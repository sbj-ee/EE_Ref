% Options for packages loaded elsewhere
\PassOptionsToPackage{unicode}{hyperref}
\PassOptionsToPackage{hyphens}{url}
\PassOptionsToPackage{dvipsnames,svgnames,x11names}{xcolor}
\documentclass[
  11pt,
]{report}
\usepackage{xcolor}
\usepackage[inner=1.25in,outer=0.85in,top=1in,bottom=1in]{geometry}
\usepackage{amsmath,amssymb}
\setcounter{secnumdepth}{-\maxdimen} % remove section numbering
\usepackage{iftex}
\ifPDFTeX
  \usepackage[T1]{fontenc}
  \usepackage[utf8]{inputenc}
  \usepackage{textcomp} % provide euro and other symbols
\else % if luatex or xetex
  \usepackage{unicode-math} % this also loads fontspec
  \defaultfontfeatures{Scale=MatchLowercase}
  \defaultfontfeatures[\rmfamily]{Ligatures=TeX,Scale=1}
\fi
\usepackage{lmodern}
\ifPDFTeX\else
  % xetex/luatex font selection
  \setmainfont[]{STIX Two Text}
  \setmonofont[]{Menlo}
\fi
% Use upquote if available, for straight quotes in verbatim environments
\IfFileExists{upquote.sty}{\usepackage{upquote}}{}
\IfFileExists{microtype.sty}{% use microtype if available
  \usepackage[]{microtype}
  \UseMicrotypeSet[protrusion]{basicmath} % disable protrusion for tt fonts
}{}
\makeatletter
\@ifundefined{KOMAClassName}{% if non-KOMA class
  \IfFileExists{parskip.sty}{%
    \usepackage{parskip}
  }{% else
    \setlength{\parindent}{0pt}
    \setlength{\parskip}{6pt plus 2pt minus 1pt}}
}{% if KOMA class
  \KOMAoptions{parskip=half}}
\makeatother
\usepackage{color}
\usepackage{fancyvrb}
\newcommand{\VerbBar}{|}
\newcommand{\VERB}{\Verb[commandchars=\\\{\}]}
\DefineVerbatimEnvironment{Highlighting}{Verbatim}{commandchars=\\\{\}}
% Add ',fontsize=\small' for more characters per line
\usepackage{framed}
\definecolor{shadecolor}{RGB}{248,248,248}
\newenvironment{Shaded}{\begin{snugshade}}{\end{snugshade}}
\newcommand{\AlertTok}[1]{\textcolor[rgb]{0.94,0.16,0.16}{#1}}
\newcommand{\AnnotationTok}[1]{\textcolor[rgb]{0.56,0.35,0.01}{\textbf{\textit{#1}}}}
\newcommand{\AttributeTok}[1]{\textcolor[rgb]{0.13,0.29,0.53}{#1}}
\newcommand{\BaseNTok}[1]{\textcolor[rgb]{0.00,0.00,0.81}{#1}}
\newcommand{\BuiltInTok}[1]{#1}
\newcommand{\CharTok}[1]{\textcolor[rgb]{0.31,0.60,0.02}{#1}}
\newcommand{\CommentTok}[1]{\textcolor[rgb]{0.56,0.35,0.01}{\textit{#1}}}
\newcommand{\CommentVarTok}[1]{\textcolor[rgb]{0.56,0.35,0.01}{\textbf{\textit{#1}}}}
\newcommand{\ConstantTok}[1]{\textcolor[rgb]{0.56,0.35,0.01}{#1}}
\newcommand{\ControlFlowTok}[1]{\textcolor[rgb]{0.13,0.29,0.53}{\textbf{#1}}}
\newcommand{\DataTypeTok}[1]{\textcolor[rgb]{0.13,0.29,0.53}{#1}}
\newcommand{\DecValTok}[1]{\textcolor[rgb]{0.00,0.00,0.81}{#1}}
\newcommand{\DocumentationTok}[1]{\textcolor[rgb]{0.56,0.35,0.01}{\textbf{\textit{#1}}}}
\newcommand{\ErrorTok}[1]{\textcolor[rgb]{0.64,0.00,0.00}{\textbf{#1}}}
\newcommand{\ExtensionTok}[1]{#1}
\newcommand{\FloatTok}[1]{\textcolor[rgb]{0.00,0.00,0.81}{#1}}
\newcommand{\FunctionTok}[1]{\textcolor[rgb]{0.13,0.29,0.53}{\textbf{#1}}}
\newcommand{\ImportTok}[1]{#1}
\newcommand{\InformationTok}[1]{\textcolor[rgb]{0.56,0.35,0.01}{\textbf{\textit{#1}}}}
\newcommand{\KeywordTok}[1]{\textcolor[rgb]{0.13,0.29,0.53}{\textbf{#1}}}
\newcommand{\NormalTok}[1]{#1}
\newcommand{\OperatorTok}[1]{\textcolor[rgb]{0.81,0.36,0.00}{\textbf{#1}}}
\newcommand{\OtherTok}[1]{\textcolor[rgb]{0.56,0.35,0.01}{#1}}
\newcommand{\PreprocessorTok}[1]{\textcolor[rgb]{0.56,0.35,0.01}{\textit{#1}}}
\newcommand{\RegionMarkerTok}[1]{#1}
\newcommand{\SpecialCharTok}[1]{\textcolor[rgb]{0.81,0.36,0.00}{\textbf{#1}}}
\newcommand{\SpecialStringTok}[1]{\textcolor[rgb]{0.31,0.60,0.02}{#1}}
\newcommand{\StringTok}[1]{\textcolor[rgb]{0.31,0.60,0.02}{#1}}
\newcommand{\VariableTok}[1]{\textcolor[rgb]{0.00,0.00,0.00}{#1}}
\newcommand{\VerbatimStringTok}[1]{\textcolor[rgb]{0.31,0.60,0.02}{#1}}
\newcommand{\WarningTok}[1]{\textcolor[rgb]{0.56,0.35,0.01}{\textbf{\textit{#1}}}}
\usepackage{longtable,booktabs,array}
\newcounter{none} % for unnumbered tables
\usepackage{calc} % for calculating minipage widths
% Correct order of tables after \paragraph or \subparagraph
\usepackage{etoolbox}
\makeatletter
\patchcmd\longtable{\par}{\if@noskipsec\mbox{}\fi\par}{}{}
\makeatother
% Allow footnotes in longtable head/foot
\IfFileExists{footnotehyper.sty}{\usepackage{footnotehyper}}{\usepackage{footnote}}
\makesavenoteenv{longtable}
\setlength{\emergencystretch}{3em} % prevent overfull lines
\providecommand{\tightlist}{%
  \setlength{\itemsep}{0pt}\setlength{\parskip}{0pt}}
% STIX Two Text is a variable font; fontspec cannot auto-detect the bold
% weight, so we re-declare the main font with an explicit BoldFont.
\setmainfont{STIX Two Text}[BoldFont={STIX Two Text Bold}]

% Fallback font for math symbols not in STIX Two Text
\usepackage{newunicodechar}
\newfontfamily\mathsymfont{STIX Two Math}
\setmathfont{STIX Two Math}

% Math symbols must be deferred to \AtBeginDocument because pandoc's
% template loads unicode-math, which claims these characters for math mode.
% Without this wrapper, newunicodechar definitions get overridden.
\AtBeginDocument{%
  % Math operators and relations
  \newunicodechar{√}{\ensuremath{\sqrt{\,}}}%
  \newunicodechar{≈}{\ensuremath{\approx}}%
  \newunicodechar{≠}{\ensuremath{\neq}}%
  \newunicodechar{≤}{\ensuremath{\leq}}%
  \newunicodechar{≥}{\ensuremath{\geq}}%
  \newunicodechar{∞}{\ensuremath{\infty}}%
  \newunicodechar{∠}{\ensuremath{\angle}}%
  \newunicodechar{∂}{\ensuremath{\partial}}%
  \newunicodechar{∇}{\ensuremath{\nabla}}%
  \newunicodechar{∫}{{\mathsymfont\char"222B}}%
  \newunicodechar{∮}{{\mathsymfont\char"222E}}%
  \newunicodechar{→}{\ensuremath{\to}}%
  \newunicodechar{↑}{\ensuremath{\uparrow}}%
  \newunicodechar{↓}{\ensuremath{\downarrow}}%
  \newunicodechar{↔}{\ensuremath{\leftrightarrow}}%
  \newunicodechar{∝}{\ensuremath{\propto}}%
  \newunicodechar{≪}{\ensuremath{\ll}}%
  \newunicodechar{≫}{\ensuremath{\gg}}%
  \newunicodechar{⊕}{\ensuremath{\oplus}}%
  \newunicodechar{✓}{{\mathsymfont\char"2713}}%
  \newunicodechar{✗}{\textbf{×}}%
  % Floor and ceiling brackets
  \newunicodechar{⌈}{\ensuremath{\lceil}}%
  \newunicodechar{⌉}{\ensuremath{\rceil}}%
  \newunicodechar{⌊}{\ensuremath{\lfloor}}%
  \newunicodechar{⌋}{\ensuremath{\rfloor}}%
}

% Modifier letters used as superscripts/subscripts in text
\newunicodechar{ᴮ}{\textsuperscript{B}}
\newunicodechar{ᴹ}{\textsuperscript{M}}
\newunicodechar{ᴺ}{\textsuperscript{N}}
\newunicodechar{ᵏ}{\textsuperscript{k}}
\newunicodechar{ᵐ}{\textsuperscript{m}}
\newunicodechar{ₖ}{\textsubscript{k}}
\newunicodechar{ₙ}{\textsubscript{n}}
\newunicodechar{ᵣ}{\textsubscript{r}}

% Page headers and footers for bound book
\usepackage{fancyhdr}
\pagestyle{fancy}
\fancyhf{}
% Even (left-hand) pages: page number on left, section title on right
\fancyhead[LE]{\thepage}
\fancyhead[RE]{\small\itshape\nouppercase{\leftmark}}
% Odd (right-hand) pages: section title on left, page number on right
\fancyhead[LO]{\small\itshape\nouppercase{\rightmark}}
\fancyhead[RO]{\thepage}
% Footer: centered book title
\fancyfoot[CE,CO]{\small\textit{Electrical Engineering Reference}}
\renewcommand{\headrulewidth}{0.4pt}
\renewcommand{\footrulewidth}{0.2pt}
% Plain pages (chapter openings, TOC): page number centered in footer
\fancypagestyle{plain}{%
  \fancyhf{}%
  \fancyfoot[C]{\thepage}%
  \renewcommand{\headrulewidth}{0pt}%
  \renewcommand{\footrulewidth}{0pt}%
}

% Figure placement and captions
\usepackage{float}
\usepackage{caption}
\captionsetup{font=small, skip=8pt, justification=centering, labelformat=empty}

% Table formatting
\usepackage{booktabs}
\usepackage{longtable}
\renewcommand{\arraystretch}{1.3}

% Green-shaded example boxes
\usepackage[most]{tcolorbox}
\newtcolorbox{examplebox}{
  colback=green!5!white,
  colframe=green!40!black,
  boxrule=0.5pt,
  arc=2pt,
  left=6pt,
  right=6pt,
  top=6pt,
  bottom=6pt,
  before skip=12pt plus 4pt,
  after skip=12pt plus 4pt,
  breakable,
  enhanced,
  before upper={\setlength{\parindent}{0pt}},
}
\usepackage{bookmark}
\IfFileExists{xurl.sty}{\usepackage{xurl}}{} % add URL line breaks if available
\urlstyle{same}
\hypersetup{
  pdftitle={Electrical Engineering Reference --- Problem Sets},
  pdfauthor={Stephen B. Johnson},
  colorlinks=true,
  linkcolor={blue},
  filecolor={Maroon},
  citecolor={Blue},
  urlcolor={blue},
  pdfcreator={LaTeX via pandoc}}

\title{Electrical Engineering Reference --- Problem Sets}
\usepackage{etoolbox}
\makeatletter
\providecommand{\subtitle}[1]{% add subtitle to \maketitle
  \apptocmd{\@title}{\par {\large #1 \par}}{}{}
}
\makeatother
\subtitle{Editio Unica}
\author{Stephen B. Johnson}
\date{}

\begin{document}
\maketitle

{
\hypersetup{linkcolor=blue}
\setcounter{tocdepth}{2}
\tableofcontents
}
\chapter{Chapter 1 --- Section 1.1: Power
Generation}\label{chapter-1-section-1.1-power-generation}

Practice problems covering fossil fuel plants, hydroelectric plants,
nuclear plants, solar PV, wind turbines, microgrids, GSU transformers,
and auxiliary transformers.

\begin{center}\rule{0.5\linewidth}{0.5pt}\end{center}

\section{Problem 1.1.1}\label{problem-1.1.1}

\textbf{Given:} A combined-cycle gas turbine plant has a gas turbine
(Brayton cycle) efficiency of 35\% and a heat recovery steam generator
(HRSG) that captures 50\% of the waste heat for a steam turbine (Rankine
cycle). The plant burns natural gas at a thermal input of 800 MW.

\textbf{Find:} (a) The gas turbine electrical output, (b) the steam
turbine electrical output, (c) the total plant output, and (d) the
overall plant efficiency.

\textbf{Solution:}

\begin{enumerate}
\def\labelenumi{(\alph{enumi})}
\item
  Gas turbine output: P\textsubscript{GT} = η\textsubscript{GT} ×
  Q\textsubscript{in} = 0.35 × 800 = \textbf{280 MW}
\item
  Waste heat from gas turbine: Q\textsubscript{waste} =
  Q\textsubscript{in} - P\textsubscript{GT} = 800 - 280 = 520 MW
\end{enumerate}

Steam turbine output: P\textsubscript{ST} = η\textsubscript{HRSG} ×
Q\textsubscript{waste} = 0.50 × 520 = \textbf{260 MW}

\begin{enumerate}
\def\labelenumi{(\alph{enumi})}
\setcounter{enumi}{2}
\item
  Total plant output: P\textsubscript{total} = P\textsubscript{GT} +
  P\textsubscript{ST} = 280 + 260 = \textbf{540 MW}
\item
  Overall efficiency: η\textsubscript{overall} = P\textsubscript{total}
  / Q\textsubscript{in} = 540 / 800 = 0.675 = \textbf{67.5\%}
\end{enumerate}

\begin{center}\rule{0.5\linewidth}{0.5pt}\end{center}

\section{Problem 1.1.2}\label{problem-1.1.2}

\textbf{Given:} A hydroelectric plant has a hydraulic head of 120 m and
a water flow rate of 40 m³/s. The turbine efficiency is 92\% and the
generator efficiency is 97\%. Use ρ = 1000 kg/m³ and g = 9.81 m/s².

\textbf{Find:} (a) The available hydraulic power, (b) the turbine
mechanical output, and (c) the electrical power output.

\textbf{Solution:}

\begin{enumerate}
\def\labelenumi{(\alph{enumi})}
\item
  Hydraulic power: P\textsubscript{hydraulic} = ρ × g × Q × h = 1000 ×
  9.81 × 40 × 120 = 47,088,000 W = \textbf{47.09 MW}
\item
  Turbine mechanical output: P\textsubscript{mech} =
  η\textsubscript{turbine} × P\textsubscript{hydraulic} = 0.92 × 47.09 =
  \textbf{43.32 MW}
\item
  Electrical power output: P\textsubscript{elec} = η\textsubscript{gen}
  × P\textsubscript{mech} = 0.97 × 43.32 = \textbf{42.02 MW}
\end{enumerate}

Overall efficiency: η = P\textsubscript{elec} /
P\textsubscript{hydraulic} = 42.02 / 47.09 = 89.2\%

\begin{center}\rule{0.5\linewidth}{0.5pt}\end{center}

\section{Problem 1.1.3}\label{problem-1.1.3}

\textbf{Given:} A nuclear power plant has a rated thermal output of
4,000 MW(th) and operates a Rankine cycle with a thermal efficiency of
34\%. The plant operates at a capacity factor of 88\% over one year
(8,760 hours).

\textbf{Find:} (a) The electrical output at full power, (b) the annual
electrical energy generated, and (c) the amount of waste heat that must
be rejected (in MW).

\textbf{Solution:}

\begin{enumerate}
\def\labelenumi{(\alph{enumi})}
\item
  Electrical output: P\textsubscript{elec} = η ×
  P\textsubscript{thermal} = 0.34 × 4,000 = \textbf{1,360 MW}
\item
  Annual energy at capacity factor: E = P\textsubscript{elec} × CF ×
  8,760 = 1,360 × 0.88 × 8,760 = 10,483,968 MWh = \textbf{1.048 × 10⁷
  MWh}
\item
  Waste heat: Q\textsubscript{reject} = P\textsubscript{thermal} -
  P\textsubscript{elec} = 4,000 - 1,360 = \textbf{2,640 MW}
\end{enumerate}

This waste heat is rejected through cooling towers or to a body of
water.

\begin{center}\rule{0.5\linewidth}{0.5pt}\end{center}

\section{Problem 1.1.4}\label{problem-1.1.4}

\textbf{Given:} A utility-scale solar PV farm is rated at 75
MW\textsubscript{DC}. The site receives an average of 4.8 peak sun hours
(PSH) per day. The inverter efficiency is 97\%, DC wiring losses are
2\%, and soiling losses are 3\%.

\textbf{Find:} (a) The effective daily DC energy after wiring and
soiling losses, (b) the daily AC energy output, (c) the annual AC energy
output, and (d) the capacity factor.

\textbf{Solution:}

\begin{enumerate}
\def\labelenumi{(\alph{enumi})}
\item
  Effective DC energy: Loss factor = (1 - 0.02)(1 - 0.03) = 0.98 × 0.97
  = 0.9506 E\textsubscript{DC,eff} = 75 × 4.8 × 0.9506 = \textbf{342.2
  MWh/day}
\item
  Daily AC energy: E\textsubscript{AC} = η\textsubscript{inv} ×
  E\textsubscript{DC,eff} = 0.97 × 342.2 = \textbf{331.9 MWh/day}
\item
  Annual AC energy: E\textsubscript{annual} = 331.9 × 365 =
  \textbf{121,143 MWh/year}
\item
  Capacity factor: CF = E\textsubscript{annual} /
  (P\textsubscript{rated} × 8,760) = 121,143 / (75 × 8,760) = 121,143 /
  657,000 = 0.184 = \textbf{18.4\%}
\end{enumerate}

\begin{center}\rule{0.5\linewidth}{0.5pt}\end{center}

\section{Problem 1.1.5}\label{problem-1.1.5}

\textbf{Given:} A wind turbine has a rotor diameter of 150 m and
operates at a hub height where air density is ρ = 1.18 kg/m³. The wind
speed is 10 m/s. The turbine power coefficient is C\textsubscript{p} =
0.45 and the gearbox/generator efficiency is 94\%.

\textbf{Find:} (a) The rotor swept area, (b) the total kinetic power in
the wind, (c) the rotor mechanical power, (d) the electrical output, and
(e) the percentage of the Betz limit achieved.

\textbf{Solution:}

\begin{enumerate}
\def\labelenumi{(\alph{enumi})}
\item
  Rotor swept area: A = π(D/2)² = π(75)² = \textbf{17,671.5 m²}
\item
  Wind power: P\textsubscript{wind} = 0.5 × ρ × A × v³ = 0.5 × 1.18 ×
  17,671.5 × (10)³ P\textsubscript{wind} = 0.5 × 1.18 × 17,671.5 × 1,000
  = \textbf{10,426,185 W = 10.43 MW}
\item
  Rotor mechanical power: P\textsubscript{rotor} = C\textsubscript{p} ×
  P\textsubscript{wind} = 0.45 × 10.43 = \textbf{4.69 MW}
\item
  Electrical output: P\textsubscript{elec} = η\textsubscript{drive} ×
  P\textsubscript{rotor} = 0.94 × 4.69 = \textbf{4.41 MW}
\item
  Betz limit is C\textsubscript{p,max} = 16/27 = 0.5926. Fraction
  achieved: 0.45 / 0.5926 = 0.759 = \textbf{75.9\% of the Betz limit}
\end{enumerate}

\begin{center}\rule{0.5\linewidth}{0.5pt}\end{center}

\section{Problem 1.1.6}\label{problem-1.1.6}

\textbf{Given:} A microgrid serves a data center with a peak load of
1,200 kW. Resources include a 600 kW solar array (currently producing
450 kW), a 1,000 kWh battery at 75\% state of charge (minimum SOC =
15\%), a 500 kW natural gas generator, and inverter efficiency of 95\%.

\textbf{Find:} (a) The power deficit that must be supplied by the
battery or generator, (b) whether the gas generator alone can meet the
deficit, and (c) how long the battery alone could sustain the deficit.

\textbf{Solution:}

\begin{enumerate}
\def\labelenumi{(\alph{enumi})}
\item
  Power deficit: Deficit = Peak load - Solar output = 1,200 - 450 =
  \textbf{750 kW}
\item
  The gas generator is rated at 500 kW, which is less than 750 kW. The
  generator alone \textbf{cannot meet the deficit}. It would need 750 -
  500 = 250 kW from the battery simultaneously.
\item
  Usable battery energy: E\textsubscript{usable} =
  (SOC\textsubscript{current} - SOC\textsubscript{min}) × Capacity =
  (0.75 - 0.15) × 1,000 = 600 kWh AC energy delivered:
  E\textsubscript{AC} = η\textsubscript{inv} × E\textsubscript{usable} =
  0.95 × 600 = 570 kWh Duration at 750 kW: t = 570 / 750 = \textbf{0.76
  hours (approximately 45.6 minutes)}
\end{enumerate}

\begin{center}\rule{0.5\linewidth}{0.5pt}\end{center}

\section{Problem 1.1.7}\label{problem-1.1.7}

\textbf{Given:} A 350 MW natural gas plant has a generator rated at 412
MVA, 18 kV, 0.85 power factor. The GSU transformer is rated 412 MVA, 18
kV delta / 230 kV wye-grounded, with an impedance of 9.0\%.

\textbf{Find:} (a) The rated current on the 18 kV and 230 kV sides, (b)
the maximum symmetrical fault current on the 230 kV side (assuming
infinite bus on the generator side), and (c) the fault MVA.

\textbf{Solution:}

\begin{enumerate}
\def\labelenumi{(\alph{enumi})}
\item
  Rated currents: I\textsubscript{LV} = S / (√3 × V\textsubscript{LV}) =
  412 × 10⁶ / (√3 × 18,000) = 412 × 10⁶ / 31,177 = \textbf{13,216 A}
  I\textsubscript{HV} = S / (√3 × V\textsubscript{HV}) = 412 × 10⁶ / (√3
  × 230,000) = 412 × 10⁶ / 398,372 = \textbf{1,034 A}
\item
  Fault current at 230 kV: I\textsubscript{fault} =
  I\textsubscript{HV,rated} / Z\textsubscript{pu} = 1,034 / 0.09 =
  \textbf{11,489 A = 11.5 kA}
\item
  Fault MVA: S\textsubscript{fault} = S\textsubscript{rated} /
  Z\textsubscript{pu} = 412 / 0.09 = \textbf{4,578 MVA = 4.58 GVA}
\end{enumerate}

\begin{center}\rule{0.5\linewidth}{0.5pt}\end{center}

\section{Problem 1.1.8}\label{problem-1.1.8}

\textbf{Given:} A 700 MW coal-fired generating unit has a generator
rated at 824 MVA, 24 kV. Station service loads total 52 MW at 0.88 power
factor lagging. The unit auxiliary transformer (UAT) steps down from 24
kV to 6.9 kV.

\textbf{Find:} (a) Station service as a percentage of gross generation,
(b) the net plant output, (c) the required UAT MVA rating with 20\%
margin, and (d) the UAT rated current on the 6.9 kV side.

\textbf{Solution:}

\begin{enumerate}
\def\labelenumi{(\alph{enumi})}
\item
  Station service percentage: SS\% = P\textsubscript{aux} /
  P\textsubscript{gross} × 100 = 52 / 700 × 100 = \textbf{7.43\%}
\item
  Net plant output: P\textsubscript{net} = 700 - 52 = \textbf{648 MW}
\item
  Required UAT apparent power: S\textsubscript{UAT} =
  P\textsubscript{aux} / PF = 52 / 0.88 = 59.1 MVA With 20\% margin:
  S\textsubscript{rated} = 59.1 × 1.20 = \textbf{70.9 MVA} (select a
  standard 75 MVA rating)
\item
  UAT current at 6.9 kV: I\textsubscript{6.9kV} = S\textsubscript{rated}
  / (√3 × V) = 75 × 10⁶ / (√3 × 6,900) = 75 × 10⁶ / 11,950 =
  \textbf{6,276 A}
\end{enumerate}

\begin{center}\rule{0.5\linewidth}{0.5pt}\end{center}

\section{Problem 1.1.9}\label{problem-1.1.9}

\textbf{Given:} A Rankine cycle coal plant has the following operating
parameters: boiler thermal input of 1,200 MW, boiler efficiency of 88\%,
turbine isentropic efficiency of 90\%, generator efficiency of 98\%, and
the condenser rejects heat at 35 degrees C.

\textbf{Find:} (a) The steam energy entering the turbine, (b) the
turbine shaft power, (c) the electrical output, and (d) the overall
plant heat rate in BTU/kWh (1 MW = 3,412,141 BTU/hr).

\textbf{Solution:}

\begin{enumerate}
\def\labelenumi{(\alph{enumi})}
\item
  Steam energy into turbine: P\textsubscript{steam} =
  η\textsubscript{boiler} × Q\textsubscript{fuel} = 0.88 × 1,200 =
  \textbf{1,056 MW}
\item
  Turbine shaft power: P\textsubscript{shaft} = η\textsubscript{turbine}
  × P\textsubscript{steam} = 0.90 × 1,056 = \textbf{950.4 MW}
\item
  Electrical output: P\textsubscript{elec} = η\textsubscript{gen} ×
  P\textsubscript{shaft} = 0.98 × 950.4 = \textbf{931.4 MW}
\item
  Overall efficiency: η = P\textsubscript{elec} / Q\textsubscript{fuel}
  = 931.4 / 1,200 = 0.7762 Heat rate = 3,412,141 /
  η\textsubscript{overall} = 3,412,141 / (931.4/1,200 × 1,000) =
  (3,412,141 × 1,200) / (931.4 × 1,000) Simpler: HR = 3,412,141 / (η ×
  1,000) BTU/kWh η = 0.7762, so HR = 3,412 / 0.7762 = \textbf{4,396
  BTU/kWh}
\end{enumerate}

Note: This is the net heat rate from fuel input to electrical output.
Typical coal plants have heat rates of 9,000-10,500 BTU/kWh because the
Carnot efficiency is not accounted for here. The turbine isentropic
efficiency is applied to the actual steam energy, not ideal Carnot
output. A more realistic calculation would use the Carnot-limited steam
cycle efficiency (typically 40-45\%) rather than the isentropic turbine
efficiency alone.

\begin{center}\rule{0.5\linewidth}{0.5pt}\end{center}

\section{Problem 1.1.10}\label{problem-1.1.10}

\textbf{Given:} Compare three generation sources for a 200 MW load
operating 6,000 hours per year: (a) Natural gas combined-cycle at 58\%
efficiency with gas cost of \$4.50/MMBTU, (b) Solar PV with a capacity
factor of 22\% and installation cost of \$1,200/kW amortized at
\$120/kW-yr, (c) Onshore wind with a capacity factor of 35\% and
installation cost of \$1,500/kW amortized at \$150/kW-yr. Assume 1 MWh =
3.412 MMBTU of thermal input.

\textbf{Find:} The annual fuel or levelized energy cost for each source
to supply 200 MW × 6,000 hr = 1,200,000 MWh of energy.

\textbf{Solution:}

\begin{enumerate}
\def\labelenumi{(\alph{enumi})}
\item
  Natural gas combined-cycle: Thermal input = E / η = 1,200,000 / 0.58 =
  2,068,966 MWh\textsubscript{thermal} In MMBTU: 2,068,966 × 3.412 =
  7,059,229 MMBTU Annual fuel cost = 7,059,229 × \$4.50 =
  \textbf{\$31,766,531/year} LCOE (fuel only) = \$31,766,531 / 1,200,000
  = \textbf{\$26.47/MWh}
\item
  Solar PV: Required installed capacity = 200 MW / 0.22 × (6,000/8,760)
  = effective capacity needed. Actually, annual energy from installed
  capacity P: E = P × CF × 8,760. For 1,200,000 MWh: P = 1,200,000 /
  (0.22 × 8,760) = 1,200,000 / 1,927.2 = 622.6 MW Annual cost = 622.6 ×
  1,000 × \$120 = \textbf{\$74,712,000/year} LCOE = \$74,712,000 /
  1,200,000 = \textbf{\$62.26/MWh}
\item
  Onshore wind: P = 1,200,000 / (0.35 × 8,760) = 1,200,000 / 3,066 =
  391.4 MW Annual cost = 391.4 × 1,000 × \$150 =
  \textbf{\$58,710,000/year} LCOE = \$58,710,000 / 1,200,000 =
  \textbf{\$48.93/MWh}
\end{enumerate}

\chapter{Chapter 1 --- Section 1.2: Power
Transmission}\label{chapter-1-section-1.2-power-transmission}

Practice problems covering short, medium, and long transmission lines,
underground transmission cables, and transmission line parameters.

\begin{center}\rule{0.5\linewidth}{0.5pt}\end{center}

\section{Problem 1.2.1}\label{problem-1.2.1}

\textbf{Given:} A 50 km, 115 kV, single-phase short transmission line
has a series impedance of z = 0.20 + j0.50 Ω/km. The line delivers 15
MVA at 115 kV with a 0.90 lagging power factor at the receiving end.

\textbf{Find:} (a) The total line impedance, (b) the receiving-end
current, (c) the sending-end voltage, and (d) the voltage regulation.

\textbf{Solution:}

\begin{enumerate}
\def\labelenumi{(\alph{enumi})}
\item
  Total line impedance: Z\textsubscript{total} = 50 × (0.20 + j0.50) =
  10.0 + j25.0 Ω \textbar Z\textsubscript{total}\textbar{} = √(10² +
  25²) = √(725) = 26.93 Ω ∠Z = tan⁻¹(25/10) = 68.20°
  Z\textsubscript{total} = \textbf{26.93∠68.20° Ω}
\item
  Receiving-end current: I\textsubscript{R} = S / V\textsubscript{R} =
  15 × 10⁶ / 115,000 = 130.43 A Power factor angle: θ = cos⁻¹(0.90) =
  25.84° I\textsubscript{R} = 130.43∠-25.84° A
\item
  Sending-end voltage: V\textsubscript{S} = V\textsubscript{R} +
  I\textsubscript{R} × Z\textsubscript{total} I\textsubscript{R} × Z =
  130.43∠-25.84° × 26.93∠68.20° = 3,512.5∠42.36° = 3,512.5(cos 42.36° +
  j sin 42.36°) = 2,596.0 + j2,365.8 V V\textsubscript{S} = 115,000 +
  2,596.0 + j2,365.8 = 117,596.0 + j2,365.8
  \textbar V\textsubscript{S}\textbar{} = √(117,596² + 2,365.8²) =
  \textbf{117,620 V = 117.62 kV}
\item
  Voltage regulation: VR\% = (\textbar V\textsubscript{S}\textbar{} -
  \textbar V\textsubscript{R}\textbar) /
  \textbar V\textsubscript{R}\textbar{} × 100 = (117,620 - 115,000) /
  115,000 × 100 = \textbf{2.28\%}
\end{enumerate}

\begin{center}\rule{0.5\linewidth}{0.5pt}\end{center}

\section{Problem 1.2.2}\label{problem-1.2.2}

\textbf{Given:} A 180 km, 345 kV, three-phase medium transmission line
has per-phase parameters: z = 0.04 + j0.40 Ω/km and y = j3.0 × 10⁻⁶
S/km. Use the nominal π model.

\textbf{Find:} (a) The total series impedance Z, (b) the total shunt
admittance Y, (c) the ABCD parameters, and (d) the sending-end voltage
if the line delivers 400 MVA at 345 kV, 0.95 lagging power factor.

\textbf{Solution:}

\begin{enumerate}
\def\labelenumi{(\alph{enumi})}
\item
  Total series impedance: Z = z × l = (0.04 + j0.40) × 180 = 7.2 + j72.0
  Ω \textbar Z\textbar{} = √(7.2² + 72²) = √(51.84 + 5,184) = √5,235.84
  = \textbf{72.36 Ω} ∠Z = tan⁻¹(72/7.2) = 84.29°
\item
  Total shunt admittance: Y = y × l = j3.0 × 10⁻⁶ × 180 = \textbf{j5.4 ×
  10⁻⁴ S}
\item
  ABCD parameters: YZ/2 = (j5.4 × 10⁻⁴)(7.2 + j72)/2 = (j0.003888 +
  j²0.03888)/2 = (-0.03888 + j0.003888)/2 YZ/2 = -0.01944 + j0.001944
\end{enumerate}

A = D = 1 + YZ/2 = 1 - 0.01944 + j0.001944 = \textbf{0.9806 + j0.0019 =
0.9806∠0.11°}

B = Z = \textbf{7.2 + j72.0 = 72.36∠84.29° Ω}

YZ/4 = -0.00972 + j0.000972 C = Y(1 + YZ/4) = j5.4 × 10⁻⁴ × (0.99028 +
j0.000972) C = j5.347 × 10⁻⁴ - 5.249 × 10⁻⁷ = \textbf{-5.25 × 10⁻⁷ +
j5.347 × 10⁻⁴ ≈ 5.347 × 10⁻⁴∠90.06° S}

\begin{enumerate}
\def\labelenumi{(\alph{enumi})}
\setcounter{enumi}{3}
\tightlist
\item
  Receiving-end quantities (per phase): V\textsubscript{R} = 345,000 /
  √3 = 199,186 V (phase voltage) I\textsubscript{R} = S / (√3 ×
  V\textsubscript{LL}) = 400 × 10⁶ / (√3 × 345,000) = 669.4 A θ =
  cos⁻¹(0.95) = 18.19°, so I\textsubscript{R} = 669.4∠-18.19° A
\end{enumerate}

Sending-end voltage (per phase): V\textsubscript{S} = A ×
V\textsubscript{R} + B × I\textsubscript{R} A × V\textsubscript{R} =
0.9806∠0.11° × 199,186∠0° = 195,322∠0.11° B × I\textsubscript{R} =
72.36∠84.29° × 669.4∠-18.19° = 48,438∠66.10° = 48,438(cos 66.10° + j sin
66.10°) = 19,612 + j44,295

V\textsubscript{S} = 195,322 + 19,612 + j(375 + 44,295) = 214,934 +
j44,670 \textbar V\textsubscript{S}\textbar{} = √(214,934² + 44,670²) =
\textbf{219,528 V per phase} V\textsubscript{S,LL} = √3 × 219,521 =
\textbf{380.2 kV}

Voltage regulation = (380.2 - 345) / 345 × 100 = \textbf{10.2\%}

\begin{center}\rule{0.5\linewidth}{0.5pt}\end{center}

\section{Problem 1.2.3}\label{problem-1.2.3}

\textbf{Given:} A 500 km, 500 kV, three-phase long transmission line has
per-phase distributed parameters: z = 0.02 + j0.30 Ω/km and y = j4.0 ×
10⁻⁶ S/km.

\textbf{Find:} (a) The characteristic impedance Z\textsubscript{c}, (b)
the propagation constant γ, (c) the surge impedance loading (SIL), and
(d) the attenuation and phase shift over the full line length.

\textbf{Solution:}

\begin{enumerate}
\def\labelenumi{(\alph{enumi})}
\item
  Characteristic impedance: z/y = (0.02 + j0.30) / (j4.0 × 10⁻⁶)
  Multiply by -j/-j: = (0.02 + j0.30)(-j) / (4.0 × 10⁻⁶) = (-j0.02 +
  0.30) / (4.0 × 10⁻⁶) = (0.30 - j0.02) / (4.0 × 10⁻⁶) = 75,000 - j5,000
  \textbar z/y\textbar{} = √(75,000² + 5,000²) = √(5.625 × 10⁹ + 25 ×
  10⁶) = √(5.65 × 10⁹) = 75,166.5 ∠(z/y) = tan⁻¹(-5,000/75,000) = -3.81°
  Z\textsubscript{c} = √75,166.5 ∠(-3.81°/2) = \textbf{274.2∠-1.91° Ω ≈
  274.2 Ω}
\item
  Propagation constant: z × y = (0.02 + j0.30)(j4.0 × 10⁻⁶) = j8.0 ×
  10⁻⁸ - 1.2 × 10⁻⁶ = -1.2 × 10⁻⁶ + j8.0 × 10⁻⁸ = 1.2027 × 10⁻⁶∠176.18°
  γ = √(1.2027 × 10⁻⁶)∠(176.18°/2) = 1.097 × 10⁻³∠88.09° γ = α + jβ =
  \textbf{3.66 × 10⁻⁵ + j1.096 × 10⁻³ per km}
\item
  Surge impedance loading: SIL = V\textsubscript{LL}² /
  Z\textsubscript{c} = (500,000)² / 274.2 = \textbf{912 MW}
\item
  Over 500 km: Attenuation: αl = 3.66 × 10⁻⁵ × 500 = 0.0183 Np = 0.0183
  × 8.686 = \textbf{0.159 dB} Phase shift: βl = 1.096 × 10⁻³ × 500 =
  0.548 rad = \textbf{31.4°}
\end{enumerate}

\begin{center}\rule{0.5\linewidth}{0.5pt}\end{center}

\section{Problem 1.2.4}\label{problem-1.2.4}

\textbf{Given:} A 230 kV underground XLPE cable is 40 km long. Each
phase has a capacitance of 0.25 μF/km and an ampacity of 700 A. System
frequency is 60 Hz.

\textbf{Find:} (a) The charging current per phase, (b) the three-phase
reactive power generated, (c) the remaining ampacity for load current,
(d) the maximum load power transfer, and (e) the required shunt reactor
compensation.

\textbf{Solution:}

\begin{enumerate}
\def\labelenumi{(\alph{enumi})}
\item
  Total capacitance per phase: C\textsubscript{total} = 0.25 × 40 = 10.0
  μF Line-to-neutral voltage: V\textsubscript{LN} = 230,000 / √3 =
  132,791 V I\textsubscript{charging} = 2πfCV\textsubscript{LN} = 2π ×
  60 × 10.0 × 10⁻⁶ × 132,791 = \textbf{500.7 A per phase}
\item
  Three-phase reactive power: Q = 3 × V\textsubscript{LN} ×
  I\textsubscript{charging} = 3 × 132,791 × 500.7 = 199.4 × 10⁶ VAR =
  \textbf{199.4 MVAR}
\item
  Remaining ampacity (charging and load currents are \textasciitilde90°
  apart): I\textsubscript{load,max} = √(I\textsubscript{rated}² -
  I\textsubscript{charging}²) = √(700² - 500.7²) = √(490,000 - 250,700)
  = √239,300 = \textbf{489.2 A}
\item
  Maximum load power: S\textsubscript{max} = √3 × V\textsubscript{LL} ×
  I\textsubscript{load,max} = √3 × 230,000 × 489.2 = 194.8 × 10⁶ VA =
  \textbf{194.8 MVA}
\end{enumerate}

Without charging current: S = √3 × 230,000 × 700 = 278.8 MVA. The
charging current reduces capacity by (278.8 - 194.8)/278.8 =
\textbf{30.1\%}.

\begin{enumerate}
\def\labelenumi{(\alph{enumi})}
\setcounter{enumi}{4}
\tightlist
\item
  A shunt reactor of approximately \textbf{200 MVAR} is needed to
  compensate the capacitive charging current, which would restore the
  full 700 A ampacity for load current.
\end{enumerate}

\begin{center}\rule{0.5\linewidth}{0.5pt}\end{center}

\section{Problem 1.2.5}\label{problem-1.2.5}

\textbf{Given:} Two parallel 138 kV transmission lines each have a
series impedance of Z = 5 + j40 Ω and a thermal rating of 400 MVA. One
line trips due to a fault. The total load is 600 MVA at 0.92 lagging
power factor.

\textbf{Find:} (a) The current through the remaining line, (b) whether
the remaining line exceeds its thermal rating, (c) the voltage drop
across the remaining line, and (d) the voltage regulation.

\textbf{Solution:}

\begin{enumerate}
\def\labelenumi{(\alph{enumi})}
\item
  Current through the remaining line: I = S / (√3 × V) = 600 × 10⁶ / (√3
  × 138,000) = \textbf{2,510 A}
\item
  The thermal rating current: I\textsubscript{rated} =
  S\textsubscript{rated} / (√3 × V) = 400 × 10⁶ / (√3 × 138,000) = 1,674
  A Since 2,510 A \textgreater{} 1,674 A, the line \textbf{exceeds its
  thermal rating by 50\%}. This is an N-1 contingency violation.
\item
  Voltage drop (magnitude): \textbar V\textsubscript{drop}\textbar{} =
  \textbar I × Z\textbar{} = 2,510 × \textbar5 + j40\textbar{} = 2,510 ×
  √(25 + 1,600) = 2,510 × 40.31 = \textbf{101,178 V}
\end{enumerate}

More precisely, with power factor angle θ = cos⁻¹(0.92) = 23.07°:
V\textsubscript{drop} = I(R cos θ + X sin θ) = 2,510 × (5 × 0.92 + 40 ×
0.3919) = 2,510 × (4.60 + 15.68) = 2,510 × 20.28 = \textbf{50,903 V}

\begin{enumerate}
\def\labelenumi{(\alph{enumi})}
\setcounter{enumi}{3}
\tightlist
\item
  Voltage regulation (using the approximate formula): VR\% =
  V\textsubscript{drop} / V\textsubscript{R} × 100 = 50,903 /
  (138,000/√3) × 100 = 50,903 / 79,674 × 100 = \textbf{63.9\%}
\end{enumerate}

This extreme voltage regulation confirms the N-1 violation and the need
for load shedding or alternate supply paths.

\begin{center}\rule{0.5\linewidth}{0.5pt}\end{center}

\section{Problem 1.2.6}\label{problem-1.2.6}

\textbf{Given:} A 345 kV three-phase transmission line is 300 km long
with a characteristic impedance of Z\textsubscript{c} = 290 Ω. The line
currently transfers 500 MW.

\textbf{Find:} (a) The surge impedance loading (SIL), (b) whether the
line is above or below SIL, (c) the reactive power behavior of the line
(absorbing or generating VARs), and (d) the approximate receiving-end
voltage if the sending-end voltage is 345 kV (use the simplified
relationship for lines near SIL).

\textbf{Solution:}

\begin{enumerate}
\def\labelenumi{(\alph{enumi})}
\item
  SIL: SIL = V² / Z\textsubscript{c} = (345,000)² / 290 =
  119,025,000,000 / 290 = \textbf{410.4 MW}
\item
  The line transfers 500 MW, which is \textbf{above the SIL} (500
  \textgreater{} 410.4).
\item
  When a line operates above SIL, the reactive power absorbed by the
  series inductance exceeds the reactive power generated by the shunt
  capacitance. The line \textbf{absorbs reactive power} (net inductive),
  causing the voltage to drop along the line.
\item
  For a line operating above SIL, the receiving-end voltage is lower
  than the sending-end voltage. Using the approximate relationship:
  V\textsubscript{R}/V\textsubscript{S} ≈ cos(βl) for lines at SIL, with
  correction for loading above SIL. At 300 km with β ≈ 1.1 × 10⁻³
  rad/km: βl = 0.33 rad = 18.9° V\textsubscript{R} ≈ V\textsubscript{S}
  × cos(βl) × (SIL/P)\^{}(0.5) ≈ 345 × cos(18.9°) × √(410.4/500)
  V\textsubscript{R} ≈ 345 × 0.946 × 0.906 = \textbf{295.7 kV}
\end{enumerate}

The receiving-end voltage drops to approximately \textbf{296 kV},
requiring reactive compensation (shunt capacitors) at the receiving end
to maintain acceptable voltage.

\chapter{Chapter 1 --- Section 1.3: Power
Distribution}\label{chapter-1-section-1.3-power-distribution}

Practice problems covering substations (transformers, autotransformers,
circuit breakers, voltage regulators, CTs, VTs, switching), distribution
poles, underground distribution, three-phase connections, AC analysis,
and power factor correction.

\begin{center}\rule{0.5\linewidth}{0.5pt}\end{center}

\section{Problem 1.3.1}\label{problem-1.3.1}

\textbf{Given:} A three-phase, 75 MVA, 230 kV / 34.5 kV, wye-delta
transformer has a nameplate impedance of 10\%. The source (230 kV side)
has a short-circuit capacity of 2,000 MVA.

\textbf{Find:} (a) The turns ratio, (b) the rated current on each side,
(c) the total impedance seen at the 34.5 kV bus (source + transformer)
in per-unit on the transformer base, and (d) the fault current at the
34.5 kV bus.

\textbf{Solution:}

\begin{enumerate}
\def\labelenumi{(\alph{enumi})}
\item
  Turns ratio: N₁/N₂ = V₁/V₂ = 230,000 / 34,500 = \textbf{6.667:1}
\item
  Rated currents: I\textsubscript{HV} = S / (√3 × V\textsubscript{HV}) =
  75 × 10⁶ / (√3 × 230,000) = \textbf{188.3 A} I\textsubscript{LV} = S /
  (√3 × V\textsubscript{LV}) = 75 × 10⁶ / (√3 × 34,500) = \textbf{1,255
  A}
\item
  Source impedance on transformer base: Z\textsubscript{source} =
  S\textsubscript{base} / S\textsubscript{SC} = 75 / 2,000 = 0.0375 pu
  Total impedance: Z\textsubscript{total} = Z\textsubscript{source} +
  Z\textsubscript{xfmr} = 0.0375 + 0.10 = \textbf{0.1375 pu}
\item
  Fault current at 34.5 kV bus: I\textsubscript{fault} = 1.0 /
  Z\textsubscript{total} × I\textsubscript{base} = (1/0.1375) × 1,255 =
  7.273 × 1,255 = \textbf{9,128 A = 9.13 kA}
\end{enumerate}

\begin{center}\rule{0.5\linewidth}{0.5pt}\end{center}

\section{Problem 1.3.2}\label{problem-1.3.2}

\textbf{Given:} A single-phase, 200 MVA, 345/138 kV autotransformer
supplies a 138 kV bus.

\textbf{Find:} (a) The turns ratio, (b) the rated load current at 138
kV, (c) the source current at 345 kV, (d) the current in the series
winding, (e) the transformed (coupled) power, and (f) the power
advantage ratio.

\textbf{Solution:}

\begin{enumerate}
\def\labelenumi{(\alph{enumi})}
\item
  Turns ratio: a = V₁/V₂ = 345/138 = \textbf{2.5:1}
\item
  Load current: I\textsubscript{load} = S / V₂ = 200 × 10⁶ / 138,000 =
  \textbf{1,449 A}
\item
  Source current: I\textsubscript{source} = S / V₁ = 200 × 10⁶ / 345,000
  = \textbf{580 A}
\item
  Series winding current: I\textsubscript{series} =
  I\textsubscript{load} - I\textsubscript{source} = 1,449 - 580 =
  \textbf{869 A}
\item
  Transformed power: S\textsubscript{transformed} =
  S\textsubscript{auto} × (1 - 1/a) = 200 × (1 - 1/2.5) = 200 × 0.60 =
  \textbf{120 MVA}
\item
  Power advantage: PA = S\textsubscript{auto} /
  S\textsubscript{transformed} = 200 / 120 = \textbf{1.667}
\end{enumerate}

A two-winding transformer rated at only 120 MVA can deliver 200 MVA as
an autotransformer at this ratio.

\begin{center}\rule{0.5\linewidth}{0.5pt}\end{center}

\section{Problem 1.3.3}\label{problem-1.3.3}

\textbf{Given:} A 230 kV substation bus has a rated short-circuit
current of 50 kA symmetrical. The circuit breaker must interrupt the
fault within 3 cycles at 60 Hz.

\textbf{Find:} (a) The interrupting time in milliseconds, (b) the
three-phase short-circuit MVA, and (c) the asymmetrical peak current
assuming an X/R ratio of 20 (asymmetry factor = 2.6 × symmetrical RMS).

\textbf{Solution:}

\begin{enumerate}
\def\labelenumi{(\alph{enumi})}
\item
  Interrupting time: t = 3 × (1/60) = 0.05 s = \textbf{50 ms}
\item
  Short-circuit MVA: S\textsubscript{fault} = √3 × V\textsubscript{LL} ×
  I\textsubscript{fault} = √3 × 230,000 × 50,000 = 19.92 × 10⁹ VA =
  \textbf{19,919 MVA ≈ 19.9 GVA}
\item
  Asymmetrical peak current: I\textsubscript{peak} = 2.6 ×
  I\textsubscript{sym} = 2.6 × 50,000 = \textbf{130 kA peak}
\end{enumerate}

The circuit breaker must have a close-and-latch rating of at least 130
kA.

\begin{center}\rule{0.5\linewidth}{0.5pt}\end{center}

\section{Problem 1.3.4}\label{problem-1.3.4}

\textbf{Given:} A step voltage regulator on a 14.4 kV (line-to-neutral)
distribution feeder has R = 2.5 V and X = 7.5 V compensator settings (on
a 120 V base). The CT ratio is 400:5 (80:1). The feeder current is 250 A
at 0.88 lagging power factor.

\textbf{Find:} (a) The secondary current, (b) the compensator voltage
drop, and (c) the required regulator output voltage (on 120 V base and
converted to primary).

\textbf{Solution:}

\begin{enumerate}
\def\labelenumi{(\alph{enumi})}
\item
  Secondary current: I\textsubscript{sec} = 250 / 80 = \textbf{3.125 A}
\item
  Power factor angle: θ = cos⁻¹(0.88) = 28.36° I\textsubscript{sec} =
  3.125∠-28.36° = 2.75 - j1.484 A
\end{enumerate}

V\textsubscript{drop} = I\textsubscript{sec} × (R + jX) = (2.75 -
j1.484)(2.5 + j7.5) = 6.875 + j20.625 - j3.710 + 11.13 = 18.005 +
j16.915 \textbar V\textsubscript{drop}\textbar{} = √(18.005² + 16.915²)
= √(324.2 + 286.1) = √610.3 = \textbf{24.7 V}

\begin{enumerate}
\def\labelenumi{(\alph{enumi})}
\setcounter{enumi}{2}
\tightlist
\item
  Regulator output on 120 V base: V\textsubscript{reg} = 120 + 24.7 =
  144.7 V Convert to primary: V\textsubscript{out} = (144.7/120) ×
  14,400 = 1.206 × 14,400 = \textbf{17,362 V}
\end{enumerate}

Tap position needed: (17,362 - 14,400) / 14,400 = 0.2057 = 20.6\% Since
regulator range is ±10\%, this exceeds the single regulator range. The
regulator would operate at its maximum boost (+10\%), providing 14,400 ×
1.10 = 15,840 V. A second regulator or capacitor bank is required.

\begin{center}\rule{0.5\linewidth}{0.5pt}\end{center}

\section{Problem 1.3.5}\label{problem-1.3.5}

\textbf{Given:} A 1200:5 CT with accuracy class C400 is connected to a
relay with impedance of 1.5 Ω and lead wire resistance of 2.0 Ω (total
loop). A primary fault current of 18,000 A flows.

\textbf{Find:} (a) The secondary current, (b) the voltage across the
burden, and (c) whether the CT maintains accuracy.

\textbf{Solution:}

\begin{enumerate}
\def\labelenumi{(\alph{enumi})}
\item
  CT ratio = 1200/5 = 240 Secondary current: I\textsubscript{sec} =
  18,000 / 240 = \textbf{75 A}
\item
  Total burden: Z\textsubscript{burden} = 1.5 + 2.0 = 3.5 Ω Voltage:
  V\textsubscript{sec} = I\textsubscript{sec} × Z\textsubscript{burden}
  = 75 × 3.5 = \textbf{262.5 V}
\item
  C400 means the CT can deliver up to 400 V at 20 times rated secondary
  (100 A) without exceeding 10\% ratio error. At 75 A (15× rated), the
  required 262.5 V is within the 400 V rating. The CT \textbf{maintains
  accuracy} with margin of 400 - 262.5 = 137.5 V.
\end{enumerate}

\begin{center}\rule{0.5\linewidth}{0.5pt}\end{center}

\section{Problem 1.3.6}\label{problem-1.3.6}

\textbf{Given:} A 34,500:120 V voltage transformer is connected to two
relays (each 30 VA burden) and two meters (each 40 VA burden). The VT
has a thermal burden rating of 250 VA and an accuracy class of 0.3 at
100 VA.

\textbf{Find:} (a) The total connected burden, (b) the secondary
current, (c) whether the VT thermal rating is exceeded, and (d) whether
the metering accuracy is maintained.

\textbf{Solution:}

\begin{enumerate}
\def\labelenumi{(\alph{enumi})}
\item
  Total burden: S\textsubscript{total} = 2 × 30 + 2 × 40 = 60 + 80 =
  \textbf{140 VA}
\item
  Secondary current: I\textsubscript{sec} = S\textsubscript{total} /
  V\textsubscript{sec} = 140 / 120 = \textbf{1.167 A}
\item
  140 VA \textless{} 250 VA thermal rating: \textbf{within thermal
  rating} (56\% loaded).
\item
  The accuracy class 0.3 is specified at 100 VA burden. At 140 VA, the
  VT is loaded beyond its accuracy-rated burden. The \textbf{metering
  accuracy may not meet the 0.3\% specification} at this burden. Revenue
  metering should be placed on a separate VT secondary winding with
  burden within 100 VA.
\end{enumerate}

\begin{center}\rule{0.5\linewidth}{0.5pt}\end{center}

\section{Problem 1.3.7}\label{problem-1.3.7}

\textbf{Given:} A three-phase, 480Y/277 V wye-connected system supplies
three loads: Phase A = 60 kW at unity PF, Phase B = 45 kW at 0.85
lagging PF, Phase C = 70 kW at 0.90 lagging PF.

\textbf{Find:} (a) The current in each phase, (b) the neutral current
(magnitude), and (c) the total three-phase apparent power.

\textbf{Solution:}

\begin{enumerate}
\def\labelenumi{(\alph{enumi})}
\item
  Phase currents: I\textsubscript{A} = P\textsubscript{A} /
  (V\textsubscript{LN} × PF\textsubscript{A}) = 60,000 / (277 × 1.0) =
  \textbf{216.6 A} at 0° I\textsubscript{B} = P\textsubscript{B} /
  (V\textsubscript{LN} × PF\textsubscript{B}) = 45,000 / (277 × 0.85) =
  \textbf{191.2 A} at θ = -cos⁻¹(0.85) = -31.79° relative to Phase B
  I\textsubscript{C} = P\textsubscript{C} / (V\textsubscript{LN} ×
  PF\textsubscript{C}) = 70,000 / (277 × 0.90) = \textbf{280.8 A} at θ =
  -cos⁻¹(0.90) = -25.84° relative to Phase C
\item
  Phase angles referenced to Phase A: I\textsubscript{A} = 216.6∠0° =
  216.6 + j0 I\textsubscript{B} = 191.2∠(-120° - 31.79°) =
  191.2∠-151.79° = -168.6 - j90.5 I\textsubscript{C} = 280.8∠(120° -
  25.84°) = 280.8∠94.16° = -20.4 + j280.1
\end{enumerate}

I\textsubscript{N} = I\textsubscript{A} + I\textsubscript{B} +
I\textsubscript{C} = (216.6 - 168.6 - 20.4) + j(0 - 90.5 + 280.1) = 27.6
+ j189.6 \textbar I\textsubscript{N}\textbar{} = √(27.6² + 189.6²) =
√(762 + 35,948) = √36,710 = \textbf{191.6 A}

The significant neutral current results from the unbalanced loading.

\begin{enumerate}
\def\labelenumi{(\alph{enumi})}
\setcounter{enumi}{2}
\tightlist
\item
  Total apparent power: S\textsubscript{A} = 60/1.0 = 60 kVA;
  S\textsubscript{B} = 45/0.85 = 52.9 kVA; S\textsubscript{C} = 70/0.90
  = 77.8 kVA S\textsubscript{total} = 60 + 52.9 + 77.8 = \textbf{190.7
  kVA} (arithmetic sum; vector sum would differ)
\end{enumerate}

\begin{center}\rule{0.5\linewidth}{0.5pt}\end{center}

\section{Problem 1.3.8}\label{problem-1.3.8}

\textbf{Given:} A three-phase, 240 V delta-connected heater bank has
three identical resistance elements of 12 Ω each.

\textbf{Find:} (a) The phase current, (b) the line current, (c) the
total three-phase power, and (d) the equivalent wye impedance per phase.

\textbf{Solution:}

\begin{enumerate}
\def\labelenumi{(\alph{enumi})}
\item
  Phase current (delta: phase voltage = line voltage):
  I\textsubscript{phase} = V\textsubscript{LL} / R = 240 / 12 =
  \textbf{20.0 A}
\item
  Line current: I\textsubscript{L} = √3 × I\textsubscript{phase} = 1.732
  × 20.0 = \textbf{34.64 A}
\item
  Total power (resistive, PF = 1): P = √3 × V\textsubscript{LL} ×
  I\textsubscript{L} × PF = √3 × 240 × 34.64 × 1.0 = \textbf{14,400 W =
  14.4 kW} Alternatively: P = 3 × V\textsubscript{phase}² / R = 3 × 240²
  / 12 = 3 × 4,800 = 14,400 W. Check.
\item
  Equivalent wye impedance: Z\textsubscript{Y} = Z\textsubscript{Δ} / 3
  = 12 / 3 = \textbf{4 Ω per phase}
\end{enumerate}

Verify: I\textsubscript{L} = V\textsubscript{LN} / Z\textsubscript{Y} =
(240/√3) / 4 = 138.56 / 4 = 34.64 A. Check.

\begin{center}\rule{0.5\linewidth}{0.5pt}\end{center}

\section{Problem 1.3.9}\label{problem-1.3.9}

\textbf{Given:} A series RLC circuit has R = 15 Ω, L = 30 mH, and C =
200 μF, connected to a 240 V\textsubscript{rms}, 50 Hz source.

\textbf{Find:} (a) The inductive and capacitive reactances, (b) the
impedance in polar form, (c) the current magnitude and phase, (d) the
power factor, and (e) the real, reactive, and apparent power.

\textbf{Solution:}

\begin{enumerate}
\def\labelenumi{(\alph{enumi})}
\item
  X\textsubscript{L} = 2πfL = 2π × 50 × 0.030 = \textbf{9.42 Ω}
  X\textsubscript{C} = 1/(2πfC) = 1/(2π × 50 × 200 × 10⁻⁶) =
  \textbf{15.92 Ω}
\item
  Net reactance: X = X\textsubscript{L} - X\textsubscript{C} = 9.42 -
  15.92 = -6.50 Ω (capacitive) Z = 15 - j6.50 Ω \textbar Z\textbar{} =
  √(15² + 6.50²) = √(225 + 42.25) = √267.25 = 16.35 Ω θ =
  tan⁻¹(-6.50/15) = -23.43° Z = \textbf{16.35∠-23.43° Ω}
\item
  Current: I = V/\textbar Z\textbar{} = 240/16.35 = \textbf{14.68 A} at
  phase angle +23.43° (leading)
\item
  Power factor: PF = cos(23.43°) = \textbf{0.918 leading}
\item
  Real power: P = V × I × PF = 240 × 14.68 × 0.918 = \textbf{3,233 W}
  Reactive power: Q = V × I × sin(23.43°) = 240 × 14.68 × 0.3975 =
  \textbf{1,400 VAR} (capacitive) Apparent power: S = V × I = 240 ×
  14.68 = \textbf{3,523 VA}
\end{enumerate}

\begin{center}\rule{0.5\linewidth}{0.5pt}\end{center}

\section{Problem 1.3.10}\label{problem-1.3.10}

\textbf{Given:} An industrial plant draws 800 kW at 0.68 lagging power
factor from a 600 V, 60 Hz, three-phase supply. The utility requires
correction to 0.92 lagging.

\textbf{Find:} (a) The original and target reactive power, (b) the
required capacitor bank in kVAR, (c) the capacitance per phase for a
wye-connected bank, and (d) the line current reduction achieved.

\textbf{Solution:}

\begin{enumerate}
\def\labelenumi{(\alph{enumi})}
\tightlist
\item
  Original: θ₁ = cos⁻¹(0.68) = 47.16° Q₁ = P × tan(θ₁) = 800 ×
  tan(47.16°) = 800 × 1.0785 = \textbf{862.8 kVAR}
\end{enumerate}

Target: θ₂ = cos⁻¹(0.92) = 23.07° Q₂ = P × tan(θ₂) = 800 × tan(23.07°) =
800 × 0.4259 = \textbf{340.7 kVAR}

\begin{enumerate}
\def\labelenumi{(\alph{enumi})}
\setcounter{enumi}{1}
\item
  Required capacitor bank: Q\textsubscript{cap} = Q₁ - Q₂ = 862.8 -
  340.7 = \textbf{522.1 kVAR}
\item
  For a wye-connected bank, each capacitor sees line-to-neutral voltage:
  V\textsubscript{LN} = 600 / √3 = 346.4 V Q per phase:
  Q\textsubscript{phase} = 522.1 / 3 = 174.0 kVAR X\textsubscript{C} =
  V\textsubscript{LN}² / Q\textsubscript{phase} = (346.4)² / 174,000 =
  119,993 / 174,000 = 0.6896 Ω C = 1/(2πfX\textsubscript{C}) = 1/(2π ×
  60 × 0.6896) = \textbf{3,847 μF per phase}
\item
  Original current: I₁ = S₁/(√3 × V) = (800/0.68)/(√3 × 600) =
  1,176.5/1,039.2 = \textbf{1,132 A} Corrected current: I₂ = S₂/(√3 × V)
  = (800/0.92)/(√3 × 600) = 869.6/1,039.2 = \textbf{836.8 A} Reduction:
  (1,132 - 836.8)/1,132 × 100 = \textbf{26.1\% reduction in line
  current}
\end{enumerate}

\chapter{Chapter 1 --- Section 1.4: Power System
Protection}\label{chapter-1-section-1.4-power-system-protection}

Practice problems covering protective relays, fault analysis, protection
coordination, and symmetrical components.

\begin{center}\rule{0.5\linewidth}{0.5pt}\end{center}

\section{Problem 1.4.1}\label{problem-1.4.1}

\textbf{Given:} A time-overcurrent relay (51) protects a 13.8 kV feeder
with a pickup current of 800 A and uses the IEEE Extremely Inverse
characteristic: t = (28.2 / (M² - 1) + 0.1217) × TDS, where M =
I\textsubscript{fault}/I\textsubscript{pickup}. The time dial setting is
TDS = 3.0.

\textbf{Find:} The relay operating time for fault currents of (a) 2,400
A, (b) 4,800 A, and (c) 12,000 A.

\textbf{Solution:}

\begin{enumerate}
\def\labelenumi{(\alph{enumi})}
\item
  M = 2,400 / 800 = 3.0 t = (28.2/(9 - 1) + 0.1217) × 3.0 = (28.2/8 +
  0.1217) × 3.0 = (3.525 + 0.1217) × 3.0 = 3.647 × 3.0 = \textbf{10.94
  s}
\item
  M = 4,800 / 800 = 6.0 t = (28.2/(36 - 1) + 0.1217) × 3.0 = (28.2/35 +
  0.1217) × 3.0 = (0.806 + 0.1217) × 3.0 = 0.928 × 3.0 = \textbf{2.78 s}
\item
  M = 12,000 / 800 = 15.0 t = (28.2/(225 - 1) + 0.1217) × 3.0 =
  (28.2/224 + 0.1217) × 3.0 = (0.126 + 0.1217) × 3.0 = 0.248 × 3.0 =
  \textbf{0.74 s}
\end{enumerate}

The extremely inverse curve shows dramatic time reduction with
increasing fault current: a 5x increase in fault current (from 2,400 to
12,000 A) reduces operating time by a factor of nearly 15.

\begin{center}\rule{0.5\linewidth}{0.5pt}\end{center}

\section{Problem 1.4.2}\label{problem-1.4.2}

\textbf{Given:} A 4.16 kV industrial bus is supplied through a
transformer with Z = 6.5\% on a 10 MVA base. The source impedance on the
same base is 1.5\%. The system X/R ratio is 10.

\textbf{Find:} (a) The base current, (b) the three-phase fault current
in per-unit and amperes, (c) the fault MVA, and (d) the asymmetrical
fault current (use asymmetry factor of 1.0 + e\textsuperscript{-π/(X/R)}
for the peak).

\textbf{Solution:}

\begin{enumerate}
\def\labelenumi{(\alph{enumi})}
\item
  Base current: I\textsubscript{base} = S\textsubscript{base} / (√3 ×
  V\textsubscript{base}) = 10 × 10⁶ / (√3 × 4,160) = \textbf{1,388 A}
\item
  Total impedance: Z\textsubscript{total} = 0.015 + 0.065 = 0.080 pu
  I\textsubscript{fault} = 1.0 / 0.080 = 12.5 pu = 12.5 × 1,388 =
  \textbf{17,350 A = 17.35 kA}
\item
  Fault MVA: S\textsubscript{fault} = S\textsubscript{base} /
  Z\textsubscript{total} = 10 / 0.080 = \textbf{125 MVA}
\item
  Peak asymmetrical factor: Peak multiplier = √2 × (1 +
  e\textsuperscript{-π/10}) = 1.414 × (1 + e\textsuperscript{-0.3142}) =
  1.414 × (1 + 0.7304) = 1.414 × 1.7304 = 2.447 I\textsubscript{peak} =
  2.447 × 17,350 = \textbf{42,455 A = 42.5 kA peak}
\end{enumerate}

\begin{center}\rule{0.5\linewidth}{0.5pt}\end{center}

\section{Problem 1.4.3}\label{problem-1.4.3}

\textbf{Given:} Three series protective devices on a radial feeder must
be coordinated: - Fuse C (downstream): 100K fuse link, total clearing
time = 0.03 s at 5,000 A - Relay B (midstream): pickup = 500 A, IEEE
Very Inverse, TDS = 2.5 - Relay A (upstream): pickup = 800 A, IEEE Very
Inverse, must coordinate with Relay B

The IEEE Very Inverse characteristic is: t = (19.61/(M² - 1) + 0.491) ×
TDS. CTI between relays = 0.3 s.

\textbf{Find:} (a) Relay B operating time at 5,000 A, (b) whether Relay
B coordinates with Fuse C (margin \textgreater{} 0.2 s), (c) required
Relay A TDS to coordinate with Relay B at the maximum fault of 8,000 A.

\textbf{Solution:}

\begin{enumerate}
\def\labelenumi{(\alph{enumi})}
\item
  Relay B at 5,000 A: M\textsubscript{B} = 5,000/500 = 10.0
  t\textsubscript{B} = (19.61/(100 - 1) + 0.491) × 2.5 = (0.198 + 0.491)
  × 2.5 = 0.689 × 2.5 = \textbf{1.72 s}
\item
  Coordination with Fuse C: Margin = t\textsubscript{B} -
  t\textsubscript{Fuse C} = 1.72 - 0.03 = 1.69 s 1.69 s \textgreater{}
  0.2 s: \textbf{Relay B coordinates with Fuse C} with ample margin.
\item
  Relay A at 8,000 A: First, Relay B at 8,000 A: M\textsubscript{B} =
  8,000/500 = 16.0 t\textsubscript{B} = (19.61/(256 - 1) + 0.491) × 2.5
  = (0.0769 + 0.491) × 2.5 = 0.568 × 2.5 = \textbf{1.42 s}
\end{enumerate}

Required Relay A time: t\textsubscript{A} = t\textsubscript{B} + CTI =
1.42 + 0.3 = 1.72 s

Relay A at 8,000 A: M\textsubscript{A} = 8,000/800 = 10.0 1.72 =
(19.61/(100 - 1) + 0.491) × TDS\textsubscript{A} = (0.198 + 0.491) ×
TDS\textsubscript{A} = 0.689 × TDS\textsubscript{A} TDS\textsubscript{A}
= 1.72/0.689 = \textbf{2.50}

\begin{center}\rule{0.5\linewidth}{0.5pt}\end{center}

\section{Problem 1.4.4}\label{problem-1.4.4}

\textbf{Given:} A 138 kV generator has sequence impedances: Z₁ = j0.20
pu, Z₂ = j0.18 pu, Z₀ = j0.06 pu. The generator neutral is grounded
through a reactor with Z\textsubscript{n} = j0.05 pu. Pre-fault voltage
is 1.0 pu. S\textsubscript{base} = 50 MVA.

\textbf{Find:} (a) The single-line-to-ground (SLG) fault current, (b)
the line-to-line (LL) fault current, and (c) the three-phase fault
current. Express results in per-unit and amperes.

\textbf{Solution:}

Base current: I\textsubscript{base} = 50 × 10⁶ / (√3 × 138,000) =
\textbf{209.2 A}

\begin{enumerate}
\def\labelenumi{(\alph{enumi})}
\item
  SLG fault (all sequences in series): For SLG: I₁ = V / (Z₁ + Z₂ + Z₀ +
  3Z\textsubscript{n}) Z\textsubscript{total} = j0.20 + j0.18 + j0.06 +
  j0.15 = j0.59 I₁ = I₂ = I₀ = 1.0/j0.59 = -j1.695 pu
  I\textsubscript{fault} = 3 × I₁ = 3 × 1.695 = 5.085 pu
  I\textsubscript{fault} = 5.085 × 209.2 = \textbf{1,064 A}
\item
  Line-to-line fault (positive and negative in series): I₁ = V / (Z₁ +
  Z₂) = 1.0 / (j0.20 + j0.18) = 1.0/j0.38 = -j2.632 pu
  I\textsubscript{fault} = √3 × \textbar I₁\textbar{} = √3 × 2.632 =
  4.557 pu I\textsubscript{fault} = 4.557 × 209.2 = \textbf{953 A}
\item
  Three-phase fault (positive sequence only): I\textsubscript{fault} = V
  / Z₁ = 1.0/j0.20 = -j5.0 pu = 5.0 pu I\textsubscript{fault} = 5.0 ×
  209.2 = \textbf{1,046 A}
\end{enumerate}

Note: In this case the SLG fault (1,064 A) exceeds the three-phase fault
(1,046 A) because of the relatively low zero-sequence impedance and
grounding impedance. This is common for solidly or low-impedance
grounded generators.

\begin{center}\rule{0.5\linewidth}{0.5pt}\end{center}

\section{Problem 1.4.5}\label{problem-1.4.5}

\textbf{Given:} A 13.8 kV distribution system has the following sequence
impedance data on a 100 MVA base: generator Z₁ = j0.12, Z₂ = j0.12, Z₀ =
j0.04; transformer Z₁ = Z₂ = Z₀ = j0.08 (delta-wye, with grounded wye).
A single-line-to-ground fault occurs on the 13.8 kV bus.

\textbf{Find:} (a) The total sequence impedances, (b) the sequence
currents, (c) the fault current, and (d) the fault current in amperes.

\textbf{Solution:}

Base current: I\textsubscript{base} = 100 × 10⁶ / (√3 × 13,800) =
\textbf{4,184 A}

\begin{enumerate}
\def\labelenumi{(\alph{enumi})}
\item
  Total sequence impedances: Z₁\textsubscript{total} =
  Z₁\textsubscript{gen} + Z₁\textsubscript{xfmr} = j0.12 + j0.08 =
  \textbf{j0.20 pu} Z₂\textsubscript{total} = Z₂\textsubscript{gen} +
  Z₂\textsubscript{xfmr} = j0.12 + j0.08 = \textbf{j0.20 pu}
  Z₀\textsubscript{total} = Z₀\textsubscript{xfmr} = \textbf{j0.08 pu}
  (delta winding blocks zero-sequence from generator)
\item
  Sequence currents: I₁ = I₂ = I₀ = V/(Z₁ + Z₂ + Z₀) = 1.0/(j0.20 +
  j0.20 + j0.08) = 1.0/j0.48 = \textbf{-j2.083 pu}
\item
  Phase A fault current: I\textsubscript{a} = 3 × I₁ = 3 × 2.083 =
  \textbf{6.25 pu}
\item
  In amperes: I\textsubscript{fault} = 6.25 × 4,184 = \textbf{26,150 A =
  26.15 kA}
\end{enumerate}

\begin{center}\rule{0.5\linewidth}{0.5pt}\end{center}

\section{Problem 1.4.6}\label{problem-1.4.6}

\textbf{Given:} A differential relay protects a 50 MVA, 138/13.8 kV,
delta-wye transformer. The CT ratios are 300:5 on the 138 kV side and
2500:5 on the 13.8 kV side. The relay has a minimum pickup of 0.3 A and
a slope setting of 25\%.

\textbf{Find:} (a) The rated currents on each side of the transformer,
(b) the CT secondary currents at rated load, (c) the relay operating and
restraint currents at rated load (accounting for the delta-wye 30° phase
shift and √3 factor), and (d) the minimum internal fault current (as a
percentage of rated) that will trip the relay.

\textbf{Solution:}

\begin{enumerate}
\def\labelenumi{(\alph{enumi})}
\item
  Rated currents: I\textsubscript{HV} = 50 × 10⁶/(√3 × 138,000) =
  \textbf{209.2 A} I\textsubscript{LV} = 50 × 10⁶/(√3 × 13,800) =
  \textbf{2,091.8 A}
\item
  CT secondary currents: HV side: I\textsubscript{CT,HV} = 209.2 ×
  (5/300) = \textbf{3.49 A} LV side: I\textsubscript{CT,LV} = 2,091.8 ×
  (5/2500) = \textbf{4.18 A}
\end{enumerate}

The CTs on the delta (HV) side are connected in wye, and those on the
wye (LV) side in delta to compensate for the transformer phase shift.
The delta-connected CTs multiply the current by √3:
I\textsubscript{CT,LV,delta} = 4.18 × √3 = \textbf{7.24 A}

There is a mismatch: 3.49 A vs.~7.24 A. Auxiliary CTs or relay tap
settings are used to match them. Assuming the relay compensates (modern
numerical relays do this internally): Ratio correction applied so both
sides read approximately equal at rated load.

\begin{enumerate}
\def\labelenumi{(\alph{enumi})}
\setcounter{enumi}{2}
\item
  At rated, balanced load with proper matching: Operating current:
  I\textsubscript{op} = \textbar I₁ - I₂\textbar{} ≈ \textbf{0 A}
  (ideally) Restraint current: I\textsubscript{res} =
  (\textbar I₁\textbar{} + \textbar I₂\textbar)/2 ≈ \textbf{3.49 A} (per
  relay scaling)
\item
  Minimum fault current to trip: With 25\% slope: I\textsubscript{op} =
  0.25 × I\textsubscript{res} must exceed 0.3 A pickup. At minimum:
  I\textsubscript{op} = 0.3 A This corresponds to a differential current
  of 0.3/3.49 = 8.6\% of rated, so a fault producing
  \textbf{approximately 8.6\% of rated current} on the HV side would
  trip the relay.
\end{enumerate}

\chapter{Chapter 1 --- Section 1.5: Power
Quality}\label{chapter-1-section-1.5-power-quality}

Practice problems covering harmonics, voltage sags and swells, power
quality monitoring, and arc flash analysis.

\begin{center}\rule{0.5\linewidth}{0.5pt}\end{center}

\section{Problem 1.5.1}\label{problem-1.5.1}

\textbf{Given:} A 12-pulse VFD produces the following harmonic current
spectrum (as percentage of fundamental): 11th = 8.5\%, 13th = 6.5\%,
23rd = 3.2\%, 25th = 2.8\%. The VFD draws 200 A fundamental at 480 V on
a system where the available short-circuit current is
I\textsubscript{SC} = 25,000 A.

\textbf{Find:} (a) The current THD, (b) the
I\textsubscript{SC}/I\textsubscript{L} ratio, (c) the IEEE 519 TDD limit
for that ratio, and (d) whether the installation is compliant.

\textbf{Solution:}

\begin{enumerate}
\def\labelenumi{(\alph{enumi})}
\item
  THD = √(8.5² + 6.5² + 3.2² + 2.8²) = √(72.25 + 42.25 + 10.24 + 7.84) =
  √132.58 = \textbf{11.5\%}
\item
  I\textsubscript{SC}/I\textsubscript{L} = 25,000 / 200 = \textbf{125}
\item
  For I\textsubscript{SC}/I\textsubscript{L} \textgreater{} 100: IEEE
  519 TDD limit = \textbf{15.0\%}. Individual harmonic limits: 11th-16th
  ≤ 7.0\%, 17th-22nd ≤ 2.5\%, 23rd-34th ≤ 1.4\%.
\item
  TDD = 11.5\% \textless{} 15.0\%: TDD is compliant. Individual
  harmonics: 11th = 8.5\% \textgreater{} 7.0\%: \textbf{Non-compliant}
  on the 11th harmonic. The 23rd = 3.2\% \textgreater{} 1.4\%:
  \textbf{Non-compliant} on the 23rd harmonic. The 25th = 2.8\%
  \textgreater{} 1.4\%: \textbf{Non-compliant} on the 25th harmonic.
\end{enumerate}

Overall: \textbf{Does not comply} due to individual harmonic violations.
A passive tuned filter at the 11th harmonic or an active harmonic filter
would be needed.

\begin{center}\rule{0.5\linewidth}{0.5pt}\end{center}

\section{Problem 1.5.2}\label{problem-1.5.2}

\textbf{Given:} A 208 V bus experiences a voltage sag to 72\% of nominal
lasting 150 ms. A sensitive semiconductor fabrication tool requires a
minimum of 85\% voltage. The tool draws 50 kVA during the sag.

\textbf{Find:} (a) The sag magnitude in volts, (b) whether the tool will
malfunction, (c) the voltage that a dynamic voltage restorer (DVR) must
inject, and (d) the energy the DVR must supply during the event.

\textbf{Solution:}

\begin{enumerate}
\def\labelenumi{(\alph{enumi})}
\item
  Sag voltage: 0.72 × 208 = \textbf{149.8 V}
\item
  Required minimum: 0.85 × 208 = 176.8 V. Since 149.8 V \textless{}
  176.8 V, the tool \textbf{will malfunction}.
\item
  DVR injection voltage: V\textsubscript{DVR} = V\textsubscript{nominal}
  - V\textsubscript{sag} = 208 - 149.8 = \textbf{58.2 V} (28\% of
  nominal)
\item
  DVR power: S\textsubscript{DVR} =
  (V\textsubscript{DVR}/V\textsubscript{nominal}) ×
  S\textsubscript{load} = 0.28 × 50 = 14 kVA Energy: E =
  S\textsubscript{DVR} × t = 14,000 × 0.150 = \textbf{2,100 J = 2.1 kJ}
\end{enumerate}

\begin{center}\rule{0.5\linewidth}{0.5pt}\end{center}

\section{Problem 1.5.3}\label{problem-1.5.3}

\textbf{Given:} A power quality monitor at a pharmaceutical plant
records the following over 90 days: 120 total voltage sags, of which 35
are between 80-90\%, 55 between 70-80\%, 20 between 50-70\%, and 10
below 50\%. Each sag below 70\% causes a batch loss costing \$25,000.

\textbf{Find:} (a) The SARFI-90 index (events below 90\%) per 30 days,
(b) the SARFI-70 index per 30 days, (c) the average sag frequency per
week, and (d) the annualized cost of power quality events.

\textbf{Solution:}

\begin{enumerate}
\def\labelenumi{(\alph{enumi})}
\item
  SARFI-90 = all events below 90\% per period. Total below 90\% = 35 +
  55 + 20 + 10 = 120 in 90 days. Per 30 days: SARFI-90 = 120/3 =
  \textbf{40 events per 30 days}
\item
  SARFI-70 = events below 70\%. Total below 70\% = 20 + 10 = 30 in 90
  days. Per 30 days: SARFI-70 = 30/3 = \textbf{10 events per 30 days}
\item
  Average sag frequency: 120 sags / (90/7 weeks) = 120/12.86 =
  \textbf{9.3 sags per week}
\item
  Events causing batch loss: below 70\% = 30 in 90 days. Annualized: 30
  × (365/90) = 121.7 events per year. Annual cost: 121.7 × \$25,000 =
  \textbf{\$3,041,667/year}
\end{enumerate}

This cost would justify a UPS or DVR system costing up to several
million dollars with a payback under two years.

\begin{center}\rule{0.5\linewidth}{0.5pt}\end{center}

\section{Problem 1.5.4}\label{problem-1.5.4}

\textbf{Given:} A 13.8 kV switchgear has a bolted fault current of 25
kA. The upstream relay clears faults in 0.5 seconds (30 cycles at 60
Hz). The working distance is 910 mm (36 inches). Using the IEEE 1584
simplified method with E\textsubscript{n} = 4.0 J/cm²,
C\textsubscript{f} = 1.0 (voltage \textgreater{} 1 kV), and distance
exponent x = 2.0.

\textbf{Find:} (a) The incident energy at the working distance, (b) the
required PPE category, and (c) the incident energy if an arc flash relay
reduces clearing time to 0.05 s (3 cycles).

\textbf{Solution:}

\begin{enumerate}
\def\labelenumi{(\alph{enumi})}
\item
  E = 4.184 × C\textsubscript{f} × E\textsubscript{n} × (t/0.2) ×
  (610\textsuperscript{x}/D\textsuperscript{x}) E = 4.184 × 1.0 × 4.0 ×
  (0.5/0.2) × (610²/910²) E = 4.184 × 4.0 × 2.5 × (372,100/828,100) E =
  41.84 × 0.4493 = \textbf{18.80 J/cm² = 4.49 cal/cm²}
\item
  PPE categories: Cat 1 ≤ 4 cal/cm², Cat 2 ≤ 8 cal/cm², Cat 3 ≤ 25
  cal/cm², Cat 4 ≤ 40 cal/cm². At 4.49 cal/cm²: \textbf{Category 2}
  (arc-rated clothing rated 8 cal/cm² minimum).
\item
  With arc flash relay (t = 0.05 s): E = 4.184 × 1.0 × 4.0 × (0.05/0.2)
  × (610²/910²) E = 4.184 × 4.0 × 0.25 × 0.4493 = \textbf{1.88 J/cm² =
  0.45 cal/cm²}
\end{enumerate}

This is below Category 1 (4 cal/cm²), meaning \textbf{no arc-rated PPE
is required beyond standard work clothing}. The arc flash relay reduces
incident energy by a factor of 10, from 4.49 to 0.45 cal/cm².

\begin{center}\rule{0.5\linewidth}{0.5pt}\end{center}

\section{Problem 1.5.5}\label{problem-1.5.5}

\textbf{Given:} A three-phase rectifier bridge draws a square-wave
current from the AC source. The fundamental component is 500 A, and the
harmonic amplitudes follow the ideal pattern of I\textsubscript{h} =
I₁/h for odd harmonics (no even harmonics). The system has a transformer
with 5\% impedance on a 2,000 kVA base at 480 V.

\textbf{Find:} (a) The current THD, (b) the voltage THD at the PCC
(using V\textsubscript{h} = I\textsubscript{h} × Z\textsubscript{h},
where Z\textsubscript{h} = h × Z₁), (c) whether the voltage THD meets
IEEE 519 limits (5\% for systems \textless{} 69 kV).

\textbf{Solution:}

\begin{enumerate}
\def\labelenumi{(\alph{enumi})}
\item
  Harmonic currents: I₃ = 500/3 = 166.7 A, I₅ = 100 A, I₇ = 71.4 A, I₉ =
  55.6 A, I₁₁ = 45.5 A, I₁₃ = 38.5 A. THD = √(166.7² + 100² + 71.4² +
  55.6² + 45.5² + 38.5²) / 500 = √(27,789 + 10,000 + 5,098 + 3,091 +
  2,070 + 1,482) / 500 = √49,530 / 500 = 222.6 / 500 = \textbf{44.5\%}
\item
  System impedance at fundamental: Z\textsubscript{base} = V²/S =
  480²/(2,000 × 10³) = 0.1152 Ω Z₁ = 0.05 × 0.1152 = 0.00576 Ω
  (primarily reactive)
\end{enumerate}

Harmonic voltage drops: V₃ = I₃ × 3 × Z₁ = 166.7 × 3 × 0.00576 = 2.88 V
V₅ = 100 × 5 × 0.00576 = 2.88 V V₇ = 71.4 × 7 × 0.00576 = 2.88 V V₉ =
55.6 × 9 × 0.00576 = 2.88 V V₁₁ = 45.5 × 11 × 0.00576 = 2.88 V V₁₃ =
38.5 × 13 × 0.00576 = 2.88 V

V\textsubscript{1,LN} = 480/√3 = 277.1 V THD\textsubscript{V} = √(6 ×
2.88²) / 277.1 = √49.79 / 277.1 = 7.06 / 277.1 = \textbf{2.55\%}

\begin{enumerate}
\def\labelenumi{(\alph{enumi})}
\setcounter{enumi}{2}
\tightlist
\item
  Voltage THD = 2.55\% \textless{} 5.0\% limit: \textbf{Compliant with
  IEEE 519}. The individual harmonic voltages are 2.88/277.1 = 1.04\%,
  each well below the 3\% individual limit.
\end{enumerate}

\begin{center}\rule{0.5\linewidth}{0.5pt}\end{center}

\section{Problem 1.5.6}\label{problem-1.5.6}

\textbf{Given:} A 480 V motor control center (MCC) has a bolted fault
current of 42 kA. Two scenarios are compared: (a) Standard relay with
0.3 s clearing time, and (b) Current-limiting fuse with 0.004 s
(quarter-cycle) clearing time. Working distance = 610 mm,
C\textsubscript{f} = 1.5, E\textsubscript{n} = 3.2 J/cm², x = 1.641.

\textbf{Find:} The incident energy and PPE category for each scenario.

\textbf{Solution:}

\begin{enumerate}
\def\labelenumi{(\alph{enumi})}
\item
  Standard relay (t = 0.3 s): E = 4.184 × 1.5 × 3.2 × (0.3/0.2) ×
  (610\textsuperscript{1.641}/610\textsuperscript{1.641}) The distance
  ratio = 1.0 since D = 610 mm (reference distance). E = 4.184 × 1.5 ×
  3.2 × 1.5 × 1.0 = \textbf{30.13 J/cm² = 7.20 cal/cm²} PPE:
  \textbf{Category 2} (8 cal/cm² minimum)
\item
  Current-limiting fuse (t = 0.004 s): E = 4.184 × 1.5 × 3.2 ×
  (0.004/0.2) × 1.0 E = 4.184 × 1.5 × 3.2 × 0.02 = \textbf{0.40 J/cm² =
  0.096 cal/cm²} PPE: \textbf{No arc-rated PPE required} (below Category
  1)
\end{enumerate}

The current-limiting fuse reduces incident energy from 7.20 to 0.096
cal/cm² -- a factor of 75 reduction -- by clearing in a quarter-cycle
instead of 18 cycles.

\chapter{Chapter 1 --- Section 1.6: HVDC
Transmission}\label{chapter-1-section-1.6-hvdc-transmission}

Practice problems covering HVDC fundamentals and converter technologies.

\begin{center}\rule{0.5\linewidth}{0.5pt}\end{center}

\section{Problem 1.6.1}\label{problem-1.6.1}

\textbf{Given:} A bipolar HVDC link operates at ±400 kV and transmits
2,000 MW over a distance of 800 km. Each conductor has a resistance of
0.015 Ω/km.

\textbf{Find:} (a) The DC current per pole, (b) the total I²R losses in
both conductors, (c) the loss percentage, and (d) the equivalent AC
losses for comparison using three conductors at 400 kV AC with the same
resistance per conductor.

\textbf{Solution:}

\begin{enumerate}
\def\labelenumi{(\alph{enumi})}
\item
  Each pole carries half the power: P\textsubscript{pole} = 2,000/2 =
  1,000 MW I\textsubscript{dc} =
  P\textsubscript{pole}/V\textsubscript{dc} = 1,000 × 10⁶/(400 × 10³) =
  \textbf{2,500 A per pole}
\item
  Conductor resistance: R = 0.015 × 800 = 12 Ω per conductor Losses per
  pole: P\textsubscript{loss} = I²R = 2,500² × 12 = 75 MW Total losses:
  P\textsubscript{total} = 2 × 75 = \textbf{150 MW}
\item
  Loss percentage: 150/2,000 × 100 = \textbf{7.5\%}
\item
  AC equivalent at 400 kV (three-phase): I\textsubscript{AC} = P/(√3 × V
  × PF) = 2,000 × 10⁶/(√3 × 400 × 10³ × 1.0) = 2,887 A (assuming unity
  PF for comparison) AC losses = 3 × I² × R = 3 × 2,887² × 12 = 3 ×
  100.1 × 10⁶ = 300.2 MW = 15.0\%
\end{enumerate}

HVDC losses (7.5\%) are \textbf{half} the AC losses (15.0\%).

\begin{center}\rule{0.5\linewidth}{0.5pt}\end{center}

\section{Problem 1.6.2}\label{problem-1.6.2}

\textbf{Given:} A 12-pulse LCC-HVDC rectifier operates with firing angle
α = 20° and commutation overlap μ = 25°. The no-load DC voltage per
6-pulse bridge is V\textsubscript{d0} = 300 kV. The DC link transmits
1,500 MW.

\textbf{Find:} (a) The actual DC voltage per bridge, (b) the total DC
voltage (12-pulse), (c) the DC current, (d) the converter power factor,
and (e) the reactive power consumed.

\textbf{Solution:}

\begin{enumerate}
\def\labelenumi{(\alph{enumi})}
\item
  DC voltage per bridge: V\textsubscript{d} = V\textsubscript{d0} × (cos
  α + cos(α + μ))/2 = 300 × (cos 20° + cos 45°)/2 = 300 × (0.9397 +
  0.7071)/2 = 300 × 0.8234 = \textbf{247.0 kV per bridge}
\item
  Total DC voltage (two bridges in series): V\textsubscript{dc} = 2 ×
  247.0 = \textbf{494.0 kV}
\item
  DC current: I\textsubscript{dc} = P/V\textsubscript{dc} = 1,500 ×
  10⁶/494,000 = \textbf{3,036 A}
\item
  Converter power factor: cos φ ≈ (cos α + cos(α + μ))/2 = 0.8234 φ =
  cos⁻¹(0.8234) = \textbf{34.6°} PF = \textbf{0.823 lagging}
\item
  Reactive power: Q = P × tan φ = 1,500 × tan(34.6°) = 1,500 × 0.691 =
  \textbf{1,036 MVAR}
\end{enumerate}

This large reactive power demand requires shunt capacitor banks and
harmonic filters totaling approximately 1,000+ MVAR at the converter
station.

\begin{center}\rule{0.5\linewidth}{0.5pt}\end{center}

\section{Problem 1.6.3}\label{problem-1.6.3}

\textbf{Given:} A VSC-HVDC link connects an offshore wind farm to the
onshore grid. The link is rated at ±320 kV, 1,000 MW, with submarine
cables 150 km long. Each cable has resistance of 0.010 Ω/km. The VSC
converter has a power loss of 1.0\% per conversion stage.

\textbf{Find:} (a) The DC current, (b) the cable I²R losses, (c) the
total converter losses (two converters), (d) the total system losses,
and (e) the overall transmission efficiency.

\textbf{Solution:}

\begin{enumerate}
\def\labelenumi{(\alph{enumi})}
\item
  DC current per pole: P\textsubscript{pole} = 1,000/2 = 500 MW
  I\textsubscript{dc} = 500 × 10⁶/(320 × 10³) = \textbf{1,562.5 A per
  pole}
\item
  Cable losses: R = 0.010 × 150 = 1.5 Ω per cable P\textsubscript{cable}
  = 2 × I² × R = 2 × 1,562.5² × 1.5 = 2 × 3,662,109 = \textbf{7.32 MW}
\item
  Converter losses (rectifier + inverter): P\textsubscript{conv} = 2 ×
  0.01 × 1,000 = \textbf{20 MW}
\item
  Total losses: P\textsubscript{total loss} = P\textsubscript{cable} +
  P\textsubscript{conv} = 7.32 + 20 = \textbf{27.32 MW}
\item
  Transmission efficiency: η =
  (P\textsubscript{delivered})/(P\textsubscript{generated}) = (1,000 -
  27.32)/1,000 = 972.68/1,000 = \textbf{97.3\%}
\end{enumerate}

The VSC converter losses (2.0\%) dominate over cable losses (0.73\%),
which is typical for shorter HVDC links.

\begin{center}\rule{0.5\linewidth}{0.5pt}\end{center}

\section{Problem 1.6.4}\label{problem-1.6.4}

\textbf{Given:} An HVDC submarine cable link is being evaluated as an
alternative to an AC cable. The route is 60 km at 230 kV. The AC cable
has capacitance of 0.22 μF/km per phase and ampacity of 600 A. The HVDC
option operates at ±200 kV with the same conductors.

\textbf{Find:} (a) The AC cable charging current per phase, (b) the
remaining AC ampacity for load, (c) the maximum AC power transfer, (d)
the HVDC power transfer (no charging current at DC), and (e) the power
transfer advantage of HVDC.

\textbf{Solution:}

\begin{enumerate}
\def\labelenumi{(\alph{enumi})}
\item
  AC charging current: C\textsubscript{total} = 0.22 × 60 = 13.2 μF per
  phase V\textsubscript{LN} = 230,000/√3 = 132,791 V
  I\textsubscript{charging} = 2πfCV = 2π × 60 × 13.2 × 10⁻⁶ × 132,791 =
  \textbf{661 A per phase}
\item
  I\textsubscript{charging} = 661 A \textgreater{}
  I\textsubscript{rated} = 600 A! The charging current \textbf{exceeds
  the cable ampacity}. The AC cable cannot carry any load current over
  60 km at 230 kV. This is a practical impossibility for AC at this
  distance and voltage.
\item
  Maximum AC power transfer = \textbf{0 MW} (the cable is completely
  consumed by charging current).
\end{enumerate}

Even with shunt reactors at each end: I\textsubscript{load} ≈ √(600² -
(661/2)²) = √(360,000 - 109,056) = √250,944 = 500.9 A (with mid-point
compensation) S\textsubscript{AC} = √3 × 230,000 × 500.9 = \textbf{199.5
MVA}

\begin{enumerate}
\def\labelenumi{(\alph{enumi})}
\setcounter{enumi}{3}
\item
  HVDC power transfer (two poles): P\textsubscript{HVDC} = 2 ×
  V\textsubscript{dc} × I = 2 × 200,000 × 600 = \textbf{240 MW}
\item
  Without reactive compensation, HVDC delivers 240 MW versus 0 MW for AC
  -- an infinite advantage. Even with ideal compensation, HVDC delivers
  240 MW versus \textasciitilde190 MW for AC = \textbf{1.26x advantage}.
  This illustrates why HVDC is required for long submarine cables.
\end{enumerate}

\chapter{Chapter 1 --- Section 1.7: Load Flow
Analysis}\label{chapter-1-section-1.7-load-flow-analysis}

Practice problems covering power flow calculations using the bus
admittance matrix, Gauss-Seidel iteration, and power balance equations.

\begin{center}\rule{0.5\linewidth}{0.5pt}\end{center}

\section{Problem 1.7.1}\label{problem-1.7.1}

\textbf{Given:} A 3-bus power system has the following configuration: -
Bus 1: Slack bus, V₁ = 1.05∠0° pu - Bus 2: PV bus, P₂ = 0.5 pu
(generation), \textbar V₂\textbar{} = 1.02 pu - Bus 3: PQ bus, P₃ = -0.8
pu (load), Q₃ = -0.3 pu (load)

Line admittances: y₁₂ = 10 - j30 pu, y₁₃ = 8 - j24 pu, y₂₃ = 5 - j15 pu.

\textbf{Find:} (a) The Y\textsubscript{bus} matrix, and (b) the initial
power mismatch at Bus 3 using a flat start (V₃⁽⁰⁾ = 1.0∠0°, V₂⁽⁰⁾ =
1.02∠0°).

\textbf{Solution:}

\begin{enumerate}
\def\labelenumi{(\alph{enumi})}
\tightlist
\item
  Y\textsubscript{bus} diagonal elements (sum of admittances connected
  to each bus): Y₁₁ = y₁₂ + y₁₃ = (10 - j30) + (8 - j24) = \textbf{18 -
  j54 pu} Y₂₂ = y₁₂ + y₂₃ = (10 - j30) + (5 - j15) = \textbf{15 - j45
  pu} Y₃₃ = y₁₃ + y₂₃ = (8 - j24) + (5 - j15) = \textbf{13 - j39 pu}
\end{enumerate}

Off-diagonal elements (negative of line admittance): Y₁₂ = Y₂₁ = -(10 -
j30) = \textbf{-10 + j30 pu} Y₁₃ = Y₃₁ = -(8 - j24) = \textbf{-8 + j24
pu} Y₂₃ = Y₃₂ = -(5 - j15) = \textbf{-5 + j15 pu}

\begin{enumerate}
\def\labelenumi{(\alph{enumi})}
\setcounter{enumi}{1}
\tightlist
\item
  Power injected at Bus 3 (calculated): P₃\textsubscript{calc} =
  \textbar V₃\textbar{} × Σ\textbar V\textsubscript{k}\textbar{} ×
  (G₃\textsubscript{k} cos(δ₃ - δ\textsubscript{k}) +
  B₃\textsubscript{k} sin(δ₃ - δ\textsubscript{k}))
\end{enumerate}

With flat start (all angles = 0°), sin terms vanish:
P₃\textsubscript{calc} = \textbar V₃\textbar{} ×
(\textbar V₁\textbar G₃₁ + \textbar V₂\textbar G₃₂ +
\textbar V₃\textbar G₃₃) = 1.0 × (1.05×(-8) + 1.02×(-5) + 1.0×13) = 1.0
× (-8.4 - 5.1 + 13.0) = \textbf{-0.5 pu}

Specified P₃ = -0.8 pu. Mismatch: ΔP₃ = P₃\textsubscript{spec} -
P₃\textsubscript{calc} = -0.8 - (-0.5) = \textbf{-0.3 pu}

Q₃\textsubscript{calc} = \textbar V₃\textbar{} ×
(\textbar V₁\textbar G₃₁ sin(0) - \textbar V₁\textbar B₃₁ cos(0) +
\ldots{} ) = \textbar V₃\textbar{} ×
Σ\textbar V\textsubscript{k}\textbar(G₃\textsubscript{k}
sin(δ₃-δ\textsubscript{k}) - B₃\textsubscript{k}
cos(δ₃-δ\textsubscript{k})) = 1.0 × (0 - 1.05×24 + 0 - 1.02×15 + 0 -
1.0×(-39)) = 1.0 × (-25.2 - 15.3 + 39.0) = \textbf{-1.5 pu}

ΔQ₃ = Q₃\textsubscript{spec} - Q₃\textsubscript{calc} = -0.3 - (-1.5) =
\textbf{+1.2 pu}

The large mismatches confirm that the flat start is far from the
solution. Newton-Raphson iterations would converge to the solution in
3-5 iterations.

\begin{center}\rule{0.5\linewidth}{0.5pt}\end{center}

\section{Problem 1.7.2}\label{problem-1.7.2}

\textbf{Given:} A simple 2-bus system has Bus 1 (slack, V₁ = 1.0∠0°)
connected to Bus 2 (PQ, P₂ = -2.0 pu, Q₂ = -0.8 pu) through a line with
admittance y₁₂ = 3 - j12 pu. Initial guess: V₂⁽⁰⁾ = 1.0∠0°.

\textbf{Find:} Perform two Gauss-Seidel iterations to find V₂.

\textbf{Solution:}

Y₂₂ = y₁₂ = 3 - j12; Y₂₁ = -(3 - j12) = -3 + j12.

GS update: V₂\textsuperscript{(k+1)} = (1/Y₂₂) × {[}(P₂ -
jQ₂)/V₂\textsuperscript{(k)*} - Y₂₁V₁{]}

\textbf{Iteration 1} (V₂⁽⁰⁾ = 1.0 + j0): (P₂ - jQ₂)/V₂* = (-2.0 +
j0.8)/1.0 = -2.0 + j0.8 -Y₂₁V₁ = (3 - j12)(1.0) = 3 - j12 Sum = (-2.0 +
j0.8) + (3 - j12) = 1.0 - j11.2 V₂⁽¹⁾ = (1.0 - j11.2)/(3 - j12) = (1.0 -
j11.2)(3 + j12)/((3)² + (12)²) = (3 + j12 - j33.6 + 134.4)/153 = (137.4
- j21.6)/153 = \textbf{0.8980 - j0.1412 = 0.9090∠-8.94° pu}

\textbf{Iteration 2} (V₂⁽¹⁾ = 0.8980 - j0.1412): V₂⁽¹⁾* = 0.8980 +
j0.1412 (P₂ - jQ₂)/V₂* = (-2.0 + j0.8)/(0.8980 + j0.1412)

Multiply by conjugate: (-2.0 + j0.8)(0.8980 - j0.1412)/(0.8980² +
0.1412²) Numerator: -1.796 + j0.2824 + j0.7184 + 0.1130 = -1.683 +
j1.0008 Denominator: 0.8064 + 0.01994 = 0.8264 = -2.036 + j1.211

Sum = (-2.036 + j1.211) + (3 - j12) = 0.964 - j10.789 V₂⁽²⁾ = (0.964 -
j10.789)/(3 - j12) = (0.964 - j10.789)(3 + j12)/153 = (2.892 + j11.568 -
j32.367 + 129.468)/153 = (132.36 - j20.799)/153 = \textbf{0.8651 -
j0.1359 = 0.8757∠-8.93° pu}

The voltage at Bus 2 is converging toward approximately 0.876∠-8.9° pu,
indicating a significant voltage drop due to the heavy load (2.0 pu on
the line).

\begin{center}\rule{0.5\linewidth}{0.5pt}\end{center}

\section{Problem 1.7.3}\label{problem-1.7.3}

\textbf{Given:} A 345 kV transmission line connects Bus A (sending,
V\textsubscript{A} = 1.02∠10° pu) to Bus B (receiving,
V\textsubscript{B} = 0.98∠0° pu). The line impedance is Z = 0.01 + j0.10
pu, and the shunt admittance is negligible.

\textbf{Find:} (a) The complex power flow from A to B, (b) the real and
reactive power delivered to Bus B, (c) the real and reactive power
losses in the line, and (d) the line current in amperes
(S\textsubscript{base} = 100 MVA).

\textbf{Solution:}

\begin{enumerate}
\def\labelenumi{(\alph{enumi})}
\tightlist
\item
  Line admittance: y = 1/Z = 1/(0.01 + j0.10) = (0.01 - j0.10)/(0.01² +
  0.10²) = (0.01 - j0.10)/0.0101 = 0.990 - j9.901 pu
\end{enumerate}

Power from A to B: S\textsubscript{AB} = V\textsubscript{A} ×
(V\textsubscript{A} - V\textsubscript{B})* × y*

V\textsubscript{A} - V\textsubscript{B} = 1.02∠10° - 0.98∠0° = (1.02
cos10° + j1.02 sin10°) - 0.98 = (1.0045 - 0.98) + j0.1771 = 0.0245 +
j0.1771

(V\textsubscript{A} - V\textsubscript{B})* = 0.0245 - j0.1771

y* = 0.990 + j9.901

V\textsubscript{A}(V\textsubscript{A}-V\textsubscript{B})\emph{y} =
(1.0045 + j0.1771)(0.0245 - j0.1771)(0.990 + j9.901)

First: (0.0245 - j0.1771)(0.990 + j9.901) = 0.02426 + j0.24258 -
j0.17533 + 1.75348 = 1.77774 + j0.06725

Then: (1.0045 + j0.1771)(1.77774 + j0.06725) = 1.7858 + j0.0675 +
j0.3149 + (j²)0.01191 = (1.7858 - 0.01191) + j(0.0675 + 0.3149) = 1.7739
+ j0.3824

S\textsubscript{AB} = \textbf{1.774 + j0.382 pu = (177.4 + j38.2) MVA}

\begin{enumerate}
\def\labelenumi{(\alph{enumi})}
\setcounter{enumi}{1}
\tightlist
\item
  Power at Bus B: S\textsubscript{BA} =
  V\textsubscript{B}(V\textsubscript{B} - V\textsubscript{A})\emph{y}
  (power into Bus B = -S\textsubscript{BA})
\end{enumerate}

The real power delivered to B equals sending power minus losses:
P\textsubscript{losses} = I² × R. We need I first.

I = (V\textsubscript{A} - V\textsubscript{B})/Z = (0.0245 +
j0.1771)/(0.01 + j0.10) = (0.0245 + j0.1771)(0.01 - j0.10)/(0.0101) =
(0.000245 - j0.00245 + j0.001771 + 0.01771)/0.0101 = (0.017955 -
j0.000679)/0.0101 = 1.778 - j0.0672

\textbar I\textbar{} = √(1.778² + 0.0672²) = \textbf{1.779 pu}

\begin{enumerate}
\def\labelenumi{(\alph{enumi})}
\setcounter{enumi}{2}
\tightlist
\item
  Line losses: P\textsubscript{loss} = \textbar I\textbar² × R = 1.779²
  × 0.01 = 0.0316 pu = \textbf{3.16 MW} Q\textsubscript{loss} =
  \textbar I\textbar² × X = 1.779² × 0.10 = 0.3165 pu = \textbf{31.65
  MVAR}
\end{enumerate}

Real power to Bus B: P\textsubscript{B} = 177.4 - 3.16 = \textbf{174.2
MW}

\begin{enumerate}
\def\labelenumi{(\alph{enumi})}
\setcounter{enumi}{3}
\tightlist
\item
  Base current: I\textsubscript{base} = 100 × 10⁶/(√3 × 345,000) = 167.3
  A I = 1.779 × 167.3 = \textbf{297.6 A}
\end{enumerate}

\chapter{Chapter 1 --- Section 1.8: SCADA
Systems}\label{chapter-1-section-1.8-scada-systems}

Practice problems covering SCADA architecture, RTU data acquisition,
communication protocols (DNP3, IEC 61850), scan rates, data throughput,
cybersecurity (NERC CIP, IDS performance), and system reliability.

\begin{center}\rule{0.5\linewidth}{0.5pt}\end{center}

\section{Problem 1.8.1}\label{problem-1.8.1}

\textbf{Given:} A regional utility operates a SCADA system monitoring 85
substations. Each RTU reports 120 analog measurements (4 bytes each) and
96 digital status points (1 bit each). The master station uses DNP3 over
TCP/IP with a scan cycle of 6 seconds and a protocol overhead of 25\%.

\textbf{Find:} (a) The total number of monitored data points, (b) the
raw payload per scan cycle, (c) the total data per scan with overhead,
and (d) the minimum communication link bandwidth required.

\textbf{Solution:}

\begin{enumerate}
\def\labelenumi{(\alph{enumi})}
\item
  Total data points: Analog: 85 × 120 = 10,200 Digital: 85 × 96 = 8,160
  Total = 10,200 + 8,160 = \textbf{18,360 points}
\item
  Raw payload per scan: Analog data: 10,200 × 4 = 40,800 bytes Digital
  data: 8,160 bits / 8 = 1,020 bytes Raw payload = 40,800 + 1,020 =
  \textbf{41,820 bytes}
\item
  With 25\% DNP3 overhead: Total = 41,820 × 1.25 = \textbf{52,275 bytes
  per scan}
\item
  Minimum bandwidth: Throughput = 52,275 bytes / 6 s = 8,712.5 bytes/s
  Bandwidth = 8,712.5 × 8 = 69,700 bps = \textbf{69.7 kbps}
\end{enumerate}

\begin{center}\rule{0.5\linewidth}{0.5pt}\end{center}

\section{Problem 1.8.2}\label{problem-1.8.2}

\textbf{Given:} A SCADA system uses DNP3 polling with a master station
that communicates with 200 RTUs. Each RTU poll-response transaction
takes 45 ms (including request, response, and processing). The utility
requires a maximum scan cycle time of 10 seconds.

\textbf{Find:} (a) The time to poll all RTUs sequentially, (b) whether
the scan cycle requirement is met, (c) the minimum number of parallel
communication channels needed to meet the requirement, and (d) the RTU
polling rate per channel.

\textbf{Solution:}

\begin{enumerate}
\def\labelenumi{(\alph{enumi})}
\item
  Sequential polling time: T\textsubscript{seq} = 200 × 45 ms = 9,000 ms
  = \textbf{9.0 s}
\item
  9.0 s \textless{} 10.0 s, so the requirement \textbf{is met} with
  sequential polling (with 1.0 s margin).
\item
  Since 9.0 s \textless{} 10.0 s, the minimum is \textbf{1 channel}.
  However, for redundancy and margin, consider that if RTU count grows
  to 250: T\textsubscript{seq} = 250 × 45 ms = 11.25 s \textgreater{} 10
  s Minimum channels = ⌈11.25 / 10⌉ = \textbf{2 channels}
\item
  With 2 channels and 200 RTUs: RTUs per channel = 200 / 2 = 100 Polling
  rate per channel = 100 / 10 s = \textbf{10 RTUs/s}
\end{enumerate}

\begin{center}\rule{0.5\linewidth}{0.5pt}\end{center}

\section{Problem 1.8.3}\label{problem-1.8.3}

\textbf{Given:} A substation RTU samples 64 analog channels using a
16-bit ADC with a full-scale range of ±10 V. The transducer scaling is
configured so that one channel monitors a 345 kV bus voltage through a
VT with ratio 345,000:115 V and a transducer output of 0--10 V for
0--120 V secondary.

\textbf{Find:} (a) The ADC resolution in volts, (b) the transducer
output voltage when the bus voltage is 342 kV, (c) the corresponding ADC
digital count (assuming unipolar 0--10 V maps to 0--65535), and (d) the
measurement resolution in kV at the primary bus.

\textbf{Solution:}

\begin{enumerate}
\def\labelenumi{(\alph{enumi})}
\item
  ADC resolution (full bipolar range): Resolution = 20 V / 2¹⁶ = 20 /
  65,536 = \textbf{305.2 μV/count}
\item
  VT secondary voltage at 342 kV primary: V\textsubscript{sec} = 342,000
  × (115 / 345,000) = 342,000 / 3,000 = 114.0 V
\end{enumerate}

Transducer output (0--10 V for 0--120 V): V\textsubscript{transducer} =
(114.0 / 120) × 10 = \textbf{9.500 V}

\begin{enumerate}
\def\labelenumi{(\alph{enumi})}
\setcounter{enumi}{2}
\item
  ADC count (unipolar 0--10 V → 0--65535): Count = (9.500 / 10) × 65,535
  = \textbf{62,258}
\item
  Primary bus resolution: One ADC count in transducer volts = 10 /
  65,536 = 152.6 μV In secondary volts = 152.6 μV × (120 / 10) = 1.831
  mV In primary kV = 1.831 × 10⁻³ × 3,000 / 1,000 = \textbf{0.00549 kV =
  5.49 V}
\end{enumerate}

\begin{center}\rule{0.5\linewidth}{0.5pt}\end{center}

\section{Problem 1.8.4}\label{problem-1.8.4}

\textbf{Given:} An IEC 61850 substation automation system has 12 IEDs
communicating over a redundant Ethernet LAN (PRP --- Parallel Redundancy
Protocol). Each IED publishes GOOSE messages at a normal rate of 1
message per second, but during a protection event the rate increases to
1 message every 4 ms for 2 seconds (exponential retransmission). Each
GOOSE frame is 150 bytes including Ethernet overhead.

\textbf{Find:} (a) The normal aggregate GOOSE traffic rate, (b) the peak
traffic rate during a protection event from a single IED, (c) the peak
aggregate rate if 3 IEDs simultaneously detect a fault, and (d) the
percentage of a 100 Mbps Ethernet link consumed during the peak event.

\textbf{Solution:}

\begin{enumerate}
\def\labelenumi{(\alph{enumi})}
\item
  Normal aggregate traffic: Rate = 12 IEDs × 1 msg/s × 150 bytes = 1,800
  bytes/s = 1,800 × 8 = \textbf{14,400 bps = 14.4 kbps}
\item
  Peak rate from one IED: Messages per second = 1000 ms / 4 ms = 250
  msg/s Data rate = 250 × 150 = 37,500 bytes/s = 37,500 × 8 =
  \textbf{300,000 bps = 300 kbps}
\item
  Peak aggregate (3 event IEDs + 9 normal IEDs): Event traffic = 3 × 300
  kbps = 900 kbps Normal traffic = 9 × 1.2 kbps = 10.8 kbps Total = 900
  + 10.8 = \textbf{910.8 kbps}
\item
  Link utilization: Utilization = 910.8 kbps / 100,000 kbps × 100\% =
  \textbf{0.91\%}
\end{enumerate}

The low utilization confirms that GOOSE messaging has minimal impact on
network capacity, even during fault events.

\begin{center}\rule{0.5\linewidth}{0.5pt}\end{center}

\section{Problem 1.8.5}\label{problem-1.8.5}

\textbf{Given:} A SCADA master station performs state estimation using
telemetered measurements from RTUs. A particular 5-bus system has 14
measurements available (voltages, power flows, power injections) but
only 9 state variables (voltage magnitudes and angles; slack bus angle
is reference). The measurement residual vector after state estimation
has a weighted sum of squared residuals (objective function) J = 8.73.

\textbf{Find:} (a) The measurement redundancy ratio, (b) the degrees of
freedom for the chi-squared (χ²) bad data test, (c) whether the
measurement set passes the χ² test at the 95\% confidence level
(χ²\textsubscript{0.05,5} = 11.07), and (d) the minimum number of
measurements that could be lost while still maintaining observability
(redundancy \textgreater{} 1.0).

\textbf{Solution:}

\begin{enumerate}
\def\labelenumi{(\alph{enumi})}
\item
  Measurement redundancy ratio: Redundancy = m / n = 14 / 9 =
  \textbf{1.556}
\item
  Degrees of freedom: ν = m − n = 14 − 9 = \textbf{5}
\item
  Chi-squared test: J = 8.73 \textless{} χ²\textsubscript{0.05,5} =
  11.07 Since J \textless{} threshold, the measurement set
  \textbf{passes} the bad data test --- no gross errors detected at 95\%
  confidence.
\item
  Minimum measurements for observability: For observability, m ≥ n, so
  m\textsubscript{min} = 9 Maximum measurements that can be lost = 14 −
  9 = \textbf{5 measurements}
\end{enumerate}

(After losing 5, redundancy = 9/9 = 1.0 --- the system is barely
observable with zero redundancy for bad data detection.)

\begin{center}\rule{0.5\linewidth}{0.5pt}\end{center}

\section{Problem 1.8.6}\label{problem-1.8.6}

\textbf{Given:} A utility's SCADA cybersecurity monitoring system
processes an average of 150,000 network packets per hour on the OT
network. The IDS has a true positive rate (sensitivity) of 99.2\% and a
false positive rate of 0.05\%. During a 12-hour shift, 8 actual
malicious packets are injected by a penetration tester.

\textbf{Find:} (a) The total packets processed in 12 hours, (b) the
expected true positives, (c) the expected false positives, (d) the
precision (positive predictive value), and (e) the F1 score.

\textbf{Solution:}

\begin{enumerate}
\def\labelenumi{(\alph{enumi})}
\item
  Total packets in 12 hours: N = 150,000 × 12 = \textbf{1,800,000
  packets}
\item
  True positives: TP = 0.992 × 8 = \textbf{7.94 ≈ 8}
\item
  False positives: Legitimate packets = 1,800,000 − 8 = 1,799,992 FP =
  0.0005 × 1,799,992 = \textbf{900.0}
\item
  Precision: Precision = TP / (TP + FP) = 8 / (8 + 900) = 8 / 908 =
  \textbf{0.00881 = 0.88\%}
\item
  F1 score: Recall = TP / (TP + FN) = 8 / (8 + 0) = 1.0 (assuming all 8
  detected) F1 = 2 × (Precision × Recall) / (Precision + Recall) F1 = 2
  × (0.00881 × 1.0) / (0.00881 + 1.0) = 0.01762 / 1.00881 =
  \textbf{0.01747 ≈ 1.75\%}
\end{enumerate}

The extremely low F1 score highlights the base-rate paradox: the
overwhelming number of legitimate packets makes even a very low false
positive rate generate far more false alarms than true detections.

\begin{center}\rule{0.5\linewidth}{0.5pt}\end{center}

\section{Problem 1.8.7}\label{problem-1.8.7}

\textbf{Given:} A SCADA communication network uses three independent
paths between a control center and a critical substation: a primary
fiber link (availability 99.95\%), a backup microwave link (availability
99.80\%), and an emergency cellular link (availability 99.50\%). The
system uses automatic failover --- it fails only if all three paths are
simultaneously unavailable.

\textbf{Find:} (a) The unavailability of each individual link, (b) the
combined unavailability (probability all three are down simultaneously),
(c) the combined availability, and (d) the expected downtime per year
for each link and for the combined system.

\textbf{Solution:}

\begin{enumerate}
\def\labelenumi{(\alph{enumi})}
\item
  Individual unavailabilities: U\textsubscript{fiber} = 1 − 0.9995 =
  \textbf{0.0005} U\textsubscript{microwave} = 1 − 0.9980 =
  \textbf{0.0020} U\textsubscript{cellular} = 1 − 0.9950 =
  \textbf{0.0050}
\item
  Combined unavailability (independent failures):
  U\textsubscript{combined} = U₁ × U₂ × U₃ = 0.0005 × 0.0020 × 0.0050 =
  \textbf{5.0 × 10⁻⁹}
\item
  Combined availability: A\textsubscript{combined} = 1 − 5.0 × 10⁻⁹ =
  \textbf{0.999999995 (nine 9's)}
\item
  Expected downtime per year (8,760 hours): Fiber: 0.0005 × 8,760 =
  \textbf{4.38 hours/year} Microwave: 0.0020 × 8,760 = \textbf{17.52
  hours/year} Cellular: 0.0050 × 8,760 = \textbf{43.80 hours/year}
  Combined: 5.0 × 10⁻⁹ × 8,760 = 4.38 × 10⁻⁵ hours = \textbf{0.158
  seconds/year}
\end{enumerate}

\begin{center}\rule{0.5\linewidth}{0.5pt}\end{center}

\section{Problem 1.8.8}\label{problem-1.8.8}

\textbf{Given:} A DNP3 Secure Authentication implementation adds a
32-byte HMAC (Hash-based Message Authentication Code) challenge-response
to each critical control command. The control center sends an average of
80 supervisory control commands per hour. Each authenticated command
exchange consists of: (1) command request (60 bytes), (2) authentication
challenge from RTU (48 bytes), (3) HMAC response from master (92 bytes),
and (4) command execution confirmation (40 bytes). The link operates at
9,600 bps (serial).

\textbf{Find:} (a) The total bytes per authenticated command exchange,
(b) the time to complete one authenticated exchange, (c) the total
authentication traffic per hour, and (d) the percentage of link capacity
consumed by authenticated control traffic.

\textbf{Solution:}

\begin{enumerate}
\def\labelenumi{(\alph{enumi})}
\item
  Total bytes per exchange: 60 + 48 + 92 + 40 = \textbf{240 bytes}
\item
  Time per exchange: Bits = 240 × 8 = 1,920 bits Transmission time =
  1,920 / 9,600 = \textbf{0.200 s = 200 ms}
\end{enumerate}

(This is transmission time only; actual round-trip includes propagation
and processing delays.)

\begin{enumerate}
\def\labelenumi{(\alph{enumi})}
\setcounter{enumi}{2}
\item
  Traffic per hour: Bytes/hour = 80 × 240 = 19,200 bytes/hour Bits/hour
  = 19,200 × 8 = \textbf{153,600 bits/hour}
\item
  Link utilization: Link capacity per hour = 9,600 × 3,600 = 34,560,000
  bits/hour Utilization = 153,600 / 34,560,000 × 100\% = \textbf{0.44\%}
\end{enumerate}

\begin{center}\rule{0.5\linewidth}{0.5pt}\end{center}

\section{Problem 1.8.9}\label{problem-1.8.9}

\textbf{Given:} A SCADA system uses report-by-exception (RBE) to reduce
communication traffic. An RTU monitors 200 analog points with a deadband
of 0.5\% of full scale. Under normal steady-state conditions, 5\% of
analog points exceed their deadband per scan cycle. During a system
disturbance, 60\% of points change per scan. The scan cycle is 4
seconds, each analog report is 12 bytes (including timestamp and quality
flags), and the link operates at 56 kbps.

\textbf{Find:} (a) The steady-state report rate (reports per scan), (b)
the steady-state data rate, (c) the disturbance report rate, (d) the
disturbance data rate, and (e) the bandwidth reduction factor of RBE
versus polling all points every scan.

\textbf{Solution:}

\begin{enumerate}
\def\labelenumi{(\alph{enumi})}
\item
  Steady-state reports per scan: Reports = 0.05 × 200 = \textbf{10
  reports/scan}
\item
  Steady-state data rate: Data per scan = 10 × 12 = 120 bytes Rate = 120
  / 4 = 30 bytes/s = 30 × 8 = \textbf{240 bps}
\item
  Disturbance reports per scan: Reports = 0.60 × 200 = \textbf{120
  reports/scan}
\item
  Disturbance data rate: Data per scan = 120 × 12 = 1,440 bytes Rate =
  1,440 / 4 = 360 bytes/s = 360 × 8 = \textbf{2,880 bps}
\item
  Full polling data rate (all 200 points every scan): Data per scan =
  200 × 12 = 2,400 bytes Rate = 2,400 / 4 = 600 bytes/s = 4,800 bps
\end{enumerate}

RBE reduction factor: Steady-state: 4,800 / 240 = \textbf{20× reduction}
Disturbance: 4,800 / 2,880 = \textbf{1.67× reduction}

Even during disturbances, RBE reduces traffic by 40\%. Under normal
conditions, RBE achieves a 95\% bandwidth savings.

\begin{center}\rule{0.5\linewidth}{0.5pt}\end{center}

\section{Problem 1.8.10}\label{problem-1.8.10}

\textbf{Given:} A NERC CIP compliance audit requires a utility to
demonstrate that all Electronic Security Perimeter (ESP) access points
have logging enabled. The utility has 3 control centers and 45
substations classified as medium-impact BES Cyber Systems. Each control
center has 4 ESP access points (firewalls), and each substation has 2
ESP access points. Each access point generates an average of 500 log
entries per day. Log entries are retained for 90 days as required by
CIP-007. Each log entry averages 250 bytes.

\textbf{Find:} (a) The total number of ESP access points, (b) the total
log entries generated per day, (c) the total storage required for 90-day
retention, and (d) the storage required if the utility compresses logs
at a 10:1 ratio.

\textbf{Solution:}

\begin{enumerate}
\def\labelenumi{(\alph{enumi})}
\item
  Total ESP access points: Control centers: 3 × 4 = 12 Substations: 45 ×
  2 = 90 Total = 12 + 90 = \textbf{102 access points}
\item
  Log entries per day: Entries = 102 × 500 = \textbf{51,000 entries/day}
\item
  Storage for 90-day retention: Daily storage = 51,000 × 250 bytes =
  12,750,000 bytes = 12.75 MB/day 90-day storage = 12.75 × 90 =
  \textbf{1,147.5 MB ≈ 1.15 GB}
\item
  With 10:1 compression: Compressed = 1,147.5 / 10 = \textbf{114.75 MB ≈
  115 MB}
\end{enumerate}

\chapter{Chapter 2 --- Section 2.1: Analog Signal
Processing}\label{chapter-2-section-2.1-analog-signal-processing}

Practice problems covering AM radio and FM radio fundamentals.

\begin{center}\rule{0.5\linewidth}{0.5pt}\end{center}

\section{Problem 2.1.1}\label{problem-2.1.1}

\textbf{Given:} An AM transmitter has a carrier power of 25 kW and is
modulated by a single-tone audio signal with a modulation index of m =
0.85.

\textbf{Find:} (a) The total transmitted power, (b) the power in each
sideband, (c) the efficiency (fraction of power carrying information),
and (d) the total power if the modulation index is increased to m = 1.0
(100\% modulation).

\textbf{Solution:}

\begin{enumerate}
\def\labelenumi{(\alph{enumi})}
\item
  Total power: P\textsubscript{total} = P\textsubscript{c}(1 + m²/2) =
  25(1 + 0.85²/2) = 25(1 + 0.3613) = 25 × 1.3613 = \textbf{34.03 kW}
\item
  Power in both sidebands: P\textsubscript{sb} = P\textsubscript{c} ×
  m²/2 = 25 × 0.7225/2 = 25 × 0.3613 = 9.03 kW Each sideband:
  P\textsubscript{each} = 9.03/2 = \textbf{4.52 kW}
\item
  Efficiency: η = P\textsubscript{sb}/P\textsubscript{total} =
  9.03/34.03 = 0.2654 = \textbf{26.5\%}
\item
  At m = 1.0: P\textsubscript{total} = 25(1 + 1.0/2) = 25 × 1.5 =
  \textbf{37.5 kW} Efficiency = (25 × 0.5)/37.5 = 12.5/37.5 =
  \textbf{33.3\%}
\end{enumerate}

Even at 100\% modulation, two-thirds of the power is wasted in the
carrier.

\begin{center}\rule{0.5\linewidth}{0.5pt}\end{center}

\section{Problem 2.1.2}\label{problem-2.1.2}

\textbf{Given:} A DSB-SC (Double Sideband Suppressed Carrier)
transmitter produces a total power of 10 kW with a 5 kHz audio signal.
The carrier frequency is 1 MHz.

\textbf{Find:} (a) The bandwidth of the transmitted signal, (b) the
power in each sideband, (c) the efficiency compared to standard AM with
m = 1.0 transmitting the same sideband power, and (d) the required
receiver complexity difference.

\textbf{Solution:}

\begin{enumerate}
\def\labelenumi{(\alph{enumi})}
\item
  Bandwidth: B = 2 × f\textsubscript{m} = 2 × 5,000 = \textbf{10 kHz}
\item
  Since the carrier is suppressed, all power is in the sidebands:
  P\textsubscript{each sideband} = 10/2 = \textbf{5 kW each}
\item
  For standard AM at m = 1.0 to deliver 10 kW of sideband power:
  P\textsubscript{sb} = P\textsubscript{c} × m²/2 = P\textsubscript{c} ×
  0.5 So P\textsubscript{c} = 10/0.5 = 20 kW P\textsubscript{total,AM} =
  20 + 10 = 30 kW
\end{enumerate}

DSB-SC needs 10 kW vs.~AM needing 30 kW for the same information power.
Efficiency improvement = 30/10 = \textbf{3x (or 4.77 dB)}

\begin{enumerate}
\def\labelenumi{(\alph{enumi})}
\setcounter{enumi}{3}
\tightlist
\item
  DSB-SC requires \textbf{coherent detection} (a local oscillator
  synchronized to the carrier frequency and phase), unlike AM which uses
  a simple envelope detector. This is why AM broadcasting uses standard
  AM despite its inefficiency -- receiver simplicity was critical for
  mass adoption.
\end{enumerate}

\begin{center}\rule{0.5\linewidth}{0.5pt}\end{center}

\section{Problem 2.1.3}\label{problem-2.1.3}

\textbf{Given:} An FM signal has a carrier frequency of 100 MHz and
modulates with an audio signal of bandwidth 20 kHz. The maximum
frequency deviation is Δf = 50 kHz.

\textbf{Find:} (a) The modulation index, (b) Carson's rule bandwidth,
(c) the number of significant sideband pairs (for β = 2.5, approximately
5 pairs), and (d) the exact bandwidth from significant sidebands.

\textbf{Solution:}

\begin{enumerate}
\def\labelenumi{(\alph{enumi})}
\item
  Modulation index: β = Δf/f\textsubscript{m} = 50,000/20,000 =
  \textbf{2.5}
\item
  Carson's rule: B = 2(Δf + f\textsubscript{m}) = 2(50,000 + 20,000) =
  \textbf{140 kHz}
\item
  For β = 2.5, the Bessel function table shows approximately \textbf{5
  significant sideband pairs} (J₀ through J₅ with magnitude
  \textgreater{} 0.01).
\item
  Exact bandwidth: B\textsubscript{exact} = 2 × 5 × f\textsubscript{m} =
  2 × 5 × 20,000 = \textbf{200 kHz}
\end{enumerate}

This is a wideband FM signal (β \textgreater{} 1). Carson's rule
captures about 98\% of the power in 140 kHz, while the full sideband
extent is 200 kHz.

\begin{center}\rule{0.5\linewidth}{0.5pt}\end{center}

\section{Problem 2.1.4}\label{problem-2.1.4}

\textbf{Given:} An FM broadcast system uses pre-emphasis with a time
constant of τ = 75 μs (North American standard) and de-emphasis at the
receiver. The audio bandwidth extends to 15 kHz.

\textbf{Find:} (a) The pre-emphasis corner frequency, (b) the
pre-emphasis boost at 15 kHz relative to 1 kHz, (c) the SNR improvement
provided by the pre-emphasis/de-emphasis system (approximate), and (d)
the effective modulation index if Δf = 75 kHz at the maximum
pre-emphasized frequency.

\textbf{Solution:}

\begin{enumerate}
\def\labelenumi{(\alph{enumi})}
\item
  Corner frequency: f\textsubscript{c} = 1/(2πτ) = 1/(2π × 75 × 10⁻⁶) =
  \textbf{2,122 Hz}
\item
  Pre-emphasis is a first-order high-pass characteristic: Boost at
  frequency f = 20 log₁₀(√(1 + (f/f\textsubscript{c})²)) At 15 kHz:
  boost = 20 log₁₀(√(1 + (15,000/2,122)²)) = 20 log₁₀(√(1 + 50.0)) = 20
  log₁₀(7.14) = \textbf{17.1 dB} At 1 kHz: boost = 20 log₁₀(√(1 +
  (1,000/2,122)²)) = 20 log₁₀(√(1 + 0.222)) = 20 log₁₀(1.105) =
  \textbf{0.87 dB} Relative boost at 15 kHz vs 1 kHz: 17.1 - 0.87 =
  \textbf{16.2 dB}
\item
  The SNR improvement from pre-emphasis/de-emphasis is approximately:
  SNR improvement ≈ (f\textsubscript{m}/f\textsubscript{c})² / 3 (for
  high-frequency emphasis) = (15,000/2,122)² / 3 = 50/3 = 16.7 =
  \textbf{12.2 dB}
\item
  β = Δf/f\textsubscript{m} = 75,000/15,000 = \textbf{5} (same as
  standard FM broadcast)
\end{enumerate}

\begin{center}\rule{0.5\linewidth}{0.5pt}\end{center}

\section{Problem 2.1.5}\label{problem-2.1.5}

\textbf{Given:} An SSB (Single Sideband) transmitter operates with a
carrier frequency of 14.2 MHz and transmits only the upper sideband. The
audio bandwidth is 300 Hz to 3,000 Hz.

\textbf{Find:} (a) The transmitted bandwidth, (b) the frequency range of
the transmitted signal, (c) the power savings compared to standard AM (m
= 1.0) for a peak envelope power (PEP) of 100 W, and (d) the bandwidth
savings compared to standard AM.

\textbf{Solution:}

\begin{enumerate}
\def\labelenumi{(\alph{enumi})}
\item
  Bandwidth: B = f\textsubscript{max} - f\textsubscript{min} = 3,000 -
  300 = \textbf{2,700 Hz}
\item
  USB frequency range: f\textsubscript{lower} = f\textsubscript{c} + 300
  = 14,200,300 Hz f\textsubscript{upper} = f\textsubscript{c} + 3,000 =
  14,203,000 Hz Range: \textbf{14.2003 to 14.203 MHz}
\item
  SSB PEP = 100 W. All power carries information. For AM at m = 1.0 with
  equivalent sideband power: The SSB signal has power in one sideband =
  100 W PEP. AM total power for same PEP sideband: P\textsubscript{c} =
  2 × P\textsubscript{ssb}/m² × 2 = needs clarification. More directly:
  AM at m = 1.0 has P\textsubscript{total} = 1.5 × P\textsubscript{c},
  with P\textsubscript{sb} = 0.5 × P\textsubscript{c} total in both
  sidebands. For the same information as SSB at 100 W:
  P\textsubscript{c} = 2 × 100/0.5 = 400 W, P\textsubscript{total} = 600
  W. Power savings: (600 - 100)/600 = \textbf{83.3\% power savings}
\item
  AM bandwidth = 2 × 3,000 = 6,000 Hz. SSB bandwidth = 2,700 Hz.
  Savings: (6,000 - 2,700)/6,000 = \textbf{55\% bandwidth savings}
\end{enumerate}

\chapter{Chapter 2 --- Section 2.2: Digital Signal
Processing}\label{chapter-2-section-2.2-digital-signal-processing}

Practice problems covering Fourier analysis, Nyquist sampling,
quantization, encoding, DFT/FFT, SQNR, and analog-to-digital conversion.

\begin{center}\rule{0.5\linewidth}{0.5pt}\end{center}

\section{Problem 2.2.1}\label{problem-2.2.1}

\textbf{Given:} A periodic triangular wave has a fundamental frequency
of 500 Hz and a peak amplitude of 3 V. The Fourier series of a
triangular wave contains only odd harmonics with coefficients
a\textsubscript{n} = 8A/(n²π²) for n = 1, 3, 5, \ldots{}

\textbf{Find:} (a) The amplitude of the 1st, 3rd, and 5th harmonic
components, (b) the frequency of each component, and (c) the RMS voltage
of the signal truncated to these three harmonics.

\textbf{Solution:}

\begin{enumerate}
\def\labelenumi{(\alph{enumi})}
\item
  Harmonic amplitudes (peak): 1st harmonic: a₁ = 8 × 3 / (1² × π²) = 24
  / 9.8696 = \textbf{2.431 V} 3rd harmonic: a₃ = 8 × 3 / (3² × π²) = 24
  / 88.826 = \textbf{0.270 V} 5th harmonic: a₅ = 8 × 3 / (5² × π²) = 24
  / 246.740 = \textbf{0.0973 V}
\item
  Frequencies: 1st harmonic: f₁ = \textbf{500 Hz} 3rd harmonic: f₃ = 3 ×
  500 = \textbf{1,500 Hz} 5th harmonic: f₅ = 5 × 500 = \textbf{2,500 Hz}
\item
  RMS voltage of truncated signal: V\textsubscript{rms} = √(a₁²/2 +
  a₃²/2 + a₅²/2) = √(2.431²/2 + 0.270²/2 + 0.0973²/2)
  V\textsubscript{rms} = √(2.955 + 0.0365 + 0.00473) = √2.996 =
  \textbf{1.731 V}
\end{enumerate}

The true RMS of a triangular wave is A/√3 = 3/√3 = 1.732 V, so the
three-harmonic approximation captures (1.731/1.732)² = 99.9\% of the
total power --- triangular waves converge very rapidly due to the 1/n²
coefficient rolloff.

\begin{center}\rule{0.5\linewidth}{0.5pt}\end{center}

\section{Problem 2.2.2}\label{problem-2.2.2}

\textbf{Given:} A signal contains three frequency components: 2 kHz, 5
kHz, and 11 kHz. It is sampled at a rate of 20 kHz.

\textbf{Find:} (a) The Nyquist rate for this signal, (b) the Nyquist
frequency (folding frequency) of the sampling system, (c) whether
aliasing occurs, and (d) the aliased frequency of any component that
aliases.

\textbf{Solution:}

\begin{enumerate}
\def\labelenumi{(\alph{enumi})}
\item
  Nyquist rate: f\textsubscript{Nyquist rate} = 2 × f\textsubscript{max}
  = 2 × 11,000 = \textbf{22 kHz}
\item
  Nyquist frequency (folding frequency): f\textsubscript{N} =
  f\textsubscript{s} / 2 = 20,000 / 2 = \textbf{10 kHz}
\item
  The 2 kHz and 5 kHz components are below the Nyquist frequency of 10
  kHz, so they are sampled correctly. The 11 kHz component exceeds 10
  kHz. \textbf{Aliasing occurs} for the 11 kHz component.
\item
  The aliased frequency of the 11 kHz component: f\textsubscript{alias}
  = f\textsubscript{s} − f\textsubscript{signal} = 20,000 − 11,000 =
  \textbf{9 kHz}
\end{enumerate}

The 11 kHz component appears as a spurious 9 kHz tone in the sampled
data, corrupting any legitimate signal content near 9 kHz. An
anti-aliasing filter with a cutoff below 10 kHz would prevent this.

\begin{center}\rule{0.5\linewidth}{0.5pt}\end{center}

\section{Problem 2.2.3}\label{problem-2.2.3}

\textbf{Given:} A 12-bit ADC has an input range of 0 to 5 V (unipolar).
A full-scale sinusoidal signal is applied.

\textbf{Find:} (a) The number of quantization levels, (b) the
quantization step size, (c) the maximum quantization error, (d) the SQNR
for a full-scale sinusoid, and (e) the dynamic range in dB.

\textbf{Solution:}

\begin{enumerate}
\def\labelenumi{(\alph{enumi})}
\item
  Number of levels: 2¹² = \textbf{4,096 levels}
\item
  Step size: Δ = V\textsubscript{range} / 2\textsuperscript{N} = 5 /
  4,096 = \textbf{1.221 mV}
\item
  Maximum quantization error: ±Δ/2 = ±0.610 mV = \textbf{±0.610 mV}
\item
  SQNR = 6.02N + 1.76 = 6.02 × 12 + 1.76 = 72.24 + 1.76 = \textbf{74.0
  dB}
\item
  Dynamic range = 20 log₁₀(2\textsuperscript{N}) = 20 log₁₀(4,096) = 20
  × 3.6124 = \textbf{72.2 dB}
\end{enumerate}

The 74 dB SQNR is adequate for high-quality audio recording but falls
short of the 96 dB achieved by 16-bit CD-quality systems.

\begin{center}\rule{0.5\linewidth}{0.5pt}\end{center}

\section{Problem 2.2.4}\label{problem-2.2.4}

\textbf{Given:} An 8-bit μ-law companded PCM system (μ = 255) digitizes
voice at 8 kHz sampling rate. The input signal range is ±1 V.

\textbf{Find:} (a) The bit rate, (b) the equivalent dynamic range of the
companded system (μ-law provides approximately 13-bit equivalent dynamic
range with 8 bits), (c) the step size at the smallest signal level (near
zero), and (d) the step size at the largest signal level (near ±1 V).

\textbf{Solution:}

\begin{enumerate}
\def\labelenumi{(\alph{enumi})}
\item
  Bit rate: R\textsubscript{b} = N × f\textsubscript{s} = 8 × 8,000 =
  \textbf{64 kbps}
\item
  The dynamic range of μ-law with μ = 255: DR = 6.02 × 8 + 1.76 + 10
  log₁₀(3(μ)² / (ln(1 + μ))²) The compression gain: 10 log₁₀(3 × 255² /
  (ln(256))²) = 10 log₁₀(195,075 / 30.68) = 10 log₁₀(6,358) = 38.0 dB
  Total DR ≈ 49.9 + 38.0 − 38.0 ≈ \textbf{72 dB} (equivalent to
  approximately 12 uniform bits)
\end{enumerate}

More directly, μ-law companding with 8 bits achieves an SQNR that is
approximately constant at \textbf{38 dB} over a 40 dB input dynamic
range, equivalent to roughly 12--13 uniform bits for small signals.

\begin{enumerate}
\def\labelenumi{(\alph{enumi})}
\setcounter{enumi}{2}
\item
  Near zero, the compression curve slope is steepest. The effective step
  size is: Δ\textsubscript{min} ≈ (2V\textsubscript{max}) /
  (2\textsuperscript{N}) × 1/(ln(1 + μ)) = 2 / 256 × 1/ln(256) = 0.00781
  × 0.1803 = \textbf{1.41 mV}
\item
  Near full scale, the compression curve slope is flattest. The
  effective step size is: Δ\textsubscript{max} ≈ (2V\textsubscript{max})
  / (2\textsuperscript{N}) × (1 + μ)/(ln(1 + μ)) = 2 / 256 × 256/5.55 =
  0.00781 × 46.1 = \textbf{360 mV}
\end{enumerate}

The ratio Δ\textsubscript{max}/Δ\textsubscript{min} = 360/1.41 = 255 =
μ, confirming that the step size varies by a factor of μ across the
input range.

\begin{center}\rule{0.5\linewidth}{0.5pt}\end{center}

\section{Problem 2.2.5}\label{problem-2.2.5}

\textbf{Given:} A 1,024-point FFT is applied to a signal sampled at
f\textsubscript{s} = 48 kHz. The signal contains tones at 3 kHz and 7.5
kHz.

\textbf{Find:} (a) The frequency resolution (bin spacing), (b) the FFT
bin indices corresponding to the two tones, (c) the total number of
unique frequency bins (considering the Nyquist symmetry), and (d) the
maximum unambiguous frequency.

\textbf{Solution:}

\begin{enumerate}
\def\labelenumi{(\alph{enumi})}
\item
  Frequency resolution: Δf = f\textsubscript{s} / N = 48,000 / 1,024 =
  \textbf{46.875 Hz}
\item
  Bin indices: For 3 kHz: k₁ = f₁ / Δf = 3,000 / 46.875 = \textbf{64}
  For 7.5 kHz: k₂ = f₂ / Δf = 7,500 / 46.875 = \textbf{160}
\item
  Due to the conjugate symmetry of the FFT for real signals, only the
  first N/2 + 1 bins contain unique information: Unique bins = 1,024/2 +
  1 = \textbf{513 bins} (from DC through Nyquist)
\item
  Maximum unambiguous frequency: f\textsubscript{max} =
  f\textsubscript{s} / 2 = 48,000 / 2 = \textbf{24 kHz} (bin 512)
\end{enumerate}

The FFT computation requires N log₂(N) = 1,024 × 10 = 10,240 complex
multiply-accumulate operations, compared to N² = 1,048,576 for a direct
DFT --- a 102× speedup.

\begin{center}\rule{0.5\linewidth}{0.5pt}\end{center}

\section{Problem 2.2.6}\label{problem-2.2.6}

\textbf{Given:} A biomedical signal (ECG) has a bandwidth of 150 Hz. It
is digitized using a 10-bit ADC with a ±5 mV input range. The sampling
rate is set to 500 Hz.

\textbf{Find:} (a) Whether the sampling rate meets the Nyquist
criterion, (b) the quantization step size in μV, (c) the SQNR, (d) the
data rate in bps, and (e) the data volume for a 24-hour Holter monitor
recording.

\textbf{Solution:}

\begin{enumerate}
\def\labelenumi{(\alph{enumi})}
\item
  Nyquist rate = 2 × 150 = 300 Hz. The sampling rate of 500 Hz exceeds
  300 Hz, so \textbf{the Nyquist criterion is met} with a guard band of
  500/2 − 150 = 100 Hz.
\item
  Step size: Δ = (5 − (−5)) mV / 2¹⁰ = 10 mV / 1,024 = \textbf{9.766 μV}
\item
  SQNR = 6.02 × 10 + 1.76 = 60.2 + 1.76 = \textbf{61.96 dB}
\item
  Data rate: R = N × f\textsubscript{s} = 10 × 500 = \textbf{5,000 bps =
  5 kbps}
\item
  Data volume for 24 hours: Volume = 5,000 × 24 × 3,600 = 432,000,000
  bits = 432 Mbits In bytes: 432,000,000 / 8 = 54,000,000 bytes =
  \textbf{54 MB}
\end{enumerate}

With lossless compression (typically 2:1 for ECG), this reduces to
approximately 27 MB --- easily stored on modern portable devices.

\begin{center}\rule{0.5\linewidth}{0.5pt}\end{center}

\section{Problem 2.2.7}\label{problem-2.2.7}

\textbf{Given:} A real-valued signal is sampled at 10 kHz and processed
with a 256-point DFT. The magnitude spectrum shows a peak at bin k = 40.

\textbf{Find:} (a) The frequency corresponding to bin 40, (b) the
minimum frequency separation between two tones that can be resolved by
this DFT, (c) the total observation time (window length), and (d) the
bin index where a 2,800 Hz tone would appear.

\textbf{Solution:}

\begin{enumerate}
\def\labelenumi{(\alph{enumi})}
\item
  Frequency at bin k: f = k × f\textsubscript{s} / N = 40 × 10,000 / 256
  = 400,000 / 256 = \textbf{1,562.5 Hz}
\item
  Minimum resolvable frequency separation equals the bin spacing: Δf =
  f\textsubscript{s} / N = 10,000 / 256 = \textbf{39.0625 Hz}
\end{enumerate}

Two tones must be separated by at least 39.06 Hz to appear in distinct
bins. With windowing (e.g., Hanning window), the effective resolution is
approximately 2Δf = 78.1 Hz due to main lobe widening.

\begin{enumerate}
\def\labelenumi{(\alph{enumi})}
\setcounter{enumi}{2}
\item
  Observation time: T = N / f\textsubscript{s} = 256 / 10,000 =
  \textbf{25.6 ms}
\item
  Bin index for 2,800 Hz: k = f × N / f\textsubscript{s} = 2,800 × 256 /
  10,000 = 71.68
\end{enumerate}

Since this is not an integer, the energy of the 2,800 Hz tone will be
spread across adjacent bins (spectral leakage), with the peak at
\textbf{bin 72} and significant energy in bins 71 and 73. Windowing
reduces the sidelobe leakage at the cost of wider main lobe width.

\begin{center}\rule{0.5\linewidth}{0.5pt}\end{center}

\section{Problem 2.2.8}\label{problem-2.2.8}

\textbf{Given:} A 14-bit ADC samples a signal at 100 MSPS (mega-samples
per second). The ADC has an effective number of bits (ENOB) of 11.5 due
to nonlinearities and clock jitter.

\textbf{Find:} (a) The ideal SQNR for 14 bits, (b) the actual SQNR based
on the ENOB, (c) the SQNR degradation due to imperfections, (d) the
Nyquist bandwidth, and (e) the raw data throughput.

\textbf{Solution:}

\begin{enumerate}
\def\labelenumi{(\alph{enumi})}
\item
  Ideal SQNR = 6.02 × 14 + 1.76 = 84.28 + 1.76 = \textbf{86.04 dB}
\item
  Actual SQNR = 6.02 × ENOB + 1.76 = 6.02 × 11.5 + 1.76 = 69.23 + 1.76 =
  \textbf{70.99 dB}
\item
  Degradation = 86.04 − 70.99 = \textbf{15.05 dB}
\end{enumerate}

This 15 dB degradation (equivalent to 2.5 lost bits) is caused by
differential nonlinearity (DNL), integral nonlinearity (INL), and
aperture jitter of the sampling clock.

\begin{enumerate}
\def\labelenumi{(\alph{enumi})}
\setcounter{enumi}{3}
\item
  Nyquist bandwidth = f\textsubscript{s} / 2 = 100 × 10⁶ / 2 =
  \textbf{50 MHz}
\item
  Raw data throughput = N × f\textsubscript{s} = 14 × 100 × 10⁶ =
  \textbf{1,400 Mbps = 1.4 Gbps}
\end{enumerate}

This high data rate requires careful digital interface design ---
typically using JESD204B or LVDS serial interfaces.

\begin{center}\rule{0.5\linewidth}{0.5pt}\end{center}

\section{Problem 2.2.9}\label{problem-2.2.9}

\textbf{Given:} A speech signal is band-limited to 3.4 kHz and sampled
at 8 kHz. The signal is quantized using a uniform quantizer with 256
levels (8 bits). The signal has a probability density that approximates
a Laplacian distribution with a peak-to-RMS ratio (crest factor) of 4:1,
meaning the signal uses only 1/4 of the full ADC range on average.

\textbf{Find:} (a) The Nyquist rate, (b) the SQNR for a full-scale
sinusoid, (c) the effective SQNR penalty due to the high crest factor,
and (d) the actual SQNR for the speech signal.

\textbf{Solution:}

\begin{enumerate}
\def\labelenumi{(\alph{enumi})}
\item
  Nyquist rate = 2 × 3,400 = \textbf{6,800 Hz}. The 8 kHz sampling rate
  exceeds this by 1,200 Hz.
\item
  SQNR for a full-scale sinusoid: SQNR = 6.02 × 8 + 1.76 = 48.16 + 1.76
  = \textbf{49.92 dB}
\item
  The crest factor penalty occurs because the signal's RMS value is 1/4
  of the peak value (−12 dB below full scale). This means the signal
  power is reduced while the quantization noise power stays the same:
  Crest factor penalty = 20 log₁₀(4) = \textbf{12.04 dB}
\item
  Actual SQNR for speech = 49.92 − 12.04 = \textbf{37.88 dB}
\end{enumerate}

This 38 dB SQNR is considered marginal for toll-quality speech (which
requires approximately 34--38 dB). This is exactly why telephone systems
use μ-law or A-law companding: companding matches the quantization step
size to the signal amplitude, recovering most of the 12 dB crest factor
penalty and achieving consistent SQNR across the speech dynamic range.

\begin{center}\rule{0.5\linewidth}{0.5pt}\end{center}

\section{Problem 2.2.10}\label{problem-2.2.10}

\textbf{Given:} A radar system samples return pulses using a
dual-channel (I and Q) ADC at 50 MSPS per channel with 16-bit
resolution. The pulse repetition interval (PRI) is 1 ms, and each pulse
return window is 200 μs. The system processes 128 pulses per coherent
processing interval (CPI).

\textbf{Find:} (a) The number of samples per pulse return window per
channel, (b) the total samples per CPI (both channels), (c) the total
data per CPI in bytes, (d) the frequency resolution after a 128-point
FFT across pulses (Doppler processing), and (e) the sustained data rate
during operation.

\textbf{Solution:}

\begin{enumerate}
\def\labelenumi{(\alph{enumi})}
\item
  Samples per pulse return window per channel: N\textsubscript{samples}
  = f\textsubscript{s} × T\textsubscript{window} = 50 × 10⁶ × 200 × 10⁻⁶
  = \textbf{10,000 samples}
\item
  Total samples per CPI (both I and Q channels): Total = 2 channels ×
  10,000 samples × 128 pulses = \textbf{2,560,000 samples}
\item
  Data per CPI: Data = 2,560,000 × 16 bits / 8 bits/byte = 2,560,000 × 2
  = \textbf{5,120,000 bytes = 5.12 MB}
\item
  Doppler frequency resolution: The FFT is performed across 128 pulses
  with PRI = 1 ms, so the coherent integration time is 128 × 1 ms = 128
  ms. Δf\textsubscript{Doppler} = 1 / (N\textsubscript{pulses} × PRI) =
  1 / (128 × 0.001) = \textbf{7.8125 Hz}
\item
  CPI duration = 128 × 1 ms = 128 ms. Sustained data rate = 5,120,000 ×
  8 / 0.128 = \textbf{320 Mbps} (combining both channels)
\end{enumerate}

This high data rate requires dedicated FPGA-based processing to perform
the range-Doppler FFT in real time.

\chapter{Chapter 2 --- Section 2.3: Digital
Modulation}\label{chapter-2-section-2.3-digital-modulation}

Practice problems covering ASK, FSK, PSK, QAM, constellation diagrams,
BER performance, and MIMO spatial multiplexing.

\begin{center}\rule{0.5\linewidth}{0.5pt}\end{center}

\section{Problem 2.3.1}\label{problem-2.3.1}

\textbf{Given:} A 4-ASK (4-level amplitude shift keying) system
transmits at a symbol rate of 4,800 symbols/s using raised-cosine pulse
shaping with roll-off factor α = 0.5. The four amplitude levels are 1,
3, 5, and 7 volts.

\textbf{Find:} (a) The bit rate, (b) the required bandwidth, (c) the
bandwidth efficiency, and (d) the average transmitted power (assuming
equal probability of each level and a 1 Ω load).

\textbf{Solution:}

\begin{enumerate}
\def\labelenumi{(\alph{enumi})}
\item
  Bits per symbol: log₂(4) = 2 bits/symbol Bit rate: R\textsubscript{b}
  = 2 × 4,800 = \textbf{9,600 bps}
\item
  Bandwidth: B = R\textsubscript{s}(1 + α) = 4,800 × (1 + 0.5) = 4,800 ×
  1.5 = \textbf{7,200 Hz}
\item
  Bandwidth efficiency: η = R\textsubscript{b} / B = 9,600 / 7,200 =
  \textbf{1.33 bits/s/Hz}
\item
  Average power with equal probability (p = 1/4 each):
  P\textsubscript{avg} = (1/4)(1² + 3² + 5² + 7²) / 1 = (1 + 9 + 25 +
  49) / 4 = 84 / 4 = \textbf{21 W}
\end{enumerate}

The peak-to-average power ratio is 7²/21 = 49/21 = 2.33 = \textbf{3.68
dB}, which is significantly worse than constant-envelope modulations
like FSK.

\begin{center}\rule{0.5\linewidth}{0.5pt}\end{center}

\section{Problem 2.3.2}\label{problem-2.3.2}

\textbf{Given:} A Gaussian Minimum Shift Keying (GMSK) system (as used
in GSM) has a bit rate of 270.833 kbps and a bandwidth-time product BT =
0.3. The 99\% power containment bandwidth for GMSK is approximately 1.22
× R\textsubscript{b} × BT/0.3.

\textbf{Find:} (a) The 99\% power containment bandwidth, (b) the
spectral efficiency, (c) the number of GSM channels in a 200 kHz
allocation, and (d) the comparison to standard MSK bandwidth (1.5 ×
R\textsubscript{b}).

\textbf{Solution:}

\begin{enumerate}
\def\labelenumi{(\alph{enumi})}
\tightlist
\item
  99\% power bandwidth for GMSK with BT = 0.3: B\textsubscript{99\%} =
  1.22 × R\textsubscript{b} = 1.22 × 270,833 = \textbf{330,416 Hz ≈
  330.4 kHz}
\end{enumerate}

However, GSM allocates 200 kHz per channel, which contains approximately
99.8\% of the power due to the Gaussian filtering. The occupied
bandwidth within the 200 kHz channel is effectively:
B\textsubscript{GSM} = \textbf{200 kHz} (allocated channel bandwidth)

\begin{enumerate}
\def\labelenumi{(\alph{enumi})}
\setcounter{enumi}{1}
\item
  Spectral efficiency: η = R\textsubscript{b} / B = 270,833 / 200,000 =
  \textbf{1.354 bits/s/Hz}
\item
  Each GSM carrier occupies one 200 kHz channel. Using TDMA with 8 time
  slots per carrier: Users per carrier = \textbf{8} (each user transmits
  during one time slot per frame) Total spectral efficiency = 8 ×
  270.833/8 / 200 = 270.833 / 200 = 1.354 bits/s/Hz per user slot
\item
  Standard MSK bandwidth: B\textsubscript{MSK} = 1.5 ×
  R\textsubscript{b} = 1.5 × 270,833 = \textbf{406,250 Hz} GMSK with BT
  = 0.3 reduces the bandwidth by: 330.4/406.25 = 0.813, a \textbf{18.7\%
  reduction} over standard MSK, at the cost of slight intersymbol
  interference introduced by the Gaussian filter.
\end{enumerate}

\begin{center}\rule{0.5\linewidth}{0.5pt}\end{center}

\section{Problem 2.3.3}\label{problem-2.3.3}

\textbf{Given:} An 8-PSK satellite modem operates at a symbol rate of 5
Msymbols/s with raised-cosine shaping (α = 0.2). The required
E\textsubscript{b}/N₀ for BER = 10⁻⁵ with 8-PSK is approximately 14.0
dB.

\textbf{Find:} (a) The bit rate, (b) the occupied bandwidth, (c) the
bandwidth efficiency, (d) the minimum symbol spacing in the
constellation (normalized to unit average power), and (e) the
E\textsubscript{b}/N₀ penalty compared to QPSK at the same BER.

\textbf{Solution:}

\begin{enumerate}
\def\labelenumi{(\alph{enumi})}
\item
  8-PSK: log₂(8) = 3 bits/symbol R\textsubscript{b} = 3 × 5 × 10⁶ =
  \textbf{15 Mbps}
\item
  Bandwidth: B = R\textsubscript{s}(1 + α) = 5 × 10⁶ × 1.2 = \textbf{6
  MHz}
\item
  Bandwidth efficiency: η = 15 / 6 = \textbf{2.5 bits/s/Hz}
\item
  For unit-power 8-PSK, the constellation points lie on a unit circle at
  angles 2πk/8 for k = 0, 1, \ldots, 7. Minimum distance:
  d\textsubscript{min} = 2 sin(π/8) = 2 × 0.3827 = \textbf{0.7654} For
  comparison, QPSK has d\textsubscript{min} = 2 sin(π/4) = √2 = 1.414
  for unit power.
\item
  QPSK requires E\textsubscript{b}/N₀ ≈ 9.6 dB for BER = 10⁻⁵. 8-PSK
  penalty = 14.0 − 9.6 = \textbf{4.4 dB}
\end{enumerate}

This 4.4 dB penalty is the price paid for the 50\% increase in bandwidth
efficiency (2.5 vs.~2.0 bits/s/Hz). For this reason, 16-QAM (which
achieves 4 bits/s/Hz at \textasciitilde13.4 dB) is generally preferred
over 8-PSK when amplitude variations are acceptable.

\begin{center}\rule{0.5\linewidth}{0.5pt}\end{center}

\section{Problem 2.3.4}\label{problem-2.3.4}

\textbf{Given:} A 256-QAM cable modem (DOCSIS 3.0) operates in a 6 MHz
channel with a symbol rate of 5.361 Msymbols/s. The channel SNR is
measured at 34 dB.

\textbf{Find:} (a) The bits per symbol, (b) the raw bit rate, (c) the
net data rate after 7\% FEC and framing overhead, (d) the required
E\textsubscript{b}/N₀ for BER = 10⁻⁸ with 256-QAM, and (e) whether the
measured SNR is sufficient.

\textbf{Solution:}

\begin{enumerate}
\def\labelenumi{(\alph{enumi})}
\item
  Bits per symbol: log₂(256) = \textbf{8 bits/symbol}
\item
  Raw bit rate: R\textsubscript{b} = 8 × 5.361 × 10⁶ = \textbf{42.88
  Mbps}
\item
  Net data rate: R\textsubscript{net} = 42.88 × (1 − 0.07) = 42.88 ×
  0.93 = \textbf{39.88 Mbps}
\item
  For 256-QAM, approximate BER: BER ≈ (4/8)(1 − 1/16) × Q(√(3 × 8 ×
  E\textsubscript{b} / (255 × N₀))) Setting BER = 10⁻⁸: Q(√(24
  E\textsubscript{b} / (255 N₀))) ≈ 10⁻⁸ / 0.469 = 2.13 × 10⁻⁸ Q⁻¹(2.13
  × 10⁻⁸) ≈ 5.50 24 E\textsubscript{b} / (255 N₀) = 30.25
  E\textsubscript{b}/N₀ = 30.25 × 255 / 24 = 321.4 = 10 log₁₀(321.4) =
  \textbf{25.07 dB}
\end{enumerate}

Converting to SNR: SNR = E\textsubscript{b}/N₀ + 10
log₁₀(R\textsubscript{b}/B) = 25.07 + 10 log₁₀(42.88/6) = 25.07 + 8.54 =
\textbf{33.61 dB}

\begin{enumerate}
\def\labelenumi{(\alph{enumi})}
\setcounter{enumi}{4}
\tightlist
\item
  The measured SNR of 34 dB exceeds the required 33.61 dB by 0.39 dB.
  The link \textbf{marginally meets} the requirement, but the margin is
  slim. In practice, DOCSIS systems require approximately 33 dB SNR for
  reliable 256-QAM operation, so 34 dB provides about \textbf{1 dB of
  margin}.
\end{enumerate}

\begin{center}\rule{0.5\linewidth}{0.5pt}\end{center}

\section{Problem 2.3.5}\label{problem-2.3.5}

\textbf{Given:} A wireless system operates at E\textsubscript{b}/N₀ = 20
dB. Three modulation schemes are available: QPSK (BER =
Q(√(2E\textsubscript{b}/N₀))), 16-QAM (BER ≈
(3/8)Q(√(4E\textsubscript{b}/(5N₀)))), and 64-QAM (BER ≈
(7/24)Q(√(2E\textsubscript{b}/(7N₀)))).

\textbf{Find:} (a) The BER for each scheme, (b) the bandwidth efficiency
of each, (c) the throughput at symbol rate R\textsubscript{s} = 1
Msymbol/s in a 1.25 MHz channel (α = 0.25), and (d) the highest-order
modulation meeting BER \textless{} 10⁻⁶.

\textbf{Solution:}

E\textsubscript{b}/N₀ = 20 dB = 10² = 100 (linear).

\begin{enumerate}
\def\labelenumi{(\alph{enumi})}
\tightlist
\item
  QPSK: BER = Q(√(2 × 100)) = Q(√200) = Q(14.14) BER ≈
  \textbf{\textless{} 10⁻⁴⁴} (essentially error-free)
\end{enumerate}

16-QAM: BER = (3/8) × Q(√(4 × 100/5)) = 0.375 × Q(√80) = 0.375 ×
Q(8.944) BER ≈ 0.375 × 1.8 × 10⁻¹⁹ = \textbf{6.75 × 10⁻²⁰} (essentially
error-free)

64-QAM: BER = (7/24) × Q(√(2 × 100/7)) = 0.2917 × Q(√28.57) = 0.2917 ×
Q(5.345) BER ≈ 0.2917 × 4.5 × 10⁻⁸ = \textbf{1.31 × 10⁻⁸}

\begin{enumerate}
\def\labelenumi{(\alph{enumi})}
\setcounter{enumi}{1}
\item
  Bandwidth efficiencies: QPSK = \textbf{2 bits/s/Hz}, 16-QAM =
  \textbf{4 bits/s/Hz}, 64-QAM = \textbf{6 bits/s/Hz}
\item
  With R\textsubscript{s} = 1 Msymbol/s: QPSK: 2 × 10⁶ = \textbf{2 Mbps}
  16-QAM: 4 × 10⁶ = \textbf{4 Mbps} 64-QAM: 6 × 10⁶ = \textbf{6 Mbps}
\item
  All three schemes meet BER \textless{} 10⁻⁶. The highest-order
  modulation meeting the target is \textbf{64-QAM} with BER = 1.31 ×
  10⁻⁸, providing 6 Mbps throughput. At 20 dB E\textsubscript{b}/N₀,
  64-QAM has comfortable margin.
\end{enumerate}

\begin{center}\rule{0.5\linewidth}{0.5pt}\end{center}

\section{Problem 2.3.6}\label{problem-2.3.6}

\textbf{Given:} A BPSK coherent demodulator has a carrier phase error of
θ = 15° due to imperfect carrier recovery. The channel
E\textsubscript{b}/N₀ = 10 dB. For BPSK with phase error, BER =
Q(√(2E\textsubscript{b}/N₀) × cos(θ)).

\textbf{Find:} (a) The ideal BER with no phase error, (b) the BER with
the 15° phase error, (c) the effective E\textsubscript{b}/N₀ loss due to
the phase error, and (d) the maximum tolerable phase error if BER must
remain below 10⁻⁵.

\textbf{Solution:}

E\textsubscript{b}/N₀ = 10 dB = 10 (linear).

\begin{enumerate}
\def\labelenumi{(\alph{enumi})}
\item
  Ideal BER: Q(√(2 × 10)) = Q(√20) = Q(4.472) = \textbf{3.87 × 10⁻⁶}
\item
  BER with θ = 15°: Q(√20 × cos(15°)) = Q(4.472 × 0.9659) = Q(4.320) BER
  = \textbf{7.8 × 10⁻⁶}
\item
  The effective E\textsubscript{b}/N₀ with phase error =
  (E\textsubscript{b}/N₀) × cos²(θ) = 10 × cos²(15°) = 10 × 0.9330 =
  9.330 In dB: 10 log₁₀(9.330) = 9.698 dB Loss = 10.0 − 9.698 =
  \textbf{0.30 dB}
\item
  For BER = 10⁻⁵, we need Q(√20 × cos(θ)) = 10⁻⁵. Q⁻¹(10⁻⁵) = 4.265 √20
  × cos(θ) = 4.265 cos(θ) = 4.265/4.472 = 0.9537 θ = arccos(0.9537) =
  \textbf{17.5°}
\end{enumerate}

Phase errors beyond 17.5° at this E\textsubscript{b}/N₀ will violate the
BER requirement.

\begin{center}\rule{0.5\linewidth}{0.5pt}\end{center}

\section{Problem 2.3.7}\label{problem-2.3.7}

\textbf{Given:} A 16-QAM transmitter has a constellation with minimum
distance d\textsubscript{min} = 2 (arbitrary units) and average symbol
energy E\textsubscript{avg}. The constellation is a 4×4 grid with points
at (±1, ±3) and (±3, ±1) and (±1, ±1) and (±3, ±3).

\textbf{Find:} (a) The average symbol energy, (b) the peak symbol
energy, (c) the peak-to-average power ratio (PAPR), (d) the number of
nearest neighbors for a corner point versus an inner point, and (e) the
approximate symbol error rate (SER) at E\textsubscript{s}/N₀ = 20 dB.

\textbf{Solution:}

\begin{enumerate}
\def\labelenumi{(\alph{enumi})}
\item
  The 16 constellation points have coordinates (±1, ±1), (±1, ±3), (±3,
  ±1), (±3, ±3). Energies: \textbar±1, ±1\textbar² = 2 (4 points);
  \textbar±1, ±3\textbar² = 10 (4 points); \textbar±3, ±1\textbar² = 10
  (4 points); \textbar±3, ±3\textbar² = 18 (4 points)
  E\textsubscript{avg} = (4×2 + 4×10 + 4×10 + 4×18) / 16 = (8 + 40 + 40
  + 72) / 16 = 160/16 = \textbf{10}
\item
  Peak energy (corner points): E\textsubscript{peak} = 3² + 3² =
  \textbf{18}
\item
  PAPR = E\textsubscript{peak}/E\textsubscript{avg} = 18/10 = 1.8 =
  \textbf{2.55 dB}
\item
  Corner point (3, 3): 2 nearest neighbors at distance 2 (points (1,3)
  and (3,1)). Inner point (1, 1): 4 nearest neighbors at distance 2
  (points (−1,1), (3,1), (1,−1), (1,3)). Corner points have \textbf{2
  neighbors}; inner points have \textbf{4 neighbors}.
\item
  SER ≈ 4(1 − 1/√M) × Q(√(3E\textsubscript{s}/((M−1)N₀)))
  E\textsubscript{s}/N₀ = 20 dB = 100. SER = 4(1 − 1/4) × Q(√(3 ×
  100/15)) = 3 × Q(√20) = 3 × Q(4.472) SER = 3 × 3.87 × 10⁻⁶ =
  \textbf{1.16 × 10⁻⁵}
\end{enumerate}

\begin{center}\rule{0.5\linewidth}{0.5pt}\end{center}

\section{Problem 2.3.8}\label{problem-2.3.8}

\textbf{Given:} A differential QPSK (DQPSK) system encodes data in the
phase change between consecutive symbols. The phase changes are 0°, 90°,
180°, and 270° corresponding to bit pairs 00, 01, 11, and 10. The symbol
rate is 2 Msymbols/s and the channel E\textsubscript{b}/N₀ is 12 dB.

\textbf{Find:} (a) The bit rate, (b) the BER for coherent DQPSK
detection (BER ≈ Q(√(2E\textsubscript{b}/N₀)) × 2), (c) the BER for
noncoherent DQPSK detection (BER ≈ e\textsuperscript{−Eb/N₀}), and (d)
the E\textsubscript{b}/N₀ penalty relative to coherent QPSK.

\textbf{Solution:}

\begin{enumerate}
\def\labelenumi{(\alph{enumi})}
\tightlist
\item
  Bit rate: R\textsubscript{b} = 2 × R\textsubscript{s} = 2 × 2 × 10⁶ =
  \textbf{4 Mbps}
\end{enumerate}

E\textsubscript{b}/N₀ = 12 dB = 15.85 (linear).

\begin{enumerate}
\def\labelenumi{(\alph{enumi})}
\setcounter{enumi}{1}
\item
  Coherent DQPSK BER: BER ≈ 2 × Q(√(2 × 15.85)) = 2 × Q(√31.70) = 2 ×
  Q(5.630) BER = 2 × 9.0 × 10⁻⁹ = \textbf{1.80 × 10⁻⁸}
\item
  Noncoherent DQPSK BER: BER ≈ e\textsuperscript{−Eb/N₀} = e⁻¹⁵·⁸⁵ =
  \textbf{1.32 × 10⁻⁷}
\item
  Coherent QPSK at the same E\textsubscript{b}/N₀: BER = Q(√(2 × 15.85))
  = Q(5.630) = 9.0 × 10⁻⁹
\end{enumerate}

Comparing at BER = 10⁻⁵: Coherent QPSK requires E\textsubscript{b}/N₀ =
9.6 dB. Coherent DQPSK requires approximately 9.6 + 0.5 = \textbf{10.1
dB} (about 0.5 dB penalty). Noncoherent DQPSK requires approximately 9.6
+ 1.8 = \textbf{11.4 dB} (about 1.8 dB penalty).

The penalty is the cost of not requiring a coherent carrier reference,
which simplifies the receiver significantly.

\begin{center}\rule{0.5\linewidth}{0.5pt}\end{center}

\section{Problem 2.3.9}\label{problem-2.3.9}

\textbf{Given:} A 1024-QAM system (used in Wi-Fi 6E) transmits at a
symbol rate of 234 data subcarriers per OFDM symbol in an 80 MHz
channel. The OFDM symbol duration is 12.8 μs (useful part) plus 0.8 μs
guard interval. The coding rate is 5/6.

\textbf{Find:} (a) The bits per symbol (per subcarrier), (b) the raw
data rate per spatial stream, (c) the net data rate after coding, (d)
the required SNR per subcarrier for BER = 10⁻⁵ with 1024-QAM, and (e)
the bandwidth efficiency.

\textbf{Solution:}

\begin{enumerate}
\def\labelenumi{(\alph{enumi})}
\item
  Bits per symbol: log₂(1024) = \textbf{10 bits/subcarrier}
\item
  Total OFDM symbol time = 12.8 + 0.8 = 13.6 μs Raw rate = 234 × 10 /
  13.6 × 10⁻⁶ = 2,340 / 13.6 × 10⁻⁶ = \textbf{172.06 Mbps}
\item
  Net data rate = 172.06 × 5/6 = \textbf{143.38 Mbps}
\end{enumerate}

The Wi-Fi 6 (802.11ax) standard specifies MCS 11 (1024-QAM, 5/6) at
approximately 143.4 Mbps per spatial stream for 80 MHz with 0.8 μs GI,
confirming this calculation.

\begin{enumerate}
\def\labelenumi{(\alph{enumi})}
\setcounter{enumi}{3}
\item
  For 1024-QAM at BER = 10⁻⁵: E\textsubscript{b}/N₀ ≈ 25.1 dB (from
  tables). SNR per subcarrier = E\textsubscript{b}/N₀ + 10
  log₁₀(bits/symbol) − 10 log₁₀(code rate) SNR = 25.1 + 10 log₁₀(10) −
  10 log₁₀(5/6) = 25.1 + 10.0 − (−0.79) = \textbf{35.9 dB}
\item
  Bandwidth efficiency = 143.38 / 80 = \textbf{1.79 bits/s/Hz} per
  stream With 2 spatial streams: 2 × 143.38 / 80 = 3.58 bits/s/Hz. This
  demonstrates why 1024-QAM requires extremely clean channels.
\end{enumerate}

\begin{center}\rule{0.5\linewidth}{0.5pt}\end{center}

\section{Problem 2.3.10}\label{problem-2.3.10}

\textbf{Given:} A 2×2 MIMO system uses spatial multiplexing with QPSK
modulation on each stream. The channel bandwidth is 20 MHz, the symbol
rate is 18 Msymbols/s per stream, and the coding rate is 3/4. The
channel matrix H has singular values σ₁ = 1.8 and σ₂ = 0.6.

\textbf{Find:} (a) The aggregate data rate, (b) the effective SNR on
each spatial stream if the total transmit SNR is 20 dB (equally split
between antennas), (c) the BER on each stream, and (d) the average BER
across both streams.

\textbf{Solution:}

\begin{enumerate}
\def\labelenumi{(\alph{enumi})}
\item
  QPSK: 2 bits/symbol per stream. Rate per stream = 2 × 18 × 10⁶ × 3/4 =
  27 Mbps Aggregate rate = 2 × 27 = \textbf{54 Mbps}
\item
  Total transmit SNR = 20 dB = 100 (linear). With equal power allocation
  (SNR = 50 per antenna): SNR on stream 1 = σ₁² × SNR\textsubscript{per
  antenna} = 1.8² × 50 = 3.24 × 50 = 162 = \textbf{22.1 dB} SNR on
  stream 2 = σ₂² × SNR\textsubscript{per antenna} = 0.6² × 50 = 0.36 ×
  50 = 18 = \textbf{12.6 dB}
\item
  For QPSK with coding rate 3/4, E\textsubscript{b}/N₀ = SNR / (bits per
  symbol × code rate) = SNR / 1.5: Stream 1: E\textsubscript{b}/N₀ =
  162/1.5 = 108 → BER = Q(√(2 × 108)) = Q(14.7) ≈ \textbf{\textless{}
  10⁻⁴⁰} (error-free) Stream 2: E\textsubscript{b}/N₀ = 18/1.5 = 12 →
  BER = Q(√(2 × 12)) = Q(4.899) = \textbf{4.8 × 10⁻⁷}
\item
  Average BER = (BER₁ + BER₂) / 2 ≈ (0 + 4.8 × 10⁻⁷) / 2 = \textbf{2.4 ×
  10⁻⁷}
\end{enumerate}

The weaker stream (σ₂ = 0.6) dominates the error performance.
Water-filling power allocation would shift more power to stream 1 and
less to stream 2, but at this SNR both streams are usable. If σ₂ were
too small (rank-deficient channel), the system would fall back to
single-stream transmission with transmit diversity.

\chapter{Chapter 2 --- Section 2.4: Channel Coding and Error
Correction}\label{chapter-2-section-2.4-channel-coding-and-error-correction}

Practice problems covering error detection (parity, CRC), forward error
correction (Hamming, Reed-Solomon, convolutional codes, LDPC),
interleaving, and burst error protection.

\begin{center}\rule{0.5\linewidth}{0.5pt}\end{center}

\section{Problem 2.4.1}\label{problem-2.4.1}

\textbf{Given:} A data link transmits 512-byte frames protected by a
CRC-16 checksum. The channel has a random bit error rate of 10⁻⁶. The
CRC-16 has a Hamming distance of 4, guaranteeing detection of all 1, 2,
and 3-bit errors, and its undetected error probability for random
patterns longer than 16 bits is approximately 2⁻¹⁶.

\textbf{Find:} (a) The total number of bits per frame, (b) the
probability that a frame contains at least one bit error, (c) the
probability of an undetected error per frame, and (d) the expected
number of frames transmitted between undetected errors at a data rate of
10 Mbps.

\textbf{Solution:}

\begin{enumerate}
\def\labelenumi{(\alph{enumi})}
\item
  Bits per frame: 512 × 8 = 4,096 data bits + 16 CRC bits =
  \textbf{4,112 bits}
\item
  P(one or more errors) = 1 − (1 − 10⁻⁶)⁴¹¹² ≈ 4,112 × 10⁻⁶ =
  \textbf{4.112 × 10⁻³} (about 1 in 243 frames)
\item
  The CRC-16 fails to detect with probability ≈ 2⁻¹⁶ = 1.526 × 10⁻⁵ for
  random error patterns of more than 16 bits. For single, double, and
  triple bit errors (which dominate at BER = 10⁻⁶), all are detected
  (Hamming distance 4).
\end{enumerate}

Probability of 4+ errors in a frame is very small: P(≥4 errors) ≈
C(4112, 4) × (10⁻⁶)⁴ ≈ 1.19 × 10¹³ × 10⁻²⁴ = 1.19 × 10⁻¹¹ P(undetected)
≈ P(≥4 errors) × 2⁻¹⁶ = 1.19 × 10⁻¹¹ × 1.526 × 10⁻⁵ = \textbf{1.82 ×
10⁻¹⁶}

\begin{enumerate}
\def\labelenumi{(\alph{enumi})}
\setcounter{enumi}{3}
\tightlist
\item
  Frames per second: 10 × 10⁶ / 4,112 = 2,432 frames/s Mean time between
  undetected errors: 1 / (2,432 × 1.82 × 10⁻¹⁶) = 2.26 × 10¹² s ≈
  \textbf{71,610 years}
\end{enumerate}

This extraordinary reliability is why CRC combined with retransmission
(ARQ) is used in Ethernet and most data links rather than FEC alone.

\begin{center}\rule{0.5\linewidth}{0.5pt}\end{center}

\section{Problem 2.4.2}\label{problem-2.4.2}

\textbf{Given:} A (7, 4) Hamming code is used to protect data on a
memory bus. The 4 data bits are d₃d₂d₁d₀ = 1011, and the 3 parity bits
are computed as: p₁ = d₃ ⊕ d₂ ⊕ d₀, p₂ = d₃ ⊕ d₁ ⊕ d₀, p₃ = d₂ ⊕ d₁ ⊕
d₀.

\textbf{Find:} (a) The encoded 7-bit codeword, (b) the syndrome if bit
position 5 (d₁) is flipped during transmission, (c) the corrected
codeword, and (d) the code rate and overhead percentage.

\textbf{Solution:}

\begin{enumerate}
\def\labelenumi{(\alph{enumi})}
\tightlist
\item
  Computing parity bits for d₃d₂d₁d₀ = 1011: p₁ = 1 ⊕ 0 ⊕ 1 = 0 p₂ = 1 ⊕
  1 ⊕ 1 = 1 p₃ = 0 ⊕ 1 ⊕ 1 = 0
\end{enumerate}

Codeword (positions 1-7): p₁p₂d₃p₃d₂d₁d₀ = \textbf{0 1 1 0 0 1 1}

\begin{enumerate}
\def\labelenumi{(\alph{enumi})}
\setcounter{enumi}{1}
\item
  If bit 5 (d₁ at position 6 in standard Hamming) is flipped: received =
  0 1 1 0 0 0 1 Syndrome bits: s₁ = p₁ ⊕ d₃ ⊕ d₂ ⊕ d₀ = 0 ⊕ 1 ⊕ 0 ⊕ 1 =
  0 s₂ = p₂ ⊕ d₃ ⊕ d₁ ⊕ d₀ = 1 ⊕ 1 ⊕ 0 ⊕ 1 = 1 s₃ = p₃ ⊕ d₂ ⊕ d₁ ⊕ d₀ =
  0 ⊕ 0 ⊕ 0 ⊕ 1 = 1 Syndrome = s₃s₂s₁ = 110₂ = \textbf{6} (decimal),
  indicating error at position 6
\item
  Flip bit 6: corrected = 0 1 1 0 0 1 1 → data = \textbf{1011} (matches
  original)
\item
  Code rate: R\textsubscript{c} = k/n = 4/7 = \textbf{0.571} Overhead:
  (n − k)/k × 100 = 3/4 × 100 = \textbf{75\%}
\end{enumerate}

\begin{center}\rule{0.5\linewidth}{0.5pt}\end{center}

\section{Problem 2.4.3}\label{problem-2.4.3}

\textbf{Given:} A Reed-Solomon RS(255, 239) code is used in a fiber
optic communication system. The code operates over GF(2⁸) with 8-bit
symbols. The pre-FEC bit error rate is 2 × 10⁻⁴.

\textbf{Find:} (a) The number of parity symbols, (b) the maximum number
of correctable symbol errors, (c) the code rate, (d) the probability of
a symbol error at the input, and (e) the post-FEC BER (using the
approximation for RS code output BER).

\textbf{Solution:}

\begin{enumerate}
\def\labelenumi{(\alph{enumi})}
\item
  Parity symbols: n − k = 255 − 239 = \textbf{16 symbols} (128 bits of
  overhead per block)
\item
  Correctable symbol errors: t = (n − k)/2 = 16/2 = \textbf{8 symbol
  errors per block}
\item
  Code rate: R\textsubscript{c} = 239/255 = \textbf{0.937} (6.3\%
  overhead)
\item
  Symbol error probability (each symbol is 8 bits): P\textsubscript{s} =
  1 − (1 − BER)⁸ = 1 − (1 − 2 × 10⁻⁴)⁸ ≈ 8 × 2 × 10⁻⁴ = \textbf{1.6 ×
  10⁻³}
\item
  Expected symbol errors per block: μ = 255 × 1.6 × 10⁻³ = 0.408 Since μ
  = 0.408 is far less than t = 8, the probability of decoder failure
  (more than 8 errors) is extremely small.
\end{enumerate}

Using the binomial approximation: P(block failure) = Σ(j=9 to 255)
C(255,j) × P\textsubscript{s}ʲ × (1−P\textsubscript{s})²⁵⁵⁻ʲ ≈ C(255,9)
× (1.6 × 10⁻³)⁹

The post-FEC BER is approximately \textbf{\textless{} 10⁻¹⁵}, well below
the 10⁻¹² target for optical transport systems.

\begin{center}\rule{0.5\linewidth}{0.5pt}\end{center}

\section{Problem 2.4.4}\label{problem-2.4.4}

\textbf{Given:} A rate-1/2 convolutional code with constraint length K =
3 has generator polynomials g₁ = {[}1, 1, 1{]} and g₂ = {[}1, 0, 1{]}.
The input data sequence is 1 0 1 1.

\textbf{Find:} (a) The output coded sequence (encoding each bit through
both generators), (b) the total number of output bits, (c) the code
rate, and (d) the free distance d\textsubscript{free} of this code
(which is 5 for K=3, rate-1/2).

\textbf{Solution:}

\begin{enumerate}
\def\labelenumi{(\alph{enumi})}
\tightlist
\item
  The encoder uses a 3-stage shift register (initialized to 000).
  Process each input bit:
\end{enumerate}

Input 1: register = 1 0 0 Output₁ = 1⊕0⊕0 = 1, Output₂ = 1⊕0 = 1 →
\textbf{11}

Input 0: register = 0 1 0 Output₁ = 0⊕1⊕0 = 1, Output₂ = 0⊕0 = 0 →
\textbf{10}

Input 1: register = 1 0 1 Output₁ = 1⊕0⊕1 = 0, Output₂ = 1⊕1 = 0 →
\textbf{00}

Input 1: register = 1 1 0 Output₁ = 1⊕1⊕0 = 0, Output₂ = 1⊕0 = 1 →
\textbf{01}

Flush (input 0): register = 0 1 1 Output₁ = 0⊕1⊕1 = 0, Output₂ = 0⊕1 = 1
→ \textbf{01}

Flush (input 0): register = 0 0 1 Output₁ = 0⊕0⊕1 = 1, Output₂ = 0⊕1 = 1
→ \textbf{11}

Complete output sequence: \textbf{11 10 00 01 01 11}

\begin{enumerate}
\def\labelenumi{(\alph{enumi})}
\setcounter{enumi}{1}
\item
  Total output bits: 4 data bits + 2 flush bits = 6 input bits × 2 =
  \textbf{12 output bits}
\item
  Code rate = 4 data bits / 12 output bits = 1/3 for finite block. For
  continuous operation: R\textsubscript{c} = \textbf{1/2} (each input
  bit produces 2 output bits)
\item
  The free distance d\textsubscript{free} = \textbf{5} for this code.
  This means the Viterbi decoder can correct up to ⌊(5−1)/2⌋ = 2 errors
  in any window of decoded bits, and can detect up to 4 errors.
\end{enumerate}

\begin{center}\rule{0.5\linewidth}{0.5pt}\end{center}

\section{Problem 2.4.5}\label{problem-2.4.5}

\textbf{Given:} A block interleaver is used with an RS(15, 11) code that
corrects t = 2 symbol errors per codeword. The channel produces burst
errors up to 8 symbols long.

\textbf{Find:} (a) The required interleaving depth, (b) the interleaver
matrix dimensions, (c) the total interleaver memory in symbols, (d) the
latency introduced by the interleaver/de-interleaver pair, and (e) the
effective burst error correction capability after de-interleaving.

\textbf{Solution:}

\begin{enumerate}
\def\labelenumi{(\alph{enumi})}
\tightlist
\item
  Required depth: D = ⌈L/t⌉ = ⌈8/2⌉ = \textbf{4 rows}
\end{enumerate}

With D = 4, a burst of 8 consecutive symbols distributes as 2 per
codeword across 4 codewords --- exactly at the correction limit. For
safety margin, use D = \textbf{5 rows} (handles bursts up to 10
symbols).

\begin{enumerate}
\def\labelenumi{(\alph{enumi})}
\setcounter{enumi}{1}
\item
  Matrix dimensions: D × n = 5 × 15 = \textbf{5 rows × 15 columns}
\item
  Memory: 5 × 15 = \textbf{75 symbols} (at the interleaver; another 75
  at the de-interleaver)
\item
  Latency: The interleaver must fill the complete matrix before reading.
  With D = 5: One-way latency = D × n = 75 symbols Total round-trip
  latency = 2 × 75 = \textbf{150 symbols}
\end{enumerate}

If each symbol is 4 bits (GF(2⁴) for RS(15,11)) and the symbol rate is
10,000 symbols/s: Latency = 150 / 10,000 = \textbf{15 ms}

\begin{enumerate}
\def\labelenumi{(\alph{enumi})}
\setcounter{enumi}{4}
\tightlist
\item
  Without interleaving: corrects bursts up to 2 symbols. With D = 5
  interleaving: corrects bursts up to D × t = 5 × 2 = \textbf{10
  symbols} --- a 5× improvement.
\end{enumerate}

\begin{center}\rule{0.5\linewidth}{0.5pt}\end{center}

\section{Problem 2.4.6}\label{problem-2.4.6}

\textbf{Given:} A turbo code consists of two rate-1/2 recursive
systematic convolutional (RSC) encoders connected in parallel through a
pseudo-random interleaver of length 1,024 bits. The overall code rate
after puncturing is 1/3 (no puncturing) or 1/2 (with puncturing). The
code operates at E\textsubscript{b}/N₀ = 1.0 dB with 8 iterations of
turbo decoding.

\textbf{Find:} (a) The number of parity bits produced by each encoder
(for rate 1/3), (b) the overhead percentage for rate 1/2, (c) the
Shannon limit for a rate-1/2 code on an AWGN channel, (d) the gap to the
Shannon limit at E\textsubscript{b}/N₀ = 1.0 dB, and (e) the decoding
complexity in terms of trellis states visited.

\textbf{Solution:}

\begin{enumerate}
\def\labelenumi{(\alph{enumi})}
\item
  For rate 1/3: each RSC encoder produces 1,024 parity bits from 1,024
  data bits. Total output = 1,024 (systematic) + 1,024 (parity 1) +
  1,024 (parity 2) = 3,072 bits. Each encoder produces \textbf{1,024
  parity bits}.
\item
  For rate 1/2 (puncturing every other parity bit from each encoder):
  Total output = 2,048 bits for 1,024 data bits. Overhead = (2,048 −
  1,024)/1,024 × 100 = \textbf{100\%} (doubles the transmitted bits)
\item
  Shannon limit for rate-1/2 AWGN: C = R\textsubscript{c} × log₂(1 +
  SNR), where SNR = 2 × R\textsubscript{c} × E\textsubscript{b}/N₀ At
  capacity: 1/2 = 1/2 × log₂(1 + E\textsubscript{b}/N₀) log₂(1 +
  E\textsubscript{b}/N₀) = 1 → E\textsubscript{b}/N₀ = 1 = \textbf{0 dB}
\item
  Gap to Shannon limit = 1.0 − 0 = \textbf{1.0 dB} At this operating
  point, turbo codes achieve BER ≈ 10⁻⁵, operating just 1 dB from the
  theoretical limit.
\item
  Each RSC encoder has K = 4 (constraint length), giving
  2\textsuperscript{K−1} = 8 trellis states. Per iteration: 2 decoders ×
  1,024 trellis stages × 8 states = 16,384 state computations. For 8
  iterations: 8 × 16,384 = \textbf{131,072 state computations} per block
  of 1,024 bits.
\end{enumerate}

\begin{center}\rule{0.5\linewidth}{0.5pt}\end{center}

\section{Problem 2.4.7}\label{problem-2.4.7}

\textbf{Given:} An LDPC code used in DVB-S2 satellite broadcasting has a
block length of n = 64,800 bits and a code rate of 3/4 (k = 48,600 data
bits). The LDPC code achieves BER = 10⁻⁷ at E\textsubscript{b}/N₀ = 3.10
dB on an AWGN channel.

\textbf{Find:} (a) The number of parity bits, (b) the Shannon limit for
rate 3/4, (c) the gap to the Shannon limit, (d) the coding gain compared
to uncoded QPSK, and (e) the net data rate for a 36 MHz transponder
using QPSK at 27.5 Msymbols/s.

\textbf{Solution:}

\begin{enumerate}
\def\labelenumi{(\alph{enumi})}
\item
  Parity bits: n − k = 64,800 − 48,600 = \textbf{16,200 parity bits} per
  block
\item
  Shannon limit for rate 3/4: 3/4 = 3/4 × log₂(1 + SNR) → log₂(1 + SNR)
  = 1 → SNR = 1 E\textsubscript{b}/N₀ = SNR / (2R\textsubscript{c}) = 1
  / (2 × 3/4) = 2/3 But more precisely: C/B = log₂(1 +
  2R\textsubscript{c} × E\textsubscript{b}/N₀) = R\textsubscript{c}
  log₂(1 + 1.5 × E\textsubscript{b}/N₀) = 0.75 1.5 ×
  E\textsubscript{b}/N₀ = 2⁰·⁷⁵ − 1 = 1.6818 − 1 = 0.6818
  E\textsubscript{b}/N₀ = 0.4545 = −3.42 dB\ldots{}
\end{enumerate}

Correction: E\textsubscript{b}/N₀\textsubscript{min} =
(2\textsuperscript{R} − 1) / R = (2⁰·⁷⁵ − 1) / 0.75 = 0.6818 / 0.75 =
0.909 = \textbf{−0.41 dB}

\begin{enumerate}
\def\labelenumi{(\alph{enumi})}
\setcounter{enumi}{2}
\tightlist
\item
  Gap to Shannon limit = 3.10 − (−0.41) = \textbf{3.51 dB}
\end{enumerate}

More commonly referenced: for rate 3/4 on AWGN, the Shannon limit for
coded BER = 10⁻⁷ is approximately \textbf{1.89 dB}, giving a gap of 3.10
− 1.89 = \textbf{1.21 dB} --- remarkably close to the theoretical limit.

\begin{enumerate}
\def\labelenumi{(\alph{enumi})}
\setcounter{enumi}{3}
\item
  Uncoded QPSK requires E\textsubscript{b}/N₀ ≈ 11.3 dB for BER = 10⁻⁷.
  Coding gain = 11.3 − 3.10 = \textbf{8.2 dB}
\item
  QPSK: 2 bits/symbol. Raw rate = 2 × 27.5 = 55 Mbps. Net rate = 55 ×
  3/4 = \textbf{41.25 Mbps}
\end{enumerate}

\begin{center}\rule{0.5\linewidth}{0.5pt}\end{center}

\section{Problem 2.4.8}\label{problem-2.4.8}

\textbf{Given:} A wireless system uses a convolutional interleaver with
parameters (I, M) = (12, 17), where I is the number of delay branches
and M is the delay increment per branch (in symbols). The FEC code is a
(204, 188) Reed-Solomon code over GF(2⁸).

\textbf{Find:} (a) The maximum delay in the interleaver in symbols, (b)
the total interleaver memory, (c) the maximum burst length (in symbols)
that can be corrected after de-interleaving, (d) the total system
latency, and (e) the effective burst correction capability in bytes.

\textbf{Solution:}

\begin{enumerate}
\def\labelenumi{(\alph{enumi})}
\item
  Maximum delay: The deepest branch has delay = (I − 1) × M = 11 × 17 =
  \textbf{187 symbols}
\item
  Total memory = I × (I − 1) × M / 2 = 12 × 11 × 17 / 2 = \textbf{1,122
  symbols}
\item
  The RS(204, 188) code corrects t = (204 − 188)/2 = 8 symbol errors per
  codeword. After de-interleaving, a burst of B consecutive symbols maps
  to at most ⌈B/I⌉ errors per codeword. Maximum correctable burst:
  B\textsubscript{max} = I × t = 12 × 8 = \textbf{96 symbols}
\item
  The convolutional interleaver introduces a one-way delay equal to the
  maximum branch delay: Interleaver delay = 187 symbols; De-interleaver
  delay = 187 symbols. Total latency = 2 × 187 = \textbf{374 symbols}
\end{enumerate}

At a symbol rate of 6.75 Msymbols/s (DVB-T standard), this is 374 / 6.75
× 10⁶ = 55.4 μs.

\begin{enumerate}
\def\labelenumi{(\alph{enumi})}
\setcounter{enumi}{4}
\tightlist
\item
  Since each RS symbol is 8 bits = 1 byte: Burst correction = 96 × 1 =
  \textbf{96 bytes} (768 bits)
\end{enumerate}

This is the burst correction capability used in DVB-T digital
television, protecting against impulse noise and multipath fading in the
terrestrial channel.

\begin{center}\rule{0.5\linewidth}{0.5pt}\end{center}

\section{Problem 2.4.9}\label{problem-2.4.9}

\textbf{Given:} A deep-space probe communicates with Earth using a
rate-1/6 convolutional code with K = 15 and Viterbi decoding, achieving
a coding gain of 9.5 dB at BER = 10⁻⁵. The raw channel capacity is C =
500 bps due to the extreme distance (2 AU). The transmitter power is 20
W, antenna gain is 40 dBi, and the link operates at 8.4 GHz (X-band).

\textbf{Find:} (a) The effective data rate after coding, (b) the uncoded
E\textsubscript{b}/N₀ required for BER = 10⁻⁵ with BPSK, (c) the coded
E\textsubscript{b}/N₀ required, (d) the margin gained by using the code,
and (e) the minimum link SNR.

\textbf{Solution:}

\begin{enumerate}
\def\labelenumi{(\alph{enumi})}
\tightlist
\item
  Effective data rate: R\textsubscript{data} = R\textsubscript{channel}
  × R\textsubscript{c} = 500 × 1/6 = \textbf{83.3 bps}
\end{enumerate}

Wait --- the code rate reduces the data rate. The 500 bps is the coded
symbol rate on the channel. Actual data rate = 500 × (1/6) = 83.3 bps.
However, if 500 bps is the channel's information capacity:
R\textsubscript{data} = 500 / 6 × 6 = 500 bps at the channel rate →
\textbf{83.3 bps} of actual user data.

\begin{enumerate}
\def\labelenumi{(\alph{enumi})}
\setcounter{enumi}{1}
\item
  Uncoded BPSK at BER = 10⁻⁵: E\textsubscript{b}/N₀ = \textbf{9.6 dB}
\item
  Coded E\textsubscript{b}/N₀ = 9.6 − 9.5 = \textbf{0.1 dB} (just barely
  above 0 dB)
\item
  The code allows the system to operate at E\textsubscript{b}/N₀ = 0.1
  dB instead of 9.6 dB, reducing the required transmitter power by a
  factor of 10\textsuperscript{9.5/10} = 8.91, or equivalently: Power
  savings = 9.5 dB → the 20 W transmitter with coding is equivalent to
  \textbf{178 W without coding}.
\item
  Minimum link SNR for the coded system: SNR = E\textsubscript{b}/N₀ +
  10 log₁₀(R\textsubscript{data}/B) With B ≈ 500 Hz (matched to channel
  rate): SNR = 0.1 + 10 log₁₀(83.3/500) = 0.1 + (−7.78) = \textbf{−7.68
  dB}
\end{enumerate}

The system operates at a negative SNR --- the signal is buried below the
noise floor, but the powerful code extracts it reliably. This is the
hallmark of deep-space communication.

\begin{center}\rule{0.5\linewidth}{0.5pt}\end{center}

\section{Problem 2.4.10}\label{problem-2.4.10}

\textbf{Given:} A 5G NR system uses an LDPC code for the data channel
(PDSCH) and a Polar code for the control channel (PDCCH). The data
channel has a transport block size of 8,424 bits encoded to 16,848 bits
(rate 1/2). The control channel carries 48 bits of DCI encoded to 432
bits. Both codes use successive cancellation list (SCL) decoding with
list size L = 8 for Polar and min-sum decoding with 10 iterations for
LDPC.

\textbf{Find:} (a) The LDPC code rate, (b) the Polar code rate, (c) the
overhead percentage for each, (d) the approximate E\textsubscript{b}/N₀
performance advantage of the Polar code for short blocks compared to
LDPC, and (e) why different codes are used for data versus control.

\textbf{Solution:}

\begin{enumerate}
\def\labelenumi{(\alph{enumi})}
\item
  LDPC code rate: R\textsubscript{LDPC} = 8,424 / 16,848 = \textbf{1/2 =
  0.500}
\item
  Polar code rate: R\textsubscript{Polar} = 48 / 432 = \textbf{1/9 =
  0.111}
\item
  LDPC overhead: (16,848 − 8,424) / 8,424 × 100 = \textbf{100\%} Polar
  overhead: (432 − 48) / 48 × 100 = \textbf{800\%}
\item
  For short block lengths (\textless{} 200 bits), Polar codes with SCL
  decoding outperform LDPC by approximately \textbf{0.5--1.5 dB} at BER
  = 10⁻³. At long block lengths (\textgreater{} 1,000 bits), LDPC codes
  perform as well or better. The crossover point is typically around
  200--400 bits.
\item
  \textbf{Different codes are optimal for different block sizes:}
\end{enumerate}

\begin{itemize}
\tightlist
\item
  The control channel carries small payloads (20--140 bits) where Polar
  codes excel due to their provable capacity-achieving property for
  binary-input memoryless channels and superior short-block performance.
\item
  The data channel carries large payloads (hundreds to thousands of
  bits) where LDPC codes achieve near-Shannon-limit performance with
  efficient parallelizable hardware decoders.
\item
  LDPC decoding is highly parallelizable (message-passing on a sparse
  graph), enabling high-throughput implementation for multi-Gbps data
  rates.
\item
  Polar decoding (SCL) is inherently sequential but excellent for the
  low-rate, latency-sensitive control channel.
\end{itemize}

\chapter{Chapter 2 --- Section 2.5:
Multiplexing}\label{chapter-2-section-2.5-multiplexing}

Practice problems covering frequency division multiplexing (FDM), OFDM,
time division multiplexing (TDM), SONET/SDH, code division multiple
access (CDMA), OFDMA, and SC-FDMA.

\begin{center}\rule{0.5\linewidth}{0.5pt}\end{center}

\section{Problem 2.5.1}\label{problem-2.5.1}

\textbf{Given:} A traditional analog FDM telephone system combines 12
voice channels into a Group. Each voice channel has a 4 kHz bandwidth
(300 Hz to 3,400 Hz of voice plus guard bands), and channels are stacked
starting at 60 kHz using SSB modulation with carrier suppression.

\textbf{Find:} (a) The total bandwidth of one Group, (b) the frequency
range occupied, (c) the number of Groups in a Supergroup (5 Groups), (d)
the Supergroup bandwidth, and (e) the total voice channels in a
Mastergroup (10 Supergroups).

\textbf{Solution:}

\begin{enumerate}
\def\labelenumi{(\alph{enumi})}
\item
  Group bandwidth: 12 channels × 4 kHz = \textbf{48 kHz}
\item
  Frequency range: The first channel occupies 60--64 kHz, the last
  channel occupies 104--108 kHz. Range = \textbf{60 kHz to 108 kHz}
\item
  Groups per Supergroup: \textbf{5 Groups} = 5 × 12 = 60 voice channels
\item
  Supergroup bandwidth: 5 × 48 kHz = 240 kHz, plus guard bands between
  groups. Standard Supergroup occupies 312--552 kHz = \textbf{240 kHz}
  (60 channels)
\item
  Mastergroup = 10 Supergroups: Total channels = 10 × 60 = \textbf{600
  voice channels} Bandwidth ≈ 10 × 240 = 2,400 kHz = \textbf{2.4 MHz}
\end{enumerate}

This hierarchical FDM structure was the backbone of the analog telephone
network for decades, carrying thousands of simultaneous calls over
coaxial cables and microwave radio links.

\begin{center}\rule{0.5\linewidth}{0.5pt}\end{center}

\section{Problem 2.5.2}\label{problem-2.5.2}

\textbf{Given:} An OFDM system for a broadband wireless access network
uses 2,048 subcarriers (2K FFT) in an 8 MHz channel. Of the 2,048
subcarriers, 1,705 carry data, 193 are pilot/reference subcarriers, and
150 are null guard subcarriers. The guard interval (cyclic prefix) is
1/4 of the useful symbol duration. Each data subcarrier uses 16-QAM with
a 2/3 FEC code rate.

\textbf{Find:} (a) The subcarrier spacing, (b) the useful symbol
duration, (c) the total symbol duration (including guard interval), (d)
the net data rate, and (e) the spectral efficiency.

\textbf{Solution:}

\begin{enumerate}
\def\labelenumi{(\alph{enumi})}
\tightlist
\item
  Subcarrier spacing: Δf = B / N = 8 × 10⁶ / 2,048 = \textbf{3,906.25 Hz
  ≈ 3.906 kHz}
\end{enumerate}

Actually, for DVB-T style systems, Δf = 8 MHz × (7/8) / 2,048 = 7 × 10⁶
/ 2,048 = \textbf{3,418 Hz} (using 7 MHz usable bandwidth in 8 MHz
channel). Let us use the simpler relationship: Δf = 1/T\textsubscript{u}
where T\textsubscript{u} is the useful symbol duration.

\begin{enumerate}
\def\labelenumi{(\alph{enumi})}
\setcounter{enumi}{1}
\tightlist
\item
  Useful symbol duration: T\textsubscript{u} = N /
  B\textsubscript{usable} For 2K mode in 8 MHz: T\textsubscript{u} =
  2,048 / (8 × 10⁶ × 7/8) = 2,048 / 7,000,000 = \textbf{224 μs} (This
  gives Δf = 1/224 μs = 4,464 Hz)
\end{enumerate}

More standard: T\textsubscript{u} = 1/Δf. With Δf = 8 × 10⁶ / 2,048:
T\textsubscript{u} = 2,048 / (8 × 10⁶) = \textbf{256 μs}

\begin{enumerate}
\def\labelenumi{(\alph{enumi})}
\setcounter{enumi}{2}
\item
  Total symbol time with 1/4 guard interval: T\textsubscript{total} =
  T\textsubscript{u} + T\textsubscript{u}/4 = 256 × (1 + 1/4) = 256 ×
  1.25 = \textbf{320 μs}
\item
  Net data rate: Bits per OFDM symbol = 1,705 data subcarriers × 4 bits
  (16-QAM) × 2/3 (FEC) = 1,705 × 2.667 = 4,547 bits R\textsubscript{net}
  = 4,547 / 320 × 10⁻⁶ = \textbf{14.21 Mbps}
\item
  Spectral efficiency: η = 14.21 / 8 = \textbf{1.78 bits/s/Hz}
\end{enumerate}

\begin{center}\rule{0.5\linewidth}{0.5pt}\end{center}

\section{Problem 2.5.3}\label{problem-2.5.3}

\textbf{Given:} A DS-1 (T1) frame consists of 24 DS-0 channels
multiplexed using TDM. Each DS-0 carries one 8-bit voice sample per
frame. A framing bit is added to each frame. The frame rate is 8,000
frames per second. Four T1 circuits are multiplexed into a DS-2.

\textbf{Find:} (a) The number of bits per T1 frame, (b) the T1 line
rate, (c) the payload capacity (voice data only), (d) the DS-2 line rate
(4 × T1 plus overhead), and (e) the number of T1s in a DS-3 (28 T1s) and
the DS-3 line rate.

\textbf{Solution:}

\begin{enumerate}
\def\labelenumi{(\alph{enumi})}
\item
  Bits per T1 frame: 24 channels × 8 bits + 1 framing bit = 192 + 1 =
  \textbf{193 bits}
\item
  T1 line rate: 193 × 8,000 = \textbf{1,544,000 bps = 1.544 Mbps}
\item
  Payload capacity: 24 × 8 × 8,000 = 1,536,000 bps = \textbf{1.536 Mbps}
  Overhead = 1,544 − 1,536 = 8 kbps (the framing channel)
\item
  DS-2 rate: 4 × 1.544 Mbps = 6.176 Mbps (payload) + stuffing and
  overhead bits. Actual DS-2 line rate = \textbf{6.312 Mbps} Overhead =
  6.312 − 6.176 = 0.136 Mbps (bit stuffing for clock synchronization)
\item
  DS-3: 28 × T1 = 28 × 1.544 = 43.232 Mbps (payload) + overhead. DS-3
  line rate = \textbf{44.736 Mbps} This carries 28 × 24 = \textbf{672
  voice channels}
\end{enumerate}

\begin{center}\rule{0.5\linewidth}{0.5pt}\end{center}

\section{Problem 2.5.4}\label{problem-2.5.4}

\textbf{Given:} An OC-192 SONET link operates at 9.953 Gbps. The link
carries concatenated STS-192c payload for a high-bandwidth Ethernet
connection. The SONET overhead is approximately 3.3\% (27 columns out of
810 per STS-1 row, but for concatenated payloads, path overhead is
shared).

\textbf{Find:} (a) The number of STS-1 signals multiplexed, (b) the
STS-1 line rate (for verification), (c) the approximate payload
bandwidth, (d) the maximum Ethernet payload that can be carried (after
mapping overhead), and (e) the number of DS-3 signals that could
alternatively be carried.

\textbf{Solution:}

\begin{enumerate}
\def\labelenumi{(\alph{enumi})}
\item
  STS-1 count: OC-192 = \textbf{192 STS-1 signals}
\item
  STS-1 line rate: 9,953.28 / 192 = \textbf{51.84 Mbps} (confirmed)
\item
  Approximate payload: 9,953.28 × (1 − 0.033) = 9,953.28 × 0.967 =
  \textbf{9,624.8 Mbps}
\end{enumerate}

For STS-192c: payload = 192 × 50.112 Mbps (SPE per STS-1) = 9,621.5 Mbps
Actual usable payload ≈ \textbf{9.58 Gbps} (after path overhead and
pointer adjustments)

\begin{enumerate}
\def\labelenumi{(\alph{enumi})}
\setcounter{enumi}{3}
\item
  An OC-192 carries one 10 Gigabit Ethernet (10GbE) WAN PHY signal. The
  10GbE WAN PHY rate is 9.953 Gbps (matching OC-192 exactly). Maximum
  Ethernet payload = 9,953 × (1 − overhead) ≈ \textbf{9.29 Gbps} after
  64B/66B encoding and SONET mapping overhead.
\item
  DS-3 capacity: Each STS-1 carries 1 DS-3 (or 28 DS-1s). OC-192
  carries: 192 × 1 = \textbf{192 DS-3 signals} (or 192 × 28 = 5,376 DS-1
  signals = 129,024 voice channels)
\end{enumerate}

\begin{center}\rule{0.5\linewidth}{0.5pt}\end{center}

\section{Problem 2.5.5}\label{problem-2.5.5}

\textbf{Given:} A CDMA system (IS-95) operates in a 1.25 MHz band. Each
user transmits at 9.6 kbps, and the Walsh code length is 64 chips per
bit (processing gain = 64). The system uses a 3-sector cell with voice
activity factor v = 0.4 and frequency reuse factor f = 0.6 (indicating
60\% of interference comes from other cells). The required
E\textsubscript{b}/N₀ is 6 dB.

\textbf{Find:} (a) The chip rate, (b) the processing gain in dB, (c) the
number of users per sector, (d) the total users per cell, and (e) the
sector capacity improvement if soft handoff and power control reduce the
required E\textsubscript{b}/N₀ by 2 dB.

\textbf{Solution:}

\begin{enumerate}
\def\labelenumi{(\alph{enumi})}
\item
  Chip rate: R\textsubscript{c} = processing gain × R\textsubscript{b} =
  64 × 9,600 = 614,400 chips/s However, IS-95 uses R\textsubscript{c} =
  \textbf{1.2288 Mchips/s} (128 chips/bit for rate 9.6 kbps, or the
  actual spread bandwidth is 1.25 MHz) Processing gain:
  G\textsubscript{p} = W/R = 1,228,800 / 9,600 = \textbf{128}
\item
  Processing gain in dB: 10 log₁₀(128) = \textbf{21.07 dB}
\item
  Users per sector: N = G\textsubscript{p} / ((E\textsubscript{b}/N₀) ×
  (1 + f) × v) + 1 E\textsubscript{b}/N₀ = 10\textsuperscript{6/10} =
  3.981 N = 128 / (3.981 × (1 + 0.6) × 0.4) + 1 = 128 / (3.981 × 1.6 ×
  0.4) + 1 N = 128 / 2.548 + 1 = 50.2 + 1 ≈ \textbf{51 users per sector}
\end{enumerate}

Wait, the formula accounts for other-cell interference: N =
G\textsubscript{p} / ((E\textsubscript{b}/N₀) × v × (1 + f)) N = 128 /
(3.981 × 0.4 × 1.6) = 128 / 2.548 = \textbf{50 users per sector}

\begin{enumerate}
\def\labelenumi{(\alph{enumi})}
\setcounter{enumi}{3}
\item
  Total per cell (3 sectors): 3 × 50 = \textbf{150 users per cell}
\item
  With 2 dB reduction: E\textsubscript{b}/N₀ = 4 dB = 2.512
  N\textsubscript{new} = 128 / (2.512 × 0.4 × 1.6) = 128 / 1.608 =
  \textbf{80 users per sector} Improvement: 80/50 = \textbf{1.6× (60\%
  more users)}
\end{enumerate}

\begin{center}\rule{0.5\linewidth}{0.5pt}\end{center}

\section{Problem 2.5.6}\label{problem-2.5.6}

\textbf{Given:} A 5G NR OFDMA system uses a 100 MHz channel at 3.5 GHz
(sub-6 GHz band) with 30 kHz subcarrier spacing. The channel has 273
resource blocks (RBs) of 12 subcarriers each. A slot consists of 14 OFDM
symbols at duration 0.5 ms. Two users are allocated: User A gets 200 RBs
with 256-QAM (8 bits) and 0.93 code rate; User B gets 73 RBs with QPSK
(2 bits) and 0.5 code rate.

\textbf{Find:} (a) The total number of data subcarriers, (b) the OFDM
symbol duration (useful + CP), (c) the data rate for each user, and (d)
the total cell throughput per slot.

\textbf{Solution:}

\begin{enumerate}
\def\labelenumi{(\alph{enumi})}
\item
  Total subcarriers: 273 × 12 = \textbf{3,276 subcarriers} User A: 200 ×
  12 = 2,400 subcarriers User B: 73 × 12 = 876 subcarriers
\item
  Slot = 14 symbols in 0.5 ms: Symbol duration = 500 μs / 14 =
  \textbf{35.71 μs} With 30 kHz spacing: T\textsubscript{u} = 1/30,000 =
  33.33 μs useful + 2.38 μs CP = 35.71 μs ✓
\item
  Data rates (assuming 12 data symbols per slot, 2 for
  control/reference): User A: 2,400 × 8 × 0.93 × 12 / 0.5 × 10⁻³ = 2,400
  × 7.44 × 12 / 5 × 10⁻⁴ = 214,272 / 5 × 10⁻⁴ = \textbf{428.5 Mbps}
\end{enumerate}

User B: 876 × 2 × 0.5 × 12 / 0.5 × 10⁻³ = 876 × 1.0 × 12 / 5 × 10⁻⁴ =
10,512 / 5 × 10⁻⁴ = \textbf{21.0 Mbps}

\begin{enumerate}
\def\labelenumi{(\alph{enumi})}
\setcounter{enumi}{3}
\tightlist
\item
  Total throughput: 428.5 + 21.0 = \textbf{449.5 Mbps}
\end{enumerate}

This represents a single-layer (single antenna) throughput. With 4×4
MIMO, the peak would approach 4 × 449.5 ≈ 1.8 Gbps, which aligns with
typical 5G sub-6 GHz peak rates.

\begin{center}\rule{0.5\linewidth}{0.5pt}\end{center}

\section{Problem 2.5.7}\label{problem-2.5.7}

\textbf{Given:} An LTE uplink uses SC-FDMA with 15 kHz subcarrier
spacing. A user is allocated 25 resource blocks (300 subcarriers) for a
5 MHz channel allocation. The modulation is 16-QAM with 1/2 coding rate.
The subframe duration is 1 ms with 14 OFDM symbols (2 used for reference
signals, 12 for data).

\textbf{Find:} (a) The occupied bandwidth, (b) the DFT size used for
SC-FDMA precoding, (c) the data rate, (d) the PAPR advantage compared to
OFDMA, and (e) the power amplifier efficiency improvement.

\textbf{Solution:}

\begin{enumerate}
\def\labelenumi{(\alph{enumi})}
\item
  Occupied bandwidth: 25 RBs × 12 subcarriers × 15 kHz = 300 × 15,000 =
  \textbf{4.5 MHz} (within the 5 MHz allocation)
\item
  The DFT precoding block size equals the number of allocated
  subcarriers: DFT size = \textbf{300 points} (one DFT per OFDM symbol)
\end{enumerate}

The 300-point DFT converts the time-domain signal to frequency-domain,
which is then mapped to the allocated subcarriers before the standard
IFFT. This converts the multicarrier OFDM signal into a
single-carrier-like waveform.

\begin{enumerate}
\def\labelenumi{(\alph{enumi})}
\setcounter{enumi}{2}
\item
  Data rate: Bits per subframe = 300 subcarriers × 4 bits (16-QAM) × 1/2
  (code rate) × 12 data symbols = 300 × 2 × 12 = 7,200 bits R = 7,200 /
  1 × 10⁻³ = \textbf{7.2 Mbps}
\item
  OFDMA PAPR for 300 subcarriers: approximately \textbf{10--12 dB} (for
  99.9\% CCDF) SC-FDMA PAPR: approximately \textbf{6--8 dB} (for 99.9\%
  CCDF) PAPR reduction: \textbf{3--4 dB}
\item
  A 3 dB PAPR reduction allows the power amplifier to operate 3 dB
  closer to saturation: PA efficiency at 6 dB backoff ≈ 15--20\% PA
  efficiency at 3 dB backoff ≈ 30--35\% Improvement: approximately
  \textbf{2× better PA efficiency}, which directly translates to longer
  battery life for mobile devices --- a critical advantage for the
  uplink.
\end{enumerate}

\begin{center}\rule{0.5\linewidth}{0.5pt}\end{center}

\section{Problem 2.5.8}\label{problem-2.5.8}

\textbf{Given:} A Wavelength Division Multiplexing (WDM) fiber optic
system uses Dense WDM (DWDM) in the C-band (1530--1565 nm) with 50 GHz
channel spacing. Each channel carries a 100 Gbps signal using DP-QPSK
(dual-polarization QPSK) modulation.

\textbf{Find:} (a) The number of channels that fit in the C-band, (b)
the total system capacity, (c) the channel bandwidth in nm at 1550 nm
center wavelength, (d) the spectral efficiency per channel, and (e) the
spectral efficiency of the full system.

\textbf{Solution:}

\begin{enumerate}
\def\labelenumi{(\alph{enumi})}
\tightlist
\item
  C-band range in frequency: f\textsubscript{start} =
  c/λ\textsubscript{end} = 3 × 10⁸ / 1,565 × 10⁻⁹ = 191.69 THz
  f\textsubscript{end} = 3 × 10⁸ / 1,530 × 10⁻⁹ = 196.08 THz Total
  bandwidth = 196.08 − 191.69 = 4.39 THz Channels = 4,390 GHz / 50 GHz =
  \textbf{87 channels} (88 with edge channels)
\end{enumerate}

Using the ITU grid: typically \textbf{80 channels} in the C-band with 50
GHz spacing.

\begin{enumerate}
\def\labelenumi{(\alph{enumi})}
\setcounter{enumi}{1}
\item
  Total capacity: 80 × 100 = \textbf{8,000 Gbps = 8 Tbps}
\item
  Channel spacing in wavelength at 1,550 nm: Δλ = λ² × Δf / c = (1,550 ×
  10⁻⁹)² × 50 × 10⁹ / (3 × 10⁸) = 2.4025 × 10⁻¹² × 50 × 10⁹ / 3 × 10⁸ Δλ
  = 1.2013 × 10⁻¹ × 10⁻⁹ = \textbf{0.40 nm}
\item
  Per channel: DP-QPSK carries 4 bits/symbol (2 polarizations × 2
  bits/symbol). Symbol rate = 100 Gbps / 4 = 25 Gbaud. In 50 GHz
  spacing: η = 100 / 50 = \textbf{2 bits/s/Hz}
\item
  Full system spectral efficiency: Total capacity / Total bandwidth =
  8,000 / 4,000 = \textbf{2 bits/s/Hz}
\end{enumerate}

Modern systems using DP-16QAM at 32 Gbaud achieve 400 Gbps per channel,
pushing spectral efficiency to 8 bits/s/Hz per channel.

\begin{center}\rule{0.5\linewidth}{0.5pt}\end{center}

\section{Problem 2.5.9}\label{problem-2.5.9}

\textbf{Given:} An OFDMA system (similar to Wi-Fi 6, 802.11ax) operates
in a 20 MHz channel with 256 subcarriers (9.765625 kHz spacing). The
resource allocation divides the channel into resource units (RUs): 9 RUs
of 26 subcarriers each (234 data subcarriers total). Three stations are
scheduled simultaneously: STA1 gets 4 RUs (106 subcarriers) with
256-QAM, 3/4 coding; STA2 gets 3 RUs (78 subcarriers) with 64-QAM, 2/3
coding; STA3 gets 2 RUs (52 subcarriers) with QPSK, 1/2 coding.

\textbf{Find:} (a) The OFDM symbol duration (with 0.8 μs GI), (b) the
data rate per station, (c) the total aggregate throughput, and (d) the
per-station spectral efficiency.

\textbf{Solution:}

\begin{enumerate}
\def\labelenumi{(\alph{enumi})}
\item
  Useful symbol duration: T\textsubscript{u} = 1/Δf = 1/9,765.625 =
  102.4 μs Wait --- 802.11ax uses 78.125 kHz spacing for 20 MHz with 256
  subcarriers: Δf = 78.125 kHz, T\textsubscript{u} = 1/78,125 =
  \textbf{12.8 μs} Total: T\textsubscript{total} = 12.8 + 0.8 =
  \textbf{13.6 μs}
\item
  Data rates: STA1: 106 subcarriers × 8 bits × 3/4 / 13.6 μs = 106 × 6 /
  13.6 × 10⁻⁶ = 636 / 13.6 × 10⁻⁶ = \textbf{46.76 Mbps} STA2: 78 × 6 ×
  2/3 / 13.6 × 10⁻⁶ = 78 × 4 / 13.6 × 10⁻⁶ = 312 / 13.6 × 10⁻⁶ =
  \textbf{22.94 Mbps} STA3: 52 × 2 × 1/2 / 13.6 × 10⁻⁶ = 52 × 1 / 13.6 ×
  10⁻⁶ = 52 / 13.6 × 10⁻⁶ = \textbf{3.82 Mbps}
\item
  Total throughput: 46.76 + 22.94 + 3.82 = \textbf{73.52 Mbps}
\item
  Per-station spectral efficiency: STA1: 46.76 / (106 × 78.125 × 10⁻³) =
  46.76 / 8.28 = \textbf{5.65 bits/s/Hz} STA2: 22.94 / (78 × 78.125 ×
  10⁻³) = 22.94 / 6.09 = \textbf{3.77 bits/s/Hz} STA3: 3.82 / (52 ×
  78.125 × 10⁻³) = 3.82 / 4.06 = \textbf{0.94 bits/s/Hz} System: 73.52 /
  20 = \textbf{3.68 bits/s/Hz}
\end{enumerate}

\begin{center}\rule{0.5\linewidth}{0.5pt}\end{center}

\section{Problem 2.5.10}\label{problem-2.5.10}

\textbf{Given:} A statistical TDM (stat-mux) system aggregates 20 bursty
data sources, each with a peak rate of 2 Mbps and an average utilization
of 30\%. The shared output link has a capacity of 15 Mbps. Overflow
traffic is buffered in a 500 KB queue.

\textbf{Find:} (a) The total peak aggregate input rate, (b) the average
aggregate input rate, (c) the oversubscription ratio (peak input / link
capacity), (d) the probability that instantaneous demand exceeds link
capacity (using the Gaussian approximation), and (e) the maximum burst
duration before the buffer overflows.

\textbf{Solution:}

\begin{enumerate}
\def\labelenumi{(\alph{enumi})}
\item
  Total peak rate: 20 × 2 = \textbf{40 Mbps}
\item
  Average aggregate rate: 20 × 2 × 0.30 = \textbf{12 Mbps}
\item
  Oversubscription ratio: 40 / 15 = \textbf{2.67:1}
\item
  Using the Gaussian approximation for the sum of 20 independent
  sources: Each source: mean = 0.3 × 2 = 0.6 Mbps, variance = 0.3 × 0.7
  × 2² = 0.84 Mbps² Aggregate: μ = 20 × 0.6 = 12 Mbps, σ² = 20 × 0.84 =
  16.8, σ = 4.099 Mbps
\end{enumerate}

P(demand \textgreater{} 15) = Q((15 − 12) / 4.099) = Q(0.732) =
\textbf{0.232} (23.2\%)

This is the probability that instantaneous demand exceeds the link rate,
requiring buffering.

\begin{enumerate}
\def\labelenumi{(\alph{enumi})}
\setcounter{enumi}{4}
\tightlist
\item
  Maximum burst scenario: all 20 sources transmit at peak simultaneously
  (40 Mbps). Buffer drain rate during overflow: input − output = 40 − 15
  = 25 Mbps. Buffer capacity: 500 KB = 500 × 8 = 4,000 kbits = 4 Mbits.
  Maximum burst before overflow: t = 4,000 / 25,000 = 0.16 s =
  \textbf{160 ms}
\end{enumerate}

In practice, the statistical multiplexing gain is substantial: 20
sources at 2 Mbps peak are served by a 15 Mbps link (62.5\% savings)
because the probability that all sources transmit simultaneously is
negligibly small ((0.3)²⁰ = 3.5 × 10⁻¹¹).

\chapter{Chapter 2 --- Section 2.6: Information
Theory}\label{chapter-2-section-2.6-information-theory}

Practice problems covering Shannon's channel capacity, entropy, source
coding, Huffman coding, rate-distortion theory, and capacity limits of
various channels.

\begin{center}\rule{0.5\linewidth}{0.5pt}\end{center}

\section{Problem 2.6.1}\label{problem-2.6.1}

\textbf{Given:} A digital subscriber line (DSL) uses a frequency band
from 25 kHz to 1.1 MHz. The average SNR across the band is 40 dB, but
the actual SNR varies: 50 dB from 25--300 kHz, 40 dB from 300--600 kHz,
30 dB from 600--900 kHz, and 20 dB from 900 kHz--1.1 MHz.

\textbf{Find:} (a) The Shannon capacity assuming a flat 40 dB SNR across
the entire band, (b) the capacity of each sub-band, (c) the total
capacity using the sub-band calculation, and (d) the percentage
difference between the flat and sub-band estimates.

\textbf{Solution:}

\begin{enumerate}
\def\labelenumi{(\alph{enumi})}
\item
  Flat SNR model: B\textsubscript{total} = 1,100 − 25 = 1,075 kHz SNR =
  40 dB = 10,000 C\textsubscript{flat} = 1,075,000 × log₂(1 + 10,000) =
  1,075,000 × 13.29 = \textbf{14.29 Mbps}
\item
  Sub-band capacities: Band 1 (25--300 kHz, B = 275 kHz, SNR = 50 dB =
  100,000): C₁ = 275,000 × log₂(100,001) = 275,000 × 16.61 =
  \textbf{4.568 Mbps}
\end{enumerate}

Band 2 (300--600 kHz, B = 300 kHz, SNR = 40 dB = 10,000): C₂ = 300,000 ×
log₂(10,001) = 300,000 × 13.29 = \textbf{3.987 Mbps}

Band 3 (600--900 kHz, B = 300 kHz, SNR = 30 dB = 1,000): C₃ = 300,000 ×
log₂(1,001) = 300,000 × 9.968 = \textbf{2.990 Mbps}

Band 4 (900--1,100 kHz, B = 200 kHz, SNR = 20 dB = 100): C₄ = 200,000 ×
log₂(101) = 200,000 × 6.658 = \textbf{1.332 Mbps}

\begin{enumerate}
\def\labelenumi{(\alph{enumi})}
\setcounter{enumi}{2}
\item
  Total sub-band capacity: 4.568 + 3.987 + 2.990 + 1.332 =
  \textbf{12.877 Mbps}
\item
  Difference: (14.29 − 12.88) / 14.29 × 100 = \textbf{9.9\%}
\end{enumerate}

The flat model overestimates capacity by about 10\% because it ignores
the degraded SNR at higher frequencies. Real DSL systems use DMT
(Discrete Multi-Tone, a form of OFDM) to allocate bits per subcarrier
based on the actual SNR at each frequency --- a practical implementation
of water-filling that approaches the sub-band capacity.

\begin{center}\rule{0.5\linewidth}{0.5pt}\end{center}

\section{Problem 2.6.2}\label{problem-2.6.2}

\textbf{Given:} A discrete memoryless source emits four symbols A, B, C,
D with probabilities P(A) = 0.5, P(B) = 0.25, P(C) = 0.125, P(D) =
0.125.

\textbf{Find:} (a) The entropy of the source, (b) the Huffman code for
these symbols, (c) the average code length, (d) the coding efficiency (η
= H/L\textsubscript{avg}), and (e) the maximum compression ratio
relative to a fixed 2-bit code.

\textbf{Solution:}

\begin{enumerate}
\def\labelenumi{(\alph{enumi})}
\item
  Entropy: H(X) = −0.5 log₂(0.5) − 0.25 log₂(0.25) − 0.125 log₂(0.125) −
  0.125 log₂(0.125) H(X) = 0.5 × 1 + 0.25 × 2 + 0.125 × 3 + 0.125 × 3
  H(X) = 0.5 + 0.5 + 0.375 + 0.375 = \textbf{1.75 bits/symbol}
\item
  Huffman code construction: Combine D(0.125) and C(0.125) → CD(0.25)
  Combine CD(0.25) and B(0.25) → BCD(0.5) Combine BCD(0.5) and A(0.5) →
  root(1.0)
\end{enumerate}

Assignments: A = 0, B = 10, C = 110, D = 111

{\def\LTcaptype{none} % do not increment counter
\begin{longtable}[]{@{}llll@{}}
\toprule\noalign{}
Symbol & Probability & Code & Length \\
\midrule\noalign{}
\endhead
\bottomrule\noalign{}
\endlastfoot
A & 0.500 & 0 & 1 \\
B & 0.250 & 10 & 2 \\
C & 0.125 & 110 & 3 \\
D & 0.125 & 111 & 3 \\
\end{longtable}
}

\begin{enumerate}
\def\labelenumi{(\alph{enumi})}
\setcounter{enumi}{2}
\item
  Average code length: L\textsubscript{avg} = 0.5 × 1 + 0.25 × 2 + 0.125
  × 3 + 0.125 × 3 = 0.5 + 0.5 + 0.375 + 0.375 = \textbf{1.75
  bits/symbol}
\item
  Efficiency: η = H(X) / L\textsubscript{avg} = 1.75 / 1.75 =
  \textbf{100\%}
\end{enumerate}

This is an optimal Huffman code because the probabilities are all
negative powers of 2 (dyadic distribution).

\begin{enumerate}
\def\labelenumi{(\alph{enumi})}
\setcounter{enumi}{4}
\tightlist
\item
  Compression ratio: 2.0 / 1.75 = \textbf{1.143:1} (14.3\% compression)
\end{enumerate}

\begin{center}\rule{0.5\linewidth}{0.5pt}\end{center}

\section{Problem 2.6.3}\label{problem-2.6.3}

\textbf{Given:} A binary symmetric channel (BSC) has a crossover
probability p = 0.01 (1\% error rate). The channel is used with a
transmission rate of 10 Mbps.

\textbf{Find:} (a) The channel capacity in bits per channel use, (b) the
capacity in Mbps, (c) the maximum achievable reliable data rate, (d) the
minimum code rate required to achieve arbitrarily low error probability,
and (e) the capacity at p = 0.1 for comparison.

\textbf{Solution:}

\begin{enumerate}
\def\labelenumi{(\alph{enumi})}
\item
  BSC capacity: C = 1 − H(p) = 1 − {[}−p log₂(p) − (1−p) log₂(1−p){]}
  H(0.01) = −0.01 × log₂(0.01) − 0.99 × log₂(0.99) H(0.01) = −0.01 ×
  (−6.644) − 0.99 × (−0.01449) H(0.01) = 0.06644 + 0.01434 = 0.08078 C =
  1 − 0.08078 = \textbf{0.919 bits/channel use}
\item
  Capacity: C\textsubscript{rate} = 0.919 × 10 = \textbf{9.19 Mbps}
\item
  Maximum reliable data rate = C\textsubscript{rate} = \textbf{9.19
  Mbps}
\end{enumerate}

By Shannon's theorem, any rate below 9.19 Mbps can be achieved with
arbitrarily low error probability using sufficiently long codes. Rates
above 9.19 Mbps will inevitably have a non-zero error floor.

\begin{enumerate}
\def\labelenumi{(\alph{enumi})}
\setcounter{enumi}{3}
\item
  Minimum code rate: R\textsubscript{c,min} = C = \textbf{0.919} Codes
  must have rate ≤ 0.919, meaning at least (1 − 0.919) × 100 = 8.1\%
  redundancy.
\item
  At p = 0.1: H(0.1) = −0.1 × log₂(0.1) − 0.9 × log₂(0.9) = 0.3322 +
  0.1368 = 0.4690 C = 1 − 0.469 = \textbf{0.531 bits/channel use} = 5.31
  Mbps
\end{enumerate}

A 10× increase in error rate (0.01 to 0.1) reduces capacity by only 42\%
(9.19 to 5.31 Mbps), demonstrating the logarithmic sensitivity of
capacity to channel quality.

\begin{center}\rule{0.5\linewidth}{0.5pt}\end{center}

\section{Problem 2.6.4}\label{problem-2.6.4}

\textbf{Given:} An English text source has 26 letters plus a space
character (27 symbols). In natural English, the letter frequencies are
approximately: space (18.3\%), E (10.2\%), T (7.7\%), A (6.5\%), O
(6.2\%), I (5.7\%), N (5.7\%), and the remaining 20 characters share the
remaining 39.7\%.

\textbf{Find:} (a) The maximum entropy if all 27 symbols were equally
likely, (b) the approximate first-order entropy using the given
frequencies (and assuming the 20 remaining characters have equal
probability of 39.7/20 = 1.985\% each), (c) the redundancy, (d) the
theoretical compression limit, and (e) comparison to practical
compression (gzip achieves \textasciitilde2.5 bits/character for
English).

\textbf{Solution:}

\begin{enumerate}
\def\labelenumi{(\alph{enumi})}
\item
  Maximum entropy: H\textsubscript{max} = log₂(27) = \textbf{4.755
  bits/symbol}
\item
  First-order entropy (using the given frequencies): H₁ = −{[}0.183
  log₂(0.183) + 0.102 log₂(0.102) + 0.077 log₂(0.077)
\end{enumerate}

\begin{itemize}
\tightlist
\item
  0.065 log₂(0.065) + 0.062 log₂(0.062) + 0.057 log₂(0.057)
\item
  0.057 log₂(0.057) + 20 × 0.01985 log₂(0.01985){]}
\end{itemize}

Computing each term: Space: 0.183 × 2.450 = 0.4484 E: 0.102 × 3.293 =
0.3359 T: 0.077 × 3.698 = 0.2847 A: 0.065 × 3.943 = 0.2563 O: 0.062 ×
4.012 = 0.2487 I: 0.057 × 4.133 = 0.2356 N: 0.057 × 4.133 = 0.2356 20
others: 20 × 0.01985 × 5.654 = 20 × 0.1122 = 2.2448

H₁ = 0.4484 + 0.3359 + 0.2847 + 0.2563 + 0.2487 + 0.2356 + 0.2356 +
2.2448 = \textbf{4.290 bits/symbol}

\begin{enumerate}
\def\labelenumi{(\alph{enumi})}
\setcounter{enumi}{2}
\tightlist
\item
  Redundancy: R = 1 − H₁/H\textsubscript{max} = 1 − 4.290/4.755 = 1 −
  0.902 = \textbf{0.098 = 9.8\%}
\end{enumerate}

This is only the first-order redundancy. When letter sequences (digrams,
trigrams) are considered, the true entropy of English drops to
approximately 1.0--1.5 bits/character (Shannon's estimate), giving
redundancy of 68--79\%.

\begin{enumerate}
\def\labelenumi{(\alph{enumi})}
\setcounter{enumi}{3}
\item
  Theoretical compression limit: 4.290/4.755 = 0.902 of original size
  using first-order statistics, or 4.290/8.0 = \textbf{0.536 of ASCII
  encoding} (46.4\% reduction).
\item
  Practical gzip: \textasciitilde2.5 bits/character = 2.5/8.0 = 31.25\%
  of ASCII. This exceeds the first-order limit (4.29 bits) because gzip
  exploits higher-order statistical patterns (repeated words and
  phrases). The Shannon limit (≈1.3 bits/char) suggests further
  \textbf{1.9× improvement} is theoretically possible.
\end{enumerate}

\begin{center}\rule{0.5\linewidth}{0.5pt}\end{center}

\section{Problem 2.6.5}\label{problem-2.6.5}

\textbf{Given:} A 4G LTE downlink channel has 10 MHz bandwidth and
operates with various SNR levels depending on the user's distance from
the tower. Consider three scenarios: (a) cell center with SNR = 25 dB,
(b) mid-cell with SNR = 15 dB, (c) cell edge with SNR = 3 dB.

\textbf{Find:} For each scenario: (a) the Shannon capacity, (b) the
spectral efficiency, (c) the closest standard LTE modulation and coding
scheme (MCS), and (d) the practical achievable rate (typically 60--75\%
of Shannon capacity due to overhead and imperfect coding).

\textbf{Solution:}

\begin{enumerate}
\def\labelenumi{(\alph{enumi})}
\item
  \textbf{Cell center (SNR = 25 dB = 316.2):} C = 10 × 10⁶ × log₂(1 +
  316.2) = 10 × 10⁶ × 8.31 = \textbf{83.1 Mbps} Spectral efficiency =
  \textbf{8.31 bits/s/Hz} Closest MCS: 64-QAM with 0.93 code rate → 6 ×
  0.93 = 5.58 bits/s/Hz → \textasciitilde55.8 Mbps Practical rate: 0.70
  × 83.1 = \textbf{58.2 Mbps} (matches 64-QAM, high code rate)
\item
  \textbf{Mid-cell (SNR = 15 dB = 31.62):} C = 10 × 10⁶ × log₂(1 +
  31.62) = 10 × 10⁶ × 5.03 = \textbf{50.3 Mbps} Spectral efficiency =
  \textbf{5.03 bits/s/Hz} Closest MCS: 16-QAM with 0.60 code rate → 4 ×
  0.60 = 2.40 bits/s/Hz → \textasciitilde24 Mbps Practical rate: 0.65 ×
  50.3 = \textbf{32.7 Mbps} (between 16-QAM and 64-QAM MCS levels)
\item
  \textbf{Cell edge (SNR = 3 dB = 2.0):} C = 10 × 10⁶ × log₂(1 + 2.0) =
  10 × 10⁶ × 1.585 = \textbf{15.85 Mbps} Spectral efficiency =
  \textbf{1.585 bits/s/Hz} Closest MCS: QPSK with 0.44 code rate → 2 ×
  0.44 = 0.88 bits/s/Hz → \textasciitilde8.8 Mbps Practical rate: 0.60 ×
  15.85 = \textbf{9.5 Mbps} (matches QPSK, low code rate)
\end{enumerate}

The gap between Shannon capacity and achievable rate widens at lower SNR
because fixed block-length codes are less efficient and higher overhead
fractions consume more of the limited throughput.

\begin{center}\rule{0.5\linewidth}{0.5pt}\end{center}

\section{Problem 2.6.6}\label{problem-2.6.6}

\textbf{Given:} A source emits symbols from an alphabet of 8 characters
with probabilities: \{0.30, 0.20, 0.15, 0.12, 0.10, 0.06, 0.04, 0.03\}.

\textbf{Find:} (a) The entropy, (b) the Huffman code, (c) the average
code length, (d) the coding efficiency, and (e) the average code length
if two symbols are coded jointly (2nd-order Huffman).

\textbf{Solution:}

\begin{enumerate}
\def\labelenumi{(\alph{enumi})}
\tightlist
\item
  Entropy: H = −{[}0.30 log₂(0.30) + 0.20 log₂(0.20) + 0.15 log₂(0.15) +
  0.12 log₂(0.12)
\end{enumerate}

\begin{itemize}
\tightlist
\item
  0.10 log₂(0.10) + 0.06 log₂(0.06) + 0.04 log₂(0.04) + 0.03
  log₂(0.03){]}
\end{itemize}

H = 0.30(1.737) + 0.20(2.322) + 0.15(2.737) + 0.12(3.059) + 0.10(3.322)
+ 0.06(4.059) + 0.04(4.644) + 0.03(5.059) H = 0.521 + 0.464 + 0.411 +
0.367 + 0.332 + 0.244 + 0.186 + 0.152 H = \textbf{2.677 bits/symbol}

\begin{enumerate}
\def\labelenumi{(\alph{enumi})}
\setcounter{enumi}{1}
\tightlist
\item
  Huffman code (building the tree by combining lowest probabilities):
\end{enumerate}

{\def\LTcaptype{none} % do not increment counter
\begin{longtable}[]{@{}llll@{}}
\toprule\noalign{}
Symbol & Prob & Code & Length \\
\midrule\noalign{}
\endhead
\bottomrule\noalign{}
\endlastfoot
S₁ & 0.30 & 00 & 2 \\
S₂ & 0.20 & 10 & 2 \\
S₃ & 0.15 & 010 & 3 \\
S₄ & 0.12 & 110 & 3 \\
S₅ & 0.10 & 111 & 3 \\
S₆ & 0.06 & 0110 & 4 \\
S₇ & 0.04 & 01110 & 5 \\
S₈ & 0.03 & 01111 & 5 \\
\end{longtable}
}

\begin{enumerate}
\def\labelenumi{(\alph{enumi})}
\setcounter{enumi}{2}
\item
  Average code length: L = 0.30(2) + 0.20(2) + 0.15(3) + 0.12(3) +
  0.10(3) + 0.06(4) + 0.04(5) + 0.03(5) L = 0.60 + 0.40 + 0.45 + 0.36 +
  0.30 + 0.24 + 0.20 + 0.15 = \textbf{2.70 bits/symbol}
\item
  Efficiency: η = H/L = 2.677/2.70 = \textbf{99.1\%}
\item
  Second-order Huffman codes blocks of 2 symbols (64 pairs). The average
  code length per original symbol approaches the entropy: L₂ ≈ H + 1/(2
  × 64) ≈ 2.677 + 0.008 ≈ \textbf{2.685 bits/symbol} per original symbol
\end{enumerate}

The improvement from 2.70 to 2.685 bits/symbol is small but meaningful
at high data rates. Arithmetic coding would achieve even closer to 2.677
without the combinatorial explosion of higher-order Huffman trees.

\begin{center}\rule{0.5\linewidth}{0.5pt}\end{center}

\section{Problem 2.6.7}\label{problem-2.6.7}

\textbf{Given:} A MIMO channel has N\textsubscript{t} = 4 transmit and
N\textsubscript{r} = 4 receive antennas. The channel is rich scattering
(full rank). The total transmit SNR is 20 dB, equally distributed among
transmit antennas. The singular values of the channel matrix H are σ₁ =
3.2, σ₂ = 2.1, σ₃ = 1.0, σ₄ = 0.3.

\textbf{Find:} (a) The SNR per transmit antenna, (b) the SNR on each
spatial sub-channel, (c) the capacity of each sub-channel, (d) the total
MIMO capacity, and (e) the comparison to a SISO system at the same total
power.

\textbf{Solution:}

\begin{enumerate}
\def\labelenumi{(\alph{enumi})}
\item
  Total SNR = 20 dB = 100. Per antenna: SNR\textsubscript{ant} = 100/4 =
  \textbf{25 per antenna}
\item
  SNR per spatial sub-channel: SNR\textsubscript{i} =
  σ\textsubscript{i}² × SNR\textsubscript{ant} Channel 1: 3.2² × 25 =
  10.24 × 25 = \textbf{256.0} (24.08 dB) Channel 2: 2.1² × 25 = 4.41 ×
  25 = \textbf{110.25} (20.42 dB) Channel 3: 1.0² × 25 = 1.0 × 25 =
  \textbf{25.0} (13.98 dB) Channel 4: 0.3² × 25 = 0.09 × 25 =
  \textbf{2.25} (3.52 dB)
\item
  Sub-channel capacities: C₁ = log₂(1 + 256.0) = log₂(257) =
  \textbf{8.01 bits/s/Hz} C₂ = log₂(1 + 110.25) = log₂(111.25) =
  \textbf{6.80 bits/s/Hz} C₃ = log₂(1 + 25.0) = log₂(26) = \textbf{4.70
  bits/s/Hz} C₄ = log₂(1 + 2.25) = log₂(3.25) = \textbf{1.70 bits/s/Hz}
\item
  Total MIMO capacity: C\textsubscript{MIMO} = 8.01 + 6.80 + 4.70 + 1.70
  = \textbf{21.21 bits/s/Hz}
\item
  SISO capacity at same total power (SNR = 100): C\textsubscript{SISO} =
  log₂(1 + 100) = log₂(101) = \textbf{6.66 bits/s/Hz}
\end{enumerate}

MIMO gain = 21.21 / 6.66 = \textbf{3.18×}

The MIMO system achieves 3.18× the SISO capacity. The gain is less than
4× (the number of spatial channels) because the 4th sub-channel has low
SNR (σ₄ = 0.3). Water-filling power allocation would shift power from
the weak 4th channel to the stronger channels, potentially increasing
total capacity slightly.

\begin{center}\rule{0.5\linewidth}{0.5pt}\end{center}

\section{Problem 2.6.8}\label{problem-2.6.8}

\textbf{Given:} A rate-distortion problem: a Gaussian source with
variance σ² = 1 is to be compressed with a maximum mean-squared error
distortion of D = 0.01. The rate-distortion function for a Gaussian
source is R(D) = (1/2) log₂(σ²/D) bits/sample.

\textbf{Find:} (a) The minimum bit rate required, (b) the equivalent
number of quantization bits, (c) the rate for D = 0.1, (d) the rate for
D = 0.001, and (e) the rate saving from allowing 10× more distortion (D
= 0.01 vs D = 0.1).

\textbf{Solution:}

\begin{enumerate}
\def\labelenumi{(\alph{enumi})}
\item
  R(D) = (1/2) log₂(1/0.01) = (1/2) log₂(100) = (1/2) × 6.644 =
  \textbf{3.322 bits/sample}
\item
  Equivalent bits: A uniform quantizer with N bits achieves D ≈ σ² ×
  2⁻²ᴺ. For D = 0.01: 2⁻²ᴺ = 0.01 → 2N = log₂(100) = 6.644 → N = 3.32
  bits. Equivalent to approximately \textbf{3.3 bits} per sample
  (between 3 and 4 bit uniform quantization).
\item
  R(0.1) = (1/2) log₂(1/0.1) = (1/2) × 3.322 = \textbf{1.661
  bits/sample}
\item
  R(0.001) = (1/2) log₂(1/0.001) = (1/2) × 9.966 = \textbf{4.983
  bits/sample}
\item
  Rate saving from D = 0.01 to D = 0.1: ΔR = 3.322 − 1.661 =
  \textbf{1.661 bits/sample saved}
\end{enumerate}

This is exactly (1/2) log₂(10) = 1.661 bits --- allowing 10× more
distortion saves 1.66 bits per sample. This is the rate-distortion
trade-off: each halving of the distortion costs exactly 0.5 bits/sample
for Gaussian sources.

\begin{center}\rule{0.5\linewidth}{0.5pt}\end{center}

\section{Problem 2.6.9}\label{problem-2.6.9}

\textbf{Given:} A wireless channel undergoes flat Rayleigh fading with
an average SNR of γ̄ = 20 dB. The ergodic capacity of a Rayleigh fading
channel is C = E{[}log₂(1 + γ){]} = e\textsuperscript{1/γ̄} × E₁(1/γ̄) /
ln(2), where E₁ is the exponential integral. For high average SNR (γ̄
\textgreater\textgreater{} 1), this simplifies to C ≈ log₂(γ̄) − 0.833
bits/s/Hz.

\textbf{Find:} (a) The AWGN capacity at SNR = 20 dB (for comparison),
(b) the ergodic Rayleigh fading capacity using the high-SNR
approximation, (c) the capacity loss due to fading, (d) the outage
capacity at 10\% outage probability (the rate achievable 90\% of the
time), and (e) the benefit of 2-branch receive diversity (which doubles
the average SNR).

\textbf{Solution:}

\begin{enumerate}
\def\labelenumi{(\alph{enumi})}
\item
  AWGN capacity: C\textsubscript{AWGN} = log₂(1 + 100) = log₂(101) =
  \textbf{6.66 bits/s/Hz}
\item
  Ergodic Rayleigh capacity (high-SNR approximation): γ̄ = 100 (linear)
  C\textsubscript{Rayleigh} ≈ log₂(100) − 0.833 = 6.644 − 0.833 =
  \textbf{5.81 bits/s/Hz}
\item
  Capacity loss due to fading: 6.66 − 5.81 = \textbf{0.85 bits/s/Hz}
  (12.8\% reduction)
\end{enumerate}

The ergodic capacity is the average over all fading states. Fading
reduces capacity because the channel spends some time at very low SNR,
where the capacity is near zero, and these deep fades are not fully
compensated by the high-SNR peaks (due to the logarithm's concavity).

\begin{enumerate}
\def\labelenumi{(\alph{enumi})}
\setcounter{enumi}{3}
\tightlist
\item
  Outage capacity at 10\% outage: For Rayleigh fading, P(γ \textless{}
  γ\textsubscript{th}) = 1 − e\textsuperscript{−γth/γ̄} = 0.10
  γ\textsubscript{th} = −γ̄ × ln(0.90) = −100 × (−0.1054) = 10.54 (10.23
  dB) C\textsubscript{out,10\%} = log₂(1 + 10.54) = log₂(11.54) =
  \textbf{3.53 bits/s/Hz}
\end{enumerate}

This is much lower than the ergodic capacity because 10\% of the time
the channel is in a deep fade.

\begin{enumerate}
\def\labelenumi{(\alph{enumi})}
\setcounter{enumi}{4}
\tightlist
\item
  With 2-branch MRC diversity, average SNR doubles to 2γ̄ = 200:
  C\textsubscript{diversity} ≈ log₂(200) − 0.833 = 7.644 − 0.833 =
  \textbf{6.81 bits/s/Hz} Gain from diversity: 6.81 − 5.81 = \textbf{1.0
  bits/s/Hz} --- the diversity gain recovers more than the fading loss.
\end{enumerate}

\begin{center}\rule{0.5\linewidth}{0.5pt}\end{center}

\section{Problem 2.6.10}\label{problem-2.6.10}

\textbf{Given:} A joint source-channel coding problem: a 720p video
stream requires 5 Mbps after source coding (H.264). The wireless channel
has bandwidth B = 2 MHz and SNR = 18 dB.

\textbf{Find:} (a) The Shannon capacity of the channel, (b) whether the
video can be transmitted reliably, (c) the minimum channel coding rate
needed, (d) the maximum source coding rate (video quality reduction) if
the channel cannot support the full rate, and (e) the bandwidth required
to reliably transmit the 5 Mbps stream.

\textbf{Solution:}

\begin{enumerate}
\def\labelenumi{(\alph{enumi})}
\item
  Channel capacity: SNR = 18 dB = 63.10 C = 2 × 10⁶ × log₂(1 + 63.10) =
  2 × 10⁶ × 6.00 = \textbf{12.0 Mbps}
\item
  The required rate of 5 Mbps is below the channel capacity of 12 Mbps.
  \textbf{Yes, the video can be transmitted reliably.} There is 12.0 −
  5.0 = 7.0 Mbps of margin for channel coding overhead.
\item
  The channel coding rate must satisfy: R\textsubscript{source} /
  R\textsubscript{channel code} ≤ C R\textsubscript{channel code} ≥
  R\textsubscript{source} / C = 5.0 / 12.0 = \textbf{0.417}
\end{enumerate}

So a code rate of at least 0.417 (e.g., rate 1/2) provides reliable
transmission. The remaining bandwidth supports the coding redundancy.

\begin{enumerate}
\def\labelenumi{(\alph{enumi})}
\setcounter{enumi}{3}
\item
  If the channel had lower capacity (say, C = 4 Mbps, insufficient for 5
  Mbps): Maximum video rate = C × R\textsubscript{channel code} = 4 ×
  0.75 = 3 Mbps (with rate 3/4 FEC) This would require reducing video
  quality from 5 Mbps to \textbf{3 Mbps} --- dropping resolution from
  720p to approximately 480p, or reducing frame rate from 30 fps to 18
  fps.
\item
  Minimum bandwidth for 5 Mbps with 18 dB SNR: 5 × 10⁶ = B × log₂(1 +
  63.10) = B × 6.00 B = 5 × 10⁶ / 6.00 = \textbf{833 kHz}
\end{enumerate}

This is the theoretical minimum. In practice, with a rate-1/2 code:
Required channel rate = 5 / 0.5 = 10 Mbps. With 16-QAM (4 bits/s/Hz): B
= 10 / 4 = \textbf{2.5 MHz} (practical minimum)

\chapter{Chapter 2 --- Section 2.7: Noise in Communication
Systems}\label{chapter-2-section-2.7-noise-in-communication-systems}

Practice problems covering thermal noise, noise power, noise voltage,
noise figure, noise temperature, cascaded system noise, Friis formula,
satellite link budgets, and G/T calculations.

\begin{center}\rule{0.5\linewidth}{0.5pt}\end{center}

\section{Problem 2.7.1}\label{problem-2.7.1}

\textbf{Given:} A microwave receiver front-end operates at a physical
temperature of T = 300 K. The receiver has a bandwidth of 25 MHz and an
input impedance of 50 Ω.

\textbf{Find:} (a) The thermal noise power in watts and dBm, (b) the
noise power spectral density (N₀) in dBm/Hz, (c) the RMS noise voltage
across the 50 Ω input, (d) the RMS noise voltage if the impedance were
75 Ω (common in cable TV), and (e) the noise power if the system were
cooled to 77 K (liquid nitrogen temperature).

\textbf{Solution:}

\begin{enumerate}
\def\labelenumi{(\alph{enumi})}
\tightlist
\item
  Noise power: N = kTB = 1.381 × 10⁻²³ × 300 × 25 × 10⁶ = 1.036 × 10⁻¹³
  W N(dBm) = 10 log₁₀(1.036 × 10⁻¹³ / 10⁻³) = 10 log₁₀(1.036 × 10⁻¹⁰) =
  \textbf{−99.8 dBm}
\end{enumerate}

Quick check: −174 + 10 log₁₀(25 × 10⁶) = −174 + 74.0 = −100.0 dBm (at
290 K) At 300 K: −100.0 + 10 log₁₀(300/290) = −100.0 + 0.15 =
\textbf{−99.8 dBm} ✓

\begin{enumerate}
\def\labelenumi{(\alph{enumi})}
\setcounter{enumi}{1}
\item
  N₀ = kT = 1.381 × 10⁻²³ × 300 = 4.143 × 10⁻²¹ W/Hz N₀(dBm/Hz) = 10
  log₁₀(4.143 × 10⁻²¹ / 10⁻³) = \textbf{−173.8 dBm/Hz}
\item
  RMS noise voltage (50 Ω): V\textsubscript{n} = √(4kTRB) = √(4 × 1.381
  × 10⁻²³ × 300 × 50 × 25 × 10⁶) V\textsubscript{n} = √(2.071 × 10⁻¹¹) =
  \textbf{4.55 μV}
\item
  For 75 Ω: V\textsubscript{n} = √(4kT × 75 × B) = √(4 × 1.381 × 10⁻²³ ×
  300 × 75 × 25 × 10⁶) V\textsubscript{n} = √(3.107 × 10⁻¹¹) =
  \textbf{5.57 μV}
\end{enumerate}

Note: despite the higher voltage, the noise power delivered to a matched
load remains kTB = −99.8 dBm regardless of impedance.

\begin{enumerate}
\def\labelenumi{(\alph{enumi})}
\setcounter{enumi}{4}
\tightlist
\item
  At 77 K: N = kTB = 1.381 × 10⁻²³ × 77 × 25 × 10⁶ = 2.659 × 10⁻¹⁴ W
  N(dBm) = \textbf{−105.8 dBm} Improvement: −99.8 − (−105.8) =
  \textbf{5.9 dB} reduction in noise floor by cryogenic cooling.
\end{enumerate}

\begin{center}\rule{0.5\linewidth}{0.5pt}\end{center}

\section{Problem 2.7.2}\label{problem-2.7.2}

\textbf{Given:} A receiver chain consists of three stages: (1) a
bandpass filter with 1.5 dB insertion loss, (2) an LNA with noise figure
1.2 dB and gain 25 dB, (3) a mixer with noise figure 10 dB and
conversion gain 0 dB (unity).

\textbf{Find:} (a) The linear noise figures and gains, (b) the system
noise figure using the Friis formula, (c) the system noise temperature,
and (d) the improvement if the filter is moved after the LNA.

\textbf{Solution:}

\begin{enumerate}
\def\labelenumi{(\alph{enumi})}
\item
  Converting to linear: Filter: F₁ = 10\textsuperscript{1.5/10} = 1.413,
  G₁ = 10\textsuperscript{−1.5/10} = 0.708 (passive loss) LNA: F₂ =
  10\textsuperscript{1.2/10} = 1.318, G₂ = 10\textsuperscript{25/10} =
  316.2 Mixer: F₃ = 10\textsuperscript{10/10} = 10.0, G₃ =
  10\textsuperscript{0/10} = 1.0
\item
  Friis formula: F\textsubscript{sys} = F₁ + (F₂ − 1)/G₁ + (F₃ − 1)/(G₁
  × G₂) F\textsubscript{sys} = 1.413 + (1.318 − 1)/0.708 + (10.0 −
  1)/(0.708 × 316.2) F\textsubscript{sys} = 1.413 + 0.449 + 0.0402 =
  \textbf{1.902} NF\textsubscript{sys} = 10 log₁₀(1.902) = \textbf{2.79
  dB}
\item
  Noise temperature: T\textsubscript{sys} = T₀(F\textsubscript{sys} − 1)
  = 290 × (1.902 − 1) = 290 × 0.902 = \textbf{261.6 K}
\item
  With LNA first, then filter, then mixer: F\textsubscript{sys} =
  F\textsubscript{LNA} + (F\textsubscript{filter} −
  1)/G\textsubscript{LNA} + (F\textsubscript{mixer} −
  1)/(G\textsubscript{LNA} × G\textsubscript{filter})
  F\textsubscript{sys} = 1.318 + (1.413 − 1)/316.2 + (10.0 − 1)/(316.2 ×
  0.708) F\textsubscript{sys} = 1.318 + 0.00131 + 0.0402 =
  \textbf{1.359} NF\textsubscript{sys} = 10 log₁₀(1.359) = \textbf{1.33
  dB}
\end{enumerate}

Improvement: 2.79 − 1.33 = \textbf{1.46 dB} by placing the LNA before
the filter. The system noise figure drops from 2.79 dB to 1.33 dB ---
nearly matching the LNA's own noise figure.

\begin{center}\rule{0.5\linewidth}{0.5pt}\end{center}

\section{Problem 2.7.3}\label{problem-2.7.3}

\textbf{Given:} A Ka-band satellite earth station operates at 20 GHz
(downlink). The antenna has a diameter of 1.2 m with 60\% aperture
efficiency. The system noise temperature breakdown: antenna noise
T\textsubscript{a} = 30 K (clear sky, elevation 30°), LNA noise
temperature T\textsubscript{LNA} = 65 K, and subsequent receiver stages
contribute 15 K (referred to input).

\textbf{Find:} (a) The antenna gain, (b) the system noise temperature,
(c) the G/T figure of merit, (d) the G/T degradation during rain
(T\textsubscript{a} increases to 120 K due to rain emission), and (e)
the effective noise figure of the LNA.

\textbf{Solution:}

\begin{enumerate}
\def\labelenumi{(\alph{enumi})}
\item
  Antenna gain at 20 GHz: λ = c/f = 3 × 10⁸ / 20 × 10⁹ = 0.015 m G = η ×
  (πD/λ)² = 0.60 × (π × 1.2/0.015)² = 0.60 × (251.3)² = 0.60 × 63,155 G
  = 37,893 = 10 log₁₀(37,893) = \textbf{45.8 dBi}
\item
  System noise temperature: T\textsubscript{sys} = T\textsubscript{a} +
  T\textsubscript{LNA} + T\textsubscript{rest} = 30 + 65 + 15 =
  \textbf{110 K}
\item
  G/T = G(dBi) − 10 log₁₀(T\textsubscript{sys}) = 45.8 − 10 log₁₀(110) =
  45.8 − 20.41 = \textbf{25.4 dB/K}
\item
  During rain, T\textsubscript{a} = 120 K: T\textsubscript{sys,rain} =
  120 + 65 + 15 = 200 K G/T\textsubscript{rain} = 45.8 − 10 log₁₀(200) =
  45.8 − 23.01 = \textbf{22.8 dB/K} G/T degradation = 25.4 − 22.8 =
  \textbf{2.6 dB}
\end{enumerate}

This 2.6 dB degradation adds to the direct rain attenuation loss, making
Ka-band particularly sensitive to rain.

\begin{enumerate}
\def\labelenumi{(\alph{enumi})}
\setcounter{enumi}{4}
\tightlist
\item
  LNA noise figure: NF = 10 log₁₀(1 + T\textsubscript{LNA}/T₀) = 10
  log₁₀(1 + 65/290) = 10 log₁₀(1.224) = \textbf{0.88 dB}
\end{enumerate}

\begin{center}\rule{0.5\linewidth}{0.5pt}\end{center}

\section{Problem 2.7.4}\label{problem-2.7.4}

\textbf{Given:} A radio astronomy receiver operates at 1,420 MHz
(hydrogen line) with a system noise temperature of 25 K (cryogenically
cooled). The antenna has an effective collecting area of 500 m² and the
observation bandwidth is 1 MHz.

\textbf{Find:} (a) The system noise power, (b) the antenna gain, (c) the
minimum detectable flux density for SNR = 5 after 1 hour of integration
(radiometer equation: SNR =
S\textsubscript{signal}/(S\textsubscript{noise}/√(Bt))), (d) the noise
power spectral density, and (e) the equivalent noise voltage across a 50
Ω feed.

\textbf{Solution:}

\begin{enumerate}
\def\labelenumi{(\alph{enumi})}
\item
  Noise power: N = kT\textsubscript{sys}B = 1.381 × 10⁻²³ × 25 × 10⁶ =
  3.453 × 10⁻¹⁶ W N(dBW) = 10 log₁₀(3.453 × 10⁻¹⁶) = \textbf{−154.6 dBW}
\item
  Antenna gain: G = 4πA\textsubscript{eff}/λ² λ = c/f = 3 × 10⁸ / 1.42 ×
  10⁹ = 0.2113 m G = 4π × 500 / (0.2113)² = 6,283.2 / 0.04465 = 140,730
  G(dBi) = 10 log₁₀(140,730) = \textbf{51.5 dBi}
\item
  Radiometer equation: SNR\textsubscript{out} =
  (T\textsubscript{source}/T\textsubscript{sys}) × √(Bt) For SNR = 5:
  T\textsubscript{source} = 5 × T\textsubscript{sys}/√(Bt) √(Bt) = √(10⁶
  × 3,600) = √(3.6 × 10⁹) = 60,000 T\textsubscript{source,min} = 5 × 25
  / 60,000 = \textbf{0.00208 K = 2.08 mK}
\end{enumerate}

Minimum flux density: S = 2kT\textsubscript{source}/A\textsubscript{eff}
= 2 × 1.381 × 10⁻²³ × 0.00208 / 500 S = 1.149 × 10⁻²⁸ W/m²/Hz =
\textbf{11.5 mJy} (1 Jy = 10⁻²⁶ W/m²/Hz)

\begin{enumerate}
\def\labelenumi{(\alph{enumi})}
\setcounter{enumi}{3}
\item
  N₀ = kT\textsubscript{sys} = 1.381 × 10⁻²³ × 25 = 3.453 × 10⁻²² W/Hz =
  \textbf{−214.6 dBW/Hz}
\item
  RMS noise voltage: V\textsubscript{n} = √(4kT\textsubscript{sys}RB) =
  √(4 × 1.381 × 10⁻²³ × 25 × 50 × 10⁶) V\textsubscript{n} = √(6.905 ×
  10⁻¹⁴) = \textbf{0.263 μV = 263 nV}
\end{enumerate}

\begin{center}\rule{0.5\linewidth}{0.5pt}\end{center}

\section{Problem 2.7.5}\label{problem-2.7.5}

\textbf{Given:} A cellular base station receiver chain: Stage 1:
Duplexer/filter (loss = 0.8 dB) Stage 2: Tower-mounted LNA (NF = 0.5 dB,
gain = 15 dB) Stage 3: Coaxial cable run (loss = 6 dB, length = 50 m)
Stage 4: Base station amplifier (NF = 4 dB, gain = 30 dB) Stage 5:
Mixer/IF chain (NF = 12 dB)

\textbf{Find:} (a) The overall system noise figure, (b) the system noise
temperature, (c) the receiver sensitivity for a 200 kHz channel
bandwidth at 10 dB required SNR, and (d) the noise figure if the cable
loss increases to 10 dB (longer cable run).

\textbf{Solution:}

\begin{enumerate}
\def\labelenumi{(\alph{enumi})}
\tightlist
\item
  Converting to linear: F₁ = 10\textsuperscript{0.8/10} = 1.202, G₁ =
  10\textsuperscript{−0.8/10} = 0.832 F₂ = 10\textsuperscript{0.5/10} =
  1.122, G₂ = 10\textsuperscript{15/10} = 31.62 F₃ =
  10\textsuperscript{6/10} = 3.981, G₃ = 10\textsuperscript{−6/10} =
  0.251 F₄ = 10\textsuperscript{4/10} = 2.512, G₄ =
  10\textsuperscript{30/10} = 1000 F₅ = 10\textsuperscript{12/10} =
  15.85
\end{enumerate}

Friis formula: F\textsubscript{sys} = F₁ + (F₂−1)/G₁ + (F₃−1)/(G₁G₂) +
(F₄−1)/(G₁G₂G₃) + (F₅−1)/(G₁G₂G₃G₄) F\textsubscript{sys} = 1.202 +
0.122/0.832 + 2.981/(0.832 × 31.62) + 1.512/(0.832 × 31.62 × 0.251) +
14.85/(0.832 × 31.62 × 0.251 × 1000) F\textsubscript{sys} = 1.202 +
0.147 + 0.1133 + 0.229 + 0.00225 F\textsubscript{sys} = \textbf{1.693}
NF\textsubscript{sys} = 10 log₁₀(1.693) = \textbf{2.29 dB}

\begin{enumerate}
\def\labelenumi{(\alph{enumi})}
\setcounter{enumi}{1}
\item
  T\textsubscript{sys} = 290 × (1.693 − 1) = 290 × 0.693 = \textbf{200.9
  K}
\item
  Sensitivity = −174 + 10 log₁₀(B) + NF + SNR\textsubscript{req} = −174
  + 10 log₁₀(200,000) + 2.29 + 10 = −174 + 53.01 + 2.29 + 10 =
  \textbf{−108.7 dBm}
\item
  With 10 dB cable loss: F₃ = 10, G₃ = 0.1 F\textsubscript{sys} = 1.202
  + 0.147 + 9.0/(0.832 × 31.62) + 1.512/(0.832 × 31.62 × 0.1) +
  14.85/(0.832 × 31.62 × 0.1 × 1000) F\textsubscript{sys} = 1.202 +
  0.147 + 0.342 + 0.575 + 0.00565 F\textsubscript{sys} = \textbf{2.272}
  NF\textsubscript{sys} = 10 log₁₀(2.272) = \textbf{3.56 dB}
\end{enumerate}

The 4 dB additional cable loss degrades the noise figure by 3.56 − 2.29
= \textbf{1.27 dB}, reducing sensitivity by the same amount. This is why
tower-mounted LNAs are critical.

\begin{center}\rule{0.5\linewidth}{0.5pt}\end{center}

\section{Problem 2.7.6}\label{problem-2.7.6}

\textbf{Given:} A geostationary satellite transponder at 36,000 km has
an EIRP of 48 dBW at 12 GHz (Ku-band). The earth station uses a 0.75 m
dish with 62\% aperture efficiency. The LNB (Low-Noise Block) has a
noise figure of 0.7 dB. Atmospheric losses are 0.3 dB and miscellaneous
losses are 0.5 dB. Antenna noise temperature is 35 K.

\textbf{Find:} (a) The free-space path loss, (b) the receive antenna
gain, (c) the received carrier power, (d) the system noise temperature,
and (e) the C/N in a 27 MHz transponder bandwidth.

\textbf{Solution:}

\begin{enumerate}
\def\labelenumi{(\alph{enumi})}
\item
  FSPL at 12 GHz, 36,000 km: λ = 3 × 10⁸ / 12 × 10⁹ = 0.025 m FSPL = 20
  log₁₀(4πd/λ) = 20 log₁₀(4π × 3.6 × 10⁷ / 0.025) = 20 log₁₀(1.810 ×
  10¹⁰) = 20 × 10.258 = \textbf{205.2 dB}
\item
  Receive gain: G = η(πD/λ)² = 0.62 × (π × 0.75/0.025)² = 0.62 ×
  (94.25)² = 0.62 × 8,883 = 5,507 G(dBi) = 10 log₁₀(5,507) =
  \textbf{37.4 dBi}
\item
  Received power: P\textsubscript{rx} = EIRP − FSPL +
  G\textsubscript{rx} − L\textsubscript{atm} − L\textsubscript{misc}
  P\textsubscript{rx} = 48 − 205.2 + 37.4 − 0.3 − 0.5 = \textbf{−120.6
  dBW = −90.6 dBm}
\item
  LNB noise temperature: T\textsubscript{LNB} = T₀(F − 1) = 290 ×
  (10\textsuperscript{0.7/10} − 1) = 290 × (1.175 − 1) = 290 × 0.175 =
  50.7 K System noise temperature: T\textsubscript{sys} =
  T\textsubscript{a} + T\textsubscript{LNB} = 35 + 50.7 = \textbf{85.7
  K}
\item
  Noise power: N = kT\textsubscript{sys}B = 1.381 × 10⁻²³ × 85.7 × 27 ×
  10⁶ = 3.197 × 10⁻¹⁴ W N(dBW) = 10 log₁₀(3.197 × 10⁻¹⁴) = −134.95 dBW
\end{enumerate}

Or: N = −228.6 + 10 log₁₀(85.7) + 10 log₁₀(27 × 10⁶) = −228.6 + 19.33 +
74.31 = −134.96 dBW

C/N = P\textsubscript{rx} − N = −120.6 − (−135.0) = \textbf{14.4 dB}

This provides comfortable margin for DVB-S2 QPSK (requires
\textasciitilde5.5 dB C/N) with approximately 9 dB of rain margin.

\begin{center}\rule{0.5\linewidth}{0.5pt}\end{center}

\section{Problem 2.7.7}\label{problem-2.7.7}

\textbf{Given:} A two-antenna diversity receiver combines signals using
maximal ratio combining (MRC). Each antenna path has: antenna gain = 5
dBi, cable loss = 2 dB, LNA noise figure = 2 dB, LNA gain = 20 dB. The
signal arrives at SNR = 8 dB per branch (before combining). The noise is
independent between branches.

\textbf{Find:} (a) The combined SNR after MRC (ideal:
SNR\textsubscript{MRC} = SNR₁ + SNR₂ for equal-SNR branches), (b) the
diversity gain in dB, (c) the effective system noise figure per branch,
(d) the combined noise figure, and (e) the BER improvement for BPSK (BER
with diversity vs.~without).

\textbf{Solution:}

\begin{enumerate}
\def\labelenumi{(\alph{enumi})}
\item
  Each branch: SNR = 8 dB = 6.31 (linear) MRC combined SNR = SNR₁ + SNR₂
  = 6.31 + 6.31 = 12.62 SNR\textsubscript{MRC}(dB) = 10 log₁₀(12.62) =
  \textbf{11.0 dB}
\item
  Diversity gain = 11.0 − 8.0 = \textbf{3.0 dB} This is the ideal
  2-branch MRC gain for equal-SNR branches with independent noise.
\item
  Per branch: cable (loss) then LNA: F\textsubscript{cable} =
  10\textsuperscript{2/10} = 1.585, G\textsubscript{cable} =
  10\textsuperscript{−2/10} = 0.631 F\textsubscript{LNA} =
  10\textsuperscript{2/10} = 1.585, G\textsubscript{LNA} = 100
\end{enumerate}

F\textsubscript{branch} = F\textsubscript{cable} + (F\textsubscript{LNA}
− 1)/G\textsubscript{cable} = 1.585 + 0.585/0.631 = 1.585 + 0.927 =
2.512 NF\textsubscript{branch} = 10 log₁₀(2.512) = \textbf{4.0 dB}

\begin{enumerate}
\def\labelenumi{(\alph{enumi})}
\setcounter{enumi}{3}
\item
  For MRC with independent noise, the combined noise figure is the same
  as each branch: NF\textsubscript{combined} = \textbf{4.0 dB} (the
  diversity gain comes from coherent signal addition, not noise
  reduction per se)
\item
  BPSK BER comparison at SNR per bit = 8 dB = 6.31: Without diversity:
  BER = Q(√(2 × 6.31)) = Q(3.553) = \textbf{1.90 × 10⁻⁴} With 2-branch
  MRC (SNR = 12.62): BER = Q(√(2 × 12.62)) = Q(5.024) = \textbf{2.53 ×
  10⁻⁷} Improvement: 1.90 × 10⁻⁴ / 2.53 × 10⁻⁷ = \textbf{751× reduction
  in BER}
\end{enumerate}

\begin{center}\rule{0.5\linewidth}{0.5pt}\end{center}

\section{Problem 2.7.8}\label{problem-2.7.8}

\textbf{Given:} A 5G millimeter-wave link operates at 28 GHz over a
distance of 200 m. The transmitter has P\textsubscript{tx} = 30 dBm and
uses a 16-element phased array (G\textsubscript{tx} = 18 dBi). The
receiver has a 16-element array (G\textsubscript{rx} = 18 dBi) with
overall noise figure NF = 6 dB. Additional losses include atmospheric
absorption (0.2 dB) and foliage loss (10 dB for a tree in the path). The
channel bandwidth is 400 MHz.

\textbf{Find:} (a) The free-space path loss, (b) the total path loss
(including atmospheric and foliage), (c) the received power, (d) the
noise floor, and (e) the achievable SNR and corresponding maximum
modulation order.

\textbf{Solution:}

\begin{enumerate}
\def\labelenumi{(\alph{enumi})}
\item
  FSPL at 28 GHz, 200 m: λ = 3 × 10⁸ / 28 × 10⁹ = 0.01071 m FSPL = 20
  log₁₀(4π × 200/0.01071) = 20 log₁₀(234,512) = 20 × 5.370 =
  \textbf{107.4 dB}
\item
  Total path loss: L\textsubscript{total} = 107.4 + 0.2 + 10 =
  \textbf{117.6 dB}
\item
  Received power: P\textsubscript{rx} = P\textsubscript{tx} +
  G\textsubscript{tx} − L\textsubscript{total} + G\textsubscript{rx} =
  30 + 18 − 117.6 + 18 = \textbf{−51.6 dBm}
\item
  Noise floor: N = −174 + 10 log₁₀(400 × 10⁶) + NF = −174 + 86.0 + 6 =
  \textbf{−82.0 dBm}
\item
  SNR = P\textsubscript{rx} − N = −51.6 − (−82.0) = \textbf{30.4 dB}
\end{enumerate}

At 30.4 dB SNR: - 256-QAM requires \textasciitilde27 dB SNR → margin =
3.4 dB → \textbf{feasible} but tight - 64-QAM requires \textasciitilde22
dB SNR → margin = 8.4 dB → comfortable - \textbf{64-QAM is recommended}
for reliability with the foliage loss

Without the foliage loss: SNR = 40.4 dB, supporting up to 1024-QAM. This
demonstrates the severe impact of obstructions at mmWave frequencies.

\begin{center}\rule{0.5\linewidth}{0.5pt}\end{center}

\section{Problem 2.7.9}\label{problem-2.7.9}

\textbf{Given:} A bent-pipe satellite transponder has the following link
parameters: Uplink: EIRP = 72 dBW, path loss = 207 dB, satellite G/T =
−2 dB/K, bandwidth = 36 MHz Downlink: satellite EIRP = 44 dBW, path loss
= 196 dB, earth station G/T = 20 dB/K, bandwidth = 36 MHz
Intermodulation noise: C/I = 22 dB

\textbf{Find:} (a) The uplink C/N, (b) the downlink C/N, (c) the overall
C/N using the reciprocal addition formula, and (d) the impact on overall
C/N if the uplink power is increased by 3 dB.

\textbf{Solution:}

\begin{enumerate}
\def\labelenumi{(\alph{enumi})}
\item
  Uplink C/N: (C/N)\textsubscript{up} = EIRP + G/T −
  L\textsubscript{path} − k − 10 log₁₀(B) = 72 + (−2) − 207 − (−228.6) −
  10 log₁₀(36 × 10⁶) = 72 − 2 − 207 + 228.6 − 75.56 = \textbf{16.04 dB}
\item
  Downlink C/N: (C/N)\textsubscript{down} = 44 + 20 − 196 − (−228.6) −
  75.56 = 44 + 20 − 196 + 228.6 − 75.56 = \textbf{21.04 dB}
\item
  Overall C/N (reciprocal addition in linear): (C/N)\textsubscript{up} =
  10\textsuperscript{16.04/10} = 40.18 (C/N)\textsubscript{down} =
  10\textsuperscript{21.04/10} = 127.1 (C/I) = 10\textsuperscript{22/10}
  = 158.5
\end{enumerate}

1/(C/N)\textsubscript{total} = 1/40.18 + 1/127.1 + 1/158.5 = 0.02489 +
0.00787 + 0.00631 = 0.03907 (C/N)\textsubscript{total} = 1/0.03907 =
25.60 = \textbf{14.08 dB}

The uplink (16.04 dB) is the bottleneck, pulling the overall C/N down to
14.08 dB.

\begin{enumerate}
\def\labelenumi{(\alph{enumi})}
\setcounter{enumi}{3}
\tightlist
\item
  Increasing uplink power by 3 dB: (C/N)\textsubscript{up,new} = 16.04 +
  3 = 19.04 dB = 80.22 1/(C/N)\textsubscript{total} = 1/80.22 + 1/127.1
  + 1/158.5 = 0.01247 + 0.00787 + 0.00631 = 0.02665
  (C/N)\textsubscript{total} = 37.52 = \textbf{15.74 dB}
\end{enumerate}

Improvement: 15.74 − 14.08 = \textbf{1.66 dB} --- less than the 3 dB
uplink increase because the downlink and intermodulation now become
comparable contributors.

\begin{center}\rule{0.5\linewidth}{0.5pt}\end{center}

\section{Problem 2.7.10}\label{problem-2.7.10}

\textbf{Given:} An Earth-to-Mars communication link during opposition
(closest approach, distance = 0.52 AU = 7.78 × 10¹⁰ m). The Deep Space
Network (DSN) 70 m antenna operates at 8.4 GHz (X-band) with
G\textsubscript{tx} = 74 dBi and transmit power = 400 kW. The Mars
orbiter has a 1.5 m HGA with G\textsubscript{rx} = 39 dBi and system
noise temperature of 200 K. The data is coded with a rate-1/6 turbo code
requiring E\textsubscript{b}/N₀ = 0.5 dB for BER = 10⁻⁵.

\textbf{Find:} (a) The EIRP, (b) the free-space path loss, (c) the
received power, (d) the noise spectral density, and (e) the maximum
achievable data rate.

\textbf{Solution:}

\begin{enumerate}
\def\labelenumi{(\alph{enumi})}
\item
  EIRP = P\textsubscript{tx}(dBW) + G\textsubscript{tx} = 10
  log₁₀(400,000) + 74 = 56.02 + 74 = \textbf{130.0 dBW}
\item
  FSPL at 8.4 GHz, 7.78 × 10¹⁰ m: λ = 3 × 10⁸ / 8.4 × 10⁹ = 0.03571 m
  FSPL = 20 log₁₀(4π × 7.78 × 10¹⁰ / 0.03571) = 20 log₁₀(2.741 × 10¹³) =
  20 × 13.438 = \textbf{268.8 dB}
\item
  P\textsubscript{rx} = EIRP − FSPL + G\textsubscript{rx} = 130.0 −
  268.8 + 39 = \textbf{−99.8 dBW} (= −69.8 dBm)
\end{enumerate}

Wait --- this is the uplink from Earth to Mars. Actually:
P\textsubscript{rx} = EIRP − FSPL + G\textsubscript{rx} = 130.0 − 268.8
+ 39 = \textbf{−99.8 dBW}

\begin{enumerate}
\def\labelenumi{(\alph{enumi})}
\setcounter{enumi}{3}
\item
  Noise spectral density: N₀ = kT\textsubscript{sys} = −228.6 + 10
  log₁₀(200) = −228.6 + 23.0 = \textbf{−205.6 dBW/Hz}
\item
  Maximum data rate: P\textsubscript{rx}/N₀ = −99.8 − (−205.6) = 105.8
  dB·Hz
\end{enumerate}

For coded system: E\textsubscript{b}/N₀ = 0.5 dB
R\textsubscript{b}(dB·Hz) = P\textsubscript{rx}/N₀ −
E\textsubscript{b}/N₀ = 105.8 − 0.5 = 105.3 dB·Hz R\textsubscript{b} =
10\textsuperscript{105.3/10} = 10\textsuperscript{10.53} = \textbf{3.39
× 10¹⁰ bps}

This seems too high. Let me reconsider --- 400 kW transmit from the DSN
is the uplink (Earth to Mars), and the spacecraft has limited transmit
power. Let me recalculate for the Mars-to-Earth downlink instead:

For the downlink (Mars orbiter transmitter): P\textsubscript{tx} = 25 W
(typical), G\textsubscript{tx} = 39 dBi. EIRP = 10 log₁₀(25) + 39 =
13.98 + 39 = 52.98 dBW DSN 70 m: G\textsubscript{rx} = 74 dBi,
T\textsubscript{sys} = 20 K

P\textsubscript{rx} = 52.98 − 268.8 + 74 = −141.8 dBW N₀ = −228.6 + 10
log₁₀(20) = −228.6 + 13.0 = −215.6 dBW/Hz P\textsubscript{rx}/N₀ =
−141.8 − (−215.6) = 73.8 dB·Hz R\textsubscript{b} =
10\textsuperscript{(73.8−0.5)/10} = 10\textsuperscript{7.33} =
\textbf{21.4 Mbps} (at closest approach)

For the original calculation (uplink from Earth): the 400 kW + 74 dBi
DSN can deliver: R\textsubscript{b} = 10\textsuperscript{10.53} =
\textbf{33.9 Gbps} theoretical maximum at closest approach.

In practice, the uplink capacity is limited by the spacecraft receiver
bandwidth and processing capability, not the RF link. The downlink at
\textbf{21.4 Mbps} is the practical bottleneck and matches NASA's
reported Mars Reconnaissance Orbiter data rates at opposition.

\chapter{Chapter 2 --- Section 2.8: Link Budget
Analysis}\label{chapter-2-section-2.8-link-budget-analysis}

Practice problems covering free-space path loss, link margin, receiver
sensitivity, Wi-Fi range estimation, cellular link budgets, microwave
point-to-point links, Fresnel zone clearance, and fade margin
calculations.

\begin{center}\rule{0.5\linewidth}{0.5pt}\end{center}

\section{Problem 2.8.1}\label{problem-2.8.1}

\textbf{Given:} A point-to-point microwave link operates at 6 GHz over a
distance of 30 km. The transmit power is 1 W (30 dBm), and both antennas
are 1.2 m parabolic dishes with 55\% aperture efficiency. Miscellaneous
losses (waveguide, connectors, radome) total 3 dB at each end. The
receiver requires a minimum input power of −75 dBm.

\textbf{Find:} (a) The antenna gain at each end, (b) the free-space path
loss, (c) the received power, (d) the link margin, and (e) the fade
margin available for rain and multipath fading.

\textbf{Solution:}

\begin{enumerate}
\def\labelenumi{(\alph{enumi})}
\item
  Antenna gain: λ = c/f = 3 × 10⁸ / 6 × 10⁹ = 0.05 m G = η(πD/λ)² = 0.55
  × (π × 1.2/0.05)² = 0.55 × (75.40)² = 0.55 × 5,685 = 3,127 G(dBi) = 10
  log₁₀(3,127) = \textbf{34.95 dBi ≈ 35.0 dBi}
\item
  FSPL at 6 GHz, 30 km: FSPL = 20 log₁₀(4πd/λ) = 20 log₁₀(4π ×
  30,000/0.05) = 20 log₁₀(7.540 × 10⁶) = 20 × 6.877 = \textbf{137.5 dB}
\item
  Received power: P\textsubscript{rx} = P\textsubscript{tx} +
  G\textsubscript{tx} − L\textsubscript{tx,misc} − FSPL +
  G\textsubscript{rx} − L\textsubscript{rx,misc} P\textsubscript{rx} =
  30 + 35.0 − 3 − 137.5 + 35.0 − 3 = \textbf{−43.5 dBm}
\item
  Link margin: Margin = P\textsubscript{rx} −
  P\textsubscript{sensitivity} = −43.5 − (−75) = \textbf{31.5 dB}
\item
  Typical allocations from the 31.5 dB margin:
\end{enumerate}

\begin{itemize}
\tightlist
\item
  Rain attenuation (99.99\% availability at 6 GHz): \textasciitilde5 dB
\item
  Multipath fading (99.99\% availability, Vigants model):
  \textasciitilde20 dB
\item
  Equipment aging margin: \textasciitilde3 dB
\item
  Total allocated: 28 dB
\item
  Remaining unallocated margin: \textbf{3.5 dB}
\end{itemize}

The link is well-designed with adequate margin for carrier-grade
(99.99\%) availability.

\begin{center}\rule{0.5\linewidth}{0.5pt}\end{center}

\section{Problem 2.8.2}\label{problem-2.8.2}

\textbf{Given:} A 5 GHz Wi-Fi 6 access point (802.11ax) operates at
P\textsubscript{tx} = 23 dBm (200 mW, the EIRP regulatory limit) with
G\textsubscript{tx} = 5 dBi. The client device has G\textsubscript{rx} =
0 dBi. The receiver sensitivity at various MCS rates is: MCS 11
(1024-QAM, 143 Mbps) = −55 dBm, MCS 7 (64-QAM, 86 Mbps) = −65 dBm, MCS 3
(16-QAM, 43 Mbps) = −73 dBm, MCS 0 (BPSK, 8.6 Mbps) = −82 dBm.

\textbf{Find:} (a) The FSPL at 5 GHz and 20 m, (b) the received power at
20 m (free space), (c) the maximum data rate achievable at 20 m, (d) the
maximum range for 64-QAM with a 10 dB fade margin (for indoor
multipath), and (e) the maximum range for MCS 0 with a 15 dB wall
penetration loss.

\textbf{Solution:}

\begin{enumerate}
\def\labelenumi{(\alph{enumi})}
\item
  FSPL at 5 GHz, 20 m: λ = 3 × 10⁸ / 5 × 10⁹ = 0.06 m FSPL = 20 log₁₀(4π
  × 20/0.06) = 20 log₁₀(4,189) = 20 × 3.622 = \textbf{72.4 dB}
\item
  P\textsubscript{rx} = 23 + 5 − 72.4 + 0 = \textbf{−44.4 dBm}
\item
  At −44.4 dBm, the signal exceeds the sensitivity for all MCS rates.
  Maximum rate = \textbf{MCS 11, 143 Mbps} (margin = −44.4 − (−55) =
  10.6 dB).
\item
  Maximum range for MCS 7 (−65 dBm) with 10 dB fade margin: Required
  P\textsubscript{rx} = −65 + 10 = −55 dBm (need 10 dB above
  sensitivity) Allowable FSPL = P\textsubscript{tx} +
  G\textsubscript{tx} + G\textsubscript{rx} − P\textsubscript{rx,req} =
  23 + 5 + 0 − (−55) = 83.0 dB FSPL = 20 log₁₀(4πd/λ) = 83.0 4πd/λ =
  10\textsuperscript{83/20} = 10\textsuperscript{4.15} = 14,125 d =
  14,125 × λ / (4π) = 14,125 × 0.06 / 12.566 = \textbf{67.4 m}
\item
  Maximum range for MCS 0 (−82 dBm) with 15 dB wall loss: Allowable FSPL
  = 23 + 5 + 0 − (−82) − 15 = 95.0 dB 4πd/λ = 10\textsuperscript{95/20}
  = 10\textsuperscript{4.75} = 56,234 d = 56,234 × 0.06 / 12.566 =
  \textbf{268 m}
\end{enumerate}

In practice, indoor propagation at 5 GHz experiences 3--8 dB loss per
wall and path loss exponents of 3.0--4.5 (vs.~2.0 for free space),
significantly reducing these theoretical ranges.

\begin{center}\rule{0.5\linewidth}{0.5pt}\end{center}

\section{Problem 2.8.3}\label{problem-2.8.3}

\textbf{Given:} A cellular LTE uplink operates at 700 MHz (Band 12). The
mobile device transmits at P\textsubscript{tx} = 23 dBm with
G\textsubscript{tx} = 0 dBi. The base station has G\textsubscript{rx} =
18 dBi and receiver sensitivity of −103 dBm (for QPSK, 1/3 code rate in
10 MHz). Body loss = 3 dB, building penetration loss = 15 dB. Use the
Hata model for urban propagation: L\textsubscript{path} = 69.55 + 26.16
log₁₀(f\textsubscript{MHz}) − 13.82 log₁₀(h\textsubscript{b}) −
a(h\textsubscript{m}) + (44.9 − 6.55 log₁₀(h\textsubscript{b})) ×
log₁₀(d\textsubscript{km}), with h\textsubscript{b} = 30 m (base station
height), h\textsubscript{m} = 1.5 m (mobile height), and
a(h\textsubscript{m}) = 0 dB for small city correction.

\textbf{Find:} (a) The maximum allowable path loss, (b) the Hata model
path loss coefficients, (c) the maximum range, (d) the link margin at 5
km, and (e) the range improvement by moving to 450 MHz (Band 31).

\textbf{Solution:}

\begin{enumerate}
\def\labelenumi{(\alph{enumi})}
\item
  Maximum allowable path loss (MAPL): MAPL = P\textsubscript{tx} +
  G\textsubscript{tx} + G\textsubscript{rx} − body loss − building loss
  − P\textsubscript{sensitivity} MAPL = 23 + 0 + 18 − 3 − 15 − (−103) =
  \textbf{126 dB}
\item
  Hata model coefficients at 700 MHz, h\textsubscript{b} = 30 m: A =
  69.55 + 26.16 log₁₀(700) − 13.82 log₁₀(30) − 0 A = 69.55 + 26.16 ×
  2.845 − 13.82 × 1.477 = 69.55 + 74.44 − 20.42 = \textbf{123.57 dB} B =
  44.9 − 6.55 log₁₀(30) = 44.9 − 6.55 × 1.477 = 44.9 − 9.67 =
  \textbf{35.23 dB/decade}
\end{enumerate}

L\textsubscript{Hata} = 123.57 + 35.23 × log₁₀(d\textsubscript{km})

\begin{enumerate}
\def\labelenumi{(\alph{enumi})}
\setcounter{enumi}{2}
\item
  Maximum range: Set L\textsubscript{Hata} = MAPL: 123.57 + 35.23 ×
  log₁₀(d) = 126 log₁₀(d) = (126 − 123.57) / 35.23 = 2.43 / 35.23 =
  0.0690 d = 10\textsuperscript{0.0690} = \textbf{1.17 km}
\item
  Link margin at 5 km: L\textsubscript{Hata} = 123.57 + 35.23 × log₁₀(5)
  = 123.57 + 35.23 × 0.699 = 123.57 + 24.63 = 148.2 dB Margin = MAPL −
  L\textsubscript{Hata} = 126 − 148.2 = \textbf{−22.2 dB} (link fails)
\end{enumerate}

The link cannot penetrate buildings at 5 km. Without building
penetration loss: MAPL = 141 dB, giving range =
10\textsuperscript{(141−123.57)/35.23} = 10\textsuperscript{0.495} =
\textbf{3.12 km} (outdoor coverage only).

\begin{enumerate}
\def\labelenumi{(\alph{enumi})}
\setcounter{enumi}{4}
\tightlist
\item
  At 450 MHz: A\textsubscript{450} = 69.55 + 26.16 × log₁₀(450) − 13.82
  × 1.477 = 69.55 + 69.44 − 20.42 = 118.57 dB Range at 450 MHz: log₁₀(d)
  = (126 − 118.57) / 35.23 = 7.43 / 35.23 = 0.211 d =
  10\textsuperscript{0.211} = \textbf{1.63 km}
\end{enumerate}

Improvement: 1.63 / 1.17 = \textbf{1.39× (39\% greater range)}. Lower
frequencies provide better building penetration and reduced path loss.

\begin{center}\rule{0.5\linewidth}{0.5pt}\end{center}

\section{Problem 2.8.4}\label{problem-2.8.4}

\textbf{Given:} A 60 GHz wireless backhaul link (IEEE 802.11ad/ay)
operates over 500 m between two buildings. P\textsubscript{tx} = 10 dBm,
G\textsubscript{tx} = G\textsubscript{rx} = 38 dBi (narrow-beam horn
antennas). Oxygen absorption at 60 GHz adds 15 dB/km. Rain attenuation
at 60 GHz during heavy rain (50 mm/hr) is approximately 20 dB/km. The
receiver noise figure is 8 dB and the channel bandwidth is 2.16 GHz.

\textbf{Find:} (a) The FSPL, (b) the total atmospheric loss (clear air),
(c) the received power (clear air), (d) the noise floor, (e) the SNR and
achievable data rate in clear air and during heavy rain.

\textbf{Solution:}

\begin{enumerate}
\def\labelenumi{(\alph{enumi})}
\item
  FSPL at 60 GHz, 500 m: λ = 3 × 10⁸ / 60 × 10⁹ = 0.005 m FSPL = 20
  log₁₀(4π × 500/0.005) = 20 log₁₀(1.257 × 10⁶) = 20 × 6.099 =
  \textbf{122.0 dB}
\item
  Oxygen absorption: L\textsubscript{O₂} = 15 × 0.5 = \textbf{7.5 dB}
\item
  Clear-air received power: P\textsubscript{rx} = 10 + 38 − 122.0 + 38 −
  7.5 = \textbf{−43.5 dBm}
\item
  Noise floor: N = −174 + 10 log₁₀(2.16 × 10⁹) + 8 = −174 + 93.35 + 8 =
  \textbf{−72.65 dBm}
\item
  Clear-air SNR: SNR = −43.5 − (−72.65) = \textbf{29.2 dB}
\end{enumerate}

At 29.2 dB SNR with 2.16 GHz bandwidth: Spectral efficiency ≈ log₂(1 +
10\textsuperscript{29.2/10}) = log₂(1 + 832) = log₂(833) = 9.70
bits/s/Hz Maximum rate = 9.70 × 2.16 = \textbf{20.95 Gbps} (theoretical)
Practical (802.11ay, 64-QAM OFDM): approximately \textbf{8--10 Gbps}

During heavy rain: Rain attenuation = 20 × 0.5 = 10 dB
P\textsubscript{rx,rain} = −43.5 − 10 = −53.5 dBm
SNR\textsubscript{rain} = −53.5 − (−72.65) = \textbf{19.2 dB} Rate drops
to approximately 64-QAM level: \textbf{4--5 Gbps}

Heavy rain reduces throughput by roughly 50\% but the link remains
operational.

\begin{center}\rule{0.5\linewidth}{0.5pt}\end{center}

\section{Problem 2.8.5}\label{problem-2.8.5}

\textbf{Given:} A Fresnel zone clearance calculation for a 15 GHz
microwave link spanning 10 km between two towers. Tower heights are
equal. The midpoint terrain elevation is 5 m above the direct line of
sight.

\textbf{Find:} (a) The first Fresnel zone radius at the midpoint, (b)
the required clearance (60\% of the first Fresnel zone), (c) the minimum
tower height above terrain for adequate clearance, (d) the Fresnel zone
radius at a point 3 km from one end, and (e) the effect of Earth
curvature over the 10 km path (assuming K = 4/3 effective Earth radius
factor).

\textbf{Solution:}

\begin{enumerate}
\def\labelenumi{(\alph{enumi})}
\item
  First Fresnel zone radius at midpoint: r₁ = √(nλd₁d₂ / (d₁ + d₂)) At
  midpoint: d₁ = d₂ = 5,000 m, n = 1 λ = c/f = 3 × 10⁸ / 15 × 10⁹ = 0.02
  m r₁ = √(1 × 0.02 × 5,000 × 5,000 / 10,000) = √(50) = \textbf{7.07 m}
\item
  Required clearance (60\% rule): 0.6 × r₁ = 0.6 × 7.07 = \textbf{4.24
  m}
\item
  The terrain at midpoint is 5 m above the line of sight. Required
  additional height to achieve 4.24 m clearance above the obstruction:
  h\textsubscript{clearance} = 5 + 4.24 = \textbf{9.24 m} above the
  direct LOS
\end{enumerate}

Since both towers are equal height and the obstruction is at midpoint,
each tower must be raised by 9.24 m above the current LOS. Total tower
height = terrain base height + 9.24 m above the LOS datum.

If the current tower heights place the LOS at terrain level at midpoint:
each tower needs to be at least \textbf{9.24 m above the surrounding
terrain}.

\begin{enumerate}
\def\labelenumi{(\alph{enumi})}
\setcounter{enumi}{3}
\tightlist
\item
  At 3 km from one end (d₁ = 3 km, d₂ = 7 km): r₁ = √(0.02 × 3,000 ×
  7,000 / 10,000) = √(42) = \textbf{6.48 m}
\end{enumerate}

The Fresnel zone is slightly smaller than at midpoint, as expected for
an asymmetric position.

\begin{enumerate}
\def\labelenumi{(\alph{enumi})}
\setcounter{enumi}{4}
\tightlist
\item
  Earth curvature bulge at midpoint: h\textsubscript{bulge} = d₁ × d₂ /
  (2 × K × R\textsubscript{e}) R\textsubscript{e} = 6,371 km, K = 4/3
  (standard atmosphere) h\textsubscript{bulge} = 5,000 × 5,000 / (2 ×
  4/3 × 6,371,000) = 25 × 10⁶ / 16,989,333 = \textbf{1.47 m}
\end{enumerate}

The Earth's curvature adds 1.47 m of effective obstruction at midpoint.
This must be added to the clearance requirement: Total required tower
height adjustment = 5 + 4.24 + 1.47 = \textbf{10.71 m} above the LOS
baseline.

\begin{center}\rule{0.5\linewidth}{0.5pt}\end{center}

\section{Problem 2.8.6}\label{problem-2.8.6}

\textbf{Given:} A LoRa (Long Range) IoT sensor link operates at 915 MHz
with the following parameters: P\textsubscript{tx} = 20 dBm,
G\textsubscript{tx} = 2 dBi (omnidirectional), G\textsubscript{rx} = 6
dBi (gateway antenna). LoRa spreading factor SF = 10, bandwidth = 125
kHz, coding rate 4/5. The receiver sensitivity at SF10 is −134 dBm. The
environment is suburban with log-distance path loss model: L = L₀ + 10n
× log₁₀(d/d₀), where L₀ = 31.5 dB at d₀ = 1 m and n = 3.3.

\textbf{Find:} (a) The LoRa data rate at SF10, (b) the maximum allowable
path loss, (c) the maximum range in suburban conditions, (d) the link
margin at 5 km, and (e) the range if SF is increased to 12 (sensitivity
improves to −140 dBm).

\textbf{Solution:}

\begin{enumerate}
\def\labelenumi{(\alph{enumi})}
\item
  LoRa data rate: R\textsubscript{b} = SF × BW × CR /
  2\textsuperscript{SF} R\textsubscript{b} = 10 × 125,000 × (4/5) / 2¹⁰
  = 10 × 125,000 × 0.8 / 1,024 R\textsubscript{b} = 1,000,000 / 1,024 =
  \textbf{976.6 bps ≈ 977 bps}
\item
  MAPL = P\textsubscript{tx} + G\textsubscript{tx} + G\textsubscript{rx}
  − P\textsubscript{sensitivity} MAPL = 20 + 2 + 6 − (−134) =
  \textbf{162 dB}
\item
  Maximum range: L = L₀ + 10n × log₁₀(d) 162 = 31.5 + 10 × 3.3 ×
  log₁₀(d) 33 × log₁₀(d) = 130.5 log₁₀(d) = 3.955 d =
  10\textsuperscript{3.955} = \textbf{9,016 m ≈ 9.0 km}
\item
  Path loss at 5 km: L = 31.5 + 33 × log₁₀(5,000) = 31.5 + 33 × 3.699 =
  31.5 + 122.1 = 153.6 dB Margin = 162 − 153.6 = \textbf{8.4 dB}
\item
  At SF12 (sensitivity = −140 dBm): MAPL = 20 + 2 + 6 − (−140) = 168 dB
  log₁₀(d) = (168 − 31.5) / 33 = 136.5 / 33 = 4.136 d =
  10\textsuperscript{4.136} = \textbf{13,690 m ≈ 13.7 km}
\end{enumerate}

Data rate at SF12: R\textsubscript{b} = 12 × 125,000 × 0.8 / 2¹² =
1,200,000 / 4,096 = \textbf{293 bps}

The 6 dB sensitivity improvement extends range by 13.7/9.0 = 1.52×
(52\%), but the data rate drops from 977 to 293 bps (3.3× slower). This
is the fundamental LoRa trade-off: spreading factor increases range at
the cost of data rate.

\begin{center}\rule{0.5\linewidth}{0.5pt}\end{center}

\section{Problem 2.8.7}\label{problem-2.8.7}

\textbf{Given:} A fiber optic link uses single-mode fiber (SMF) at 1,310
nm over 40 km. The transmitter output is 0 dBm (1 mW) from an SFP+
module. The fiber attenuation is 0.35 dB/km at 1,310 nm. There are 4
fusion splices (0.1 dB each) and 2 connectors (0.3 dB each). The
receiver sensitivity is −24 dBm for 10 Gbps with BER = 10⁻¹².

\textbf{Find:} (a) The total fiber loss, (b) the total connector and
splice loss, (c) the total link loss, (d) the power margin, and (e) the
maximum achievable distance with 3 dB margin.

\textbf{Solution:}

\begin{enumerate}
\def\labelenumi{(\alph{enumi})}
\item
  Fiber loss: L\textsubscript{fiber} = 0.35 × 40 = \textbf{14.0 dB}
\item
  Connector and splice loss: L\textsubscript{splice} = 4 × 0.1 = 0.4 dB
  L\textsubscript{connector} = 2 × 0.3 = 0.6 dB Total: \textbf{1.0 dB}
\item
  Total link loss: L\textsubscript{total} = 14.0 + 1.0 = \textbf{15.0
  dB}
\item
  Received power: P\textsubscript{rx} = 0 − 15.0 = −15.0 dBm Power
  margin: M = P\textsubscript{rx} − P\textsubscript{sensitivity} = −15.0
  − (−24) = \textbf{9.0 dB}
\item
  Maximum distance with 3 dB margin: Available loss budget =
  P\textsubscript{tx} − P\textsubscript{sensitivity} − margin −
  L\textsubscript{conn/splice} = 0 − (−24) − 3 − 1.0 = 20.0 dB for fiber
  Maximum distance = 20.0 / 0.35 = \textbf{57.1 km}
\end{enumerate}

At 1,550 nm with 0.20 dB/km attenuation: d\textsubscript{max} =
20.0/0.20 = 100 km. The lower attenuation at 1,550 nm nearly doubles the
reach, which is why long-haul systems use the C-band (1,550 nm).

\begin{center}\rule{0.5\linewidth}{0.5pt}\end{center}

\section{Problem 2.8.8}\label{problem-2.8.8}

\textbf{Given:} A Bluetooth Low Energy (BLE) link at 2.4 GHz is used for
an indoor asset tracking system. P\textsubscript{tx} = 0 dBm,
G\textsubscript{tx} = G\textsubscript{rx} = 0 dBi. The receiver
sensitivity is −97 dBm at 1 Mbps data rate (BLE 1M PHY). The indoor
propagation model uses a log-distance model with n = 2.8 and reference
loss L₀ = 40 dB at 1 m. Wall penetration loss is 5 dB per wall.

\textbf{Find:} (a) The MAPL, (b) the maximum range in open indoor space
(no walls), (c) the range through 2 interior walls, (d) the received
signal strength at 10 m (for RSSI-based ranging accuracy estimation),
and (e) the ranging error if RSSI varies by ±6 dB due to multipath.

\textbf{Solution:}

\begin{enumerate}
\def\labelenumi{(\alph{enumi})}
\item
  MAPL = 0 + 0 + 0 − (−97) = \textbf{97 dB}
\item
  Maximum range (no walls): 97 = 40 + 10 × 2.8 × log₁₀(d) 28 × log₁₀(d)
  = 57 log₁₀(d) = 2.036 d = 10\textsuperscript{2.036} = \textbf{108.6 m}
\item
  Through 2 walls: available loss = 97 − 2 × 5 = 87 dB 28 × log₁₀(d) =
  87 − 40 = 47 log₁₀(d) = 1.679 d = 10\textsuperscript{1.679} =
  \textbf{47.7 m}
\item
  RSSI at 10 m: L = 40 + 28 × log₁₀(10) = 40 + 28 = 68 dB
  P\textsubscript{rx} = 0 − 68 = \textbf{−68 dBm}
\item
  RSSI-based distance estimation at −68 dBm: d\textsubscript{est} =
  10\textsuperscript{(Prx − P₀ + L₀)/(10n)}
\end{enumerate}

With ±6 dB variation: At −62 dBm: L = 62 → d =
10\textsuperscript{(62−40)/28} = 10\textsuperscript{0.786} = 6.11 m At
−74 dBm: L = 74 → d = 10\textsuperscript{(74−40)/28} =
10\textsuperscript{1.214} = 16.37 m

True distance = 10 m, but RSSI estimates range from 6.1 to 16.4 m.
Ranging error: \textbf{−3.9 m to +6.4 m} (−39\% to +64\%)

This illustrates why RSSI-based indoor positioning is limited to
room-level accuracy (3--5 m). BLE 5.1 direction-finding (Angle of
Arrival) achieves sub-meter accuracy by using phase information instead
of amplitude.

\begin{center}\rule{0.5\linewidth}{0.5pt}\end{center}

\section{Problem 2.8.9}\label{problem-2.8.9}

\textbf{Given:} A 5G NR mmWave outdoor urban small cell operates at 39
GHz. The base station transmits at 36 dBm using a 256-element phased
array with 30 dBi beamforming gain. The UE (user equipment) has
G\textsubscript{rx} = 5 dBi (small phone array). NF\textsubscript{rx} =
7 dB, channel bandwidth = 100 MHz. The propagation model is the 3GPP UMi
street canyon model: L = 32.4 + 21.0 log₁₀(d\textsubscript{3D}) + 20
log₁₀(f\textsubscript{GHz}). Minimum required SNR is 5 dB for QPSK 1/2.

\textbf{Find:} (a) The noise floor, (b) the minimum required received
power, (c) the allowable path loss, (d) the maximum LOS range, and (e)
the link margin at 100 m.

\textbf{Solution:}

\begin{enumerate}
\def\labelenumi{(\alph{enumi})}
\item
  Noise floor: N = −174 + 10 log₁₀(100 × 10⁶) + 7 = −174 + 80 + 7 =
  \textbf{−87 dBm}
\item
  Minimum received power: P\textsubscript{rx,min} = N +
  SNR\textsubscript{req} = −87 + 5 = \textbf{−82 dBm}
\item
  Allowable path loss: MAPL = P\textsubscript{tx} + G\textsubscript{tx}
  + G\textsubscript{rx} − P\textsubscript{rx,min} = 36 + 30 + 5 − (−82)
  = \textbf{153 dB}
\item
  Maximum LOS range: L = 32.4 + 21.0 log₁₀(d) + 20 log₁₀(39) = 32.4 +
  21.0 log₁₀(d) + 31.82 L = 64.22 + 21.0 log₁₀(d)
\end{enumerate}

Setting L = MAPL: 153 = 64.22 + 21.0 log₁₀(d) log₁₀(d) = 88.78 / 21.0 =
4.228 d = 10\textsuperscript{4.228} = \textbf{16,896 m ≈ 16.9 km}
(theoretical LOS)

This seems very large --- this is because the 30 dBi array gain is
substantial. However, mmWave is limited by blockage and NLOS conditions,
not LOS range.

\begin{enumerate}
\def\labelenumi{(\alph{enumi})}
\setcounter{enumi}{4}
\tightlist
\item
  Link margin at 100 m: L = 64.22 + 21.0 × log₁₀(100) = 64.22 + 42.0 =
  106.22 dB P\textsubscript{rx} = 36 + 30 + 5 − 106.22 = \textbf{−35.22
  dBm} SNR = −35.22 − (−87) = 51.78 dB Margin over QPSK 1/2 requirement:
  51.78 − 5 = \textbf{46.8 dB}
\end{enumerate}

At 100 m with 51.8 dB SNR, the system can use 256-QAM (\textasciitilde27
dB SNR) with massive margin, achieving peak rates of \textasciitilde2
Gbps in 100 MHz. The real challenge at mmWave is blockage: a human body
attenuates 20--35 dB, and building materials cause 40+ dB loss, which is
why 5G mmWave requires dense small cell deployment.

\begin{center}\rule{0.5\linewidth}{0.5pt}\end{center}

\section{Problem 2.8.10}\label{problem-2.8.10}

\textbf{Given:} A maritime VHF radio link operates at 156 MHz (Channel
16, distress frequency). The ship transmitter has P\textsubscript{tx} =
25 W (43.98 dBm) and the antenna height is 5 m above sea level with
G\textsubscript{tx} = 3 dBi. The coast station antenna is at 50 m height
with G\textsubscript{rx} = 6 dBi. Receiver sensitivity is −107 dBm. The
radio horizon distance is given by d(km) ≈ 4.12 × (√h₁ + √h₂) where h is
in meters.

\textbf{Find:} (a) The radio horizon distance, (b) the FSPL at the
horizon distance, (c) the received power at the horizon, (d) the link
margin at the horizon, and (e) the maximum range for reliable
communication (accounting for a 10 dB fade margin for sea-state fading).

\textbf{Solution:}

\begin{enumerate}
\def\labelenumi{(\alph{enumi})}
\item
  Radio horizon: d = 4.12 × (√5 + √50) = 4.12 × (2.236 + 7.071) = 4.12 ×
  9.307 = \textbf{38.3 km}
\item
  FSPL at 156 MHz, 38.3 km: λ = 3 × 10⁸ / 156 × 10⁶ = 1.923 m FSPL = 20
  log₁₀(4π × 38,300 / 1.923) = 20 log₁₀(249,870) = 20 × 5.398 =
  \textbf{107.96 dB}
\item
  Received power at horizon: P\textsubscript{rx} = 43.98 + 3 − 107.96 +
  6 = \textbf{−54.98 dBm}
\item
  Link margin: M = −54.98 − (−107) = \textbf{52.0 dB}
\item
  With 10 dB fade margin, maximum allowable path loss: MAPL = 43.98 + 3
  + 6 − (−107) − 10 = 149.98 dB
\end{enumerate}

FSPL = 149.98: 20 log₁₀(4πd/1.923) = 149.98 4πd/1.923 =
10\textsuperscript{149.98/20} = 10\textsuperscript{7.499} = 3.157 × 10⁷
d = 3.157 × 10⁷ × 1.923 / (4π) = 4.833 × 10⁶ m = \textbf{4,833 km}

This far exceeds the radio horizon (38.3 km), so the limiting factor is
\textbf{not RF power but the line of sight}. The maximum reliable range
is the radio horizon: \textbf{38.3 km} --- beyond this, the signal is
blocked by Earth's curvature.

This is why maritime VHF is considered a short-range service (typically
20--40 nautical miles = 37--74 km depending on antenna heights), and
long-range maritime communication uses HF (sky-wave) or satellite
systems.

\chapter{Chapter 3 --- Section 3.1: Semiconductor
Fundamentals}\label{chapter-3-section-3.1-semiconductor-fundamentals}

Practice problems covering energy bands, doping, carrier concentrations,
drift, and diffusion in semiconductor materials.

\begin{center}\rule{0.5\linewidth}{0.5pt}\end{center}

\section{Problem 3.1.1}\label{problem-3.1.1}

\textbf{Given:} A germanium sample (E\textsubscript{g} = 0.66 eV) has an
intrinsic carrier concentration n\textsubscript{i} = 2.4 x
10\textsuperscript{13} cm\textsuperscript{-3} at 300 K. The material
constant B is the same for both temperatures in the approximation
n\textsubscript{i} \textasciitilde{} B x T\textsuperscript{3/2} x
e\textsuperscript{-Eg/(2kT)}.

\textbf{Find:} The intrinsic carrier concentration at 350 K. Use k =
8.617 x 10\textsuperscript{-5} eV/K.

\textbf{Solution:} Taking the ratio at two temperatures (B cancels):

n\textsubscript{i}(T₂)/n\textsubscript{i}(T₁) =
(T₂/T₁)\textsuperscript{3/2} x e\textsuperscript{-Eg/(2k) x (1/T₂ -
1/T₁)}

Temperature ratio: (350/300)\textsuperscript{3/2} =
(1.1667)\textsuperscript{3/2} = 1.260

Exponent: -E\textsubscript{g}/(2k) x (1/T₂ - 1/T₁) = -0.66/(2 x 8.617 x
10\textsuperscript{-5}) x (1/350 - 1/300) = -3,830 x (0.002857 -
0.003333) = -3,830 x (-4.762 x 10\textsuperscript{-4}) = 1.824

e\textsuperscript{1.824} = 6.197

n\textsubscript{i}(350 K) = 2.4 x 10\textsuperscript{13} x 1.260 x 6.197
= \textbf{1.87 x 10\textsuperscript{14} cm\textsuperscript{-3}}

The carrier concentration increases by a factor of \textasciitilde7.8
for a 50 K rise, illustrating the strong temperature dependence of
germanium (which has a smaller bandgap than silicon).

\begin{center}\rule{0.5\linewidth}{0.5pt}\end{center}

\section{Problem 3.1.2}\label{problem-3.1.2}

\textbf{Given:} A silicon sample is doped with boron (acceptor) at a
concentration of N\textsubscript{A} = 8 x 10\textsuperscript{15}
cm\textsuperscript{-3}. The hole mobility is mu\textsubscript{p} = 450
cm\textsuperscript{2}/(V*s). Use n\textsubscript{i} = 1.5 x
10\textsuperscript{10} cm\textsuperscript{-3} and q = 1.602 x
10\textsuperscript{-19} C.

\textbf{Find:} (a) The majority and minority carrier concentrations, (b)
the conductivity, and (c) the resistivity.

\textbf{Solution:}

\begin{enumerate}
\def\labelenumi{(\alph{enumi})}
\tightlist
\item
  This is P-type silicon (boron is a Group III acceptor).
\end{enumerate}

Majority carriers (holes): p \textasciitilde{} N\textsubscript{A} =
\textbf{8 x 10\textsuperscript{15} cm\textsuperscript{-3}}

Minority carriers (electrons): n = n\textsubscript{i}\textsuperscript{2}
/ p = (1.5 x 10\textsuperscript{10})\textsuperscript{2} / (8 x
10\textsuperscript{15}) = 2.25 x 10\textsuperscript{20} / 8 x
10\textsuperscript{15} = \textbf{2.81 x 10\textsuperscript{4}
cm\textsuperscript{-3}}

\begin{enumerate}
\def\labelenumi{(\alph{enumi})}
\setcounter{enumi}{1}
\item
  Conductivity (dominated by majority carriers): sigma = q x p x
  mu\textsubscript{p} = 1.602 x 10\textsuperscript{-19} x 8 x
  10\textsuperscript{15} x 450 = **0.577 (ohm*cm)\textsuperscript{-1}**
\item
  Resistivity: rho = 1/sigma = 1/0.577 = **1.73 ohm*cm**
\end{enumerate}

\begin{center}\rule{0.5\linewidth}{0.5pt}\end{center}

\section{Problem 3.1.3}\label{problem-3.1.3}

\textbf{Given:} A silicon wafer is doped with both phosphorus
(N\textsubscript{D} = 3 x 10\textsuperscript{16} cm\textsuperscript{-3})
and boron (N\textsubscript{A} = 5 x 10\textsuperscript{15}
cm\textsuperscript{-3}) --- a compensated semiconductor. Use
n\textsubscript{i} = 1.5 x 10\textsuperscript{10}
cm\textsuperscript{-3}, mu\textsubscript{n} = 1,200
cm\textsuperscript{2}/(V*s), and q = 1.602 x 10\textsuperscript{-19} C.

\textbf{Find:} (a) The net doping type, (b) the majority and minority
carrier concentrations, and (c) the resistivity.

\textbf{Solution:}

\begin{enumerate}
\def\labelenumi{(\alph{enumi})}
\item
  Since N\textsubscript{D} \textgreater{} N\textsubscript{A}, the net
  doping is \textbf{N-type}.
\item
  Net donor concentration: N\textsubscript{D} - N\textsubscript{A} = 3 x
  10\textsuperscript{16} - 5 x 10\textsuperscript{15} = 2.5 x
  10\textsuperscript{16} cm\textsuperscript{-3}
\end{enumerate}

Majority carriers (electrons): n \textasciitilde{} N\textsubscript{D} -
N\textsubscript{A} = \textbf{2.5 x 10\textsuperscript{16}
cm\textsuperscript{-3}}

Minority carriers (holes): p = n\textsubscript{i}\textsuperscript{2} / n
= (1.5 x 10\textsuperscript{10})\textsuperscript{2} / (2.5 x
10\textsuperscript{16}) = 2.25 x 10\textsuperscript{20} / 2.5 x
10\textsuperscript{16} = \textbf{9.0 x 10\textsuperscript{3}
cm\textsuperscript{-3}}

\begin{enumerate}
\def\labelenumi{(\alph{enumi})}
\setcounter{enumi}{2}
\tightlist
\item
  sigma = q x n x mu\textsubscript{n} = 1.602 x 10\textsuperscript{-19}
  x 2.5 x 10\textsuperscript{16} x 1,200 = 4.806
  (ohm*cm)\textsuperscript{-1}
\end{enumerate}

rho = 1/sigma = 1/4.806 = **0.208 ohm*cm**

\begin{center}\rule{0.5\linewidth}{0.5pt}\end{center}

\section{Problem 3.1.4}\label{problem-3.1.4}

\textbf{Given:} An N-type silicon bar (N\textsubscript{D} = 5 x
10\textsuperscript{16} cm\textsuperscript{-3}) is 1 cm long with a
cross-sectional area of 0.01 cm\textsuperscript{2}. A voltage of 10 V is
applied across its length. The electron mobility is mu\textsubscript{n}
= 1,100 cm\textsuperscript{2}/(V*s). Use q = 1.602 x
10\textsuperscript{-19} C.

\textbf{Find:} (a) The electric field, (b) the electron drift velocity,
(c) the current density, (d) the total current, and (e) the resistance
of the bar.

\textbf{Solution:}

\begin{enumerate}
\def\labelenumi{(\alph{enumi})}
\item
  Electric field: E = V/L = 10/1 = \textbf{10 V/cm}
\item
  Drift velocity: v\textsubscript{d} = mu\textsubscript{n} x E = 1,100 x
  10 = \textbf{11,000 cm/s = 110 m/s}
\item
  Current density: J = q x n x mu\textsubscript{n} x E = 1.602 x
  10\textsuperscript{-19} x 5 x 10\textsuperscript{16} x 1,100 x 10 =
  1.602 x 10\textsuperscript{-19} x 5.5 x 10\textsuperscript{20} =
  \textbf{88.1 A/cm\textsuperscript{2}}
\item
  Total current: I = J x A = 88.1 x 0.01 = \textbf{0.881 A}
\item
  Resistance: R = V/I = 10/0.881 = \textbf{11.35 ohm}
\end{enumerate}

Alternatively: rho = 1/(q x n x mu\textsubscript{n}) = 1/(1.602 x
10\textsuperscript{-19} x 5 x 10\textsuperscript{16} x 1,100) = 0.1135
ohm*cm R = rho x L/A = 0.1135 x 1/0.01 = 11.35 ohm (confirms).

\begin{center}\rule{0.5\linewidth}{0.5pt}\end{center}

\section{Problem 3.1.5}\label{problem-3.1.5}

\textbf{Given:} In a P-type silicon region, the hole concentration
decreases linearly from p₁ = 10\textsuperscript{17}
cm\textsuperscript{-3} to p₂ = 10\textsuperscript{15}
cm\textsuperscript{-3} over a distance of 2 um. The hole mobility is
mu\textsubscript{p} = 400 cm\textsuperscript{2}/(V*s) and T = 300 K. Use
q = 1.602 x 10\textsuperscript{-19} C and V\textsubscript{T} = kT/q =
0.02585 V.

\textbf{Find:} (a) The hole diffusion coefficient, (b) the concentration
gradient, and (c) the hole diffusion current density.

\textbf{Solution:}

\begin{enumerate}
\def\labelenumi{(\alph{enumi})}
\item
  Diffusion coefficient (Einstein relation): D\textsubscript{p} =
  V\textsubscript{T} x mu\textsubscript{p} = 0.02585 x 400 =
  \textbf{10.34 cm\textsuperscript{2}/s}
\item
  Concentration gradient: dp/dx = (p₂ - p₁) / delta\_x =
  (10\textsuperscript{15} - 10\textsuperscript{17}) / (2 x
  10\textsuperscript{-4}) = -9.9 x 10\textsuperscript{16} / (2 x
  10\textsuperscript{-4}) = \textbf{-4.95 x 10\textsuperscript{20}
  cm\textsuperscript{-4}}
\item
  Hole diffusion current density: J\textsubscript{diff,p} = -q x
  D\textsubscript{p} x (dp/dx) = -1.602 x 10\textsuperscript{-19} x
  10.34 x (-4.95 x 10\textsuperscript{20}) = \textbf{819.7
  A/cm\textsuperscript{2}}
\end{enumerate}

The positive value indicates current flows in the direction of
decreasing hole concentration (from high to low), which is physically
correct for hole diffusion.

\begin{center}\rule{0.5\linewidth}{0.5pt}\end{center}

\section{Problem 3.1.6}\label{problem-3.1.6}

\textbf{Given:} An N-type silicon sample (N\textsubscript{D} =
10\textsuperscript{16} cm\textsuperscript{-3}) is illuminated,
generating excess carriers at a uniform rate of G =
10\textsuperscript{20} cm\textsuperscript{-3}/s. The minority carrier
(hole) lifetime is tau\textsubscript{p} = 10 us.

\textbf{Find:} (a) The steady-state excess minority carrier
concentration, (b) the total hole concentration under illumination, and
(c) the percentage change in majority carrier concentration. Use
n\textsubscript{i} = 1.5 x 10\textsuperscript{10}
cm\textsuperscript{-3}.

\textbf{Solution:}

\begin{enumerate}
\def\labelenumi{(\alph{enumi})}
\item
  Steady-state excess carrier concentration: delta\_p = G x
  tau\textsubscript{p} = 10\textsuperscript{20} x 10 x
  10\textsuperscript{-6} = \textbf{10\textsuperscript{15}
  cm\textsuperscript{-3}}
\item
  Equilibrium hole concentration: p₀ =
  n\textsubscript{i}\textsuperscript{2}/N\textsubscript{D} = (1.5 x
  10\textsuperscript{10})\textsuperscript{2} / 10\textsuperscript{16} =
  2.25 x 10\textsuperscript{4} cm\textsuperscript{-3}
\end{enumerate}

Total hole concentration: p = p₀ + delta\_p = 2.25 x
10\textsuperscript{4} + 10\textsuperscript{15} \textasciitilde{}
\textbf{10\textsuperscript{15} cm\textsuperscript{-3}} (delta\_p
dominates over p₀ by 11 orders of magnitude)

\begin{enumerate}
\def\labelenumi{(\alph{enumi})}
\setcounter{enumi}{2}
\tightlist
\item
  Majority carrier concentration also increases by delta\_p to maintain
  charge neutrality: n = N\textsubscript{D} + delta\_p =
  10\textsuperscript{16} + 10\textsuperscript{15} = 1.1 x
  10\textsuperscript{16} cm\textsuperscript{-3}
\end{enumerate}

Percentage change = delta\_p/N\textsubscript{D} x 100 =
10\textsuperscript{15}/10\textsuperscript{16} x 100 = \textbf{10\%}

While the minority carrier concentration changed by 11 orders of
magnitude, the majority carrier concentration changed by only 10\%.

\chapter{Chapter 3 --- Section 3.2: Semiconductor
Materials}\label{chapter-3-section-3.2-semiconductor-materials}

Practice problems covering silicon, gallium arsenide, and gallium
nitride material properties and device comparisons.

\begin{center}\rule{0.5\linewidth}{0.5pt}\end{center}

\section{Problem 3.2.1}\label{problem-3.2.1}

\textbf{Given:} Intrinsic silicon at T = 300 K has n\textsubscript{i} =
1.5 x 10\textsuperscript{10} cm\textsuperscript{-3}. A silicon sample is
doped with arsenic (donor) at N\textsubscript{D} = 2 x
10\textsuperscript{17} cm\textsuperscript{-3}. Use kT = 0.02585 eV at
300 K.

\textbf{Find:} (a) The electron and hole concentrations, (b) the Fermi
level position relative to the intrinsic Fermi level, and (c) the Fermi
level position relative to the conduction band edge if
E\textsubscript{g} = 1.12 eV.

\textbf{Solution:}

\begin{enumerate}
\def\labelenumi{(\alph{enumi})}
\tightlist
\item
  Electron concentration: n \textasciitilde{} N\textsubscript{D} =
  \textbf{2 x 10\textsuperscript{17} cm\textsuperscript{-3}}
\end{enumerate}

Hole concentration: p = n\textsubscript{i}\textsuperscript{2}/n = (1.5 x
10\textsuperscript{10})\textsuperscript{2} / (2 x
10\textsuperscript{17}) = 2.25 x 10\textsuperscript{20} / 2 x
10\textsuperscript{17} = \textbf{1.125 x 10\textsuperscript{3}
cm\textsuperscript{-3}}

\begin{enumerate}
\def\labelenumi{(\alph{enumi})}
\setcounter{enumi}{1}
\item
  Fermi level position: E\textsubscript{F} - E\textsubscript{i} = kT x
  ln(n/n\textsubscript{i}) = 0.02585 x ln(2 x 10\textsuperscript{17} /
  1.5 x 10\textsuperscript{10}) = 0.02585 x ln(1.333 x
  10\textsuperscript{7}) = 0.02585 x 16.41 = \textbf{0.424 eV above
  E\textsubscript{i}}
\item
  The intrinsic Fermi level is approximately at mid-gap:
  E\textsubscript{i} \textasciitilde{} E\textsubscript{C} -
  E\textsubscript{g}/2 = E\textsubscript{C} - 0.56 eV.
\end{enumerate}

E\textsubscript{C} - E\textsubscript{F} = E\textsubscript{g}/2 -
(E\textsubscript{F} - E\textsubscript{i}) = 0.56 - 0.424 = \textbf{0.136
eV below E\textsubscript{C}}

The Fermi level is close to the conduction band, confirming strong
N-type behavior.

\begin{center}\rule{0.5\linewidth}{0.5pt}\end{center}

\section{Problem 3.2.2}\label{problem-3.2.2}

\textbf{Given:} A GaAs laser diode emits light at a wavelength
corresponding to its bandgap of E\textsubscript{g} = 1.42 eV. A silicon
photodetector has a bandgap of E\textsubscript{g} = 1.12 eV. Use h =
6.626 x 10\textsuperscript{-34} J*s, c = 3 x 10\textsuperscript{8} m/s,
and 1 eV = 1.602 x 10\textsuperscript{-19} J.

\textbf{Find:} (a) The emission wavelength of the GaAs laser, (b)
whether the silicon detector can absorb this wavelength, and (c) the
maximum (cutoff) wavelength the silicon detector can absorb.

\textbf{Solution:}

\begin{enumerate}
\def\labelenumi{(\alph{enumi})}
\item
  Emission wavelength: lambda = hc/E\textsubscript{g} = (6.626 x
  10\textsuperscript{-34} x 3 x 10\textsuperscript{8}) / (1.42 x 1.602 x
  10\textsuperscript{-19}) = 1.988 x 10\textsuperscript{-25} / 2.275 x
  10\textsuperscript{-19} = 8.74 x 10\textsuperscript{-7} m =
  \textbf{874 nm} (near infrared)
\item
  For the silicon detector to absorb the photon, the photon energy must
  exceed silicon's bandgap: Photon energy = 1.42 eV \textgreater{}
  E\textsubscript{g,Si} = 1.12 eV. \textbf{Yes}, silicon can absorb the
  874 nm GaAs emission.
\item
  Silicon cutoff wavelength: lambda\textsubscript{cutoff} =
  hc/E\textsubscript{g,Si} = 1.988 x 10\textsuperscript{-25} / (1.12 x
  1.602 x 10\textsuperscript{-19}) = 1.988 x 10\textsuperscript{-25} /
  1.794 x 10\textsuperscript{-19} = 1.108 x 10\textsuperscript{-6} m =
  \textbf{1,108 nm}
\end{enumerate}

Silicon can detect wavelengths shorter than 1,108 nm, so the 874 nm GaAs
emission falls well within the detector's range.

\begin{center}\rule{0.5\linewidth}{0.5pt}\end{center}

\section{Problem 3.2.3}\label{problem-3.2.3}

\textbf{Given:} A GaAs HEMT has electron mobility mu\textsubscript{n} =
8,000 cm\textsuperscript{2}/(V\emph{s) and a silicon MOSFET has
mu\textsubscript{n} = 1,200 cm\textsuperscript{2}/(V}s). Both devices
have a channel length of L = 0.25 um. The GaAs device operates at
V\textsubscript{DD} = 3.3 V and the silicon device at
V\textsubscript{DD} = 1.8 V.

\textbf{Find:} (a) The electron transit time for each device at an
average field of E = 10 kV/cm, (b) the estimated f\textsubscript{T} for
each, and (c) the ratio of f\textsubscript{T} values.

\textbf{Solution:}

\begin{enumerate}
\def\labelenumi{(\alph{enumi})}
\tightlist
\item
  Drift velocity: GaAs: v\textsubscript{d} = 8,000 x 10,000 = 8.0 x
  10\textsuperscript{7} cm/s Silicon: v\textsubscript{d} = 1,200 x
  10,000 = 1.2 x 10\textsuperscript{7} cm/s
\end{enumerate}

Transit time tau = L/v\textsubscript{d}: GaAs: tau = 0.25 x
10\textsuperscript{-4} / 8.0 x 10\textsuperscript{7} = \textbf{3.13 x
10\textsuperscript{-13} s = 0.313 ps} Silicon: tau = 0.25 x
10\textsuperscript{-4} / 1.2 x 10\textsuperscript{7} = \textbf{2.08 x
10\textsuperscript{-12} s = 2.08 ps}

\begin{enumerate}
\def\labelenumi{(\alph{enumi})}
\setcounter{enumi}{1}
\item
  f\textsubscript{T} \textasciitilde{} 1/(2 x pi x tau): GaAs:
  f\textsubscript{T} = 1/(2 x pi x 3.13 x 10\textsuperscript{-13}) =
  \textbf{509 GHz} Silicon: f\textsubscript{T} = 1/(2 x pi x 2.08 x
  10\textsuperscript{-12}) = \textbf{76.5 GHz}
\item
  Ratio: 509/76.5 = \textbf{6.7x} advantage for GaAs, consistent with
  its use in mmWave and 5G applications.
\end{enumerate}

\begin{center}\rule{0.5\linewidth}{0.5pt}\end{center}

\section{Problem 3.2.4}\label{problem-3.2.4}

\textbf{Given:} A GaN HEMT power amplifier for a 5G base station
operates at 3.5 GHz with V\textsubscript{DS} = 28 V and
I\textsubscript{D} = 3.5 A. The device achieves a drain efficiency of
65\%. An equivalent GaAs device at the same frequency achieves 45\%
efficiency at V\textsubscript{DS} = 12 V and I\textsubscript{D} = 7 A.

\textbf{Find:} (a) The DC input power and RF output power for each
device, (b) the heat dissipated in each, and (c) the power density
advantage of GaN (assuming the GaN die is 4 mm x 1 mm and the GaAs die
is 8 mm x 1.5 mm).

\textbf{Solution:}

\begin{enumerate}
\def\labelenumi{(\alph{enumi})}
\tightlist
\item
  GaN: P\textsubscript{DC} = V\textsubscript{DS} x I\textsubscript{D} =
  28 x 3.5 = \textbf{98 W} P\textsubscript{RF} = eta x
  P\textsubscript{DC} = 0.65 x 98 = \textbf{63.7 W}
\end{enumerate}

GaAs: P\textsubscript{DC} = 12 x 7 = \textbf{84 W} P\textsubscript{RF} =
0.45 x 84 = \textbf{37.8 W}

\begin{enumerate}
\def\labelenumi{(\alph{enumi})}
\setcounter{enumi}{1}
\item
  Heat dissipated: GaN: P\textsubscript{heat} = P\textsubscript{DC} -
  P\textsubscript{RF} = 98 - 63.7 = \textbf{34.3 W} GaAs:
  P\textsubscript{heat} = 84 - 37.8 = \textbf{46.2 W}
\item
  Power density (RF output per die area): GaN: 63.7 / (4 x 1) =
  \textbf{15.9 W/mm\textsuperscript{2}} GaAs: 37.8 / (8 x 1.5) =
  \textbf{3.15 W/mm\textsuperscript{2}}
\end{enumerate}

Power density advantage: 15.9 / 3.15 = \textbf{5.0x}, demonstrating why
GaN enables smaller, higher-power amplifiers.

\begin{center}\rule{0.5\linewidth}{0.5pt}\end{center}

\section{Problem 3.2.5}\label{problem-3.2.5}

\textbf{Given:} A GaN-on-SiC power switching transistor operates at
V\textsubscript{DS} = 650 V with R\textsubscript{DS(on)} = 25 mohm at 25
degrees C. The R\textsubscript{DS(on)} temperature coefficient is
+0.8\%/degree C. An equivalent SiC MOSFET has R\textsubscript{DS(on)} =
40 mohm at 25 degrees C with a temperature coefficient of +1.2\%/degree
C. Both devices carry 30 A at a junction temperature of 150 degrees C.

\textbf{Find:} (a) R\textsubscript{DS(on)} at 150 degrees C for each
device, and (b) the conduction losses at 150 degrees C for each.

\textbf{Solution:}

\begin{enumerate}
\def\labelenumi{(\alph{enumi})}
\tightlist
\item
  Temperature rise: delta\_T = 150 - 25 = 125 degrees C.
\end{enumerate}

GaN: R\textsubscript{DS(on)}(150) = 25 x (1 + 0.008 x 125) = 25 x 2.0 =
\textbf{50 mohm} SiC: R\textsubscript{DS(on)}(150) = 40 x (1 + 0.012 x
125) = 40 x 2.5 = \textbf{100 mohm}

\begin{enumerate}
\def\labelenumi{(\alph{enumi})}
\setcounter{enumi}{1}
\tightlist
\item
  Conduction losses at 150 degrees C: GaN: P\textsubscript{cond} =
  I\textsuperscript{2} x R\textsubscript{DS(on)} = 30\textsuperscript{2}
  x 0.050 = 900 x 0.050 = \textbf{45.0 W} SiC: P\textsubscript{cond} =
  30\textsuperscript{2} x 0.100 = 900 x 0.100 = \textbf{90.0 W}
\end{enumerate}

The GaN device has \textbf{50\% lower} conduction losses at operating
temperature, though both devices show significant
R\textsubscript{DS(on)} increases (2x and 2.5x) from the cold-start
values --- thermal derating must be included in the design.

\chapter{Chapter 3 --- Section 3.3: PN
Junction}\label{chapter-3-section-3.3-pn-junction}

Practice problems covering depletion regions, built-in potential,
forward/reverse bias characteristics, and junction capacitance.

\begin{center}\rule{0.5\linewidth}{0.5pt}\end{center}

\section{Problem 3.3.1}\label{problem-3.3.1}

\textbf{Given:} A silicon PN junction has N\textsubscript{A} =
10\textsuperscript{18} cm\textsuperscript{-3} and N\textsubscript{D} = 5
x 10\textsuperscript{15} cm\textsuperscript{-3}. T = 300 K. Use
n\textsubscript{i} = 1.5 x 10\textsuperscript{10}
cm\textsuperscript{-3}, V\textsubscript{T} = 0.02585 V,
epsilon\textsubscript{Si} = 1.036 x 10\textsuperscript{-12} F/cm, and q
= 1.602 x 10\textsuperscript{-19} C.

\textbf{Find:} (a) The built-in potential, (b) the depletion width at
zero bias, (c) the depletion width at a reverse bias of 10 V, and (d)
the ratio of depletion widths.

\textbf{Solution:}

\begin{enumerate}
\def\labelenumi{(\alph{enumi})}
\item
  Built-in potential: V\textsubscript{bi} = V\textsubscript{T} x
  ln(N\textsubscript{A} x N\textsubscript{D} /
  n\textsubscript{i}\textsuperscript{2}) = 0.02585 x
  ln(10\textsuperscript{18} x 5 x 10\textsuperscript{15} / (1.5 x
  10\textsuperscript{10})\textsuperscript{2}) = 0.02585 x ln(5 x
  10\textsuperscript{33} / 2.25 x 10\textsuperscript{20}) = 0.02585 x
  ln(2.222 x 10\textsuperscript{13}) = 0.02585 x 30.73 = \textbf{0.795
  V}
\item
  Since N\textsubscript{A} \textgreater\textgreater{}
  N\textsubscript{D}, the depletion extends mostly into the N-side:
  1/N\textsubscript{A} + 1/N\textsubscript{D} \textasciitilde{}
  1/N\textsubscript{D} = 2 x 10\textsuperscript{-16}
  cm\textsuperscript{3}
\end{enumerate}

W = sqrt(2 x epsilon\textsubscript{Si} x (V\textsubscript{bi} -
V\textsubscript{A}) / (q x N\textsubscript{D})) for one-sided junction
approximation: W = sqrt(2 x 1.036 x 10\textsuperscript{-12} x 0.795 /
(1.602 x 10\textsuperscript{-19} x 5 x 10\textsuperscript{15})) =
sqrt(1.647 x 10\textsuperscript{-12} / 8.01 x 10\textsuperscript{-4}) =
sqrt(2.056 x 10\textsuperscript{-9}) = \textbf{4.54 x
10\textsuperscript{-5} cm = 0.454 um}

\begin{enumerate}
\def\labelenumi{(\alph{enumi})}
\setcounter{enumi}{2}
\item
  At V\textsubscript{R} = 10 V (V\textsubscript{A} = -10 V): W = sqrt(2
  x 1.036 x 10\textsuperscript{-12} x (0.795 + 10) / (1.602 x
  10\textsuperscript{-19} x 5 x 10\textsuperscript{15})) = sqrt(2 x
  1.036 x 10\textsuperscript{-12} x 10.795 / 8.01 x
  10\textsuperscript{-4}) = sqrt(2.238 x 10\textsuperscript{-11} / 8.01
  x 10\textsuperscript{-4}) = sqrt(2.794 x 10\textsuperscript{-8}) =
  \textbf{1.672 x 10\textsuperscript{-4} cm = 1.672 um}
\item
  Ratio: 1.672/0.454 = \textbf{3.68x}
\end{enumerate}

The depletion width increases by a factor of sqrt((V\textsubscript{bi} +
V\textsubscript{R})/V\textsubscript{bi}) = sqrt(10.795/0.795) = 3.68,
confirming the square-root voltage dependence.

\begin{center}\rule{0.5\linewidth}{0.5pt}\end{center}

\section{Problem 3.3.2}\label{problem-3.3.2}

\textbf{Given:} A silicon diode has I\textsubscript{S} = 5 x
10\textsuperscript{-15} A and ideality factor n = 1.05 at T = 300 K. The
diode carries a forward current of I\textsubscript{F} = 1.5 mA.

\textbf{Find:} (a) The forward voltage across the diode, (b) the dynamic
(small-signal) resistance at this operating point, and (c) the forward
voltage if the temperature increases to 350 K, using the empirical
temperature coefficient of approximately −2 mV/°C at constant current.

\textbf{Solution:}

\begin{enumerate}
\def\labelenumi{(\alph{enumi})}
\tightlist
\item
  From the Shockley equation, I =
  I\textsubscript{S}(e\textsuperscript{V/(nVT)} - 1) \textasciitilde{}
  I\textsubscript{S} x e\textsuperscript{V/(nVT)} (since I
  \textgreater\textgreater{} I\textsubscript{S}):
\end{enumerate}

V = n x V\textsubscript{T} x ln(I/I\textsubscript{S}) = 1.05 x 0.02585 x
ln(1.5 x 10\textsuperscript{-3} / 5 x 10\textsuperscript{-15}) = 0.02714
x ln(3 x 10\textsuperscript{11}) = 0.02714 x 26.43 = \textbf{0.717 V}

\begin{enumerate}
\def\labelenumi{(\alph{enumi})}
\setcounter{enumi}{1}
\item
  Dynamic resistance: r\textsubscript{d} = n x V\textsubscript{T} / I =
  1.05 x 0.02585 / 0.0015 = 0.02714 / 0.0015 = \textbf{18.1 ohm}
\item
  At 350 K (ΔT = +50°C above 300 K), using the empirical forward voltage
  temperature coefficient of approximately −2 mV/°C at constant current:
\end{enumerate}

V(350 K) ≈ 0.717 − 50 × 0.002 = 0.717 − 0.100 = \textbf{0.617 V}

The forward voltage decreases by approximately 100 mV over a 50°C
temperature rise, consistent with the well-known rule that silicon diode
forward voltage drops about 2 mV per °C increase in temperature at
constant current.

\begin{center}\rule{0.5\linewidth}{0.5pt}\end{center}

\section{Problem 3.3.3}\label{problem-3.3.3}

\textbf{Given:} A silicon solar cell PN junction has N\textsubscript{A}
= 10\textsuperscript{17} cm\textsuperscript{-3}, N\textsubscript{D} =
10\textsuperscript{16} cm\textsuperscript{-3}, and a junction area of A
= 100 cm\textsuperscript{2}. The built-in potential is
V\textsubscript{bi} = 0.757 V and epsilon\textsubscript{Si} = 1.036 x
10\textsuperscript{-12} F/cm. The junction operates at a reverse bias of
V\textsubscript{R} = 0.3 V (the solar cell's maximum power point voltage
creates a slight forward bias reducing the depletion capacitance, but
here we consider the dark capacitance at reverse bias).

\textbf{Find:} (a) The depletion width, (b) the zero-bias junction
capacitance per unit area, and (c) the total junction capacitance at
V\textsubscript{R} = 0.3 V.

\textbf{Solution:}

\begin{enumerate}
\def\labelenumi{(\alph{enumi})}
\tightlist
\item
  Using 1/N\textsubscript{A} + 1/N\textsubscript{D} =
  10\textsuperscript{-17} + 10\textsuperscript{-16} = 1.1 x
  10\textsuperscript{-16} cm\textsuperscript{3}:
\end{enumerate}

W = sqrt(2 x 1.036 x 10\textsuperscript{-12} x 1.1 x
10\textsuperscript{-16} x (0.757 + 0.3) / 1.602 x
10\textsuperscript{-19}) = sqrt(2.279 x 10\textsuperscript{-28} x 1.057
/ 1.602 x 10\textsuperscript{-19}) = sqrt(2.409 x
10\textsuperscript{-28} / 1.602 x 10\textsuperscript{-19}) = sqrt(1.504
x 10\textsuperscript{-9}) = \textbf{3.88 x 10\textsuperscript{-5} cm =
0.388 um}

\begin{enumerate}
\def\labelenumi{(\alph{enumi})}
\setcounter{enumi}{1}
\tightlist
\item
  Zero-bias junction capacitance per unit area: C\textsubscript{j0}/A =
  epsilon\textsubscript{Si}/W₀
\end{enumerate}

W₀ = sqrt(2 x 1.036 x 10\textsuperscript{-12} x 1.1 x
10\textsuperscript{-16} x 0.757 / 1.602 x 10\textsuperscript{-19}) =
sqrt(1.724 x 10\textsuperscript{-28} / 1.602 x 10\textsuperscript{-19})
= sqrt(1.076 x 10\textsuperscript{-9}) = 3.28 x 10\textsuperscript{-5}
cm

C\textsubscript{j0}/A = 1.036 x 10\textsuperscript{-12} / 3.28 x
10\textsuperscript{-5} = \textbf{3.16 x 10\textsuperscript{-8}
F/cm\textsuperscript{2} = 31.6 nF/cm\textsuperscript{2}}

\begin{enumerate}
\def\labelenumi{(\alph{enumi})}
\setcounter{enumi}{2}
\tightlist
\item
  At V\textsubscript{R} = 0.3 V: C\textsubscript{j} =
  C\textsubscript{j0} / sqrt(1 + V\textsubscript{R}/V\textsubscript{bi})
\end{enumerate}

C\textsubscript{j0} = 31.6 x 10\textsuperscript{-9} x 100 = 3.16 uF
(total zero-bias capacitance) C\textsubscript{j} = 3.16 x
10\textsuperscript{-6} / sqrt(1 + 0.3/0.757) = 3.16 x
10\textsuperscript{-6} / sqrt(1.396) = 3.16 x 10\textsuperscript{-6} /
1.182 = \textbf{2.67 uF}

\begin{center}\rule{0.5\linewidth}{0.5pt}\end{center}

\section{Problem 3.3.4}\label{problem-3.3.4}

\textbf{Given:} A varactor diode has C\textsubscript{j0} = 50 pF,
V\textsubscript{bi} = 0.7 V, and m = 0.45 (hyperabrupt junction). It is
used in a VCO tank circuit with L = 10 nH.

\textbf{Find:} (a) The junction capacitance at reverse bias voltages of
0.5 V, 2 V, 5 V, and 12 V, and (b) the corresponding resonant
frequencies and the tuning range.

\textbf{Solution:}

\begin{enumerate}
\def\labelenumi{(\alph{enumi})}
\tightlist
\item
  C\textsubscript{j} = C\textsubscript{j0} / (1 +
  V\textsubscript{R}/V\textsubscript{bi})\textsuperscript{m}:
\end{enumerate}

At V\textsubscript{R} = 0.5 V: C = 50 / (1.714)\textsuperscript{0.45} =
50 / 1.274 = \textbf{39.2 pF} At V\textsubscript{R} = 2 V: C = 50 /
(3.857)\textsuperscript{0.45} = 50 / 1.836 = \textbf{27.2 pF} At
V\textsubscript{R} = 5 V: C = 50 / (8.143)\textsuperscript{0.45} = 50 /
2.570 = \textbf{19.5 pF} At V\textsubscript{R} = 12 V: C = 50 /
(18.14)\textsuperscript{0.45} = 50 / 3.684 = \textbf{13.6 pF}

\begin{enumerate}
\def\labelenumi{(\alph{enumi})}
\setcounter{enumi}{1}
\tightlist
\item
  Resonant frequency f₀ = 1/(2π√(LC)):
\end{enumerate}

At 0.5 V: f₀ = 1/(2π√(10 x 10\textsuperscript{-9} x 39.2 x
10\textsuperscript{-12})) = 1/(2π x 6.261 x 10\textsuperscript{-10}) =
\textbf{254 MHz} At 2 V: f₀ = 1/(2π√(10 x 10\textsuperscript{-9} x 27.2
x 10\textsuperscript{-12})) = 1/(2π x 5.215 x 10\textsuperscript{-10}) =
\textbf{305 MHz} At 5 V: f₀ = 1/(2π√(10 x 10\textsuperscript{-9} x 19.5
x 10\textsuperscript{-12})) = 1/(2π x 4.416 x 10\textsuperscript{-10}) =
\textbf{361 MHz} At 12 V: f₀ = 1/(2π√(10 x 10\textsuperscript{-9} x 13.6
x 10\textsuperscript{-12})) = 1/(2π x 3.688 x 10\textsuperscript{-10}) =
\textbf{432 MHz}

Tuning range: 432/254 = \textbf{1.70:1} (254 MHz to 432 MHz), suitable
for a wideband VCO.

\begin{center}\rule{0.5\linewidth}{0.5pt}\end{center}

\section{Problem 3.3.5}\label{problem-3.3.5}

\textbf{Given:} Two silicon diodes are connected in series (same
polarity) and driven by a constant current source of I = 2 mA. Diode 1
has I\textsubscript{S1} = 10\textsuperscript{-14} A and Diode 2 has
I\textsubscript{S2} = 5 x 10\textsuperscript{-14} A. Both have n = 1.0
and T = 300 K.

\textbf{Find:} (a) The forward voltage across each diode, (b) the total
voltage, and (c) the voltage across each diode if the temperature rises
to 85 degrees C (358 K), assuming I\textsubscript{S} doubles every 10
degrees C.

\textbf{Solution:}

\begin{enumerate}
\def\labelenumi{(\alph{enumi})}
\tightlist
\item
  At 300 K, V\textsubscript{T} = 0.02585 V:
\end{enumerate}

V₁ = V\textsubscript{T} x ln(I/I\textsubscript{S1}) = 0.02585 x ln(2 x
10\textsuperscript{-3} / 10\textsuperscript{-14}) = 0.02585 x ln(2 x
10\textsuperscript{11}) = 0.02585 x 26.02 = \textbf{0.673 V}

V₂ = V\textsubscript{T} x ln(I/I\textsubscript{S2}) = 0.02585 x ln(2 x
10\textsuperscript{-3} / 5 x 10\textsuperscript{-14}) = 0.02585 x ln(4 x
10\textsuperscript{10}) = 0.02585 x 24.41 = \textbf{0.631 V}

\begin{enumerate}
\def\labelenumi{(\alph{enumi})}
\setcounter{enumi}{1}
\item
  Total voltage: V\textsubscript{total} = 0.673 + 0.631 = \textbf{1.304
  V}
\item
  At 358 K (85 degrees C): V\textsubscript{T}(358) = 8.617 x
  10\textsuperscript{-5} x 358 = 0.03085 V Temperature rise = 58 degrees
  C, so I\textsubscript{S} multiplier = 2\textsuperscript{5.8} = 55.7
\end{enumerate}

I\textsubscript{S1}(358) = 10\textsuperscript{-14} x 55.7 = 5.57 x
10\textsuperscript{-13} A I\textsubscript{S2}(358) = 5 x
10\textsuperscript{-14} x 55.7 = 2.785 x 10\textsuperscript{-12} A

V₁ = 0.03085 x ln(2 x 10\textsuperscript{-3} / 5.57 x
10\textsuperscript{-13}) = 0.03085 x ln(3.59 x 10\textsuperscript{9}) =
0.03085 x 22.00 = \textbf{0.679 V} V₂ = 0.03085 x ln(2 x
10\textsuperscript{-3} / 2.785 x 10\textsuperscript{-12}) = 0.03085 x
ln(7.18 x 10\textsuperscript{8}) = 0.03085 x 20.39 = \textbf{0.629 V}

Total: 0.679 + 0.629 = \textbf{1.308 V} --- remarkably stable despite a
58 degrees C temperature increase, demonstrating why series diode
strings are used as voltage references.

\chapter{Chapter 3 --- Section 3.4:
Diodes}\label{chapter-3-section-3.4-diodes}

Practice problems covering rectifier diodes, Zener diodes, Schottky
diodes, and LEDs.

\begin{center}\rule{0.5\linewidth}{0.5pt}\end{center}

\section{Problem 3.4.1}\label{problem-3.4.1}

\textbf{Given:} A half-wave rectifier uses a single silicon diode
(V\textsubscript{f} = 0.7 V) powered by a 24 V\textsubscript{rms}
transformer secondary. The load is 50 ohm with a 2,200 uF filter
capacitor. Frequency is 60 Hz.

\textbf{Find:} (a) The peak output voltage, (b) the DC load current, (c)
the peak-to-peak ripple voltage, and (d) the ripple as a percentage of
the DC output.

\textbf{Solution:}

\begin{enumerate}
\def\labelenumi{(\alph{enumi})}
\item
  Peak transformer voltage: V\textsubscript{peak} = 24 x sqrt(2) = 33.94
  V One diode conducts: V\textsubscript{out\_peak} = 33.94 - 0.7 =
  \textbf{33.24 V}
\item
  DC load current: I\textsubscript{DC} \textasciitilde{}
  V\textsubscript{out\_peak}/R\textsubscript{L} = 33.24/50 =
  \textbf{664.8 mA}
\item
  For a half-wave rectifier, f\textsubscript{ripple} = 60 Hz:
  V\textsubscript{ripple} = I\textsubscript{DC}/(f\textsubscript{ripple}
  x C) = 0.6648/(60 x 0.0022) = 0.6648/0.132 = \textbf{5.04 V
  peak-to-peak}
\item
  Ripple percentage: 5.04/33.24 x 100 = \textbf{15.2\%}
\end{enumerate}

This poor ripple performance illustrates why full-wave rectifiers are
preferred for most applications.

\begin{center}\rule{0.5\linewidth}{0.5pt}\end{center}

\section{Problem 3.4.2}\label{problem-3.4.2}

\textbf{Given:} A 12 V Zener diode has dynamic resistance
r\textsubscript{z} = 15 ohm and is rated for P\textsubscript{max} = 5 W.
The input voltage is 20 V and the series resistor R\textsubscript{S} =
47 ohm. The load can vary from R\textsubscript{L} = 100 ohm (heavy load)
to R\textsubscript{L} = open circuit (no load).

\textbf{Find:} (a) The Zener current at full load, (b) the Zener current
at no load, (c) the maximum Zener power dissipation, and (d) the output
voltage regulation (change in V\textsubscript{out}) from full load to no
load, accounting for r\textsubscript{z}.

\textbf{Solution:}

\begin{enumerate}
\def\labelenumi{(\alph{enumi})}
\item
  At full load (R\textsubscript{L} = 100 ohm): I\textsubscript{L} =
  V\textsubscript{Z}/R\textsubscript{L} = 12/100 = 120 mA
  I\textsubscript{total} = (V\textsubscript{in} -
  V\textsubscript{Z})/R\textsubscript{S} = (20 - 12)/47 = 8/47 = 170.2
  mA I\textsubscript{Z} = I\textsubscript{total} - I\textsubscript{L} =
  170.2 - 120 = \textbf{50.2 mA}
\item
  At no load (R\textsubscript{L} = open, I\textsubscript{L} = 0):
  I\textsubscript{Z} = I\textsubscript{total} = (20 - 12)/47 =
  \textbf{170.2 mA}
\item
  Maximum power: P\textsubscript{Z} = V\textsubscript{Z} x
  I\textsubscript{Z\_max} = 12 x 0.1702 = \textbf{2.04 W} (within 5 W
  rating)
\item
  Voltage change due to Zener current variation:
  delta\_I\textsubscript{Z} = 170.2 - 50.2 = 120 mA
  delta\_V\textsubscript{out} = delta\_I\textsubscript{Z} x
  r\textsubscript{z} = 0.120 x 15 = \textbf{1.8 V}
\end{enumerate}

The output varies from approximately 12.0 V at full load to 12.0 + 1.8 =
13.8 V at no load. Load regulation = 1.8/12 x 100 = \textbf{15\%} ---
mediocre, suggesting a lower r\textsubscript{z} Zener or an active
regulator would be preferable.

\begin{center}\rule{0.5\linewidth}{0.5pt}\end{center}

\section{Problem 3.4.3}\label{problem-3.4.3}

\textbf{Given:} A synchronous buck converter switching at 500 kHz uses a
Schottky body diode with V\textsubscript{f} = 0.45 V during dead time.
The dead time is t\textsubscript{dead} = 30 ns per transition (two
transitions per cycle). The inductor current (and diode current during
dead time) is I\textsubscript{L} = 15 A. The high-side MOSFET body diode
would have V\textsubscript{f} = 1.0 V during the same dead time.

\textbf{Find:} (a) The energy lost in the Schottky diode per dead-time
event, (b) the total dead-time power loss with the Schottky diode, and
(c) the power savings compared to using the MOSFET body diode.

\textbf{Solution:}

\begin{enumerate}
\def\labelenumi{(\alph{enumi})}
\item
  Energy per dead-time event (Schottky): E\textsubscript{dead} =
  V\textsubscript{f} x I\textsubscript{L} x t\textsubscript{dead} = 0.45
  x 15 x 30 x 10\textsuperscript{-9} = \textbf{202.5 nJ}
\item
  Two dead-time events per switching cycle: P\textsubscript{dead} = 2 x
  E\textsubscript{dead} x f\textsubscript{sw} = 2 x 202.5 x
  10\textsuperscript{-9} x 500,000 = \textbf{0.203 W}
\item
  With MOSFET body diode: E\textsubscript{dead} = 1.0 x 15 x 30 x
  10\textsuperscript{-9} = 450 nJ P\textsubscript{dead} = 2 x 450 x
  10\textsuperscript{-9} x 500,000 = 0.450 W
\end{enumerate}

Power savings: 0.450 - 0.203 = \textbf{0.247 W} (55\% reduction in
dead-time losses).

\begin{center}\rule{0.5\linewidth}{0.5pt}\end{center}

\section{Problem 3.4.4}\label{problem-3.4.4}

\textbf{Given:} An LED streetlight module contains 60 white LEDs
arranged as 10 parallel strings of 6 LEDs in series. Each LED has
V\textsubscript{f} = 3.2 V at I\textsubscript{f} = 350 mA and a luminous
efficacy of 140 lm/W. The LED driver provides a constant current of 350
mA per string from a 24 V DC bus using a buck converter at 92\%
efficiency.

\textbf{Find:} (a) The forward voltage of each string, (b) the total
luminous output, (c) the electrical power consumed by the LEDs, (d) the
total input power from the 24 V bus, and (e) the system luminous
efficacy (lumens per watt from the bus).

\textbf{Solution:}

\begin{enumerate}
\def\labelenumi{(\alph{enumi})}
\item
  Each string has 6 LEDs in series: V\textsubscript{string} = 6 x 3.2 =
  \textbf{19.2 V}
\item
  Power per LED: P\textsubscript{LED} = V\textsubscript{f} x
  I\textsubscript{f} = 3.2 x 0.350 = 1.12 W Lumens per LED: 140 x 1.12 =
  156.8 lm Total: 60 x 156.8 = \textbf{9,408 lumens}
\item
  Total LED power: 60 x 1.12 = \textbf{67.2 W}
\item
  Input power from bus: P\textsubscript{in} = P\textsubscript{LED}/eta =
  67.2/0.92 = \textbf{73.0 W}
\item
  System luminous efficacy: eta\textsubscript{system} = 9,408/73.0 =
  \textbf{128.9 lm/W}
\end{enumerate}

This is approximately 8.6x more efficient than a 15 lm/W incandescent
source producing the same light output.

\begin{center}\rule{0.5\linewidth}{0.5pt}\end{center}

\section{Problem 3.4.5}\label{problem-3.4.5}

\textbf{Given:} A full-wave bridge rectifier with four silicon diodes
(V\textsubscript{f} = 0.7 V each) is connected to a 120
V\textsubscript{rms} / 18 V\textsubscript{rms} transformer. The
secondary winding resistance is R\textsubscript{w} = 0.5 ohm. The load
draws 2 A DC through a 4,700 uF filter capacitor at 60 Hz.

\textbf{Find:} (a) The no-load peak output voltage, (b) the approximate
loaded DC output voltage accounting for diode drops and winding
resistance, (c) the ripple voltage, and (d) the peak repetitive diode
current (during the short capacitor charging pulse).

\textbf{Solution:}

\begin{enumerate}
\def\labelenumi{(\alph{enumi})}
\item
  V\textsubscript{peak} = 18 x sqrt(2) = 25.46 V Two diodes conduct:
  V\textsubscript{out\_peak} = 25.46 - 2 x 0.7 = \textbf{24.06 V}
\item
  With winding resistance loss: V\textsubscript{DC} \textasciitilde{}
  V\textsubscript{out\_peak} - I\textsubscript{DC} x R\textsubscript{w}
  - V\textsubscript{ripple}/2 Ripple first: V\textsubscript{ripple} =
  I\textsubscript{DC}/(f\textsubscript{ripple} x C) = 2/(120 x 0.0047) =
  2/0.564 = 3.55 V V\textsubscript{DC} \textasciitilde{} 24.06 - 2 x 0.5
  - 3.55/2 = 24.06 - 1.0 - 1.78 = \textbf{21.28 V}
\item
  Ripple voltage: V\textsubscript{ripple} = \textbf{3.55 V peak-to-peak}
  (calculated above)
\item
  The capacitor charging pulse occurs during a fraction of each half
  cycle. The conduction angle is approximately: theta\textsubscript{c}
  \textasciitilde{} sqrt(2 x
  V\textsubscript{ripple}/V\textsubscript{peak}) = sqrt(2 x 3.55/25.46)
  = sqrt(0.279) = 0.528 rad = 30.3 degrees
\end{enumerate}

Peak diode current: I\textsubscript{peak} \textasciitilde{}
I\textsubscript{DC} x (2 x pi / theta\textsubscript{c}) = 2 x
(6.283/0.528) = 2 x 11.9 = \textbf{23.8 A}

This large peak current is why rectifier diodes must have high surge
current ratings and why transformer current ratings exceed the DC load
current.

\begin{center}\rule{0.5\linewidth}{0.5pt}\end{center}

\section{Problem 3.4.6}\label{problem-3.4.6}

\textbf{Given:} An engineer needs to design a Zener-based voltage clamp
to protect a 3.3 V microcontroller ADC input. The ADC input voltage
range is 0 to 3.3 V. A 3.3 V Zener diode (r\textsubscript{z} = 30 ohm)
is placed from the ADC input to ground, with a 1 kohm series resistor
from the signal source. The signal source can produce 0 to 12 V.

\textbf{Find:} (a) The ADC input voltage when the source is at 3.0 V
(below Zener threshold), (b) the ADC input voltage when the source is at
5.0 V, (c) the ADC input voltage when the source is at 12 V, and (d) the
maximum Zener power dissipation.

\textbf{Solution:}

\begin{enumerate}
\def\labelenumi{(\alph{enumi})}
\item
  At V\textsubscript{source} = 3.0 V: The Zener is not conducting (V
  \textless{} V\textsubscript{Z}). The ADC has very high input
  impedance, so virtually no current flows through R\textsubscript{S}:
  V\textsubscript{ADC} \textasciitilde{} \textbf{3.0 V}
\item
  At V\textsubscript{source} = 5.0 V: The Zener conducts. Using a
  simplified model: I\textsubscript{Z} = (V\textsubscript{source} -
  V\textsubscript{Z})/(R\textsubscript{S} + r\textsubscript{z}) = (5.0 -
  3.3)/(1,000 + 30) = 1.7/1,030 = 1.65 mA V\textsubscript{ADC} =
  V\textsubscript{Z} + I\textsubscript{Z} x r\textsubscript{z} = 3.3 +
  0.00165 x 30 = 3.3 + 0.050 = \textbf{3.35 V}
\item
  At V\textsubscript{source} = 12 V: I\textsubscript{Z} = (12 -
  3.3)/1,030 = 8.7/1,030 = 8.45 mA V\textsubscript{ADC} = 3.3 + 0.00845
  x 30 = 3.3 + 0.254 = \textbf{3.55 V}
\end{enumerate}

This exceeds 3.3 V by 0.25 V, which may damage the ADC. The dynamic
resistance r\textsubscript{z} limits the clamping accuracy.

\begin{enumerate}
\def\labelenumi{(\alph{enumi})}
\setcounter{enumi}{3}
\tightlist
\item
  Maximum Zener power: P\textsubscript{Z} = V\textsubscript{Z} x
  I\textsubscript{Z} \textasciitilde{} 3.3 x 0.00845 = \textbf{27.9 mW}
  (negligible for most Zener ratings)
\end{enumerate}

A better protection approach would use a Schottky diode clamp to
V\textsubscript{DD} (3.3 V) with a lower voltage drop, or use a
precision TVS diode with lower r\textsubscript{z}.

\chapter{Chapter 3 --- Section 3.5:
Transistors}\label{chapter-3-section-3.5-transistors}

Practice problems covering BJTs, MOSFETs, and power MOSFETs in
amplification and switching applications.

\begin{center}\rule{0.5\linewidth}{0.5pt}\end{center}

\section{Problem 3.5.1}\label{problem-3.5.1}

\textbf{Given:} An NPN BJT has beta = 200 and V\textsubscript{BE} = 0.7
V. It is used in a common-emitter configuration with V\textsubscript{CC}
= 15 V, R\textsubscript{C} = 3.3 kohm, R\textsubscript{E} = 680 ohm, and
a voltage divider bias network (R₁ = 56 kohm, R₂ = 12 kohm).

\textbf{Find:} (a) The base voltage V\textsubscript{B}, (b) the emitter
current I\textsubscript{E}, (c) the collector current
I\textsubscript{C}, (d) V\textsubscript{CE}, and (e) whether the
transistor is in the active region.

\textbf{Solution:}

\begin{enumerate}
\def\labelenumi{(\alph{enumi})}
\item
  Base voltage (voltage divider, assuming I\textsubscript{B} is small):
  V\textsubscript{B} = V\textsubscript{CC} x R₂/(R₁ + R₂) = 15 x
  12,000/(56,000 + 12,000) = 15 x 0.1765 = \textbf{2.65 V}
\item
  Emitter voltage: V\textsubscript{E} = V\textsubscript{B} -
  V\textsubscript{BE} = 2.65 - 0.7 = 1.95 V Emitter current:
  I\textsubscript{E} = V\textsubscript{E}/R\textsubscript{E} = 1.95/680
  = \textbf{2.87 mA}
\item
  I\textsubscript{C} \textasciitilde{} I\textsubscript{E} x beta/(beta +
  1) = 2.87 x 200/201 = \textbf{2.85 mA}
\item
  V\textsubscript{CE} = V\textsubscript{CC} - I\textsubscript{C} x
  R\textsubscript{C} - I\textsubscript{E} x R\textsubscript{E} = 15 -
  0.00285 x 3,300 - 0.00287 x 680 = 15 - 9.41 - 1.95 = \textbf{3.64 V}
\item
  V\textsubscript{CE} = 3.64 V \textgreater{} V\textsubscript{CE(sat)}
  \textasciitilde{} 0.2 V and V\textsubscript{CB} = V\textsubscript{CE}
  - V\textsubscript{BE} = 3.64 - 0.7 = 2.94 V \textgreater{} 0. The
  transistor is in the \textbf{active region} (suitable for linear
  amplification).
\end{enumerate}

\begin{center}\rule{0.5\linewidth}{0.5pt}\end{center}

\section{Problem 3.5.2}\label{problem-3.5.2}

\textbf{Given:} An NPN BJT switching circuit has V\textsubscript{CC} = 5
V, R\textsubscript{C} = 1 kohm, and a base resistor R\textsubscript{B} =
10 kohm driven by a 5 V logic signal. The transistor has
beta\textsubscript{min} = 80 and V\textsubscript{CE(sat)} = 0.2 V,
V\textsubscript{BE(sat)} = 0.8 V.

\textbf{Find:} (a) The base current, (b) the collector current in
saturation, (c) the forced beta, (d) whether the transistor is indeed
saturated, and (e) the power dissipated in the transistor when on.

\textbf{Solution:}

\begin{enumerate}
\def\labelenumi{(\alph{enumi})}
\item
  Base current: I\textsubscript{B} = (V\textsubscript{in} -
  V\textsubscript{BE(sat)})/R\textsubscript{B} = (5 - 0.8)/10,000 =
  4.2/10,000 = \textbf{0.42 mA}
\item
  Collector current in saturation: I\textsubscript{C(sat)} =
  (V\textsubscript{CC} - V\textsubscript{CE(sat)})/R\textsubscript{C} =
  (5 - 0.2)/1,000 = \textbf{4.8 mA}
\item
  Forced beta: beta\textsubscript{forced} =
  I\textsubscript{C(sat)}/I\textsubscript{B} = 4.8/0.42 = \textbf{11.4}
\item
  Since beta\textsubscript{forced} = 11.4 \textless{}
  beta\textsubscript{min} = 80, the transistor has excess base drive and
  is \textbf{confirmed saturated} with an overdrive factor of 80/11.4 =
  7.0x.
\item
  Power dissipation when on: P = V\textsubscript{CE(sat)} x
  I\textsubscript{C} + V\textsubscript{BE(sat)} x I\textsubscript{B} =
  0.2 x 0.0048 + 0.8 x 0.00042 = 0.96 mW + 0.34 mW = \textbf{1.30 mW}
\end{enumerate}

\begin{center}\rule{0.5\linewidth}{0.5pt}\end{center}

\section{Problem 3.5.3}\label{problem-3.5.3}

\textbf{Given:} An N-channel enhancement MOSFET has V\textsubscript{th}
= 1.5 V and k\textsubscript{n} = 2.0 mA/V\textsuperscript{2}. It is
biased at V\textsubscript{GS} = 4 V and used with R\textsubscript{D} = 2
kohm and V\textsubscript{DD} = 12 V.

\textbf{Find:} (a) The drain current, (b) V\textsubscript{DS}, (c) the
operating region, (d) the small-signal transconductance
g\textsubscript{m}, and (e) the voltage gain with R\textsubscript{D} as
the load.

\textbf{Solution:}

\begin{enumerate}
\def\labelenumi{(\alph{enumi})}
\item
  Assume saturation first: I\textsubscript{D} = (k\textsubscript{n}/2) x
  (V\textsubscript{GS} - V\textsubscript{th})\textsuperscript{2} = (2.0
  x 10\textsuperscript{-3}/2) x (4 - 1.5)\textsuperscript{2} =
  10\textsuperscript{-3} x 6.25 = \textbf{6.25 mA}
\item
  V\textsubscript{DS} = V\textsubscript{DD} - I\textsubscript{D} x
  R\textsubscript{D} = 12 - 0.00625 x 2,000 = 12 - 12.5 = -0.5 V
\end{enumerate}

Negative V\textsubscript{DS} is impossible. The MOSFET is in the
\textbf{triode (linear) region}, not saturation.

\begin{enumerate}
\def\labelenumi{(\alph{enumi})}
\setcounter{enumi}{2}
\tightlist
\item
  In triode: I\textsubscript{D} = k\textsubscript{n} x
  {[}(V\textsubscript{GS} - V\textsubscript{th}) x V\textsubscript{DS} -
  V\textsubscript{DS}\textsuperscript{2}/2{]} KVL: V\textsubscript{DS} =
  V\textsubscript{DD} - I\textsubscript{D} x R\textsubscript{D}
\end{enumerate}

Substituting: V\textsubscript{DS} = 12 - 2,000 x 2 x
10\textsuperscript{-3} x {[}(2.5) x V\textsubscript{DS} -
V\textsubscript{DS}\textsuperscript{2}/2{]} V\textsubscript{DS} = 12 - 4
x {[}2.5 x V\textsubscript{DS} -
V\textsubscript{DS}\textsuperscript{2}/2{]} V\textsubscript{DS} = 12 -
10 x V\textsubscript{DS} + 2 x V\textsubscript{DS}\textsuperscript{2}
2V\textsubscript{DS}\textsuperscript{2} - 11V\textsubscript{DS} + 12 = 0
V\textsubscript{DS} = (11 +/- sqrt(121 - 96))/4 = (11 +/- 5)/4

V\textsubscript{DS} = 4.0 V or 1.5 V. Since V\textsubscript{DS}
\textless{} V\textsubscript{GS} - V\textsubscript{th} = 2.5 V for
triode: V\textsubscript{DS} = \textbf{1.5 V} (the other root gives
V\textsubscript{DS} = 4 V \textgreater{} 2.5 V, which would be
saturation)

Recalculate I\textsubscript{D}: I\textsubscript{D} = (12 - 1.5)/2,000 =
\textbf{5.25 mA}

\begin{enumerate}
\def\labelenumi{(\alph{enumi})}
\setcounter{enumi}{3}
\item
  In saturation, g\textsubscript{m} would be: g\textsubscript{m} =
  k\textsubscript{n} x (V\textsubscript{GS} - V\textsubscript{th}) = 2 x
  10\textsuperscript{-3} x 2.5 = \textbf{5.0 mA/V}
\item
  In saturation, voltage gain: A\textsubscript{v} = -g\textsubscript{m}
  x R\textsubscript{D} = -5.0 x 10\textsuperscript{-3} x 2,000 =
  \textbf{-10} But since the device is in triode, it acts more like a
  resistor. To use this MOSFET as an amplifier with gain = -10,
  R\textsubscript{D} must be reduced or V\textsubscript{DD} increased to
  keep V\textsubscript{DS} \textgreater{} 2.5 V.
\end{enumerate}

\begin{center}\rule{0.5\linewidth}{0.5pt}\end{center}

\section{Problem 3.5.4}\label{problem-3.5.4}

\textbf{Given:} A power MOSFET drives a 24 V brushless DC motor at
f\textsubscript{sw} = 20 kHz using PWM. The MOSFET has
R\textsubscript{DS(on)} = 12 mohm at 25 degrees C (increasing to 18 mohm
at T\textsubscript{J} = 100 degrees C), Q\textsubscript{g} = 80 nC,
t\textsubscript{rise} = 25 ns, t\textsubscript{fall} = 35 ns. The motor
draws 30 A at a PWM duty cycle of 75\%.

\textbf{Find:} (a) The RMS current through the MOSFET, (b) the
conduction loss at T\textsubscript{J} = 100 degrees C, (c) the switching
loss, (d) the gate drive loss at V\textsubscript{GS} = 10 V, and (e) the
total loss and junction temperature if theta\textsubscript{JA} = 40
degrees C/W and T\textsubscript{A} = 40 degrees C.

\textbf{Solution:}

\begin{enumerate}
\def\labelenumi{(\alph{enumi})}
\item
  RMS current: I\textsubscript{RMS} = I\textsubscript{load} x sqrt(D) =
  30 x sqrt(0.75) = 30 x 0.866 = \textbf{25.98 A}
\item
  Conduction loss at 100 degrees C: P\textsubscript{cond} =
  I\textsubscript{RMS}\textsuperscript{2} x R\textsubscript{DS(on)} =
  25.98\textsuperscript{2} x 0.018 = 675 x 0.018 = \textbf{12.15 W}
\item
  Switching loss: P\textsubscript{sw} = 0.5 x V\textsubscript{DS} x
  I\textsubscript{load} x (t\textsubscript{rise} +
  t\textsubscript{fall}) x f\textsubscript{sw} = 0.5 x 24 x 30 x (25 +
  35) x 10\textsuperscript{-9} x 20,000 = 360 x 60 x
  10\textsuperscript{-9} x 20,000 = \textbf{0.432 W}
\item
  Gate drive loss: P\textsubscript{gate} = Q\textsubscript{g} x
  V\textsubscript{GS} x f\textsubscript{sw} = 80 x
  10\textsuperscript{-9} x 10 x 20,000 = \textbf{0.016 W}
\item
  Total loss: P\textsubscript{total} = 12.15 + 0.432 + 0.016 =
  \textbf{12.60 W}
\end{enumerate}

Junction temperature: T\textsubscript{J} = T\textsubscript{A} +
P\textsubscript{total} x theta\textsubscript{JA} = 40 + 12.60 x 40 = 40
+ 504 = \textbf{544 degrees C}

This far exceeds the 175 degrees C maximum junction temperature. The
thermal resistance theta\textsubscript{JA} = 40 degrees C/W (no
heatsink, free-standing package) is wholly inadequate. A heatsink with
theta\textsubscript{JA} \textless{} (100 - 40)/12.60 = 4.76 degrees C/W
is required.

\begin{center}\rule{0.5\linewidth}{0.5pt}\end{center}

\section{Problem 3.5.5}\label{problem-3.5.5}

\textbf{Given:} Two identical power MOSFETs (R\textsubscript{DS(on)} =
5.0 mohm each at 25 degrees C, temperature coefficient +0.4\%/degree C)
are connected in parallel to share a 50 A load current. Due to layout
asymmetry, MOSFET A runs 10 degrees C hotter than MOSFET B.

\textbf{Find:} (a) The R\textsubscript{DS(on)} of each MOSFET at their
respective operating temperatures if T\textsubscript{A} = 60 degrees C
and T\textsubscript{B} = 50 degrees C, (b) the current sharing ratio,
and (c) whether the positive temperature coefficient helps or hurts
current sharing.

\textbf{Solution:}

\begin{enumerate}
\def\labelenumi{(\alph{enumi})}
\item
  R\textsubscript{DS(on)} at operating temperature: MOSFET A:
  R\textsubscript{A} = 5.0 x (1 + 0.004 x (60 - 25)) = 5.0 x (1 + 0.14)
  = 5.0 x 1.14 = \textbf{5.70 mohm} MOSFET B: R\textsubscript{B} = 5.0 x
  (1 + 0.004 x (50 - 25)) = 5.0 x (1 + 0.10) = 5.0 x 1.10 = \textbf{5.50
  mohm}
\item
  In parallel with common V\textsubscript{DS}, current divides inversely
  with resistance: I\textsubscript{A} = I\textsubscript{total} x
  R\textsubscript{B}/(R\textsubscript{A} + R\textsubscript{B}) = 50 x
  5.50/(5.70 + 5.50) = 50 x 0.491 = \textbf{24.55 A} I\textsubscript{B}
  = 50 x 5.70/11.20 = 50 x 0.509 = \textbf{25.45 A}
\item
  The hotter MOSFET (A) carries \textbf{less} current (24.55 A vs 25.45
  A) because its resistance increased. This is a \textbf{self-balancing}
  mechanism --- the positive temperature coefficient of
  R\textsubscript{DS(on)} naturally promotes current sharing in parallel
  MOSFETs, unlike BJTs where thermal runaway causes current hogging.
\end{enumerate}

\begin{center}\rule{0.5\linewidth}{0.5pt}\end{center}

\section{Problem 3.5.6}\label{problem-3.5.6}

\textbf{Given:} An N-channel MOSFET with V\textsubscript{th} = 1.0 V and
k\textsubscript{n} = 4 mA/V\textsuperscript{2} is used in a
common-source amplifier with a drain resistor R\textsubscript{D} = 1
kohm and V\textsubscript{DD} = 5 V. The gate is biased at
V\textsubscript{GS} = 2.0 V. Channel-length modulation parameter lambda
= 0.02 V\textsuperscript{-1}.

\textbf{Find:} (a) The DC operating point (I\textsubscript{D},
V\textsubscript{DS}), (b) g\textsubscript{m}, (c) the output resistance
r\textsubscript{o} due to channel-length modulation, (d) the
small-signal voltage gain, and (e) the output voltage swing (limited by
saturation constraint V\textsubscript{DS} \textgreater=
V\textsubscript{GS} - V\textsubscript{th}).

\textbf{Solution:}

\begin{enumerate}
\def\labelenumi{(\alph{enumi})}
\tightlist
\item
  I\textsubscript{D} = (k\textsubscript{n}/2)(V\textsubscript{GS} -
  V\textsubscript{th})\textsuperscript{2}(1 + lambda x
  V\textsubscript{DS}) First approximation (ignore lambda):
  I\textsubscript{D} = (4 x
  10\textsuperscript{-3}/2)(1.0)\textsuperscript{2} = 2.0 mA
  V\textsubscript{DS} = V\textsubscript{DD} - I\textsubscript{D} x
  R\textsubscript{D} = 5 - 2.0 x 1 = 3.0 V Check: V\textsubscript{DS} =
  3.0 V \textgreater{} V\textsubscript{GS} - V\textsubscript{th} = 1.0
  V, so \textbf{saturation confirmed}.
\end{enumerate}

Refine with lambda: I\textsubscript{D} = 2.0 x (1 + 0.02 x 3.0) = 2.0 x
1.06 = \textbf{2.12 mA} V\textsubscript{DS} = 5 - 2.12 x 1 =
\textbf{2.88 V}

\begin{enumerate}
\def\labelenumi{(\alph{enumi})}
\setcounter{enumi}{1}
\item
  g\textsubscript{m} = k\textsubscript{n}(V\textsubscript{GS} -
  V\textsubscript{th}) = 4 x 10\textsuperscript{-3} x 1.0 = \textbf{4.0
  mA/V}
\item
  r\textsubscript{o} = 1/(lambda x I\textsubscript{D}) = 1/(0.02 x
  0.00212) = 1/4.24 x 10\textsuperscript{-5} = \textbf{23.6 kohm}
\item
  Voltage gain: A\textsubscript{v} = -g\textsubscript{m} x
  (R\textsubscript{D} \textbar\textbar{} r\textsubscript{o}) = -4 x
  10\textsuperscript{-3} x (1,000 x 23,600)/(1,000 + 23,600) = -4 x
  10\textsuperscript{-3} x 23,600,000/24,600 = -4 x
  10\textsuperscript{-3} x 959 = \textbf{-3.84}
\item
  Minimum V\textsubscript{DS} for saturation = V\textsubscript{GS} -
  V\textsubscript{th} = 1.0 V. Maximum I\textsubscript{D} =
  (V\textsubscript{DD} - V\textsubscript{DS,min})/R\textsubscript{D} =
  (5 - 1.0)/1,000 = 4.0 mA Maximum output swing downward: 2.88 - 1.0 =
  \textbf{1.88 V} below Q-point. Maximum output swing upward
  (I\textsubscript{D} -\textgreater{} 0): 5.0 - 2.88 = \textbf{2.12 V}
  above Q-point.
\end{enumerate}

\chapter{Chapter 3 --- Section 3.6: Voltage
Regulators}\label{chapter-3-section-3.6-voltage-regulators}

Practice problems covering linear regulators, LDOs, and the TL431
programmable reference.

\begin{center}\rule{0.5\linewidth}{0.5pt}\end{center}

\section{Problem 3.6.1}\label{problem-3.6.1}

\textbf{Given:} An LM7812 fixed +12 V linear regulator is powered from
an unregulated 18 V DC supply. It provides 800 mA to a load. The
regulator has a dropout voltage of 2 V, a quiescent current of 6 mA, and
is mounted in a TO-220 package with thermal resistance
theta\textsubscript{JC} = 5 degrees C/W. A heatsink with
theta\textsubscript{CS} = 0.5 degrees C/W and theta\textsubscript{SA} =
8 degrees C/W is attached. Ambient temperature is 35 degrees C.

\textbf{Find:} (a) The power dissipated in the regulator, (b) the
efficiency, (c) the junction temperature, and (d) the maximum ambient
temperature for safe operation (T\textsubscript{J,max} = 125 degrees C).

\textbf{Solution:}

\begin{enumerate}
\def\labelenumi{(\alph{enumi})}
\item
  Power dissipation: P\textsubscript{D} = (V\textsubscript{in} -
  V\textsubscript{out}) x (I\textsubscript{out} + I\textsubscript{Q}) =
  (18 - 12) x (0.800 + 0.006) = 6.0 x 0.806 = \textbf{4.84 W}
\item
  Efficiency: eta = P\textsubscript{out}/P\textsubscript{in} =
  (V\textsubscript{out} x I\textsubscript{out}) / (V\textsubscript{in} x
  (I\textsubscript{out} + I\textsubscript{Q})) = (12 x 0.800) / (18 x
  0.806) = 9.6 / 14.51 = \textbf{66.2\%}
\item
  Total thermal resistance: theta\textsubscript{JA} =
  theta\textsubscript{JC} + theta\textsubscript{CS} +
  theta\textsubscript{SA} = 5 + 0.5 + 8 = 13.5 degrees C/W
  T\textsubscript{J} = T\textsubscript{A} + P\textsubscript{D} x
  theta\textsubscript{JA} = 35 + 4.84 x 13.5 = 35 + 65.3 = \textbf{100.3
  degrees C}
\item
  Maximum ambient temperature: T\textsubscript{A,max} =
  T\textsubscript{J,max} - P\textsubscript{D} x theta\textsubscript{JA}
  = 125 - 4.84 x 13.5 = 125 - 65.3 = \textbf{59.7 degrees C}
\end{enumerate}

\begin{center}\rule{0.5\linewidth}{0.5pt}\end{center}

\section{Problem 3.6.2}\label{problem-3.6.2}

\textbf{Given:} An LDO regulator converts 3.6 V (single Li-ion cell,
ranging from 4.2 V fully charged to 3.0 V discharged) to 2.5 V for an
FPGA I/O bank drawing 200 mA. The LDO has a dropout voltage of 200 mV,
quiescent current I\textsubscript{Q} = 50 uA, and PSRR = 60 dB at 1 kHz.

\textbf{Find:} (a) Whether the LDO can regulate over the full battery
range, (b) the power dissipation at full charge and at near-discharge,
(c) the battery efficiency at each point, and (d) the input ripple
attenuation if the battery charger generates 10 mV\textsubscript{rms}
ripple at 1 kHz.

\textbf{Solution:}

\begin{enumerate}
\def\labelenumi{(\alph{enumi})}
\item
  Minimum input for regulation: V\textsubscript{in,min} =
  V\textsubscript{out} + V\textsubscript{dropout} = 2.5 + 0.2 = 2.7 V.
  Battery range is 3.0 to 4.2 V, so 3.0 V \textgreater{} 2.7 V.
  \textbf{Yes}, the LDO regulates over the full range.
\item
  Power dissipation: At full charge (4.2 V): P\textsubscript{D} = (4.2 -
  2.5) x 0.200 = 1.7 x 0.200 = \textbf{340 mW} At near-discharge (3.0
  V): P\textsubscript{D} = (3.0 - 2.5) x 0.200 = 0.5 x 0.200 =
  \textbf{100 mW}
\item
  Efficiency: At 4.2 V: eta = 2.5/4.2 = \textbf{59.5\%} At 3.0 V: eta =
  2.5/3.0 = \textbf{83.3\%}
\end{enumerate}

The LDO is most efficient when the battery is nearly discharged, which
is fortunate since that is when energy conservation matters most.

\begin{enumerate}
\def\labelenumi{(\alph{enumi})}
\setcounter{enumi}{3}
\tightlist
\item
  PSRR = 60 dB means 1,000:1 attenuation: Output ripple = 10 mV / 1,000
  = \textbf{10 uV\textsubscript{rms}} at the FPGA supply
\end{enumerate}

This excellent noise rejection is a key advantage of LDOs over switching
regulators for sensitive analog/digital circuits.

\begin{center}\rule{0.5\linewidth}{0.5pt}\end{center}

\section{Problem 3.6.3}\label{problem-3.6.3}

\textbf{Given:} A TL431 is used to regulate a 5.0 V output rail. The
internal reference voltage is V\textsubscript{ref} = 2.495 V. The
feedback resistor divider uses R₂ = 4.7 kohm (lower resistor). The TL431
cathode is connected through R\textsubscript{S} = 33 ohm to an
unregulated 8.0 V supply. The load draws 50 mA.

\textbf{Find:} (a) The required R₁ (upper resistor) for 5.0 V output,
(b) the divider current, (c) the total cathode current, (d) the TL431
power dissipation, and (e) the output voltage shift if R₁ has a +1\%
tolerance and R₂ has a -1\% tolerance (worst case).

\textbf{Solution:}

\begin{enumerate}
\def\labelenumi{(\alph{enumi})}
\tightlist
\item
  R₁ = R₂ x (V\textsubscript{out}/V\textsubscript{ref} - 1) = 4,700 x
  (5.0/2.495 - 1) = 4,700 x (2.004 - 1) = 4,700 x 1.004 R₁ =
  \textbf{4,719 ohm} (use standard 4.7 kohm, giving V\textsubscript{out}
  = 2.495 x (4,700 + 4,700)/4,700 = 4.99 V)
\end{enumerate}

With R₁ = 4.7 kohm: V\textsubscript{out} = 2.495 x (4,700 + 4,700)/4,700
= 2.495 x 2 = \textbf{4.99 V}

\begin{enumerate}
\def\labelenumi{(\alph{enumi})}
\setcounter{enumi}{1}
\item
  Divider current: I\textsubscript{div} = V\textsubscript{out}/(R₁ + R₂)
  = 4.99/(4,700 + 4,700) = 4.99/9,400 = \textbf{0.531 mA}
\item
  Total current through R\textsubscript{S}: I\textsubscript{total} =
  (V\textsubscript{supply} - V\textsubscript{out})/R\textsubscript{S} =
  (8.0 - 4.99)/33 = 3.01/33 = 91.2 mA I\textsubscript{load} = 50 mA,
  I\textsubscript{div} = 0.531 mA I\textsubscript{cathode} =
  I\textsubscript{total} - I\textsubscript{load} - I\textsubscript{div}
  = 91.2 - 50 - 0.531 = \textbf{40.7 mA}
\item
  TL431 power dissipation: P = V\textsubscript{KA} x I\textsubscript{K}
  = 4.99 x 0.0407 = \textbf{203 mW}
\item
  Worst-case tolerance analysis: R₁(+1\%) = 4,700 x 1.01 = 4,747 ohm
  R₂(-1\%) = 4,700 x 0.99 = 4,653 ohm V\textsubscript{out,max} = 2.495 x
  (4,747 + 4,653)/4,653 = 2.495 x 9,400/4,653 = 2.495 x 2.020 = 5.040 V
\end{enumerate}

Voltage shift: 5.040 - 4.99 = \textbf{+50 mV} (1\% resistor tolerances
produce a 1\% output shift)

\begin{center}\rule{0.5\linewidth}{0.5pt}\end{center}

\section{Problem 3.6.4}\label{problem-3.6.4}

\textbf{Given:} An engineer must select between an LDO and a buck
converter for a 12 V to 3.3 V, 500 mA power supply in a battery-powered
portable instrument. The LDO has I\textsubscript{Q} = 30 uA. The buck
converter has 88\% efficiency and I\textsubscript{Q} = 2 mA. Operating
time from a 12 V, 5 Ah battery must be maximized.

\textbf{Find:} (a) The LDO power dissipation and efficiency, (b) the
buck converter input current, (c) the total input current for each
solution, (d) the battery runtime for each, and (e) the crossover load
current where both solutions have equal efficiency.

\textbf{Solution:}

\begin{enumerate}
\def\labelenumi{(\alph{enumi})}
\item
  LDO: P\textsubscript{D} = (12 - 3.3) x 0.500 = 8.7 x 0.500 = 4.35 W
  eta = 3.3/12 = \textbf{27.5\%} I\textsubscript{in} =
  I\textsubscript{out} + I\textsubscript{Q} = 500 + 0.030 =
  \textbf{500.03 mA}
\item
  Buck converter input current: P\textsubscript{out} = 3.3 x 0.500 =
  1.65 W P\textsubscript{in} = P\textsubscript{out}/eta = 1.65/0.88 =
  1.875 W I\textsubscript{in,load} =
  P\textsubscript{in}/V\textsubscript{in} = 1.875/12 = \textbf{156.3 mA}
\item
  Total input current: LDO: \textbf{500.0 mA} (essentially all flows
  through) Buck: 156.3 + 2.0 = \textbf{158.3 mA}
\item
  Battery runtime (capacity / current): LDO: 5,000 mAh / 500.0 =
  \textbf{10.0 hours} Buck: 5,000 / 158.3 = \textbf{31.6 hours}
\item
  Crossover: LDO efficiency = V\textsubscript{out}/V\textsubscript{in} =
  27.5\% regardless of load. Buck efficiency depends on quiescent
  overhead. At load current I\textsubscript{L}: Buck total input power =
  3.3 x I\textsubscript{L}/0.88 + 12 x 0.002 LDO total input power = 12
  x I\textsubscript{L}
\end{enumerate}

Setting equal: 3.75 x I\textsubscript{L} + 0.024 = 12 x
I\textsubscript{L} 0.024 = 8.25 x I\textsubscript{L} I\textsubscript{L}
= 0.024/8.25 = \textbf{2.9 mA}

Below 2.9 mA load, the LDO is more efficient due to the buck converter's
higher quiescent current. Above 2.9 mA, the buck converter wins
decisively.

\chapter{Chapter 3 --- Section 3.7: Semiconductor
Fabrication}\label{chapter-3-section-3.7-semiconductor-fabrication}

Practice problems covering crystal growth, lithography, ion
implantation, etching, CMOS integration, and advanced nodes.

\begin{center}\rule{0.5\linewidth}{0.5pt}\end{center}

\section{Problem 3.7.1}\label{problem-3.7.1}

\textbf{Given:} A 200 mm diameter silicon wafer is used to fabricate
power management ICs with die dimensions of 3 mm x 4 mm. The edge
exclusion is 5 mm.

\textbf{Find:} (a) The usable wafer area, (b) the maximum number of dies
(before edge correction), (c) the estimated dies per wafer after edge
loss correction, and (d) the number of good dies at 85\% yield.

\textbf{Solution:}

\begin{enumerate}
\def\labelenumi{(\alph{enumi})}
\item
  Usable wafer area: A = pi x (D/2 - edge)\textsuperscript{2} = pi x
  (100 - 5)\textsuperscript{2} = pi x 95\textsuperscript{2} = pi x 9,025
  = \textbf{28,353 mm\textsuperscript{2}}
\item
  Die area: 3 x 4 = 12 mm\textsuperscript{2} Maximum dies: 28,353/12 =
  \textbf{2,363}
\item
  Edge loss correction: Die diagonal = sqrt(3\textsuperscript{2} +
  4\textsuperscript{2}) = sqrt(25) = 5.0 mm Edge dies lost
  \textasciitilde{} pi x (D - 2 x edge)/die\_diagonal = pi x 190/5.0 =
  119 Estimated dies: 2,363 - 119 = \textbf{\textasciitilde2,244 dies
  per wafer}
\item
  Good dies at 85\% yield: 2,244 x 0.85 = \textbf{\textasciitilde1,907
  good dies}
\end{enumerate}

\begin{center}\rule{0.5\linewidth}{0.5pt}\end{center}

\section{Problem 3.7.2}\label{problem-3.7.2}

\textbf{Given:} An advanced lithography system uses EUV at lambda = 13.5
nm with NA = 0.33 and k₁ = 0.32. A next-generation high-NA EUV tool uses
NA = 0.55 with the same wavelength and k₁ = 0.28.

\textbf{Find:} (a) The minimum feature size (CD\textsubscript{min}) for
each tool, (b) the minimum half-pitch (feature + space), and (c) the
transistor density improvement assuming density scales as
1/pitch\textsuperscript{2}.

\textbf{Solution:}

\begin{enumerate}
\def\labelenumi{(\alph{enumi})}
\tightlist
\item
  Rayleigh criterion: CD\textsubscript{min} = k₁ x lambda/NA
\end{enumerate}

Standard EUV: CD\textsubscript{min} = 0.32 x 13.5/0.33 = \textbf{13.1
nm} High-NA EUV: CD\textsubscript{min} = 0.28 x 13.5/0.55 = \textbf{6.87
nm}

\begin{enumerate}
\def\labelenumi{(\alph{enumi})}
\setcounter{enumi}{1}
\item
  Minimum half-pitch (assuming pitch = 2 x CD): Standard EUV: pitch = 2
  x 13.1 = \textbf{26.2 nm} High-NA EUV: pitch = 2 x 6.87 = \textbf{13.7
  nm}
\item
  Density improvement: Density ratio =
  (pitch\textsubscript{old}/pitch\textsubscript{new})\textsuperscript{2}
  = (26.2/13.7)\textsuperscript{2} = (1.912)\textsuperscript{2} =
  \textbf{3.65x}
\end{enumerate}

High-NA EUV enables approximately 3.65x higher transistor density,
supporting the transition from 3 nm to sub-2 nm nodes.

\begin{center}\rule{0.5\linewidth}{0.5pt}\end{center}

\section{Problem 3.7.3}\label{problem-3.7.3}

\textbf{Given:} Boron is implanted into an N-type silicon substrate
(N\textsubscript{D} = 5 x 10\textsuperscript{15} cm\textsuperscript{-3})
at an energy of 50 keV. At this energy, the projected range is
R\textsubscript{p} = 170 nm and the standard deviation is
delta\_R\textsubscript{p} = 60 nm. The implant dose is phi = 5 x
10\textsuperscript{13} cm\textsuperscript{-2}.

\textbf{Find:} (a) The peak boron concentration, (b) the junction depth
where the boron concentration equals the background N\textsubscript{D},
and (c) the surface concentration.

\textbf{Solution:}

\begin{enumerate}
\def\labelenumi{(\alph{enumi})}
\item
  Peak concentration (at x = R\textsubscript{p}) for a Gaussian profile:
  N\textsubscript{peak} = phi / (sqrt(2 x pi) x
  delta\_R\textsubscript{p}) = 5 x 10\textsuperscript{13} / (sqrt(2 x
  pi) x 60 x 10\textsuperscript{-7}) = 5 x 10\textsuperscript{13} /
  (2.507 x 6 x 10\textsuperscript{-6}) = 5 x 10\textsuperscript{13} /
  1.504 x 10\textsuperscript{-5} = \textbf{3.32 x 10\textsuperscript{18}
  cm\textsuperscript{-3}}
\item
  Junction depth where N(x\textsubscript{j}) = N\textsubscript{D} = 5 x
  10\textsuperscript{15}: N(x) = N\textsubscript{peak} x exp(-(x -
  R\textsubscript{p})\textsuperscript{2} / (2 x
  delta\_R\textsubscript{p}\textsuperscript{2})) 5 x
  10\textsuperscript{15} = 3.32 x 10\textsuperscript{18} x
  exp(-(x\textsubscript{j} - 170)\textsuperscript{2} / (2 x
  60\textsuperscript{2}))
\end{enumerate}

exp(-(x\textsubscript{j} - 170)\textsuperscript{2}/7,200) = 5 x
10\textsuperscript{15} / 3.32 x 10\textsuperscript{18} = 1.506 x
10\textsuperscript{-3}

-(x\textsubscript{j} - 170)\textsuperscript{2}/7,200 = ln(1.506 x
10\textsuperscript{-3}) = -6.499 (x\textsubscript{j} -
170)\textsuperscript{2} = 7,200 x 6.499 = 46,793 x\textsubscript{j} -
170 = +/- 216.3 nm

Taking the deeper junction: x\textsubscript{j} = 170 + 216 = \textbf{386
nm} (into the substrate) The shallow junction would be at 170 - 216 =
-46 nm (surface side, not physically meaningful as it is above the
surface).

\begin{enumerate}
\def\labelenumi{(\alph{enumi})}
\setcounter{enumi}{2}
\tightlist
\item
  Surface concentration (x = 0): N(0) = N\textsubscript{peak} x
  exp(-R\textsubscript{p}\textsuperscript{2} / (2 x
  delta\_R\textsubscript{p}\textsuperscript{2})) = 3.32 x
  10\textsuperscript{18} x exp(-170\textsuperscript{2} / (2 x
  60\textsuperscript{2})) = 3.32 x 10\textsuperscript{18} x
  exp(-28,900/7,200) = 3.32 x 10\textsuperscript{18} x exp(-4.014) =
  3.32 x 10\textsuperscript{18} x 0.01810 = \textbf{6.01 x
  10\textsuperscript{16} cm\textsuperscript{-3}}
\end{enumerate}

\begin{center}\rule{0.5\linewidth}{0.5pt}\end{center}

\section{Problem 3.7.4}\label{problem-3.7.4}

\textbf{Given:} A PECVD process deposits SiO\textsubscript{2} at a rate
of 80 nm/min. The target film thickness is 400 nm +/- 5\%. The
subsequent RIE etch process has an etch rate of 150 nm/min for
SiO\textsubscript{2} and 8 nm/min for the underlying silicon nitride
(Si\textsubscript{3}N\textsubscript{4}). A 15\% overetch is standard.

\textbf{Find:} (a) The deposition time, (b) the selectivity of
SiO\textsubscript{2} to Si\textsubscript{3}N\textsubscript{4}, (c) the
etch time with overetch, and (d) the
Si\textsubscript{3}N\textsubscript{4} loss during overetch.

\textbf{Solution:}

\begin{enumerate}
\def\labelenumi{(\alph{enumi})}
\item
  Deposition time: t\textsubscript{dep} = 400/80 = \textbf{5.0 minutes}
\item
  Selectivity: S =
  Rate(SiO\textsubscript{2})/Rate(Si\textsubscript{3}N\textsubscript{4})
  = 150/8 = \textbf{18.75:1}
\item
  Main etch time: t\textsubscript{etch} = 400/150 = 2.667 min Overetch
  time: 0.15 x 2.667 = 0.400 min Total: 2.667 + 0.400 = \textbf{3.07
  minutes}
\item
  Si\textsubscript{3}N\textsubscript{4} loss during overetch: Loss = 8 x
  0.400 = \textbf{3.2 nm}
\end{enumerate}

If the SiO\textsubscript{2} film is 5\% thicker (420 nm), the main etch
takes 420/150 = 2.80 min, and the overetch time becomes 0.15 x 2.80 =
0.42 min, with Si\textsubscript{3}N\textsubscript{4} loss of 8 x 0.42 =
3.4 nm. The high selectivity ensures minimal damage to the underlayer.

\begin{center}\rule{0.5\linewidth}{0.5pt}\end{center}

\section{Problem 3.7.5}\label{problem-3.7.5}

\textbf{Given:} A 5 nm GAA nanosheet CMOS process has a gate pitch of 48
nm and a minimum metal pitch of 28 nm. The SRAM bit cell area is 0.021
um\textsuperscript{2}. A processor requires 8 MB of L2 cache (8 x 8 x
10\textsuperscript{6} x 8 = 64 x 10\textsuperscript{6} bits, plus
\textasciitilde30\% overhead for control logic), and the logic portion
requires 15 billion transistors at a density of 310
MTr/mm\textsuperscript{2}.

\textbf{Find:} (a) The SRAM area for L2 cache, (b) the logic area, (c)
the total die area, and (d) the estimated die cost if the wafer cost is
\$18,000 for a 300 mm wafer (assume 90\% yield and use the die count
from a simplified calculation).

\textbf{Solution:}

\begin{enumerate}
\def\labelenumi{(\alph{enumi})}
\item
  Total SRAM bits with overhead: 64 x 10\textsuperscript{6} x 1.30 =
  83.2 x 10\textsuperscript{6} cells Each SRAM cell = 6 transistors, but
  area is per bit cell: A\textsubscript{SRAM} = 83.2 x
  10\textsuperscript{6} x 0.021 x 10\textsuperscript{-6}
  mm\textsuperscript{2} = 83.2 x 10\textsuperscript{6} x 2.1 x
  10\textsuperscript{-8} = \textbf{1.75 mm\textsuperscript{2}}
\item
  Logic area: A\textsubscript{logic} = 15 x 10\textsuperscript{9} / (310
  x 10\textsuperscript{6}) = \textbf{48.4 mm\textsuperscript{2}}
\item
  Total die area (logic + SRAM + I/O and misc \textasciitilde20\%
  overhead): A\textsubscript{total} = (48.4 + 1.75) x 1.20 = 50.15 x
  1.20 = \textbf{60.2 mm\textsuperscript{2}}
\item
  Dies per 300 mm wafer: Usable area: pi x (150 - 3)\textsuperscript{2}
  = 67,929 mm\textsuperscript{2} Gross dies: 67,929/60.2 = 1,128 Edge
  loss: pi x 294/sqrt(2 x 60.2) \textasciitilde{} pi x 294/10.96
  \textasciitilde{} 84 Net dies: 1,128 - 84 = 1,044 Good dies at 90\%:
  1,044 x 0.90 = 940
\end{enumerate}

Die cost: \$18,000/940 = \textbf{\$19.15 per die}

\begin{center}\rule{0.5\linewidth}{0.5pt}\end{center}

\section{Problem 3.7.6}\label{problem-3.7.6}

\textbf{Given:} A chiplet-based processor uses a 3 nm compute chiplet
(80 mm\textsuperscript{2}, \$35/die), two 5 nm I/O chiplets (40
mm\textsuperscript{2} each, \$12/die each), and an HBM memory stack
(\$25/stack, 4 stacks). These are assembled on a silicon interposer (500
mm\textsuperscript{2}, \$15). The assembly and test cost is \$20 per
package. The equivalent monolithic design would be a single 250
mm\textsuperscript{2} die at 3 nm (\$95/die) with assembly cost of \$8.

\textbf{Find:} (a) The total cost of the chiplet solution, (b) the total
cost of the monolithic solution, (c) the yield advantage of the chiplet
approach (assuming 3 nm yield = 75\% for 80 mm\textsuperscript{2} and
55\% for 250 mm\textsuperscript{2}), and (d) the effective cost per good
package for each approach.

\textbf{Solution:}

\begin{enumerate}
\def\labelenumi{(\alph{enumi})}
\item
  Chiplet solution cost per package: Compute chiplet: \$35 I/O chiplets:
  2 x \$12 = \$24 HBM stacks: 4 x \$25 = \$100 Interposer: \$15 Assembly
  + test: \$20 \textbf{Total: \$194 per package}
\item
  Monolithic solution: Die: \$95 Assembly + test: \$8 \textbf{Total:
  \$103 per package}
\item
  The die costs above already assume testing, but not yield on final
  assembly. If the chiplet assembly yield is 95\% and monolithic
  assembly yield is 98\%:
\end{enumerate}

Chiplet effective cost: \$194/0.95 = \textbf{\$204 per good package}
Monolithic effective cost: \$103/0.98 = \textbf{\$105 per good package}

\begin{enumerate}
\def\labelenumi{(\alph{enumi})}
\setcounter{enumi}{3}
\tightlist
\item
  However, the given die costs may not include yield. If we factor in
  raw die costs:
\end{enumerate}

Chiplet: The 3 nm die at 75\% yield has a raw cost already factored in
at \$35. The monolithic 3 nm die at 55\% yield costs more per good die.

If raw wafer cost is \$18,000 and the monolithic die gets 67,929/250
\textasciitilde{} 230 gross dies (minus \textasciitilde30 edge = 200
net), then at 55\% yield = 110 good dies, raw cost = \$18,000/110 =
\$163/die (not \$95). This demonstrates the \textbf{chiplet yield
advantage}: smaller dies have exponentially better yield, making the
chiplet approach more cost-effective for large designs despite higher
assembly complexity.

\chapter{Chapter 4 --- Section 4.1: Open
Loop}\label{chapter-4-section-4.1-open-loop}

Practice problems covering open-loop control system analysis and
steady-state response.

\begin{center}\rule{0.5\linewidth}{0.5pt}\end{center}

\section{Problem 4.1.1}\label{problem-4.1.1}

\textbf{Given:} An open-loop conveyor belt speed controller has a
forward path transfer function G(s) = 120 / (s + 15). A step input of
magnitude 2 is applied as R(s) = 2/s.

\textbf{Find:} (a) The output C(s), (b) the time-domain response c(t),
(c) the steady-state output, and (d) the time constant.

\textbf{Solution:}

\begin{enumerate}
\def\labelenumi{(\alph{enumi})}
\tightlist
\item
  C(s) = G(s) x R(s) = 120 / {[}s(s + 15){]} x 2 = 240 / {[}s(s + 15){]}
\end{enumerate}

Partial fractions: 240/{[}s(s + 15){]} = A/s + B/(s + 15) A = 240/15 =
16, B = 240/(-15) = -16 C(s) = \textbf{16/s - 16/(s + 15)}

\begin{enumerate}
\def\labelenumi{(\alph{enumi})}
\setcounter{enumi}{1}
\item
  Inverse Laplace: c(t) = \textbf{16(1 - e\textsuperscript{-15t})} for t
  \textgreater= 0
\item
  Steady-state: c(infinity) = \textbf{16} The desired output for a step
  of magnitude 2 would ideally be 2, but the open-loop gain is G(0) =
  120/15 = 8, so the output is 8 x 2 = 16.
\item
  Time constant: tau = 1/15 = \textbf{0.0667 s = 66.7 ms}
\end{enumerate}

\begin{center}\rule{0.5\linewidth}{0.5pt}\end{center}

\section{Problem 4.1.2}\label{problem-4.1.2}

\textbf{Given:} An open-loop heating system has G(s) = 30 / {[}(s +
0.5)(s + 3){]}. A unit step input R(s) = 1/s is applied.

\textbf{Find:} (a) The steady-state output, (b) the steady-state error
if the desired output is 1, (c) the time-domain response, and (d) how a
10\% increase in the plant gain (from 30 to 33) affects the steady-state
output.

\textbf{Solution:}

\begin{enumerate}
\def\labelenumi{(\alph{enumi})}
\item
  Steady-state (Final Value Theorem): c\textsubscript{ss} =
  lim(s-\textgreater0) s x G(s) x R(s) = lim(s-\textgreater0) s x
  30/{[}s(s + 0.5)(s + 3){]} = 30/(0.5 x 3) = \textbf{20}
\item
  Steady-state error: e\textsubscript{ss} = 1 - 20 = \textbf{-19}
  (massive gain error; the system overshoots by 1,900\%)
\item
  Partial fractions of C(s) = 30/{[}s(s + 0.5)(s + 3){]}: A/s + B/(s +
  0.5) + C/(s + 3) A = 30/(0.5 x 3) = 20 B = 30/((-0.5)(3 - 0.5)) =
  30/(-0.5 x 2.5) = -24 C = 30/((-3)(-3 + 0.5)) = 30/((-3)(-2.5)) =
  30/7.5 = 4
\end{enumerate}

c(t) = \textbf{20 - 24e\textsuperscript{-0.5t} +
4e\textsuperscript{-3t}} for t \textgreater= 0

\begin{enumerate}
\def\labelenumi{(\alph{enumi})}
\setcounter{enumi}{3}
\tightlist
\item
  With gain increased 10\%: c\textsubscript{ss} = 33/(0.5 x 3) =
  \textbf{22} The 10\% gain change produces a 10\% output change (from
  20 to 22), demonstrating the sensitivity of open-loop systems to
  parameter variations.
\end{enumerate}

\begin{center}\rule{0.5\linewidth}{0.5pt}\end{center}

\section{Problem 4.1.3}\label{problem-4.1.3}

\textbf{Given:} An open-loop motor positioning system has G(s) = 200 /
{[}s(s + 20){]}. The input is a ramp r(t) = 5t, so R(s) =
5/s\textsuperscript{2}.

\textbf{Find:} (a) The output in the Laplace domain, (b) the
steady-state output, and (c) whether the system can track the ramp
input.

\textbf{Solution:}

\begin{enumerate}
\def\labelenumi{(\alph{enumi})}
\tightlist
\item
  C(s) = G(s) x R(s) = 200 x 5 / {[}s\textsuperscript{2}(s + 20) x s{]}
  = 1,000 / {[}s\textsuperscript{3}(s + 20){]}
\end{enumerate}

Wait --- R(s) = 5/s\textsuperscript{2}, so: C(s) = 200 / {[}s(s + 20){]}
x 5/s\textsuperscript{2} = 1,000 / {[}s\textsuperscript{3}(s + 20){]}

\begin{enumerate}
\def\labelenumi{(\alph{enumi})}
\setcounter{enumi}{1}
\tightlist
\item
  Applying the Final Value Theorem: c\textsubscript{ss} =
  lim(s-\textgreater0) s x C(s) = lim(s-\textgreater0) 1,000 /
  {[}s\textsuperscript{2}(s + 20){]}
\end{enumerate}

As s -\textgreater{} 0, this approaches infinity. The
\textbf{steady-state output is unbounded} --- it grows without limit.

\begin{enumerate}
\def\labelenumi{(\alph{enumi})}
\setcounter{enumi}{2}
\tightlist
\item
  The system \textbf{cannot} track the ramp in a meaningful open-loop
  sense. With G(s) containing an integrator (1/s factor), the open-loop
  output to a ramp input grows as t\textsuperscript{2}/2, diverging from
  the linear ramp. This illustrates that open-loop systems with
  integrators are inherently unsuitable for tracking inputs without
  feedback to regulate the output.
\end{enumerate}

\chapter{Chapter 4 --- Section 4.2: Closed
Loop}\label{chapter-4-section-4.2-closed-loop}

Practice problems covering closed-loop transfer functions, feedback
effects, and steady-state error analysis.

\begin{center}\rule{0.5\linewidth}{0.5pt}\end{center}

\section{Problem 4.2.1}\label{problem-4.2.1}

\textbf{Given:} A unity feedback system has G(s) = 50 / {[}(s + 1)(s +
10){]}. The reference input is a unit step.

\textbf{Find:} (a) The closed-loop transfer function T(s), (b) the
closed-loop poles, (c) the steady-state output, and (d) the steady-state
error.

\textbf{Solution:}

\begin{enumerate}
\def\labelenumi{(\alph{enumi})}
\item
  T(s) = G(s)/(1 + G(s)) = 50/{[}(s + 1)(s + 10){]} / {[}1 + 50/((s +
  1)(s + 10)){]} = 50 / {[}(s + 1)(s + 10) + 50{]} = 50 /
  {[}s\textsuperscript{2} + 11s + 10 + 50{]} = \textbf{50 /
  (s\textsuperscript{2} + 11s + 60)}
\item
  Poles: s\textsuperscript{2} + 11s + 60 = 0 s = (-11 +/- sqrt(121 -
  240))/2 = (-11 +/- sqrt(-119))/2 = (-11 +/- j10.91)/2 Poles: \textbf{s
  = -5.5 +/- j5.45} (complex conjugate, stable)
\item
  Steady-state: T(0) = 50/60 = \textbf{0.833}
\item
  e\textsubscript{ss} = 1 - T(0) = 1 - 0.833 = \textbf{0.167} (16.7\%
  error)
\end{enumerate}

This is a Type 0 system (no integrator in G(s)), so finite steady-state
error to a step is expected.

\begin{center}\rule{0.5\linewidth}{0.5pt}\end{center}

\section{Problem 4.2.2}\label{problem-4.2.2}

\textbf{Given:} A closed-loop system has a forward path G(s) = K /
{[}s(s + 8){]} and feedback H(s) = 0.5. The system must have a
steady-state error of less than 2\% for a unit step input.

\textbf{Find:} (a) The closed-loop transfer function, (b) the minimum
value of K to meet the error specification, and (c) the closed-loop
poles at that K.

\textbf{Solution:}

\begin{enumerate}
\def\labelenumi{(\alph{enumi})}
\item
  T(s) = G(s)/(1 + G(s)H(s)) = {[}K/(s(s + 8)){]} / {[}1 + 0.5K/(s(s +
  8)){]} = \textbf{K / (s\textsuperscript{2} + 8s + 0.5K)}
\item
  For a unity-feedback-equivalent analysis: e\textsubscript{ss} = 1/(1 +
  K\textsubscript{p}) where K\textsubscript{p} = lim(s-\textgreater0)
  G(s)H(s) = lim(s-\textgreater0) 0.5K/{[}s(s + 8){]}
\end{enumerate}

Since G(s)H(s) has a pole at s = 0, this is Type 1, and
K\textsubscript{p} = infinity, giving e\textsubscript{ss} = 0 for a
step.

Actually, for a non-unity feedback system, we must use:
e\textsubscript{ss} = lim(s-\textgreater0) s x R(s) / (1 + G(s)H(s))

For unit step R(s) = 1/s: e\textsubscript{ss} = lim(s-\textgreater0)
1/(1 + G(s)H(s)) = lim(s-\textgreater0) 1/(1 + 0.5K/{[}s(s+8){]})

As s -\textgreater{} 0, G(s)H(s) -\textgreater{} infinity, so
e\textsubscript{ss} = \textbf{0} regardless of K (as long as K
\textgreater{} 0).

However, the actual output is Y(s) = T(s) x R(s), and the steady-state
output is: y\textsubscript{ss} = T(0) = K/(0.5K) = 2.0

The reference was 1, so the output is 2.0, meaning the error based on
the reference is 1 - 2 = -1 (overshoot in DC). With non-unity feedback,
the reference must be scaled. If we define error as r - H x y: e = r -
0.5y -\textgreater{} at steady state: 0 = 1 - 0.5 x y\textsubscript{ss},
so y\textsubscript{ss} = 2.0 and e\textsubscript{ss} = \textbf{0}.

The system tracks with zero steady-state error for any K \textgreater{}
0 due to the integrator.

\begin{enumerate}
\def\labelenumi{(\alph{enumi})}
\setcounter{enumi}{2}
\tightlist
\item
  For stability, require K \textgreater{} 0. Choose K = 50 for good
  transient response: s\textsuperscript{2} + 8s + 25 = 0 s = (-8 +/-
  sqrt(64 - 100))/2 = (-8 +/- j6)/2 = \textbf{-4 +/- j3}
\end{enumerate}

Natural frequency omega\textsubscript{n} = sqrt(25) = 5 rad/s, damping
zeta = 4/5 = 0.8.

\begin{center}\rule{0.5\linewidth}{0.5pt}\end{center}

\section{Problem 4.2.3}\label{problem-4.2.3}

\textbf{Given:} A unity feedback system has G(s) = 200(s + 5) / {[}s(s +
10)(s + 20){]}. A unit step input is applied.

\textbf{Find:} (a) The system type, (b) the position error constant
K\textsubscript{p}, (c) the steady-state error to a unit step, (d) the
velocity error constant K\textsubscript{v}, and (e) the steady-state
error to a unit ramp.

\textbf{Solution:}

\begin{enumerate}
\def\labelenumi{(\alph{enumi})}
\item
  G(s) has one free s in the denominator, so this is a \textbf{Type 1}
  system.
\item
  K\textsubscript{p} = lim(s-\textgreater0) G(s) = lim(s-\textgreater0)
  200(s + 5)/{[}s(s + 10)(s + 20){]} = infinity (due to the 1/s factor)
\item
  e\textsubscript{ss}(step) = 1/(1 + K\textsubscript{p}) = 1/(1 +
  infinity) = \textbf{0} (zero steady-state error to step)
\item
  K\textsubscript{v} = lim(s-\textgreater0) s x G(s) =
  lim(s-\textgreater0) 200(s + 5)/{[}(s + 10)(s + 20){]} = 200(5)/(10 x
  20) = 1,000/200 = \textbf{5}
\item
  e\textsubscript{ss}(ramp) = 1/K\textsubscript{v} = 1/5 = \textbf{0.2}
  (the system lags behind the ramp by 0.2 units)
\end{enumerate}

\chapter{Chapter 4 --- Section 4.3: Control
Signals}\label{chapter-4-section-4.3-control-signals}

Practice problems covering step, ramp, impulse, and sinusoidal input
responses.

\begin{center}\rule{0.5\linewidth}{0.5pt}\end{center}

\section{Problem 4.3.1}\label{problem-4.3.1}

\textbf{Given:} A first-order system has the transfer function G(s) = 15
/ (s + 5). A unit step input is applied.

\textbf{Find:} (a) The time-domain step response, (b) the time constant,
(c) the steady-state value, (d) the 2\% settling time, and (e) the
10-90\% rise time.

\textbf{Solution:} Rewrite in standard form: G(s) = 3 / (0.2s + 1), so K
= 3, tau = 0.2 s.

\begin{enumerate}
\def\labelenumi{(\alph{enumi})}
\item
  C(s) = 15/{[}s(s + 5){]} = 3/s - 3/(s + 5) c(t) = \textbf{3(1 -
  e\textsuperscript{-5t})} for t \textgreater= 0
\item
  Time constant: tau = 1/5 = \textbf{0.2 s}
\item
  Steady-state: c(infinity) = \textbf{3}
\item
  Settling time (2\%): t\textsubscript{s} = 4 x tau = 4 x 0.2 =
  \textbf{0.8 s}
\item
  Rise time: t\textsubscript{r} = 2.2 x tau = 2.2 x 0.2 = \textbf{0.44
  s}
\end{enumerate}

\begin{center}\rule{0.5\linewidth}{0.5pt}\end{center}

\section{Problem 4.3.2}\label{problem-4.3.2}

\textbf{Given:} A unity feedback system has G(s) = 500 / {[}s(s + 5)(s +
50){]}. A ramp input r(t) = 2t is applied.

\textbf{Find:} (a) The system type, (b) the velocity error constant
K\textsubscript{v}, and (c) the steady-state error.

\textbf{Solution:}

\begin{enumerate}
\def\labelenumi{(\alph{enumi})}
\item
  One free integrator in G(s), so this is a \textbf{Type 1} system.
\item
  K\textsubscript{v} = lim(s-\textgreater0) s x G(s) =
  lim(s-\textgreater0) 500/{[}(s + 5)(s + 50){]} = 500/(5 x 50) =
  \textbf{2.0}
\item
  For ramp input r(t) = 2t, R(s) = 2/s\textsuperscript{2}:
  e\textsubscript{ss} = A/K\textsubscript{v} where A = 2 (ramp slope)
  e\textsubscript{ss} = 2/2.0 = \textbf{1.0}
\end{enumerate}

The system lags behind the ramp by a constant 1.0 unit.

\begin{center}\rule{0.5\linewidth}{0.5pt}\end{center}

\section{Problem 4.3.3}\label{problem-4.3.3}

\textbf{Given:} A system has the transfer function G(s) = 24 / {[}(s +
2)(s + 4)(s + 6){]}. A unit impulse is applied.

\textbf{Find:} (a) The impulse response g(t) using partial fractions,
and (b) the peak value and time of the peak.

\textbf{Solution:}

\begin{enumerate}
\def\labelenumi{(\alph{enumi})}
\tightlist
\item
  For an impulse input, C(s) = G(s) x 1 = 24/{[}(s + 2)(s + 4)(s +
  6){]}.
\end{enumerate}

Partial fractions: A/(s + 2) + B/(s + 4) + C/(s + 6)

A = 24/{[}(4 - 2)(6 - 2){]} = 24/{[}2 x 4{]} = 3 B = 24/{[}(-4 + 2)(6 -
4){]} = 24/{[}(-2)(2){]} = -6 C = 24/{[}(-6 + 2)(-6 + 4){]} =
24/{[}(-4)(-2){]} = 3

g(t) = \textbf{3e\textsuperscript{-2t} - 6e\textsuperscript{-4t} +
3e\textsuperscript{-6t}} for t \textgreater= 0

\begin{enumerate}
\def\labelenumi{(\alph{enumi})}
\setcounter{enumi}{1}
\tightlist
\item
  At t = 0: g(0) = 3 - 6 + 3 = 0. Find peak by setting dg/dt = 0: dg/dt
  = -6e\textsuperscript{-2t} + 24e\textsuperscript{-4t} -
  18e\textsuperscript{-6t} = 0
\end{enumerate}

Dividing by -6e\textsuperscript{-6t}: e\textsuperscript{4t} -
4e\textsuperscript{2t} + 3 = 0 Let u = e\textsuperscript{2t}:
u\textsuperscript{2} - 4u + 3 = 0 -\textgreater{} (u - 1)(u - 3) = 0 u =
1 (t = 0) or u = 3 (t = ln(3)/2 = 0.549 s)

At t = 0.549 s: g(0.549) = 3 x e\textsuperscript{-1.099} - 6 x
e\textsuperscript{-2.197} + 3 x e\textsuperscript{-3.296} = 3 x 0.333 -
6 x 0.111 + 3 x 0.037 = 1.0 - 0.667 + 0.111 = \textbf{0.444}

Peak value is \textbf{0.444} at t = \textbf{0.549 s}.

\begin{center}\rule{0.5\linewidth}{0.5pt}\end{center}

\section{Problem 4.3.4}\label{problem-4.3.4}

\textbf{Given:} A stable system has G(s) = 20 / {[}(s + 1)(s + 5){]}. A
sinusoidal input r(t) = 4 sin(3t) is applied.

\textbf{Find:} (a) The magnitude \textbar G(j3)\textbar{} and phase
angle, (b) the steady-state output, and (c) the output amplitude and
phase at omega = 10 rad/s for the same input amplitude.

\textbf{Solution:}

\begin{enumerate}
\def\labelenumi{(\alph{enumi})}
\tightlist
\item
  G(j3) = 20/{[}(j3 + 1)(j3 + 5){]} = 20/{[}(1 + j3)(5 + j3){]} =
  20/{[}(5 - 9) + j(3 + 15){]} = 20/{[}-4 + j18{]}
\end{enumerate}

\textbar G(j3)\textbar{} = 20/sqrt(16 + 324) = 20/sqrt(340) = 20/18.44 =
\textbf{1.085}

Phase: angle(-4 + j18) = 180 degrees - arctan(18/4) = 180 - 77.3 = 102.7
degrees Since the denominator has positive imaginary and negative real
parts (second quadrant): Phase of G = -102.7 degrees

Alternatively, compute directly: Phase = -arctan(3/1) - arctan(3/5) =
-71.57 - 30.96 = \textbf{-102.5 degrees}

\begin{enumerate}
\def\labelenumi{(\alph{enumi})}
\setcounter{enumi}{1}
\item
  Steady-state output: y\textsubscript{ss}(t) = 4 x 1.085 x sin(3t -
  102.5 degrees) = \textbf{4.34 sin(3t - 102.5 degrees)}
\item
  At omega = 10: G(j10) = 20/{[}(1 + j10)(5 + j10){]} = 20/{[}(5 - 100)
  + j(10 + 50){]} = 20/{[}-95 + j60{]} \textbar G(j10)\textbar{} =
  20/sqrt(9,025 + 3,600) = 20/sqrt(12,625) = 20/112.4 = \textbf{0.178}
  Phase = -arctan(10/1) - arctan(10/5) = -84.3 - 63.4 = \textbf{-147.7
  degrees}
\end{enumerate}

Output: y\textsubscript{ss} = 4 x 0.178 x sin(10t - 147.7 degrees) =
\textbf{0.712 sin(10t - 147.7 degrees)}

The amplitude drops from 4.34 at omega = 3 to 0.712 at omega = 10,
showing the low-pass filtering effect.

\begin{center}\rule{0.5\linewidth}{0.5pt}\end{center}

\section{Problem 4.3.5}\label{problem-4.3.5}

\textbf{Given:} A second-order system has G(s) = 100 /
(s\textsuperscript{2} + 6s + 100). A unit step input is applied.

\textbf{Find:} (a) The natural frequency and damping ratio, (b) the
damped frequency, (c) the peak time, (d) the percent overshoot, and (e)
the 2\% settling time.

\textbf{Solution:}

\begin{enumerate}
\def\labelenumi{(\alph{enumi})}
\item
  Comparing with
  omega\textsubscript{n}\textsuperscript{2}/(s\textsuperscript{2} +
  2\emph{zeta}omega\textsubscript{n}\emph{s +
  omega\textsubscript{n}\textsuperscript{2}):
  omega\textsubscript{n}\textsuperscript{2} = 100 -\textgreater{}
  omega\textsubscript{n} = \textbf{10 rad/s}
  2}zeta*omega\textsubscript{n} = 6 -\textgreater{} zeta = 6/20 =
  \textbf{0.3} (underdamped)
\item
  omega\textsubscript{d} = omega\textsubscript{n} x sqrt(1 -
  zeta\textsuperscript{2}) = 10 x sqrt(1 - 0.09) = 10 x 0.9539 =
  \textbf{9.54 rad/s}
\item
  Peak time: t\textsubscript{p} = pi/omega\textsubscript{d} = pi/9.54 =
  \textbf{0.329 s}
\item
  \%OS = 100 x e\textsuperscript{-zeta\emph{pi/sqrt(1 -
  zeta\textsuperscript{2}) = 100 x e\textsuperscript{-0.3}}pi/0.9539} =
  100 x e\textsuperscript{-0.988} = 100 x 0.3730 = \textbf{37.3\%}
\item
  t\textsubscript{s} = 4/(zeta x omega\textsubscript{n}) = 4/(0.3 x 10)
  = 4/3 = \textbf{1.33 s}
\end{enumerate}

\begin{center}\rule{0.5\linewidth}{0.5pt}\end{center}

\section{Problem 4.3.6}\label{problem-4.3.6}

\textbf{Given:} A unity feedback system with G(s) = K/{[}s(s + 4){]}
requires a 2\% settling time of 2 seconds and a percent overshoot no
greater than 20\%.

\textbf{Find:} (a) The required damping ratio for the overshoot
specification, (b) the required sigma = zeta x omega\textsubscript{n}
for the settling time, (c) the required omega\textsubscript{n}, and (d)
the value of K.

\textbf{Solution:}

\begin{enumerate}
\def\labelenumi{(\alph{enumi})}
\item
  \%OS = 20\%: 20 = 100 x e\textsuperscript{-zeta\emph{pi/sqrt(1 -
  zeta\textsuperscript{2}) ln(0.20) = -zeta}pi/sqrt(1 - zeta2}) -1.609 =
  -zeta\emph{pi/sqrt(1 - zeta\textsuperscript{2}) zeta}pi/sqrt(1 -
  zeta\textsuperscript{2}) = 1.609 zeta\textsuperscript{2} x
  pi\textsuperscript{2} = 1.609\textsuperscript{2} x (1 -
  zeta\textsuperscript{2}) = 2.589 - 2.589*zeta\textsuperscript{2}
  zeta\textsuperscript{2}(9.870 + 2.589) = 2.589 zeta\textsuperscript{2}
  = 2.589/12.459 = 0.2078 zeta = \textbf{0.456}
\item
  sigma = 4/t\textsubscript{s} = 4/2 = \textbf{2} (for 2\% criterion:
  t\textsubscript{s} = 4/sigma)
\item
  sigma = zeta x omega\textsubscript{n} -\textgreater{}
  omega\textsubscript{n} = sigma/zeta = 2/0.456 = \textbf{4.386 rad/s}
\item
  Closed-loop: T(s) = K/(s\textsuperscript{2} + 4s + K)
  omega\textsubscript{n}\textsuperscript{2} = K -\textgreater{} K =
  4.386\textsuperscript{2} = \textbf{19.2}
\end{enumerate}

Verify: 2\emph{zeta}omega\textsubscript{n} = 4 -\textgreater{} zeta =
4/(2 x 4.386) = 0.456. Confirmed.

\chapter{Chapter 4 --- Section 4.4: Transfer Functions and Block
Diagrams}\label{chapter-4-section-4.4-transfer-functions-and-block-diagrams}

Practice problems covering transfer function representation, poles and
zeros, DC gain, block diagram algebra, series/parallel/feedback
reduction, signal flow graphs, and Mason's gain formula.

\begin{center}\rule{0.5\linewidth}{0.5pt}\end{center}

\section{Problem 4.4.1}\label{problem-4.4.1}

\textbf{Given:} A system has the transfer function H(s) = 120(s + 4) /
(s² + 12s + 36).

\textbf{Find:} (a) The poles and zeros, (b) the natural frequency and
damping ratio, (c) the DC gain, and (d) whether the system is stable.

\textbf{Solution:}

\begin{enumerate}
\def\labelenumi{(\alph{enumi})}
\item
  Zero: s + 4 = 0 → s = \textbf{-4} (one real zero). Poles: s² + 12s +
  36 = (s + 6)² = 0 → s = \textbf{-6} (repeated real pole, multiplicity
  2).
\item
  Comparing s² + 2ζω\textsubscript{n}s + ω\textsubscript{n}² = s² + 12s
  + 36: ω\textsubscript{n} = √36 = \textbf{6 rad/s} 2ζω\textsubscript{n}
  = 12 → ζ = 12 / (2 × 6) = \textbf{1.0} (critically damped)
\item
  DC gain: H(0) = 120(4) / 36 = 480 / 36 = \textbf{13.33}
\item
  Both poles are at s = -6 (negative real part). The system is
  \textbf{stable}.
\end{enumerate}

\begin{center}\rule{0.5\linewidth}{0.5pt}\end{center}

\section{Problem 4.4.2}\label{problem-4.4.2}

\textbf{Given:} A system has the transfer function G(s) = 200 / (s³ +
9s² + 26s + 24).

\textbf{Find:} (a) The poles of the system, (b) the DC gain, and (c) the
partial fraction expansion of G(s).

\textbf{Solution:}

\begin{enumerate}
\def\labelenumi{(\alph{enumi})}
\item
  Factor the denominator. Test s = -1: (-1)³ + 9(-1)² + 26(-1) + 24 = -1
  + 9 - 26 + 24 = 6 (not a root). Test s = -2: (-2)³ + 9(-2)² + 26(-2) +
  24 = -8 + 36 - 52 + 24 = 0 (root). Factor out (s + 2): s³ + 9s² + 26s
  + 24 = (s + 2)(s² + 7s + 12) = (s + 2)(s + 3)(s + 4). Poles: s =
  \textbf{-2, -3, -4} (all real, all stable).
\item
  DC gain: G(0) = 200 / (2 × 3 × 4) = 200 / 24 = \textbf{8.33}
\item
  G(s) = A/(s + 2) + B/(s + 3) + C/(s + 4). A = 200 / {[}(s + 3)(s +
  4){]} at s = -2 = 200 / (1 × 2) = 100 B = 200 / {[}(s + 2)(s + 4){]}
  at s = -3 = 200 / ((-1)(1)) = -200 C = 200 / {[}(s + 2)(s + 3){]} at s
  = -4 = 200 / ((-2)(-1)) = 100
\end{enumerate}

G(s) = \textbf{100/(s + 2) - 200/(s + 3) + 100/(s + 4)}

\begin{center}\rule{0.5\linewidth}{0.5pt}\end{center}

\section{Problem 4.4.3}\label{problem-4.4.3}

\textbf{Given:} A control system has three blocks in the forward path:
G₁(s) = 5, G₂(s) = 1/(s + 3), and G₃(s) = 1/(s + 8). The system uses
unity feedback H(s) = 1.

\textbf{Find:} (a) The open-loop transfer function, (b) the closed-loop
transfer function, (c) the closed-loop poles, and (d) the steady-state
error to a unit step input.

\textbf{Solution:}

\begin{enumerate}
\def\labelenumi{(\alph{enumi})}
\item
  Open-loop: G(s) = G₁G₂G₃ = 5 / {[}(s + 3)(s + 8){]} = \textbf{5 / (s²
  + 11s + 24)}
\item
  Closed-loop: T(s) = G(s) / (1 + G(s)) = 5 / (s² + 11s + 24 + 5) =
  \textbf{5 / (s² + 11s + 29)}
\item
  Poles: s = (-11 ± √(121 - 116)) / 2 = (-11 ± √5) / 2 = (-11 ± 2.236) /
  2 s₁ = \textbf{-4.382}, s₂ = \textbf{-6.618} (both real, stable,
  overdamped)
\item
  This is a Type 0 system. K\textsubscript{p} = G(0) = 5/24 = 0.2083.
  e\textsubscript{ss} = 1 / (1 + K\textsubscript{p}) = 1 / 1.2083 =
  \textbf{0.828} (82.8\% steady-state error)
\end{enumerate}

\begin{center}\rule{0.5\linewidth}{0.5pt}\end{center}

\section{Problem 4.4.4}\label{problem-4.4.4}

\textbf{Given:} A feedback control system has forward path G(s) = K /
{[}s(s + 6){]} and feedback path H(s) = 2.

\textbf{Find:} (a) The closed-loop transfer function, (b) the value of K
that produces a damping ratio ζ = 0.5, (c) the natural frequency at that
K, and (d) the steady-state error to a unit step for the non-unity
feedback system.

\textbf{Solution:}

\begin{enumerate}
\def\labelenumi{(\alph{enumi})}
\item
  Closed-loop: T(s) = G(s) / (1 + G(s)H(s)) = {[}K / s(s + 6){]} / {[}1
  + 2K / s(s + 6){]} T(s) = \textbf{K / (s² + 6s + 2K)}
\item
  Comparing with s² + 2ζω\textsubscript{n}s + ω\textsubscript{n}²:
  ω\textsubscript{n}² = 2K and 2ζω\textsubscript{n} = 6.
  ω\textsubscript{n} = 6 / (2 × 0.5) = 6 rad/s. 2K = ω\textsubscript{n}²
  = 36 → K = \textbf{18}
\item
  ω\textsubscript{n} = \textbf{6 rad/s}
\item
  For a non-unity feedback system, e\textsubscript{ss} = 1 - T(0) where
  T(0) = K / (2K) = 1/2. e\textsubscript{ss} = 1 - 0.5 = \textbf{0.5}
  (50\% error due to the feedback gain of 2)
\end{enumerate}

\begin{center}\rule{0.5\linewidth}{0.5pt}\end{center}

\section{Problem 4.4.5}\label{problem-4.4.5}

\textbf{Given:} A control system has two forward paths and two feedback
loops. The forward path contains G₁(s) = 4/(s + 1) and G₂(s) = 3/(s +
5). There is a minor loop with feedback H₁(s) = 0.5 around G₂, and unity
feedback H₂(s) = 1 around the entire system.

\textbf{Find:} (a) The inner-loop closed-loop transfer function, (b) the
overall open-loop transfer function, (c) the overall closed-loop
transfer function, and (d) the DC gain.

\textbf{Solution:}

\begin{enumerate}
\def\labelenumi{(\alph{enumi})}
\item
  Inner loop (G₂ with feedback H₁): G₂\textsubscript{CL}(s) = G₂ / (1 +
  G₂H₁) = {[}3/(s + 5){]} / {[}1 + 1.5/(s + 5){]} = 3 / (s + 5 + 1.5) =
  \textbf{3 / (s + 6.5)}
\item
  Overall open-loop: G\textsubscript{OL}(s) = G₁ × G₂\textsubscript{CL}
  = {[}4/(s + 1){]} × {[}3/(s + 6.5){]} = \textbf{12 / {[}(s + 1)(s +
  6.5){]}}
\item
  Overall closed-loop: T(s) = G\textsubscript{OL} / (1 +
  G\textsubscript{OL}) = 12 / {[}(s + 1)(s + 6.5) + 12{]} = 12 / (s² +
  7.5s + 6.5 + 12) = \textbf{12 / (s² + 7.5s + 18.5)}
\item
  DC gain: T(0) = 12 / 18.5 = \textbf{0.649}
\end{enumerate}

\begin{center}\rule{0.5\linewidth}{0.5pt}\end{center}

\section{Problem 4.4.6}\label{problem-4.4.6}

\textbf{Given:} A second-order system has poles at s = -3 + j4 and s =
-3 - j4, a zero at s = -10, and a DC gain of 5.

\textbf{Find:} (a) The transfer function H(s), (b) ω\textsubscript{n}
and ζ, (c) the damped natural frequency, and (d) the expected percent
overshoot (ignoring the zero's effect).

\textbf{Solution:}

\begin{enumerate}
\def\labelenumi{(\alph{enumi})}
\item
  Denominator: (s + 3 - j4)(s + 3 + j4) = (s + 3)² + 16 = s² + 6s + 25.
  Transfer function form: H(s) = K(s + 10) / (s² + 6s + 25). DC gain:
  H(0) = K(10) / 25 = 5 → K = 12.5. H(s) = \textbf{12.5(s + 10) / (s² +
  6s + 25)}
\item
  From the denominator: ω\textsubscript{n}² = 25 → ω\textsubscript{n} =
  \textbf{5 rad/s}. 2ζω\textsubscript{n} = 6 → ζ = 6 / 10 = \textbf{0.6}
\item
  ω\textsubscript{d} = ω\textsubscript{n}√(1 - ζ²) = 5√(1 - 0.36) = 5 ×
  0.8 = \textbf{4 rad/s}
\item
  \%OS = 100 × e\textsuperscript{-ζπ/√(1-ζ²)} = 100 ×
  e\textsuperscript{-0.6π/0.8} = 100 × e\textsuperscript{-2.356} = 100 ×
  0.0948 = \textbf{9.48\%}
\end{enumerate}

Note: The zero at s = -10 will increase the actual overshoot slightly
above this value.

\begin{center}\rule{0.5\linewidth}{0.5pt}\end{center}

\section{Problem 4.4.7}\label{problem-4.4.7}

\textbf{Given:} A system has the signal flow graph with nodes R, X₁, X₂,
X₃, and Y. The branch gains are: R to X₁ = 1, X₁ to X₂ = G₁ = 4, X₂ to
X₃ = G₂ = 3, X₃ to Y = G₃ = 2, X₂ to X₁ = H₁ = -0.25 (feedback), and X₃
to X₂ = H₂ = -0.5 (feedback).

\textbf{Find:} The overall transfer function Y/R using Mason's gain
formula.

\textbf{Solution:}

Forward path: P₁ = 1 × G₁ × G₂ × G₃ = 1 × 4 × 3 × 2 = 24.

Loop gains: L₁ = G₁ × H₁ = 4 × (-0.25) = -1.0 (loop: X₁ → X₂ → X₁) L₂ =
G₂ × H₂ = 3 × (-0.5) = -1.5 (loop: X₂ → X₃ → X₂)

Non-touching loops: L₁ and L₂ share node X₂, so they are touching. No
non-touching pairs exist.

Graph determinant: Δ = 1 - (L₁ + L₂) = 1 - (-1.0 - 1.5) = 1 + 2.5 = 3.5

Cofactor: Δ₁ = 1 (all loops touch the forward path).

Y/R = P₁Δ₁ / Δ = 24 × 1 / 3.5 = \textbf{6.857}

\begin{center}\rule{0.5\linewidth}{0.5pt}\end{center}

\section{Problem 4.4.8}\label{problem-4.4.8}

\textbf{Given:} A signal flow graph has two forward paths and three
loops. Forward paths: P₁ = 5 × 3 = 15 (R → X₁ → X₃ → Y), P₂ = 2 × 4 = 8
(R → X₂ → X₃ → Y). Loop gains: L₁ = -2 (X₁ → X₃ → X₁), L₂ = -3 (X₂ → X₃
→ X₂), L₃ = -0.5 (X₁ → X₂ → X₁). L₁ and L₂ are touching. L₁ and L₃ are
touching. L₂ and L₃ are touching.

\textbf{Find:} (a) The graph determinant Δ, (b) the cofactors Δ₁ and Δ₂,
and (c) the overall transfer function Y/R.

\textbf{Solution:}

\begin{enumerate}
\def\labelenumi{(\alph{enumi})}
\item
  Since all loop pairs are touching (no non-touching pairs): Δ = 1 - (L₁
  + L₂ + L₃) = 1 - (-2 - 3 - 0.5) = 1 + 5.5 = \textbf{6.5}
\item
  Cofactor Δ₁: Remove all loops touching P₁. Path P₁ goes through X₁ and
  X₃, which touches L₁ (X₁, X₃), L₃ (X₁, X₂). L₂ (X₂, X₃) touches P₁
  through X₃. All loops touch P₁, so Δ₁ = \textbf{1}.
\end{enumerate}

Cofactor Δ₂: Remove all loops touching P₂. Path P₂ goes through X₂ and
X₃, which touches L₂ (X₂, X₃), L₃ (X₁, X₂). L₁ (X₁, X₃) touches P₂
through X₃. All loops touch P₂, so Δ₂ = \textbf{1}.

\begin{enumerate}
\def\labelenumi{(\alph{enumi})}
\setcounter{enumi}{2}
\tightlist
\item
  Y/R = (P₁Δ₁ + P₂Δ₂) / Δ = (15 × 1 + 8 × 1) / 6.5 = 23 / 6.5 =
  \textbf{3.538}
\end{enumerate}

\begin{center}\rule{0.5\linewidth}{0.5pt}\end{center}

\section{Problem 4.4.9}\label{problem-4.4.9}

\textbf{Given:} Two systems are connected in parallel. System 1: G₁(s) =
8/(s + 2). System 2: G₂(s) = -3/(s + 7). The combined output feeds into
a unity feedback loop.

\textbf{Find:} (a) The combined parallel transfer function, (b) the
closed-loop transfer function, (c) the closed-loop poles, and (d) the DC
gain.

\textbf{Solution:}

\begin{enumerate}
\def\labelenumi{(\alph{enumi})}
\item
  Parallel combination: G(s) = G₁(s) + G₂(s) = 8/(s + 2) + (-3)/(s + 7)
  = {[}8(s + 7) - 3(s + 2){]} / {[}(s + 2)(s + 7){]} G(s) = (8s + 56 -
  3s - 6) / (s² + 9s + 14) = \textbf{(5s + 50) / (s² + 9s + 14)}
\item
  Closed-loop: T(s) = G / (1 + G) = (5s + 50) / (s² + 9s + 14 + 5s + 50)
  = \textbf{(5s + 50) / (s² + 14s + 64)}
\item
  Poles: s = (-14 ± √(196 - 256)) / 2 = (-14 ± √(-60)) / 2 = (-14 ±
  j7.746) / 2 s = \textbf{-7 ± j3.873} (complex conjugate, underdamped,
  stable)
\item
  DC gain: T(0) = 50 / 64 = \textbf{0.781}
\end{enumerate}

\begin{center}\rule{0.5\linewidth}{0.5pt}\end{center}

\section{Problem 4.4.10}\label{problem-4.4.10}

\textbf{Given:} A control system has a disturbance D(s) entering at the
plant input. The controller is G\textsubscript{c}(s) = 10(s + 3)/(s +
15), the plant is G\textsubscript{p}(s) = 4/{[}s(s + 2){]}, and the
system uses unity feedback.

\textbf{Find:} (a) The closed-loop transfer function Y(s)/R(s), (b) the
disturbance transfer function Y(s)/D(s), (c) the steady-state output due
to a unit step disturbance (with R = 0), and (d) the steady-state error
to a unit step reference (with D = 0).

\textbf{Solution:}

\begin{enumerate}
\def\labelenumi{(\alph{enumi})}
\tightlist
\item
  Forward path: G(s) = G\textsubscript{c}(s) × G\textsubscript{p}(s) =
  {[}10(s + 3) / (s + 15){]} × {[}4 / s(s + 2){]} = 40(s + 3) / {[}s(s +
  2)(s + 15){]} Closed-loop: Y/R = G / (1 + G) = \textbf{40(s + 3) /
  {[}s(s + 2)(s + 15) + 40(s + 3){]}}
\end{enumerate}

Expand denominator: s³ + 17s² + 30s + 40s + 120 = s³ + 17s² + 70s + 120.
Y/R = \textbf{40(s + 3) / (s³ + 17s² + 70s + 120)}

\begin{enumerate}
\def\labelenumi{(\alph{enumi})}
\setcounter{enumi}{1}
\item
  Disturbance enters at plant input: Y/D = G\textsubscript{p} / (1 +
  G\textsubscript{c}G\textsubscript{p}) = {[}4/s(s + 2){]} / {[}1 + 40(s
  + 3)/s(s + 2)(s + 15){]} Y/D = 4(s + 15) / {[}s(s + 2)(s + 15) + 40(s
  + 3){]} = \textbf{4(s + 15) / (s³ + 17s² + 70s + 120)}
\item
  Steady-state output to unit step disturbance: y\textsubscript{ss} =
  lim(s→0) s × {[}4(s + 15) / (s³ + 17s² + 70s + 120){]} × (1/s) = 4(15)
  / 120 = 60/120 = \textbf{0.5}
\item
  The open-loop transfer function has one free integrator (Type 1
  system). K\textsubscript{p} = lim(s→0) G(s) = lim(s→0) 40(s + 3) /
  {[}s(s + 2)(s + 15){]} = ∞ (Type 1). e\textsubscript{ss} = 1 / (1 +
  K\textsubscript{p}) = 1 / ∞ = \textbf{0} (zero steady-state error to a
  step input)
\end{enumerate}

\chapter{Chapter 4 --- Section 4.5: Time-Domain
Performance}\label{chapter-4-section-4.5-time-domain-performance}

Practice problems covering first-order system response, second-order
system response, time constants, rise time, settling time, peak time,
percent overshoot, damping ratio, natural frequency, and steady-state
error analysis.

\begin{center}\rule{0.5\linewidth}{0.5pt}\end{center}

\section{Problem 4.5.1}\label{problem-4.5.1}

\textbf{Given:} A thermal process has a first-order transfer function
G(s) = 150 / (40s + 1), where the input is valve position (0-100\%) and
the output is temperature in degrees C.

\textbf{Find:} (a) The DC gain and time constant, (b) the steady-state
temperature for a 30\% valve input, (c) the time to reach 63.2\% of
final temperature, (d) the 2\% settling time, and (e) the 10-90\% rise
time.

\textbf{Solution:}

\begin{enumerate}
\def\labelenumi{(\alph{enumi})}
\item
  Comparing G(s) = K / (τs + 1): DC gain K = \textbf{150}, time constant
  τ = \textbf{40 seconds}
\item
  Steady-state temperature: T\textsubscript{ss} = K × input = 150 × 0.30
  = \textbf{45 degrees C}
\item
  Time to 63.2\%: t = τ = \textbf{40 seconds}
\item
  2\% settling time: t\textsubscript{s} = 4τ = 4 × 40 = \textbf{160
  seconds}
\item
  Rise time (10-90\%): t\textsubscript{r} = 2.2τ = 2.2 × 40 = \textbf{88
  seconds}
\end{enumerate}

\begin{center}\rule{0.5\linewidth}{0.5pt}\end{center}

\section{Problem 4.5.2}\label{problem-4.5.2}

\textbf{Given:} A position control servo has a second-order closed-loop
transfer function G(s) = 400 / (s² + 16s + 400).

\textbf{Find:} (a) ω\textsubscript{n} and ζ, (b) the damped natural
frequency ω\textsubscript{d}, (c) the percent overshoot, (d) the peak
time, (e) the 2\% settling time, and (f) the pole locations.

\textbf{Solution:}

\begin{enumerate}
\def\labelenumi{(\alph{enumi})}
\item
  Comparing with ω\textsubscript{n}²/(s² + 2ζω\textsubscript{n}s +
  ω\textsubscript{n}²): ω\textsubscript{n} = √400 = \textbf{20 rad/s}
  2ζω\textsubscript{n} = 16 → ζ = 16 / (2 × 20) = \textbf{0.4}
  (underdamped)
\item
  ω\textsubscript{d} = ω\textsubscript{n}√(1 - ζ²) = 20√(1 - 0.16) = 20
  × 0.9165 = \textbf{18.33 rad/s}
\item
  \%OS = 100 × e\textsuperscript{-ζπ/√(1-ζ²)} = 100 ×
  e\textsuperscript{-0.4π/0.9165} = 100 × e\textsuperscript{-1.371} =
  100 × 0.254 = \textbf{25.4\%}
\item
  Peak time: t\textsubscript{p} = π / ω\textsubscript{d} = π / 18.33 =
  \textbf{0.171 s}
\item
  2\% settling time: t\textsubscript{s} = 4 / (ζω\textsubscript{n}) = 4
  / (0.4 × 20) = 4 / 8 = \textbf{0.5 s}
\item
  Poles: s = -ζω\textsubscript{n} ± jω\textsubscript{d} = \textbf{-8 ±
  j18.33}
\end{enumerate}

\begin{center}\rule{0.5\linewidth}{0.5pt}\end{center}

\section{Problem 4.5.3}\label{problem-4.5.3}

\textbf{Given:} A control system must meet the following time-domain
specifications: percent overshoot no greater than 10\%, 2\% settling
time no greater than 0.8 s, and peak time no greater than 0.3 s.

\textbf{Find:} (a) The minimum damping ratio, (b) the minimum σ =
ζω\textsubscript{n} (real part of poles), (c) the minimum
ω\textsubscript{d} (damped frequency), and (d) the minimum
ω\textsubscript{n}.

\textbf{Solution:}

\begin{enumerate}
\def\labelenumi{(\alph{enumi})}
\item
  From \%OS = 100 × e\textsuperscript{-ζπ/√(1-ζ²)} ≤ 10:
  e\textsuperscript{-ζπ/√(1-ζ²)} ≤ 0.10 → -ζπ/√(1-ζ²) ≤ ln(0.10) =
  -2.303 ζπ/√(1-ζ²) ≥ 2.303 → ζ²π² ≥ 2.303²(1-ζ²) → ζ²(π² + 5.304) =
  5.304 ζ² = 5.304 / 15.17 = 0.3497 → ζ ≥ \textbf{0.591}
\item
  From t\textsubscript{s} = 4/(ζω\textsubscript{n}) ≤ 0.8: σ =
  ζω\textsubscript{n} ≥ 4 / 0.8 = \textbf{5 s⁻¹}
\item
  From t\textsubscript{p} = π/ω\textsubscript{d} ≤ 0.3:
  ω\textsubscript{d} ≥ π / 0.3 = \textbf{10.47 rad/s}
\item
  ω\textsubscript{n} = √(σ² + ω\textsubscript{d}²) = √(25 + 109.7) =
  √134.7 = \textbf{11.6 rad/s}
\end{enumerate}

Check: ζ = σ/ω\textsubscript{n} = 5/11.6 = 0.431 \textless{} 0.591, so
the overshoot constraint is more restrictive. Using ζ = 0.591:
ω\textsubscript{n} = σ/ζ = 5/0.591 = \textbf{8.46 rad/s}, and
ω\textsubscript{d} = 8.46 × √(1 - 0.349) = 8.46 × 0.807 = 6.83 rad/s.
But t\textsubscript{p} = π/6.83 = 0.46 s \textgreater{} 0.3 s, which
violates the peak time spec.

So ω\textsubscript{d} must be at least 10.47 rad/s. With ζ = 0.591:
ω\textsubscript{n} = ω\textsubscript{d}/√(1 - ζ²) = 10.47/0.807 =
\textbf{12.98 rad/s}. σ = ζω\textsubscript{n} = 0.591 × 12.98 = 7.67
\textgreater{} 5, which satisfies the settling time spec.

Final answer: ζ ≥ \textbf{0.591}, ω\textsubscript{n} ≥ \textbf{12.98
rad/s}, with poles at s = -7.67 ± j10.47.

\begin{center}\rule{0.5\linewidth}{0.5pt}\end{center}

\section{Problem 4.5.4}\label{problem-4.5.4}

\textbf{Given:} A unity feedback system has the open-loop transfer
function G(s) = 1000(s + 8) / {[}s²(s + 15)(s + 40){]}.

\textbf{Find:} (a) The system type, (b) the position error constant
K\textsubscript{p}, (c) the velocity error constant K\textsubscript{v},
(d) the acceleration error constant K\textsubscript{a}, and (e) the
steady-state errors for a unit step, unit ramp, and unit parabolic
input.

\textbf{Solution:}

\begin{enumerate}
\def\labelenumi{(\alph{enumi})}
\item
  G(s) has two free integrators (s² in the denominator). This is a
  \textbf{Type 2} system.
\item
  K\textsubscript{p} = lim(s→0) G(s) = lim(s→0) 1000(s + 8) / {[}s²(s +
  15)(s + 40){]} = \textbf{∞}
\item
  K\textsubscript{v} = lim(s→0) sG(s) = lim(s→0) 1000(s + 8) / {[}s(s +
  15)(s + 40){]} = \textbf{∞}
\item
  K\textsubscript{a} = lim(s→0) s²G(s) = lim(s→0) 1000(s + 8) / {[}(s +
  15)(s + 40){]} = 1000 × 8 / (15 × 40) = 8000 / 600 = \textbf{13.33}
\item
  Steady-state errors:
\end{enumerate}

\begin{itemize}
\tightlist
\item
  Unit step: e\textsubscript{ss} = 1/(1 + K\textsubscript{p}) = 1/∞ =
  \textbf{0}
\item
  Unit ramp: e\textsubscript{ss} = 1/K\textsubscript{v} = 1/∞ =
  \textbf{0}
\item
  Unit parabola r(t) = t²/2: e\textsubscript{ss} = 1/K\textsubscript{a}
  = 1/13.33 = \textbf{0.075}
\end{itemize}

\begin{center}\rule{0.5\linewidth}{0.5pt}\end{center}

\section{Problem 4.5.5}\label{problem-4.5.5}

\textbf{Given:} A second-order system has a step response that exhibits
the first peak at t = 0.25 s with a value of 1.35 (the final value is
1.0).

\textbf{Find:} (a) The percent overshoot, (b) the damping ratio ζ, (c)
the damped natural frequency ω\textsubscript{d}, (d) the natural
frequency ω\textsubscript{n}, and (e) the 2\% settling time.

\textbf{Solution:}

\begin{enumerate}
\def\labelenumi{(\alph{enumi})}
\item
  \%OS = (peak - final) / final × 100 = (1.35 - 1.0) / 1.0 × 100 =
  \textbf{35\%}
\item
  From \%OS = 100 × e\textsuperscript{-ζπ/√(1-ζ²)}: 0.35 =
  e\textsuperscript{-ζπ/√(1-ζ²)} → ln(0.35) = -1.0498 = -ζπ/√(1-ζ²)
  ζπ/√(1-ζ²) = 1.0498 → ζ²π² = 1.0498²(1-ζ²) = 1.1021(1-ζ²) ζ²(π² +
  1.1021) = 1.1021 → ζ² = 1.1021/10.97 = 0.1005 ζ = \textbf{0.317}
\item
  Peak time t\textsubscript{p} = π/ω\textsubscript{d} = 0.25 s.
  ω\textsubscript{d} = π / 0.25 = \textbf{12.57 rad/s}
\item
  ω\textsubscript{n} = ω\textsubscript{d} / √(1 - ζ²) = 12.57 / √(1 -
  0.1005) = 12.57 / 0.9487 = \textbf{13.25 rad/s}
\item
  t\textsubscript{s} = 4 / (ζω\textsubscript{n}) = 4 / (0.317 × 13.25) =
  4 / 4.20 = \textbf{0.952 s}
\end{enumerate}

\begin{center}\rule{0.5\linewidth}{0.5pt}\end{center}

\section{Problem 4.5.6}\label{problem-4.5.6}

\textbf{Given:} An overdamped second-order system has the transfer
function G(s) = 36 / (s² + 13s + 36).

\textbf{Find:} (a) The poles, (b) ω\textsubscript{n} and ζ, (c) the unit
step response c(t), and (d) the 2\% settling time (using the exact
response, not the approximate formula).

\textbf{Solution:}

\begin{enumerate}
\def\labelenumi{(\alph{enumi})}
\item
  s² + 13s + 36 = (s + 4)(s + 9) = 0. Poles: s = \textbf{-4} and s =
  \textbf{-9} (both real, overdamped).
\item
  ω\textsubscript{n} = √36 = \textbf{6 rad/s}. 2ζω\textsubscript{n} = 13
  → ζ = 13/12 = \textbf{1.083} (overdamped).
\item
  Step response: C(s) = 36 / {[}s(s + 4)(s + 9){]}. Partial fractions:
  C(s) = A/s + B/(s + 4) + C/(s + 9). A = 36/(4 × 9) = 1 B =
  36/{[}(-4)(-4 + 9){]} = 36/(-4 × 5) = -1.8 C = 36/{[}(-9)(-9 + 4){]} =
  36/(-9 × (-5)) = 0.8
\end{enumerate}

c(t) = \textbf{1 - 1.8e\textsuperscript{-4t} +
0.8e\textsuperscript{-9t}}

\begin{enumerate}
\def\labelenumi{(\alph{enumi})}
\setcounter{enumi}{3}
\tightlist
\item
  The dominant pole at s = -4 controls the settling. The 2\% criterion
  requires \textbar c(t) - 1\textbar{} ≤ 0.02. At large t, the
  e\textsuperscript{-9t} term is negligible: 1.8e\textsuperscript{-4t} =
  0.02 → e\textsuperscript{-4t} = 0.0111 → -4t = ln(0.0111) = -4.50.
  t\textsubscript{s} ≈ 4.50/4 = \textbf{1.125 s}
\end{enumerate}

Note: Using the approximate formula t\textsubscript{s} =
4/(ζω\textsubscript{n}) = 4/6.5 = 0.615 s would significantly
underestimate the settling time for overdamped systems.

\begin{center}\rule{0.5\linewidth}{0.5pt}\end{center}

\section{Problem 4.5.7}\label{problem-4.5.7}

\textbf{Given:} A unity feedback system has open-loop transfer function
G(s) = 50 / {[}(s + 5)(s + 10){]}. This is a Type 0 system with
steady-state error to a unit step.

\textbf{Find:} (a) The steady-state error to a unit step, (b) the value
of a proportional gain K placed in the forward path that reduces the
steady-state error to 2\%, (c) the closed-loop poles with that gain, and
(d) whether the system is underdamped or overdamped at that gain.

\textbf{Solution:}

\begin{enumerate}
\def\labelenumi{(\alph{enumi})}
\item
  K\textsubscript{p} = G(0) = 50 / (5 × 10) = 1. e\textsubscript{ss} =
  1/(1 + K\textsubscript{p}) = 1/(1 + 1) = \textbf{0.5} (50\% error)
\item
  For 2\% error: e\textsubscript{ss} = 1/(1 + KK\textsubscript{p,plant})
  = 0.02, where K\textsubscript{p,plant} = 50/50 = 1. 1/(1 + K) = 0.02 →
  1 + K = 50 → K = \textbf{49} (Verification: open-loop DC gain = 49 ×
  50/50 = 49, e\textsubscript{ss} = 1/50 = 0.02)
\end{enumerate}

Actually, with gain K in forward path: G\textsubscript{OL}(s) =
50K/{[}(s+5)(s+10){]}. K\textsubscript{p} = 50K/50 = K. For
e\textsubscript{ss} = 0.02: 1/(1+K) = 0.02 → K = \textbf{49}.

\begin{enumerate}
\def\labelenumi{(\alph{enumi})}
\setcounter{enumi}{2}
\item
  Closed-loop characteristic: s² + 15s + 50 + 50K = s² + 15s + 50 + 2450
  = s² + 15s + 2500 = 0. s = (-15 ± √(225 - 10000))/2 = (-15 ±
  √(-9775))/2 = (-15 ± j98.87)/2 Poles: s = \textbf{-7.5 ± j49.44}
\item
  The poles are complex conjugate → \textbf{underdamped}. ζ = 7.5/50 =
  0.15 (very low damping). \%OS = 100 × e\textsuperscript{-0.15π/0.989}
  = 100 × e\textsuperscript{-0.476} = 62.1\%. The gain needed for low
  steady-state error causes very poor transient response --- this
  illustrates the fundamental trade-off in proportional-only control.
\end{enumerate}

\begin{center}\rule{0.5\linewidth}{0.5pt}\end{center}

\section{Problem 4.5.8}\label{problem-4.5.8}

\textbf{Given:} A unity feedback system has open-loop transfer function
G(s) = 300 / {[}s(s + 12)(s + 25){]}. Determine the steady-state error
performance.

\textbf{Find:} (a) The system type, (b) K\textsubscript{p},
K\textsubscript{v}, and K\textsubscript{a}, (c) the steady-state error
for r(t) = 5u(t) (step of magnitude 5), (d) the steady-state error for
r(t) = 2t (ramp of slope 2), and (e) the steady-state error for r(t) =
0.5t² (parabolic input).

\textbf{Solution:}

\begin{enumerate}
\def\labelenumi{(\alph{enumi})}
\item
  One free integrator in G(s) → \textbf{Type 1} system.
\item
  K\textsubscript{p} = lim(s→0) G(s) = lim(s→0) 300/{[}s(s+12)(s+25){]}
  = \textbf{∞} K\textsubscript{v} = lim(s→0) sG(s) = 300/(12 × 25) =
  300/300 = \textbf{1.0} K\textsubscript{a} = lim(s→0) s²G(s) = lim(s→0)
  300s/{[}(s+12)(s+25){]} = \textbf{0}
\item
  Step input of magnitude 5: e\textsubscript{ss} = 5/(1 +
  K\textsubscript{p}) = 5/∞ = \textbf{0}
\item
  Ramp input r(t) = 2t: e\textsubscript{ss} = 2/K\textsubscript{v} =
  2/1.0 = \textbf{2.0}
\item
  Parabolic input r(t) = 0.5t² (A = 1): e\textsubscript{ss} =
  A/K\textsubscript{a} = 1/0 = \textbf{∞} (unbounded error; a Type 1
  system cannot track a parabolic input)
\end{enumerate}

\begin{center}\rule{0.5\linewidth}{0.5pt}\end{center}

\section{Problem 4.5.9}\label{problem-4.5.9}

\textbf{Given:} A second-order system has the closed-loop transfer
function T(s) = ω\textsubscript{n}² / (s² + 2ζω\textsubscript{n}s +
ω\textsubscript{n}²) with ω\textsubscript{n} = 10 rad/s. The damping
ratio is varied from ζ = 0.1 to ζ = 2.0.

\textbf{Find:} The percent overshoot, peak time, and 2\% settling time
for: (a) ζ = 0.1, (b) ζ = 0.3, (c) ζ = 0.707, (d) ζ = 1.0, and (e) ζ =
2.0.

\textbf{Solution:}

\begin{enumerate}
\def\labelenumi{(\alph{enumi})}
\item
  ζ = 0.1: ω\textsubscript{d} = 10√(1-0.01) = 9.95 rad/s. \%OS = 100 ×
  e\textsuperscript{-0.1π/0.995} = 100 × e\textsuperscript{-0.316} =
  \textbf{72.9\%} t\textsubscript{p} = π/9.95 = \textbf{0.316 s}
  t\textsubscript{s} = 4/(0.1 × 10) = \textbf{4.0 s}
\item
  ζ = 0.3: ω\textsubscript{d} = 10√(1-0.09) = 9.54 rad/s. \%OS = 100 ×
  e\textsuperscript{-0.3π/0.954} = 100 × e\textsuperscript{-0.987} =
  \textbf{37.3\%} t\textsubscript{p} = π/9.54 = \textbf{0.329 s}
  t\textsubscript{s} = 4/(0.3 × 10) = \textbf{1.333 s}
\item
  ζ = 0.707: ω\textsubscript{d} = 10√(1-0.5) = 7.07 rad/s. \%OS = 100 ×
  e\textsuperscript{-0.707π/0.707} = 100 × e\textsuperscript{-π} = 100 ×
  0.0432 = \textbf{4.3\%} t\textsubscript{p} = π/7.07 = \textbf{0.444 s}
  t\textsubscript{s} = 4/(0.707 × 10) = \textbf{0.566 s}
\item
  ζ = 1.0 (critically damped): No overshoot → \textbf{\%OS = 0\%}. No
  oscillation, so t\textsubscript{p} = \textbf{∞} (no peak; monotonic
  approach to final value). t\textsubscript{s} = 4/(1.0 × 10) =
  \textbf{0.4 s}
\item
  ζ = 2.0 (overdamped): Poles at s = -10(2 ± √(4-1)) = -10(2 ± 1.732) →
  s = -2.68 and -37.32. \textbf{\%OS = 0\%}, t\textsubscript{p} =
  \textbf{∞} (no peak). Settling dominated by slow pole:
  t\textsubscript{s} ≈ 4/2.68 = \textbf{1.49 s}
\end{enumerate}

Note: ζ = 0.707 provides the best balance --- fast settling (0.566 s)
with minimal overshoot (4.3\%).

\begin{center}\rule{0.5\linewidth}{0.5pt}\end{center}

\section{Problem 4.5.10}\label{problem-4.5.10}

\textbf{Given:} A unity feedback system with open-loop transfer function
G(s) = K(s + 6) / {[}s(s + 3)(s + 10){]} is required to have a
steady-state error of no more than 0.1 for a ramp input r(t) = t.

\textbf{Find:} (a) The system type, (b) the minimum K to meet the ramp
error specification, (c) the closed-loop characteristic equation at that
K, and (d) verify stability using the Routh criterion.

\textbf{Solution:}

\begin{enumerate}
\def\labelenumi{(\alph{enumi})}
\item
  One free integrator → \textbf{Type 1} system.
\item
  K\textsubscript{v} = lim(s→0) sG(s) = lim(s→0)
  K(s+6)/{[}(s+3)(s+10){]} = 6K/(3 × 10) = K/5. For e\textsubscript{ss}
  = 1/K\textsubscript{v} ≤ 0.1: K\textsubscript{v} ≥ 10 → K/5 ≥ 10 → K ≥
  \textbf{50}
\item
  At K = 50: closed-loop characteristic equation: s(s + 3)(s + 10) +
  50(s + 6) = 0 s³ + 13s² + 30s + 50s + 300 = 0 \textbf{s³ + 13s² + 80s
  + 300 = 0}
\item
  Routh array: \textbar{} s³ \textbar{} 1 \textbar{} 80 \textbar{}
  \textbar{} s² \textbar{} 13 \textbar{} 300 \textbar{} \textbar{} s¹
  \textbar{} (13 × 80 - 300)/13 \textbar{} 0 \textbar{} \textbar{} s⁰
  \textbar{} 300 \textbar{} \textbar{}
\end{enumerate}

s¹ row: (1040 - 300)/13 = 740/13 = 56.9 \textgreater{} 0. All
first-column entries are positive: 1, 13, 56.9, 300 → \textbf{system is
stable} at K = 50.

\chapter{Chapter 4 --- Section 4.6: PID
Control}\label{chapter-4-section-4.6-pid-control}

Practice problems covering proportional control, integral control,
derivative control, PID controller design, Ziegler-Nichols tuning,
steady-state error elimination, and transient response improvement.

\begin{center}\rule{0.5\linewidth}{0.5pt}\end{center}

\section{Problem 4.6.1}\label{problem-4.6.1}

\textbf{Given:} A proportional controller with gain K\textsubscript{p}
controls a plant G(s) = 5 / (s + 8) in a unity feedback configuration.

\textbf{Find:} (a) The closed-loop transfer function, (b) the
steady-state error to a unit step for K\textsubscript{p} = 10, (c) the
K\textsubscript{p} required to limit the steady-state error to 5\%, (d)
the closed-loop pole and time constant at that K\textsubscript{p}.

\textbf{Solution:}

\begin{enumerate}
\def\labelenumi{(\alph{enumi})}
\item
  Closed-loop: T(s) = K\textsubscript{p}G(s) / (1 +
  K\textsubscript{p}G(s)) = 5K\textsubscript{p} / (s + 8 +
  5K\textsubscript{p}) = \textbf{5K\textsubscript{p} / (s + 8 +
  5K\textsubscript{p})}
\item
  K\textsubscript{p,position} = lim(s→0) K\textsubscript{p} × 5/(s + 8)
  = 5K\textsubscript{p}/8 = 5(10)/8 = 6.25. e\textsubscript{ss} = 1/(1 +
  6.25) = 1/7.25 = \textbf{0.138} (13.8\% error)
\item
  For 5\% error: e\textsubscript{ss} = 1/(1 + 5K\textsubscript{p}/8) =
  0.05. 1 + 5K\textsubscript{p}/8 = 20 → 5K\textsubscript{p}/8 = 19 →
  K\textsubscript{p} = \textbf{30.4}
\item
  Closed-loop pole: s = -(8 + 5 × 30.4) = -(8 + 152) = \textbf{-160}.
  Time constant: τ = 1/160 = \textbf{6.25 ms}. The high gain makes the
  system very fast but leaves a 5\% residual error.
\end{enumerate}

\begin{center}\rule{0.5\linewidth}{0.5pt}\end{center}

\section{Problem 4.6.2}\label{problem-4.6.2}

\textbf{Given:} A PI controller (K\textsubscript{p} = 3,
K\textsubscript{i} = 6) controls a plant G(s) = 4/(s + 2) with unity
feedback.

\textbf{Find:} (a) The open-loop transfer function, (b) the system type,
(c) the steady-state error to a unit step, (d) the steady-state error to
a unit ramp, and (e) the closed-loop characteristic equation and its
roots.

\textbf{Solution:}

\begin{enumerate}
\def\labelenumi{(\alph{enumi})}
\item
  PI controller: G\textsubscript{c}(s) = K\textsubscript{p} +
  K\textsubscript{i}/s = (3s + 6)/s. Open-loop: G\textsubscript{OL}(s) =
  G\textsubscript{c}(s)G(s) = 4(3s + 6)/{[}s(s + 2){]} = \textbf{(12s +
  24) / {[}s(s + 2){]}}
\item
  One free integrator from the PI controller → \textbf{Type 1} system.
\item
  Type 1 system: e\textsubscript{ss} for step = \textbf{0} (zero
  steady-state error to a step).
\item
  K\textsubscript{v} = lim(s→0) s × (12s + 24)/{[}s(s + 2){]} = 24/2 =
  12. e\textsubscript{ss} = 1/K\textsubscript{v} = 1/12 =
  \textbf{0.0833} (8.33\% ramp lag)
\item
  Characteristic equation: s(s + 2) + 12s + 24 = s² + 14s + 24 = 0. s =
  (-14 ± √(196 - 96))/2 = (-14 ± 10)/2. Roots: s = \textbf{-2} and s =
  \textbf{-12} (both real, stable, overdamped).
\end{enumerate}

\begin{center}\rule{0.5\linewidth}{0.5pt}\end{center}

\section{Problem 4.6.3}\label{problem-4.6.3}

\textbf{Given:} A PD controller (K\textsubscript{p} = 40,
K\textsubscript{d} = 4) controls a plant G(s) = 1/(s² + 6s) with unity
feedback. Compare the performance with proportional-only control
(K\textsubscript{d} = 0).

\textbf{Find:} (a) The closed-loop transfer function with PD control,
(b) ω\textsubscript{n} and ζ with PD control, (c) the percent overshoot
with PD control, (d) ω\textsubscript{n} and ζ with P-only control, and
(e) the percent overshoot with P-only control.

\textbf{Solution:}

\begin{enumerate}
\def\labelenumi{(\alph{enumi})}
\item
  PD controller: G\textsubscript{c}(s) = 40 + 4s. Closed-loop: T(s) =
  (4s + 40)/(s² + 6s + 4s + 40) = \textbf{(4s + 40) / (s² + 10s + 40)}
\item
  ω\textsubscript{n} = √40 = \textbf{6.32 rad/s}, 2ζω\textsubscript{n} =
  10 → ζ = 10/(2 × 6.32) = \textbf{0.791}
\item
  \%OS\textsubscript{PD} = 100 × e\textsuperscript{-0.791π/√(1-0.626)} =
  100 × e\textsuperscript{-0.791π/0.612} = 100 ×
  e\textsuperscript{-4.06} = 100 × 0.0173 = \textbf{1.7\%}
\end{enumerate}

(Note: The zero in the numerator at s = -10 will increase actual
overshoot slightly above this value computed from the denominator
alone.)

\begin{enumerate}
\def\labelenumi{(\alph{enumi})}
\setcounter{enumi}{3}
\item
  P-only (K\textsubscript{d} = 0): T(s) = 40/(s² + 6s + 40).
  ω\textsubscript{n} = √40 = \textbf{6.32 rad/s}, 2ζω\textsubscript{n} =
  6 → ζ = 6/12.65 = \textbf{0.474}
\item
  \%OS\textsubscript{P} = 100 × e\textsuperscript{-0.474π/√(1-0.225)} =
  100 × e\textsuperscript{-0.474π/0.881} = 100 ×
  e\textsuperscript{-1.690} = 100 × 0.185 = \textbf{18.5\%}
\end{enumerate}

The derivative action reduced overshoot from \textbf{18.5\% to 1.7\%} by
increasing ζ from 0.474 to 0.791.

\begin{center}\rule{0.5\linewidth}{0.5pt}\end{center}

\section{Problem 4.6.4}\label{problem-4.6.4}

\textbf{Given:} A temperature control loop produces sustained
oscillations at K\textsubscript{u} = 12 with a period P\textsubscript{u}
= 8 s when using proportional-only control. Use the Ziegler-Nichols
ultimate gain method.

\textbf{Find:} (a) P-only controller parameters, (b) PI controller
parameters, (c) PID controller parameters, and (d) the PID controller
transfer function.

\textbf{Solution:}

\begin{enumerate}
\def\labelenumi{(\alph{enumi})}
\item
  P-only: K\textsubscript{p} = 0.5 × K\textsubscript{u} = 0.5 × 12 =
  \textbf{6.0}
\item
  PI: K\textsubscript{p} = 0.45 × K\textsubscript{u} = 0.45 × 12 =
  \textbf{5.4} K\textsubscript{i} = 0.54 ×
  K\textsubscript{u}/P\textsubscript{u} = 0.54 × 12/8 = \textbf{0.81
  s⁻¹} (T\textsubscript{i} = K\textsubscript{p}/K\textsubscript{i} =
  5.4/0.81 = 6.67 s)
\item
  PID: K\textsubscript{p} = 0.6 × K\textsubscript{u} = 0.6 × 12 =
  \textbf{7.2} K\textsubscript{i} = 1.2 ×
  K\textsubscript{u}/P\textsubscript{u} = 1.2 × 12/8 = \textbf{1.8 s⁻¹}
  K\textsubscript{d} = 0.075 × K\textsubscript{u} × P\textsubscript{u} =
  0.075 × 12 × 8 = \textbf{7.2 s}
\item
  G\textsubscript{c}(s) = K\textsubscript{p} + K\textsubscript{i}/s +
  K\textsubscript{d}s = 7.2 + 1.8/s + 7.2s = \textbf{(7.2s² + 7.2s +
  1.8) / s}
\end{enumerate}

Verification: T\textsubscript{i} = K\textsubscript{p}/K\textsubscript{i}
= 7.2/1.8 = 4.0 s = P\textsubscript{u}/2 = 4.0 s. T\textsubscript{d} =
K\textsubscript{d}/K\textsubscript{p} = 7.2/7.2 = 1.0 s =
P\textsubscript{u}/8 = 1.0 s. Both check out.

\begin{center}\rule{0.5\linewidth}{0.5pt}\end{center}

\section{Problem 4.6.5}\label{problem-4.6.5}

\textbf{Given:} A process plant has the first-order-plus-dead-time
(FOPDT) model G(s) = 2e\textsuperscript{-3s} / (10s + 1), identified
from an open-loop step response. The reaction curve parameters are: DC
gain K\textsubscript{p} = 2, dead time L = 3 s, and time constant T = 10
s. Use the Ziegler-Nichols step response method.

\textbf{Find:} The PID parameters using the Ziegler-Nichols first method
(reaction curve): K\textsubscript{p} = 1.2T/(KL), T\textsubscript{i} =
2L, T\textsubscript{d} = 0.5L.

\textbf{Solution:}

Proportional gain: K\textsubscript{p} = 1.2T / (K × L) = 1.2 × 10 / (2 ×
3) = 12/6 = \textbf{2.0}

Integral time: T\textsubscript{i} = 2L = 2 × 3 = \textbf{6.0 s}
K\textsubscript{i} = K\textsubscript{p}/T\textsubscript{i} = 2.0/6.0 =
\textbf{0.333 s⁻¹}

Derivative time: T\textsubscript{d} = 0.5L = 0.5 × 3 = \textbf{1.5 s}
K\textsubscript{d} = K\textsubscript{p} × T\textsubscript{d} = 2.0 × 1.5
= \textbf{3.0 s}

PID transfer function: G\textsubscript{c}(s) = 2.0(1 + 1/(6s) + 1.5s) =
\textbf{2.0 + 0.333/s + 3.0s}

The ratio T/L = 10/3 = 3.33. Ziegler-Nichols works best when 3
\textless{} T/L \textless{} 10, so this plant is at the lower end of the
useful range. Some detuning may be needed due to the relatively large
dead time.

\begin{center}\rule{0.5\linewidth}{0.5pt}\end{center}

\section{Problem 4.6.6}\label{problem-4.6.6}

\textbf{Given:} A PI controller is used to eliminate steady-state error
in a Type 0 plant G(s) = 10 / {[}(s + 1)(s + 5){]}. The PI gains are
K\textsubscript{p} = 4 and K\textsubscript{i} = 2.

\textbf{Find:} (a) The open-loop transfer function, (b) the closed-loop
transfer function, (c) the closed-loop poles, (d) the percent overshoot
(approximate, using dominant poles), and (e) the 2\% settling time.

\textbf{Solution:}

\begin{enumerate}
\def\labelenumi{(\alph{enumi})}
\item
  G\textsubscript{c}(s) = (4s + 2)/s. G\textsubscript{OL}(s) = 10(4s +
  2)/{[}s(s + 1)(s + 5){]} = \textbf{(40s + 20) / {[}s(s + 1)(s + 5){]}}
\item
  Closed-loop characteristic: s(s + 1)(s + 5) + 40s + 20 = 0. s³ + 6s² +
  5s + 40s + 20 = s³ + 6s² + 45s + 20 = 0.
\end{enumerate}

T(s) = \textbf{(40s + 20) / (s³ + 6s² + 45s + 20)}

\begin{enumerate}
\def\labelenumi{(\alph{enumi})}
\setcounter{enumi}{2}
\item
  Using numerical methods, the real root is approximately s ≈
  \textbf{-0.472}. Factor: s³ + 6s² + 45s + 20 ≈ (s + 0.472)(s² + 5.528s
  + 42.37). Verification: expanding gives s³ + 6.0s² + (2.609 + 42.37)s
  + 20.0 = s³ + 6s² + 44.98s + 20 ≈ original. Complex roots: s = (-5.528
  ± √(30.56 - 169.48))/2 = (-5.528 ± j11.79)/2. s = \textbf{-2.764 ±
  j5.894}
\item
  Dominant complex poles: ζ = 2.764/√(2.764² + 5.894²) = 2.764/6.510 =
  0.425. \%OS ≈ 100 × e\textsuperscript{-0.425π/√(1-0.180)} = 100 ×
  e\textsuperscript{-0.425π/0.905} = 100 × e\textsuperscript{-1.474} =
  \textbf{22.9\%}
\item
  t\textsubscript{s} ≈ 4/σ = 4/2.764 = \textbf{1.448 s}
\end{enumerate}

\begin{center}\rule{0.5\linewidth}{0.5pt}\end{center}

\section{Problem 4.6.7}\label{problem-4.6.7}

\textbf{Given:} A PID controller with K\textsubscript{p} = 8,
K\textsubscript{i} = 15, and K\textsubscript{d} = 1.5 controls a plant
G(s) = 1/(s + 3) with unity feedback.

\textbf{Find:} (a) The open-loop transfer function, (b) the closed-loop
characteristic equation, (c) the system type, (d) K\textsubscript{v},
and (e) the steady-state error for a ramp input r(t) = 2t.

\textbf{Solution:}

\begin{enumerate}
\def\labelenumi{(\alph{enumi})}
\item
  G\textsubscript{c}(s) = K\textsubscript{p} + K\textsubscript{i}/s +
  K\textsubscript{d}s = (1.5s² + 8s + 15)/s. G\textsubscript{OL}(s) =
  (1.5s² + 8s + 15)/{[}s(s + 3){]} = \textbf{(1.5s² + 8s + 15) / {[}s(s
  + 3){]}}
\item
  Characteristic equation: s(s + 3) + 1.5s² + 8s + 15 = 0. s² + 3s +
  1.5s² + 8s + 15 = 2.5s² + 11s + 15 = 0.
\end{enumerate}

Divide by 2.5: \textbf{s² + 4.4s + 6 = 0}

\begin{enumerate}
\def\labelenumi{(\alph{enumi})}
\setcounter{enumi}{2}
\item
  One free integrator in the open-loop (from the integral term) →
  \textbf{Type 1} system.
\item
  K\textsubscript{v} = lim(s→0) sG\textsubscript{OL}(s) = lim(s→0)
  (1.5s² + 8s + 15)/(s + 3) = 15/3 = \textbf{5.0}
\item
  For ramp r(t) = 2t: e\textsubscript{ss} = 2/K\textsubscript{v} = 2/5.0
  = \textbf{0.4}
\end{enumerate}

\begin{center}\rule{0.5\linewidth}{0.5pt}\end{center}

\section{Problem 4.6.8}\label{problem-4.6.8}

\textbf{Given:} An industrial pressure loop with plant G(s) = 3 / {[}(s
+ 1)(s + 4){]} uses a PI controller. The system must have zero
steady-state error to a step and a damping ratio of at least ζ = 0.5.
The PI controller has K\textsubscript{i} = 2K\textsubscript{p} (the
integral gain is twice the proportional gain).

\textbf{Find:} (a) The open-loop transfer function in terms of
K\textsubscript{p}, (b) the closed-loop characteristic equation, (c) the
value of K\textsubscript{p} that yields ζ = 0.5, and (d) the resulting
K\textsubscript{i}.

\textbf{Solution:}

\begin{enumerate}
\def\labelenumi{(\alph{enumi})}
\item
  G\textsubscript{c}(s) = K\textsubscript{p} + 2K\textsubscript{p}/s =
  K\textsubscript{p}(s + 2)/s. G\textsubscript{OL}(s) =
  3K\textsubscript{p}(s + 2) / {[}s(s + 1)(s + 4){]} =
  \textbf{3K\textsubscript{p}(s + 2) / {[}s(s + 1)(s + 4){]}}
\item
  Characteristic equation: s(s + 1)(s + 4) + 3K\textsubscript{p}(s + 2)
  = 0. s³ + 5s² + 4s + 3K\textsubscript{p}s + 6K\textsubscript{p} = 0.
  \textbf{s³ + 5s² + (4 + 3K\textsubscript{p})s + 6K\textsubscript{p} =
  0}
\item
  For a third-order system, exact damping ratio assignment requires
  numerical methods. Using the dominant second-order pole approximation,
  assume the characteristic equation factors as (s + a)(s² +
  2ζω\textsubscript{n}s + ω\textsubscript{n}²) where a is a distant real
  pole.
\end{enumerate}

Expanding: s³ + (a + 2ζω\textsubscript{n})s² + (2aζω\textsubscript{n} +
ω\textsubscript{n}²)s + aω\textsubscript{n}². Match: a +
2ζω\textsubscript{n} = 5, 2aζω\textsubscript{n} + ω\textsubscript{n}² =
4 + 3K\textsubscript{p}, aω\textsubscript{n}² = 6K\textsubscript{p}.

With ζ = 0.5, try K\textsubscript{p} = 2: s³ + 5s² + 10s + 12 = 0. Test
s = -1: -1 + 5 - 10 + 12 = 6 (no). Test s = -2: -8 + 20 - 20 + 12 = 4
(no). Test s = -3: -27 + 45 - 30 + 12 = 0 (yes!). Factor: (s + 3)(s² +
2s + 4) = 0. From s² + 2s + 4: ω\textsubscript{n} = 2, ζ = 1/2 = 0.5.

K\textsubscript{p} = \textbf{2}

\begin{enumerate}
\def\labelenumi{(\alph{enumi})}
\setcounter{enumi}{3}
\tightlist
\item
  K\textsubscript{i} = 2K\textsubscript{p} = \textbf{4 s⁻¹}
\end{enumerate}

Verification: The closed-loop poles are s = -3 and s = -1 ± j√3 = -1 ±
j1.732. The real pole at -3 is 3× the real part of the complex poles
(-1), so the dominant second-order approximation is reasonable.

\begin{center}\rule{0.5\linewidth}{0.5pt}\end{center}

\section{Problem 4.6.9}\label{problem-4.6.9}

\textbf{Given:} A PID controller must be designed for a plant G(s) =
1/{[}s(s + 10){]} in a unity feedback system. The specifications
require: zero steady-state error to a ramp input (guaranteed by Type 2
system), ζ ≥ 0.6 for the dominant poles, and ω\textsubscript{n} ≥ 8
rad/s.

\textbf{Find:} (a) The required system type with PID control, (b) design
K\textsubscript{p}, K\textsubscript{i}, and K\textsubscript{d} to place
the closed-loop poles at s = -5 ± j6.33 and s = -20 (a fast real pole).

\textbf{Solution:}

\begin{enumerate}
\def\labelenumi{(\alph{enumi})}
\item
  The plant has one integrator (1/s). The PID integral term adds
  another, making the open-loop \textbf{Type 2}. This guarantees zero
  ramp error.
\item
  Desired characteristic polynomial: (s + 20)(s + 5 - j6.33)(s + 5 +
  j6.33) = (s + 20)(s² + 10s + 65.06). Expanding: s³ + 30s² + 265.06s +
  1301.2.
\end{enumerate}

PID controller: G\textsubscript{c}(s) = (K\textsubscript{d}s² +
K\textsubscript{p}s + K\textsubscript{i})/s. Open-loop:
G\textsubscript{OL}(s) = (K\textsubscript{d}s² + K\textsubscript{p}s +
K\textsubscript{i})/{[}s²(s + 10){]}.

Closed-loop characteristic: s²(s + 10) + K\textsubscript{d}s² +
K\textsubscript{p}s + K\textsubscript{i} = 0. s³ + (10 +
K\textsubscript{d})s² + K\textsubscript{p}s + K\textsubscript{i} = 0.

Match with desired: s³ + 30s² + 265.06s + 1301.2: 10 +
K\textsubscript{d} = 30 → K\textsubscript{d} = \textbf{20}
K\textsubscript{p} = \textbf{265.06} K\textsubscript{i} =
\textbf{1301.2}

Verification: ω\textsubscript{n} = √(25 + 40.07) = √65.06 = 8.07 rad/s ≥
8. ζ = 5/8.07 = 0.620 ≥ 0.6. Both specifications met.

\begin{center}\rule{0.5\linewidth}{0.5pt}\end{center}

\section{Problem 4.6.10}\label{problem-4.6.10}

\textbf{Given:} A motor speed control system uses a PID controller. An
integral windup test is performed: the setpoint is 1000 RPM, the motor
saturates at 500 RPM due to a mechanical load, and the error accumulates
for 10 seconds. PID parameters are K\textsubscript{p} = 0.5,
K\textsubscript{i} = 2.0, K\textsubscript{d} = 0.02, with sample period
T = 0.01 s.

\textbf{Find:} (a) The steady-state error during saturation, (b) the
accumulated integral term after 10 seconds (assuming constant error),
(c) the total control signal at t = 10 s, and (d) the time to unwind the
integral if the load is suddenly removed and error reverses at 500 RPM.

\textbf{Solution:}

\begin{enumerate}
\def\labelenumi{(\alph{enumi})}
\item
  Error during saturation: e = 1000 - 500 = \textbf{500 RPM}
\item
  Accumulated integral term after 10 s with constant error: Integral sum
  = Σe{[}k{]} × T for k = 0 to 999 (10 s / 0.01 s = 1000 samples). Each
  sample contributes e × T = 500 × 0.01 = 5. Total integral sum = 1000 ×
  5 = 5000. Integral contribution: K\textsubscript{i} × integral sum =
  2.0 × 5000 = \textbf{10,000}
\item
  At t = 10 s (error still 500 RPM, de/dt = 0): u = K\textsubscript{p} ×
  500 + 10,000 + K\textsubscript{d} × 0 = 250 + 10,000 = \textbf{10,250}
\end{enumerate}

This is far beyond any reasonable actuator range, illustrating the
windup problem.

\begin{enumerate}
\def\labelenumi{(\alph{enumi})}
\setcounter{enumi}{3}
\tightlist
\item
  If the error reverses to -500 RPM (load removed, motor overshoots to
  1500 RPM): The integral decreases at K\textsubscript{i} × (-500) × T =
  2.0 × (-500) × 0.01 = -10 per sample. The accumulated integral is
  10,000. Time to unwind to zero: 10,000 / 10 = 1000 samples =
  \textbf{10 seconds}.
\end{enumerate}

During this 10-second unwind, the controller output remains positive
despite a negative error, causing prolonged overshoot. This demonstrates
why \textbf{anti-windup} mechanisms (clamping, back-calculation) are
essential in practical PID implementations.

\chapter{Chapter 4 --- Section 4.7: Stability
Analysis}\label{chapter-4-section-4.7-stability-analysis}

Practice problems covering Routh-Hurwitz criterion, root locus
construction, gain margins, stability boundaries, special Routh cases,
and root locus design.

\begin{center}\rule{0.5\linewidth}{0.5pt}\end{center}

\section{Problem 4.7.1}\label{problem-4.7.1}

\textbf{Given:} A unity feedback system has open-loop transfer function
G(s) = K / {[}s(s + 4)(s + 12){]}.

\textbf{Find:} (a) The closed-loop characteristic equation, (b) the
Routh array, (c) the range of K for stability, and (d) the frequency of
oscillation at the stability boundary.

\textbf{Solution:}

\begin{enumerate}
\def\labelenumi{(\alph{enumi})}
\item
  Characteristic equation: s(s + 4)(s + 12) + K = 0. s³ + 16s² + 48s + K
  = 0.
\item
  Routh array: \textbar{} s³ \textbar{} 1 \textbar{} 48 \textbar{}
  \textbar{} s² \textbar{} 16 \textbar{} K \textbar{} \textbar{} s¹
  \textbar{} (16 × 48 - K)/16 \textbar{} 0 \textbar{} \textbar{} s⁰
  \textbar{} K \textbar{} \textbar{}
\end{enumerate}

s¹ row: (768 - K)/16

\begin{enumerate}
\def\labelenumi{(\alph{enumi})}
\setcounter{enumi}{2}
\tightlist
\item
  For stability, all first-column entries must be positive:
\end{enumerate}

\begin{itemize}
\tightlist
\item
  s³: 1 \textgreater{} 0
\item
  s²: 16 \textgreater{} 0
\item
  s¹: (768 - K)/16 \textgreater{} 0 → K \textless{} 768
\item
  s⁰: K \textgreater{} 0
\end{itemize}

Range: \textbf{0 \textless{} K \textless{} 768}

\begin{enumerate}
\def\labelenumi{(\alph{enumi})}
\setcounter{enumi}{3}
\tightlist
\item
  At K = 768, the s¹ row is zero. Form the auxiliary equation from the
  s² row: 16s² + 768 = 0 → s² = -48 → s = ±j√48 = ±j6.93. Frequency of
  oscillation: ω = \textbf{6.93 rad/s}
\end{enumerate}

\begin{center}\rule{0.5\linewidth}{0.5pt}\end{center}

\section{Problem 4.7.2}\label{problem-4.7.2}

\textbf{Given:} A control system has the characteristic equation s⁴ +
3s³ + 5s² + 9s + K = 0.

\textbf{Find:} (a) The Routh array in terms of K, (b) the range of K for
stability, and (c) the value of K at which the system is marginally
stable and the corresponding oscillation frequency.

\textbf{Solution:}

\begin{enumerate}
\def\labelenumi{(\alph{enumi})}
\item
  Routh array: \textbar{} s⁴ \textbar{} 1 \textbar{} 5 \textbar{} K
  \textbar{} \textbar{} s³ \textbar{} 3 \textbar{} 9 \textbar{} 0
  \textbar{} \textbar{} s² \textbar{} (15 - 9)/3 = 2 \textbar{} K
  \textbar{} \textbar{} \textbar{} s¹ \textbar{} (2 × 9 - 3K)/2 = (18 -
  3K)/2 \textbar{} 0 \textbar{} \textbar{} \textbar{} s⁰ \textbar{} K
  \textbar{} \textbar{} \textbar{}
\item
  For stability:
\end{enumerate}

\begin{itemize}
\tightlist
\item
  s⁴: 1 \textgreater{} 0
\item
  s³: 3 \textgreater{} 0
\item
  s²: 2 \textgreater{} 0
\item
  s¹: (18 - 3K)/2 \textgreater{} 0 → K \textless{} 6
\item
  s⁰: K \textgreater{} 0
\end{itemize}

Range: \textbf{0 \textless{} K \textless{} 6}

\begin{enumerate}
\def\labelenumi{(\alph{enumi})}
\setcounter{enumi}{2}
\tightlist
\item
  At K = 6, the s¹ row is zero. Auxiliary equation from s² row: 2s² + 6
  = 0 → s² = -3 → s = ±j√3 = ±j1.732. Frequency of oscillation: ω =
  \textbf{1.732 rad/s}
\end{enumerate}

The system is marginally stable at K = \textbf{6} with sustained
oscillations at 1.732 rad/s.

\begin{center}\rule{0.5\linewidth}{0.5pt}\end{center}

\section{Problem 4.7.3}\label{problem-4.7.3}

\textbf{Given:} A characteristic equation has the form s³ + 2s² + (K +
3)s + 2K = 0, where K is a positive real parameter.

\textbf{Find:} (a) The Routh array in terms of K, (b) the range of K for
stability, (c) whether K = 4 produces a stable system, and (d) the
number of RHP poles at K = 10.

\textbf{Solution:}

\begin{enumerate}
\def\labelenumi{(\alph{enumi})}
\item
  Routh array: \textbar{} s³ \textbar{} 1 \textbar{} K + 3 \textbar{}
  \textbar{} s² \textbar{} 2 \textbar{} 2K \textbar{} \textbar{} s¹
  \textbar{} (2(K + 3) - 2K)/2 = 6/2 = 3 \textbar{} 0 \textbar{}
  \textbar{} s⁰ \textbar{} 2K \textbar{} \textbar{}
\item
  First-column entries: 1, 2, 3, 2K. For stability: 2K \textgreater{} 0
  → K \textgreater{} 0. All other entries are always positive. Range:
  \textbf{K \textgreater{} 0} (stable for all positive K).
\item
  At K = 4: all first-column entries are 1, 2, 3, 8 --- all positive.
  \textbf{Yes, the system is stable.}
\item
  At K = 10: first-column entries are 1, 2, 3, 20 --- all positive.
  \textbf{Zero RHP poles} (the system is stable). Note that the s¹ row
  entry is 3 regardless of K, so the system is stable for all K
  \textgreater{} 0.
\end{enumerate}

\begin{center}\rule{0.5\linewidth}{0.5pt}\end{center}

\section{Problem 4.7.4}\label{problem-4.7.4}

\textbf{Given:} The open-loop transfer function is G(s)H(s) = K / {[}s(s
+ 1)(s + 3)(s + 6){]}.

\textbf{Find:} (a) The number of root locus branches, (b) the real-axis
segments of the root locus, (c) the asymptote angles and centroid, and
(d) the value of K at the imaginary axis crossing.

\textbf{Solution:}

\begin{enumerate}
\def\labelenumi{(\alph{enumi})}
\item
  Poles: s = 0, -1, -3, -6 (n = 4 poles, m = 0 zeros). Number of
  branches: \textbf{4}
\item
  Real-axis segments lie to the left of an odd number of poles/zeros:
\end{enumerate}

\begin{itemize}
\tightlist
\item
  Between 0 and -1: 1 pole to the right (at 0) → odd → on locus
\item
  Between -1 and -3: 2 poles to the right → even → not on locus
\item
  Between -3 and -6: 3 poles to the right → odd → on locus
\item
  Left of -6: 4 poles to the right → even → not on locus
\end{itemize}

Segments: \textbf{{[}0, -1{]} and {[}-3, -6{]}}

\begin{enumerate}
\def\labelenumi{(\alph{enumi})}
\setcounter{enumi}{2}
\item
  Asymptote angles: (2k+1) × 180°/(n-m) = (2k+1) × 45°. Angles:
  \textbf{45°, 135°, 225°, 315°} Centroid: σ\textsubscript{a} = (0 - 1 -
  3 - 6)/(4 - 0) = -10/4 = \textbf{-2.5}
\item
  Characteristic equation: s⁴ + 10s³ + 27s² + 18s + K = 0. Routh array:
  \textbar{} s⁴ \textbar{} 1 \textbar{} 27 \textbar{} K \textbar{}
  \textbar{} s³ \textbar{} 10 \textbar{} 18 \textbar{} 0 \textbar{}
  \textbar{} s² \textbar{} (270 - 18)/10 = 25.2 \textbar{} K \textbar{}
  \textbar{} \textbar{} s¹ \textbar{} (25.2 × 18 - 10K)/25.2 = (453.6 -
  10K)/25.2 \textbar{} 0 \textbar{} \textbar{} \textbar{} s⁰ \textbar{}
  K \textbar{} \textbar{} \textbar{}
\end{enumerate}

s¹ row: (453.6 - 10K)/25.2 = 0 → K = \textbf{45.36}

Auxiliary equation at K = 45.36: 25.2s² + 45.36 = 0 → s² = -1.8 → s =
±j1.342. Imaginary axis crossing at ω = \textbf{1.342 rad/s} with K =
\textbf{45.36}.

\begin{center}\rule{0.5\linewidth}{0.5pt}\end{center}

\section{Problem 4.7.5}\label{problem-4.7.5}

\textbf{Given:} A root locus has open-loop transfer function G(s)H(s) =
K(s + 5) / {[}s(s + 2)(s + 8){]}.

\textbf{Find:} (a) The number of branches, start/end points, (b)
real-axis segments, (c) asymptote angles and centroid, (d) the breakaway
point (between 0 and -2), and (e) whether the root locus crosses the
imaginary axis.

\textbf{Solution:}

\begin{enumerate}
\def\labelenumi{(\alph{enumi})}
\item
  n = 3 poles (0, -2, -8), m = 1 zero (-5). Branches: \textbf{3}. Start
  at s = 0, -2, -8 (K = 0). Two branches end at the zero s = -5 and at
  infinity. Actually: 1 branch ends at the zero s = -5, and 2 branches
  go to infinity along asymptotes.
\item
  Real-axis segments (to the left of an odd total of poles+zeros):
\end{enumerate}

\begin{itemize}
\tightlist
\item
  Between 0 and -2: 1 pole (at 0) to the right → odd → \textbf{on locus}
\item
  Between -2 and -5: 2 poles to the right → even → not on locus
\item
  Between -5 and -8: 2 poles + 1 zero to the right = 3 → odd →
  \textbf{on locus}
\item
  Left of -8: 3 poles + 1 zero = 4 → even → not on locus
\end{itemize}

Segments: \textbf{{[}0, -2{]} and {[}-5, -8{]}}

\begin{enumerate}
\def\labelenumi{(\alph{enumi})}
\setcounter{enumi}{2}
\item
  Asymptotes (n - m = 2 branches go to infinity): Angles: (2k+1) ×
  180°/2 = \textbf{90° and 270°} Centroid: σ\textsubscript{a} = (0 - 2 -
  8 - (-5))/(3 - 1) = (-10 + 5)/2 = \textbf{-2.5}
\item
  Breakaway point between 0 and -2: K = -s(s + 2)(s + 8)/(s + 5). Set
  dK/ds = 0. Using the simplified method: 1/s + 1/(s+2) + 1/(s+8) =
  1/(s+5). Numerically: at s = -0.8: 1/(-0.8) + 1/(1.2) + 1/(7.2) =
  -1.25 + 0.833 + 0.139 = -0.278. 1/(s+5): 1/4.2 = 0.238. Difference =
  -0.516. At s = -1.0: 1/(-1) + 1/(1) + 1/(7) = -1 + 1 + 0.143 = 0.143.
  1/4 = 0.25. Difference = -0.107. At s = -1.1: 1/(-1.1) + 1/(0.9) +
  1/(6.9) = -0.909 + 1.111 + 0.145 = 0.347. 1/3.9 = 0.256. Difference =
  0.091. By interpolation, breakaway at s ≈ \textbf{-1.05}
\item
  Characteristic equation: s(s+2)(s+8) + K(s+5) = s³ + 10s² + 16s + Ks +
  5K = s³ + 10s² + (16+K)s + 5K = 0. Routh s¹: {[}10(16+K) - 5K{]}/10 =
  (160 + 10K - 5K)/10 = (160 + 5K)/10. Since K \textgreater{} 0: 160 +
  5K \textgreater{} 0 always. s⁰: 5K \textgreater{} 0 for K
  \textgreater{} 0. The root locus \textbf{never crosses the imaginary
  axis} --- the system is stable for all K \textgreater{} 0.
\end{enumerate}

\begin{center}\rule{0.5\linewidth}{0.5pt}\end{center}

\section{Problem 4.7.6}\label{problem-4.7.6}

\textbf{Given:} A characteristic equation with a zero row in the Routh
array: s⁴ + 6s³ + 11s² + 6s + K = 0.

\textbf{Find:} (a) The Routh array, (b) the value of K that produces a
zero row, (c) the auxiliary polynomial and its roots at that K, and (d)
all four closed-loop poles at that K.

\textbf{Solution:}

\begin{enumerate}
\def\labelenumi{(\alph{enumi})}
\item
  Routh array: \textbar{} s⁴ \textbar{} 1 \textbar{} 11 \textbar{} K
  \textbar{} \textbar{} s³ \textbar{} 6 \textbar{} 6 \textbar{} 0
  \textbar{} \textbar{} s² \textbar{} (66-6)/6 = 10 \textbar{} K
  \textbar{} \textbar{} \textbar{} s¹ \textbar{} (60 - 6K)/10 \textbar{}
  0 \textbar{} \textbar{} \textbar{} s⁰ \textbar{} K \textbar{}
  \textbar{} \textbar{}
\item
  Zero s¹ row: 60 - 6K = 0 → K = \textbf{10}
\item
  At K = 10, the auxiliary polynomial is from the s² row: 10s² + 10 = 0.
  s² = -1 → s = \textbf{±j1} (purely imaginary poles --- sustained
  oscillations at ω = 1 rad/s).
\item
  At K = 10: s⁴ + 6s³ + 11s² + 6s + 10 = 0. Factor out (s² + 1): s⁴ +
  6s³ + 11s² + 6s + 10 = (s² + 1)(s² + 6s + 10) = 0. Remaining roots: s
  = (-6 ± √(36 - 40))/2 = (-6 ± j2)/2 = \textbf{-3 ± j1}.
\end{enumerate}

All four poles: s = \textbf{±j1} (marginally stable) and s = \textbf{-3
± j1} (stable).

\begin{center}\rule{0.5\linewidth}{0.5pt}\end{center}

\section{Problem 4.7.7}\label{problem-4.7.7}

\textbf{Given:} A unity feedback system has G(s) = K / {[}(s + 1)(s² +
4s + 13){]}. The denominator s² + 4s + 13 has complex poles at s = -2 ±
j3.

\textbf{Find:} (a) All open-loop poles, (b) the root locus real-axis
segments, (c) asymptote angles and centroid, (d) the range of K for
stability, and (e) the K value at the imaginary axis crossing.

\textbf{Solution:}

\begin{enumerate}
\def\labelenumi{(\alph{enumi})}
\item
  Open-loop poles: s = \textbf{-1}, s = \textbf{-2 + j3}, s = \textbf{-2
  - j3} (n = 3, m = 0).
\item
  Complex poles are not on the real axis, so only the real pole at -1
  matters for the real-axis test. To the left of -1: 1 real pole to the
  right → odd → on locus. Segment: \textbf{(-∞, -1{]}} --- however, this
  is only valid to the left of -1 with no other real poles/zeros.
\end{enumerate}

Actually: Between 0 and -1: 0 poles/zeros to the right → even → not on
locus (the pole at -1 is at the boundary). Left of -1: 1 real pole to
the right → odd → \textbf{on locus}. Real-axis segment: \textbf{(-∞,
-1{]}}

\begin{enumerate}
\def\labelenumi{(\alph{enumi})}
\setcounter{enumi}{2}
\item
  Asymptote angles: (2k+1) × 180°/3 = \textbf{60°, 180°, 300°} Centroid:
  σ\textsubscript{a} = (-1 - 2 - 2)/3 = \textbf{-5/3 = -1.667}
\item
  Characteristic equation: (s + 1)(s² + 4s + 13) + K = s³ + 5s² + 17s +
  13 + K = 0. Routh array: \textbar{} s³ \textbar{} 1 \textbar{} 17
  \textbar{} \textbar{} s² \textbar{} 5 \textbar{} 13 + K \textbar{}
  \textbar{} s¹ \textbar{} (85 - 13 - K)/5 = (72 - K)/5 \textbar{} 0
  \textbar{} \textbar{} s⁰ \textbar{} 13 + K \textbar{} \textbar{}
\end{enumerate}

Stability: K \textgreater{} -13 (always true for positive K) and K
\textless{} 72. Range: \textbf{0 \textless{} K \textless{} 72} (or -13
\textless{} K \textless{} 72 allowing negative gain)

\begin{enumerate}
\def\labelenumi{(\alph{enumi})}
\setcounter{enumi}{4}
\tightlist
\item
  At K = 72: auxiliary equation from s² row: 5s² + 85 = 0 → s² = -17 → s
  = ±j√17 = ±j4.123. Imaginary axis crossing at ω = \textbf{4.123 rad/s}
  with K = \textbf{72}.
\end{enumerate}

\begin{center}\rule{0.5\linewidth}{0.5pt}\end{center}

\section{Problem 4.7.8}\label{problem-4.7.8}

\textbf{Given:} A root locus with open-loop transfer function G(s)H(s) =
K / {[}s²(s + 4){]} (double pole at origin).

\textbf{Find:} (a) Number of branches and asymptotes, (b) the asymptote
angles and centroid, (c) the breakaway angle at the double pole at the
origin, (d) whether the system is stable for any K \textgreater{} 0, and
(e) the number of RHP poles for K = 10.

\textbf{Solution:}

\begin{enumerate}
\def\labelenumi{(\alph{enumi})}
\item
  n = 3 poles (double at 0, one at -4), m = 0 zeros. Branches:
  \textbf{3}. All 3 go to infinity along asymptotes.
\item
  Asymptote angles: (2k+1) × 180°/3 = \textbf{60°, 180°, 300°} Centroid:
  σ\textsubscript{a} = (0 + 0 - 4)/3 = \textbf{-1.333}
\item
  The two branches starting from the double pole at the origin depart at
  ±90° (perpendicular to the real axis) because they must be symmetric
  about the real axis and cannot continue along the real axis (the
  segment to the left of -4 is on the locus, but between 0 and -4 there
  are 2 poles to the right --- even --- so it is not on the locus).
  Departure angle: \textbf{±90°}
\item
  Characteristic equation: s³ + 4s² + K = 0. Routh array: \textbar{} s³
  \textbar{} 1 \textbar{} 0 \textbar{} \textbar{} s² \textbar{} 4
  \textbar{} K \textbar{} \textbar{} s¹ \textbar{} (0 - K)/4 = -K/4
  \textbar{} 0 \textbar{} \textbar{} s⁰ \textbar{} K \textbar{}
  \textbar{}
\end{enumerate}

s¹ row: -K/4 \textless{} 0 for all K \textgreater{} 0. The system is
\textbf{unstable for all K \textgreater{} 0}.

\begin{enumerate}
\def\labelenumi{(\alph{enumi})}
\setcounter{enumi}{4}
\tightlist
\item
  At K = 10: first-column entries are 1, 4, -2.5, 10. There are two sign
  changes (4 to -2.5, then -2.5 to 10), so there are \textbf{2 RHP
  poles} (with 1 LHP pole).
\end{enumerate}

\begin{center}\rule{0.5\linewidth}{0.5pt}\end{center}

\section{Problem 4.7.9}\label{problem-4.7.9}

\textbf{Given:} A system has the characteristic equation s⁵ + 2s⁴ + 4s³
+ 8s² + 3s + 6 = 0.

\textbf{Find:} (a) The Routh array, (b) the number of RHP poles, and (c)
whether the system is stable.

\textbf{Solution:}

\begin{enumerate}
\def\labelenumi{(\alph{enumi})}
\tightlist
\item
  Routh array: \textbar{} s⁵ \textbar{} 1 \textbar{} 4 \textbar{} 3
  \textbar{} \textbar{} s⁴ \textbar{} 2 \textbar{} 8 \textbar{} 6
  \textbar{} \textbar{} s³ \textbar{} (8 - 8)/2 = 0 \textbar{} (6 - 6)/2
  = 0 \textbar{} \textbar{}
\end{enumerate}

The entire s³ row is zero. This indicates symmetric root patterns. Use
the auxiliary polynomial from the s⁴ row: P(s) = 2s⁴ + 8s² + 6. Take the
derivative: dP/ds = 8s³ + 16s.

Replace the s³ row with coefficients of 8s³ + 16s: \textbar{} s³
\textbar{} 8 \textbar{} 16 \textbar{} \textbar{} \textbar{} s²
\textbar{} (64 - 32)/8 = 4 \textbar{} 6 \textbar{} \textbar{} \textbar{}
s¹ \textbar{} (64 - 48)/4 = 4 \textbar{} 0 \textbar{} \textbar{}
\textbar{} s⁰ \textbar{} 6 \textbar{} \textbar{} \textbar{}

\begin{enumerate}
\def\labelenumi{(\alph{enumi})}
\setcounter{enumi}{1}
\item
  First column (from s⁵ down): 1, 2, 8, 4, 4, 6. No sign changes →
  \textbf{0 RHP poles}.
\item
  Although no poles are in the RHP, the zero row indicates poles on the
  imaginary axis. Solve 2s⁴ + 8s² + 6 = 0: let u = s², then 2u² + 8u + 6
  = 0 → u = (-8 ± √(64-48))/4 = (-8 ± 4)/4. u = -1 → s = ±j1, and u = -3
  → s = ±j√3.
\end{enumerate}

The system has poles on the imaginary axis at s = ±j1 and s = ±j√3,
making it \textbf{marginally stable} (sustained oscillations, not
asymptotically stable).

\begin{center}\rule{0.5\linewidth}{0.5pt}\end{center}

\section{Problem 4.7.10}\label{problem-4.7.10}

\textbf{Given:} A unity feedback system has G(s) = K(s + 3) / {[}s(s +
1)(s + 2)(s + 6){]}.

\textbf{Find:} (a) The number of root locus branches and asymptotes, (b)
the asymptote angles and centroid, (c) the breakaway point between s =
-1 and s = -2, (d) the break-in point between s = -3 and s = -6, and (e)
the maximum K for stability.

\textbf{Solution:}

\begin{enumerate}
\def\labelenumi{(\alph{enumi})}
\item
  n = 4 poles (0, -1, -2, -6), m = 1 zero (-3). Branches: \textbf{4}.
  Asymptotic branches: n - m = \textbf{3}.
\item
  Asymptote angles: (2k+1) × 180°/3 = \textbf{60°, 180°, 300°} Centroid:
  σ\textsubscript{a} = (0 - 1 - 2 - 6 - (-3))/3 = (-9 + 3)/3 =
  \textbf{-2.0}
\item
  1/(s+3) = 1/s + 1/(s+1) + 1/(s+2) + 1/(s+6). This is derived from
  dK/ds = 0. Between s = -1 and s = -2, try s = -1.5: LHS: 1/(-1.5+3) =
  1/1.5 = 0.667 RHS: 1/(-1.5) + 1/(-0.5) + 1/(0.5) + 1/(4.5) = -0.667 -
  2.0 + 2.0 + 0.222 = -0.445 Not equal. Try s = -1.4: LHS: 1/1.6 =
  0.625. RHS: 1/(-1.4) + 1/(-0.4) + 1/(0.6) + 1/(4.6) = -0.714 - 2.5 +
  1.667 + 0.217 = -1.330 Try s = -1.2: LHS: 1/1.8 = 0.556. RHS: 1/(-1.2)
  + 1/(-.2) + 1/(0.8) + 1/(4.8) = -0.833 - 5.0 + 1.25 + 0.208 = -4.375
\end{enumerate}

The breakaway actually occurs very close to the midpoint. Using K =
-s(s+1)(s+2)(s+6)/(s+3): dK/ds = 0 is solved numerically. At s = -0.446:
K = 0.446 × 0.554 × 1.554 × 5.554 / 2.554 = 0.834. Breakaway point ≈ s =
\textbf{-0.45}

\begin{enumerate}
\def\labelenumi{(\alph{enumi})}
\setcounter{enumi}{3}
\item
  Break-in between -3 and -6 by numerical evaluation: at s = -4.3, K =
  4.3 × 3.3 × 2.3 × 1.7 / 1.3 = 42.7. This is a local minimum of K.
  Break-in point ≈ s = \textbf{-4.3}
\item
  Characteristic equation: s(s+1)(s+2)(s+6) + K(s+3) = 0. Expand: s⁴ +
  9s³ + 20s² + 12s + Ks + 3K = s⁴ + 9s³ + 20s² + (12+K)s + 3K = 0.
\end{enumerate}

Routh array: \textbar{} s⁴ \textbar{} 1 \textbar{} 20 \textbar{} 3K
\textbar{} \textbar{} s³ \textbar{} 9 \textbar{} 12+K \textbar{} 0
\textbar{} \textbar{} s² \textbar{} (180-12-K)/9 = (168-K)/9 \textbar{}
3K \textbar{} \textbar{} \textbar{} s¹ \textbar{} {[}(168-K)(12+K)/9 -
27K{]} / {[}(168-K)/9{]} \textbar{} 0 \textbar{} \textbar{} \textbar{}
s⁰ \textbar{} 3K \textbar{} \textbar{} \textbar{}

s¹ numerator: (168-K)(12+K) - 243K = 2016 + 168K - 12K - K² - 243K =
2016 - 87K - K². Set to zero: K² + 87K - 2016 = 0 → K = (-87 + √(7569 +
8064))/2 = (-87 + √15633)/2 = (-87 + 125.03)/2 = \textbf{19.0}

Maximum K for stability: K = \textbf{19.0}

\chapter{Chapter 4 --- Section 4.8: Frequency Response
Design}\label{chapter-4-section-4.8-frequency-response-design}

Practice problems covering Bode plots, gain and phase margins, Nyquist
criterion, lead compensator design, lag compensator design, lead-lag
compensation, and bandwidth analysis.

\begin{center}\rule{0.5\linewidth}{0.5pt}\end{center}

\section{Problem 4.8.1}\label{problem-4.8.1}

\textbf{Given:} A system has the open-loop transfer function G(s) = 200
/ {[}s(s + 5)(s + 20){]}.

\textbf{Find:} (a) The magnitude in dB and phase at ω = 1, 5, 10, and 20
rad/s, (b) the gain crossover frequency (where \textbar G(jω)\textbar{}
= 0 dB), and (c) the low-frequency and high-frequency asymptotic slopes.

\textbf{Solution:}

\begin{enumerate}
\def\labelenumi{(\alph{enumi})}
\tightlist
\item
  \textbar G(jω)\textbar{} = 200 / {[}ω × √(ω² + 25) × √(ω² + 400){]}
\end{enumerate}

At ω = 1: \textbar G\textbar{} = 200/(1 × √26 × √401) = 200/(1 × 5.099 ×
20.025) = 200/102.1 = 1.96 → \textbf{5.84 dB} Phase: -90° - arctan(1/5)
- arctan(1/20) = -90° - 11.31° - 2.86° = \textbf{-104.2°}

At ω = 5: \textbar G\textbar{} = 200/(5 × √50 × √425) = 200/(5 × 7.071 ×
20.616) = 200/728.9 = 0.274 → \textbf{-11.24 dB} Phase: -90° -
arctan(5/5) - arctan(5/20) = -90° - 45° - 14.04° = \textbf{-149.0°}

At ω = 10: \textbar G\textbar{} = 200/(10 × √125 × √500) = 200/(10 ×
11.18 × 22.36) = 200/2499 = 0.0800 → \textbf{-21.94 dB} Phase: -90° -
arctan(10/5) - arctan(10/20) = -90° - 63.43° - 26.57° = \textbf{-180.0°}

At ω = 20: \textbar G\textbar{} = 200/(20 × √425 × √800) = 200/(20 ×
20.616 × 28.284) = 200/11,664 = 0.01714 → \textbf{-35.32 dB} Phase: -90°
- arctan(20/5) - arctan(20/20) = -90° - 75.96° - 45° = \textbf{-210.96°}

\begin{enumerate}
\def\labelenumi{(\alph{enumi})}
\setcounter{enumi}{1}
\item
  From the data, gain crosses 0 dB between ω = 1 and ω = 5. Solving
  numerically: At ω = 3: \textbar G\textbar{} = 200/(3 × √34 × √409) =
  200/(3 × 5.831 × 20.224) = 200/353.8 = 0.565 → -4.96 dB. At ω = 2:
  \textbar G\textbar{} = 200/(2 × √29 × √404) = 200/(2 × 5.385 × 20.1) =
  200/216.5 = 0.924 → -0.69 dB. At ω = 1.9: \textbar G\textbar{} =
  200/(1.9 × √28.61 × √403.61) = 200/(1.9 × 5.349 × 20.09) = 200/204.1 =
  0.980 → -0.18 dB. Gain crossover: ω\textsubscript{gc} ≈ \textbf{1.95
  rad/s}
\item
  Low frequency (ω \textless\textless{} 5): \textbar G\textbar{} ≈
  200/(ω × 5 × 20) = 2/ω → slope of \textbf{-20 dB/decade}. High
  frequency (ω \textgreater\textgreater{} 20): \textbar G\textbar{} ≈
  200/(ω³) → slope of \textbf{-60 dB/decade}.
\end{enumerate}

\begin{center}\rule{0.5\linewidth}{0.5pt}\end{center}

\section{Problem 4.8.2}\label{problem-4.8.2}

\textbf{Given:} A unity feedback system has G(s) = 40 / {[}(s + 2)(s +
10){]}.

\textbf{Find:} (a) The gain crossover frequency, (b) the phase at gain
crossover, (c) the phase margin, (d) the phase crossover frequency
(where phase = -180°), and (e) the gain margin.

\textbf{Solution:}

\begin{enumerate}
\def\labelenumi{(\alph{enumi})}
\item
  \textbar G(jω)\textbar{} = 40 / {[}√(ω² + 4) × √(ω² + 100){]} = 1. 40²
  = (ω² + 4)(ω² + 100). Let u = ω²: 1600 = u² + 104u + 400 → u² + 104u -
  1200 = 0. u = (-104 + √(10816 + 4800))/2 = (-104 + √15616)/2 = (-104 +
  124.96)/2 = 10.48. ω\textsubscript{gc} = √10.48 = \textbf{3.24 rad/s}
\item
  Phase at ω\textsubscript{gc}: ∠G = -arctan(3.24/2) - arctan(3.24/10) =
  -58.3° - 17.95° = \textbf{-76.3°}
\item
  Phase margin: PM = 180° - 76.3° = \textbf{103.7°} (very large --- this
  is a Type 0 system with no integrators, hence very stable)
\item
  Phase = -180° requires -arctan(ω/2) - arctan(ω/10) = -180°. Since the
  maximum phase lag from two real poles is -180° only as ω → ∞, there is
  \textbf{no finite phase crossover frequency}.
\item
  Since the phase never reaches -180° at any finite frequency, the gain
  margin is \textbf{infinite} (∞ dB). The system is stable for all
  finite gains (though with increasing steady-state error as gain
  decreases).
\end{enumerate}

\begin{center}\rule{0.5\linewidth}{0.5pt}\end{center}

\section{Problem 4.8.3}\label{problem-4.8.3}

\textbf{Given:} An open-loop transfer function G(s) = 100 / {[}s(s +
5)(s + 20){]} with unity feedback.

\textbf{Find:} (a) The phase crossover frequency, (b) the gain at the
phase crossover frequency, (c) the gain margin in dB, (d) the gain
crossover frequency, and (e) the phase margin.

\textbf{Solution:}

\begin{enumerate}
\def\labelenumi{(\alph{enumi})}
\item
  Phase = -180°: -90° - arctan(ω/5) - arctan(ω/20) = -180°. arctan(ω/5)
  + arctan(ω/20) = 90°. Using the identity: tan⁻¹(a) + tan⁻¹(b) = 90°
  when ab = 1. (ω/5)(ω/20) = 1 → ω²/100 = 1 → ω = \textbf{10 rad/s}
\item
  \textbar G(j10)\textbar{} = 100/(10 × √(100+25) × √(100+400)) =
  100/(10 × 11.18 × 22.36) = 100/2499 = 0.04002
\item
  GM = 20 log₁₀(1/0.04002) = 20 log₁₀(24.99) = 20 × 1.398 =
  \textbf{27.96 dB}
\item
  \textbar G(jω)\textbar{} = 1: 100/(ω√(ω²+25)√(ω²+400)) = 1. 10,000 =
  ω²(ω²+25)(ω²+400). Try ω = 1.5: 2.25 × 27.25 × 402.25 = 24,659 (too
  high). Try ω = 2: 4 × 29 × 404 = 46,864 (too high). Try ω = 1: 1 × 26
  × 401 = 10,426 (close). Try ω = 0.99: 0.98 × 25.98 × 400.98 = 10,205
  (close). ω\textsubscript{gc} ≈ \textbf{1.0 rad/s}
\item
  Phase at ω\textsubscript{gc} = 1: -90° - arctan(1/5) - arctan(1/20) =
  -90° - 11.31° - 2.86° = -104.2°. PM = 180° - 104.2° = \textbf{75.8°}
  (very comfortable margin)
\end{enumerate}

\begin{center}\rule{0.5\linewidth}{0.5pt}\end{center}

\section{Problem 4.8.4}\label{problem-4.8.4}

\textbf{Given:} A system has G(jω)H(jω) that, when plotted as a Nyquist
diagram, crosses the negative real axis at -0.6 + j0 at ω = 8 rad/s. The
system has no open-loop RHP poles (P = 0) and makes zero clockwise
encirclements of the (-1, 0) point.

\textbf{Find:} (a) Whether the closed-loop system is stable, (b) the
gain margin, (c) the factor by which the gain could be increased before
instability, and (d) the gain margin in dB.

\textbf{Solution:}

\begin{enumerate}
\def\labelenumi{(\alph{enumi})}
\item
  Z = P + N = 0 + 0 = 0. No closed-loop RHP poles → \textbf{stable}.
\item
  The Nyquist plot crosses the real axis at -0.6. Gain margin =
  1/\textbar real axis crossing\textbar{} = 1/0.6 = \textbf{1.667}
\item
  The gain can be multiplied by \textbf{1.667} before the Nyquist plot
  passes through (-1, 0).
\item
  GM = 20 log₁₀(1.667) = 20 × 0.2218 = \textbf{4.44 dB}
\end{enumerate}

This is below the typical recommended minimum of 6 dB, indicating the
system has marginal stability.

\begin{center}\rule{0.5\linewidth}{0.5pt}\end{center}

\section{Problem 4.8.5}\label{problem-4.8.5}

\textbf{Given:} A unity feedback system has G(s) = 15 / {[}s(s + 3){]}.
The current phase margin is PM = 40°. A lead compensator is needed to
increase the phase margin to 55°.

\textbf{Find:} (a) The current gain crossover frequency, (b) the
additional phase needed from the lead compensator, (c) the lead
compensator parameters (zero z, pole p, gain K\textsubscript{c}), and
(d) the new gain crossover frequency.

\textbf{Solution:}

\begin{enumerate}
\def\labelenumi{(\alph{enumi})}
\tightlist
\item
  \textbar G(jω)\textbar{} = 15/(ω√(ω²+9)) = 1 → 225 = ω²(ω²+9) → ω⁴ +
  9ω² - 225 = 0. ω² = (-9 + √(81+900))/2 = (-9 + 31.32)/2 = 11.16.
  ω\textsubscript{gc} = √11.16 = \textbf{3.34 rad/s}
\end{enumerate}

Phase: -90° - arctan(3.34/3) = -90° - 48.1° = -138.1°. PM = 180° -
138.1° = 41.9° (confirms ≈ 40°).

\begin{enumerate}
\def\labelenumi{(\alph{enumi})}
\setcounter{enumi}{1}
\item
  Additional phase: 55° - 41.9° + 8° (safety margin for crossover shift)
  = \textbf{21.1°}
\item
  sin(φ\textsubscript{max}) = sin(21.1°) = 0.360. α = (1 - 0.360)/(1 +
  0.360) = 0.640/1.360 = \textbf{0.471}
\end{enumerate}

New gain crossover frequency (slightly shifted; place the lead
compensator peak here): ω\textsubscript{gc,new} ≈ ω\textsubscript{gc} /
√(√α) --- actually, we target the lead peak at the new crossover. Use
ω\textsubscript{m} = ω\textsubscript{gc} as a first approximation. The
lead compensator attenuation at ω\textsubscript{m} is √α = 0.686, and
\textbar G(jω\textsubscript{gc})\textbar{} = 1, so K\textsubscript{c} =
1/√α = 1/0.686 = \textbf{1.458}

Compensator zero and pole: z = ω\textsubscript{gc} × √α = 3.34 × 0.686 =
\textbf{2.29 rad/s} p = ω\textsubscript{gc} / √α = 3.34 / 0.686 =
\textbf{4.87 rad/s}

Lead compensator: G\textsubscript{c}(s) = \textbf{1.458(s + 2.29) / (s +
4.87)}

\begin{enumerate}
\def\labelenumi{(\alph{enumi})}
\setcounter{enumi}{3}
\tightlist
\item
  The new gain crossover frequency is approximately
  ω\textsubscript{gc,new} ≈ \textbf{3.34 rad/s} (the lead compensator is
  centered at this frequency to provide maximum phase boost there).
\end{enumerate}

\begin{center}\rule{0.5\linewidth}{0.5pt}\end{center}

\section{Problem 4.8.6}\label{problem-4.8.6}

\textbf{Given:} A unity feedback system has G(s) = 10 / {[}s(s + 1)(s +
5){]}. The steady-state error to a ramp input must be reduced from the
current value to e\textsubscript{ss} = 0.1, while maintaining the
current phase margin (approximately 50°).

\textbf{Find:} (a) The current K\textsubscript{v} and ramp error, (b)
the required gain increase, and (c) the lag compensator parameters to
achieve the gain increase without significantly affecting the phase
margin.

\textbf{Solution:}

\begin{enumerate}
\def\labelenumi{(\alph{enumi})}
\item
  K\textsubscript{v} = lim(s→0) sG(s) = 10/(1 × 5) = 2.
  e\textsubscript{ss} = 1/K\textsubscript{v} = 1/2 = \textbf{0.5}
\item
  Required K\textsubscript{v} = 1/0.1 = 10. Gain increase needed: 10/2 =
  \textbf{5} (or 13.98 dB).
\item
  A lag compensator G\textsubscript{c}(s) = (s +
  z\textsubscript{lag})/(s + p\textsubscript{lag}) where
  z\textsubscript{lag}/p\textsubscript{lag} = 5 (the required gain
  increase).
\end{enumerate}

First, find the current gain crossover: \textbar G(jω)\textbar{} =
10/(ω√(ω²+1)√(ω²+25)) = 1. Try ω = 1.3: 10/(1.3 × √2.69 × √26.69) =
10/(1.3 × 1.640 × 5.166) = 10/11.01 = 0.908. Close to 1.
ω\textsubscript{gc} ≈ \textbf{1.25 rad/s}

Place the lag compensator corner frequencies well below
ω\textsubscript{gc} (at least one decade below): z\textsubscript{lag} =
ω\textsubscript{gc}/10 = \textbf{0.125 rad/s} p\textsubscript{lag} =
z\textsubscript{lag}/5 = \textbf{0.025 rad/s}

Lag compensator: G\textsubscript{c}(s) = \textbf{(s + 0.125) / (s +
0.025)}

At DC: G\textsubscript{c}(0) = 0.125/0.025 = 5. This provides the
required 5× gain increase.

At ω = ω\textsubscript{gc} = 1.25: the lag compensator magnitude ≈ 1
(since both corners are far below crossover) and phase ≈
-arctan(1.25/0.125) + arctan(1.25/0.025) ≈ -84.3° + 88.9° = +4.6°. The
net phase contribution is small, preserving the phase margin.

\begin{center}\rule{0.5\linewidth}{0.5pt}\end{center}

\section{Problem 4.8.7}\label{problem-4.8.7}

\textbf{Given:} A unity feedback system has G(s) = K / {[}s(s + 4){]}.
The specifications are: steady-state ramp error ≤ 0.05 and phase margin
≥ 50°.

\textbf{Find:} (a) The minimum K from the ramp error requirement, (b)
the phase margin at that K, (c) whether lead compensation is needed, and
(d) if so, design a lead compensator.

\textbf{Solution:}

\begin{enumerate}
\def\labelenumi{(\alph{enumi})}
\item
  K\textsubscript{v} = lim(s→0) sG(s) = K/4. For e\textsubscript{ss} ≤
  0.05: K\textsubscript{v} ≥ 20 → K/4 ≥ 20 → K ≥ \textbf{80}
\item
  At K = 80: \textbar G(jω)\textbar{} = 80/(ω√(ω²+16)) = 1 → 6400 =
  ω²(ω²+16) → ω⁴+16ω²-6400 = 0. ω² = (-16+√(256+25600))/2 =
  (-16+160.8)/2 = 72.4. ω\textsubscript{gc} = 8.51 rad/s. Phase: -90° -
  arctan(8.51/4) = -90° - 64.8° = -154.8°. PM = 180° - 154.8° =
  \textbf{25.2°} (below the 50° requirement).
\item
  PM = 25.2° \textless{} 50° → \textbf{yes, lead compensation is
  needed}.
\item
  Additional phase: 50° - 25.2° + 7° (safety) = \textbf{31.8°}. α = (1 -
  sin 31.8°)/(1 + sin 31.8°) = (1 - 0.527)/(1 + 0.527) = 0.473/1.527 =
  \textbf{0.310}
\end{enumerate}

The lead compensator attenuates by √α = 0.557 at ω\textsubscript{m}, so
the new crossover shifts. Find ω where \textbar KG\textbar{} = 1/√α:
80/(ω√(ω²+16)) = 1/0.557 = 1.796 → ω√(ω²+16) = 44.54. Try ω = 6: 6√52 =
43.27 (close). ω\textsubscript{m} ≈ \textbf{6.1 rad/s}.

z = ω\textsubscript{m}√α = 6.1 × 0.557 = \textbf{3.40 rad/s} p =
ω\textsubscript{m}/√α = 6.1/0.557 = \textbf{10.95 rad/s}

Lead compensator: G\textsubscript{c}(s) = \textbf{(s + 3.40) / (s +
10.95)}

\begin{center}\rule{0.5\linewidth}{0.5pt}\end{center}

\section{Problem 4.8.8}\label{problem-4.8.8}

\textbf{Given:} A system has transfer function G(s) = 50(s + 2) / {[}s(s
+ 10)(s + 50){]}.

\textbf{Find:} (a) The Bode magnitude plot corner frequencies and
slopes, (b) the low-frequency asymptote magnitude at ω = 1, (c) the
high-frequency asymptotic slope, and (d) the approximate gain crossover
frequency from the asymptotic plot.

\textbf{Solution:}

\begin{enumerate}
\def\labelenumi{(\alph{enumi})}
\tightlist
\item
  Rewrite in standard form: G(s) = 50 × 2 × (s/2 + 1) / {[}s × 10 × 50 ×
  (s/10 + 1)(s/50 + 1){]} G(s) = 0.2(s/2 + 1) / {[}s(s/10 + 1)(s/50 +
  1){]}
\end{enumerate}

Corner frequencies: ω = \textbf{2 rad/s} (zero), ω = \textbf{10 rad/s}
(pole), ω = \textbf{50 rad/s} (pole).

Slopes: - ω \textless{} 2: -20 dB/dec (from 1/s) - 2 \textless{} ω
\textless{} 10: -20 + 20 = 0 dB/dec (zero cancels the 1/s slope) - 10
\textless{} ω \textless{} 50: 0 - 20 = -20 dB/dec - ω \textgreater{} 50:
-20 - 20 = \textbf{-40 dB/dec}

\begin{enumerate}
\def\labelenumi{(\alph{enumi})}
\setcounter{enumi}{1}
\item
  At ω = 1 (low frequency): \textbar G\textbar{} ≈ 0.2/ω = 0.2/1 = 0.2 →
  20 log₁₀(0.2) = \textbf{-14 dB}
\item
  High-frequency slope: \textbf{-40 dB/decade}
\item
  From the asymptotic plot, the magnitude is -14 dB at ω = 1 and rises
  at -20 dB/dec, so it reaches 0 dB: -14 + 20 log₁₀(ω) = 0 is wrong
  since slope is -20 dB/dec. At ω = 1: -14 dB. The magnitude decreases
  at -20 dB/dec until ω = 2 where the zero flattens it. At ω = 2: -14 +
  20 log₁₀(1) = -14 dB\ldots{} Actually, magnitude at ω = 2 on the -20
  dB/dec line: \textbar G\textbar{} at ω = 2 = 0.2/2 = 0.1 → -20 dB. But
  the zero starts at ω = 2, flattening to 0 dB/dec. So the magnitude
  stays at -20 dB until ω = 10. Then pole at ω = 10 resumes -20 dB/dec.
  The magnitude never reaches 0 dB.
\end{enumerate}

Gain crossover: \textbf{does not cross 0 dB} (the system gain is too
low). The maximum asymptotic magnitude is about -14 dB at ω = 1,
decreasing from there. With additional gain of at least 14 dB (factor of
5), gain crossover would occur.

\begin{center}\rule{0.5\linewidth}{0.5pt}\end{center}

\section{Problem 4.8.9}\label{problem-4.8.9}

\textbf{Given:} The open-loop Nyquist plot of a system with two
open-loop RHP poles (P = 2) makes two counterclockwise (CCW)
encirclements of the (-1, 0) point. CCW encirclements are counted as N =
-2 (negative).

\textbf{Find:} (a) The number of closed-loop RHP poles, (b) whether the
closed-loop system is stable, and (c) what would happen if the gain were
increased so that the plot made only one CCW encirclement.

\textbf{Solution:}

\begin{enumerate}
\def\labelenumi{(\alph{enumi})}
\item
  Z = P + N = 2 + (-2) = \textbf{0} closed-loop RHP poles.
\item
  Z = 0 → \textbf{the closed-loop system is stable}. The feedback has
  stabilized an open-loop unstable plant.
\item
  With N = -1: Z = P + N = 2 + (-1) = \textbf{1} closed-loop RHP pole.
  The closed-loop system would be \textbf{unstable} with one RHP pole.
  This shows that the system is conditionally stable --- it is stable
  only within a specific range of gain.
\end{enumerate}

\begin{center}\rule{0.5\linewidth}{0.5pt}\end{center}

\section{Problem 4.8.10}\label{problem-4.8.10}

\textbf{Given:} A lead-lag compensator is needed for G(s) = 5 / {[}s(s +
1){]} with unity feedback. Requirements: K\textsubscript{v} ≥ 50 and PM
≥ 45°.

\textbf{Find:} (a) The gain K needed for the K\textsubscript{v}
requirement, (b) the uncompensated phase margin at that gain, (c) the
lag compensator ratio to provide the necessary gain, and (d) the lead
compensator to restore the phase margin.

\textbf{Solution:}

\begin{enumerate}
\def\labelenumi{(\alph{enumi})}
\item
  K\textsubscript{v} = lim(s→0) s × K × 5/{[}s(s+1){]} = 5K. For
  K\textsubscript{v} ≥ 50: 5K ≥ 50 → K ≥ \textbf{10}.
\item
  At K = 10: G(s) = 50/{[}s(s+1){]}. \textbar G(jω)\textbar{} =
  50/(ω√(ω²+1)) = 1. 2500 = ω²(ω²+1) → ω⁴ + ω² - 2500 = 0. ω² =
  (-1+√(1+10000))/2 = (-1+100.005)/2 = 49.5. ω\textsubscript{gc} = 7.04
  rad/s. Phase: -90° - arctan(7.04) = -90° - 81.9° = -171.9°. PM = 180°
  - 171.9° = \textbf{8.1°} (very poor).
\item
  Choose a new crossover frequency where the uncompensated plant has
  sufficient phase. For PM = 45° + lead contribution, target the
  compensated crossover at a lower frequency. After lag compensation
  reduces the crossover to ω\textsubscript{gc,new} ≈ 3 rad/s:
\end{enumerate}

Uncompensated gain at ω = 3: \textbar G(j3)\textbar{} = 50/(3√10) =
50/9.487 = 5.27 → 14.44 dB. The lag compensator must attenuate by 14.44
dB (factor of 5.27) at ω = 3 but maintain DC gain.

Lag: z\textsubscript{lag}/p\textsubscript{lag} = 1 (no DC gain change
needed --- the K = 10 already provides it). Actually, we need the lag to
reduce gain at ω = 3. Use a lag with high-frequency attenuation:
z\textsubscript{lag} = 0.3 rad/s, p\textsubscript{lag} = 0.3/5.27 =
\textbf{0.057 rad/s}

\begin{enumerate}
\def\labelenumi{(\alph{enumi})}
\setcounter{enumi}{3}
\tightlist
\item
  Phase at ω = 3 without lead: -90° - arctan(3) = -90° - 71.6° =
  -161.6°. Phase from lag at ω = 3: ≈ -5° (small lag penalty). Total:
  -166.6°. PM would be 13.4° --- still needs lead.
\end{enumerate}

Additional phase needed: 45° - 13.4° + 5° = 36.6°. α = (1 - sin
36.6°)/(1 + sin 36.6°) = (1 - 0.596)/(1 + 0.596) = 0.404/1.596 = 0.253.
z\textsubscript{lead} = 3√0.253 = \textbf{1.51 rad/s},
p\textsubscript{lead} = 3/√0.253 = \textbf{5.97 rad/s}.

Lead compensator: G\textsubscript{lead}(s) = \textbf{(s + 1.51) / (s +
5.97)} Lag compensator: G\textsubscript{lag}(s) = \textbf{(s + 0.3) / (s
+ 0.057)}

\chapter{Chapter 4 --- Section 4.9: State-Space
Representation}\label{chapter-4-section-4.9-state-space-representation}

Practice problems covering state-space models, matrix formulation,
eigenvalue analysis, controllability, observability, state feedback pole
placement, and Luenberger observer design.

\begin{center}\rule{0.5\linewidth}{0.5pt}\end{center}

\section{Problem 4.9.1}\label{problem-4.9.1}

\textbf{Given:} A second-order mechanical system (mass-spring-damper)
has mass m = 2 kg, spring constant k = 18 N/m, and damping coefficient b
= 8 N·s/m. The state variables are x₁ = position (m) and x₂ = velocity
(m/s), the input u is an applied force (N), and the output y is the
position.

\textbf{Find:} (a) The state-space matrices A, B, C, and D, (b) the
eigenvalues of A, and (c) whether the system is stable.

\textbf{Solution:}

From Newton's second law: mẍ + bẋ + kx = u, which gives ẍ = -(b/m)ẋ -
(k/m)x + (1/m)u.

Defining ẋ₁ = x₂ and ẋ₂ = -(k/m)x₁ - (b/m)x₂ + (1/m)u:

\begin{enumerate}
\def\labelenumi{(\alph{enumi})}
\tightlist
\item
  \textbf{A} = {[}0, 1; -9, -4{]}, \textbf{B} = {[}0; 0.5{]}, \textbf{C}
  = {[}1, 0{]}, \textbf{D} = {[}0{]}
\end{enumerate}

where k/m = 18/2 = 9, b/m = 8/2 = 4, and 1/m = 1/2 = 0.5.

\begin{enumerate}
\def\labelenumi{(\alph{enumi})}
\setcounter{enumi}{1}
\tightlist
\item
  Eigenvalues from det(sI - A) = 0:
\end{enumerate}

s² + 4s + 9 = 0

s = (-4 ± √(16 - 36)) / 2 = (-4 ± √(-20)) / 2 = (-4 ± j4.472) / 2 =
\textbf{-2 ± j2.236}

\begin{enumerate}
\def\labelenumi{(\alph{enumi})}
\setcounter{enumi}{2}
\tightlist
\item
  Both eigenvalues have negative real parts (Re = -2), so the system is
  \textbf{stable}. The natural frequency is ω\textsubscript{n} = √9 = 3
  rad/s and the damping ratio is ζ = 4/(2 × 3) = 0.667.
\end{enumerate}

\begin{center}\rule{0.5\linewidth}{0.5pt}\end{center}

\section{Problem 4.9.2}\label{problem-4.9.2}

\textbf{Given:} A DC motor is modeled with state variables x₁ = armature
current (A) and x₂ = angular velocity (rad/s). The parameters are
armature resistance R\textsubscript{a} = 2 Ω, armature inductance
L\textsubscript{a} = 0.5 H, back-EMF constant K\textsubscript{b} = 0.1
V·s/rad, torque constant K\textsubscript{t} = 0.1 N·m/A, moment of
inertia J = 0.01 kg·m², and friction coefficient B\textsubscript{f} =
0.05 N·m·s/rad. The input u is the applied armature voltage, and the
output y is the angular velocity.

\textbf{Find:} (a) The state-space matrices A, B, C, D, (b) the
eigenvalues of A, and (c) the transfer function from voltage to angular
velocity.

\textbf{Solution:}

From circuit and mechanical equations: L\textsubscript{a}(dx₁/dt) = u -
R\textsubscript{a}x₁ - K\textsubscript{b}x₂ J(dx₂/dt) =
K\textsubscript{t}x₁ - B\textsubscript{f}x₂

\begin{enumerate}
\def\labelenumi{(\alph{enumi})}
\tightlist
\item
  ẋ₁ = -(R\textsubscript{a}/L\textsubscript{a})x₁ -
  (K\textsubscript{b}/L\textsubscript{a})x₂ + (1/L\textsubscript{a})u =
  -4x₁ - 0.2x₂ + 2u ẋ₂ = (K\textsubscript{t}/J)x₁ -
  (B\textsubscript{f}/J)x₂ = 10x₁ - 5x₂
\end{enumerate}

\textbf{A} = {[}-4, -0.2; 10, -5{]}, \textbf{B} = {[}2; 0{]}, \textbf{C}
= {[}0, 1{]}, \textbf{D} = {[}0{]}

\begin{enumerate}
\def\labelenumi{(\alph{enumi})}
\setcounter{enumi}{1}
\tightlist
\item
  det(sI - A) = (s + 4)(s + 5) + 2 = s² + 9s + 22 = 0
\end{enumerate}

s = (-9 ± √(81 - 88)) / 2 = (-9 ± √(-7)) / 2 = \textbf{-4.5 ± j1.323}

Both poles have negative real parts --- the system is \textbf{stable}.

\begin{enumerate}
\def\labelenumi{(\alph{enumi})}
\setcounter{enumi}{2}
\tightlist
\item
  H(s) = C(sI - A)⁻¹B + D
\end{enumerate}

(sI - A)⁻¹ = (1/(s² + 9s + 22)) × {[}s + 5, -0.2; 10, s + 4{]}

H(s) = {[}0, 1{]} × (1/(s² + 9s + 22)) × {[}s + 5, -0.2; 10, s + 4{]} ×
{[}2; 0{]}

= (1/(s² + 9s + 22)) × {[}0, 1{]} × {[}2(s + 5); 20{]} = \textbf{20 /
(s² + 9s + 22)}

\begin{center}\rule{0.5\linewidth}{0.5pt}\end{center}

\section{Problem 4.9.3}\label{problem-4.9.3}

\textbf{Given:} A system has state-space matrices: A = {[}0, 1; -6,
-5{]}, B = {[}0; 1{]}, C = {[}1, 0{]}, D = {[}0{]}

\textbf{Find:} (a) Whether the system is controllable, (b) whether the
system is observable, and (c) the transfer function.

\textbf{Solution:}

\begin{enumerate}
\def\labelenumi{(\alph{enumi})}
\tightlist
\item
  Controllability matrix: C\textsubscript{M} = {[}B, AB{]}
\end{enumerate}

AB = {[}0, 1; -6, -5{]} × {[}0; 1{]} = {[}1; -5{]}

C\textsubscript{M} = {[}{[}0, 1{]}; {[}1, -5{]}{]}

det(C\textsubscript{M}) = 0 × (-5) - 1 × 1 = -1 ≠ 0

Rank = 2 = n → \textbf{fully controllable}

\begin{enumerate}
\def\labelenumi{(\alph{enumi})}
\setcounter{enumi}{1}
\tightlist
\item
  Observability matrix: O\textsubscript{M} = {[}C; CA{]}
\end{enumerate}

CA = {[}1, 0{]} × {[}0, 1; -6, -5{]} = {[}0, 1{]}

O\textsubscript{M} = {[}{[}1, 0{]}; {[}0, 1{]}{]}

det(O\textsubscript{M}) = 1 × 1 - 0 × 0 = 1 ≠ 0

Rank = 2 = n → \textbf{fully observable}

\begin{enumerate}
\def\labelenumi{(\alph{enumi})}
\setcounter{enumi}{2}
\tightlist
\item
  H(s) = C(sI - A)⁻¹B + D
\end{enumerate}

Characteristic polynomial: s² + 5s + 6 = (s + 2)(s + 3)

H(s) = {[}1, 0{]} × (1/(s² + 5s + 6)) × {[}s + 5, 1; -6, s{]} × {[}0;
1{]}

= (1/(s² + 5s + 6)) × {[}1, 0{]} × {[}1; s{]} = \textbf{1 / (s² + 5s +
6) = 1 / ((s + 2)(s + 3))}

\begin{center}\rule{0.5\linewidth}{0.5pt}\end{center}

\section{Problem 4.9.4}\label{problem-4.9.4}

\textbf{Given:} A system has A = {[}0, 1; -2, -3{]} and B = {[}1; 0{]}.

\textbf{Find:} Whether the system is controllable, and explain the
physical implication.

\textbf{Solution:}

Controllability matrix: C\textsubscript{M} = {[}B, AB{]}

AB = {[}0, 1; -2, -3{]} × {[}1; 0{]} = {[}0; -2{]}

C\textsubscript{M} = {[}{[}1, 0{]}; {[}0, -2{]}{]}

det(C\textsubscript{M}) = 1 × (-2) - 0 × 0 = -2 ≠ 0

Rank = 2 = n → \textbf{fully controllable}

Since det(C\textsubscript{M}) ≠ 0, arbitrary pole placement via state
feedback is possible. The eigenvalues of A are found from s² + 3s + 2 =
(s + 1)(s + 2) = 0, giving open-loop poles at s = -1 and s = -2. A state
feedback gain K can move these poles to any desired closed-loop
locations.

Now consider the alternative B' = {[}1; 1{]}. Then AB' = {[}1; -5{]},
and C\textsubscript{M}' = {[}{[}1, 1{]}; {[}1, -5{]}{]}, det = -5 - 1 =
-6 ≠ 0 --- still \textbf{controllable}.

But if B'\,' = {[}1; -2{]}, then AB'\,' = {[}-2; 4{]} and
C\textsubscript{M}'\,' = {[}{[}1, -2{]}; {[}-2, 4{]}{]}, det = 4 - 4 = 0
--- \textbf{not controllable}. The input aligns with the eigenvector of
the s = -2 mode, so only one mode can be influenced.

\begin{center}\rule{0.5\linewidth}{0.5pt}\end{center}

\section{Problem 4.9.5}\label{problem-4.9.5}

\textbf{Given:} A third-order system has: A = {[}-1, 0, 0; 0, -3, 0; 0,
0, -5{]}, B = {[}1; 1; 0{]}, C = {[}1, 1, 1{]}

\textbf{Find:} (a) The system eigenvalues, (b) whether the system is
controllable, and (c) whether the system is observable.

\textbf{Solution:}

\begin{enumerate}
\def\labelenumi{(\alph{enumi})}
\item
  Since A is diagonal, the eigenvalues are the diagonal entries:
  \textbf{s₁ = -1, s₂ = -3, s₃ = -5}. All are negative, so the system is
  stable.
\item
  Controllability matrix (n = 3): C\textsubscript{M} = {[}B, AB, A²B{]}
\end{enumerate}

AB = {[}-1, 0, 0; 0, -3, 0; 0, 0, -5{]} × {[}1; 1; 0{]} = {[}-1; -3;
0{]}

A²B = A × AB = {[}-1, 0, 0; 0, -3, 0; 0, 0, -5{]} × {[}-1; -3; 0{]} =
{[}1; 9; 0{]}

C\textsubscript{M} = {[}{[}1, -1, 1{]}; {[}1, -3, 9{]}; {[}0, 0, 0{]}{]}

The third row is all zeros, so rank(C\textsubscript{M}) ≤ 2 \textless{}
3. The system is \textbf{not controllable}.

The s = -5 mode (x₃) has B₃ = 0, meaning the input cannot excite this
state --- it is an uncontrollable mode.

\begin{enumerate}
\def\labelenumi{(\alph{enumi})}
\setcounter{enumi}{2}
\tightlist
\item
  Observability matrix: O\textsubscript{M} = {[}C; CA; CA²{]}
\end{enumerate}

CA = {[}1, 1, 1{]} × A = {[}-1, -3, -5{]} CA² = {[}-1, -3, -5{]} × A =
{[}1, 9, 25{]}

O\textsubscript{M} = {[}{[}1, 1, 1{]}; {[}-1, -3, -5{]}; {[}1, 9,
25{]}{]}

det(O\textsubscript{M}) = 1(-75 + 45) - 1(-25 + 5) + 1(-9 + 3) = 1(-30)
- 1(-20) + 1(-6) = -30 + 20 - 6 = -16 ≠ 0

Rank = 3 = n → \textbf{fully observable}

All three modes are visible in the output, even though x₃ cannot be
controlled by the input.

\begin{center}\rule{0.5\linewidth}{0.5pt}\end{center}

\section{Problem 4.9.6}\label{problem-4.9.6}

\textbf{Given:} A system with A = {[}0, 1; -4, -5{]} and B = {[}0; 1{]}.
Design a state feedback controller u = -Kx + r to place the closed-loop
poles at s = -10 and s = -10 (repeated real poles for critically damped
response).

\textbf{Find:} The feedback gain vector K = {[}k₁, k₂{]} and the
closed-loop characteristic polynomial.

\textbf{Solution:}

Desired characteristic polynomial: (s + 10)(s + 10) = s² + 20s + 100

Open-loop A - BK = {[}0, 1; -4 - k₁, -5 - k₂{]}

Characteristic polynomial of A\textsubscript{CL}:

det(sI - A\textsubscript{CL}) = s(s + 5 + k₂) + (4 + k₁) = s² + (5 +
k₂)s + (4 + k₁)

Matching coefficients with s² + 20s + 100:

5 + k₂ = 20 → k₂ = \textbf{15}

4 + k₁ = 100 → k₁ = \textbf{96}

K = {[}96, 15{]}. The control law is u = -96x₁ - 15x₂ + r.

Verification: A\textsubscript{CL} = {[}0, 1; -100, -20{]}. Eigenvalues
from s² + 20s + 100 = (s + 10)² = 0 → s = -10, -10. The critically
damped response has a time constant of τ = 1/10 = \textbf{0.1 s} with no
overshoot.

\begin{center}\rule{0.5\linewidth}{0.5pt}\end{center}

\section{Problem 4.9.7}\label{problem-4.9.7}

\textbf{Given:} A system with A = {[}-2, 1; 0, -1{]} and B = {[}0; 1{]}.
A state feedback controller places the poles at s = -5 ± j5.

\textbf{Find:} (a) The required gain K = {[}k₁, k₂{]}, (b) the resulting
damping ratio and natural frequency, and (c) the expected percent
overshoot for a step input.

\textbf{Solution:}

\begin{enumerate}
\def\labelenumi{(\alph{enumi})}
\tightlist
\item
  Desired characteristic polynomial: (s + 5 - j5)(s + 5 + j5) = s² + 10s
  + 50
\end{enumerate}

A\textsubscript{CL} = A - BK = {[}-2, 1; -k₁, -1 - k₂{]}

det(sI - A\textsubscript{CL}) = (s + 2)(s + 1 + k₂) + k₁ = s² + (3 +
k₂)s + (2 + 2k₂ + k₁)

Matching coefficients: 3 + k₂ = 10 → k₂ = \textbf{7} 2 + 2(7) + k₁ = 50
→ 16 + k₁ = 50 → k₁ = \textbf{34}

K = {[}34, 7{]}

\begin{enumerate}
\def\labelenumi{(\alph{enumi})}
\setcounter{enumi}{1}
\item
  From s² + 10s + 50: ω\textsubscript{n} = √50 = \textbf{7.07 rad/s}, ζ
  = 10/(2 × 7.07) = \textbf{0.707}
\item
  Percent overshoot = 100 × e\textsuperscript{-πζ/√(1-ζ²)} = 100 ×
  e\textsuperscript{-π(0.707)/√(1-0.5)} = 100 ×
  e\textsuperscript{-2.221/0.707} = 100 × e\textsuperscript{-3.142}
\end{enumerate}

Correcting: -πζ/√(1 - ζ²) = -π(0.707)/√(0.5) = -2.221/0.707 = -3.142

\%OS = 100 × e⁻³·¹⁴² = 100 × 0.0432 = \textbf{4.3\%}

\begin{center}\rule{0.5\linewidth}{0.5pt}\end{center}

\section{Problem 4.9.8}\label{problem-4.9.8}

\textbf{Given:} A system with A = {[}0, 1; -3, -4{]}, B = {[}0; 1{]}, C
= {[}1, 0{]} (only position x₁ is measured). The controller poles have
been placed at s = -6 ± j6. Design a Luenberger observer with poles 4×
faster than the controller poles.

\textbf{Find:} The observer gain vector L = {[}l₁; l₂{]} and the
observer pole locations.

\textbf{Solution:}

Controller poles at s = -6 ± j6 have ω\textsubscript{n} = √(36 + 36) =
8.49 rad/s.

Observer poles placed 4× faster: s = -24 ± j24.

Desired observer characteristic polynomial: (s + 24 - j24)(s + 24 + j24)
= s² + 48s + 1152

Observer error dynamics: A - LC where L = {[}l₁; l₂{]}:

A - LC = {[}0 - l₁, 1; -3 - l₂, -4{]}

det(sI - (A - LC)) = (s + l₁)(s + 4) + (3 + l₂) = s² + (4 + l₁)s + (4l₁
+ 3 + l₂)

Matching coefficients: 4 + l₁ = 48 → l₁ = \textbf{44} 4(44) + 3 + l₂ =
1152 → 179 + l₂ = 1152 → l₂ = \textbf{973}

L = {[}44; 973{]}

The observer estimation error decays with a time constant of
approximately 1/24 = \textbf{0.042 s}, compared to the controller time
constant of 1/6 = 0.167 s. This ensures the state estimates converge
well before the controller response settles.

\begin{center}\rule{0.5\linewidth}{0.5pt}\end{center}

\section{Problem 4.9.9}\label{problem-4.9.9}

\textbf{Given:} A thermal system has two states: x₁ = room temperature
(°C above ambient) and x₂ = wall temperature (°C above ambient). The
system matrices are: A = {[}-0.1, 0.05; 0.02, -0.04{]}, B = {[}0.08;
0{]}, C = {[}1, 0{]}, D = {[}0{]} The input u is the heater power (kW),
and only the room temperature is measured.

\textbf{Find:} (a) The eigenvalues and their physical meaning, (b) the
controllability and observability of the system, and (c) an observer
with poles at s = -0.5 and s = -0.6.

\textbf{Solution:}

\begin{enumerate}
\def\labelenumi{(\alph{enumi})}
\tightlist
\item
  det(sI - A) = (s + 0.1)(s + 0.04) - (0.05)(0.02) = s² + 0.14s + 0.004
  - 0.001 = s² + 0.14s + 0.003 = 0
\end{enumerate}

s = (-0.14 ± √(0.0196 - 0.012)) / 2 = (-0.14 ± √0.0076) / 2 = (-0.14 ±
0.0872) / 2

s₁ = (-0.14 + 0.0872) / 2 = \textbf{-0.0264} (slow thermal mode, τ =
37.9 s) s₂ = (-0.14 - 0.0872) / 2 = \textbf{-0.1136} (fast thermal mode,
τ = 8.8 s)

The slow mode represents the overall building thermal mass, while the
fast mode represents the temperature difference between room and wall
equalizing.

\begin{enumerate}
\def\labelenumi{(\alph{enumi})}
\setcounter{enumi}{1}
\tightlist
\item
  C\textsubscript{M} = {[}B, AB{]} = {[}{[}0.08, -0.008 + 0{]}; {[}0,
  0.0016 + 0{]}{]}
\end{enumerate}

AB = {[}-0.1(0.08) + 0.05(0), 0.02(0.08) + (-0.04)(0){]} = {[}-0.008;
0.0016{]}

C\textsubscript{M} = {[}{[}0.08, -0.008{]}; {[}0, 0.0016{]}{]}

det(C\textsubscript{M}) = 0.08 × 0.0016 - (-0.008) × 0 = 0.000128 ≠ 0 →
\textbf{controllable}

O\textsubscript{M} = {[}C; CA{]} = {[}{[}1, 0{]}; {[}-0.1, 0.05{]}{]}

det(O\textsubscript{M}) = 1 × 0.05 - 0 × (-0.1) = 0.05 ≠ 0 →
\textbf{observable}

\begin{enumerate}
\def\labelenumi{(\alph{enumi})}
\setcounter{enumi}{2}
\tightlist
\item
  Desired observer polynomial: (s + 0.5)(s + 0.6) = s² + 1.1s + 0.3
\end{enumerate}

A - LC = {[}-0.1 - l₁, 0.05; 0.02 - l₂, -0.04{]}

det(sI - (A - LC)) = (s + 0.1 + l₁)(s + 0.04) - 0.05(0.02 - l₂)

= s² + (0.14 + l₁)s + (0.1 + l₁)(0.04) - 0.001 + 0.05l₂

= s² + (0.14 + l₁)s + 0.004 + 0.04l₁ - 0.001 + 0.05l₂

= s² + (0.14 + l₁)s + 0.003 + 0.04l₁ + 0.05l₂

Matching: 0.14 + l₁ = 1.1 → l₁ = \textbf{0.96} 0.003 + 0.04(0.96) +
0.05l₂ = 0.3 → 0.003 + 0.0384 + 0.05l₂ = 0.3 → 0.05l₂ = 0.2586 → l₂ =
\textbf{5.172}

L = {[}0.96; 5.172{]}

\begin{center}\rule{0.5\linewidth}{0.5pt}\end{center}

\section{Problem 4.9.10}\label{problem-4.9.10}

\textbf{Given:} A combined controller-observer system for the RLC
circuit from §4.9.1 has: A = {[}0, 10; -2, -6{]}, B = {[}0; 2{]}, C =
{[}1, 0{]} Controller gain K = {[}1.5, 2{]} (poles at s = -5 ± j5) and
observer gain L = {[}34; 57.6{]} (poles at s = -20 ± j20).

\textbf{Find:} (a) The four closed-loop poles of the combined
controller-observer system, (b) the system matrix of the combined
(4th-order) system, and (c) whether the separation principle holds.

\textbf{Solution:}

\begin{enumerate}
\def\labelenumi{(\alph{enumi})}
\tightlist
\item
  By the separation principle, the combined system poles are the union
  of:
\end{enumerate}

\begin{itemize}
\tightlist
\item
  Controller poles: eigenvalues of (A - BK) = \textbf{s = -5 ± j5}
\item
  Observer poles: eigenvalues of (A - LC) = \textbf{s = -20 ± j20}
\end{itemize}

The four poles are: \textbf{s = -5 + j5, s = -5 - j5, s = -20 + j20, s =
-20 - j20}

\begin{enumerate}
\def\labelenumi{(\alph{enumi})}
\setcounter{enumi}{1}
\tightlist
\item
  The combined system with state {[}x; e{]}\textsuperscript{T} (where e
  = x - x̂) has the block triangular form:
\end{enumerate}

A\textsubscript{combined} = {[}A - BK, BK; 0, A - LC{]}

A - BK = {[}0, 10; -2 - 2(1.5), -6 - 2(2){]} = {[}0, 10; -5, -10{]}

A - LC = {[}0 - 34, 10; -2 - 57.6, -6{]} = {[}-34, 10; -59.6, -6{]}

A\textsubscript{combined} = {[}{[}0, 10, 0, -20{]}; {[}-5, -10, 3, 4{]};
{[}0, 0, -34, 10{]}; {[}0, 0, -59.6, -6{]}{]}

Note: BK = {[}0; 2{]} × {[}1.5, 2{]} = {[}0, 0; 3, 4{]}

\begin{enumerate}
\def\labelenumi{(\alph{enumi})}
\setcounter{enumi}{2}
\tightlist
\item
  The separation principle \textbf{holds}: the characteristic polynomial
  of A\textsubscript{combined} is:
\end{enumerate}

det(sI - A\textsubscript{combined}) = det(sI - (A - BK)) × det(sI - (A -
LC))

= (s² + 10s + 50)(s² + 40s + 800)

The controller and observer can be designed independently --- the
controller poles and observer poles are the eigenvalues of their
respective subsystems, and neither design affects the other's pole
locations.

\chapter{Chapter 4 --- Section 4.10: Digital Control
Systems}\label{chapter-4-section-4.10-digital-control-systems}

Practice problems covering discrete-time system analysis, zero-order
hold discretization, Tustin (bilinear) transformation, z-plane
stability, digital PID implementation, sample period selection, and
anti-windup strategies.

\begin{center}\rule{0.5\linewidth}{0.5pt}\end{center}

\section{Problem 4.10.1}\label{problem-4.10.1}

\textbf{Given:} A continuous-time first-order plant G(s) = 20/(s + 10)
is preceded by a zero-order hold (ZOH) with sample period T = 0.02 s.

\textbf{Find:} (a) The discrete-time transfer function G(z) using the
ZOH method, (b) the location of the discrete pole and whether it is
stable, and (c) the DC gain of G(z) and verify it matches G(0).

\textbf{Solution:}

\begin{enumerate}
\def\labelenumi{(\alph{enumi})}
\tightlist
\item
  G(s)/s = 20/{[}s(s + 10){]}. Partial fractions: 20/{[}s(s + 10){]} =
  2/s - 2/(s + 10)
\end{enumerate}

Z-transform: Z\{2/s - 2/(s + 10)\} = 2z/(z - 1) - 2z/(z -
e\textsuperscript{-10T})

e\textsuperscript{-10(0.02)} = e⁻⁰·² = 0.8187

G(z) = (1 - z⁻¹){[}2z/(z - 1) - 2z/(z - 0.8187){]}

= 2 - 2(z - 1)/(z - 0.8187)

= {[}2(z - 0.8187) - 2(z - 1){]} / (z - 0.8187)

= 2(1 - 0.8187) / (z - 0.8187) = \textbf{0.3625 / (z - 0.8187)}

\begin{enumerate}
\def\labelenumi{(\alph{enumi})}
\setcounter{enumi}{1}
\tightlist
\item
  The discrete pole is at z = 0.8187. Since \textbar0.8187\textbar{}
  \textless{} 1, the pole is inside the unit circle and the system is
  \textbf{stable}.
\end{enumerate}

The continuous pole at s = -10 maps to z = e\textsuperscript{sT} = e⁻⁰·²
= 0.8187, confirming the relationship.

\begin{enumerate}
\def\labelenumi{(\alph{enumi})}
\setcounter{enumi}{2}
\tightlist
\item
  DC gain: G(1) = 0.3625 / (1 - 0.8187) = 0.3625 / 0.1813 = \textbf{2.0}
\end{enumerate}

Continuous DC gain: G(0) = 20/10 = \textbf{2.0} --- the DC gains match.

\begin{center}\rule{0.5\linewidth}{0.5pt}\end{center}

\section{Problem 4.10.2}\label{problem-4.10.2}

\textbf{Given:} A continuous-time integrator G(s) = 1/s is discretized
using the Tustin (bilinear) approximation with sample period T = 0.1 s.

\textbf{Find:} (a) The discrete transfer function G(z), (b) the step
response for the first 4 samples, and (c) comparison with the exact
discrete integrator (ZOH method).

\textbf{Solution:}

\begin{enumerate}
\def\labelenumi{(\alph{enumi})}
\tightlist
\item
  Tustin approximation: s = (2/T)(z - 1)/(z + 1) = 20(z - 1)/(z + 1)
\end{enumerate}

G(z) = 1/s = (z + 1) / {[}20(z - 1){]} = \textbf{(T/2)(z + 1)/(z - 1) =
0.05(z + 1)/(z - 1)}

\begin{enumerate}
\def\labelenumi{(\alph{enumi})}
\setcounter{enumi}{1}
\tightlist
\item
  For a unit step input U(z) = z/(z - 1):
\end{enumerate}

Y(z) = G(z) × U(z) = 0.05z(z + 1) / (z - 1)²

Expanding by long division or partial fractions: y{[}0{]} = 0.05,
y{[}1{]} = 0.05 + 0.1 = 0.15, y{[}2{]} = 0.15 + 0.1 = 0.25, y{[}3{]} =
0.25 + 0.1 = \textbf{0.35}

The output increments by T = 0.1 per step (after the first half-step),
as expected for an integrator with unit input.

\begin{enumerate}
\def\labelenumi{(\alph{enumi})}
\setcounter{enumi}{2}
\tightlist
\item
  The ZOH discrete integrator is G\textsubscript{ZOH}(z) = Tz⁻¹/(1 -
  z⁻¹) = T/(z - 1) = 0.1/(z - 1).
\end{enumerate}

ZOH step response: y{[}0{]} = 0, y{[}1{]} = 0.1, y{[}2{]} = 0.2,
y{[}3{]} = 0.3.

The Tustin method introduces a half-sample advance (trapezoidal
integration averages current and previous inputs), giving a more
accurate approximation of the continuous integral at each sample
instant. At k = 3 the exact continuous value is 0.3, Tustin gives 0.35
(slight lead), and ZOH gives 0.3 (exact for step input). The
\textbf{Tustin method better preserves frequency response} while ZOH
better preserves step response.

\begin{center}\rule{0.5\linewidth}{0.5pt}\end{center}

\section{Problem 4.10.3}\label{problem-4.10.3}

\textbf{Given:} A second-order continuous plant G(s) = 100/(s² + 6s +
100) has a closed-loop bandwidth of f\textsubscript{BW} = 1.6 Hz
(ω\textsubscript{BW} = 10 rad/s).

\textbf{Find:} (a) The maximum sample period T using the rule T ≤ 1/(10
× f\textsubscript{BW}), (b) the location of the continuous poles, (c)
the corresponding z-plane pole locations for the chosen T, and (d)
whether a sample period of T = 0.2 s would be adequate.

\textbf{Solution:}

\begin{enumerate}
\def\labelenumi{(\alph{enumi})}
\item
  Maximum sample period: T\textsubscript{max} = 1/(10 ×
  f\textsubscript{BW}) = 1/(10 × 1.6) = \textbf{0.0625 s} (minimum
  sample rate = 16 Hz)
\item
  Continuous poles from s² + 6s + 100 = 0: s = (-6 ± √(36 - 400)) / 2 =
  (-6 ± j19.08) / 2 = \textbf{-3 ± j9.54}
\end{enumerate}

ω\textsubscript{n} = √100 = 10 rad/s, ζ = 6/(2 × 10) = 0.3

\begin{enumerate}
\def\labelenumi{(\alph{enumi})}
\setcounter{enumi}{2}
\tightlist
\item
  For T = 0.0625 s, the z-plane poles are: z = e\textsuperscript{sT} =
  e\textsuperscript{(-3 ± j9.54)(0.0625)} = e⁻⁰·¹⁸⁷⁵ ×
  e\textsuperscript{±j0.5963}
\end{enumerate}

\textbar z\textbar{} = e⁻⁰·¹⁸⁷⁵ = 0.829

∠z = ±0.5963 rad = ±34.2°

z = \textbf{0.829∠±34.2°} = 0.685 ± j0.466

Both poles are inside the unit circle (\textbar z\textbar{} = 0.829
\textless{} 1) → \textbf{stable}.

\begin{enumerate}
\def\labelenumi{(\alph{enumi})}
\setcounter{enumi}{3}
\tightlist
\item
  For T = 0.2 s: f\textsubscript{s} = 1/0.2 = 5 Hz. The rule requires
  f\textsubscript{s} ≥ 16 Hz. Since 5 Hz \textless{} 16 Hz, T = 0.2 s is
  \textbf{not adequate} --- significant phase lag would be introduced,
  and the natural frequency of 10 rad/s (1.59 Hz) would be above the
  Nyquist frequency of 2.5 Hz, potentially causing aliasing and
  instability.
\end{enumerate}

\begin{center}\rule{0.5\linewidth}{0.5pt}\end{center}

\section{Problem 4.10.4}\label{problem-4.10.4}

\textbf{Given:} A digital PID controller has gains K\textsubscript{p} =
3.0, K\textsubscript{i} = 8.0, K\textsubscript{d} = 0.2, and sample
period T = 0.02 s. The error sequence is e{[}0{]} = 5.0, e{[}1{]} = 3.5,
e{[}2{]} = 2.0, e{[}3{]} = 0.8. Assume e{[}-1{]} = 0 and prior integral
sum = 0.

\textbf{Find:} The control output u{[}k{]} for k = 0 through 3 using the
position-form digital PID.

\textbf{Solution:}

Position form: u{[}k{]} = K\textsubscript{p}e{[}k{]} +
K\textsubscript{i}T × Σe{[}j{]} + K\textsubscript{d}(e{[}k{]} -
e{[}k-1{]})/T

\textbf{k = 0:} P = 3.0 × 5.0 = 15.0 I = 8.0 × 0.02 × 5.0 = 0.8 D = 0.2
× (5.0 - 0)/0.02 = 50.0 u{[}0{]} = 15.0 + 0.8 + 50.0 = \textbf{65.8}

\textbf{k = 1:} P = 3.0 × 3.5 = 10.5 Integral sum = 5.0 + 3.5 = 8.5; I =
8.0 × 0.02 × 8.5 = 1.36 D = 0.2 × (3.5 - 5.0)/0.02 = -15.0 u{[}1{]} =
10.5 + 1.36 + (-15.0) = \textbf{-3.14}

\textbf{k = 2:} P = 3.0 × 2.0 = 6.0 Integral sum = 8.5 + 2.0 = 10.5; I =
8.0 × 0.02 × 10.5 = 1.68 D = 0.2 × (2.0 - 3.5)/0.02 = -15.0 u{[}2{]} =
6.0 + 1.68 + (-15.0) = \textbf{-7.32}

\textbf{k = 3:} P = 3.0 × 0.8 = 2.4 Integral sum = 10.5 + 0.8 = 11.3; I
= 8.0 × 0.02 × 11.3 = 1.808 D = 0.2 × (0.8 - 2.0)/0.02 = -12.0 u{[}3{]}
= 2.4 + 1.808 + (-12.0) = \textbf{-7.792}

The large derivative term at k = 0 (50.0) is characteristic of the
initial step --- in practice, derivative-on-measurement (rather than
derivative-on-error) is used to avoid this ``derivative kick.''

\begin{center}\rule{0.5\linewidth}{0.5pt}\end{center}

\section{Problem 4.10.5}\label{problem-4.10.5}

\textbf{Given:} A motor speed controller uses a velocity-form digital
PID with K\textsubscript{p} = 1.5, K\textsubscript{i} = 4.0,
K\textsubscript{d} = 0.08, T = 0.01 s. The error sequence is e{[}0{]} =
100, e{[}1{]} = 80, e{[}2{]} = 55, e{[}3{]} = 30. Assume e{[}-1{]} =
e{[}-2{]} = 0 and u{[}-1{]} = 0.

\textbf{Find:} The incremental control output Δu{[}k{]} and the total
control output u{[}k{]} for k = 0 through 3.

\textbf{Solution:}

Velocity form: Δu{[}k{]} = K\textsubscript{p}(e{[}k{]} - e{[}k-1{]}) +
K\textsubscript{i}Te{[}k{]} + K\textsubscript{d}(e{[}k{]} - 2e{[}k-1{]}
+ e{[}k-2{]})/T

\textbf{k = 0:} Δu{[}0{]} = 1.5(100 - 0) + 4.0(0.01)(100) + 0.08(100 - 0
+ 0)/0.01 = 150 + 4.0 + 800 = \textbf{954.0} u{[}0{]} = 0 + 954.0 =
\textbf{954.0}

\textbf{k = 1:} Δu{[}1{]} = 1.5(80 - 100) + 4.0(0.01)(80) + 0.08(80 -
200 + 0)/0.01 = -30 + 3.2 + (-960) = \textbf{-986.8} u{[}1{]} = 954.0 +
(-986.8) = \textbf{-32.8}

\textbf{k = 2:} Δu{[}2{]} = 1.5(55 - 80) + 4.0(0.01)(55) + 0.08(55 - 160
+ 100)/0.01 = -37.5 + 2.2 + (-40) = \textbf{-75.3} u{[}2{]} = -32.8 +
(-75.3) = \textbf{-108.1}

\textbf{k = 3:} Δu{[}3{]} = 1.5(30 - 55) + 4.0(0.01)(30) + 0.08(30 - 110
+ 80)/0.01 = -37.5 + 1.2 + 0 = \textbf{-36.3} u{[}3{]} = -108.1 +
(-36.3) = \textbf{-144.4}

The negative control outputs indicate the controller is reducing the
drive signal as the motor approaches the setpoint. The velocity form
naturally prevents integral windup because it computes incremental
changes rather than accumulating the integral sum.

\begin{center}\rule{0.5\linewidth}{0.5pt}\end{center}

\section{Problem 4.10.6}\label{problem-4.10.6}

\textbf{Given:} A temperature controller uses a digital PID with
K\textsubscript{p} = 5.0, K\textsubscript{i} = 0.5, K\textsubscript{d} =
1.0, and T = 1.0 s. The actuator (heater) saturates at
u\textsubscript{max} = 100\% and u\textsubscript{min} = 0\%. The error
sequence is e{[}0{]} = 50, e{[}1{]} = 48, e{[}2{]} = 45, e{[}3{]} = 40.
Assume e{[}-1{]} = 0 and initial integral sum = 0. Anti-windup uses the
clamping method (freeze integrator when output saturates).

\textbf{Find:} The unclamped and clamped control outputs for k = 0
through 3.

\textbf{Solution:}

Position form: u{[}k{]} = K\textsubscript{p}e{[}k{]} +
K\textsubscript{i}T × Σe + K\textsubscript{d}(e{[}k{]} - e{[}k-1{]})/T

\textbf{k = 0:} P = 5.0 × 50 = 250; I = 0.5 × 1.0 × 50 = 25; D = 1.0 ×
(50 - 0)/1.0 = 50 u\textsubscript{unclamped} = 250 + 25 + 50 =
\textbf{325} u\textsubscript{clamped} = min(325, 100) = \textbf{100}
(saturated --- freeze integrator at Σe = 50)

\textbf{k = 1:} Integrator frozen, Σe remains at 50 (clamping active
since output saturated at k = 0) P = 5.0 × 48 = 240; I = 0.5 × 1.0 × 50
= 25; D = 1.0 × (48 - 50)/1.0 = -2 u\textsubscript{unclamped} = 240 + 25
+ (-2) = \textbf{263} u\textsubscript{clamped} = \textbf{100} (still
saturated --- integrator remains frozen at 50)

\textbf{k = 2:} Σe still frozen at 50 P = 5.0 × 45 = 225; I = 25; D =
1.0 × (45 - 48)/1.0 = -3 u\textsubscript{unclamped} = 225 + 25 + (-3) =
\textbf{247} u\textsubscript{clamped} = \textbf{100} (still saturated)

\textbf{k = 3:} Σe still frozen at 50 P = 5.0 × 40 = 200; I = 25; D =
1.0 × (40 - 45)/1.0 = -5 u\textsubscript{unclamped} = 200 + 25 + (-5) =
\textbf{220} u\textsubscript{clamped} = \textbf{100} (still saturated)

Without clamping, the integral sum would grow to 50 + 48 + 45 + 40 =
183, and the integral term alone would be 91.5 --- causing severe
overshoot when the error finally reverses sign. The clamping method
prevents this accumulation.

\begin{center}\rule{0.5\linewidth}{0.5pt}\end{center}

\section{Problem 4.10.7}\label{problem-4.10.7}

\textbf{Given:} A continuous-time PD controller C(s) = 2.0 + 0.15s is to
be discretized using the Tustin (bilinear) method with T = 0.005 s.

\textbf{Find:} (a) The discrete transfer function C(z), (b) the
difference equation relating u{[}k{]} to e{[}k{]}, and (c) the
controller gain at 0 Hz and at the Nyquist frequency.

\textbf{Solution:}

\begin{enumerate}
\def\labelenumi{(\alph{enumi})}
\tightlist
\item
  Substitute s = (2/T)(z - 1)/(z + 1) = 400(z - 1)/(z + 1):
\end{enumerate}

C(z) = 2.0 + 0.15 × 400(z - 1)/(z + 1) = 2.0 + 60(z - 1)/(z + 1)

= {[}2.0(z + 1) + 60(z - 1){]} / (z + 1) = (2z + 2 + 60z - 60) / (z + 1)

= \textbf{(62z - 58) / (z + 1)}

\begin{enumerate}
\def\labelenumi{(\alph{enumi})}
\setcounter{enumi}{1}
\tightlist
\item
  Cross-multiplying: u\href{z\%20+\%201}{k} = (62z - 58)e{[}k{]}
\end{enumerate}

u{[}k{]} + u{[}k-1{]} = 62e{[}k{]} - 58e{[}k-1{]}

\textbf{u{[}k{]} = -u{[}k-1{]} + 62e{[}k{]} - 58e{[}k-1{]}}

\begin{enumerate}
\def\labelenumi{(\alph{enumi})}
\setcounter{enumi}{2}
\tightlist
\item
  DC gain (z = 1): C(1) = (62 - 58)/(1 + 1) = 4/2 = \textbf{2.0}
\end{enumerate}

This matches C(0) = 2.0 + 0.15(0) = 2.0.

Nyquist frequency gain (z = -1): C(-1) = (-62 - 58)/(-1 + 1) → undefined
(pole at z = -1).

In practice, \textbar C(z)\textbar{} at z = e\textsuperscript{jπ} = -1
approaches infinity, meaning the derivative term has very high gain at
the Nyquist frequency. This is why practical digital PD controllers
include a derivative filter: C(s) = K\textsubscript{p} +
K\textsubscript{d}s/(1 + τ\textsubscript{f}s) with τ\textsubscript{f} =
K\textsubscript{d}/(N × K\textsubscript{p}) to limit high-frequency
gain.

\begin{center}\rule{0.5\linewidth}{0.5pt}\end{center}

\section{Problem 4.10.8}\label{problem-4.10.8}

\textbf{Given:} A continuous second-order system G(s) = 50/(s² + 4s +
50) is discretized using the ZOH method with T = 0.05 s. The continuous
poles are at s = -2 ± j6.782.

\textbf{Find:} (a) The z-plane pole locations, (b) the magnitude of the
z-plane poles and verify stability, and (c) the angle of the z-plane
poles in degrees.

\textbf{Solution:}

\begin{enumerate}
\def\labelenumi{(\alph{enumi})}
\tightlist
\item
  z = e\textsuperscript{sT} = e\textsuperscript{(-2 ± j6.782)(0.05)} =
  e⁻⁰·¹ × e\textsuperscript{±j0.3391}
\end{enumerate}

e⁻⁰·¹ = 0.9048

z = 0.9048 × (cos(0.3391) ± j sin(0.3391))

cos(0.3391) = 0.9430, sin(0.3391) = 0.3327

z = 0.9048 × (0.9430 ± j0.3327)

z = \textbf{0.8532 ± j0.3011}

\begin{enumerate}
\def\labelenumi{(\alph{enumi})}
\setcounter{enumi}{1}
\tightlist
\item
  \textbar z\textbar{} = 0.9048, which equals e⁻⁰·¹ =
  e\textsuperscript{-σT} where σ = 2 is the real part of the continuous
  pole.
\end{enumerate}

Since \textbar z\textbar{} = 0.9048 \textless{} 1, both poles are inside
the unit circle → \textbf{stable}.

Verification: \textbar z\textbar² = 0.8532² + 0.3011² = 0.7280 + 0.0907
= 0.8187. \textbar z\textbar{} = √0.8187 = 0.9048.

\begin{enumerate}
\def\labelenumi{(\alph{enumi})}
\setcounter{enumi}{2}
\tightlist
\item
  Angle: θ = arctan(0.3011/0.8532) = arctan(0.3529) = 19.44°
\end{enumerate}

This equals ω\textsubscript{d}T in degrees: 6.782 × 0.05 × (180/π) =
0.3391 × 57.30 = \textbf{19.43°} --- confirmed.

The z-plane poles preserve the damping information in the magnitude and
the oscillation frequency in the angle.

\begin{center}\rule{0.5\linewidth}{0.5pt}\end{center}

\section{Problem 4.10.9}\label{problem-4.10.9}

\textbf{Given:} A discrete-time system has the transfer function G(z) =
0.2z / (z² - 1.2z + 0.52). A unit step input is applied.

\textbf{Find:} (a) The z-plane pole locations and stability, (b) the
equivalent continuous-time poles (assuming T = 0.1 s), (c) the
steady-state output, and (d) the first 3 output samples y{[}0{]},
y{[}1{]}, y{[}2{]} (assume zero initial conditions).

\textbf{Solution:}

\begin{enumerate}
\def\labelenumi{(\alph{enumi})}
\tightlist
\item
  Poles from z² - 1.2z + 0.52 = 0: z = (1.2 ± √(1.44 - 2.08)) / 2 = (1.2
  ± √(-0.64)) / 2 = (1.2 ± j0.8) / 2
\end{enumerate}

z = \textbf{0.6 ± j0.4}

\textbar z\textbar{} = √(0.36 + 0.16) = √0.52 = 0.7211

Since \textbar z\textbar{} = 0.7211 \textless{} 1 → \textbf{stable}

\begin{enumerate}
\def\labelenumi{(\alph{enumi})}
\setcounter{enumi}{1}
\tightlist
\item
  z = e\textsuperscript{sT}, so s = ln(z)/T:
\end{enumerate}

\textbar z\textbar{} = e\textsuperscript{σT} → σ = ln(0.7211)/0.1 =
-0.3268/0.1 = \textbf{-3.27 rad/s} (real part)

∠z = ω\textsubscript{d}T → ω\textsubscript{d} = arctan(0.4/0.6)/0.1 =
0.5880/0.1 = \textbf{5.88 rad/s} (imaginary part)

Continuous poles: s = \textbf{-3.27 ± j5.88}

\begin{enumerate}
\def\labelenumi{(\alph{enumi})}
\setcounter{enumi}{2}
\item
  Steady-state (Final Value Theorem): y\textsubscript{ss} =
  lim\textsubscript{z→1} (z - 1) × G(z) × z/(z - 1) = G(1) = 0.2(1)/(1 -
  1.2 + 0.52) = 0.2/0.32 = \textbf{0.625}
\item
  G(z) = Y(z)/U(z) → Y(z)(z² - 1.2z + 0.52) = 0.2z × U(z)
\end{enumerate}

Difference equation: y{[}k{]} = 1.2y{[}k-1{]} - 0.52y{[}k-2{]} +
0.2u{[}k-1{]}

For step input u{[}k{]} = 1: y{[}0{]} = 1.2(0) - 0.52(0) + 0.2(0) =
\textbf{0} y{[}1{]} = 1.2(0) - 0.52(0) + 0.2(1) = \textbf{0.2} y{[}2{]}
= 1.2(0.2) - 0.52(0) + 0.2(1) = 0.24 + 0.2 = \textbf{0.44}

\begin{center}\rule{0.5\linewidth}{0.5pt}\end{center}

\section{Problem 4.10.10}\label{problem-4.10.10}

\textbf{Given:} A continuous PID controller C(s) = 4.0(1 + 1/(2s) +
0.3s) is to be implemented digitally. The plant has a closed-loop
bandwidth of f\textsubscript{BW} = 25 Hz.

\textbf{Find:} (a) The K\textsubscript{p}, K\textsubscript{i}, and
K\textsubscript{d} gains in standard form, (b) the maximum sample period
T, (c) the discrete PID in position form for the selected T, and (d) the
first two control outputs for a unit step error (e{[}k{]} = 1 for k ≥ 0,
e{[}-1{]} = 0).

\textbf{Solution:}

\begin{enumerate}
\def\labelenumi{(\alph{enumi})}
\tightlist
\item
  Expanding: C(s) = 4.0 + 4.0/(2s) + 4.0(0.3)s = 4.0 + 2.0/s + 1.2s
\end{enumerate}

K\textsubscript{p} = \textbf{4.0}, K\textsubscript{i} = \textbf{2.0},
K\textsubscript{d} = \textbf{1.2}

\begin{enumerate}
\def\labelenumi{(\alph{enumi})}
\setcounter{enumi}{1}
\tightlist
\item
  T\textsubscript{max} = 1/(10 × f\textsubscript{BW}) = 1/(10 × 25) =
  \textbf{0.004 s} (4 ms, sample rate = 250 Hz)
\end{enumerate}

Select T = 0.004 s.

\begin{enumerate}
\def\labelenumi{(\alph{enumi})}
\setcounter{enumi}{2}
\tightlist
\item
  Position-form digital PID: u{[}k{]} = K\textsubscript{p}e{[}k{]} +
  K\textsubscript{i}T Σe{[}j{]} + K\textsubscript{d}(e{[}k{]} -
  e{[}k-1{]})/T
\end{enumerate}

u{[}k{]} = 4.0 × e{[}k{]} + 2.0 × 0.004 × Σe{[}j{]} + 1.2 × (e{[}k{]} -
e{[}k-1{]})/0.004

u{[}k{]} = \textbf{4.0 × e{[}k{]} + 0.008 × Σe{[}j{]} + 300 × (e{[}k{]}
- e{[}k-1{]})}

\begin{enumerate}
\def\labelenumi{(\alph{enumi})}
\setcounter{enumi}{3}
\tightlist
\item
  \textbf{k = 0:} e{[}0{]} = 1, Σe = 1 P = 4.0(1) = 4.0; I = 0.008(1) =
  0.008; D = 300(1 - 0) = 300 u{[}0{]} = 4.0 + 0.008 + 300 =
  \textbf{304.008}
\end{enumerate}

\textbf{k = 1:} e{[}1{]} = 1, Σe = 2 P = 4.0(1) = 4.0; I = 0.008(2) =
0.016; D = 300(1 - 1) = 0 u{[}1{]} = 4.0 + 0.016 + 0 = \textbf{4.016}

The enormous derivative kick at k = 0 (300) is a practical concern. In
implementation, derivative-on-measurement should be used: replace
e{[}k{]} - e{[}k-1{]} with -(y{[}k{]} - y{[}k-1{]}) in the derivative
term, eliminating the spike caused by a step change in the reference
signal.

\chapter{Chapter 5 --- Section 5.1:
Microcontrollers}\label{chapter-5-section-5.1-microcontrollers}

Practice problems covering microcontroller architecture, common MCU
families, and clock/PLL configuration.

\begin{center}\rule{0.5\linewidth}{0.5pt}\end{center}

\section{Problem 5.1.1}\label{problem-5.1.1}

\textbf{Given:} A Von Neumann architecture MCU runs at 64 MHz. Due to
bus contention between instruction fetches and data accesses, 40\% of
instructions require an additional bus cycle. A subroutine consists of
200 instructions, of which 80 are load/store instructions that trigger
the extra cycle and the remaining 120 are single-cycle instructions.

\textbf{Find:} (a) The total execution time for the subroutine. (b) How
much faster would the same subroutine run on a Harvard architecture MCU
at the same clock frequency, assuming all instructions execute in one
cycle?

\textbf{Solution:}

\begin{enumerate}
\def\labelenumi{(\alph{enumi})}
\tightlist
\item
  Clock period = 1 / 64 MHz = 15.625 ns
\end{enumerate}

Single-cycle instructions: 120 instructions x 1 cycle = 120 cycles
Two-cycle instructions (bus contention): 80 instructions x 2 cycles =
160 cycles Total cycles = 120 + 160 = 280 cycles

Execution time = 280 x 15.625 ns = \textbf{4.375 us}

\begin{enumerate}
\def\labelenumi{(\alph{enumi})}
\setcounter{enumi}{1}
\tightlist
\item
  On Harvard architecture, all 200 instructions execute in 1 cycle each:
  Total cycles = 200 Execution time = 200 x 15.625 ns = 3.125 us
\end{enumerate}

Speedup = 4.375 / 3.125 = \textbf{1.40x (40\% faster)}

The Harvard architecture eliminates the bus contention penalty by
providing separate instruction and data buses.

\begin{center}\rule{0.5\linewidth}{0.5pt}\end{center}

\section{Problem 5.1.2}\label{problem-5.1.2}

\textbf{Given:} An embedded system must read 4 analog channels at 10 kHz
each, run a PID control loop every 1 ms (requiring approximately 500
multiply-accumulate operations per iteration), communicate over UART at
115200 baud, and operate from a 3.3 V coin cell with a target battery
life of 2 years. The system must cost under \$2 per unit in volume.

\textbf{Find:} Select the most appropriate MCU family (8-bit AVR, ARM
Cortex-M0+, or ARM Cortex-M4) and justify the choice.

\textbf{Solution:} ADC requirement: 4 channels x 10,000 samples/s =
40,000 samples/s total --- achievable by all three families.

PID computation: 500 MAC operations per ms = 500,000 MAC/s. - 8-bit AVR
(16 MHz): software multiply takes \textasciitilde2 cycles, total = 500 x
2 = 1000 cycles/ms. At 16 MHz, that is 1000/16,000 = 6.25\% CPU ---
feasible but leaves little headroom. - Cortex-M0+ (48 MHz): single-cycle
32-bit multiply, 500 cycles/ms = 500/48,000 = 1.04\% CPU ---
comfortable. - Cortex-M4 (up to 168 MHz): single-cycle MAC with FPU ---
vastly overqualified.

Power: Cortex-M0+ devices (e.g., STM32L0 series) are specifically
designed for ultra-low-power applications with sleep currents as low as
0.3 uA and active current of \textasciitilde100 uA/MHz.

Cost: 8-bit AVR and Cortex-M0+ both fall well under \$2; Cortex-M4
devices typically cost \$3-5.

\textbf{The ARM Cortex-M0+ is the best choice.} It provides sufficient
computation for the PID loop, meets the ADC and UART requirements,
offers best-in-class low-power modes for battery life, and stays within
the cost target.

\begin{center}\rule{0.5\linewidth}{0.5pt}\end{center}

\section{Problem 5.1.3}\label{problem-5.1.3}

\textbf{Given:} A microcontroller uses a 16 MHz external crystal (HSE)
with a PLL configured as follows: input divider M = 4, multiplication
factor N = 192, output divider P = 4. The AHB prescaler is 1, the APB1
prescaler is 2, and the APB2 prescaler is 1.

\textbf{Find:} (a) The system clock (PLL output), (b) the AHB bus
frequency, (c) the APB1 peripheral clock, (d) the APB1 timer clock, and
(e) the APB2 peripheral clock.

\textbf{Solution:}

\begin{enumerate}
\def\labelenumi{(\alph{enumi})}
\tightlist
\item
  f\textsubscript{PLL} = (HSE / M) x N / P = (16 MHz / 4) x 192 / 4 = 4
  MHz x 48 = \textbf{192 MHz}
\end{enumerate}

Wait --- let's recalculate: (16 / 4) = 4 MHz VCO input; 4 x 192 = 768
MHz VCO output; 768 / 4 = \textbf{192 MHz system clock}

\begin{enumerate}
\def\labelenumi{(\alph{enumi})}
\setcounter{enumi}{1}
\item
  f\textsubscript{AHB} = f\textsubscript{PLL} / AHB prescaler = 192 / 1
  = \textbf{192 MHz}
\item
  f\textsubscript{APB1} = f\textsubscript{AHB} / APB1 prescaler = 192 /
  2 = \textbf{96 MHz}
\item
  When the APB prescaler is not 1, the timer clock is doubled:
  f\textsubscript{APB1\_timer} = 2 x f\textsubscript{APB1} = 2 x 96 =
  \textbf{192 MHz}
\item
  f\textsubscript{APB2} = f\textsubscript{AHB} / APB2 prescaler = 192 /
  1 = \textbf{192 MHz}
\end{enumerate}

Since APB2 prescaler is 1, APB2 timers also run at 192 MHz (no doubling
needed when prescaler = 1).

\begin{center}\rule{0.5\linewidth}{0.5pt}\end{center}

\section{Problem 5.1.4}\label{problem-5.1.4}

\textbf{Given:} An engineer needs a timer to generate precise 1-second
interrupts for a real-time clock function. The available clock sources
are: HSI at 16 MHz (+/- 1\% accuracy) and LSE at 32.768 kHz (+/- 20 ppm
accuracy).

\textbf{Find:} (a) The maximum timing error per day using each clock
source. (b) Which source is appropriate for the RTC function?

\textbf{Solution:}

\begin{enumerate}
\def\labelenumi{(\alph{enumi})}
\tightlist
\item
  HSI at +/- 1\%: Error per second = 1\% of 1 s = 10 ms Error per day =
  0.01 x 86,400 s = \textbf{864 seconds/day = 14.4 minutes/day}
\end{enumerate}

LSE at +/- 20 ppm: Error per second = 20 x 10\textsuperscript{-6} x 1 s
= 20 us Error per day = 20 x 10\textsuperscript{-6} x 86,400 =
\textbf{1.728 seconds/day}

\begin{enumerate}
\def\labelenumi{(\alph{enumi})}
\setcounter{enumi}{1}
\tightlist
\item
  \textbf{The LSE (32.768 kHz crystal) is the correct choice for RTC.}
  Its 20 ppm accuracy yields less than 2 seconds of drift per day, while
  the HSI would drift nearly 15 minutes per day. The 32.768 kHz
  frequency also divides evenly by powers of 2 (32,768 =
  2\textsuperscript{15}), producing an exact 1 Hz output with a 15-bit
  prescaler.
\end{enumerate}

\begin{center}\rule{0.5\linewidth}{0.5pt}\end{center}

\section{Problem 5.1.5}\label{problem-5.1.5}

\textbf{Given:} A Cortex-M7 MCU at 216 MHz has a 5-stage pipeline. A
branch misprediction flushes the pipeline. In a particular function,
15\% of instructions are branches, and the branch predictor has a 90\%
accuracy rate.

\textbf{Find:} (a) The number of wasted cycles per 1000 instructions due
to mispredictions. (b) The effective CPI (cycles per instruction) if the
ideal CPI is 1.0.

\textbf{Solution:}

\begin{enumerate}
\def\labelenumi{(\alph{enumi})}
\item
  Branches per 1000 instructions = 0.15 x 1000 = 150 branches
  Mispredictions = 150 x (1 - 0.90) = 150 x 0.10 = 15 mispredictions
  Each misprediction wastes the pipeline depth - 1 = 5 - 1 = 4 cycles
  (flushing 4 in-flight instructions) Wasted cycles = 15 x 4 =
  \textbf{60 cycles per 1000 instructions}
\item
  Total cycles for 1000 instructions = 1000 (ideal) + 60 (penalty) =
  1060 cycles Effective CPI = 1060 / 1000 = \textbf{1.06 cycles per
  instruction}
\end{enumerate}

This represents a 6\% performance degradation from the ideal case.
Improving branch prediction accuracy to 95\% would reduce wasted cycles
to 30, giving CPI = 1.03.

\begin{center}\rule{0.5\linewidth}{0.5pt}\end{center}

\section{Problem 5.1.6}\label{problem-5.1.6}

\textbf{Given:} An MCU system clock is configured at 100 MHz. A UART
peripheral requires a 48 MHz clock derived from the PLL. The available
PLL output dividers for the UART clock are integers from 1 to 16.

\textbf{Find:} Can the PLL provide exactly 48 MHz for the UART? If not,
determine the closest achievable frequency and the resulting baud rate
error for a target of 115200 baud.

\textbf{Solution:} The PLL output is 100 MHz. Available UART clocks =
100 / N for N = 1 to 16: 100, 50, 33.33, 25, 20, 16.67, 14.29, 12.5,
11.11, 10, 9.09, 8.33, 7.69, 7.14, 6.67, 6.25 MHz

The closest to 48 MHz is \textbf{50 MHz} (N = 2). Exact 48 MHz is not
achievable with integer division from 100 MHz.

Baud rate generation at 115200 from 50 MHz: Divider = 50,000,000 /
115,200 = 434.028 Using integer divider = 434: actual baud = 50,000,000
/ 434 = 115,207.4 baud Error = (115,207.4 - 115,200) / 115,200 x 100 =
\textbf{0.0064\%}

This error is well within the +/- 3\% tolerance for UART communication,
so \textbf{50 MHz is acceptable} as the UART peripheral clock.

\chapter{Chapter 5 --- Section 5.2:
Memory}\label{chapter-5-section-5.2-memory}

Practice problems covering flash memory, SRAM allocation, and EEPROM
endurance calculations.

\begin{center}\rule{0.5\linewidth}{0.5pt}\end{center}

\section{Problem 5.2.1}\label{problem-5.2.1}

\textbf{Given:} A firmware image is 192 KB. The MCU flash is organized
as 8 sectors: sector 0 is 16 KB, sector 1 is 16 KB, sector 2 is 16 KB,
sector 3 is 16 KB, sector 4 is 64 KB, and sectors 5-7 are each 128 KB.
The flash endurance is 10,000 write-erase cycles. A bootloader occupies
sector 0. Firmware updates are performed via OTA 3 times per day.

\textbf{Find:} (a) Which sectors are needed for the application image?
(b) Time to reach the flash endurance limit. (c) If a dual-bank A/B
update scheme is used, how does this change the answer?

\textbf{Solution:}

\begin{enumerate}
\def\labelenumi{(\alph{enumi})}
\tightlist
\item
  Available sectors (excluding sector 0 for bootloader): Sector 1: 16
  KB, Sector 2: 16 KB, Sector 3: 16 KB, Sector 4: 64 KB, Sector 5: 128
  KB, Sector 6: 128 KB, Sector 7: 128 KB
\end{enumerate}

192 KB firmware requires: Sector 1 (16) + Sector 2 (16) + Sector 3 (16)
+ Sector 4 (64) + Sector 5 (128) = 240 KB total capacity, using
\textbf{sectors 1 through 5} (5 sectors).

Alternatively, Sector 5 (128 KB) + Sector 4 (64 KB) = 192 KB exactly,
using \textbf{sectors 4 and 5} for a more efficient layout.

\begin{enumerate}
\def\labelenumi{(\alph{enumi})}
\setcounter{enumi}{1}
\item
  At 3 updates/day: 3 x 365 = 1,095 erase cycles per year Time to
  endurance limit = 10,000 / 1,095 = \textbf{9.13 years}
\item
  With dual-bank (A/B): each slot is updated on alternating cycles, so
  each bank sees half the updates: Effective cycles per bank per year =
  1,095 / 2 = 547.5 Time to endurance limit = 10,000 / 547.5 =
  \textbf{18.26 years}
\end{enumerate}

The dual-bank scheme doubles the effective flash lifetime and also
provides rollback protection.

\begin{center}\rule{0.5\linewidth}{0.5pt}\end{center}

\section{Problem 5.2.2}\label{problem-5.2.2}

\textbf{Given:} An RTOS-based embedded application on an MCU with 64 KB
SRAM has the following memory requirements: - RTOS kernel: 2 KB - Task 1
stack: 512 bytes - Task 2 stack: 1024 bytes - Task 3 stack: 2048 bytes -
Idle task stack: 256 bytes - Global variables: 4.5 KB - DMA receive
buffer (UART): 256 bytes (double-buffered) - ADC sample buffer: 2048
samples at 16 bits each (circular, double-buffered) - Display frame
buffer: 128 x 64 pixels, 1 bit per pixel

\textbf{Find:} (a) Total SRAM usage. (b) Remaining SRAM. (c) Maximum
additional task stack size that could be allocated while keeping 25\%
SRAM free as safety margin.

\textbf{Solution:}

\begin{enumerate}
\def\labelenumi{(\alph{enumi})}
\tightlist
\item
  Itemized memory usage:
\end{enumerate}

\begin{itemize}
\tightlist
\item
  RTOS kernel: 2,048 bytes
\item
  Task 1 stack: 512 bytes
\item
  Task 2 stack: 1,024 bytes
\item
  Task 3 stack: 2,048 bytes
\item
  Idle task stack: 256 bytes
\item
  Global variables: 4,608 bytes (4.5 x 1024)
\item
  UART DMA buffer: 2 x 256 = 512 bytes (double-buffered)
\item
  ADC buffer: 2 x 2048 x 2 = 8,192 bytes (double-buffered, 16-bit
  samples)
\item
  Display frame buffer: (128 x 64) / 8 = 1,024 bytes
\end{itemize}

Total = 2,048 + 512 + 1,024 + 2,048 + 256 + 4,608 + 512 + 8,192 + 1,024
= \textbf{20,224 bytes = 19.75 KB}

\begin{enumerate}
\def\labelenumi{(\alph{enumi})}
\setcounter{enumi}{1}
\item
  Remaining = 64 KB - 19.75 KB = \textbf{44.25 KB}
\item
  25\% safety margin = 0.25 x 64 = 16 KB reserved Available for tasks =
  64 - 19.75 - 16 = \textbf{28.25 KB maximum additional stack}
\end{enumerate}

\begin{center}\rule{0.5\linewidth}{0.5pt}\end{center}

\section{Problem 5.2.3}\label{problem-5.2.3}

\textbf{Given:} A data logger stores sensor readings in EEPROM. Each
reading is a 6-byte record (4-byte timestamp + 2-byte sensor value). The
EEPROM is 4 KB (4096 bytes) with 1,000,000 write cycles per byte
endurance. Readings are taken every 30 seconds. The logger uses
sequential writes with wrap-around (circular buffer).

\textbf{Find:} (a) Maximum number of records that fit in the EEPROM. (b)
Time before the buffer wraps around. (c) Time to reach the endurance
limit on the first byte.

\textbf{Solution:}

\begin{enumerate}
\def\labelenumi{(\alph{enumi})}
\tightlist
\item
  Records = 4096 / 6 = 682 complete records (682 x 6 = 4092 bytes used,
  4 bytes unused)
\end{enumerate}

Maximum records = \textbf{682 records}

\begin{enumerate}
\def\labelenumi{(\alph{enumi})}
\setcounter{enumi}{1}
\tightlist
\item
  Readings per hour = 3600 / 30 = 120 readings/hour Time to fill = 682 /
  120 = 5.683 hours = \textbf{5 hours 41 minutes}
\end{enumerate}

After this, the buffer wraps and begins overwriting the oldest records.

\begin{enumerate}
\def\labelenumi{(\alph{enumi})}
\setcounter{enumi}{2}
\tightlist
\item
  Each wrap-around writes 1 new record (6 bytes) to the first location.
  Wraps occur every 5.683 hours. Writes per year to byte 0 = (365 x 24)
  / 5.683 = 8,760 / 5.683 = 1,541 writes/year Time to endurance limit =
  1,000,000 / 1,541 = \textbf{649 years}
\end{enumerate}

The EEPROM endurance is not a concern for this application.

\begin{center}\rule{0.5\linewidth}{0.5pt}\end{center}

\section{Problem 5.2.4}\label{problem-5.2.4}

\textbf{Given:} An MCU has 256 KB of flash. The linker output reports
the following section sizes: - .text (code): 87,432 bytes - .rodata
(constants): 12,288 bytes - .data (initialized variables, stored in
flash): 3,072 bytes - .bss (zero-initialized variables): 8,192 bytes
(not stored in flash)

\textbf{Find:} (a) Total flash usage. (b) Flash utilization percentage.
(c) If a 16 KB flash sector is reserved for non-volatile parameter
storage and 32 KB for a bootloader, what is the remaining space for the
application?

\textbf{Solution:}

\begin{enumerate}
\def\labelenumi{(\alph{enumi})}
\tightlist
\item
  Flash-resident sections: .text + .rodata + .data (initial values)
  Flash usage = 87,432 + 12,288 + 3,072 = \textbf{102,792 bytes = 100.38
  KB}
\end{enumerate}

Note: .bss is not stored in flash; it is zeroed in SRAM at startup.

\begin{enumerate}
\def\labelenumi{(\alph{enumi})}
\setcounter{enumi}{1}
\item
  Utilization = 102,792 / (256 x 1024) x 100 = 102,792 / 262,144 x 100 =
  \textbf{39.2\%}
\item
  Reserved space = 32 KB (bootloader) + 16 KB (parameters) = 48 KB
  Available for application = 256 - 48 = 208 KB Remaining after current
  application = 208 - 100.38 = \textbf{107.62 KB free}
\end{enumerate}

The application uses 100.38 / 208 = 48.3\% of its allocated space.

\begin{center}\rule{0.5\linewidth}{0.5pt}\end{center}

\section{Problem 5.2.5}\label{problem-5.2.5}

\textbf{Given:} A wear-leveling algorithm distributes writes evenly
across 64 EEPROM pages of 32 bytes each (2 KB total). The application
writes a 4-byte counter value once per minute. Without wear leveling,
only one page is written. With wear leveling, writes rotate across all
64 pages. The EEPROM endurance is 100,000 cycles per page.

\textbf{Find:} (a) Time to endurance limit without wear leveling. (b)
Time to endurance limit with wear leveling. (c) The improvement factor.

\textbf{Solution:}

\begin{enumerate}
\def\labelenumi{(\alph{enumi})}
\item
  Without wear leveling: Writes per year = 60 x 24 x 365 = 525,600
  writes/year Time = 100,000 / 525,600 = \textbf{0.190 years = 69.4
  days}
\item
  With wear leveling across 64 pages: Each page sees 525,600 / 64 =
  8,212.5 writes/year Time = 100,000 / 8,212.5 = \textbf{12.18 years}
\item
  Improvement factor = 12.18 / 0.190 = \textbf{64x improvement}
\end{enumerate}

The wear-leveling factor equals the number of pages, as expected. This
demonstrates why wear leveling is critical for high-frequency write
applications.

\chapter{Chapter 5 --- Section 5.3:
Peripherals}\label{chapter-5-section-5.3-peripherals}

Practice problems covering GPIO, timers/PWM, ADC, DAC, DMA, and watchdog
timers.

\begin{center}\rule{0.5\linewidth}{0.5pt}\end{center}

\section{Problem 5.3.1}\label{problem-5.3.1}

\textbf{Given:} A 3.3 V MCU GPIO pin (maximum sink current 12 mA) is
used to drive a common-anode RGB LED. Each LED color element has a
forward voltage of 2.1 V (red), 3.0 V (green), and 3.0 V (blue), and the
desired forward current for each is 8 mA. The GPIO pins drive low (sink
current) to turn on each LED element, with the common anode connected to
3.3 V through individual current-limiting resistors.

\textbf{Find:} (a) The resistor value for each color. (b) Whether the
GPIO pins can directly drive the LEDs. (c) The total power dissipated in
the resistors when all three colors are on (white).

\textbf{Solution:}

\begin{enumerate}
\def\labelenumi{(\alph{enumi})}
\tightlist
\item
  R = (V\textsubscript{CC} - V\textsubscript{f} - V\textsubscript{OL}) /
  I\textsubscript{f}, where V\textsubscript{OL} \textasciitilde{} 0.1 V
  for CMOS output sinking current.
\end{enumerate}

Red: R = (3.3 - 2.1 - 0.1) / 0.008 = 1.1 / 0.008 = 137.5 ohm
-\textgreater{} use \textbf{150 ohm} (standard value) Actual
I\textsubscript{red} = 1.1 / 150 = 7.33 mA

Green: R = (3.3 - 3.0 - 0.1) / 0.008 = 0.2 / 0.008 = 25 ohm
-\textgreater{} use \textbf{27 ohm} (standard value) Actual
I\textsubscript{green} = 0.2 / 27 = 7.41 mA

Blue: R = (3.3 - 3.0 - 0.1) / 0.008 = 25 ohm -\textgreater{} use
\textbf{27 ohm} (standard value) Actual I\textsubscript{blue} = 7.41 mA

\begin{enumerate}
\def\labelenumi{(\alph{enumi})}
\setcounter{enumi}{1}
\item
  Each GPIO pin sinks at most 7.41 mA, well below the 12 mA maximum.
  \textbf{Yes, direct drive is feasible.}
\item
  Total resistor power: P\textsubscript{red} = (7.33 x
  10\textsuperscript{-3})\^{}2 x 150 = 8.06 mW P\textsubscript{green} =
  (7.41 x 10\textsuperscript{-3})\^{}2 x 27 = 1.48 mW
  P\textsubscript{blue} = (7.41 x 10\textsuperscript{-3})\^{}2 x 27 =
  1.48 mW Total = \textbf{11.02 mW}
\end{enumerate}

\begin{center}\rule{0.5\linewidth}{0.5pt}\end{center}

\section{Problem 5.3.2}\label{problem-5.3.2}

\textbf{Given:} An MCU running at 84 MHz uses a 16-bit timer to generate
a 50 Hz PWM signal for a hobby servo motor. Servos require a 20 ms
period with a pulse width ranging from 1.0 ms (0 degrees) to 2.0 ms (180
degrees).

\textbf{Find:} (a) The minimum prescaler value to fit the 20 ms period
within the 16-bit counter (max 65535). (b) The auto-reload register
(ARR) value with this prescaler. (c) The compare register (CCR) values
for 0, 90, and 180 degrees. (d) The angular resolution in degrees.

\textbf{Solution:}

\begin{enumerate}
\def\labelenumi{(\alph{enumi})}
\tightlist
\item
  Timer ticks for 20 ms at 84 MHz (no prescaler) = 84,000,000 x 0.020 =
  1,680,000 ticks. This exceeds 65,535, so a prescaler is needed.
  Minimum prescaler = ceil(1,680,000 / 65,536) = ceil(25.63) = 26.
\end{enumerate}

But prescaler values are typically N-1, so prescaler register = 26 - 1 =
25 (divides by 26). Timer frequency = 84 MHz / 26 = 3,230,769.2 Hz. Not
a clean number.

Better: use prescaler = 84. Timer freq = 84 MHz / 84 = 1 MHz (1 us per
tick). Ticks for 20 ms = 20,000. This fits in 16 bits. \textbf{Prescaler
= 84 (register value = 83)}

\begin{enumerate}
\def\labelenumi{(\alph{enumi})}
\setcounter{enumi}{1}
\item
  ARR = 20,000 - 1 = \textbf{19,999}
\item
  Pulse widths at 1 us per tick: 0 degrees: 1.0 ms = 1000 ticks
  -\textgreater{} CCR = \textbf{1000} 90 degrees: 1.5 ms = 1500 ticks
  -\textgreater{} CCR = \textbf{1500} 180 degrees: 2.0 ms = 2000 ticks
  -\textgreater{} CCR = \textbf{2000}
\item
  Servo range = 2000 - 1000 = 1000 ticks over 180 degrees Angular
  resolution = 180 / 1000 = \textbf{0.18 degrees per tick}
\end{enumerate}

\begin{center}\rule{0.5\linewidth}{0.5pt}\end{center}

\section{Problem 5.3.3}\label{problem-5.3.3}

\textbf{Given:} A 10-bit ADC with V\textsubscript{ref} = 5.0 V is used
to measure a 4-20 mA current loop signal through a 250 ohm precision
sense resistor. The current range of 4-20 mA corresponds to a process
variable range of 0-100 PSI.

\textbf{Find:} (a) The voltage range at the ADC input. (b) The ADC
digital output range. (c) The pressure resolution in PSI per ADC count.
(d) The ADC reading for a pressure of 62.5 PSI.

\textbf{Solution:}

\begin{enumerate}
\def\labelenumi{(\alph{enumi})}
\item
  Voltage range: V\textsubscript{min} = 4 mA x 250 ohm = \textbf{1.0 V}
  V\textsubscript{max} = 20 mA x 250 ohm = \textbf{5.0 V}
\item
  ADC LSB = 5.0 / 1024 = 4.883 mV ADC\textsubscript{min} = 1.0 /
  0.004883 = 204.8 -\textgreater{} \textbf{205 counts}
  ADC\textsubscript{max} = 5.0 / 0.004883 = 1023.9 -\textgreater{}
  \textbf{1023 counts}
\end{enumerate}

Digital output range = 205 to 1023 (819 usable counts)

\begin{enumerate}
\def\labelenumi{(\alph{enumi})}
\setcounter{enumi}{2}
\item
  Pressure resolution = 100 PSI / 819 counts = \textbf{0.122 PSI per
  count}
\item
  62.5 PSI corresponds to a current of: I = 4 + (62.5 / 100) x (20 - 4)
  = 4 + 10 = 14 mA V = 14 mA x 250 = 3.5 V ADC = 3.5 / 0.004883 = 716.8
  -\textgreater{} \textbf{717 counts}
\end{enumerate}

Verification: (717 - 205) / 819 x 100 = 512 / 819 x 100 = 62.5 PSI.

\begin{center}\rule{0.5\linewidth}{0.5pt}\end{center}

\section{Problem 5.3.4}\label{problem-5.3.4}

\textbf{Given:} A 12-bit DAC with V\textsubscript{ref} = 3.3 V generates
a sawtooth waveform at 500 Hz. The waveform ramps linearly from 0 V to
3.3 V over one period, then resets to 0 V. A DMA channel transfers
values from a lookup table in SRAM to the DAC data register.

\textbf{Find:} (a) The number of lookup table entries for a voltage step
size equal to the DAC's LSB. (b) The required DMA transfer rate. (c) The
SRAM consumed by the lookup table (16-bit entries). (d) If only 256
entries are used instead, what is the voltage step size?

\textbf{Solution:}

\begin{enumerate}
\def\labelenumi{(\alph{enumi})}
\item
  The DAC has 2\textsuperscript{12} = 4096 levels. For one full ramp
  from 0 to maximum: Lookup table entries = \textbf{4096 entries} (codes
  0 through 4095)
\item
  DMA transfer rate = 4096 entries x 500 Hz = 2,048,000 transfers/s =
  \textbf{2.048 MSPS} Transfer interval = 1 / 2,048,000 = 488 ns
\item
  SRAM = 4096 x 2 bytes = \textbf{8,192 bytes = 8 KB}
\item
  With 256 entries, the step size between consecutive DAC codes: Code
  step = 4095 / 255 = 16.06, round to 16 codes per step Voltage step =
  16 x (3.3 / 4096) = 16 x 0.8057 mV = \textbf{12.89 mV}
\end{enumerate}

DMA rate = 256 x 500 = \textbf{128,000 transfers/s = 128 kSPS} (much
more manageable) SRAM = 256 x 2 = \textbf{512 bytes}

\begin{center}\rule{0.5\linewidth}{0.5pt}\end{center}

\section{Problem 5.3.5}\label{problem-5.3.5}

\textbf{Given:} An audio system uses DMA to stream 16-bit PCM audio
samples from SRAM to a DAC at 44.1 kHz (CD quality). The DMA uses a
ping-pong (double) buffer scheme: while the DAC reads from buffer A, the
CPU fills buffer B, and vice versa. Each buffer holds 512 samples. The
MCU runs at 120 MHz.

\textbf{Find:} (a) The DMA transfer rate in bytes per second. (b) The
time available for the CPU to fill each buffer. (c) The minimum CPU time
to fill one buffer if a sample generation algorithm takes 15 cycles per
sample. (d) The CPU utilization percentage for audio generation.

\textbf{Solution:}

\begin{enumerate}
\def\labelenumi{(\alph{enumi})}
\item
  DMA rate = 44,100 samples/s x 2 bytes/sample = \textbf{88,200 bytes/s
  = 88.2 KB/s}
\item
  Buffer duration = 512 samples / 44,100 samples/s = \textbf{11.61 ms}
  The CPU has 11.61 ms to fill the next 512-sample buffer before the DMA
  needs it.
\item
  CPU cycles per buffer = 512 x 15 = 7,680 cycles Time = 7,680 /
  120,000,000 = \textbf{64.0 us}
\item
  CPU utilization = 64.0 us / 11,610 us x 100 = \textbf{0.55\%}
\end{enumerate}

The DMA-based audio streaming uses negligible CPU time, leaving over
99\% of the processor available for other tasks such as UI,
communication, or audio effects processing.

\begin{center}\rule{0.5\linewidth}{0.5pt}\end{center}

\section{Problem 5.3.6}\label{problem-5.3.6}

\textbf{Given:} A safety-critical industrial controller uses a window
watchdog (WWDG) clocked at 36 MHz with a prescaler of 8. The WWDG 7-bit
downcounter is loaded with a value of 127 (0x7F) and the window register
is set to 100. The WWDG generates a reset if the counter reaches 63
(0x3F) or if the counter is refreshed while the counter value is greater
than the window value.

\textbf{Find:} (a) The watchdog tick period. (b) The timeout period
(time from reload to counter reaching 0x3F). (c) The earliest allowed
refresh time after a reload. (d) The valid refresh window in
milliseconds.

\textbf{Solution:}

\begin{enumerate}
\def\labelenumi{(\alph{enumi})}
\item
  WWDG uses a fixed additional divide-by-4096 internally:
  f\textsubscript{WWDG} = 36 MHz / (8 x 4096) = 36,000,000 / 32,768 =
  1,098.63 Hz Tick period = 1 / 1,098.63 = \textbf{910.2 us per tick}
\item
  The counter counts down from 127 to 63 (reset threshold): Ticks to
  timeout = 127 - 63 = 64 ticks Timeout period = 64 x 910.2 us =
  \textbf{58.25 ms}
\item
  Refresh is forbidden while counter \textgreater{} window value (100).
  Ticks before window opens = 127 - 100 = 27 ticks Earliest refresh time
  = 27 x 910.2 us = \textbf{24.58 ms after reload}
\item
  Valid refresh window = timeout - earliest refresh = 58.25 - 24.58 =
  \textbf{33.67 ms} The firmware must refresh the WWDG between 24.58 ms
  and 58.25 ms after each reload.
\end{enumerate}

This window prevents both stuck firmware (timeout) and runaway firmware
(refreshing too early).

\begin{center}\rule{0.5\linewidth}{0.5pt}\end{center}

\section{Problem 5.3.7}\label{problem-5.3.7}

\textbf{Given:} A 12-bit ADC samples at 1 MSPS with V\textsubscript{ref}
= 2.5 V. The signal of interest is a 0-1 V sensor output. An op-amp gain
stage amplifies the sensor output to span the full ADC range before
digitization.

\textbf{Find:} (a) The required amplifier gain. (b) The effective number
of bits (ENOB) for the original 0-1 V range without amplification. (c)
The ENOB with the gain stage (assuming the amplifier adds no noise). (d)
The improvement in voltage resolution.

\textbf{Solution:}

\begin{enumerate}
\def\labelenumi{(\alph{enumi})}
\item
  Gain = V\textsubscript{ref} / V\textsubscript{max\_signal} = 2.5 / 1.0
  = \textbf{2.5 V/V} The amplifier maps the 0-1 V sensor range to 0-2.5
  V, spanning the full ADC range.
\item
  Without amplification: Usable codes = 1.0 V / (2.5 V / 4096) = 1.0 /
  0.000610 = 1638 codes ENOB = log\textsubscript{2}(1638) =
  \textbf{10.68 bits}
\item
  With amplification, the signal spans all 4096 codes: ENOB =
  log\textsubscript{2}(4096) = \textbf{12.0 bits} (full ADC resolution
  utilized)
\item
  Resolution without gain: 2.5 / 4096 = 610.4 uV Resolution with gain:
  (1.0 / 4096) = 244.1 uV (referred to sensor input: 2.5 V / 4096 / 2.5
  = 244.1 uV)
\end{enumerate}

Improvement = 610.4 / 244.1 = \textbf{2.5x finer resolution}, matching
the gain factor.

\chapter{Chapter 5 --- Section 5.4: Communication
Interfaces}\label{chapter-5-section-5.4-communication-interfaces}

Practice problems covering UART, SPI, I2C, CAN bus, and USB
communication protocols.

\begin{center}\rule{0.5\linewidth}{0.5pt}\end{center}

\section{Problem 5.4.1}\label{problem-5.4.1}

\textbf{Given:} An MCU UART is configured at 9600 baud with 8E1 framing
(8 data bits, even parity, 1 stop bit). The MCU must transmit a 128-byte
data packet with a 2-byte CRC appended.

\textbf{Find:} (a) The total bits per byte including overhead. (b) The
total transmission time. (c) The effective data throughput (data bytes
only, excluding CRC). (d) The efficiency compared to a raw 8N1
configuration.

\textbf{Solution:}

\begin{enumerate}
\def\labelenumi{(\alph{enumi})}
\item
  Each byte frame: 1 start + 8 data + 1 parity + 1 stop = \textbf{11
  bits per byte}
\item
  Total bytes = 128 (data) + 2 (CRC) = 130 bytes Total bits = 130 x 11 =
  1,430 bits Transmission time = 1,430 / 9,600 = \textbf{148.96 ms}
\item
  Effective data throughput = 128 x 8 / 0.14896 = 1,024 / 0.14896 =
  \textbf{6,874 bps}
\item
  At 8N1 (10 bits/byte), the same 130 bytes would take: 130 x 10 / 9,600
  = 135.42 ms Efficiency of 8E1 vs 8N1: 135.42 / 148.96 = 90.9\%,
  meaning the parity bit adds \textbf{9.1\% overhead}
\end{enumerate}

Effective throughput as percentage of baud rate: 6,874 / 9,600 =
\textbf{71.6\%}

\begin{center}\rule{0.5\linewidth}{0.5pt}\end{center}

\section{Problem 5.4.2}\label{problem-5.4.2}

\textbf{Given:} An MCU communicates with 3 SPI slave devices: a flash
memory (max 20 MHz), an accelerometer (max 8 MHz), and a display
controller (max 4 MHz). The MCU's SPI peripheral is clocked from a 48
MHz APB bus. The SPI clock can only be divided by powers of 2 (2, 4, 8,
16, 32, 64, 128, 256).

\textbf{Find:} (a) The optimal SPI prescaler and actual clock frequency
for each device. (b) The time to read 4096 bytes from the flash at its
maximum speed. (c) The time to write a 128 x 160 pixel, 16-bit color
frame to the display.

\textbf{Solution:}

\begin{enumerate}
\def\labelenumi{(\alph{enumi})}
\tightlist
\item
  Flash (max 20 MHz): 48 / 2 = 24 MHz (too fast), 48 / 4 = 12 MHz
  (safe). \textbf{Prescaler = 4, f\textsubscript{SPI} = 12 MHz}
\end{enumerate}

Accelerometer (max 8 MHz): 48 / 8 = 6 MHz (safe). \textbf{Prescaler = 8,
f\textsubscript{SPI} = 6 MHz}

Display (max 4 MHz): 48 / 16 = 3 MHz (safe). \textbf{Prescaler = 16,
f\textsubscript{SPI} = 3 MHz}

\begin{enumerate}
\def\labelenumi{(\alph{enumi})}
\setcounter{enumi}{1}
\item
  Flash read: 1 byte opcode + 3 byte address + 4096 data bytes = 4100
  bytes Time = (4100 x 8) / 12,000,000 = 32,800 / 12,000,000 =
  \textbf{2.733 ms}
\item
  Display frame: 128 x 160 x 2 = 40,960 bytes Time = (40,960 x 8) /
  3,000,000 = 327,680 / 3,000,000 = \textbf{109.2 ms}
\end{enumerate}

At 109.2 ms per frame, the maximum display refresh rate is 1000 / 109.2
= \textbf{9.2 FPS}. For higher frame rates, DMA would be used to free
the CPU during the transfer.

\begin{center}\rule{0.5\linewidth}{0.5pt}\end{center}

\section{Problem 5.4.3}\label{problem-5.4.3}

\textbf{Given:} An I2C bus running at 400 kHz (Fast mode) has 5 slave
devices connected. The master polls each device by writing a 1-byte
register address and then reading 4 bytes of data. Each complete read
transaction consists of: START + address\_W(1 byte) + ACK + register(1
byte) + ACK + RESTART + address\_R(1 byte) + ACK + data(4 bytes with
ACK/NACK) + STOP.

\textbf{Find:} (a) The total bit count for one device read transaction.
(b) The time for one device read. (c) The total polling time for all 5
devices. (d) The maximum polling rate if I2C bus utilization must stay
below 50\%.

\textbf{Solution:}

\begin{enumerate}
\def\labelenumi{(\alph{enumi})}
\item
  Transaction breakdown: Write phase: START(1) + address\_W(8) + ACK(1)
  + register(8) + ACK(1) = 19 bits Read phase: RESTART(1) +
  address\_R(8) + ACK(1) + data1(8) + ACK(1) + data2(8) + ACK(1) +
  data3(8) + ACK(1) + data4(8) + NACK(1) + STOP(1) = 47 bits Total = 19
  + 47 = \textbf{66 bits per transaction}
\item
  Time per device = 66 / 400,000 = \textbf{165 us}
\item
  Total for 5 devices = 5 x 165 = \textbf{825 us}
\item
  At 50\% bus utilization, 825 us of every polling period must leave 825
  us idle: Minimum period = 825 / 0.50 = 1,650 us = 1.65 ms Maximum
  polling rate = 1 / 0.00165 = \textbf{606 Hz}
\end{enumerate}

\begin{center}\rule{0.5\linewidth}{0.5pt}\end{center}

\section{Problem 5.4.4}\label{problem-5.4.4}

\textbf{Given:} A CAN FD bus operates with a nominal bit rate of 500
kbps for the arbitration phase and 2 Mbps for the data phase. A CAN FD
frame carries 32 bytes of payload. The arbitration overhead (SOF + ID +
control bits) is 29 bits at the nominal rate, and the data + CRC + ACK +
EOF is 280 bits at the data rate, plus 3 bits for IFS at the nominal
rate.

\textbf{Find:} (a) The time for the arbitration phase. (b) The time for
the data phase. (c) The total frame time. (d) The maximum message
throughput and effective data rate.

\textbf{Solution:}

\begin{enumerate}
\def\labelenumi{(\alph{enumi})}
\item
  Arbitration phase: 29 bits at 500 kbps t\textsubscript{arb} = 29 /
  500,000 = \textbf{58.0 us}
\item
  Data phase: 280 bits at 2 Mbps t\textsubscript{data} = 280 / 2,000,000
  = \textbf{140.0 us}
\item
  IFS: 3 bits at 500 kbps = 6.0 us Total frame time = 58.0 + 140.0 + 6.0
  = \textbf{204.0 us}
\item
  Maximum throughput = 1 / 204.0 us = \textbf{4,902 frames/s} Effective
  data rate = 4,902 x 32 bytes = 156,863 bytes/s = \textbf{1.255 Mbps}
\end{enumerate}

Compared to classical CAN (244 kbps effective from Example 5.4.4), CAN
FD provides a \textbf{5.14x improvement} in data throughput by using a
higher bit rate during the data phase and a larger payload.

\begin{center}\rule{0.5\linewidth}{0.5pt}\end{center}

\section{Problem 5.4.5}\label{problem-5.4.5}

\textbf{Given:} An embedded device uses USB 2.0 High Speed (480 Mbps)
with a bulk transfer endpoint. The maximum bulk packet size for High
Speed is 512 bytes. Protocol overhead (token packets, handshakes,
inter-packet gaps, and microframe scheduling) reduces the usable
bandwidth to approximately 53\% for bulk transfers. The device must
stream 24-bit audio at 96 kHz in stereo.

\textbf{Find:} (a) The required data rate for the audio stream. (b) The
maximum achievable bulk transfer throughput. (c) The percentage of USB
bandwidth used by the audio stream. (d) How many additional audio
channels could be supported simultaneously?

\textbf{Solution:}

\begin{enumerate}
\def\labelenumi{(\alph{enumi})}
\item
  Audio data rate = 96,000 samples/s x 2 channels x 3 bytes/sample =
  \textbf{576,000 bytes/s = 4.608 Mbps}
\item
  Usable bandwidth = 480 Mbps x 0.53 = 254.4 Mbps Maximum bulk
  throughput = 254.4 / 8 = \textbf{31.8 MB/s}
\item
  USB bandwidth used = 0.576 / 31.8 x 100 = \textbf{1.81\%}
\item
  Available bandwidth for additional audio = 31.8 - 0.576 = 31.224 MB/s
  Additional stereo channels = floor(31.224 / 0.576) = \textbf{54
  additional stereo pairs}
\end{enumerate}

Or equivalently, \textbf{108 additional mono channels} at 24-bit/96 kHz.
USB High Speed has vastly more bandwidth than needed for audio, which is
why professional multi-channel audio interfaces commonly use USB 2.0
High Speed.

\begin{center}\rule{0.5\linewidth}{0.5pt}\end{center}

\section{Problem 5.4.6}\label{problem-5.4.6}

\textbf{Given:} An RS-485 multi-drop network connects one master and 8
slave devices on a shared bus at 19200 baud (8N1). The master polls each
slave in round-robin fashion. Each poll consists of a 5-byte command
from the master and a 10-byte response from the slave. There is a 1 ms
turnaround delay after each direction change (master-to-slave and
slave-to-master) for bus driver enable/disable.

\textbf{Find:} (a) The time for one poll-response cycle to a single
slave. (b) The total time to poll all 8 slaves. (c) The maximum polling
rate. (d) The bus efficiency (data bytes / total time).

\textbf{Solution:}

\begin{enumerate}
\def\labelenumi{(\alph{enumi})}
\tightlist
\item
  At 19200 baud, 8N1: 10 bits/byte, so 1920 bytes/s.
\end{enumerate}

Master command: 5 bytes x (10/19200) = 2.604 ms Turnaround delay: 1.0 ms
Slave response: 10 bytes x (10/19200) = 5.208 ms Turnaround delay: 1.0
ms Total per slave = 2.604 + 1.0 + 5.208 + 1.0 = \textbf{9.812 ms}

\begin{enumerate}
\def\labelenumi{(\alph{enumi})}
\setcounter{enumi}{1}
\item
  Total for 8 slaves = 8 x 9.812 = \textbf{78.5 ms}
\item
  Maximum polling rate = 1 / 0.0785 = \textbf{12.74 polls/s per slave}
\item
  Data bytes per cycle = 8 x (5 + 10) = 120 bytes in 78.5 ms Throughput
  = 120 / 0.0785 = 1,529 bytes/s Bus efficiency = 1,529 / 1,920 =
  \textbf{79.6\%}
\end{enumerate}

The turnaround delays consume 2.0 ms out of 9.812 ms per slave (20.4\%),
which is the dominant source of inefficiency.

\chapter{Chapter 5 --- Section 5.5:
Interrupts}\label{chapter-5-section-5.5-interrupts}

Practice problems covering interrupt priority, latency, NVIC behavior,
and ISR design.

\begin{center}\rule{0.5\linewidth}{0.5pt}\end{center}

\section{Problem 5.5.1}\label{problem-5.5.1}

\textbf{Given:} A Cortex-M3 MCU at 72 MHz has the following interrupt
configuration: - Timer interrupt: priority 0 (highest), ISR takes 50
cycles - UART RX interrupt: priority 1, ISR takes 80 cycles - ADC
complete interrupt: priority 2, ISR takes 120 cycles - SPI transfer
complete: priority 3 (lowest), ISR takes 40 cycles

The NVIC requires 12 cycles for context save (entry) and 12 cycles for
context restore (exit). The SPI ISR is currently executing when the
Timer, UART, and ADC interrupts all trigger simultaneously.

\textbf{Find:} (a) The sequence of ISR execution. (b) The total time
from the simultaneous trigger until all four ISRs have completed and the
main program resumes.

\textbf{Solution:}

\begin{enumerate}
\def\labelenumi{(\alph{enumi})}
\tightlist
\item
  The SPI ISR (priority 3) is preempted by the Timer interrupt (priority
  0, highest of the three new interrupts). After Timer completes, UART
  (priority 1) runs next, then ADC (priority 2), then the remainder of
  SPI (already executing, 0 cycles remaining since it was preempted at
  an arbitrary point --- assume worst case: full SPI ISR re-runs after
  restoration).
\end{enumerate}

Execution sequence: 1. SPI ISR is interrupted (context save for
preemption) 2. Timer ISR (priority 0) runs 3. UART ISR (priority 1) runs
4. ADC ISR (priority 2) runs 5. SPI ISR resumes (context restore to SPI)
6. Return to main program

\begin{enumerate}
\def\labelenumi{(\alph{enumi})}
\setcounter{enumi}{1}
\tightlist
\item
  Using ARM tail-chaining (no full context save/restore between
  consecutive ISRs of different priority):
\end{enumerate}

\begin{itemize}
\tightlist
\item
  Initial preemption of SPI: 12 cycles (save SPI context)
\item
  Timer ISR: 50 cycles
\item
  Tail-chain to UART (6 cycles, reduced overhead): 6 cycles
\item
  UART ISR: 80 cycles
\item
  Tail-chain to ADC: 6 cycles
\item
  ADC ISR: 120 cycles
\item
  Restore SPI context: 12 cycles (late arrival optimization)
\item
  SPI ISR completes remaining work: 40 cycles (worst case)
\item
  Restore main context: 12 cycles
\end{itemize}

Total cycles = 12 + 50 + 6 + 80 + 6 + 120 + 12 + 40 + 12 = 338 cycles
Total time = 338 / 72,000,000 = \textbf{4.69 us}

\begin{center}\rule{0.5\linewidth}{0.5pt}\end{center}

\section{Problem 5.5.2}\label{problem-5.5.2}

\textbf{Given:} An MCU handles a high-speed ADC sampling at 500 kHz
using an interrupt-driven approach. Each ADC ISR reads the 16-bit result
and stores it in a circular buffer (30 CPU cycles per ISR, including
context save/restore). The MCU runs at 144 MHz.

\textbf{Find:} (a) The CPU utilization consumed by the ADC interrupt.
(b) The maximum number of simultaneous 500 kHz ADC channels the CPU can
support before exceeding 50\% utilization. (c) The advantage of
switching to DMA (assume DMA interrupt fires once per 256-sample
buffer).

\textbf{Solution:}

\begin{enumerate}
\def\labelenumi{(\alph{enumi})}
\item
  ISR frequency = 500,000 interrupts/s Cycles per second = 500,000 x 30
  = 15,000,000 cycles/s CPU utilization = 15,000,000 / 144,000,000 =
  \textbf{10.42\%}
\item
  At 50\% utilization: max cycles = 0.50 x 144,000,000 = 72,000,000
  cycles/s Max channels = 72,000,000 / 15,000,000 = \textbf{4.8, so 4
  channels}
\item
  With DMA, the interrupt fires every 256 samples: DMA ISR frequency =
  500,000 / 256 = 1,953 interrupts/s Assuming 50 cycles per DMA ISR
  (slightly longer for buffer management): Cycles per second = 1,953 x
  50 = 97,656 cycles/s CPU utilization = 97,656 / 144,000,000 =
  \textbf{0.068\%}
\end{enumerate}

Reduction = 10.42 / 0.068 = \textbf{153x less CPU overhead} with DMA.

\begin{center}\rule{0.5\linewidth}{0.5pt}\end{center}

\section{Problem 5.5.3}\label{problem-5.5.3}

\textbf{Given:} A real-time system requires that a motor control ISR
always completes within 10 us of its trigger. The MCU runs at 200 MHz.
The ISR itself takes 150 cycles. There are two higher-priority
interrupts: a safety shutdown ISR (priority 0, 80 cycles) and a
communication ISR (priority 1, 200 cycles). The motor ISR is priority 2.
Context switch overhead is 12 cycles.

\textbf{Find:} (a) The worst-case latency for the motor ISR (from
trigger to completion). (b) Does the system meet the 10 us deadline? (c)
What changes could bring it within deadline if not?

\textbf{Solution:}

\begin{enumerate}
\def\labelenumi{(\alph{enumi})}
\tightlist
\item
  Worst case: motor ISR triggers just as the safety ISR begins, and the
  communication ISR also triggers.
\end{enumerate}

Worst-case timeline: - Wait for safety ISR to complete: 12 (entry) + 80
(execution) = 92 cycles - Communication ISR runs (higher priority than
motor): 6 (tail-chain) + 200 = 206 cycles - Motor ISR entry and
execution: 6 (tail-chain) + 150 = 156 cycles

Worst-case total from motor trigger to motor completion = 92 + 206 + 156
= 454 cycles Time = 454 / 200,000,000 = \textbf{2.27 us}

\begin{enumerate}
\def\labelenumi{(\alph{enumi})}
\setcounter{enumi}{1}
\item
  2.27 us \textless{} 10 us. \textbf{Yes, the system meets the deadline}
  with 7.73 us of margin.
\item
  Not needed in this case. However, if the deadline were tighter (e.g.,
  2 us), options would include:
\end{enumerate}

\begin{itemize}
\tightlist
\item
  Elevating motor ISR to priority 0 (above communication)
\item
  Reducing communication ISR execution time
\item
  Splitting the communication ISR into a short critical section and
  deferred processing
\end{itemize}

\begin{center}\rule{0.5\linewidth}{0.5pt}\end{center}

\section{Problem 5.5.4}\label{problem-5.5.4}

\textbf{Given:} An MCU application uses a bare-metal super-loop
architecture. The main loop takes 800 us per iteration. A GPIO interrupt
detects a button press and sets a flag (ISR takes 0.5 us). The main loop
checks the flag and performs a 5 ms debounce verification. The button
press must be acknowledged within 50 ms.

\textbf{Find:} (a) The worst-case response time from button press to
acknowledgment. (b) Whether the 50 ms deadline is met. (c) The response
time if the main loop period were increased to 20 ms (e.g., by adding a
complex computation).

\textbf{Solution:}

\begin{enumerate}
\def\labelenumi{(\alph{enumi})}
\tightlist
\item
  Worst-case response:
\end{enumerate}

\begin{itemize}
\tightlist
\item
  ISR latency: negligible (\textless{} 1 us for flag set)
\item
  Worst-case time until main loop checks flag: the press occurs just
  after the flag check point, so the flag isn't seen until the next
  iteration = \textbf{800 us}
\item
  Debounce wait: \textbf{5 ms = 5,000 us}
\item
  Processing after debounce: negligible
\end{itemize}

Total worst-case = 0.5 + 800 + 5,000 = \textbf{5,800.5 us = 5.8 ms}

\begin{enumerate}
\def\labelenumi{(\alph{enumi})}
\setcounter{enumi}{1}
\item
  5.8 ms \textless\textless{} 50 ms. \textbf{Yes, the deadline is easily
  met.}
\item
  With 20 ms main loop: Worst-case = 0.5 + 20,000 + 5,000 =
  \textbf{25,000.5 us = 25.0 ms}
\end{enumerate}

Still within the 50 ms deadline, with 25 ms of margin. If the main loop
grew to 45 ms, the worst case would be 50 ms, reaching the limit and
suggesting the need for an RTOS or performing the debounce in a timer
ISR instead.

\chapter{Chapter 5 --- Section 5.6: Real-Time Operating Systems
(RTOS)}\label{chapter-5-section-5.6-real-time-operating-systems-rtos}

Practice problems covering RTOS task scheduling, CPU utilization, and
Rate Monotonic Analysis.

\begin{center}\rule{0.5\linewidth}{0.5pt}\end{center}

\section{Problem 5.6.1}\label{problem-5.6.1}

\textbf{Given:} A FreeRTOS application on a Cortex-M4 at 120 MHz has
five tasks:

{\def\LTcaptype{none} % do not increment counter
\begin{longtable}[]{@{}llll@{}}
\toprule\noalign{}
Task & Priority & Period & Execution Time \\
\midrule\noalign{}
\endhead
\bottomrule\noalign{}
\endlastfoot
Motor control & 5 (highest) & 500 us & 80 us \\
Sensor sampling & 4 & 2 ms & 300 us \\
PID computation & 3 & 5 ms & 1.2 ms \\
Communication & 2 & 20 ms & 3 ms \\
Display update & 1 (lowest) & 100 ms & 15 ms \\
\end{longtable}
}

\textbf{Find:} (a) The CPU utilization for each task and the total. (b)
Whether the task set is schedulable using Rate Monotonic Analysis (RMA).
(c) The idle time percentage.

\textbf{Solution:}

\begin{enumerate}
\def\labelenumi{(\alph{enumi})}
\tightlist
\item
  Utilization per task U\textsubscript{i} = C\textsubscript{i} /
  T\textsubscript{i}:
\end{enumerate}

Motor: 80 / 500 = 0.160 = 16.0\% Sensor: 300 / 2,000 = 0.150 = 15.0\%
PID: 1,200 / 5,000 = 0.240 = 24.0\% Communication: 3,000 / 20,000 =
0.150 = 15.0\% Display: 15,000 / 100,000 = 0.150 = 15.0\%

\textbf{Total utilization = 0.160 + 0.150 + 0.240 + 0.150 + 0.150 =
0.850 = 85.0\%}

\begin{enumerate}
\def\labelenumi{(\alph{enumi})}
\setcounter{enumi}{1}
\tightlist
\item
  RMA schedulability bound for n = 5 tasks: U\textsubscript{bound} = n x
  (2\textsuperscript{1/n} - 1) = 5 x (2\textsuperscript{0.2} - 1) = 5 x
  (1.1487 - 1) = 5 x 0.1487 = \textbf{0.7435 = 74.35\%}
\end{enumerate}

Since 85.0\% \textgreater{} 74.35\%, the RMA sufficient condition is
\textbf{not satisfied}.

However, the RMA bound is a sufficient condition, not a necessary one.
The task set may still be schedulable. A detailed response-time analysis
would be needed to confirm. For guaranteed schedulability, the total
utilization should be reduced below 74.35\%.

\begin{enumerate}
\def\labelenumi{(\alph{enumi})}
\setcounter{enumi}{2}
\tightlist
\item
  Idle time = 100\% - 85.0\% = \textbf{15.0\%}
\end{enumerate}

\begin{center}\rule{0.5\linewidth}{0.5pt}\end{center}

\section{Problem 5.6.2}\label{problem-5.6.2}

\textbf{Given:} Two RTOS tasks share a resource protected by a mutex.
Task A (priority 3, high) and Task C (priority 1, low) both need the
mutex. Task B (priority 2, medium) does not use the mutex. Task C
acquires the mutex and begins its critical section (2 ms long). Task A
becomes ready and needs the mutex.

\textbf{Find:} (a) Describe the priority inversion problem that occurs
without mitigation. (b) Calculate the worst-case delay for Task A if
Task B runs for 10 ms. (c) How does priority inheritance solve this
problem?

\textbf{Solution:}

\begin{enumerate}
\def\labelenumi{(\alph{enumi})}
\tightlist
\item
  \textbf{Priority inversion scenario:}
\end{enumerate}

\begin{enumerate}
\def\labelenumi{\arabic{enumi}.}
\tightlist
\item
  Task C (priority 1) acquires the mutex and enters its critical
  section.
\item
  Task A (priority 3) preempts C and tries to acquire the mutex --- it
  blocks because C holds it.
\item
  Task B (priority 2) becomes ready. Since A is blocked and B has higher
  priority than C, \textbf{B preempts C}.
\item
  Task C cannot complete its critical section to release the mutex until
  B finishes.
\item
  Task A (highest priority) is effectively blocked by Task B (medium
  priority), which doesn't even use the shared resource. This is
  \textbf{priority inversion}.
\end{enumerate}

\begin{enumerate}
\def\labelenumi{(\alph{enumi})}
\setcounter{enumi}{1}
\tightlist
\item
  Worst-case delay for Task A:
\end{enumerate}

\begin{itemize}
\tightlist
\item
  B runs for 10 ms (preempting C)
\item
  C finishes critical section: 2 ms
\item
  Total delay for A = 10 + 2 = \textbf{12 ms}
\end{itemize}

Task A, the highest-priority task, waits 12 ms for a 2 ms critical
section.

\begin{enumerate}
\def\labelenumi{(\alph{enumi})}
\setcounter{enumi}{2}
\tightlist
\item
  \textbf{Priority inheritance protocol:} When Task A blocks on the
  mutex held by Task C, the RTOS temporarily raises C's priority to
  match A's priority (3). Now C cannot be preempted by B (priority 2),
  so C completes its 2 ms critical section immediately, releases the
  mutex, and its priority reverts to 1. Task A acquires the mutex with
  only \textbf{2 ms of delay} instead of 12 ms.
\end{enumerate}

\begin{center}\rule{0.5\linewidth}{0.5pt}\end{center}

\section{Problem 5.6.3}\label{problem-5.6.3}

\textbf{Given:} An RTOS tick timer runs at 1 kHz (1 ms tick). A task
uses vTaskDelay(10) to sleep for 10 ticks. The task's processing before
the delay takes 0.3 ms.

\textbf{Find:} (a) The actual delay range (minimum and maximum). (b) The
total period of the task (execution + delay). (c) How vTaskDelayUntil()
improves periodicity, and what period it would produce.

\textbf{Solution:}

\begin{enumerate}
\def\labelenumi{(\alph{enumi})}
\tightlist
\item
  vTaskDelay(10) delays for 10 \textbf{complete} tick periods from the
  time the call is made. However, the first tick may occur anywhere from
  0 to 1 ms after the call (depending on where in the current tick
  period the call lands).
\end{enumerate}

Minimum delay: 10 ticks x 1 ms - 1 ms (nearly aligned with tick) =
\textbf{9 ms} (just barely misses a tick boundary and counts 10 from the
next tick minus almost-full first tick)

More precisely: minimum delay = 9.0+ ms (10 ticks, first is partial),
maximum delay = 10.0 ms (call coincides with tick edge).

Delay range: \textbf{9 to 10 ms}

\begin{enumerate}
\def\labelenumi{(\alph{enumi})}
\setcounter{enumi}{1}
\item
  Total period = processing + delay = 0.3 ms + (9 to 10 ms) =
  \textbf{9.3 to 10.3 ms} The period jitters by up to 1 ms (one tick
  period).
\item
  vTaskDelayUntil() delays until an \textbf{absolute tick count},
  removing the jitter. If the wake time is set to a period of 10 ticks:
  Period = exactly \textbf{10 ms} (10 ticks), regardless of when
  processing finishes. The task always wakes at the same phase relative
  to the tick timer, providing deterministic periodicity with zero
  jitter (assuming processing completes within the period).
\end{enumerate}

\begin{center}\rule{0.5\linewidth}{0.5pt}\end{center}

\section{Problem 5.6.4}\label{problem-5.6.4}

\textbf{Given:} An RTOS message queue has a depth of 16 entries, each
holding a 32-byte message. A producer task generates messages at a rate
of 200 messages/second. A consumer task processes each message in 3 ms.

\textbf{Find:} (a) The consumer's processing rate. (b) Whether the queue
is stable (consumer keeps up with producer). (c) The maximum burst of
messages the queue can absorb if the consumer is temporarily blocked for
50 ms. (d) Whether the queue overflows during the burst.

\textbf{Solution:}

\begin{enumerate}
\def\labelenumi{(\alph{enumi})}
\item
  Consumer rate = 1 / 3 ms = \textbf{333.3 messages/s}
\item
  Producer rate = 200 msg/s, Consumer rate = 333.3 msg/s Since 333.3
  \textgreater{} 200, \textbf{the queue is stable} --- the consumer
  keeps up with the producer.
\item
  During a 50 ms blockage: Messages arriving = 200 x 0.050 = \textbf{10
  messages}
\item
  Queue depth = 16 entries. During the blockage, 10 messages accumulate.
  10 \textless{} 16, so \textbf{the queue does not overflow}.
\end{enumerate}

After the blockage, the consumer processes the backlog at its net drain
rate: Net rate = 333.3 - 200 = 133.3 msg/s Time to drain backlog = 10 /
133.3 = \textbf{75 ms}

If the consumer were blocked for 80 ms, the burst would be 200 x 0.080 =
16 messages, exactly filling the queue.

\begin{center}\rule{0.5\linewidth}{0.5pt}\end{center}

\section{Problem 5.6.5}\label{problem-5.6.5}

\textbf{Given:} An RTOS application must allocate memory for tasks. Each
task requires a stack plus a Task Control Block (TCB) of 88 bytes. The
system has 32 KB of RTOS heap. Tasks and their minimum stack
requirements:

{\def\LTcaptype{none} % do not increment counter
\begin{longtable}[]{@{}ll@{}}
\toprule\noalign{}
Task & Stack Size \\
\midrule\noalign{}
\endhead
\bottomrule\noalign{}
\endlastfoot
Main & 2048 bytes \\
Sensor & 512 bytes \\
Comm & 1024 bytes \\
Logger & 4096 bytes \\
Idle (automatic) & 128 bytes \\
Timer daemon (automatic) & 256 bytes \\
\end{longtable}
}

\textbf{Find:} (a) Total memory for all tasks. (b) RTOS heap
utilization. (c) Maximum number of additional 512-byte-stack tasks.

\textbf{Solution:}

\begin{enumerate}
\def\labelenumi{(\alph{enumi})}
\tightlist
\item
  Memory per task = stack + TCB (88 bytes)
\end{enumerate}

Main: 2048 + 88 = 2,136 bytes Sensor: 512 + 88 = 600 bytes Comm: 1024 +
88 = 1,112 bytes Logger: 4096 + 88 = 4,184 bytes Idle: 128 + 88 = 216
bytes Timer: 256 + 88 = 344 bytes

Total = 2,136 + 600 + 1,112 + 4,184 + 216 + 344 = \textbf{8,592 bytes}

\begin{enumerate}
\def\labelenumi{(\alph{enumi})}
\setcounter{enumi}{1}
\item
  Heap utilization = 8,592 / 32,768 = \textbf{26.2\%}
\item
  Remaining heap = 32,768 - 8,592 = 24,176 bytes Memory per additional
  task = 512 + 88 = 600 bytes Max additional tasks = floor(24,176 / 600)
  = \textbf{40 additional tasks}
\end{enumerate}

\chapter{Chapter 5 --- Section 5.7: Power
Management}\label{chapter-5-section-5.7-power-management}

Practice problems covering sleep modes, battery life calculations, and
low-power design techniques.

\begin{center}\rule{0.5\linewidth}{0.5pt}\end{center}

\section{Problem 5.7.1}\label{problem-5.7.1}

\textbf{Given:} A wildlife tracking collar runs on a 3.7 V, 2500 mAh
lithium-polymer battery. The MCU has three power modes: - Active (GPS
acquisition): 45 mA for 30 seconds - Active (data processing + satellite
uplink): 120 mA for 5 seconds - Stop mode: 8 uA

The collar wakes every 4 hours to acquire a GPS fix and transmit
location data.

\textbf{Find:} (a) The average current consumption. (b) The expected
battery life. (c) The battery life if the wake interval is reduced to 1
hour for tracking an endangered animal.

\textbf{Solution:}

\begin{enumerate}
\def\labelenumi{(\alph{enumi})}
\tightlist
\item
  Period T = 4 hours = 14,400 seconds.
\end{enumerate}

Charge per cycle: GPS: 45 mA x 30 s = 1,350 mA-s = 0.375 mAh Uplink: 120
mA x 5 s = 600 mA-s = 0.1667 mAh Sleep: 0.008 mA x (14,400 - 35) s =
0.008 x 14,365 = 114.92 mA-s = 0.03192 mAh

Total per cycle = 0.375 + 0.1667 + 0.03192 = 0.5736 mAh Average current
= 0.5736 mAh / 4 h = \textbf{0.1434 mA = 143.4 uA}

\begin{enumerate}
\def\labelenumi{(\alph{enumi})}
\setcounter{enumi}{1}
\item
  Battery life = 2500 / 0.1434 = 17,434 hours = \textbf{726 days = 1.99
  years}
\item
  At 1-hour intervals (T = 3600 s): Sleep charge: 0.008 x (3600 - 35) =
  28.52 mA-s = 0.007922 mAh Total per cycle = 0.375 + 0.1667 + 0.007922
  = 0.5496 mAh Average current = 0.5496 / 1 = \textbf{0.5496 mA = 549.6
  uA}
\end{enumerate}

Battery life = 2500 / 0.5496 = 4,548 hours = \textbf{189 days = 6.2
months}

The 4x increase in wake frequency increases average current by 3.83x
because the active-mode energy dominates.

\begin{center}\rule{0.5\linewidth}{0.5pt}\end{center}

\section{Problem 5.7.2}\label{problem-5.7.2}

\textbf{Given:} An MCU at V\textsubscript{DD} = 1.8 V runs at 80 MHz,
drawing 12 mA in active mode. The dynamic power follows P =
C\textsubscript{L} x V\textsubscript{DD}\textsuperscript{2} x f, and the
voltage regulator supports 1.2 V operation at frequencies up to 48 MHz.

\textbf{Find:} (a) The estimated load capacitance C\textsubscript{L}.
(b) The active current at 1.2 V / 48 MHz. (c) The power savings as a
percentage. (d) The energy to execute a fixed task of 10 million cycles
at each operating point.

\textbf{Solution:}

\begin{enumerate}
\def\labelenumi{(\alph{enumi})}
\item
  P\textsubscript{active} = V\textsubscript{DD} x I = 1.8 x 0.012 =
  0.0216 W = 21.6 mW C\textsubscript{L} = P /
  (V\textsubscript{DD}\textsuperscript{2} x f) = 0.0216 /
  (1.8\textsuperscript{2} x 80 x 10\textsuperscript{6}) = 0.0216 / (3.24
  x 8 x 10\textsuperscript{7}) C\textsubscript{L} = 0.0216 / 2.592 x
  10\textsuperscript{8} = \textbf{83.3 pF}
\item
  P at 1.2 V / 48 MHz = 83.3 x 10\textsuperscript{-12} x
  1.2\textsuperscript{2} x 48 x 10\textsuperscript{6} = 83.3 x
  10\textsuperscript{-12} x 1.44 x 4.8 x 10\textsuperscript{7} = 83.3 x
  10\textsuperscript{-12} x 6.912 x 10\textsuperscript{7} = 5.76 x
  10\textsuperscript{-3} W = 5.76 mW I = P / V\textsubscript{DD} = 5.76
  / 1.2 = \textbf{4.8 mA}
\item
  Power savings = (21.6 - 5.76) / 21.6 x 100 = \textbf{73.3\%}
\item
  At 80 MHz: time = 10\textsuperscript{7} / 80 x 10\textsuperscript{6} =
  0.125 s Energy = 21.6 mW x 0.125 s = \textbf{2.70 mJ}
\end{enumerate}

At 48 MHz: time = 10\textsuperscript{7} / 48 x 10\textsuperscript{6} =
0.2083 s Energy = 5.76 mW x 0.2083 s = \textbf{1.20 mJ}

Energy savings = (2.70 - 1.20) / 2.70 x 100 = \textbf{55.6\%}

Running slower at lower voltage saves significant energy per task
because energy scales as V\textsuperscript{2}, even though the task
takes longer.

\begin{center}\rule{0.5\linewidth}{0.5pt}\end{center}

\section{Problem 5.7.3}\label{problem-5.7.3}

\textbf{Given:} A battery-powered sensor hub polls 3 external I2C
sensors. Each sensor has a standby current of 50 uA and an active
current of 2 mA during measurement. Measurements take 10 ms each. The
hub polls every 60 seconds. An external load switch (I\textsubscript{Q}
= 0.5 uA) can cut power to all three sensors between measurements.

\textbf{Find:} (a) Average sensor current without the load switch
(sensors in standby between measurements). (b) Average sensor current
with the load switch. (c) Battery life improvement (battery = 3.3 V, 500
mAh, MCU sleep current = 5 uA).

\textbf{Solution:}

\begin{enumerate}
\def\labelenumi{(\alph{enumi})}
\tightlist
\item
  Without load switch (3 sensors always powered): Active time per poll =
  3 x 10 ms = 30 ms at 3 x 2 mA = 6 mA Standby time = 60,000 - 30 =
  59,970 ms at 3 x 50 uA = 150 uA
\end{enumerate}

I\textsubscript{avg\_sensors} = (6 x 30 + 0.150 x 59,970) / 60,000 =
(180 + 8,995.5) / 60,000 = 9,175.5 / 60,000 = \textbf{0.1529 mA = 152.9
uA}

\begin{enumerate}
\def\labelenumi{(\alph{enumi})}
\setcounter{enumi}{1}
\tightlist
\item
  With load switch: Active time = 30 ms at 6 mA (sensors powered) Off
  time = 59,970 ms at 0.5 uA (load switch quiescent only)
\end{enumerate}

I\textsubscript{avg\_sensors} = (6 x 30 + 0.0005 x 59,970) / 60,000 =
(180 + 29.985) / 60,000 = 209.985 / 60,000 = \textbf{0.003500 mA = 3.50
uA}

\begin{enumerate}
\def\labelenumi{(\alph{enumi})}
\setcounter{enumi}{2}
\tightlist
\item
  Total system current: Without switch: 5 (MCU) + 152.9 (sensors) =
  157.9 uA With switch: 5 (MCU) + 3.5 (sensors) = 8.5 uA
\end{enumerate}

Battery life without switch = 500 / 0.1579 = 3,167 hours = \textbf{132
days} Battery life with switch = 500 / 0.0085 = 58,824 hours =
\textbf{6.7 years}

Improvement factor = 58,824 / 3,167 = \textbf{18.6x}

The load switch nearly eliminates the sensor standby current, which
dominated the power budget.

\begin{center}\rule{0.5\linewidth}{0.5pt}\end{center}

\section{Problem 5.7.4}\label{problem-5.7.4}

\textbf{Given:} An IoT gateway MCU processes incoming packets. The
packet arrival rate follows a Poisson distribution with an average of 5
packets per second. Each packet requires 2 ms of active processing at 25
mA. Between packets, the MCU enters Sleep mode at 1 mA.

\textbf{Find:} (a) The average active duty cycle. (b) The average
current. (c) If Stop mode (10 uA) is used instead of Sleep mode and
waking takes an additional 0.5 ms at 25 mA, what is the new average
current?

\textbf{Solution:}

\begin{enumerate}
\def\labelenumi{(\alph{enumi})}
\item
  Active time per second = 5 packets x 2 ms = 10 ms Duty cycle = 10 /
  1000 = \textbf{1.0\%}
\item
  I\textsubscript{avg} = 25 x 0.01 + 1.0 x 0.99 = 0.25 + 0.99 =
  \textbf{1.24 mA}
\item
  With Stop mode and 0.5 ms wake penalty: Active per packet = 2.0 + 0.5
  = 2.5 ms at 25 mA Active time per second = 5 x 2.5 = 12.5 ms Sleep
  time = 987.5 ms at 0.010 mA
\end{enumerate}

I\textsubscript{avg} = 25 x 0.0125 + 0.010 x 0.9875 = 0.3125 + 0.009875
= \textbf{0.322 mA}

Improvement = 1.24 / 0.322 = \textbf{3.85x reduction in average current}

Despite the wake-up penalty, Stop mode saves significant power because
the sleep current drops from 1 mA to 10 uA, and the MCU spends 99\% of
its time sleeping.

\chapter{Chapter 5 --- Section 5.8: Development and
Debugging}\label{chapter-5-section-5.8-development-and-debugging}

Practice problems covering debug interfaces, firmware download, and
bootloader design.

\begin{center}\rule{0.5\linewidth}{0.5pt}\end{center}

\section{Problem 5.8.1}\label{problem-5.8.1}

\textbf{Given:} A development team uses a CMSIS-DAP debug probe with SWD
at 10 MHz to program a Cortex-M7 target. The firmware image is 512 KB.
The MCU flash has 8 sectors of 64 KB each. Each sector erase takes 250
ms. Flash programming speed (after erase) is 80 KB/s. After programming,
a full read-back verification is performed at the SWD link speed.

\textbf{Find:} (a) The total sector erase time. (b) The flash
programming time. (c) The SWD raw throughput (assuming 50\% protocol
efficiency). (d) The verification read-back time. (e) The total firmware
download-and-verify time.

\textbf{Solution:}

\begin{enumerate}
\def\labelenumi{(\alph{enumi})}
\item
  Sectors to erase = 512 / 64 = 8 sectors Total erase time = 8 x 250 ms
  = \textbf{2.0 seconds}
\item
  Programming time = 512 KB / 80 KB/s = \textbf{6.4 seconds}
\item
  SWD at 10 MHz, 1-bit data line, 50\% efficiency: Raw throughput = 10
  MHz x 0.50 / 8 bits = \textbf{625 KB/s}
\item
  Verification reads back the entire 512 KB at SWD speed: Read-back time
  = 512 / 625 = \textbf{0.819 seconds}
\item
  Total = erase + program + verify = 2.0 + 6.4 + 0.819 = \textbf{9.22
  seconds}
\end{enumerate}

The programming step is the bottleneck (69\% of total time), limited by
the flash write speed rather than the SWD link bandwidth.

\begin{center}\rule{0.5\linewidth}{0.5pt}\end{center}

\section{Problem 5.8.2}\label{problem-5.8.2}

\textbf{Given:} A custom bootloader for a medical device uses dual-bank
(A/B) firmware slots. The MCU has 1 MB of flash organized as: -
Bootloader: 64 KB (sectors 0-3, 16 KB each) - Configuration/keys: 16 KB
(sector 4) - Slot A: 464 KB (sectors 5-11) - Slot B: 464 KB (sectors
12-18)

The active firmware is 320 KB. Updates arrive over BLE (Bluetooth Low
Energy) at an effective throughput of 20 KB/s. Flash erase for Slot B
takes 3.5 seconds. Flash programming runs at 40 KB/s. A SHA-256 hash
verification of the programmed image takes 800 ms.

\textbf{Find:} (a) The BLE transfer time for the firmware image. (b) The
total update time (receive + erase + program + verify). (c) If the
device must remain operational during the update, explain how the A/B
scheme enables this.

\textbf{Solution:}

\begin{enumerate}
\def\labelenumi{(\alph{enumi})}
\item
  BLE transfer time = 320 KB / 20 KB/s = \textbf{16.0 seconds}
\item
  Total update time:
\end{enumerate}

\begin{itemize}
\tightlist
\item
  BLE receive: 16.0 s
\item
  Flash erase (Slot B): 3.5 s
\item
  Flash program: 320 / 40 = 8.0 s
\item
  SHA-256 verify: 0.8 s
\end{itemize}

Total = 16.0 + 3.5 + 8.0 + 0.8 = \textbf{28.3 seconds}

Note: BLE receive and flash erase can overlap if the bootloader erases
sectors progressively while receiving data. In that case: Overlapped
total = max(16.0, 3.5) + 8.0 + 0.8 = 16.0 + 8.0 + 0.8 = \textbf{24.8
seconds}

\begin{enumerate}
\def\labelenumi{(\alph{enumi})}
\setcounter{enumi}{2}
\tightlist
\item
  \textbf{A/B update scheme operation:} The device runs its current
  firmware from Slot A throughout the entire update process. The new
  firmware is written to Slot B. If the download or verification fails,
  Slot A remains untouched and the device continues operating normally.
  After successful verification, the bootloader updates a flag
  indicating Slot B is the new active image. On the next reboot, the
  bootloader loads from Slot B. If Slot B fails to boot (detected by a
  watchdog or boot counter), the bootloader automatically reverts to
  Slot A.
\end{enumerate}

This ensures the device is \textbf{never bricked} by a failed update ---
critical for medical devices under IEC 62304.

\begin{center}\rule{0.5\linewidth}{0.5pt}\end{center}

\section{Problem 5.8.3}\label{problem-5.8.3}

\textbf{Given:} A Cortex-M4 MCU supports 6 hardware breakpoints and 4
data watchpoints through the CoreSight debug architecture. A developer
is debugging a firmware with 150 functions. The bug manifests as a
corrupted 4-byte variable at address 0x2000\_1A00 that changes
unexpectedly during normal execution.

\textbf{Find:} (a) How many hardware breakpoints can be simultaneously
active? (b) How a data watchpoint can identify the source of the
corruption. (c) If the developer needs to monitor 5 different variables
simultaneously, what limitation is encountered? (d) An alternative
approach using ITM trace.

\textbf{Solution:}

\begin{enumerate}
\def\labelenumi{(\alph{enumi})}
\item
  The MCU supports \textbf{6 simultaneous hardware breakpoints}. Each
  breakpoint halts the processor when execution reaches a specific
  address. These are a limited hardware resource --- unlike software
  breakpoints (which replace an instruction with a breakpoint opcode),
  hardware breakpoints don't modify the code and work on read-only flash
  memory.
\item
  Configure a \textbf{data watchpoint} at address 0x2000\_1A00 with a
  4-byte size, triggering on write access. When any instruction writes
  to this address, the processor halts immediately after the write. The
  debugger then shows:
\end{enumerate}

\begin{itemize}
\tightlist
\item
  The program counter (PC) of the instruction that performed the write
\item
  The call stack showing which function path led to the write
\item
  Register values at the point of the write
\end{itemize}

This directly identifies the offending code without needing to know
which of the 150 functions is responsible.

\begin{enumerate}
\def\labelenumi{(\alph{enumi})}
\setcounter{enumi}{2}
\tightlist
\item
  With only \textbf{4 data watchpoints} available, the developer cannot
  monitor all 5 variables simultaneously. Options:
\end{enumerate}

\begin{itemize}
\tightlist
\item
  Monitor the 4 most likely candidates first, then swap in the 5th
\item
  Use conditional watchpoints to filter specific write patterns
\item
  Combine adjacent variables into a single watchpoint if addresses are
  within a watchpoint's address range
\end{itemize}

\begin{enumerate}
\def\labelenumi{(\alph{enumi})}
\setcounter{enumi}{3}
\tightlist
\item
  \textbf{ITM trace approach:} Instrument the code to output a trace
  message (via ITM stimulus port) whenever the variable is written. The
  ITM provides printf-style debug output at minimal CPU overhead
  (writing a single word to an ITM stimulus register takes 1 cycle). The
  SWO (Serial Wire Output) pin streams the trace data to the debug probe
  at up to 2 Mbit/s, allowing real-time monitoring without halting the
  processor.
\end{enumerate}

\begin{center}\rule{0.5\linewidth}{0.5pt}\end{center}

\section{Problem 5.8.4}\label{problem-5.8.4}

\textbf{Given:} A bootloader receives firmware over UART at 921600 baud
(8N1). The incoming data is framed in packets of 128 bytes payload + 2
bytes CRC-16 + 1 byte sequence number = 131 bytes per packet. The MCU
acknowledges each packet with a 1-byte ACK/NACK. The round-trip latency
(packet send + ACK return + host processing) adds 5 ms per packet.

\textbf{Find:} (a) The raw UART throughput. (b) The effective firmware
transfer rate including protocol overhead and latency. (c) The time to
transfer a 384 KB firmware image. (d) How a sliding window protocol with
window size 4 would improve throughput.

\textbf{Solution:}

\begin{enumerate}
\def\labelenumi{(\alph{enumi})}
\item
  Raw UART throughput at 921600 baud, 8N1 (10 bits/byte): Throughput =
  921,600 / 10 = \textbf{92,160 bytes/s = 90.0 KB/s}
\item
  Per packet: Packet size = 131 bytes at 92,160 B/s -\textgreater{}
  transmit time = 131 / 92,160 = 1.422 ms ACK = 1 byte at 92,160 B/s
  -\textgreater{} 0.011 ms Round-trip overhead = 5 ms Total per packet =
  1.422 + 0.011 + 5.0 = 6.433 ms Payload per packet = 128 bytes
\end{enumerate}

Effective rate = 128 / 0.006433 = \textbf{19,898 bytes/s = 19.4 KB/s}

\begin{enumerate}
\def\labelenumi{(\alph{enumi})}
\setcounter{enumi}{2}
\item
  Packets needed = ceil(384 x 1024 / 128) = ceil(3072) = 3072 packets
  Transfer time = 3072 x 6.433 ms = \textbf{19.76 seconds}
\item
  With sliding window of 4: the host sends 4 packets before waiting for
  the first ACK. Transmit 4 packets = 4 x 1.422 = 5.688 ms Wait for ACK
  of first packet (includes 5 ms round-trip) = 5.0 ms But by the time
  the ACK arrives, 4 packets have been sent, so the pipeline stays full.
\end{enumerate}

Effective time per packet (once pipeline is full) = max(1.422 ms, 5.0/4)
= max(1.422, 1.25) = 1.422 ms Effective rate = 128 / 0.001422 =
\textbf{90,014 bytes/s = 87.9 KB/s}

Transfer time = 3072 x 1.422 ms = \textbf{4.37 seconds}

The sliding window provides a \textbf{4.52x speedup} by hiding the
round-trip latency.

\chapter{Chapter 6 --- Section 6.1: Number
Systems}\label{chapter-6-section-6.1-number-systems}

Practice problems covering binary, hexadecimal, and octal number
conversions.

\begin{center}\rule{0.5\linewidth}{0.5pt}\end{center}

\section{Problem 6.1.1}\label{problem-6.1.1}

\textbf{Given:} The decimal number 4013.

\textbf{Find:} (a) The binary representation. (b) The hexadecimal
representation. (c) Verify by converting the hex result back to decimal.

\textbf{Solution:}

\begin{enumerate}
\def\labelenumi{(\alph{enumi})}
\tightlist
\item
  Successive division by 2: 4013 / 2 = 2006 R 1 2006 / 2 = 1003 R 0 1003
  / 2 = 501 R 1 501 / 2 = 250 R 1 250 / 2 = 125 R 0 125 / 2 = 62 R 1 62
  / 2 = 31 R 0 31 / 2 = 15 R 1 15 / 2 = 7 R 1 7 / 2 = 3 R 1 3 / 2 = 1 R
  1 1 / 2 = 0 R 1
\end{enumerate}

Reading MSB to LSB: \textbf{1111 1010 1101 binary}

\begin{enumerate}
\def\labelenumi{(\alph{enumi})}
\setcounter{enumi}{1}
\item
  Group into 4-bit nibbles from LSB: 1111 1010 1101 F = 1111, A = 1010,
  D = 1101 Hexadecimal: \textbf{0xFAD}
\item
  Verification: F x 16\textsuperscript{2} + A x 16\textsuperscript{1} +
  D x 16\textsuperscript{0} = 15 x 256 + 10 x 16 + 13 = 3840 + 160 + 13
  = \textbf{4013}. Confirmed.
\end{enumerate}

\begin{center}\rule{0.5\linewidth}{0.5pt}\end{center}

\section{Problem 6.1.2}\label{problem-6.1.2}

\textbf{Given:} A 16-bit microcontroller register contains the
hexadecimal value 0xB7E3.

\textbf{Find:} (a) The full binary representation. (b) The decimal
value. (c) The value of bits {[}11:8{]} (upper nibble of the lower
byte). (d) The octal representation.

\textbf{Solution:}

\begin{enumerate}
\def\labelenumi{(\alph{enumi})}
\item
  Convert each hex digit to 4 bits: B = 1011, 7 = 0111, E = 1110, 3 =
  0011 Binary: \textbf{1011 0111 1110 0011}
\item
  Decimal: B x 16\textsuperscript{3} + 7 x 16\textsuperscript{2} + E x
  16\textsuperscript{1} + 3 x 16\textsuperscript{0} = 11 x 4096 + 7 x
  256 + 14 x 16 + 3 = 45,056 + 1,792 + 224 + 3 = \textbf{47,075}
\item
  Bits {[}11:8{]} correspond to the second nibble from the left:
  \textbf{0111 = 7 (decimal)}
\item
  Group binary into 3-bit groups from LSB: 1 011 011 111 100 011 = 1 3 3
  7 4 3 Octal: \textbf{133743}
\end{enumerate}

Verification: 1 x 8\textsuperscript{5} + 3 x 8\textsuperscript{4} + 3 x
8\textsuperscript{3} + 7 x 8\textsuperscript{2} + 4 x 8 + 3 = 32,768 +
12,288 + 1,536 + 448 + 32 + 3 = 47,075. Confirmed.

\begin{center}\rule{0.5\linewidth}{0.5pt}\end{center}

\section{Problem 6.1.3}\label{problem-6.1.3}

\textbf{Given:} Two 8-bit unsigned binary numbers: A = 1100 1010 and B =
0011 1001.

\textbf{Find:} (a) A + B in binary (show carry chain). (b) A AND B. (c)
A OR B. (d) A XOR B. (e) The decimal values of A, B, and A + B.

\textbf{Solution:}

\begin{enumerate}
\def\labelenumi{(\alph{enumi})}
\tightlist
\item
  Binary addition with carry:
\end{enumerate}

\begin{verbatim}
  1100 1010
+ 0011 1001
-----------
\end{verbatim}

Bit 0: 0 + 1 = 1, C = 0 Bit 1: 1 + 0 = 1, C = 0 Bit 2: 0 + 0 = 0, C = 0
Bit 3: 1 + 1 = 0, C = 1 Bit 4: 0 + 1 + 1 = 0, C = 1 Bit 5: 0 + 1 + 1 =
0, C = 1 Bit 6: 1 + 0 + 1 = 0, C = 1 Bit 7: 1 + 0 + 1 = 0, C = 1 (carry
out)

A + B = \textbf{(1) 0000 0011 = 9-bit result 1 0000 0011}

\begin{enumerate}
\def\labelenumi{(\alph{enumi})}
\setcounter{enumi}{1}
\item
  A AND B = 1100 1010 AND 0011 1001 = \textbf{0000 1000}
\item
  A OR B = 1100 1010 OR 0011 1001 = \textbf{1111 1011}
\item
  A XOR B = 1100 1010 XOR 0011 1001 = \textbf{1111 0011}
\item
  A = 128 + 64 + 8 + 2 = \textbf{202} B = 32 + 16 + 8 + 1 = \textbf{57}
  A + B = 202 + 57 = 259 = \textbf{259} (requires 9 bits; carry out
  indicates overflow for 8-bit unsigned)
\end{enumerate}

\begin{center}\rule{0.5\linewidth}{0.5pt}\end{center}

\section{Problem 6.1.4}\label{problem-6.1.4}

\textbf{Given:} A signed 8-bit number uses two's complement
representation. The bit pattern is 1101 0110.

\textbf{Find:} (a) The decimal value. (b) The two's complement of this
number (i.e., the negation). (c) The range of representable values in
8-bit two's complement.

\textbf{Solution:}

\begin{enumerate}
\def\labelenumi{(\alph{enumi})}
\tightlist
\item
  The MSB is 1, so the number is negative. Two's complement to
  magnitude: invert all bits and add 1. Invert: 0010 1001 Add 1: 0010
  1010 = 32 + 8 + 2 = 42
\end{enumerate}

The decimal value is \textbf{-42}.

\begin{enumerate}
\def\labelenumi{(\alph{enumi})}
\setcounter{enumi}{1}
\item
  The two's complement (negation) of -42 is +42. +42 in binary = 0010
  1010 Verification: 0010 1010 = \textbf{+42}
\item
  8-bit two's complement range: Minimum = -2\textsuperscript{7} =
  \textbf{-128} (bit pattern 1000 0000) Maximum = 2\textsuperscript{7} -
  1 = \textbf{+127} (bit pattern 0111 1111)
\end{enumerate}

Range: \textbf{-128 to +127} (256 total values)

\begin{center}\rule{0.5\linewidth}{0.5pt}\end{center}

\section{Problem 6.1.5}\label{problem-6.1.5}

\textbf{Given:} A 32-bit IEEE 754 single-precision floating-point number
has the hex representation 0x41C80000.

\textbf{Find:} (a) The sign, exponent, and mantissa fields. (b) The
decimal value.

\textbf{Solution:}

\begin{enumerate}
\def\labelenumi{(\alph{enumi})}
\tightlist
\item
  Convert to binary: 0x41C80000 = 0100 0001 1100 1000 0000 0000 0000
  0000
\end{enumerate}

Sign bit (bit 31): \textbf{0} (positive) Exponent (bits {[}30:23{]}):
1000 0011 = 131 Mantissa (bits {[}22:0{]}): 100 1000 0000 0000 0000 0000

\begin{enumerate}
\def\labelenumi{(\alph{enumi})}
\setcounter{enumi}{1}
\tightlist
\item
  Biased exponent = 131, bias = 127 Actual exponent = 131 - 127 =
  \textbf{4}
\end{enumerate}

Mantissa with implicit leading 1: 1.1001000\ldots{} = 1 + 0.5 + 0.0625 =
1.5625

Value = (-1)\textsuperscript{0} x 1.5625 x 2\textsuperscript{4} = 1 x
1.5625 x 16 = \textbf{25.0}

Verification: 25.0 = 11001.0 binary = 1.1001 x 2\textsuperscript{4}.
Confirmed.

\begin{center}\rule{0.5\linewidth}{0.5pt}\end{center}

\section{Problem 6.1.6}\label{problem-6.1.6}

\textbf{Given:} A BCD (Binary-Coded Decimal) counter displays the value
947 on three 7-segment displays. Each digit is encoded as a 4-bit BCD
nibble.

\textbf{Find:} (a) The 12-bit BCD representation. (b) The equivalent
binary value. (c) The number of bits saved by using binary instead of
BCD.

\textbf{Solution:}

\begin{enumerate}
\def\labelenumi{(\alph{enumi})}
\item
  Each decimal digit is encoded as 4 bits: 9 = 1001, 4 = 0100, 7 = 0111
  BCD: \textbf{1001 0100 0111} (12 bits)
\item
  947 in binary: successive division by 2: 947 = 512 + 256 + 128 + 32 +
  16 + 2 + 1 = 1110110011 Verification: 512 + 256 + 128 + 32 + 16 + 2 +
  1 = 947
\end{enumerate}

947 in binary: \textbf{11 1011 0011} (10 bits)

\begin{enumerate}
\def\labelenumi{(\alph{enumi})}
\setcounter{enumi}{2}
\tightlist
\item
  BCD requires 12 bits; binary requires 10 bits. Savings = 12 - 10 =
  \textbf{2 bits (16.7\% more efficient)}
\end{enumerate}

For larger numbers the savings grow: a 6-digit BCD number needs 24 bits,
while binary can represent up to 999,999 in
ceil(log\textsubscript{2}(1,000,000)) = 20 bits, saving 4 bits.

\chapter{Chapter 6 --- Section 6.2: Boolean
Algebra}\label{chapter-6-section-6.2-boolean-algebra}

Practice problems covering truth tables and Karnaugh map simplification.

\begin{center}\rule{0.5\linewidth}{0.5pt}\end{center}

\section{Problem 6.2.1}\label{problem-6.2.1}

\textbf{Given:} The Boolean function F(A, B, C) = AB + A'C + BC.

\textbf{Find:} (a) Construct the complete truth table. (b) Express F as
a sum of minterms. (c) Simplify using Boolean algebra theorems.

\textbf{Solution:}

\begin{enumerate}
\def\labelenumi{(\alph{enumi})}
\tightlist
\item
  Truth table (evaluate each term):
\end{enumerate}

{\def\LTcaptype{none} % do not increment counter
\begin{longtable}[]{@{}lllllll@{}}
\toprule\noalign{}
A & B & C & AB & A'C & BC & F = AB + A'C + BC \\
\midrule\noalign{}
\endhead
\bottomrule\noalign{}
\endlastfoot
0 & 0 & 0 & 0 & 0 & 0 & 0 \\
0 & 0 & 1 & 0 & 1 & 0 & 1 \\
0 & 1 & 0 & 0 & 0 & 0 & 0 \\
0 & 1 & 1 & 0 & 1 & 1 & 1 \\
1 & 0 & 0 & 0 & 0 & 0 & 0 \\
1 & 0 & 1 & 0 & 0 & 0 & 0 \\
1 & 1 & 0 & 1 & 0 & 0 & 1 \\
1 & 1 & 1 & 1 & 0 & 1 & 1 \\
\end{longtable}
}

\begin{enumerate}
\def\labelenumi{(\alph{enumi})}
\setcounter{enumi}{1}
\item
  F = 1 for rows 1, 3, 6, 7 (zero-indexed), which are minterms:
  \textbf{F = sum(1, 3, 6, 7)}
\item
  Simplification: F = AB + A'C + BC Apply the consensus theorem: XY +
  X'Z + YZ = XY + X'Z (since YZ is redundant). Here X = A, Y = B, Z = C:
  \textbf{F = AB + A'C}
\end{enumerate}

The term BC is the consensus term and can be eliminated. Verification:
AB + A'C covers all four minterms (AB covers m6, m7; A'C covers m1, m3).

\begin{center}\rule{0.5\linewidth}{0.5pt}\end{center}

\section{Problem 6.2.2}\label{problem-6.2.2}

\textbf{Given:} The Boolean function F(A, B, C, D) = sum of minterms (0,
2, 5, 7, 8, 10, 13, 15).

\textbf{Find:} Simplify using a 4-variable Karnaugh map.

\textbf{Solution:} K-map layout (AB on rows, CD on columns, Gray code
order):

\begin{verbatim}
        CD=00  CD=01  CD=11  CD=10
AB=00 |  1   |  0   |  0   |  1   |
AB=01 |  0   |  1   |  1   |  0   |
AB=11 |  0   |  1   |  1   |  0   |
AB=10 |  1   |  0   |  0   |  1   |
\end{verbatim}

Identify groups: Group 1 (quad): minterms 0, 2, 8, 10 (corners of the
map). These are at CD=00 and CD=10 for AB=00 and AB=10. Common
variables: B' and D'. \textbf{Term: B'D'}

Group 2 (quad): minterms 5, 7, 13, 15. These are at CD=01 and CD=11 for
AB=01 and AB=11. Common variables: B and D. \textbf{Term: BD}

All 8 minterms are covered.

\textbf{F = B'D' + BD}

This can also be written as F = (B XNOR D), meaning the function is true
whenever B and D have the same value.

\begin{center}\rule{0.5\linewidth}{0.5pt}\end{center}

\section{Problem 6.2.3}\label{problem-6.2.3}

\textbf{Given:} F(A, B, C, D) = sum of minterms (1, 3, 4, 5, 9, 11, 12,
13) with don't-care conditions d(6, 14).

\textbf{Find:} Simplify using a K-map, taking advantage of don't-care
terms.

\textbf{Solution:} K-map with 1s for minterms and Xs for don't-cares:

\begin{verbatim}
        CD=00  CD=01  CD=11  CD=10
AB=00 |  0   |  1   |  1   |  0   |
AB=01 |  1   |  1   |  0   |  X   |
AB=11 |  1   |  1   |  0   |  X   |
AB=10 |  0   |  1   |  1   |  0   |
\end{verbatim}

Groups: Group 1 (octet): minterms 1, 3, 5, 9, 11, 13 plus don't-cares
can we find a larger group?

Let's identify systematically: Group 1: columns CD=01 (all rows): cells
1, 5, 13, 9 -\textgreater{} all are 1. Quad with common variable: D and
C'. \textbf{Term: C'D}

Group 2: cells 1, 3, 9, 11 (AB=00 row CD=01,11 and AB=10 row CD=01,11):
common variables B' and D. \textbf{Term: B'D}

Group 3: cells 4, 5, 12, 13 (AB=01 row CD=00,01 and AB=11 row CD=00,01):
common variables C' and B. Wait --- checking: 4=AB01,CD00; 5=AB01,CD01;
12=AB11,CD00; 13=AB11,CD01. Common: B=1, C=0. \textbf{Term: BC'}

Include don't-cares 6 and 14 to enlarge groups: Cell 6 (AB=01, CD=10)
and cell 14 (AB=11, CD=10): with cells 4 and 12 forms a quad: 4, 6, 12,
14 -\textgreater{} common: B and D'. \textbf{Term: BD'}

Now check coverage: C'D covers \{1,5,9,13\}. B'D covers \{1,3,9,11\}.
BD' covers \{4,6,12,14\}. Missing: minterm 3 is covered by B'D. Minterm
5 is covered by C'D. All covered.

But BD' uses both don't-cares. Can we do better?

Simpler solution: C'D covers \{1,5,9,13\}. B'D covers \{1,3,9,11\}. BC'
+ don't-cares: cells 4,5,12,13,6,14 -\textgreater{} BC' covers
\{4,5,12,13\} and with X at 6,14, we can use BD' for \{4,6,12,14\}.

Essential terms: C'D and B'D cover all D=1 minterms. BC' covers
\{4,5,12,13\} (5 and 13 already covered). Actually we just need to cover
4 and 12: these share B=1, C=0, D=0 -\textgreater{} BCD' doesn't
simplify unless we use don't-cares.

Using don't-care 6: group \{4, 6, 12, 14\} = BD'. This covers 4 and 12
with a simpler term.

\textbf{F = B'D + C'D + BD'}

Simplify further: B'D + C'D = D(B' + C') = D(BC)' by De Morgan's. And
BD' is separate.

\textbf{F = B'D + C'D + BD'}

\begin{center}\rule{0.5\linewidth}{0.5pt}\end{center}

\section{Problem 6.2.4}\label{problem-6.2.4}

\textbf{Given:} Prove De Morgan's theorem: (A + B)' = A' * B' using a
truth table.

\textbf{Find:} Construct truth tables for both sides and verify they are
identical.

\textbf{Solution:}

{\def\LTcaptype{none} % do not increment counter
\begin{longtable}[]{@{}lllllll@{}}
\toprule\noalign{}
A & B & A + B & (A + B)' & A' & B' & A' * B' \\
\midrule\noalign{}
\endhead
\bottomrule\noalign{}
\endlastfoot
0 & 0 & 0 & 1 & 1 & 1 & 1 \\
0 & 1 & 1 & 0 & 1 & 0 & 0 \\
1 & 0 & 1 & 0 & 0 & 1 & 0 \\
1 & 1 & 1 & 0 & 0 & 0 & 0 \\
\end{longtable}
}

The columns (A + B)' and A' * B' are identical for all input
combinations: Row 0: 1 = 1 Row 1: 0 = 0 Row 2: 0 = 0 Row 3: 0 = 0

\textbf{De Morgan's theorem (A + B)' = A' * B' is proven.}

This theorem is fundamental to digital design because it allows
converting between OR/AND forms, which is essential for implementing
circuits using only NAND or only NOR gates.

\begin{center}\rule{0.5\linewidth}{0.5pt}\end{center}

\section{Problem 6.2.5}\label{problem-6.2.5}

\textbf{Given:} A Boolean function is specified by the product-of-sums
(POS) expression: F(A, B, C) = (A + B + C) * (A + B' + C) * (A' + B + C)
* (A' + B + C')

\textbf{Find:} (a) The equivalent sum-of-minterms form. (b) Simplify
using a K-map.

\textbf{Solution:}

\begin{enumerate}
\def\labelenumi{(\alph{enumi})}
\tightlist
\item
  A maxterm (sum term) equals 0 for one specific input combination. Each
  maxterm corresponds to the complement of a minterm: (A + B + C) = 0
  when A=0, B=0, C=0 -\textgreater{} maxterm M₀ -\textgreater{} minterm
  m₀ is absent (A + B' + C) = 0 when A=0, B=1, C=0 -\textgreater{}
  maxterm M₂ -\textgreater{} minterm m₂ is absent (A' + B + C) = 0 when
  A=1, B=0, C=0 -\textgreater{} maxterm M₄ -\textgreater{} minterm m₄ is
  absent (A' + B + C') = 0 when A=1, B=0, C=1 -\textgreater{} maxterm M₅
  -\textgreater{} minterm m₅ is absent
\end{enumerate}

The function is 0 for minterms 0, 2, 4, 5; therefore it is 1 for
minterms 1, 3, 6, 7: \textbf{F = sum(1, 3, 6, 7)}

\begin{enumerate}
\def\labelenumi{(\alph{enumi})}
\setcounter{enumi}{1}
\tightlist
\item
  K-map for 3 variables:
\end{enumerate}

\begin{verbatim}
       C=0  C=1
AB=00 | 0 | 1 |
AB=01 | 0 | 1 |
AB=11 | 1 | 1 |
AB=10 | 0 | 0 |
\end{verbatim}

Groups: Group 1: minterms 1, 3 (AB=00,01; C=1) -\textgreater{} A' and C,
but B varies. Common: A'C. Group 2: minterms 6, 7 (AB=11; C=0,1)
-\textgreater{} AB. Group 3: minterms 3, 7 (AB=01,11; C=1)
-\textgreater{} BC.

Essential: A'C covers \{1,3\}. AB covers \{6,7\}. BC is redundant (3
covered by A'C, 7 by AB).

\textbf{F = A'C + AB}

\begin{center}\rule{0.5\linewidth}{0.5pt}\end{center}

\section{Problem 6.2.6}\label{problem-6.2.6}

\textbf{Given:} A 5-variable Boolean function F(A, B, C, D, E) = sum of
minterms (0, 2, 4, 6, 16, 18, 20, 22).

\textbf{Find:} Identify the pattern and write the simplified expression
without a K-map.

\textbf{Solution:} List the minterms in binary (ABCDE): m0 = 00000 m2 =
00010 m4 = 00100 m6 = 00110 m16 = 10000 m18 = 10010 m20 = 10100 m22 =
10110

Pattern analysis: - Bit E (LSB) is always 0 in all minterms - Bit B is
always 0 in all minterms - Bits A, C, D vary freely

E = 0 -\textgreater{} E' is present in all terms B = 0 -\textgreater{}
B' is present in all terms

\textbf{F = B'E'}

Verification: B'E' is true when B = 0 and E = 0. There are
2\textsuperscript{3} = 8 combinations of A, C, D, giving minterms where
ABCDE has B=0, E=0: \{0, 2, 4, 6, 16, 18, 20, 22\}. All 8 minterms
match. Confirmed.

\chapter{Chapter 6 --- Section 6.3: Logic
Gates}\label{chapter-6-section-6.3-logic-gates}

Practice problems covering AND, OR, NOT, NAND, NOR, XOR gates, logic
families, voltage levels, noise margins, and universal gate
implementations.

\begin{center}\rule{0.5\linewidth}{0.5pt}\end{center}

\section{Problem 6.3.1}\label{problem-6.3.1}

\textbf{Given:} A digital system requires the function Y = ABCD (4-input
AND). The only available components are 2-input AND gates with a
propagation delay of t\textsubscript{pd} = 4 ns each.

\textbf{Find:} (a) The minimum number of 2-input AND gates required. (b)
Two different gate arrangements (cascaded chain vs.~balanced tree). (c)
The propagation delay of each arrangement.

\textbf{Solution:}

\begin{enumerate}
\def\labelenumi{(\alph{enumi})}
\tightlist
\item
  A 4-input AND requires three 2-input AND gates (each gate reduces two
  inputs to one; 4 inputs need 3 reductions).
\end{enumerate}

Minimum gates: \textbf{3}

\begin{enumerate}
\def\labelenumi{(\alph{enumi})}
\setcounter{enumi}{1}
\tightlist
\item
  Arrangement 1 --- cascaded chain: Gate 1: G₁ = A * B Gate 2: G₂ = G₁ *
  C = ABC Gate 3: Y = G₂ * D = ABCD Delay = 3 stages x 4 ns = \textbf{12
  ns}
\end{enumerate}

Arrangement 2 --- balanced tree: Gate 1: G₁ = A * B (level 1) Gate 2: G₂
= C * D (level 1, in parallel) Gate 3: Y = G₁ * G₂ = ABCD (level 2)
Delay = 2 levels x 4 ns = \textbf{8 ns}

\begin{enumerate}
\def\labelenumi{(\alph{enumi})}
\setcounter{enumi}{2}
\tightlist
\item
  The balanced tree is faster by 12 - 8 = \textbf{4 ns (33\% reduction)}
  because gates at the same level operate in parallel. The tree
  structure is preferred in high-speed designs to minimize critical path
  depth.
\end{enumerate}

\begin{center}\rule{0.5\linewidth}{0.5pt}\end{center}

\section{Problem 6.3.2}\label{problem-6.3.2}

\textbf{Given:} A 3-input OR gate has inputs A, B, C. The gate is built
from CMOS transistors operating at V\textsubscript{DD} = 3.3 V. The CMOS
implementation uses a NOR gate (3 series PMOS + 3 parallel NMOS)
followed by an inverter. Each NMOS transistor has on-resistance
R\textsubscript{n} = 500 Ω and each PMOS transistor has on-resistance
R\textsubscript{p} = 1000 Ω.

\textbf{Find:} (a) The truth table output for all 8 input combinations.
(b) The number of input combinations that produce a high output. (c) The
worst-case pull-down resistance of the NOR stage. (d) The worst-case
pull-up resistance of the NOR stage.

\textbf{Solution:}

\begin{enumerate}
\def\labelenumi{(\alph{enumi})}
\tightlist
\item
  Truth table for Y = A + B + C:
\end{enumerate}

{\def\LTcaptype{none} % do not increment counter
\begin{longtable}[]{@{}llll@{}}
\toprule\noalign{}
A & B & C & Y \\
\midrule\noalign{}
\endhead
\bottomrule\noalign{}
\endlastfoot
0 & 0 & 0 & 0 \\
0 & 0 & 1 & 1 \\
0 & 1 & 0 & 1 \\
0 & 1 & 1 & 1 \\
1 & 0 & 0 & 1 \\
1 & 0 & 1 & 1 \\
1 & 1 & 0 & 1 \\
1 & 1 & 1 & 1 \\
\end{longtable}
}

\begin{enumerate}
\def\labelenumi{(\alph{enumi})}
\setcounter{enumi}{1}
\item
  Number of high outputs = 2³ - 1 = \textbf{7 out of 8 combinations}
\item
  NOR stage pull-down: the NMOS transistors are in parallel. Worst case
  for pull-down is when only one NMOS is on (one input high):
  R\textsubscript{pd,worst} = R\textsubscript{n} = \textbf{500 Ω}
\end{enumerate}

Best case (all three on): R\textsubscript{pd,best} = R\textsubscript{n}
/ 3 = 500 / 3 = \textbf{167 Ω}

\begin{enumerate}
\def\labelenumi{(\alph{enumi})}
\setcounter{enumi}{3}
\tightlist
\item
  NOR stage pull-up: the PMOS transistors are in series (all three must
  be on, meaning all inputs low): R\textsubscript{pu} = 3 x
  R\textsubscript{p} = 3 x 1000 = \textbf{3000 Ω}
\end{enumerate}

The high pull-up resistance makes the low-to-high transition slower than
the high-to-low transition, a characteristic of multi-input CMOS NOR
gates.

\begin{center}\rule{0.5\linewidth}{0.5pt}\end{center}

\section{Problem 6.3.3}\label{problem-6.3.3}

\textbf{Given:} A CMOS inverter has V\textsubscript{DD} = 5.0 V. The
switching threshold is at V\textsubscript{DD}/2 = 2.5 V. The inverter
drives a load capacitance C\textsubscript{L} = 15 pF. The output drive
current during switching is I\textsubscript{avg} = 2 mA.

\textbf{Find:} (a) The approximate output transition time (0 to
V\textsubscript{DD}). (b) The dynamic power dissipation at a switching
frequency of 50 MHz. (c) The propagation delay if t\textsubscript{pd} is
approximated as half the transition time.

\textbf{Solution:}

\begin{enumerate}
\def\labelenumi{(\alph{enumi})}
\item
  Transition time using I = C dV/dt, so dt = C dV / I:
  t\textsubscript{transition} = C\textsubscript{L} x V\textsubscript{DD}
  / I\textsubscript{avg} = 15 x 10⁻¹² x 5.0 / (2 x 10⁻³)
  t\textsubscript{transition} = 75 x 10⁻¹² / 2 x 10⁻³ = 37.5 x 10⁻⁹ s =
  \textbf{37.5 ns}
\item
  Dynamic power: P\textsubscript{dyn} = C\textsubscript{L} x
  V\textsubscript{DD}² x f = 15 x 10⁻¹² x (5.0)² x 50 x 10⁶
  P\textsubscript{dyn} = 15 x 10⁻¹² x 25 x 50 x 10⁶ = 18,750 x 10⁻⁶ =
  \textbf{18.75 mW}
\item
  Propagation delay: t\textsubscript{pd} ≈ t\textsubscript{transition} /
  2 = 37.5 / 2 = \textbf{18.75 ns}
\end{enumerate}

This is the delay from input crossing the threshold to the output
crossing the threshold.

\begin{center}\rule{0.5\linewidth}{0.5pt}\end{center}

\section{Problem 6.3.4}\label{problem-6.3.4}

\textbf{Given:} An engineer needs to implement the function Y = A * B
using only 2-input NAND gates. The NAND gate propagation delay is
t\textsubscript{pd} = 3.5 ns.

\textbf{Find:} (a) The circuit implementation showing all gates. (b) The
total number of NAND gates required. (c) The total propagation delay
from input to output.

\textbf{Solution:}

\begin{enumerate}
\def\labelenumi{(\alph{enumi})}
\tightlist
\item
  Since Y = A * B = ((A * B)`)' = NAND(NAND(A,B), NAND(A,B)): Gate 1: N₁
  = (A * B)' {[}NAND of A and B{]} Gate 2: Y = (N₁ * N₁)' = (N₁)' = A *
  B {[}NAND used as inverter with both inputs tied to N₁{]}
\end{enumerate}

Alternatively, using the identity: Gate 1: N₁ = NAND(A, B) = (AB)' Gate
2: Y = NAND(N₁, N₁) = (N₁)' = ((AB)`)' = AB

\begin{enumerate}
\def\labelenumi{(\alph{enumi})}
\setcounter{enumi}{1}
\item
  Total NAND gates: \textbf{2}
\item
  Total propagation delay = 2 x t\textsubscript{pd} = 2 x 3.5 =
  \textbf{7.0 ns}
\end{enumerate}

Note: A single NAND gate produces (AB)', so one additional
NAND-as-inverter stage is needed to restore the true AND function. This
is why NAND gates are preferred as the basic building block in CMOS ---
the AND function always costs one extra gate delay.

\begin{center}\rule{0.5\linewidth}{0.5pt}\end{center}

\section{Problem 6.3.5}\label{problem-6.3.5}

\textbf{Given:} A system must implement the XOR function Y = A XOR B
using only 2-input NOR gates. Each NOR gate has a propagation delay of 4
ns.

\textbf{Find:} (a) The Boolean derivation showing how to express XOR
using NOR operations. (b) The circuit implementation with gate count.
(c) The total propagation delay.

\textbf{Solution:}

\begin{enumerate}
\def\labelenumi{(\alph{enumi})}
\tightlist
\item
  Derivation: Y = A XOR B = A'B + AB'
\end{enumerate}

Using NOR gates, first generate complements: A' = NOR(A, A) B' = NOR(B,
B)

Now express A'B + AB': Note that NOR(A', B') = (A' + B')' = AB (by De
Morgan's). That gives AND, not what we need.

Instead, use the identity: A'B + AB' = ((A'B + AB')`)' The inner
expression: (A'B + AB')' = NOR(A'B, AB')

We need to form A'B and AB'. Using NOR: A'B: NOR(A, B') gives (A + B')'.
That is not A'B.

A better approach: Let P = NOR(A, B) = (A + B)' = A'B' Let Q = NOR(A',
B') = (A' + B')' = AB Then Y = A'B + AB' and note: NOR(P, Q) = (P + Q)'
= (A'B' + AB)' = ((A XNOR B))' = A XOR B

So: Gate 1: A' = NOR(A, A) Gate 2: B' = NOR(B, B) Gate 3: P = NOR(A, B)
= A'B' Gate 4: Q = NOR(A', B') = AB Gate 5: Y = NOR(P, Q) = (A'B' + AB)'
= A'B + AB' = A XOR B

\begin{enumerate}
\def\labelenumi{(\alph{enumi})}
\setcounter{enumi}{1}
\item
  Total NOR gates: \textbf{5}
\item
  Critical path: A -\textgreater{} Gate 1 (A') -\textgreater{} Gate 4
  (Q) -\textgreater{} Gate 5 (Y) = 3 gate levels. Also: A
  -\textgreater{} Gate 3 (P) -\textgreater{} Gate 5 (Y) = 2 gate levels.
  The longest path is 3 levels (through the inverter).
\end{enumerate}

Total propagation delay = 3 x 4 = \textbf{12 ns}

\begin{center}\rule{0.5\linewidth}{0.5pt}\end{center}

\section{Problem 6.3.6}\label{problem-6.3.6}

\textbf{Given:} A 4-bit even parity generator uses XOR gates to compute
parity bit P from data bits D₃D₂D₁D₀. Each 2-input XOR gate has a
propagation delay of 6 ns. The data word is D₃D₂D₁D₀ = 1001.

\textbf{Find:} (a) The parity bit P. (b) The complete 5-bit transmitted
word. (c) The propagation delay using a cascaded (chain) structure. (d)
The propagation delay using a balanced tree structure. (e) Verify that
the received word has even parity.

\textbf{Solution:}

\begin{enumerate}
\def\labelenumi{(\alph{enumi})}
\tightlist
\item
  P = D₃ XOR D₂ XOR D₁ XOR D₀
\end{enumerate}

Using a tree structure: Level 1: X₁ = D₃ XOR D₂ = 1 XOR 0 = 1 Level 1:
X₂ = D₁ XOR D₀ = 0 XOR 1 = 1 Level 2: P = X₁ XOR X₂ = 1 XOR 1 = 0

Parity bit: \textbf{P = 0}

\begin{enumerate}
\def\labelenumi{(\alph{enumi})}
\setcounter{enumi}{1}
\item
  Transmitted word = D₃D₂D₁D₀P = \textbf{10010}
\item
  Cascaded chain: 3 XOR gates in series (D₃ XOR D₂, then XOR D₁, then
  XOR D₀). Delay = 3 x 6 = \textbf{18 ns}
\item
  Balanced tree: 2 levels (two XOR gates in parallel at level 1, one at
  level 2). Delay = 2 x 6 = \textbf{12 ns}
\item
  Verification: count the 1s in 10010 -\textgreater{} positions 4 and 1
  have 1s -\textgreater{} two 1s total. Two is even, so \textbf{even
  parity is confirmed}.
\end{enumerate}

\begin{center}\rule{0.5\linewidth}{0.5pt}\end{center}

\section{Problem 6.3.7}\label{problem-6.3.7}

\textbf{Given:} A 74HC04 CMOS hex inverter operates at
V\textsubscript{DD} = 5 V with the following specifications:
V\textsubscript{OH} = 4.9 V, V\textsubscript{OL} = 0.1 V,
V\textsubscript{IH} = 3.5 V, V\textsubscript{IL} = 1.5 V. It drives a
74LVC04 CMOS inverter operating at V\textsubscript{DD} = 3.3 V with
specifications: V\textsubscript{IH} = 2.0 V, V\textsubscript{IL} = 0.8
V, absolute maximum input voltage = 4.3 V.

\textbf{Find:} (a) The noise margins of the 74HC04. (b) Whether the
74HC04 can directly drive the 74LVC04 for logic-low outputs. (c) Whether
the 74HC04 can directly drive the 74LVC04 for logic-high outputs. (d) A
solution if direct driving is not safe.

\textbf{Solution:}

\begin{enumerate}
\def\labelenumi{(\alph{enumi})}
\item
  74HC04 noise margins: NM\textsubscript{H} = V\textsubscript{OH} -
  V\textsubscript{IH} = 4.9 - 3.5 = \textbf{1.4 V} NM\textsubscript{L} =
  V\textsubscript{IL} - V\textsubscript{OL} = 1.5 - 0.1 = \textbf{1.4 V}
\item
  Logic-low: V\textsubscript{OL}(74HC04) = 0.1 V \textless{}
  V\textsubscript{IL}(74LVC04) = 0.8 V. The low output is well within
  the acceptable range. \textbf{Compatible for logic low.}
\item
  Logic-high: V\textsubscript{OH}(74HC04) = 4.9 V \textgreater{}
  V\textsubscript{max}(74LVC04) = 4.3 V. The 5 V output \textbf{exceeds
  the absolute maximum rating} of the 3.3 V device by 0.6 V. This will
  damage the 74LVC04. \textbf{Not safe for direct connection.}
\item
  Solutions:
\end{enumerate}

\begin{enumerate}
\def\labelenumi{\arabic{enumi}.}
\tightlist
\item
  Use a resistive voltage divider to reduce the high output to below 3.3
  V.
\item
  Insert a bidirectional level shifter (e.g., TXS0108E) between the two
  devices.
\item
  Use a series resistor (e.g., 1 kΩ) to limit current through the
  74LVC04 input clamping diodes, provided the data sheet allows this
  approach.
\end{enumerate}

\textbf{A level-shifting buffer is the recommended solution.}

\begin{center}\rule{0.5\linewidth}{0.5pt}\end{center}

\section{Problem 6.3.8}\label{problem-6.3.8}

\textbf{Given:} A 74HC00 CMOS NAND gate operates at V\textsubscript{DD}
= 3.3 V with C\textsubscript{pd} = 24 pF (internal power dissipation
capacitance) and drives an external load of C\textsubscript{L} = 30 pF.
The quiescent supply current is I\textsubscript{CC} = 4 μA. The gate
switches at f = 10 MHz.

\textbf{Find:} (a) The static power dissipation. (b) The dynamic power
dissipation due to output load switching. (c) The dynamic power
dissipation due to internal switching. (d) The total power dissipation.

\textbf{Solution:}

\begin{enumerate}
\def\labelenumi{(\alph{enumi})}
\item
  Static power: P\textsubscript{static} = V\textsubscript{DD} x
  I\textsubscript{CC} = 3.3 x 4 x 10⁻⁶ = \textbf{13.2 μW}
\item
  Dynamic power from external load: P\textsubscript{load} =
  C\textsubscript{L} x V\textsubscript{DD}² x f = 30 x 10⁻¹² x (3.3)² x
  10 x 10⁶ P\textsubscript{load} = 30 x 10⁻¹² x 10.89 x 10 x 10⁶ = 3,267
  x 10⁻⁶ = \textbf{3.267 mW}
\item
  Dynamic power from internal capacitance: P\textsubscript{internal} =
  C\textsubscript{pd} x V\textsubscript{DD}² x f = 24 x 10⁻¹² x 10.89 x
  10 x 10⁶ P\textsubscript{internal} = 2,613.6 x 10⁻⁶ = \textbf{2.614
  mW}
\item
  Total power: P\textsubscript{total} = P\textsubscript{static} +
  P\textsubscript{load} + P\textsubscript{internal} = 0.013 + 3.267 +
  2.614 = \textbf{5.894 mW}
\end{enumerate}

The static power is negligible compared to dynamic power at 10 MHz. At
higher frequencies dynamic power dominates even more, which is why
reducing V\textsubscript{DD} is so effective (power scales as
V\textsubscript{DD}²).

\begin{center}\rule{0.5\linewidth}{0.5pt}\end{center}

\section{Problem 6.3.9}\label{problem-6.3.9}

\textbf{Given:} A CMOS inverter at V\textsubscript{DD} = 1.8 V drives a
fan-out of 8 identical gate inputs. Each gate input has an input
capacitance of C\textsubscript{in} = 5 pF. The driving inverter has a
maximum output current of I\textsubscript{OH} = 4 mA (sourcing) and
I\textsubscript{OL} = 8 mA (sinking). The system operates at f = 100
MHz.

\textbf{Find:} (a) The total load capacitance. (b) The rise time of the
output (10\% to 90\% of V\textsubscript{DD}) assuming constant current
charging. (c) The fall time of the output. (d) The dynamic power
dissipated driving the fan-out load. (e) Whether the fan-out is
acceptable if the maximum allowable rise/fall time is 5 ns.

\textbf{Solution:}

\begin{enumerate}
\def\labelenumi{(\alph{enumi})}
\item
  Total load capacitance: C\textsubscript{total} = 8 x
  C\textsubscript{in} = 8 x 5 = \textbf{40 pF}
\item
  Rise time (charging from 0.1 V\textsubscript{DD} to 0.9
  V\textsubscript{DD}): ΔV = 0.8 x V\textsubscript{DD} = 0.8 x 1.8 =
  1.44 V t\textsubscript{r} = C\textsubscript{total} x ΔV /
  I\textsubscript{OH} = 40 x 10⁻¹² x 1.44 / (4 x 10⁻³)
  t\textsubscript{r} = 57.6 x 10⁻¹² / 4 x 10⁻³ = 14.4 x 10⁻⁹ =
  \textbf{14.4 ns}
\item
  Fall time (discharging from 0.9 V\textsubscript{DD} to 0.1
  V\textsubscript{DD}): t\textsubscript{f} = C\textsubscript{total} x ΔV
  / I\textsubscript{OL} = 40 x 10⁻¹² x 1.44 / (8 x 10⁻³)
  t\textsubscript{f} = 57.6 x 10⁻¹² / 8 x 10⁻³ = 7.2 x 10⁻⁹ =
  \textbf{7.2 ns}
\item
  Dynamic power: P\textsubscript{dyn} = C\textsubscript{total} x
  V\textsubscript{DD}² x f = 40 x 10⁻¹² x (1.8)² x 100 x 10⁶
  P\textsubscript{dyn} = 40 x 10⁻¹² x 3.24 x 10⁸ = \textbf{12.96 mW}
\item
  The rise time of 14.4 ns exceeds the 5 ns limit. The fall time of 7.2
  ns also exceeds the limit. \textbf{The fan-out of 8 is not acceptable}
  at this speed. The maximum fan-out for t\textsubscript{r} \textless= 5
  ns: N\textsubscript{max} = I\textsubscript{OH} x
  t\textsubscript{r,max} / (ΔV x C\textsubscript{in}) = 4 x 10⁻³ x 5 x
  10⁻⁹ / (1.44 x 5 x 10⁻¹²) = 20 x 10⁻¹² / 7.2 x 10⁻¹² = 2.78
\end{enumerate}

\textbf{Maximum fan-out for the rise time constraint is 2 gates.} A
buffer or larger driver is needed for a fan-out of 8.

\begin{center}\rule{0.5\linewidth}{0.5pt}\end{center}

\section{Problem 6.3.10}\label{problem-6.3.10}

\textbf{Given:} A 5 V TTL system (74LS series) must interface with a 3.3
V LVCMOS device. The TTL gate has V\textsubscript{OH} = 2.7 V,
V\textsubscript{OL} = 0.5 V, I\textsubscript{OH} = -0.4 mA,
I\textsubscript{OL} = 8 mA. The LVCMOS device requires
V\textsubscript{IH} = 2.0 V and V\textsubscript{IL} = 0.8 V. Each LVCMOS
input draws I\textsubscript{IH} = 1 μA and I\textsubscript{IL} = -1 μA.

\textbf{Find:} (a) The noise margins for the TTL-to-LVCMOS interface.
(b) The DC fan-out of the TTL gate driving LVCMOS inputs. (c) Whether a
pull-up resistor to 3.3 V is needed, and if so, the appropriate value.
(d) The noise margins with the pull-up in place.

\textbf{Solution:}

\begin{enumerate}
\def\labelenumi{(\alph{enumi})}
\tightlist
\item
  Without pull-up: NM\textsubscript{H} = V\textsubscript{OH}(TTL) -
  V\textsubscript{IH}(LVCMOS) = 2.7 - 2.0 = \textbf{0.7 V}
  NM\textsubscript{L} = V\textsubscript{IL}(LVCMOS) -
  V\textsubscript{OL}(TTL) = 0.8 - 0.5 = \textbf{0.3 V}
\end{enumerate}

Both margins are positive, so the interface technically works, but the
margins are thin.

\begin{enumerate}
\def\labelenumi{(\alph{enumi})}
\setcounter{enumi}{1}
\tightlist
\item
  DC fan-out: For high state: N = \textbar I\textsubscript{OH}\textbar{}
  / I\textsubscript{IH} = 0.4 x 10⁻³ / 1 x 10⁻⁶ = 400 For low state: N =
  I\textsubscript{OL} / \textbar I\textsubscript{IL}\textbar{} = 8 x
  10⁻³ / 1 x 10⁻⁶ = 8000
\end{enumerate}

DC fan-out is limited by the high state: \textbf{400 gates} (effectively
unlimited for practical circuits).

\begin{enumerate}
\def\labelenumi{(\alph{enumi})}
\setcounter{enumi}{2}
\tightlist
\item
  A pull-up resistor to 3.3 V improves the high-state noise margin. When
  the TTL output is high, the pull-up helps raise V\textsubscript{OH}
  closer to 3.3 V. When the TTL output is low, the pull-up resistor must
  not exceed the sink current capability: R\textsubscript{min} =
  V\textsubscript{pullup} / I\textsubscript{OL} = 3.3 / 8 x 10⁻³ = 412.5
  Ω
\end{enumerate}

Use a standard value of \textbf{R\textsubscript{pullup} = 4.7 kΩ} (draws
only 3.3/4700 = 0.70 mA in low state, well within the 8 mA sink
capability).

\textbf{Yes, a pull-up resistor is recommended} to improve the
high-level noise margin.

\begin{enumerate}
\def\labelenumi{(\alph{enumi})}
\setcounter{enumi}{3}
\tightlist
\item
  With the 4.7 kΩ pull-up to 3.3 V, the high output approaches 3.3 V:
  NM\textsubscript{H} = 3.3 - 2.0 = \textbf{1.3 V} NM\textsubscript{L} =
  0.8 - 0.5 = \textbf{0.3 V} (unchanged)
\end{enumerate}

The pull-up nearly doubles the high noise margin from 0.7 V to
\textbf{1.3 V}.

\chapter{Chapter 6 --- Section 6.4: Combinational
Circuits}\label{chapter-6-section-6.4-combinational-circuits}

Practice problems covering adders, multiplexers, decoders, encoders,
carry-lookahead logic, and combinational circuit design.

\begin{center}\rule{0.5\linewidth}{0.5pt}\end{center}

\section{Problem 6.4.1}\label{problem-6.4.1}

\textbf{Given:} A full adder has inputs A = 1, B = 1, and
C\textsubscript{in} = 1.

\textbf{Find:} (a) The sum bit S and carry-out C\textsubscript{out}. (b)
The Boolean expressions for S and C\textsubscript{out} in terms of A, B,
C\textsubscript{in}. (c) The gate count required to implement one full
adder using AND, OR, and XOR gates.

\textbf{Solution:}

\begin{enumerate}
\def\labelenumi{(\alph{enumi})}
\tightlist
\item
  S = A XOR B XOR C\textsubscript{in} = 1 XOR 1 XOR 1 = 0 XOR 1 =
  \textbf{1} C\textsubscript{out} = AB + AC\textsubscript{in} +
  BC\textsubscript{in} = (1)(1) + (1)(1) + (1)(1) = 1 + 1 + 1 =
  \textbf{1}
\end{enumerate}

The result represents 1 + 1 + 1 = 3 in decimal = 11 in binary (S = 1,
C\textsubscript{out} = 1).

\begin{enumerate}
\def\labelenumi{(\alph{enumi})}
\setcounter{enumi}{1}
\tightlist
\item
  Boolean expressions: \textbf{S = A XOR B XOR C\textsubscript{in}}
  \textbf{C\textsubscript{out} = AB + C\textsubscript{in}(A XOR B)}
\end{enumerate}

The second form of C\textsubscript{out} uses the generate (G = AB) and
propagate (P = A XOR B) signals: C\textsubscript{out} = G + P *
C\textsubscript{in}.

\begin{enumerate}
\def\labelenumi{(\alph{enumi})}
\setcounter{enumi}{2}
\tightlist
\item
  Gate count:
\end{enumerate}

\begin{itemize}
\tightlist
\item
  2 XOR gates (for S: A XOR B, then result XOR C\textsubscript{in})
\item
  2 AND gates (for AB and (A XOR B) * C\textsubscript{in})
\item
  1 OR gate (for G + P * C\textsubscript{in})
\end{itemize}

Total: \textbf{2 XOR + 2 AND + 1 OR = 5 gates}

\begin{center}\rule{0.5\linewidth}{0.5pt}\end{center}

\section{Problem 6.4.2}\label{problem-6.4.2}

\textbf{Given:} A 16-bit ripple-carry adder is built from full adders.
Each full adder has a carry propagation delay of t\textsubscript{carry}
= 5 ns and a sum generation delay of t\textsubscript{sum} = 7 ns
(measured from C\textsubscript{in} to S).

\textbf{Find:} (a) The worst-case delay for the carry to propagate from
C\textsubscript{in} of bit 0 to C\textsubscript{out} of bit 15. (b) The
total delay for the MSB sum bit S₁₅. (c) The maximum operating
frequency. (d) How much faster a 4-bit carry-lookahead adder (CLA) block
with t\textsubscript{CLA} = 8 ns would be if four CLA blocks are
cascaded for 16 bits.

\textbf{Solution:}

\begin{enumerate}
\def\labelenumi{(\alph{enumi})}
\item
  Carry ripple delay through 16 stages: t\textsubscript{carry,total} =
  16 x t\textsubscript{carry} = 16 x 5 = \textbf{80 ns}
\item
  MSB sum delay = carry propagation through 15 stages + sum generation
  at stage 15: t\textsubscript{S15} = 15 x t\textsubscript{carry} +
  t\textsubscript{sum} = 15 x 5 + 7 = 75 + 7 = \textbf{82 ns}
\item
  Maximum frequency (limited by worst-case carry-out delay):
  f\textsubscript{max} = 1 / t\textsubscript{carry,total} = 1 / 80 ns =
  \textbf{12.5 MHz}
\item
  With 4-bit CLA blocks, the carry output of each block is available
  after one CLA delay. Four cascaded CLA blocks:
  t\textsubscript{CLA,total} = 4 x t\textsubscript{CLA} = 4 x 8 = 32 ns
  Add t\textsubscript{sum} for the final bit: 32 + 7 = 39 ns
\end{enumerate}

Speed improvement: 82 / 39 = 2.10x Time saved: 82 - 39 = \textbf{43 ns
(2.1x faster)}

\begin{center}\rule{0.5\linewidth}{0.5pt}\end{center}

\section{Problem 6.4.3}\label{problem-6.4.3}

\textbf{Given:} A 4-bit carry-lookahead adder adds A = 1011 and B =
0111. The generate and propagate signals for each bit position are
G\textsubscript{i} = A\textsubscript{i} * B\textsubscript{i} and
P\textsubscript{i} = A\textsubscript{i} XOR B\textsubscript{i}. The
input carry is C₀ = 0.

\textbf{Find:} (a) The generate and propagate signals for each bit. (b)
The carry signals C₁ through C₄ using the CLA equations. (c) The sum
bits S₃S₂S₁S₀. (d) Verify the result in decimal.

\textbf{Solution:}

\begin{enumerate}
\def\labelenumi{(\alph{enumi})}
\item
  Generate and propagate for each bit: Bit 0: G₀ = A₀ * B₀ = 1 * 1 = 1,
  P₀ = A₀ XOR B₀ = 1 XOR 1 = 0 Bit 1: G₁ = A₁ * B₁ = 1 * 1 = 1, P₁ = A₁
  XOR B₁ = 1 XOR 1 = 0 Bit 2: G₂ = A₂ * B₂ = 0 * 1 = 0, P₂ = A₂ XOR B₂ =
  0 XOR 1 = 1 Bit 3: G₃ = A₃ * B₃ = 1 * 0 = 0, P₃ = A₃ XOR B₃ = 1 XOR 0
  = 1
\item
  CLA carry equations: C₁ = G₀ + P₀C₀ = 1 + 0 x 0 = \textbf{1} C₂ = G₁ +
  P₁G₀ + P₁P₀C₀ = 1 + 0 + 0 = \textbf{1} C₃ = G₂ + P₂G₁ + P₂P₁G₀ +
  P₂P₁P₀C₀ = 0 + 1 x 1 + 0 + 0 = \textbf{1} C₄ = G₃ + P₃G₂ + P₃P₂G₁ +
  P₃P₂P₁G₀ + P₃P₂P₁P₀C₀ C₄ = 0 + 0 + 1 x 1 x 1 + 0 + 0 = \textbf{1}
\item
  Sum bits: S\textsubscript{i} = P\textsubscript{i} XOR
  C\textsubscript{i} S₀ = P₀ XOR C₀ = 0 XOR 0 = \textbf{0} S₁ = P₁ XOR
  C₁ = 0 XOR 1 = \textbf{1} S₂ = P₂ XOR C₂ = 1 XOR 1 = \textbf{0} S₃ =
  P₃ XOR C₃ = 1 XOR 1 = \textbf{0}
\end{enumerate}

Result: C₄S₃S₂S₁S₀ = \textbf{1 0010 = 10010 binary}

\begin{enumerate}
\def\labelenumi{(\alph{enumi})}
\setcounter{enumi}{3}
\tightlist
\item
  Decimal verification: A = 1011 = 11, B = 0111 = 7 11 + 7 = \textbf{18}
  = 10010 binary. \textbf{Confirmed.}
\end{enumerate}

\begin{center}\rule{0.5\linewidth}{0.5pt}\end{center}

\section{Problem 6.4.4}\label{problem-6.4.4}

\textbf{Given:} A 4:1 multiplexer with select lines S₁S₀ is used to
implement the Boolean function F(A, B, C) = sum of minterms (0, 2, 5,
7). Variables A and B are connected to S₁ and S₀ respectively. Variable
C is available as a data input.

\textbf{Find:} (a) The data input connections D₀ through D₃ in terms of
C, C', 0, or 1. (b) Verify the implementation produces the correct
output for all 8 input combinations.

\textbf{Solution:}

\begin{enumerate}
\def\labelenumi{(\alph{enumi})}
\tightlist
\item
  Group minterms by the select line values (A, B):
\end{enumerate}

S₁S₀ = AB = 00: minterms 0 (C=0) and 1 (C=1). F is 1 for m₀ only
-\textgreater{} D₀ = \textbf{C'} S₁S₀ = AB = 01: minterms 2 (C=0) and 3
(C=1). F is 1 for m₂ only -\textgreater{} D₁ = \textbf{C'} S₁S₀ = AB =
10: minterms 4 (C=0) and 5 (C=1). F is 1 for m₅ only -\textgreater{} D₂
= \textbf{C} S₁S₀ = AB = 11: minterms 6 (C=0) and 7 (C=1). F is 1 for m₇
only -\textgreater{} D₃ = \textbf{C}

\begin{enumerate}
\def\labelenumi{(\alph{enumi})}
\setcounter{enumi}{1}
\tightlist
\item
  Verification:
\end{enumerate}

{\def\LTcaptype{none} % do not increment counter
\begin{longtable}[]{@{}llllll@{}}
\toprule\noalign{}
A & B & C & S₁S₀ & D selected & F \\
\midrule\noalign{}
\endhead
\bottomrule\noalign{}
\endlastfoot
0 & 0 & 0 & 00 & D₀ = C' = 1 & 1 (m₀) \\
0 & 0 & 1 & 00 & D₀ = C' = 0 & 0 (m₁) \\
0 & 1 & 0 & 01 & D₁ = C' = 1 & 1 (m₂) \\
0 & 1 & 1 & 01 & D₁ = C' = 0 & 0 (m₃) \\
1 & 0 & 0 & 10 & D₂ = C = 0 & 0 (m₄) \\
1 & 0 & 1 & 10 & D₂ = C = 1 & 1 (m₅) \\
1 & 1 & 0 & 11 & D₃ = C = 0 & 0 (m₆) \\
1 & 1 & 1 & 11 & D₃ = C = 1 & 1 (m₇) \\
\end{longtable}
}

F = 1 for minterms \{0, 2, 5, 7\}. \textbf{Confirmed} --- matches the
specification.

\begin{center}\rule{0.5\linewidth}{0.5pt}\end{center}

\section{Problem 6.4.5}\label{problem-6.4.5}

\textbf{Given:} A 3-to-8 decoder with active-low outputs and an
active-low enable (EN') is used to implement two Boolean functions:
F₁(A, B, C) = sum of minterms (1, 3, 5) F₂(A, B, C) = sum of minterms
(0, 2, 6) Inputs A, B, C are connected to the decoder inputs I₂, I₁, I₀
respectively. EN' is tied to ground (always enabled).

\textbf{Find:} (a) Which decoder outputs are needed for F₁. (b) Which
decoder outputs are needed for F₂. (c) The external gate type and
connections for each function. (d) The total gate count (decoder +
external gates).

\textbf{Solution:}

\begin{enumerate}
\def\labelenumi{(\alph{enumi})}
\item
  F₁ uses minterms 1, 3, 5. With active-low outputs, output
  Y\textsubscript{i}' goes low when input equals i. The needed outputs
  are: \textbf{Y₁', Y₃', Y₅'} (active low for minterms 1, 3, 5)
\item
  F₂ uses minterms 0, 2, 6: \textbf{Y₀', Y₂', Y₆'} (active low for
  minterms 0, 2, 6)
\item
  Since the decoder outputs are active-low, each output
  Y\textsubscript{i}' = 0 when minterm i is selected. To OR the
  minterms, use NAND gates on the active-low outputs (because NAND of
  active-low signals implements OR of the active-high equivalents):
\end{enumerate}

F₁ = NAND(Y₁', Y₃', Y₅') = (Y₁' * Y₃' * Y₅')'

When any Y\textsubscript{i}' = 0, the NAND output goes high, which is
the desired OR behavior.

F₁: \textbf{3-input NAND gate} with inputs Y₁', Y₃', Y₅' F₂:
\textbf{3-input NAND gate} with inputs Y₀', Y₂', Y₆'

\begin{enumerate}
\def\labelenumi{(\alph{enumi})}
\setcounter{enumi}{3}
\tightlist
\item
  Gate count:
\end{enumerate}

\begin{itemize}
\tightlist
\item
  1 decoder (3-to-8): typically contains 8 AND gates + 3 inverters
  internally
\item
  2 external NAND gates (one 3-input for F₁, one 3-input for F₂)
\end{itemize}

Total external gates: \textbf{2 NAND gates} plus the decoder

\begin{center}\rule{0.5\linewidth}{0.5pt}\end{center}

\section{Problem 6.4.6}\label{problem-6.4.6}

\textbf{Given:} An 8-to-3 priority encoder has 8 inputs I₇ (highest
priority) through I₀ (lowest priority), a 3-bit output Y₂Y₁Y₀, and a
valid output V. At a given instant, inputs I₆, I₄, I₂, and I₀ are all
asserted (logic high) while I₇, I₅, I₃, and I₁ are deasserted.

\textbf{Find:} (a) The output code Y₂Y₁Y₀ and the valid bit V. (b) The
Boolean expression for the MSB output Y₂ as a function of I₇ through I₄.
(c) If the priority encoder is used as an interrupt controller handling
8 interrupt sources, how many bits are needed to mask (enable/disable)
individual interrupts, and sketch the logic for one masked input.

\textbf{Solution:}

\begin{enumerate}
\def\labelenumi{(\alph{enumi})}
\item
  Active inputs: I₆, I₄, I₂, I₀. The highest priority among these is I₆.
  Output code encodes 6: Y₂Y₁Y₀ = \textbf{110} Valid bit: \textbf{V = 1}
  (at least one input is active)
\item
  Y₂ = 1 when the highest active input is I₄, I₅, I₆, or I₇: Y₂ = I₇ +
  I₇'I₆ + I₇'I₆'I₅ + I₇'I₆'I₅'I₄ Simplified: \textbf{Y₂ = I₇ + I₆ + I₅ +
  I₄} (Y₂ is high if any of the upper 4 inputs is active, since their
  codes all have bit 2 = 1)
\end{enumerate}

Wait --- this overcounts. Y₂ should reflect the highest-priority active
input's code bit 2. For a standard priority encoder: Y₂ = I₇ + I₇'I₆ +
I₇'I₆'I₅ + I₇'I₆'I₅'I₄

This simplifies to: \textbf{Y₂ = I₄ + I₅ + I₆ + I₇} because if any of
I₄-I₇ is active, the encoded output is 4-7 (all have Y₂ = 1), and if the
highest active input is I₃ or below, Y₂ = 0.

\begin{enumerate}
\def\labelenumi{(\alph{enumi})}
\setcounter{enumi}{2}
\tightlist
\item
  An 8-bit mask register is needed, one bit per interrupt source:
  \textbf{8 bits}. For masked input i: I\textsubscript{i,masked} =
  I\textsubscript{i} AND M\textsubscript{i}, where M\textsubscript{i} is
  the mask bit (1 = enabled, 0 = disabled). This requires one
  \textbf{2-input AND gate} per interrupt input, placed between the
  interrupt source and the priority encoder input.
\end{enumerate}

\begin{center}\rule{0.5\linewidth}{0.5pt}\end{center}

\section{Problem 6.4.7}\label{problem-6.4.7}

\textbf{Given:} A BCD-to-7-segment decoder drives a common-cathode
display. The BCD input is 0111 (decimal 7). The seven segments are
labeled a through g (top, upper-right, lower-right, bottom, lower-left,
upper-left, middle).

\textbf{Find:} (a) Which segments should be lit for decimal 7. (b) The
truth table entries for segments a, b, and c for BCD inputs 0 through 9.
(c) The Boolean expression for segment a using a K-map (treating BCD
values 10-15 as don't-cares).

\textbf{Solution:}

\begin{enumerate}
\def\labelenumi{(\alph{enumi})}
\tightlist
\item
  For decimal 7, the active segments are: Segment a (top): \textbf{ON}
  Segment b (upper-right): \textbf{ON} Segment c (lower-right):
  \textbf{ON} Segments d, e, f, g: OFF
\end{enumerate}

Display shows: \textbf{three segments lit (a, b, c)} forming the digit
7.

\begin{enumerate}
\def\labelenumi{(\alph{enumi})}
\setcounter{enumi}{1}
\tightlist
\item
  Truth table for segments a, b, c (1 = lit):
\end{enumerate}

{\def\LTcaptype{none} % do not increment counter
\begin{longtable}[]{@{}lllll@{}}
\toprule\noalign{}
BCD (D₃D₂D₁D₀) & Decimal & a & b & c \\
\midrule\noalign{}
\endhead
\bottomrule\noalign{}
\endlastfoot
0000 & 0 & 1 & 1 & 1 \\
0001 & 1 & 0 & 1 & 1 \\
0010 & 2 & 1 & 1 & 0 \\
0011 & 3 & 1 & 1 & 1 \\
0100 & 4 & 0 & 1 & 1 \\
0101 & 5 & 1 & 0 & 1 \\
0110 & 6 & 1 & 0 & 1 \\
0111 & 7 & 1 & 1 & 1 \\
1000 & 8 & 1 & 1 & 1 \\
1001 & 9 & 1 & 1 & 1 \\
\end{longtable}
}

\begin{enumerate}
\def\labelenumi{(\alph{enumi})}
\setcounter{enumi}{2}
\tightlist
\item
  K-map for segment a (D₃D₂ on rows, D₁D₀ on columns), with don't-cares
  for 10-15:
\end{enumerate}

\begin{verbatim}
           D₁D₀=00  D₁D₀=01  D₁D₀=11  D₁D₀=10
D₃D₂=00 |   1    |   0    |   1    |   1    |
D₃D₂=01 |   0    |   1    |   1    |   1    |
D₃D₂=11 |   X    |   X    |   X    |   X    |
D₃D₂=10 |   1    |   1    |   X    |   X    |
\end{verbatim}

Groups: - Quad: cells (0,2,8,10) using don't-cares -\textgreater{} D₁' *
D₀' \ldots{} checking: these are at D₁D₀=00 and D₁D₀=10 for D₃D₂=00 and
D₃D₂=10. Common: D₂' and D₀'. \textbf{Term: D₂'D₀'}. Wait, cell 4
(D₃D₂=01, D₁D₀=00) = 0, so we can't group 0 and 4.

Let me redo: minterms where a=1: \{0, 2, 3, 5, 6, 7, 8, 9\},
don't-cares: \{10,11,12,13,14,15\}.

Group 1: \{0, 2, 8, 10\} (D₃D₂=00/10, D₁D₀=00/10) -\textgreater{}
common: D₂'D₀'. Includes don't-care 10. \textbf{D₂'D₀'} Group 2: \{2, 3,
6, 7\} (D₃D₂=00/01, D₁D₀=11/10) -\textgreater{} common: D₁. \textbf{D₁}
Group 3: \{5, 7, 13, 15\} -\textgreater{} common: D₂D₀. Using
don't-cares. \textbf{D₂D₀} Group 4: \{8, 9, 10, 11\} -\textgreater{} D₃
covers all. \textbf{D₃}

Check coverage: D₂'D₀' covers \{0, 8, 2\emph{, 10}\} --- wait, let me be
more careful.

Minterm 0 = 0000: D₂'D₀' covers it (D₂=0, D₀=0). Yes. Minterm 5 = 0101:
D₂D₀ covers it (D₂=1, D₀=1). Yes. Minterm 3 = 0011: D₁ covers it (D₁=1).
Yes. All minterms covered.

\textbf{a = D₃ + D₁ + D₂'D₀' + D₂D₀}

This can also be written as: \textbf{a = D₃ + D₁ + (D₂ XNOR D₀)}

\begin{center}\rule{0.5\linewidth}{0.5pt}\end{center}

\section{Problem 6.4.8}\label{problem-6.4.8}

\textbf{Given:} A 4-bit binary subtractor is built using full adders and
the two's complement method. The minuend is A = 1100 (12 decimal) and
the subtrahend is B = 0101 (5 decimal). The subtractor computes A - B by
adding A to the two's complement of B.

\textbf{Find:} (a) The one's complement of B. (b) The two's complement
of B using the adder (invert B and set C\textsubscript{in} = 1). (c) The
result of A + B' + 1 showing the carry chain. (d) The decimal result and
the overflow status.

\textbf{Solution:}

\begin{enumerate}
\def\labelenumi{(\alph{enumi})}
\item
  One's complement of B: invert each bit. B = 0101, so B' =
  \textbf{1010}
\item
  Two's complement is formed by the adder: A + B' with
  C\textsubscript{in} = 1. This computes A + B' + 1 = A + (-B) = A - B.
\item
  Addition with carry chain:
\end{enumerate}

\begin{verbatim}
  1100   (A = 12)
+ 1010   (B' = one's complement of 5)
+    1   (C_in = 1)
------
\end{verbatim}

Bit 0: 0 + 0 + 1 = 1, C₁ = 0 Bit 1: 0 + 1 + 0 = 1, C₂ = 0 Bit 2: 1 + 0 +
0 = 1, C₃ = 0 Bit 3: 1 + 1 + 0 = 0, C₄ = 1

Result: C₄ = 1, Sum = \textbf{0111}

\begin{enumerate}
\def\labelenumi{(\alph{enumi})}
\setcounter{enumi}{3}
\tightlist
\item
  In two's complement subtraction, C\textsubscript{out} = 1 indicates no
  borrow (result is positive). The 4-bit result is 0111 = \textbf{7
  decimal}. A - B = 12 - 5 = 7. \textbf{Confirmed.}
\end{enumerate}

Overflow check: C₃ XOR C₄ = 0 XOR 1 = 1? No --- C₃ = 0 (carry into bit
3) and C₄ = 1 (carry out of bit 3). Overflow = C₃ XOR C₄ = 0 XOR 1 = 1.
But this indicates signed overflow.

For unsigned subtraction: C\textsubscript{out} = 1 means no borrow. For
signed: both operands are positive and the result (7) is positive and
fits in 4 bits (range -8 to +7). Since the MSB of the result is 0, there
is \textbf{no signed overflow}. The XOR check applies to the carries
into and out of the MSB: C₃ (carry into bit 3) = 0, C₄ (carry out of bit
3) = 1, so C₃ XOR C₄ = 1 would suggest overflow, but we must
recalculate.

Rechecking carries: Bit 2 gives C₃ = 0 (correct). Bit 3: 1 + 1 + 0 = 10,
C₄ = 1. Overflow = C₃ XOR C₄ = 0 XOR 1 = 1. However, the result 0111 =
+7 is the correct answer for 12 - 5 in unsigned. The overflow flag is
meaningful only for signed interpretation. In 4-bit signed, A = 1100 =
-4 (not +12), so signed A - B = -4 - 5 = -9, which overflows the range
{[}-8, +7{]}. \textbf{Unsigned result is correct (7); signed
interpretation would overflow.}

\begin{center}\rule{0.5\linewidth}{0.5pt}\end{center}

\section{Problem 6.4.9}\label{problem-6.4.9}

\textbf{Given:} A combinational circuit must compare two 4-bit unsigned
numbers A = A₃A₂A₁A₀ and B = B₃B₂B₁B₀ and produce three outputs: G (A
\textgreater{} B), E (A = B), and L (A \textless{} B). The comparison is
done starting from the MSB.

\textbf{Find:} (a) The Boolean expression for the equality output E. (b)
The expression for G (A greater than B) for a 1-bit comparator. (c) How
1-bit comparator stages are cascaded for the full 4-bit comparison. (d)
Compute G, E, L for A = 1010 and B = 1001.

\textbf{Solution:}

\begin{enumerate}
\def\labelenumi{(\alph{enumi})}
\tightlist
\item
  Equality requires all bit pairs to match: E\textsubscript{i} =
  A\textsubscript{i} XNOR B\textsubscript{i} = (A\textsubscript{i} XOR
  B\textsubscript{i})' for each bit position.
\end{enumerate}

\textbf{E = E₃ * E₂ * E₁ * E₀ = (A₃ XNOR B₃)(A₂ XNOR B₂)(A₁ XNOR B₁)(A₀
XNOR B₀)}

\begin{enumerate}
\def\labelenumi{(\alph{enumi})}
\setcounter{enumi}{1}
\tightlist
\item
  For a single bit position i, A\textsubscript{i} \textgreater{}
  B\textsubscript{i} when A\textsubscript{i} = 1 and B\textsubscript{i}
  = 0: \textbf{G\textsubscript{i} = A\textsubscript{i} *
  B\textsubscript{i}'}
\end{enumerate}

Similarly, L\textsubscript{i} = A\textsubscript{i}' *
B\textsubscript{i}.

\begin{enumerate}
\def\labelenumi{(\alph{enumi})}
\setcounter{enumi}{2}
\tightlist
\item
  Cascading from MSB to LSB: G = G₃ + E₃G₂ + E₃E₂G₁ + E₃E₂E₁G₀
\end{enumerate}

In words: A \textgreater{} B if A₃ \textgreater{} B₃, or if A₃ = B₃ and
A₂ \textgreater{} B₂, etc. L = L₃ + E₃L₂ + E₃E₂L₁ + E₃E₂E₁L₀

\begin{enumerate}
\def\labelenumi{(\alph{enumi})}
\setcounter{enumi}{3}
\tightlist
\item
  For A = 1010 and B = 1001: Bit 3: A₃ = 1, B₃ = 1 -\textgreater{} G₃ =
  0, E₃ = 1, L₃ = 0 Bit 2: A₂ = 0, B₂ = 0 -\textgreater{} G₂ = 0, E₂ =
  1, L₂ = 0 Bit 1: A₁ = 1, B₁ = 0 -\textgreater{} G₁ = 1, E₁ = 0, L₁ = 0
  Bit 0: A₀ = 0, B₀ = 1 -\textgreater{} G₀ = 0, E₀ = 0, L₀ = 1
\end{enumerate}

G = 0 + 1 x 0 + 1 x 1 x 1 + 1 x 1 x 0 x 0 = 0 + 0 + 1 + 0 = \textbf{1} E
= 1 x 1 x 0 x 0 = \textbf{0} L = 0 + 1 x 0 + 1 x 1 x 0 + 1 x 1 x 0 x 1 =
\textbf{0}

Result: \textbf{G = 1, E = 0, L = 0} (A \textgreater{} B). Verification:
A = 1010 = 10, B = 1001 = 9, and 10 \textgreater{} 9.
\textbf{Confirmed.}

\begin{center}\rule{0.5\linewidth}{0.5pt}\end{center}

\section{Problem 6.4.10}\label{problem-6.4.10}

\textbf{Given:} A combinational lock circuit accepts a 4-bit BCD input
representing digits 0-9. The lock opens (output UNLOCK = 1) when the
correct 3-digit code 4-7-2 is entered sequentially. However, the
sequencing is handled externally --- this problem concerns only the
single-digit combinational comparator. The circuit compares the 4-bit
BCD input D₃D₂D₁D₀ against a fixed 4-bit key value K₃K₂K₁K₀.

\textbf{Find:} (a) The Boolean expression for the MATCH output. (b)
Implement the MATCH function for key digit 7 (K = 0111) using the
minimum number of gates. (c) The gate count and gate types. (d) How to
extend this to a 3-digit comparator with separate key registers.

\textbf{Solution:}

\begin{enumerate}
\def\labelenumi{(\alph{enumi})}
\item
  MATCH is true when all 4 bits of D equal the corresponding bits of K:
  \textbf{MATCH = (D₃ XNOR K₃)(D₂ XNOR K₂)(D₁ XNOR K₁)(D₀ XNOR K₀)}
\item
  For K = 0111 (key digit 7): K₃ = 0: D₃ XNOR 0 = D₃' (match when D₃ =
  0) K₂ = 1: D₂ XNOR 1 = D₂ (match when D₂ = 1) K₁ = 1: D₁ XNOR 1 = D₁
  (match when D₁ = 1) K₀ = 1: D₀ XNOR 1 = D₀ (match when D₀ = 1)
\end{enumerate}

MATCH = D₃' * D₂ * D₁ * D₀

Implementation: Gate 1: Inverter on D₃ -\textgreater{} D₃' Gate 2:
4-input AND gate on D₃', D₂, D₁, D₀ Or with 2-input gates: Gate 1: NOT
D₃ -\textgreater{} D₃' Gate 2: AND(D₃', D₂) -\textgreater{} W₁ Gate 3:
AND(D₁, D₀) -\textgreater{} W₂ Gate 4: AND(W₁, W₂) -\textgreater{} MATCH

\begin{enumerate}
\def\labelenumi{(\alph{enumi})}
\setcounter{enumi}{2}
\item
  Using 2-input gates: \textbf{1 NOT gate + 3 AND gates = 4 gates total}
  Using a 4-input AND gate with one inverter: \textbf{1 NOT + 1 AND = 2
  gates total}
\item
  For a 3-digit comparator:
\end{enumerate}

\begin{itemize}
\tightlist
\item
  Three separate 4-bit comparators, one per digit position
\item
  Each comparator has its own 4-bit key register (programmable or
  hardwired)
\item
  MATCH₁ compares digit 1 input against K\textsuperscript{(1)} = 0100
  (digit 4)
\item
  MATCH₂ compares digit 2 input against K\textsuperscript{(2)} = 0111
  (digit 7)
\item
  MATCH₃ compares digit 3 input against K\textsuperscript{(3)} = 0010
  (digit 2)
\item
  \textbf{UNLOCK = MATCH₁ AND MATCH₂ AND MATCH₃} (all three digits must
  match)
\end{itemize}

Total: 3 comparators + 1 AND gate for the final UNLOCK output.

\chapter{Chapter 6 --- Section 6.5: Sequential
Circuits}\label{chapter-6-section-6.5-sequential-circuits}

Practice problems covering latches, flip-flops, counters, shift
registers, and state machines.

\begin{center}\rule{0.5\linewidth}{0.5pt}\end{center}

\section{Problem 6.5.1}\label{problem-6.5.1}

\textbf{Given:} An SR latch is built from two cross-coupled NOR gates.
The current state is Q = 0, Q' = 1. The following input sequence is
applied:

{\def\LTcaptype{none} % do not increment counter
\begin{longtable}[]{@{}lll@{}}
\toprule\noalign{}
Time & S & R \\
\midrule\noalign{}
\endhead
\bottomrule\noalign{}
\endlastfoot
t₁ & 1 & 0 \\
t₂ & 0 & 0 \\
t₃ & 0 & 1 \\
t₄ & 0 & 0 \\
t₅ & 1 & 1 \\
\end{longtable}
}

\textbf{Find:} (a) The output Q after each time step. (b) Why the S = R
= 1 condition is forbidden. (c) What happens to Q and Q' during the
forbidden state if both S and R return to 0 simultaneously.

\textbf{Solution:}

\begin{enumerate}
\def\labelenumi{(\alph{enumi})}
\tightlist
\item
  NOR-based SR latch: S = 1 sets Q to 1; R = 1 resets Q to 0; S = R = 0
  holds the previous state.
\end{enumerate}

t₁: S = 1, R = 0 -\textgreater{} SET. \textbf{Q = 1}, Q' = 0. t₂: S = 0,
R = 0 -\textgreater{} HOLD. \textbf{Q = 1} (retains set value). t₃: S =
0, R = 1 -\textgreater{} RESET. \textbf{Q = 0}, Q' = 1. t₄: S = 0, R = 0
-\textgreater{} HOLD. \textbf{Q = 0} (retains reset value). t₅: S = 1, R
= 1 -\textgreater{} FORBIDDEN. Both NOR gates output 0: \textbf{Q = 0,
Q' = 0}.

\begin{enumerate}
\def\labelenumi{(\alph{enumi})}
\setcounter{enumi}{1}
\item
  The S = R = 1 condition is forbidden because it forces both Q = 0 and
  Q' = 0, violating the fundamental requirement that Q and Q' be
  complementary. The latch is no longer in a valid bistable state.
\item
  When both S and R return to 0 simultaneously, both NOR gates see
  inputs (0, 0). The output depends on which gate's propagation delay is
  slightly shorter --- a race condition. The latch will settle into
  either Q = 1 or Q = 0, but the outcome is \textbf{unpredictable
  (metastable)}. This is why the forbidden state must be avoided in
  reliable designs.
\end{enumerate}

\begin{center}\rule{0.5\linewidth}{0.5pt}\end{center}

\section{Problem 6.5.2}\label{problem-6.5.2}

\textbf{Given:} A gated D latch (D latch with enable) is used in an
8-bit register. The enable signal goes high for 12 ns. During this
window, the data input transitions from 0 to 1 at 4 ns after enable goes
high, and back to 0 at 9 ns after enable goes high. The latch
propagation delay is t\textsubscript{pd} = 2 ns.

\textbf{Find:} (a) The output Q waveform with timestamps (assume Q = 0
initially). (b) The value stored when enable returns low at t = 12 ns.
(c) Why this transparency behavior is problematic in multi-stage
pipeline designs.

\textbf{Solution:}

\begin{enumerate}
\def\labelenumi{(\alph{enumi})}
\tightlist
\item
  Timeline (t = 0 is when enable goes high):
\end{enumerate}

t = 0 ns: Enable goes high. D = 0. Q remains \textbf{0} (transparent,
follows D). t = 4 ns: D transitions to 1. After t\textsubscript{pd} = 2
ns: t = 6 ns: Q transitions to \textbf{1}. t = 9 ns: D transitions to 0.
After t\textsubscript{pd} = 2 ns: t = 11 ns: Q transitions to
\textbf{0}. t = 12 ns: Enable goes low. D = 0 at this instant. Q latches
at \textbf{0}.

\begin{enumerate}
\def\labelenumi{(\alph{enumi})}
\setcounter{enumi}{1}
\item
  The stored value when enable goes low: \textbf{Q = 0} (the latch
  captured D = 0 at the falling edge of enable).
\item
  Transparency is problematic in pipelines because while the enable is
  high, any input change propagates through the latch to the output. In
  a multi-stage pipeline, data from stage N could ripple through to
  stage N+1 and even N+2 during the same clock phase, corrupting
  pipeline timing. This is why \textbf{edge-triggered flip-flops} are
  preferred for synchronous designs --- they sample the input only at
  the clock edge, preventing transparency-related timing violations.
\end{enumerate}

\begin{center}\rule{0.5\linewidth}{0.5pt}\end{center}

\section{Problem 6.5.3}\label{problem-6.5.3}

\textbf{Given:} A positive-edge-triggered D flip-flop has: - Setup time:
t\textsubscript{su} = 3 ns - Hold time: t\textsubscript{h} = 1.5 ns -
Clock-to-Q delay: t\textsubscript{cq} = 2.5 ns

The clock period is T\textsubscript{clk} = 8 ns and the clock frequency
is 125 MHz. The flip-flop drives combinational logic that feeds another
identical flip-flop.

\textbf{Find:} (a) The maximum combinational logic delay between the two
flip-flops. (b) The minimum combinational logic delay to satisfy the
hold time constraint. (c) The slack (timing margin) if the actual logic
delay is 3.8 ns. (d) Whether the design meets timing at 150 MHz.

\textbf{Solution:}

\begin{enumerate}
\def\labelenumi{(\alph{enumi})}
\item
  Setup time constraint: t\textsubscript{cq} +
  t\textsubscript{logic,max} + t\textsubscript{su} \textless=
  T\textsubscript{clk} t\textsubscript{logic,max} = T\textsubscript{clk}
  - t\textsubscript{cq} - t\textsubscript{su} = 8 - 2.5 - 3 =
  \textbf{2.5 ns}
\item
  Hold time constraint: t\textsubscript{cq} + t\textsubscript{logic,min}
  \textgreater= t\textsubscript{h} t\textsubscript{logic,min} =
  t\textsubscript{h} - t\textsubscript{cq} = 1.5 - 2.5 = -1.0 ns
\end{enumerate}

Since this is negative, the constraint is automatically satisfied even
with zero logic delay: \textbf{t\textsubscript{logic,min} = 0 ns} (hold
time is met by t\textsubscript{cq} alone).

\begin{enumerate}
\def\labelenumi{(\alph{enumi})}
\setcounter{enumi}{2}
\tightlist
\item
  With t\textsubscript{logic} = 3.8 ns: Total path delay =
  t\textsubscript{cq} + t\textsubscript{logic} + t\textsubscript{su} =
  2.5 + 3.8 + 3.0 = 9.3 ns Slack = T\textsubscript{clk} - total = 8.0 -
  9.3 = \textbf{-1.3 ns (negative slack)}
\end{enumerate}

\textbf{The design fails timing} with 3.8 ns of logic delay at 125 MHz.
The logic must be reduced to 2.5 ns or less, or the clock frequency must
be lowered.

\begin{enumerate}
\def\labelenumi{(\alph{enumi})}
\setcounter{enumi}{3}
\tightlist
\item
  At 150 MHz: T\textsubscript{clk} = 1 / 150 x 10⁶ = 6.667 ns
  t\textsubscript{logic,max} = 6.667 - 2.5 - 3.0 = \textbf{1.167 ns}
\end{enumerate}

At 150 MHz, only 1.167 ns of combinational logic is allowed, which is
extremely tight. \textbf{The design does not meet timing at 150 MHz}
with the actual logic delay of 3.8 ns (slack = 6.667 - 9.3 = -2.633 ns).

\begin{center}\rule{0.5\linewidth}{0.5pt}\end{center}

\section{Problem 6.5.4}\label{problem-6.5.4}

\textbf{Given:} A JK flip-flop has inputs J and K and a clock input. The
characteristic equation is Q\textsuperscript{+} = JQ' + K'Q. The initial
state is Q = 0. The following input sequence is applied over 6 clock
cycles:

{\def\LTcaptype{none} % do not increment counter
\begin{longtable}[]{@{}lll@{}}
\toprule\noalign{}
Clock & J & K \\
\midrule\noalign{}
\endhead
\bottomrule\noalign{}
\endlastfoot
1 & 1 & 0 \\
2 & 1 & 1 \\
3 & 0 & 0 \\
4 & 0 & 1 \\
5 & 1 & 1 \\
6 & 1 & 0 \\
\end{longtable}
}

\textbf{Find:} (a) The output Q after each clock edge. (b) Which clock
cycles correspond to Set, Reset, Toggle, and Hold operations. (c) The
advantages of JK flip-flops over D flip-flops.

\textbf{Solution:}

\begin{enumerate}
\def\labelenumi{(\alph{enumi})}
\tightlist
\item
  Apply Q\textsuperscript{+} = JQ' + K'Q at each rising edge:
\end{enumerate}

Clock 1: J=1, K=0. Q\textsuperscript{+} = 1 x 1 + 1 x 0 = 1. \textbf{Q =
1} (SET) Clock 2: J=1, K=1. Q\textsuperscript{+} = 1 x 0 + 0 x 1 = 0.
\textbf{Q = 0} (TOGGLE) Clock 3: J=0, K=0. Q\textsuperscript{+} = 0 x 1
+ 1 x 0 = 0. \textbf{Q = 0} (HOLD) Clock 4: J=0, K=1.
Q\textsuperscript{+} = 0 x 1 + 0 x 0 = 0. \textbf{Q = 0} (RESET, already
0) Clock 5: J=1, K=1. Q\textsuperscript{+} = 1 x 1 + 0 x 0 = 1.
\textbf{Q = 1} (TOGGLE) Clock 6: J=1, K=0. Q\textsuperscript{+} = 1 x 0
+ 1 x 1 = 1. \textbf{Q = 1} (SET, already 1)

\begin{enumerate}
\def\labelenumi{(\alph{enumi})}
\setcounter{enumi}{1}
\tightlist
\item
  Operation summary:
\end{enumerate}

\begin{itemize}
\tightlist
\item
  \textbf{Set} (J=1, K=0): clocks 1 and 6
\item
  \textbf{Toggle} (J=1, K=1): clocks 2 and 5
\item
  \textbf{Hold} (J=0, K=0): clock 3
\item
  \textbf{Reset} (J=0, K=1): clock 4
\end{itemize}

\begin{enumerate}
\def\labelenumi{(\alph{enumi})}
\setcounter{enumi}{2}
\tightlist
\item
  Advantages of JK over D: The JK flip-flop can toggle without external
  feedback (D flip-flops require connecting Q' back to D). The JK
  eliminates the forbidden state of the SR latch. However, D flip-flops
  are simpler (one input vs.~two) and are more common in modern
  synchronous designs because synthesis tools optimize for D flip-flops.
\end{enumerate}

\begin{center}\rule{0.5\linewidth}{0.5pt}\end{center}

\section{Problem 6.5.5}\label{problem-6.5.5}

\textbf{Given:} A synchronous 4-bit binary up-counter is clocked at
f\textsubscript{clk} = 20 MHz. It counts from 0000 to 1111 and then
wraps around to 0000.

\textbf{Find:} (a) The total number of states. (b) The output frequency
at each bit (Q₀, Q₁, Q₂, Q₃). (c) The time for one complete count cycle.
(d) The counter state after 25 clock pulses starting from 0000. (e) The
excitation equations for T flip-flop implementation.

\textbf{Solution:}

\begin{enumerate}
\def\labelenumi{(\alph{enumi})}
\item
  A 4-bit counter has 2⁴ = \textbf{16 states} (0 through 15).
\item
  Each bit toggles at half the frequency of the previous bit: Q₀ (LSB):
  f₀ = f\textsubscript{clk} / 2 = 20 / 2 = \textbf{10 MHz} Q₁: f₁ =
  f\textsubscript{clk} / 4 = 20 / 4 = \textbf{5 MHz} Q₂: f₂ =
  f\textsubscript{clk} / 8 = 20 / 8 = \textbf{2.5 MHz} Q₃ (MSB): f₃ =
  f\textsubscript{clk} / 16 = 20 / 16 = \textbf{1.25 MHz}
\item
  One complete cycle = 16 clock pulses: T\textsubscript{cycle} = 16 /
  f\textsubscript{clk} = 16 / (20 x 10⁶) = 800 x 10⁻⁹ = \textbf{800 ns =
  0.8 μs}
\item
  After 25 clock pulses: 25 mod 16 = 9. State = 9 = \textbf{1001 binary}
\item
  T flip-flop excitation (T\textsubscript{i} = 1 means flip-flop i
  toggles): T₀ = 1 (Q₀ toggles every clock) T₁ = Q₀ (Q₁ toggles when Q₀
  = 1) T₂ = Q₀ * Q₁ (Q₂ toggles when Q₀ = Q₁ = 1) T₃ = Q₀ * Q₁ * Q₂ (Q₃
  toggles when Q₀ = Q₁ = Q₂ = 1)
\end{enumerate}

General: \textbf{T\textsubscript{i} = Q₀ * Q₁ * \ldots{} *
Q\textsubscript{i-1}} for i \textgreater{} 0, and \textbf{T₀ = 1}.

\begin{center}\rule{0.5\linewidth}{0.5pt}\end{center}

\section{Problem 6.5.6}\label{problem-6.5.6}

\textbf{Given:} A modulo-10 (decade) counter is needed for a digital
clock's seconds-ones digit. The counter uses 4 D flip-flops (Q₃Q₂Q₁Q₀)
and must count from 0000 to 1001, then reset to 0000. The clock
frequency is 1 Hz (one pulse per second).

\textbf{Find:} (a) The count sequence. (b) The reset detection logic.
(c) The output frequency of Q₃ (MSB). (d) The output frequency that
would drive the seconds-tens counter. (e) The total number of decade
counters needed to display hours, minutes, and seconds on a 24-hour
clock.

\textbf{Solution:}

\begin{enumerate}
\def\labelenumi{(\alph{enumi})}
\item
  Count sequence: \textbf{0, 1, 2, 3, 4, 5, 6, 7, 8, 9, 0, 1, \ldots{}}
  In binary: 0000, 0001, 0010, 0011, 0100, 0101, 0110, 0111, 1000, 1001,
  (reset) 0000.
\item
  Detect count 10 = 1010 and reset. However, in a well-designed
  synchronous counter, we detect when the next state would be 10 and
  instead load 0.
\end{enumerate}

For an asynchronous-reset approach: detect the state after 9, which
would momentarily be 1010. Reset = Q₃ AND Q₁ (since 1010 has Q₃ = 1 and
Q₁ = 1, and this combination does not occur in any valid state 0-9).

\textbf{Reset = Q₃ * Q₁}

\begin{enumerate}
\def\labelenumi{(\alph{enumi})}
\setcounter{enumi}{2}
\item
  Q₃ goes high at count 8 and returns low at count 0 (after 9). It is
  high for 2 clock periods out of 10. f\textsubscript{Q3} is not a clean
  square wave, but the frequency of its rising edge is: f =
  f\textsubscript{clk} / 10 = 1 / 10 = \textbf{0.1 Hz}
\item
  The carry/ripple output to the tens counter triggers once every 10
  counts: f\textsubscript{carry} = 1 Hz / 10 = \textbf{0.1 Hz} (one
  pulse every 10 seconds)
\item
  A 24-hour clock displays HH:MM:SS:
\end{enumerate}

\begin{itemize}
\tightlist
\item
  Seconds ones: \textbf{1 decade counter} (mod-10, 0-9)
\item
  Seconds tens: \textbf{1 mod-6 counter} (0-5)
\item
  Minutes ones: \textbf{1 decade counter} (mod-10, 0-9)
\item
  Minutes tens: \textbf{1 mod-6 counter} (0-5)
\item
  Hours ones: \textbf{1 decade counter} (mod-10, 0-9, reset at 4 when
  tens = 2)
\item
  Hours tens: \textbf{1 mod-3 counter} (0-2)
\end{itemize}

Total: \textbf{3 decade counters + 2 mod-6 counters + 1 mod-3 counter =
6 counters total} (If counting only decade counters: \textbf{3 decade
counters})

\begin{center}\rule{0.5\linewidth}{0.5pt}\end{center}

\section{Problem 6.5.7}\label{problem-6.5.7}

\textbf{Given:} An 8-bit Linear Feedback Shift Register (LFSR) uses the
polynomial x⁸ + x⁶ + x⁵ + x⁴ + 1, meaning feedback taps are at bit
positions 8, 6, 5, and 4 (XOR of these positions feeds back to the
input). The initial seed value is 0000 0001.

\textbf{Find:} (a) The maximum sequence length. (b) The first 5 states
of the LFSR (show the XOR feedback computation). (c) One application of
this LFSR. (d) Why the all-zeros state must be avoided.

\textbf{Solution:}

\begin{enumerate}
\def\labelenumi{(\alph{enumi})}
\item
  An n-bit LFSR with a maximal-length (primitive) polynomial produces a
  sequence of length 2ⁿ - 1 before repeating. Maximum sequence length =
  2⁸ - 1 = \textbf{255 states}
\item
  The polynomial x⁸ + x⁶ + x⁵ + x⁴ + 1 means the feedback is: new bit =
  Q₈ XOR Q₆ XOR Q₅ XOR Q₄ (using 1-indexed positions where Q₈ is the MSB
  being shifted out and Q₁ is the LSB).
\end{enumerate}

Using notation {[}Q₈ Q₇ Q₆ Q₅ Q₄ Q₃ Q₂ Q₁{]}, shift right, feedback
enters Q₈:

State 0: {[}0 0 0 0 0 0 0 1{]} Feedback = Q₈ XOR Q₆ XOR Q₅ XOR Q₄ = 0
XOR 0 XOR 0 XOR 0 = 0 State 1: {[}0 0 0 0 0 0 1 0{]} (shifted right, 0
enters MSB)

Wait --- let me clarify the shift direction. In a standard LFSR, bits
shift from MSB toward LSB (right), and feedback enters the MSB.

State 0: 0000 0001. Q₈Q₇Q₆Q₅Q₄Q₃Q₂Q₁ = 00000001. Feedback = Q₈ XOR Q₆
XOR Q₅ XOR Q₄ = 0 XOR 0 XOR 0 XOR 0 = 0. Shift right, feedback enters
Q₈: State 1: \textbf{0000 0000} --- but this is the all-zeros state,
which is a lockup state.

The issue is that seed 00000001 with the MSB-out shifting convention
produces zero feedback. Let me use the Fibonacci LFSR convention where
Q₁ is the output bit:

Feedback = Q₈ XOR Q₆ XOR Q₅ XOR Q₄. Shift left, output exits Q₁,
feedback enters Q₈\ldots{} Actually, let's use a cleaner convention:

Register {[}b₁ b₂ b₃ b₄ b₅ b₆ b₇ b₈{]}, shift right (b₁ is MSB input
side): Feedback to b₁ = b₈ XOR b₆ XOR b₅ XOR b₄.

State 0: {[}0 0 0 0 0 0 0 1{]} (b₈ = 1) Feedback = 1 XOR 0 XOR 0 XOR 0 =
1 State 1: {[}1 0 0 0 0 0 0 0{]} Feedback = 0 XOR 0 XOR 0 XOR 0 = 0
State 2: {[}0 1 0 0 0 0 0 0{]} Feedback = 0 XOR 0 XOR 0 XOR 0 = 0 State
3: {[}0 0 1 0 0 0 0 0{]} Feedback = 0 XOR 0 XOR 0 XOR 0 = 0 State 4:
{[}0 0 0 1 0 0 0 0{]} Feedback = 0 XOR 0 XOR 1 XOR 0 = 1

First 5 states: \textbf{00000001, 10000000, 01000000, 00100000,
00010000}

\begin{enumerate}
\def\labelenumi{(\alph{enumi})}
\setcounter{enumi}{2}
\item
  Applications: \textbf{CRC (Cyclic Redundancy Check) calculation} for
  error detection in communication protocols, pseudo-random number
  generation for built-in self-test (BIST), and data scrambling in
  serial communication links.
\item
  The all-zeros state must be avoided because XOR of all zeros produces
  zero feedback, so the LFSR would remain stuck at 00000000 forever.
  This is why LFSRs are initialized with a \textbf{nonzero seed} value.
  The all-zeros state is the only excluded state in a maximal-length
  LFSR sequence.
\end{enumerate}

\begin{center}\rule{0.5\linewidth}{0.5pt}\end{center}

\section{Problem 6.5.8}\label{problem-6.5.8}

\textbf{Given:} A 4-bit serial-in/parallel-out (SIPO) shift register
receives data at 1 Mbps (1 bit per microsecond). The serial input
sequence is: 1, 0, 1, 1, 0, 0, 1, 0 (transmitted LSB first). The
register is initially cleared to 0000.

\textbf{Find:} (a) The register contents after each of the first 4 clock
pulses. (b) The parallel output after 4 clocks (first nibble). (c) The
parallel output after 8 clocks (second nibble). (d) The total time to
receive one byte. (e) The equivalent parallel data rate.

\textbf{Solution:}

\begin{enumerate}
\def\labelenumi{(\alph{enumi})}
\tightlist
\item
  Data shifts in from the left (MSB position), LSB first:
\end{enumerate}

Clock 1: shift in 1 -\textgreater{} register = \textbf{1000} Clock 2:
shift in 0 -\textgreater{} register = \textbf{0100}

Wait --- for SIPO with LSB first, the first bit received occupies the
LSB position after the full word is loaded. Let me reconsider: as bits
shift in, the first bit ends up in the rightmost position.

Shift right convention (new data enters from the left): Clock 1: shift
in 1 -\textgreater{} {[}1 x x x{]} = \textbf{1000} Clock 2: shift in 0
-\textgreater{} {[}0 1 x x{]} = \textbf{0100} Clock 3: shift in 1
-\textgreater{} {[}1 0 1 x{]} = \textbf{1010} Clock 4: shift in 0
-\textgreater{} wait, the sequence is 1, 0, 1, 1\ldots{} Clock 3: shift
in 1 -\textgreater{} {[}1 0 1 x{]} = \textbf{1010} Clock 4: shift in 1
-\textgreater{} {[}1 1 0 1{]} = \textbf{1101}

\begin{enumerate}
\def\labelenumi{(\alph{enumi})}
\setcounter{enumi}{1}
\item
  After 4 clocks, parallel output = \textbf{1101} The first bit (1) is
  now in the LSB, and the fourth bit (1) is in the MSB. In decimal: 1101
  = \textbf{13} (or interpreting as received LSB-first: the nibble is
  1011 = 0xB = 11).
\item
  Continuing with bits 5-8 (0, 0, 1, 0): Clock 5: shift in 0
  -\textgreater{} {[}0 1 1 0{]} = \textbf{0110} Clock 6: shift in 0
  -\textgreater{} {[}0 0 1 1{]} = \textbf{0011} Clock 7: shift in 1
  -\textgreater{} {[}1 0 0 1{]} = \textbf{1001} Clock 8: shift in 0
  -\textgreater{} {[}0 1 0 0{]} = \textbf{0100}
\end{enumerate}

Second nibble parallel output: \textbf{0100}

\begin{enumerate}
\def\labelenumi{(\alph{enumi})}
\setcounter{enumi}{3}
\item
  Total time for 8 bits at 1 Mbps: t = 8 x 1 μs = \textbf{8 μs}
\item
  Parallel data rate: one byte every 8 μs: Data rate = 8 bits / 8 μs = 1
  Mbps serial = \textbf{125 kBytes/s} The parallel output is available
  every 4 μs (per nibble) or 8 μs (per byte), with the same aggregate
  bit rate.
\end{enumerate}

\begin{center}\rule{0.5\linewidth}{0.5pt}\end{center}

\section{Problem 6.5.9}\label{problem-6.5.9}

\textbf{Given:} A Moore state machine controls a traffic light at an
intersection. The light cycles through Green (30 s), Yellow (5 s), and
Red (35 s) for the main road. The cross street has the complementary
pattern (Red when main is Green or Yellow, Green when main is Red,
etc.). A 1 Hz clock drives a counter that generates timing signals.

\textbf{Find:} (a) The state diagram with states, outputs, and
transitions. (b) The number of states and flip-flops required. (c) The
state encoding (binary). (d) The counter terminal count values for each
state. (e) The total cycle time.

\textbf{Solution:}

\begin{enumerate}
\def\labelenumi{(\alph{enumi})}
\tightlist
\item
  State diagram for the main road:
\end{enumerate}

State S0 (Main Green, Cross Red): Output = Main Green / Cross Red. After
30 s -\textgreater{} transition to S1. State S1 (Main Yellow, Cross
Red): Output = Main Yellow / Cross Red. After 5 s -\textgreater{}
transition to S2. State S2 (Main Red, Cross Green): Output = Main Red /
Cross Green. After 30 s -\textgreater{} transition to S3. State S3 (Main
Red, Cross Yellow): Output = Main Red / Cross Yellow. After 5 s
-\textgreater{} transition to S0.

\begin{enumerate}
\def\labelenumi{(\alph{enumi})}
\setcounter{enumi}{1}
\item
  Number of states: \textbf{4} Flip-flops required: ceil(log₂(4)) =
  \textbf{2 flip-flops} (Q₁Q₀)
\item
  State encoding: S0 = 00 (Main Green / Cross Red) S1 = 01 (Main Yellow
  / Cross Red) S2 = 10 (Main Red / Cross Green) S3 = 11 (Main Red /
  Cross Yellow)
\item
  Using a counter clocked at 1 Hz: S0: count to 30, terminal count =
  \textbf{29} (counts 0 to 29) S1: count to 5, terminal count =
  \textbf{4} (counts 0 to 4) S2: count to 30, terminal count =
  \textbf{29} S3: count to 5, terminal count = \textbf{4}
\end{enumerate}

A 6-bit counter (max count 63) is sufficient for the longest interval.

\begin{enumerate}
\def\labelenumi{(\alph{enumi})}
\setcounter{enumi}{4}
\tightlist
\item
  Total cycle time: T\textsubscript{cycle} = 30 + 5 + 30 + 5 =
  \textbf{70 seconds}
\end{enumerate}

\begin{center}\rule{0.5\linewidth}{0.5pt}\end{center}

\section{Problem 6.5.10}\label{problem-6.5.10}

\textbf{Given:} A Mealy state machine detects the bit sequence ``1010''
on a serial input line X. The output Z = 1 during the clock cycle in
which the final bit of the sequence is received. The detector is
overlapping (the last bits of one detection can be part of the next).
The clock rate is 10 MHz.

\textbf{Find:} (a) The state diagram with 4 states. (b) The state
transition table. (c) The next-state equations using D flip-flops with
binary state encoding. (d) The output equation. (e) The maximum
detection rate.

\textbf{Solution:}

\begin{enumerate}
\def\labelenumi{(\alph{enumi})}
\tightlist
\item
  States based on progress toward ``1010'':
\end{enumerate}

S0 = no progress (idle). S1 = received ``1''. S2 = received ``10''. S3 =
received ``101''.

When in S3 and X = 0, the full sequence ``1010'' is detected (Z = 1),
and the machine transitions to S2 (overlapping: ``10'' is a prefix of
the next potential ``1010'').

\begin{enumerate}
\def\labelenumi{(\alph{enumi})}
\setcounter{enumi}{1}
\tightlist
\item
  State transition table (Mealy: output depends on state and input):
\end{enumerate}

{\def\LTcaptype{none} % do not increment counter
\begin{longtable}[]{@{}lllll@{}}
\toprule\noalign{}
Current State & X = 0 & Z (X=0) & X = 1 & Z (X=1) \\
\midrule\noalign{}
\endhead
\bottomrule\noalign{}
\endlastfoot
S0 (00) & S0 & 0 & S1 & 0 \\
S1 (01) & S2 & 0 & S1 & 0 \\
S2 (10) & S0 & 0 & S3 & 0 \\
S3 (11) & S2 & \textbf{1} & S1 & 0 \\
\end{longtable}
}

\begin{enumerate}
\def\labelenumi{(\alph{enumi})}
\setcounter{enumi}{2}
\tightlist
\item
  State encoding: S0=00, S1=01, S2=10, S3=11 (Q₁Q₀).
\end{enumerate}

Next-state equations from the table: Q₁\textsuperscript{+}: is 1 in
transitions to S2 (10) and S3 (11). - S1,X=0 -\textgreater{} S2: Q₁'Q₀X'
-\textgreater{} Q₁\textsuperscript{+} = 1 - S2,X=1 -\textgreater{} S3:
Q₁Q₀'X -\textgreater{} Q₁\textsuperscript{+} = 1 - S3,X=0
-\textgreater{} S2: Q₁Q₀X' -\textgreater{} Q₁\textsuperscript{+} = 1

\textbf{Q₁\textsuperscript{+} = Q₁'Q₀X' + Q₁Q₀'X + Q₁Q₀X'} Simplify:
Q₁\textsuperscript{+} = Q₀X' + Q₁Q₀'X = \textbf{Q₀X' + Q₁X(Q₀')}
Further: = X'Q₀ + XQ₁Q₀'

Q₀\textsuperscript{+}: is 1 in transitions to S1 (01) and S3 (11). -
S0,X=1 -\textgreater{} S1: Q₁'Q₀'X - S1,X=1 -\textgreater{} S1: Q₁'Q₀X -
S2,X=1 -\textgreater{} S3: Q₁Q₀'X - S3,X=1 -\textgreater{} S1: Q₁Q₀X

All transitions to states with Q₀=1 have X=1:
\textbf{Q₀\textsuperscript{+} = X}

\begin{enumerate}
\def\labelenumi{(\alph{enumi})}
\setcounter{enumi}{3}
\item
  Output equation (Mealy --- depends on state and input): Z = 1 only
  when in S3 (Q₁Q₀ = 11) and X = 0: \textbf{Z = Q₁ * Q₀ * X'}
\item
  The minimum time between detections in an overlapping detector occurs
  with the sequence ``\ldots10101010\ldots{}'': The pattern ``1010'' is
  detected every 2 clock cycles (the next ``1010'' starts 2 bits after
  the previous one since they overlap by 2 bits). Maximum detection rate
  = f\textsubscript{clk} / 2 = 10 MHz / 2 = \textbf{5 million detections
  per second}
\end{enumerate}

\chapter{Chapter 6 --- Section 6.6: Programmable
Logic}\label{chapter-6-section-6.6-programmable-logic}

Practice problems covering FPGAs, CPLDs, timing analysis, LUT
utilization, clock domain crossing, and design trade-offs.

\begin{center}\rule{0.5\linewidth}{0.5pt}\end{center}

\section{Problem 6.6.1}\label{problem-6.6.1}

\textbf{Given:} An FPGA contains 10,000 6-input lookup tables (LUTs) and
10,000 flip-flops. A digital design requires the following resources: -
32-bit ALU: 180 LUTs, 32 flip-flops - 16 KB block RAM: 0 LUTs, 0
flip-flops (uses dedicated BRAM) - UART controller: 85 LUTs, 48
flip-flops - SPI master: 60 LUTs, 35 flip-flops - I2C controller: 70
LUTs, 40 flip-flops - PWM generator (8 channels): 120 LUTs, 64
flip-flops - Interrupt controller (16 sources): 95 LUTs, 32 flip-flops -
GPIO block (32 pins): 96 LUTs, 64 flip-flops

\textbf{Find:} (a) The total LUT and flip-flop usage. (b) The
utilization percentages. (c) Whether the design fits. (d) The remaining
resources available for additional logic. (e) The recommended maximum
utilization for routability.

\textbf{Solution:}

\begin{enumerate}
\def\labelenumi{(\alph{enumi})}
\tightlist
\item
  Total LUT usage: 180 + 85 + 60 + 70 + 120 + 95 + 96 = \textbf{706
  LUTs}
\end{enumerate}

Total flip-flop usage: 32 + 48 + 35 + 40 + 64 + 32 + 64 = \textbf{315
flip-flops}

\begin{enumerate}
\def\labelenumi{(\alph{enumi})}
\setcounter{enumi}{1}
\item
  Utilization: LUT utilization = 706 / 10,000 = 7.06\% -\textgreater{}
  \textbf{7.1\%} Flip-flop utilization = 315 / 10,000 = 3.15\%
  -\textgreater{} \textbf{3.2\%}
\item
  Both resources are well below 100\%. \textbf{The design fits
  comfortably.}
\item
  Remaining resources: LUTs available = 10,000 - 706 = \textbf{9,294
  LUTs} Flip-flops available = 10,000 - 315 = \textbf{9,685 flip-flops}
\item
  FPGA designs should generally stay below \textbf{70-80\% utilization}
  to ensure the place-and-route tool can find valid routing paths and
  meet timing. Beyond 80\%, routing congestion increases dramatically,
  timing closure becomes difficult, and compilation times grow
  significantly. For this design at 7.1\% LUT usage, there is ample
  margin.
\end{enumerate}

\begin{center}\rule{0.5\linewidth}{0.5pt}\end{center}

\section{Problem 6.6.2}\label{problem-6.6.2}

\textbf{Given:} A CPLD has 256 macrocells with a guaranteed maximum
pin-to-pin delay of 5.0 ns. It is used to implement address decoding for
a microprocessor system with the following requirements: - 4 chip select
outputs for SRAM (each 64 KB, starting at 0x00000) - 2 chip select
outputs for flash memory (each 256 KB, starting at 0x40000) - 1 chip
select for an I/O peripheral block (4 KB at 0xC0000) - Wait-state
generation logic (8 macrocells) - Bus arbitration logic for 2 bus
masters (12 macrocells)

The address bus is 20 bits wide (A₁₉ through A₀).

\textbf{Find:} (a) The address bits used to decode each chip select. (b)
The macrocell count for address decoding. (c) The total macrocell usage.
(d) The utilization percentage. (e) The maximum bus speed supported.

\textbf{Solution:}

\begin{enumerate}
\def\labelenumi{(\alph{enumi})}
\tightlist
\item
  Address decoding:
\end{enumerate}

SRAM (4 x 64 KB = 256 KB total, 0x00000 to 0x3FFFF): - Each 64 KB block
uses 16 address bits (A₁₅ to A₀) internally. - Decode using A₁₉A₁₈ = 00,
then A₁₇A₁₆ selects the chip: - CS₀: A₁₉A₁₈ = 00, A₁₇A₁₆ = 00
-\textgreater{} 0x00000 to 0x0FFFF - CS₁: A₁₉A₁₈ = 00, A₁₇A₁₆ = 01
-\textgreater{} 0x10000 to 0x1FFFF - CS₂: A₁₉A₁₈ = 00, A₁₇A₁₆ = 10
-\textgreater{} 0x20000 to 0x2FFFF - CS₃: A₁₉A₁₈ = 00, A₁₇A₁₆ = 11
-\textgreater{} 0x30000 to 0x3FFFF - Decode bits: \textbf{A₁₉, A₁₈, A₁₇,
A₁₆}

Flash (2 x 256 KB = 512 KB, 0x40000 to 0xBFFFF): - Each 256 KB block
uses 18 address bits (A₁₇ to A₀) internally. - Decode using A₁₉A₁₈:
flash region starts at 0x40000 (A₁₉A₁₈ = 01) through 0xBFFFF (A₁₉A₁₈ =
10). - CS₄: A₁₉ = 0, A₁₈ = 1 -\textgreater{} 0x40000 to 0x7FFFF - CS₅:
A₁₉ = 1, A₁₈ = 0 -\textgreater{} 0x80000 to 0xBFFFF - Decode bits:
\textbf{A₁₉, A₁₈}

I/O peripheral (4 KB at 0xC0000 to 0xC0FFF): - 4 KB = 12 address bits
(A₁₁ to A₀) for internal decoding. - Decode bits: \textbf{A₁₉ through
A₁₂} (8 bits must match 0xC0 = 1100 0000). - CS₆: A₁₉A₁₈A₁₇A₁₆ = 1100,
A₁₅A₁₄A₁₃A₁₂ = 0000

\begin{enumerate}
\def\labelenumi{(\alph{enumi})}
\setcounter{enumi}{1}
\tightlist
\item
  Macrocells for address decoding:
\end{enumerate}

\begin{itemize}
\tightlist
\item
  4 SRAM chip selects: 4 macrocells (each is a simple AND of 2-4 address
  bits)
\item
  2 flash chip selects: 2 macrocells
\item
  1 I/O chip select: 1 macrocell (AND of 8 address bits, fits in one
  macrocell with product term sharing)
\end{itemize}

Total decoding macrocells: \textbf{7 macrocells}

\begin{enumerate}
\def\labelenumi{(\alph{enumi})}
\setcounter{enumi}{2}
\item
  Total macrocell usage: Decoding: 7 Wait-state generation: 8 Bus
  arbitration: 12 Total = 7 + 8 + 12 = \textbf{27 macrocells}
\item
  Utilization = 27 / 256 = \textbf{10.5\%}
\item
  The pin-to-pin delay of 5.0 ns allows address decode to complete
  within one bus cycle. For a synchronous bus, the maximum speed:
  f\textsubscript{max} = 1 / (2 x t\textsubscript{pd}) = 1 / (2 x 5.0
  ns) = 1 / 10 ns = \textbf{100 MHz}
\end{enumerate}

For an asynchronous bus where decode must complete within the address
setup time (typically one-half cycle): f\textsubscript{bus,max} =
\textbf{100 MHz}

\begin{center}\rule{0.5\linewidth}{0.5pt}\end{center}

\section{Problem 6.6.3}\label{problem-6.6.3}

\textbf{Given:} A synchronous digital circuit in an FPGA has the
following timing parameters: - Clock period: T\textsubscript{clk} = 6.25
ns (160 MHz) - Source flip-flop t\textsubscript{cq} = 0.5 ns -
Destination flip-flop t\textsubscript{su} = 0.3 ns, t\textsubscript{h} =
0.15 ns - Routing delay (source FF to logic): t\textsubscript{route1} =
0.8 ns - Combinational logic: 3 LUTs in series, each with
t\textsubscript{LUT} = 0.4 ns - Routing delay (logic to destination FF):
t\textsubscript{route2} = 0.6 ns - Clock skew: t\textsubscript{skew} =
0.2 ns (destination clock arrives late)

\textbf{Find:} (a) The total data path delay. (b) The setup slack. (c)
The hold slack (minimum path delay through logic is 0.3 ns with 0.2 ns
routing). (d) Whether the design meets timing. (e) The maximum
achievable frequency.

\textbf{Solution:}

\begin{enumerate}
\def\labelenumi{(\alph{enumi})}
\item
  Total data path delay: t\textsubscript{data} = t\textsubscript{cq} +
  t\textsubscript{route1} + 3 x t\textsubscript{LUT} +
  t\textsubscript{route2} t\textsubscript{data} = 0.5 + 0.8 + 3 x 0.4 +
  0.6 = 0.5 + 0.8 + 1.2 + 0.6 = \textbf{3.1 ns}
\item
  Setup time analysis (with clock skew --- late arrival at destination
  means less time for data): Setup slack = T\textsubscript{clk} -
  t\textsubscript{data} - t\textsubscript{su} - t\textsubscript{skew}
  Setup slack = 6.25 - 3.1 - 0.3 - 0.2 = \textbf{2.65 ns (positive ---
  met)}
\item
  Hold time analysis (minimum path delay): t\textsubscript{data,min} =
  t\textsubscript{cq} + t\textsubscript{route,min} +
  t\textsubscript{logic,min} = 0.5 + 0.2 + 0.3 = 1.0 ns Hold slack =
  t\textsubscript{data,min} - t\textsubscript{h} - t\textsubscript{skew}
\end{enumerate}

With late-arriving destination clock, hold is easier to meet (the data
has more time): Hold slack = t\textsubscript{data,min} +
t\textsubscript{skew} - t\textsubscript{h} = 1.0 + 0.2 - 0.15 =
\textbf{1.05 ns (positive --- met)}

\begin{enumerate}
\def\labelenumi{(\alph{enumi})}
\setcounter{enumi}{3}
\item
  Both setup slack (2.65 ns) and hold slack (1.05 ns) are positive.
  \textbf{The design meets timing at 160 MHz.}
\item
  Maximum frequency (setup-limited): T\textsubscript{clk,min} =
  t\textsubscript{data} + t\textsubscript{su} + t\textsubscript{skew} =
  3.1 + 0.3 + 0.2 = 3.6 ns f\textsubscript{max} = 1 /
  T\textsubscript{clk,min} = 1 / 3.6 ns = \textbf{277.8 MHz}
\end{enumerate}

\begin{center}\rule{0.5\linewidth}{0.5pt}\end{center}

\section{Problem 6.6.4}\label{problem-6.6.4}

\textbf{Given:} An FPGA design must cross between two asynchronous clock
domains: a 100 MHz domain (source) and a 75 MHz domain (destination). A
two-flip-flop synchronizer is used to safely transfer a single-bit
control signal from the source to the destination domain. The
destination flip-flops have t\textsubscript{su} = 0.3 ns and a
metastability resolution time constant τ = 0.2 ns.

\textbf{Find:} (a) The destination clock period. (b) The time available
for metastability resolution in the synchronizer (settling window). (c)
The mean time between failures (MTBF) if the metastability window
t\textsubscript{w} = 0.04 ns and the data change rate is 100 million
transitions/second. Use MTBF = e\textsuperscript{tr/τ} /
(t\textsubscript{w} x f\textsubscript{clk} x f\textsubscript{data}). (d)
Whether this MTBF is acceptable for a system requiring 10 years of
reliable operation.

\textbf{Solution:}

\begin{enumerate}
\def\labelenumi{(\alph{enumi})}
\item
  Destination clock period: T\textsubscript{dst} = 1 / 75 x 10⁶ =
  \textbf{13.33 ns}
\item
  The settling window is the time between when the first synchronizer
  flip-flop captures data and when the second flip-flop samples it (one
  full destination clock period, minus setup time): t\textsubscript{r} =
  T\textsubscript{dst} - t\textsubscript{su} = 13.33 - 0.3 =
  \textbf{13.03 ns}
\item
  MTBF calculation: MTBF = e\textsuperscript{tr/τ} / (t\textsubscript{w}
  x f\textsubscript{clk} x f\textsubscript{data})
\end{enumerate}

Numerator: e\textsuperscript{13.03/0.2} = e\textsuperscript{65.15}

e\textsuperscript{65.15} is extremely large. Using ln(10) = 2.3026:
65.15 / 2.3026 = 28.3, so e\textsuperscript{65.15} ≈
10\textsuperscript{28.3} ≈ 2.0 x 10²⁸

Denominator: t\textsubscript{w} x f\textsubscript{clk} x
f\textsubscript{data} = 0.04 x 10⁻⁹ x 75 x 10⁶ x 100 x 10⁶ = 0.04 x 10⁻⁹
x 7.5 x 10¹⁵ = 0.3 x 10⁶ = 3.0 x 10⁵

MTBF = 2.0 x 10²⁸ / 3.0 x 10⁵ = \textbf{6.67 x 10²² seconds}

Converting to years: 6.67 x 10²² / (3.156 x 10⁷) = \textbf{2.11 x 10¹⁵
years}

\begin{enumerate}
\def\labelenumi{(\alph{enumi})}
\setcounter{enumi}{3}
\tightlist
\item
  The required lifetime is 10 years. The MTBF of 2.11 x 10¹⁵ years
  exceeds this by a factor of 2.11 x 10¹⁴. \textbf{The two-flip-flop
  synchronizer is more than adequate.} This enormous margin is typical
  for two-stage synchronizers at moderate clock frequencies.
\end{enumerate}

\begin{center}\rule{0.5\linewidth}{0.5pt}\end{center}

\section{Problem 6.6.5}\label{problem-6.6.5}

\textbf{Given:} An FPGA has 6-input LUTs. A designer needs to implement
several combinational functions: - Function A: 4-input Boolean function
(1 minterm table) - Function B: 6-input Boolean function - Function C:
8-input Boolean function - Function D: 3-input Boolean function combined
with a 2-input function (sharing inputs)

\textbf{Find:} (a) The number of LUTs required for each function. (b)
The total LUT count. (c) How a 6-input LUT implements a function (truth
table size). (d) The technique used to implement Function C across
multiple LUTs.

\textbf{Solution:}

\begin{enumerate}
\def\labelenumi{(\alph{enumi})}
\tightlist
\item
  LUT requirements:
\end{enumerate}

Function A (4 inputs): A 6-input LUT can implement any function of up to
6 variables. A 4-input function fits in one LUT with 2 inputs unused
(tied to constant values). LUTs for A: \textbf{1 LUT}

Function B (6 inputs): exactly matches the LUT size. LUTs for B:
\textbf{1 LUT}

Function C (8 inputs): exceeds the 6-input LUT capacity. The function
must be decomposed. The synthesis tool uses Shannon decomposition:
F(x₁,\ldots,x₈) = x₇' * F(x₁,\ldots,x₆,0,x₈) + x₇ * F(x₁,\ldots,x₆,1,x₈)
Wait --- this still has 7 variables per sub-function. For 8 inputs
decomposed across 6-input LUTs:

Method: Fix 2 variables (e.g., x₇, x₈) and implement 4 sub-functions of
6 variables, then use a 4:1 MUX (another LUT) to select. - 4 LUTs for
the sub-functions of x₁ through x₆ (one for each combination of x₇, x₈)
- 1 LUT for the 4:1 MUX (inputs: 4 sub-function outputs + x₇ + x₈ = 6
inputs)

LUTs for C: \textbf{5 LUTs}

Function D (3+2 inputs, 5 total): fits in one 6-input LUT since 5
\textless{} 6. LUTs for D: \textbf{1 LUT}

\begin{enumerate}
\def\labelenumi{(\alph{enumi})}
\setcounter{enumi}{1}
\item
  Total LUT count = 1 + 1 + 5 + 1 = \textbf{8 LUTs}
\item
  A 6-input LUT stores a truth table with 2⁶ = \textbf{64 entries} (one
  bit per entry). The 6 inputs serve as address lines to this 64-bit
  memory, and the stored bit at the addressed location becomes the
  output. This is equivalent to a 64x1 ROM.
\item
  Function C uses \textbf{Shannon decomposition} (also called function
  decomposition or cofactor-based decomposition). The function is split
  into sub-functions of fewer variables, each fitting in a single LUT,
  with a multiplexer LUT combining the results.
\end{enumerate}

\begin{center}\rule{0.5\linewidth}{0.5pt}\end{center}

\section{Problem 6.6.6}\label{problem-6.6.6}

\textbf{Given:} A combinational logic path in an FPGA consists of: -
Source register: t\textsubscript{cq} = 0.4 ns - 4 levels of LUT logic:
t\textsubscript{LUT} = 0.3 ns each - 5 routing segments between
elements: t\textsubscript{route} = 0.5 ns, 0.3 ns, 0.7 ns, 0.4 ns, 0.6
ns - Destination register: t\textsubscript{su} = 0.2 ns

The designer considers pipelining by inserting a register after the
second LUT to split the path into two stages.

\textbf{Find:} (a) The critical path delay without pipelining. (b) The
maximum frequency without pipelining. (c) The critical path delay of
each stage with pipelining. (d) The maximum frequency with pipelining.
(e) The throughput improvement. (f) The latency increase.

\textbf{Solution:}

\begin{enumerate}
\def\labelenumi{(\alph{enumi})}
\item
  Without pipelining: t\textsubscript{path} = t\textsubscript{cq} +
  t\textsubscript{route1} + t\textsubscript{LUT1} +
  t\textsubscript{route2} + t\textsubscript{LUT2} +
  t\textsubscript{route3} + t\textsubscript{LUT3} +
  t\textsubscript{route4} + t\textsubscript{LUT4} +
  t\textsubscript{route5} + t\textsubscript{su} t\textsubscript{path} =
  0.4 + 0.5 + 0.3 + 0.3 + 0.3 + 0.7 + 0.3 + 0.4 + 0.3 + 0.6 + 0.2
  t\textsubscript{path} = \textbf{4.3 ns}
\item
  f\textsubscript{max} = 1 / 4.3 ns = \textbf{232.6 MHz}
\item
  With pipeline register after LUT2: Stage 1: t\textsubscript{cq} +
  t\textsubscript{route1} + t\textsubscript{LUT1} +
  t\textsubscript{route2} + t\textsubscript{LUT2} +
  t\textsubscript{route3}(to pipeline reg) + t\textsubscript{su} Assume
  the pipeline register is placed at the route3 boundary. The routing
  splits approximately: Stage 1 = 0.4 + 0.5 + 0.3 + 0.3 + 0.3 + 0.35 +
  0.2 = \textbf{2.35 ns}
\end{enumerate}

Stage 2: t\textsubscript{cq}(pipe reg) + t\textsubscript{route3b} +
t\textsubscript{LUT3} + t\textsubscript{route4} + t\textsubscript{LUT4}
+ t\textsubscript{route5} + t\textsubscript{su} Stage 2 = 0.4 + 0.35 +
0.3 + 0.4 + 0.3 + 0.6 + 0.2 = \textbf{2.55 ns}

Critical path = max(Stage 1, Stage 2) = \textbf{2.55 ns}

\begin{enumerate}
\def\labelenumi{(\alph{enumi})}
\setcounter{enumi}{3}
\item
  f\textsubscript{max,pipelined} = 1 / 2.55 ns = \textbf{392.2 MHz}
\item
  Throughput improvement = 392.2 / 232.6 = \textbf{1.69x (69\%
  improvement)}
\item
  Latency without pipelining: 1 clock cycle = 4.3 ns Latency with
  pipelining: 2 clock cycles x 2.55 ns = 5.1 ns Latency increase: 5.1 -
  4.3 = \textbf{0.8 ns increase} (or 1 additional clock cycle of
  latency)
\end{enumerate}

The trade-off: pipelining increases throughput at the cost of slightly
higher latency and one additional flip-flop per data bit.

\begin{center}\rule{0.5\linewidth}{0.5pt}\end{center}

\section{Problem 6.6.7}\label{problem-6.6.7}

\textbf{Given:} A static timing analysis report for an FPGA design
operating at 200 MHz shows the following 5 critical paths:

{\def\LTcaptype{none} % do not increment counter
\begin{longtable}[]{@{}
  >{\raggedright\arraybackslash}p{(\linewidth - 10\tabcolsep) * \real{0.0845}}
  >{\raggedright\arraybackslash}p{(\linewidth - 10\tabcolsep) * \real{0.1127}}
  >{\raggedright\arraybackslash}p{(\linewidth - 10\tabcolsep) * \real{0.1831}}
  >{\raggedright\arraybackslash}p{(\linewidth - 10\tabcolsep) * \real{0.2394}}
  >{\raggedright\arraybackslash}p{(\linewidth - 10\tabcolsep) * \real{0.2113}}
  >{\raggedright\arraybackslash}p{(\linewidth - 10\tabcolsep) * \real{0.1690}}@{}}
\toprule\noalign{}
\begin{minipage}[b]{\linewidth}\raggedright
Path
\end{minipage} & \begin{minipage}[b]{\linewidth}\raggedright
Source
\end{minipage} & \begin{minipage}[b]{\linewidth}\raggedright
Destination
\end{minipage} & \begin{minipage}[b]{\linewidth}\raggedright
Data Delay (ns)
\end{minipage} & \begin{minipage}[b]{\linewidth}\raggedright
Required (ns)
\end{minipage} & \begin{minipage}[b]{\linewidth}\raggedright
Slack (ns)
\end{minipage} \\
\midrule\noalign{}
\endhead
\bottomrule\noalign{}
\endlastfoot
1 & REG\_A & REG\_B & 4.8 & 4.7 & -0.1 \\
2 & REG\_C & REG\_D & 4.5 & 4.7 & +0.2 \\
3 & REG\_E & REG\_F & 4.6 & 4.7 & +0.1 \\
4 & REG\_A & REG\_G & 4.9 & 4.7 & -0.2 \\
5 & REG\_B & REG\_H & 4.3 & 4.7 & +0.4 \\
\end{longtable}
}

The required time is T\textsubscript{clk} - t\textsubscript{su} = 5.0 -
0.3 = 4.7 ns.

\textbf{Find:} (a) Which paths fail timing. (b) The worst negative slack
(WNS). (c) The total negative slack (TNS). (d) The maximum safe
operating frequency. (e) Three strategies to fix the timing violations.

\textbf{Solution:}

\begin{enumerate}
\def\labelenumi{(\alph{enumi})}
\item
  Paths with negative slack fail timing: \textbf{Path 1: slack = -0.1 ns
  (FAIL)} \textbf{Path 4: slack = -0.2 ns (FAIL)}
\item
  Worst negative slack (WNS): the most negative slack value. \textbf{WNS
  = -0.2 ns} (Path 4, REG\_A to REG\_G)
\item
  Total negative slack (TNS): sum of all negative slacks. TNS = (-0.1) +
  (-0.2) = \textbf{-0.3 ns}
\item
  Maximum safe frequency: determined by the worst path. Worst path delay
  = 4.9 ns (Path 4). T\textsubscript{clk,min} = 4.9 +
  t\textsubscript{su} = 4.9 + 0.3 = 5.2 ns f\textsubscript{max} = 1 /
  5.2 ns = \textbf{192.3 MHz}
\item
  Three strategies to fix timing violations:
\end{enumerate}

\begin{enumerate}
\def\labelenumi{\arabic{enumi}.}
\item
  \textbf{Retiming/pipelining}: Insert a pipeline register in the
  combinational logic between REG\_A and REG\_B/REG\_G. Since REG\_A is
  the common source of both failing paths, pipelining its output path
  will fix both violations.
\item
  \textbf{Logic optimization}: Restructure the combinational logic to
  reduce the number of logic levels. Replace cascaded logic with
  parallel structures (tree reduction), or use FPGA-specific primitives
  (carry chains, DSP blocks) that have faster propagation.
\item
  \textbf{Physical constraints}: Apply placement constraints
  (floorplanning) to place REG\_A closer to REG\_B and REG\_G, reducing
  routing delay. Use the FPGA tool's physical optimization features such
  as post-route optimization or register duplication to reduce fanout on
  critical nets.
\end{enumerate}

\begin{center}\rule{0.5\linewidth}{0.5pt}\end{center}

\section{Problem 6.6.8}\label{problem-6.6.8}

\textbf{Given:} A designer must choose between an FPGA and a CPLD for a
glue logic application. The requirements are: - 45 I/O pins - 180
macrocells equivalent of logic - Must be operational within 10 ms of
power-up - Maximum pin-to-pin delay: 8 ns - Production volume: 500
units/year - Power budget: 200 mW maximum

Available options: - CPLD: 256 macrocells, 64 I/O pins,
t\textsubscript{pd} = 5 ns, non-volatile, P\textsubscript{static} = 50
mW, unit cost \$3.50 - FPGA: 1,500 LUTs, 100 I/O pins,
t\textsubscript{pd} = 3 ns (after configuration), requires external
flash, configuration time = 50 ms, P\textsubscript{static} = 150 mW,
unit cost \$8.00 (+ \$1.50 for config flash)

\textbf{Find:} (a) Whether each device meets all requirements. (b) A
comparison table. (c) The recommended choice with justification. (d) The
annual cost difference.

\textbf{Solution:}

\begin{enumerate}
\def\labelenumi{(\alph{enumi})}
\tightlist
\item
  Requirements check:
\end{enumerate}

CPLD: - I/O pins: 64 \textgreater= 45. \textbf{PASS} - Logic: 256
macrocells \textgreater= 180. \textbf{PASS} - Power-up time: instant
(non-volatile). \textbf{PASS} (\textless{} 10 ms) - Pin-to-pin delay: 5
ns \textless= 8 ns. \textbf{PASS} - Power: 50 mW \textless= 200 mW.
\textbf{PASS}

FPGA: - I/O pins: 100 \textgreater= 45. \textbf{PASS} - Logic: 1,500
LUTs \textgreater\textgreater{} 180. \textbf{PASS} (more than enough) -
Power-up time: 50 ms \textgreater{} 10 ms. \textbf{FAIL} - Pin-to-pin
delay: 3 ns \textless= 8 ns. \textbf{PASS} - Power: 150 mW \textless=
200 mW. \textbf{PASS}

\begin{enumerate}
\def\labelenumi{(\alph{enumi})}
\setcounter{enumi}{1}
\tightlist
\item
  Comparison table:
\end{enumerate}

{\def\LTcaptype{none} % do not increment counter
\begin{longtable}[]{@{}llll@{}}
\toprule\noalign{}
Parameter & CPLD & FPGA & Requirement \\
\midrule\noalign{}
\endhead
\bottomrule\noalign{}
\endlastfoot
I/O pins & 64 & 100 & 45 \\
Logic capacity & 256 MC & 1,500 LUTs & 180 MC \\
Power-up & Instant & 50 ms & \textless{} 10 ms \\
Delay & 5 ns & 3 ns & \textless{} 8 ns \\
Power & 50 mW & 150 mW & \textless{} 200 mW \\
Unit cost & \$3.50 & \$9.50 & --- \\
Config memory & Built-in & External & --- \\
\end{longtable}
}

\begin{enumerate}
\def\labelenumi{(\alph{enumi})}
\setcounter{enumi}{2}
\tightlist
\item
  \textbf{Recommended choice: CPLD}
\end{enumerate}

Justification: The FPGA fails the 10 ms power-up requirement since it
needs 50 ms to load its configuration from external flash. The CPLD
meets all requirements and has lower power (50 mW vs.~150 mW), lower
cost (\$3.50 vs.~\$9.50), and simpler BOM (no external configuration
flash). The CPLD's deterministic timing also simplifies design
validation.

\begin{enumerate}
\def\labelenumi{(\alph{enumi})}
\setcounter{enumi}{3}
\tightlist
\item
  Annual cost difference: CPLD: 500 x \$3.50 = \$1,750/year FPGA: 500 x
  (\$8.00 + \$1.50) = 500 x \$9.50 = \$4,750/year Difference = \$4,750 -
  \$1,750 = \textbf{\$3,000/year savings with CPLD}
\end{enumerate}

\begin{center}\rule{0.5\linewidth}{0.5pt}\end{center}

\section{Problem 6.6.9}\label{problem-6.6.9}

\textbf{Given:} An FPGA-based design has a static hazard in a
combinational output. The Boolean function is F = AC + A'B, and a
transition occurs where A changes from 1 to 0 while B = 1 and C = 1
(input changes from ABC = 111 to ABC = 011). The gate delays are: AND
gate = 3 ns, OR gate = 2 ns, inverter = 2 ns.

\textbf{Find:} (a) The expected output value before and after the
transition. (b) The timing of each signal during the transition (show
the glitch). (c) The type of hazard. (d) The redundant term needed to
eliminate the hazard. (e) The corrected Boolean expression.

\textbf{Solution:}

\begin{enumerate}
\def\labelenumi{(\alph{enumi})}
\item
  Before: ABC = 111. F = (1)(1) + (0)(1) = 1 + 0 = \textbf{1} After: ABC
  = 011. F = (0)(1) + (1)(1) = 0 + 1 = \textbf{1} The output should
  remain at 1 throughout the transition.
\item
  Signal timing (A changes from 1 to 0 at t = 0):
\end{enumerate}

t = 0 ns: A transitions from 1 to 0.

Path through AC (direct): - AC: A goes to 0, C = 1. AC drops from 1 to 0
at t = 3 ns (AND gate delay).

Path through A'B (through inverter): - A' rises from 0 to 1 at t = 2 ns
(inverter delay). - A'B: rises from 0 to 1 at t = 2 + 3 = 5 ns (inverter
+ AND gate delay).

OR gate output F: - At t = 3 ns: AC = 0, A'B still = 0 (hasn't risen
yet). F = 0 + 0 = 0. - At t = 5 ns: AC = 0, A'B = 1 at AND output. F
updates at t = 5 + 2 = 7 ns: F = 0 + 1 = 1.

\textbf{Glitch: F drops from 1 to 0 during t = 3+2 = 5 ns to t = 7 ns.}
Actually, let's be more precise:

F (OR gate output) responds to its inputs: - F was 1 (from AC=1). At t =
3 ns, AC input to OR drops. OR delay = 2 ns, so F drops at t = 3 + 2 =
\textbf{5 ns}. - A'B rises at t = 5 ns. OR responds at t = 5 + 2 =
\textbf{7 ns}, F returns to 1.

\textbf{The glitch lasts from t = 5 ns to t = 7 ns (2 ns wide).}

\begin{enumerate}
\def\labelenumi{(\alph{enumi})}
\setcounter{enumi}{2}
\item
  This is a \textbf{static-1 hazard} (output should remain at 1 but
  momentarily dips to 0).
\item
  The hazard occurs because BC = 1 during the transition, but there is
  no term in F covering BC. On the K-map, the hazard exists between the
  two prime implicants AC and A'B when B = C = 1.
\end{enumerate}

The redundant consensus term is: \textbf{BC}

\begin{enumerate}
\def\labelenumi{(\alph{enumi})}
\setcounter{enumi}{4}
\tightlist
\item
  Corrected expression: \textbf{F = AC + A'B + BC}
\end{enumerate}

The added BC term holds the output at 1 during the transition, since B =
C = 1 throughout. This eliminates the static-1 hazard at the cost of one
additional AND gate.

\begin{center}\rule{0.5\linewidth}{0.5pt}\end{center}

\section{Problem 6.6.10}\label{problem-6.6.10}

\textbf{Given:} An FPGA implements a digital FIR filter that processes
8-bit audio samples at a sample rate of 48 kHz. The filter has 64 taps
with 12-bit coefficients. Each tap requires one multiply-accumulate
(MAC) operation. The FPGA has dedicated DSP blocks that can perform one
18x18 bit multiply + 48-bit accumulate per clock cycle with a DSP block
delay of 4 ns. The FPGA fabric clock is 100 MHz.

\textbf{Find:} (a) The number of clock cycles available per sample. (b)
The minimum number of DSP blocks needed to process all 64 taps within
one sample period (serial approach). (c) The number of DSP blocks needed
for a fully parallel implementation. (d) The minimum DSP clock frequency
for a single-DSP serial approach. (e) The trade-off between serial and
parallel implementations.

\textbf{Solution:}

\begin{enumerate}
\def\labelenumi{(\alph{enumi})}
\item
  Clock cycles per sample: Cycles = f\textsubscript{clk} /
  f\textsubscript{sample} = 100 x 10⁶ / 48 x 10³ = \textbf{2,083 clock
  cycles per sample}
\item
  Serial approach with minimum DSP blocks: Each DSP block can perform
  2,083 MAC operations per sample period. The filter needs 64 MACs per
  sample. Since 64 \textless{} 2,083, a single DSP block can easily
  handle all 64 taps.
\end{enumerate}

Minimum DSP blocks: \textbf{1 DSP block} (performing 64 MACs
sequentially out of 2,083 available cycles)

DSP utilization: 64 / 2,083 = 3.1\%

\begin{enumerate}
\def\labelenumi{(\alph{enumi})}
\setcounter{enumi}{2}
\item
  Fully parallel: each tap gets its own DSP block, all operating
  simultaneously. DSP blocks needed: \textbf{64 DSP blocks} All MACs
  complete in 1 clock cycle (10 ns). The remaining 2,082 cycles are
  idle.
\item
  For a single DSP, minimum clock frequency to complete 64 MACs in one
  sample period: T\textsubscript{sample} = 1 / 48,000 = 20.833 μs
  f\textsubscript{min} = 64 / T\textsubscript{sample} = 64 / 20.833 x
  10⁻⁶ = \textbf{3.072 MHz}
\end{enumerate}

Since the DSP block delay is 4 ns, f\textsubscript{max} = 1/4 ns = 250
MHz, so 3.072 MHz is easily achievable.

\begin{enumerate}
\def\labelenumi{(\alph{enumi})}
\setcounter{enumi}{4}
\tightlist
\item
  Trade-off summary:
\end{enumerate}

{\def\LTcaptype{none} % do not increment counter
\begin{longtable}[]{@{}llll@{}}
\toprule\noalign{}
Approach & DSP Blocks & Latency (clocks) & Utilization \\
\midrule\noalign{}
\endhead
\bottomrule\noalign{}
\endlastfoot
Serial (1 DSP) & 1 & 64 cycles = 640 ns & 3.1\% \\
Parallel (64 DSPs) & 64 & 1 cycle = 10 ns & 0.048\% \\
Semi-parallel (4 DSPs) & 4 & 16 cycles = 160 ns & 0.77\% \\
\end{longtable}
}

\textbf{Serial} uses minimal resources but has higher latency (640 ns,
still far below the 20.8 μs sample period). \textbf{Parallel} has the
lowest latency but uses 64x more DSP resources. A \textbf{semi-parallel}
approach (e.g., 4 DSPs each processing 16 taps) balances resource usage
and latency. For audio at 48 kHz, the serial approach is optimal because
the sample period is so long relative to the processing time that
resource efficiency far outweighs the negligible latency difference.

\chapter{Chapter 7 --- Section 7.1: Electric Charge and
Current}\label{chapter-7-section-7.1-electric-charge-and-current}

Practice problems covering electric charge, current, voltage, energy,
drift velocity, and power relationships.

\begin{center}\rule{0.5\linewidth}{0.5pt}\end{center}

\section{Problem 7.1.1}\label{problem-7.1.1}

\textbf{Given:} A lightning bolt transfers 5 C of charge from a cloud to
the ground in 0.001 seconds.

\textbf{Find:} (a) The average current during the lightning strike. (b)
The number of electrons transferred.

\textbf{Solution:}

\begin{enumerate}
\def\labelenumi{(\alph{enumi})}
\tightlist
\item
  Current is the rate of charge flow: I = Q / t
\end{enumerate}

I = 5 C / 0.001 s = \textbf{5,000 A (5 kA)}

\begin{enumerate}
\def\labelenumi{(\alph{enumi})}
\setcounter{enumi}{1}
\tightlist
\item
  Number of electrons: N = Q / e
\end{enumerate}

N = 5 / (1.602 x 10\textsuperscript{-19}) = \textbf{3.12 x
10\textsuperscript{19} electrons}

\begin{center}\rule{0.5\linewidth}{0.5pt}\end{center}

\section{Problem 7.1.2}\label{problem-7.1.2}

\textbf{Given:} An aluminum conductor has a cross-sectional area of 53.5
mm\textsuperscript{2} (1/0 AWG). Aluminum has a free electron density of
n = 18.1 x 10\textsuperscript{28} electrons/m\textsuperscript{3}. The
conductor carries a current of 100 A.

\textbf{Find:} (a) The drift velocity of the electrons. (b) The time it
takes an electron to travel 1 meter along the conductor.

\textbf{Solution:}

\begin{enumerate}
\def\labelenumi{(\alph{enumi})}
\tightlist
\item
  Drift velocity: v\textsubscript{d} = I / (n x A x e)
\end{enumerate}

v\textsubscript{d} = 100 / (18.1 x 10\textsuperscript{28} x 53.5 x
10\textsuperscript{-6} x 1.602 x 10\textsuperscript{-19})

v\textsubscript{d} = 100 / (18.1 x 10\textsuperscript{28} x 53.5 x
10\textsuperscript{-6} x 1.602 x 10\textsuperscript{-19})

Denominator = 18.1 x 53.5 x 1.602 x 10\textsuperscript{28-6-19} = 1551.3
x 10\textsuperscript{3} = 1.551 x 10\textsuperscript{6}

v\textsubscript{d} = 100 / 1.551 x 10\textsuperscript{6} = \textbf{6.45
x 10\textsuperscript{-5} m/s = 0.0645 mm/s}

\begin{enumerate}
\def\labelenumi{(\alph{enumi})}
\setcounter{enumi}{1}
\tightlist
\item
  Time to travel 1 meter: t = d / v\textsubscript{d}
\end{enumerate}

t = 1 / 6.45 x 10\textsuperscript{-5} = \textbf{15,504 s = 4.31 hours}

\begin{center}\rule{0.5\linewidth}{0.5pt}\end{center}

\section{Problem 7.1.3}\label{problem-7.1.3}

\textbf{Given:} A smartphone battery is rated at 3,000 mAh at a nominal
voltage of 3.7 V. The phone draws an average current of 250 mA during
normal use.

\textbf{Find:} (a) The total charge stored in the battery in coulombs.
(b) The total energy stored in watt-hours and joules. (c) The expected
battery life. (d) The average power consumption.

\textbf{Solution:}

\begin{enumerate}
\def\labelenumi{(\alph{enumi})}
\tightlist
\item
  Total charge: Q = I x t = capacity in Ah x 3600 s/h
\end{enumerate}

Q = 3.0 Ah x 3600 = \textbf{10,800 C}

\begin{enumerate}
\def\labelenumi{(\alph{enumi})}
\setcounter{enumi}{1}
\tightlist
\item
  Energy: W = capacity x voltage = 3.0 Ah x 3.7 V = \textbf{11.1 Wh}
\end{enumerate}

In joules: W = 11.1 x 3600 = \textbf{39,960 J = 40.0 kJ}

\begin{enumerate}
\def\labelenumi{(\alph{enumi})}
\setcounter{enumi}{2}
\item
  Battery life: t = capacity / I = 3,000 mAh / 250 mA = \textbf{12
  hours}
\item
  Average power: P = V x I = 3.7 x 0.250 = \textbf{0.925 W}
\end{enumerate}

\begin{center}\rule{0.5\linewidth}{0.5pt}\end{center}

\section{Problem 7.1.4}\label{problem-7.1.4}

\textbf{Given:} A residential electric water heater has two 4,500 W
heating elements operating at 240 V. The utility charges \$0.12 per kWh.
The heater runs an average of 3 hours per day.

\textbf{Find:} (a) The current drawn by one heating element. (b) The
resistance of one heating element. (c) The daily energy consumption in
kWh. (d) The monthly energy cost (30 days).

\textbf{Solution:}

\begin{enumerate}
\def\labelenumi{(\alph{enumi})}
\item
  Current per element: I = P / V = 4,500 / 240 = \textbf{18.75 A}
\item
  Resistance: R = V\textsuperscript{2} / P = 240\textsuperscript{2} /
  4,500 = 57,600 / 4,500 = \textbf{12.8 ohm}
\item
  Daily energy (assuming one element operates at a time, total on-time 3
  hours):
\end{enumerate}

W = P x t = 4.5 kW x 3 h = \textbf{13.5 kWh per day}

\begin{enumerate}
\def\labelenumi{(\alph{enumi})}
\setcounter{enumi}{3}
\tightlist
\item
  Monthly cost: Cost = 13.5 x 30 x \$0.12 = \textbf{\$48.60 per month}
\end{enumerate}

\begin{center}\rule{0.5\linewidth}{0.5pt}\end{center}

\section{Problem 7.1.5}\label{problem-7.1.5}

\textbf{Given:} A solar panel produces 8.5 A at a terminal voltage of 32
V under full sun conditions. The panel is connected to a charge
controller and battery system.

\textbf{Find:} (a) The power output of the panel. (b) The total energy
produced during 5 peak sun hours. (c) The total charge delivered to the
battery in one day of 5 peak sun hours.

\textbf{Solution:}

\begin{enumerate}
\def\labelenumi{(\alph{enumi})}
\item
  Power: P = V x I = 32 x 8.5 = \textbf{272 W}
\item
  Energy: W = P x t = 272 W x 5 h = \textbf{1,360 Wh = 1.36 kWh}
\item
  Charge: Q = I x t = 8.5 A x 5 x 3600 s = \textbf{153,000 C = 153 kC}
\end{enumerate}

Equivalently, Q = 8.5 A x 5 h = \textbf{42.5 Ah}

\begin{center}\rule{0.5\linewidth}{0.5pt}\end{center}

\chapter{Chapter 7 --- Section 7.2: Passive
Components}\label{chapter-7-section-7.2-passive-components}

Practice problems covering resistors, capacitors, inductors, and
wire/cable characteristics.

\begin{center}\rule{0.5\linewidth}{0.5pt}\end{center}

\section{Problem 7.2.1}\label{problem-7.2.1}

\textbf{Given:} A precision voltage divider uses a 10 kohm metal film
resistor (tolerance +/-0.1\%, temperature coefficient +/-50 ppm/degC).
The resistor operates at 25 degC in a lab environment, but the circuit
must also function at 75 degC.

\textbf{Find:} (a) The resistance range at 25 degC due to tolerance. (b)
The resistance change due to temperature at 75 degC. (c) The total
worst-case resistance at 75 degC. (d) The power dissipated if 15 V is
applied across the resistor.

\textbf{Solution:}

\begin{enumerate}
\def\labelenumi{(\alph{enumi})}
\tightlist
\item
  Tolerance range at 25 degC:
\end{enumerate}

R\textsubscript{min} = 10,000 x (1 - 0.001) = \textbf{9,990 ohm}

R\textsubscript{max} = 10,000 x (1 + 0.001) = \textbf{10,010 ohm}

\begin{enumerate}
\def\labelenumi{(\alph{enumi})}
\setcounter{enumi}{1}
\tightlist
\item
  Temperature change: deltaT = 75 - 25 = 50 degC
\end{enumerate}

deltaR = R x TC x deltaT = 10,000 x 50 x 10\textsuperscript{-6} x 50 =
\textbf{25 ohm}

\begin{enumerate}
\def\labelenumi{(\alph{enumi})}
\setcounter{enumi}{2}
\tightlist
\item
  Worst-case resistance at 75 degC:
\end{enumerate}

R\textsubscript{max,total} = 10,010 + 25 = \textbf{10,035 ohm}

R\textsubscript{min,total} = 9,990 - 25 = \textbf{9,965 ohm}

Total uncertainty: +/-35 ohm or +/-0.35\%

\begin{enumerate}
\def\labelenumi{(\alph{enumi})}
\setcounter{enumi}{3}
\tightlist
\item
  Power: P = V\textsuperscript{2} / R = 15\textsuperscript{2} / 10,000 =
  225 / 10,000 = \textbf{22.5 mW}
\end{enumerate}

\begin{center}\rule{0.5\linewidth}{0.5pt}\end{center}

\section{Problem 7.2.2}\label{problem-7.2.2}

\textbf{Given:} A timing circuit uses a 47 uF aluminum electrolytic
capacitor (ESR = 0.8 ohm, voltage rating 25 V) charged to 12 V.

\textbf{Find:} (a) The energy stored in the capacitor. (b) If the
capacitor is discharged through a 100 ohm load, the peak discharge
current. (c) The peak power dissipated in the ESR during discharge. (d)
If the capacitor must hold its charge for 10 minutes with less than 1 V
droop, the minimum allowable leakage resistance.

\textbf{Solution:}

\begin{enumerate}
\def\labelenumi{(\alph{enumi})}
\item
  Energy: W = 1/2 x C x V\textsuperscript{2} = 0.5 x 47 x
  10\textsuperscript{-6} x 12\textsuperscript{2} = 0.5 x 47 x
  10\textsuperscript{-6} x 144 = \textbf{3.38 mJ}
\item
  Peak discharge current (total resistance = R\textsubscript{load} +
  ESR):
\end{enumerate}

I\textsubscript{peak} = V / (R\textsubscript{load} + ESR) = 12 / (100 +
0.8) = 12 / 100.8 = \textbf{119 mA}

\begin{enumerate}
\def\labelenumi{(\alph{enumi})}
\setcounter{enumi}{2}
\item
  Peak power in ESR: P\textsubscript{ESR} =
  I\textsubscript{peak}\textsuperscript{2} x ESR =
  0.119\textsuperscript{2} x 0.8 = 0.01416 x 0.8 = \textbf{11.3 mW}
\item
  For voltage droop \textless{} 1 V in 10 minutes, using V(t) =
  V\textsubscript{0} x e\textsuperscript{-t/(Rleak x C)}:
\end{enumerate}

11 = 12 x e\textsuperscript{-600/(Rleak x 47 x 10-6})

e\textsuperscript{-600/(Rleak x 47 x 10-6}) = 11/12 = 0.9167

-600 / (R\textsubscript{leak} x 47 x 10\textsuperscript{-6}) =
ln(0.9167) = -0.0870

R\textsubscript{leak} = 600 / (0.0870 x 47 x 10\textsuperscript{-6}) =
600 / (4.089 x 10\textsuperscript{-6}) = \textbf{146.7 Mohm}

\begin{center}\rule{0.5\linewidth}{0.5pt}\end{center}

\section{Problem 7.2.3}\label{problem-7.2.3}

\textbf{Given:} A switching power supply uses a 22 uH toroidal inductor
with a DC resistance of 0.015 ohm. The inductor operates at 250 kHz with
a DC current of 5 A and a ripple current of 1.5 A peak-to-peak.

\textbf{Find:} (a) The peak inductor current. (b) The energy stored at
peak current. (c) The quality factor Q at the switching frequency. (d)
The minimum saturation current rating needed with a 20\% safety margin.
(e) The minimum self-resonant frequency (SRF) for reliable operation.

\textbf{Solution:}

\begin{enumerate}
\def\labelenumi{(\alph{enumi})}
\item
  Peak current: I\textsubscript{peak} = I\textsubscript{DC} + deltaI/2 =
  5 + 1.5/2 = 5 + 0.75 = \textbf{5.75 A}
\item
  Energy at peak: W = 1/2 x L x I\textsubscript{peak}\textsuperscript{2}
  = 0.5 x 22 x 10\textsuperscript{-6} x 5.75\textsuperscript{2} = 0.5 x
  22 x 10\textsuperscript{-6} x 33.06 = \textbf{363.7 uJ}
\item
  Quality factor: Q = omega x L / R\textsubscript{DC}
\end{enumerate}

omega = 2 x pi x 250,000 = 1,570,796 rad/s

Q = 1,570,796 x 22 x 10\textsuperscript{-6} / 0.015 = 34.56 / 0.015 =
\textbf{2,304}

\begin{enumerate}
\def\labelenumi{(\alph{enumi})}
\setcounter{enumi}{3}
\item
  Minimum saturation rating: I\textsubscript{sat} \textgreater= 1.2 x
  I\textsubscript{peak} = 1.2 x 5.75 = \textbf{6.9 A}
\item
  Minimum SRF \textgreater= 3 x f\textsubscript{sw} = 3 x 250 kHz =
  \textbf{750 kHz}
\end{enumerate}

\begin{center}\rule{0.5\linewidth}{0.5pt}\end{center}

\section{Problem 7.2.4}\label{problem-7.2.4}

\textbf{Given:} A 240 V, 30 A branch circuit uses 10 AWG copper
conductors (resistance = 3.277 ohm/km) in a 200-foot (61 m) run from the
panel to a workshop subpanel.

\textbf{Find:} (a) The one-way conductor resistance. (b) The total
voltage drop at full load (single-phase, accounting for both
conductors). (c) The voltage drop as a percentage. (d) Whether it meets
the NEC 3\% recommendation for feeders. (e) What conductor size would
meet the 3\% recommendation.

\textbf{Solution:}

\begin{enumerate}
\def\labelenumi{(\alph{enumi})}
\item
  One-way resistance: R = 3.277 ohm/km x 0.061 km = \textbf{0.200 ohm}
\item
  Voltage drop (both conductors): V\textsubscript{drop} = 2 x I x R = 2
  x 30 x 0.200 = \textbf{12.0 V}
\item
  Percentage: V\textsubscript{drop}\% = 12.0 / 240 x 100 =
  \textbf{5.0\%}
\item
  The NEC recommends \textless= 3\% for feeders. At 5.0\%, this circuit
  \textbf{does not meet} the recommendation. The maximum allowable drop
  is 240 x 0.03 = 7.2 V.
\item
  Required maximum resistance per conductor: R\textsubscript{max} = 7.2
  / (2 x 30) = 0.12 ohm
\end{enumerate}

Required resistance per km: 0.12 / 0.061 = 1.967 ohm/km

From the AWG table: 6 AWG has 1.296 ohm/km. Check: V\textsubscript{drop}
= 2 x 30 x 1.296 x 0.061 = \textbf{4.74 V = 1.97\%}

\textbf{6 AWG copper} meets the 3\% recommendation.

\begin{center}\rule{0.5\linewidth}{0.5pt}\end{center}

\section{Problem 7.2.5}\label{problem-7.2.5}

\textbf{Given:} A DC motor circuit requires a 0.1 ohm, 50 W wirewound
power resistor for current sensing. The resistor has a temperature
coefficient of +30 ppm/degC and operates in an ambient of 40 degC. At
full power, the resistor surface temperature rises by 80 degC above
ambient.

\textbf{Find:} (a) The current at which the resistor dissipates its full
rated power. (b) The voltage across the resistor at full power. (c) The
resistance shift at full operating temperature. (d) The actual
current-sense voltage error percentage due to resistance change.

\textbf{Solution:}

\begin{enumerate}
\def\labelenumi{(\alph{enumi})}
\item
  Current: P = I\textsuperscript{2} x R, so I = sqrt(P/R) = sqrt(50/0.1)
  = sqrt(500) = \textbf{22.36 A}
\item
  Voltage: V = I x R = 22.36 x 0.1 = \textbf{2.236 V}
\item
  Operating temperature: T = 40 + 80 = 120 degC. Temperature rise above
  reference (25 degC): deltaT = 120 - 25 = 95 degC.
\end{enumerate}

deltaR = R x TC x deltaT = 0.1 x 30 x 10\textsuperscript{-6} x 95 =
\textbf{0.000285 ohm}

R\textsubscript{hot} = 0.1 + 0.000285 = \textbf{0.100285 ohm}

\begin{enumerate}
\def\labelenumi{(\alph{enumi})}
\setcounter{enumi}{3}
\tightlist
\item
  Voltage error: The sense voltage at the same current would be
  V\textsubscript{hot} = 22.36 x 0.100285 = 2.2424 V.
\end{enumerate}

Error = (0.100285 - 0.1) / 0.1 x 100 = \textbf{0.285\%}

This error may be acceptable for motor current sensing but would need
compensation for precision measurements.

\begin{center}\rule{0.5\linewidth}{0.5pt}\end{center}

\section{Problem 7.2.6}\label{problem-7.2.6}

\textbf{Given:} A decoupling network uses a 100 nF ceramic capacitor
(X7R, rated 50 V) in parallel with a 10 uF tantalum capacitor (ESR = 1.5
ohm) for a 3.3 V digital supply. The X7R capacitor loses 40\% of its
capacitance at its rated voltage. A load transient of 500 mA occurs.

\textbf{Find:} (a) The effective capacitance of the ceramic capacitor at
3.3 V and at 50 V. (b) The ESR voltage spike from the tantalum alone
during the load transient. (c) The total capacitance available for
energy storage at 3.3 V. (d) The voltage droop if the regulator takes 20
us to respond.

\textbf{Solution:}

\begin{enumerate}
\def\labelenumi{(\alph{enumi})}
\tightlist
\item
  At 3.3 V (well below the 50 V rating), derating is minimal:
  C\textsubscript{eff} = \textbf{100 nF} (essentially full capacitance).
\end{enumerate}

At 50 V (rated voltage), C\textsubscript{eff} = 100 x (1 - 0.40) =
\textbf{60 nF}

\begin{enumerate}
\def\labelenumi{(\alph{enumi})}
\setcounter{enumi}{1}
\item
  ESR spike from tantalum: deltaV = I x ESR = 0.5 x 1.5 = \textbf{0.75
  V} (22.7\% of 3.3 V -- unacceptable alone)
\item
  Total capacitance at 3.3 V: C\textsubscript{total} = 100 nF + 10 uF =
  10 uF + 0.1 uF = \textbf{10.1 uF}
\item
  Voltage droop: deltaV = I x dt / C = 0.5 x 20 x 10\textsuperscript{-6}
  / (10.1 x 10\textsuperscript{-6}) = \textbf{0.99 V}
\end{enumerate}

This 30\% droop on a 3.3 V rail is excessive. Additional bulk
capacitance or a faster regulator is needed.

\begin{center}\rule{0.5\linewidth}{0.5pt}\end{center}

\chapter{Chapter 7 --- Section 7.3: Fundamental
Laws}\label{chapter-7-section-7.3-fundamental-laws}

Practice problems covering Ohm's Law, Kirchhoff's Voltage Law (KVL), and
Kirchhoff's Current Law (KCL).

\begin{center}\rule{0.5\linewidth}{0.5pt}\end{center}

\section{Problem 7.3.1}\label{problem-7.3.1}

\textbf{Given:} A heating element in a toaster oven draws 10 A from a
120 V supply.

\textbf{Find:} (a) The resistance of the heating element. (b) The power
dissipated. (c) If the supply voltage drops to 108 V (a 10\% sag), the
new current and power. (d) The percentage reduction in heating power
during the voltage sag.

\textbf{Solution:}

\begin{enumerate}
\def\labelenumi{(\alph{enumi})}
\item
  Resistance: R = V / I = 120 / 10 = \textbf{12 ohm}
\item
  Power: P = V x I = 120 x 10 = \textbf{1,200 W}
\item
  At 108 V (resistance assumed constant):
\end{enumerate}

I\textsubscript{new} = 108 / 12 = \textbf{9 A}

P\textsubscript{new} = V\textsuperscript{2} / R = 108\textsuperscript{2}
/ 12 = 11,664 / 12 = \textbf{972 W}

\begin{enumerate}
\def\labelenumi{(\alph{enumi})}
\setcounter{enumi}{3}
\tightlist
\item
  Percentage reduction: (1200 - 972) / 1200 x 100 = \textbf{19\%}
\end{enumerate}

A 10\% voltage reduction causes a 19\% power reduction because power is
proportional to V\textsuperscript{2}.

\begin{center}\rule{0.5\linewidth}{0.5pt}\end{center}

\section{Problem 7.3.2}\label{problem-7.3.2}

\textbf{Given:} A series circuit contains a 36 V battery, a 1.5 kohm
resistor R₁, a 2.2 kohm resistor R₂, and a 3.3 kohm resistor R₃.

\textbf{Find:} (a) The total series resistance. (b) The loop current.
(c) The voltage drop across each resistor. (d) Verify KVL around the
loop.

\textbf{Solution:}

\begin{enumerate}
\def\labelenumi{(\alph{enumi})}
\item
  R\textsubscript{total} = R₁ + R₂ + R₃ = 1,500 + 2,200 + 3,300 =
  \textbf{7,000 ohm = 7 kohm}
\item
  Loop current: I = V / R\textsubscript{total} = 36 / 7,000 =
  \textbf{5.143 mA}
\item
  Voltage drops:
\end{enumerate}

V₁ = I x R₁ = 5.143 x 10\textsuperscript{-3} x 1,500 = \textbf{7.714 V}

V₂ = I x R₂ = 5.143 x 10\textsuperscript{-3} x 2,200 = \textbf{11.314 V}

V₃ = I x R₃ = 5.143 x 10\textsuperscript{-3} x 3,300 = \textbf{16.971 V}

\begin{enumerate}
\def\labelenumi{(\alph{enumi})}
\setcounter{enumi}{3}
\tightlist
\item
  KVL check: V₁ + V₂ + V₃ = 7.714 + 11.314 + 16.971 = \textbf{36.0 V =
  V\textsubscript{source}} (verified)
\end{enumerate}

\begin{center}\rule{0.5\linewidth}{0.5pt}\end{center}

\section{Problem 7.3.3}\label{problem-7.3.3}

\textbf{Given:} At a node in a circuit, five branches meet. Branch 1
carries 2.5 A into the node. Branch 2 carries 1.8 A out of the node.
Branch 3 carries 3.2 A into the node. Branch 4 carries 0.7 A out of the
node.

\textbf{Find:} (a) The current in branch 5 (magnitude and direction).
(b) Verify KCL at the node.

\textbf{Solution:}

\begin{enumerate}
\def\labelenumi{(\alph{enumi})}
\tightlist
\item
  By KCL, the sum of currents entering a node equals the sum of currents
  leaving:
\end{enumerate}

I\textsubscript{in} = I₁ + I₃ = 2.5 + 3.2 = 5.7 A

I\textsubscript{out} = I₂ + I₄ = 1.8 + 0.7 = 2.5 A

I₅ = I\textsubscript{in} - I\textsubscript{out} = 5.7 - 2.5 =
\textbf{3.2 A leaving the node}

\begin{enumerate}
\def\labelenumi{(\alph{enumi})}
\setcounter{enumi}{1}
\tightlist
\item
  KCL check: Total in = I₁ + I₃ = 2.5 + 3.2 = 5.7 A. Total out = I₂ + I₄
  + I₅ = 1.8 + 0.7 + 3.2 = 5.7 A. \textbf{Verified.}
\end{enumerate}

\begin{center}\rule{0.5\linewidth}{0.5pt}\end{center}

\section{Problem 7.3.4}\label{problem-7.3.4}

\textbf{Given:} A series-parallel circuit is powered by a 48 V source. A
200 ohm resistor R₁ is in series with the source. Two parallel branches
connect from the junction of R₁ to ground: Branch A has a 600 ohm
resistor R₂, and Branch B has a 300 ohm resistor R₃ in series with a 100
ohm resistor R₄.

\textbf{Find:} (a) The equivalent resistance of the parallel
combination. (b) The total circuit resistance. (c) The source current.
(d) The voltage at the junction node (after R₁). (e) The current through
each parallel branch.

\textbf{Solution:}

\begin{enumerate}
\def\labelenumi{(\alph{enumi})}
\tightlist
\item
  Branch B resistance: R\textsubscript{B} = R₃ + R₄ = 300 + 100 = 400
  ohm
\end{enumerate}

Parallel combination: R\textsubscript{parallel} = (R₂ x
R\textsubscript{B}) / (R₂ + R\textsubscript{B}) = (600 x 400) / (600 +
400) = 240,000 / 1,000 = \textbf{240 ohm}

\begin{enumerate}
\def\labelenumi{(\alph{enumi})}
\setcounter{enumi}{1}
\item
  Total resistance: R\textsubscript{total} = R₁ +
  R\textsubscript{parallel} = 200 + 240 = \textbf{440 ohm}
\item
  Source current: I\textsubscript{source} = 48 / 440 = \textbf{109.1 mA}
\item
  Voltage at junction: V\textsubscript{junction} =
  I\textsubscript{source} x R\textsubscript{parallel} = 0.1091 x 240 =
  \textbf{26.18 V}
\end{enumerate}

Or: V\textsubscript{junction} = 48 - I\textsubscript{source} x R₁ = 48 -
0.1091 x 200 = 48 - 21.82 = \textbf{26.18 V}

\begin{enumerate}
\def\labelenumi{(\alph{enumi})}
\setcounter{enumi}{4}
\tightlist
\item
  Branch currents:
\end{enumerate}

I\textsubscript{A} = V\textsubscript{junction} / R₂ = 26.18 / 600 =
\textbf{43.6 mA}

I\textsubscript{B} = V\textsubscript{junction} / R\textsubscript{B} =
26.18 / 400 = \textbf{65.5 mA}

Check: I\textsubscript{A} + I\textsubscript{B} = 43.6 + 65.5 =
\textbf{109.1 mA = I\textsubscript{source}} (verified)

\begin{center}\rule{0.5\linewidth}{0.5pt}\end{center}

\section{Problem 7.3.5}\label{problem-7.3.5}

\textbf{Given:} An LED circuit uses a red LED (forward voltage
V\textsubscript{LED} = 2.0 V, desired current I\textsubscript{LED} = 20
mA) powered by a 5 V supply.

\textbf{Find:} (a) The required current-limiting resistor value. (b) The
power dissipated by the resistor. (c) The power dissipated by the LED.
(d) The minimum resistor power rating.

\textbf{Solution:}

\begin{enumerate}
\def\labelenumi{(\alph{enumi})}
\tightlist
\item
  By KVL: V\textsubscript{supply} = V\textsubscript{R} +
  V\textsubscript{LED}
\end{enumerate}

V\textsubscript{R} = 5 - 2.0 = 3.0 V

R = V\textsubscript{R} / I = 3.0 / 0.020 = \textbf{150 ohm}

\begin{enumerate}
\def\labelenumi{(\alph{enumi})}
\setcounter{enumi}{1}
\item
  Power in resistor: P\textsubscript{R} = V\textsubscript{R} x I = 3.0 x
  0.020 = \textbf{60 mW}
\item
  Power in LED: P\textsubscript{LED} = V\textsubscript{LED} x I = 2.0 x
  0.020 = \textbf{40 mW}
\item
  A standard \textbf{1/8 W (125 mW)} resistor provides a 2:1 derating
  margin and is the minimum standard rating above 60 mW.
\end{enumerate}

Check total power: P\textsubscript{total} = P\textsubscript{R} +
P\textsubscript{LED} = 60 + 40 = 100 mW = V\textsubscript{supply} x I =
5 x 0.020 = 100 mW (verified).

\begin{center}\rule{0.5\linewidth}{0.5pt}\end{center}

\section{Problem 7.3.6}\label{problem-7.3.6}

\textbf{Given:} Three 120 V branch circuits share a common neutral in a
residential panel (single-phase, 120/240 V system). Circuit A draws 12
A, Circuit B draws 8 A, and Circuit C draws 15 A. Circuits A and C are
on phase A (same hot leg), and Circuit B is on phase B (opposite hot
leg).

\textbf{Find:} (a) The neutral current if all three circuits are active
simultaneously. (b) Verify using KCL at the neutral bus.

\textbf{Solution:}

\begin{enumerate}
\def\labelenumi{(\alph{enumi})}
\tightlist
\item
  In a 120/240 V single-phase system, currents on opposite phases
  partially cancel on the neutral.
\end{enumerate}

Phase A contribution to neutral: I\textsubscript{A-phase} =
I\textsubscript{A} + I\textsubscript{C} = 12 + 15 = 27 A

Phase B contribution to neutral: I\textsubscript{B-phase} = 8 A

Neutral current: I\textsubscript{N} = \textbar I\textsubscript{A-phase}
- I\textsubscript{B-phase}\textbar{} = \textbar27 - 8\textbar{} =
\textbf{19 A}

\begin{enumerate}
\def\labelenumi{(\alph{enumi})}
\setcounter{enumi}{1}
\tightlist
\item
  KCL verification at the neutral bus: The neutral carries the
  unbalanced current between the two phases. Current flowing from phase
  A loads into the neutral = 27 A. Current flowing from neutral to phase
  B loads = 8 A. Net neutral current returning to the transformer = 27 -
  8 = \textbf{19 A} (verified).
\end{enumerate}

\begin{center}\rule{0.5\linewidth}{0.5pt}\end{center}

\chapter{Chapter 7 --- Section 7.4: DC Circuit
Analysis}\label{chapter-7-section-7.4-dc-circuit-analysis}

Practice problems covering series circuits, parallel circuits, and
series-parallel combinations.

\begin{center}\rule{0.5\linewidth}{0.5pt}\end{center}

\section{Problem 7.4.1}\label{problem-7.4.1}

\textbf{Given:} A series string of five identical 1 kohm resistors is
connected across a 25 V supply. A voltmeter is connected across the
third resistor in the string.

\textbf{Find:} (a) The total resistance. (b) The series current. (c) The
voltmeter reading using the voltage divider rule. (d) The power
dissipated by each resistor and the total power.

\textbf{Solution:}

\begin{enumerate}
\def\labelenumi{(\alph{enumi})}
\item
  R\textsubscript{total} = 5 x 1,000 = \textbf{5,000 ohm = 5 kohm}
\item
  Series current: I = V / R\textsubscript{total} = 25 / 5,000 =
  \textbf{5 mA}
\item
  Voltage across one resistor: V₃ = V\textsubscript{source} x (R₃ /
  R\textsubscript{total}) = 25 x (1,000 / 5,000) = 25 x 0.2 = \textbf{5
  V}
\item
  Power per resistor: P = I\textsuperscript{2} x R =
  (0.005)\textsuperscript{2} x 1,000 = 25 x 10\textsuperscript{-6} x
  1,000 = \textbf{25 mW}
\end{enumerate}

Total power: P\textsubscript{total} = 5 x 25 mW = \textbf{125 mW}

Check: P\textsubscript{total} = V\textsuperscript{2} /
R\textsubscript{total} = 625 / 5,000 = 125 mW (verified).

\begin{center}\rule{0.5\linewidth}{0.5pt}\end{center}

\section{Problem 7.4.2}\label{problem-7.4.2}

\textbf{Given:} Four resistors are connected in parallel across a 24 V
source: R₁ = 100 ohm, R₂ = 220 ohm, R₃ = 470 ohm, R₄ = 1 kohm.

\textbf{Find:} (a) The total parallel resistance. (b) The total current
from the source. (c) The current through each branch. (d) The total
power consumed.

\textbf{Solution:}

\begin{enumerate}
\def\labelenumi{(\alph{enumi})}
\tightlist
\item
  1/R\textsubscript{total} = 1/100 + 1/220 + 1/470 + 1/1000
\end{enumerate}

1/R\textsubscript{total} = 0.01000 + 0.004545 + 0.002128 + 0.001000 =
0.017673 S

R\textsubscript{total} = 1 / 0.017673 = \textbf{56.58 ohm}

\begin{enumerate}
\def\labelenumi{(\alph{enumi})}
\setcounter{enumi}{1}
\item
  Total current: I\textsubscript{total} = 24 / 56.58 = \textbf{424.2 mA}
\item
  Branch currents:
\end{enumerate}

I₁ = 24 / 100 = \textbf{240 mA}

I₂ = 24 / 220 = \textbf{109.1 mA}

I₃ = 24 / 470 = \textbf{51.1 mA}

I₄ = 24 / 1000 = \textbf{24.0 mA}

Check: 240 + 109.1 + 51.1 + 24.0 = 424.2 mA (verified).

\begin{enumerate}
\def\labelenumi{(\alph{enumi})}
\setcounter{enumi}{3}
\tightlist
\item
  Total power: P = V x I\textsubscript{total} = 24 x 0.4242 =
  \textbf{10.18 W}
\end{enumerate}

\begin{center}\rule{0.5\linewidth}{0.5pt}\end{center}

\section{Problem 7.4.3}\label{problem-7.4.3}

\textbf{Given:} A circuit consists of a 9 V battery with an internal
resistance of 0.5 ohm, connected to an external network. The external
network has R₁ = 10 ohm in series with a parallel combination of R₂ = 30
ohm and R₃ = 60 ohm.

\textbf{Find:} (a) The equivalent external resistance. (b) The total
circuit resistance. (c) The current drawn from the battery. (d) The
terminal voltage of the battery (voltage across the external network).
(e) The current through R₂ and R₃.

\textbf{Solution:}

\begin{enumerate}
\def\labelenumi{(\alph{enumi})}
\tightlist
\item
  Parallel combination: R\textsubscript{23} = (30 x 60) / (30 + 60) =
  1,800 / 90 = 20 ohm
\end{enumerate}

External resistance: R\textsubscript{ext} = R₁ + R\textsubscript{23} =
10 + 20 = \textbf{30 ohm}

\begin{enumerate}
\def\labelenumi{(\alph{enumi})}
\setcounter{enumi}{1}
\item
  Total resistance: R\textsubscript{total} = R\textsubscript{int} +
  R\textsubscript{ext} = 0.5 + 30 = \textbf{30.5 ohm}
\item
  Current: I = 9 / 30.5 = \textbf{295.1 mA}
\item
  Terminal voltage: V\textsubscript{terminal} = V\textsubscript{battery}
  - I x R\textsubscript{int} = 9 - 0.2951 x 0.5 = 9 - 0.148 =
  \textbf{8.852 V}
\end{enumerate}

Or: V\textsubscript{terminal} = I x R\textsubscript{ext} = 0.2951 x 30 =
8.852 V (verified).

\begin{enumerate}
\def\labelenumi{(\alph{enumi})}
\setcounter{enumi}{4}
\tightlist
\item
  Voltage across parallel combination: V\textsubscript{23} = I x
  R\textsubscript{23} = 0.2951 x 20 = 5.902 V
\end{enumerate}

I₂ = V\textsubscript{23} / R₂ = 5.902 / 30 = \textbf{196.7 mA}

I₃ = V\textsubscript{23} / R₃ = 5.902 / 60 = \textbf{98.4 mA}

Check: I₂ + I₃ = 196.7 + 98.4 = 295.1 mA = I (verified).

\begin{center}\rule{0.5\linewidth}{0.5pt}\end{center}

\section{Problem 7.4.4}\label{problem-7.4.4}

\textbf{Given:} A Wheatstone bridge has R₁ = 1 kohm, R₂ = 2 kohm, R₃ =
1.5 kohm, and R₄ is unknown. The bridge is powered by a 10 V source.
When balanced, the galvanometer reads zero current.

\textbf{Find:} (a) The value of R₄ for a balanced bridge. (b) Verify the
balance condition. (c) If R₄ is actually 3.1 kohm (slightly unbalanced),
find the open-circuit voltage across the galvanometer terminals.

\textbf{Solution:}

\begin{enumerate}
\def\labelenumi{(\alph{enumi})}
\tightlist
\item
  Balance condition: R₁/R₂ = R₃/R₄
\end{enumerate}

R₄ = R₂ x R₃ / R₁ = 2,000 x 1,500 / 1,000 = \textbf{3,000 ohm = 3 kohm}

\begin{enumerate}
\def\labelenumi{(\alph{enumi})}
\setcounter{enumi}{1}
\item
  Check: R₁/R₂ = 1/2 = 0.5. R₃/R₄ = 1.5/3 = 0.5. \textbf{Balanced.}
\item
  With R₄ = 3.1 kohm, the voltage at each midpoint:
\end{enumerate}

V\textsubscript{A} = V\textsubscript{source} x R₂ / (R₁ + R₂) = 10 x
2,000 / (1,000 + 2,000) = 10 x 2/3 = 6.667 V

V\textsubscript{B} = V\textsubscript{source} x R₄ / (R₃ + R₄) = 10 x
3,100 / (1,500 + 3,100) = 10 x 3,100/4,600 = 6.739 V

V\textsubscript{galvanometer} = V\textsubscript{B} - V\textsubscript{A}
= 6.739 - 6.667 = \textbf{72.5 mV}

\begin{center}\rule{0.5\linewidth}{0.5pt}\end{center}

\section{Problem 7.4.5}\label{problem-7.4.5}

\textbf{Given:} A voltage divider supplies 3.3 V to a microcontroller
from a 12 V source. The microcontroller input has an input impedance of
10 Mohm and draws negligible current. The divider uses R₁ (upper,
connected to 12 V) and R₂ (lower, connected to ground) with R₂ = 10
kohm.

\textbf{Find:} (a) The required value of R₁. (b) The current through the
divider (quiescent current). (c) The power dissipated by the divider.
(d) If the microcontroller input impedance drops to 100 kohm (a
different MCU), the new output voltage.

\textbf{Solution:}

\begin{enumerate}
\def\labelenumi{(\alph{enumi})}
\tightlist
\item
  Voltage divider: V\textsubscript{out} = V\textsubscript{in} x R₂ / (R₁
  + R₂)
\end{enumerate}

3.3 = 12 x 10,000 / (R₁ + 10,000)

R₁ + 10,000 = 12 x 10,000 / 3.3 = 120,000 / 3.3 = 36,364

R₁ = 36,364 - 10,000 = \textbf{26,364 ohm} (use standard value 27 kohm)

With R₁ = 27 kohm: V\textsubscript{out} = 12 x 10,000 / (27,000 +
10,000) = 120,000 / 37,000 = \textbf{3.24 V}

\begin{enumerate}
\def\labelenumi{(\alph{enumi})}
\setcounter{enumi}{1}
\item
  Quiescent current: I\textsubscript{q} = 12 / (27,000 + 10,000) = 12 /
  37,000 = \textbf{324 uA}
\item
  Power: P = V\textsubscript{in} x I\textsubscript{q} = 12 x 0.000324 =
  \textbf{3.89 mW}
\item
  With 100 kohm load in parallel with R₂:
\end{enumerate}

R₂\textsubscript{eff} = (10,000 x 100,000) / (10,000 + 100,000) =
1,000,000,000 / 110,000 = 9,091 ohm

V\textsubscript{out} = 12 x 9,091 / (27,000 + 9,091) = 109,091 / 36,091
= \textbf{3.02 V}

The output drops from 3.24 V to 3.02 V due to loading, which may be
below the MCU minimum voltage.

\begin{center}\rule{0.5\linewidth}{0.5pt}\end{center}

\chapter{Chapter 7 --- Section 7.5: Analysis
Methods}\label{chapter-7-section-7.5-analysis-methods}

Practice problems covering nodal analysis, mesh analysis, superposition,
and dependent sources.

\begin{center}\rule{0.5\linewidth}{0.5pt}\end{center}

\section{Problem 7.5.1}\label{problem-7.5.1}

\textbf{Given:} A circuit has three non-reference nodes (V₁, V₂, V₃). A
5 mA current source feeds into node V₁. A 1 kohm resistor connects V₁ to
ground, a 2 kohm resistor connects V₁ to V₂, a 2 kohm resistor connects
V₂ to ground, a 3 kohm resistor connects V₂ to V₃, and a 1 kohm resistor
connects V₃ to ground.

\textbf{Find:} V₁, V₂, and V₃ using nodal analysis.

\textbf{Solution:} KCL at node V₁: 0.005 = V₁/1000 + (V₁ - V₂)/2000

Multiply by 2000: 10 = 2V₁ + V₁ - V₂ = 3V₁ - V₂ \ldots{} (1)

KCL at node V₂: (V₁ - V₂)/2000 = V₂/2000 + (V₂ - V₃)/3000

Multiply by 6000: 3(V₁ - V₂) = 3V₂ + 2(V₂ - V₃)

3V₁ - 3V₂ = 3V₂ + 2V₂ - 2V₃

3V₁ - 8V₂ + 2V₃ = 0 \ldots{} (2)

KCL at node V₃: (V₂ - V₃)/3000 = V₃/1000

Multiply by 3000: V₂ - V₃ = 3V₃, so V₂ = 4V₃ \ldots{} (3)

From (3): V₂ = 4V₃. Substitute into (2): 3V₁ - 32V₃ + 2V₃ = 0, so 3V₁ =
30V₃, V₁ = 10V₃ \ldots{} (4)

Substitute (3) and (4) into (1): 10 = 3(10V₃) - 4V₃ = 30V₃ - 4V₃ = 26V₃

V₃ = 10/26 = \textbf{0.385 V}

V₂ = 4 x 0.385 = \textbf{1.538 V}

V₁ = 10 x 0.385 = \textbf{3.846 V}

\begin{center}\rule{0.5\linewidth}{0.5pt}\end{center}

\section{Problem 7.5.2}\label{problem-7.5.2}

\textbf{Given:} A circuit has two meshes. The left mesh contains a 20 V
source and resistors R₁ = 4 kohm and R₂ = 8 kohm (shared). The right
mesh contains R₂ (shared), R₃ = 6 kohm, and a 10 V source. Both mesh
currents are defined clockwise.

\textbf{Find:} The mesh currents I₁ and I₂ using mesh analysis.

\textbf{Solution:} KVL for mesh 1 (clockwise): 20 = 4000I₁ + 8000(I₁ -
I₂)

20 = 12000I₁ - 8000I₂ \ldots{} (1)

KVL for mesh 2 (clockwise): -10 = 6000I₂ + 8000(I₂ - I₁)

-10 = -8000I₁ + 14000I₂ \ldots{} (2)

From (2): 8000I₁ = 14000I₂ + 10, so I₁ = 1.75I₂ + 0.00125

Substitute into (1): 20 = 12000(1.75I₂ + 0.00125) - 8000I₂

20 = 21000I₂ + 15 - 8000I₂

5 = 13000I₂

I₂ = 5/13000 = \textbf{0.385 mA}

I₁ = 1.75 x 0.000385 + 0.00125 = 0.000673 + 0.00125 = \textbf{1.923 mA}

Current through R₂: I\textsubscript{R₂} = I₁ - I₂ = 1.923 - 0.385 =
\textbf{1.538 mA} (flowing downward)

\begin{center}\rule{0.5\linewidth}{0.5pt}\end{center}

\section{Problem 7.5.3}\label{problem-7.5.3}

\textbf{Given:} A circuit has a 12 V voltage source on the left, a 4 mA
current source on the right, and three resistors: R₁ = 3 kohm (connected
from the voltage source to a central node), R₂ = 6 kohm (from the
central node to ground), and R₃ = 2 kohm (from the central node to the
current source, which connects to ground).

\textbf{Find:} The voltage at the central node using superposition.

\textbf{Solution:} \textbf{Source 1 alone (12 V active, current source
open):}

When the current source is opened, R₃ is effectively disconnected. R₂
connects the central node to ground, and R₁ connects it to the voltage
source:

V\textsubscript{node,1} = 12 x R₂ / (R₁ + R₂) = 12 x 6,000 / (3,000 +
6,000) = 12 x 2/3 = \textbf{8.0 V}

\textbf{Source 2 alone (4 mA active, voltage source shorted):}

With the 12 V source shorted, R₁ connects the node to ground. R₁ and R₂
are in parallel:

R\textsubscript{12} = (3,000 x 6,000) / (3,000 + 6,000) = 18,000,000 /
9,000 = 2,000 ohm

A current source in series with R₃ still forces 4 mA into the central
node. The node voltage is determined by the parallel resistance to
ground:

V\textsubscript{node,2} = 4 x 10\textsuperscript{-3} x (R₁
\textbar\textbar{} R₂) = 0.004 x 2,000 = \textbf{8.0 V}

\textbf{Superposition result:}

V\textsubscript{node} = V\textsubscript{node,1} +
V\textsubscript{node,2} = 8.0 + 8.0 = \textbf{16.0 V}

\begin{center}\rule{0.5\linewidth}{0.5pt}\end{center}

\section{Problem 7.5.4}\label{problem-7.5.4}

\textbf{Given:} A circuit contains a VCVS (voltage-controlled voltage
source). A 10 V independent source is in series with a 5 kohm resistor
R₁ connected to node A. A dependent voltage source V\textsubscript{x} =
3V\textsubscript{A} is in series with a 2 kohm resistor R₂, and this
branch connects node A to ground.

\textbf{Find:} The voltage V\textsubscript{A} at node A.

\textbf{Solution:} The branch from node A through the dependent source
(V\textsubscript{x} = 3V\textsubscript{A}, + at the node A side) and R₂
to ground:

I₂ = (V\textsubscript{A} - 3V\textsubscript{A}) / 2000 =
-2V\textsubscript{A} / 2000 = -V\textsubscript{A}/1000

Current entering node A from the 10 V source through R₁:

I₁ = (10 - V\textsubscript{A}) / 5000

KCL at node A: I₁ = I₂ (current entering = current leaving)

(10 - V\textsubscript{A}) / 5000 = -V\textsubscript{A} / 1000

Multiply by 5000: 10 - V\textsubscript{A} = -5V\textsubscript{A}

10 = -5V\textsubscript{A} + V\textsubscript{A} = -4V\textsubscript{A}

V\textsubscript{A} = 10 / (-4) = \textbf{-2.5 V}

The negative voltage indicates the dependent source forces a reversal of
the expected polarity. With V\textsubscript{A} = -2.5 V, the dependent
source produces V\textsubscript{x} = 3(-2.5) = -7.5 V.

\begin{center}\rule{0.5\linewidth}{0.5pt}\end{center}

\section{Problem 7.5.5}\label{problem-7.5.5}

\textbf{Given:} A circuit has three voltage sources (6 V, 12 V, and 18
V) and four resistors forming a two-mesh network. Mesh 1 (left): 6 V
source, R₁ = 2 kohm, R₂ = 4 kohm (shared). Mesh 2 (right): R₂ = 4 kohm
(shared), R₃ = 3 kohm, 12 V source. Additionally, an 18 V source is in
series with R₄ = 6 kohm between the top node (junction of R₁ and R₂) and
ground, forming a third branch.

\textbf{Find:} The voltage at the top node using superposition
(considering each source independently).

\textbf{Solution:} Let V\textsubscript{N} be the voltage at the top
node. Using nodal analysis with superposition:

\textbf{18 V source alone} (6 V and 12 V replaced by short circuits):

The 18 V source is in series with R₄ = 6 kohm. The other branches
present R₁ = 2 kohm and R₃ = 3 kohm (since the voltage sources are
shorted, R₂ connects directly through the shorted sources).

Wait -- R₂ is shared between the two meshes. With both 6 V and 12 V
sources shorted:

From the top node: R₁ to ground (through shorted 6 V), R₃ to ground
(through shorted 12 V), and the 18 V source with R₄.

R\textsubscript{parallel} = R₁ \textbar\textbar{} R₃ = (2000 x 3000) /
(2000 + 3000) = 6,000,000 / 5,000 = 1,200 ohm

V\textsubscript{N,18} = 18 x R\textsubscript{parallel} / (R₄ +
R\textsubscript{parallel}) = 18 x 1200 / (6000 + 1200) = 21,600 / 7,200
= \textbf{3.0 V}

\textbf{6 V source alone} (12 V and 18 V shorted):

From the top node: R₃ to ground (through shorted 12 V), R₄ to ground
(through shorted 18 V).

R\textsubscript{parallel} = R₃ \textbar\textbar{} R₄ = (3000 x 6000) /
(3000 + 6000) = 18,000,000 / 9,000 = 2,000 ohm

V\textsubscript{N,6} = 6 x R\textsubscript{parallel} / (R₁ +
R\textsubscript{parallel}) = 6 x 2000 / (2000 + 2000) = 12,000 / 4,000 =
\textbf{3.0 V}

\textbf{12 V source alone} (6 V and 18 V shorted):

From the top node: R₁ to ground (through shorted 6 V), R₄ to ground
(through shorted 18 V).

R\textsubscript{parallel} = R₁ \textbar\textbar{} R₄ = (2000 x 6000) /
(2000 + 6000) = 12,000,000 / 8,000 = 1,500 ohm

V\textsubscript{N,12} = 12 x R\textsubscript{parallel} / (R₃ +
R\textsubscript{parallel}) = 12 x 1500 / (3000 + 1500) = 18,000 / 4,500
= \textbf{4.0 V}

\textbf{Total by superposition:}

V\textsubscript{N} = V\textsubscript{N,18} + V\textsubscript{N,6} +
V\textsubscript{N,12} = 3.0 + 3.0 + 4.0 = \textbf{10.0 V}

\begin{center}\rule{0.5\linewidth}{0.5pt}\end{center}

\section{Problem 7.5.6}\label{problem-7.5.6}

\textbf{Given:} A MOSFET amplifier small-signal model contains a VCCS
(voltage-controlled current source) with g\textsubscript{m} = 10 mA/V.
The gate-to-source voltage V\textsubscript{gs} appears across a 50 kohm
input resistance. The drain current source
g\textsubscript{m}V\textsubscript{gs} drives into a 5 kohm drain
resistor R\textsubscript{D} in parallel with a 20 kohm load
R\textsubscript{L}.

\textbf{Find:} (a) The effective load resistance seen by the current
source. (b) The voltage gain A\textsubscript{v} =
V\textsubscript{out}/V\textsubscript{gs}. (c) If V\textsubscript{gs} =
0.1 V, the output voltage.

\textbf{Solution:}

\begin{enumerate}
\def\labelenumi{(\alph{enumi})}
\item
  Effective load: R\textsubscript{eff} = R\textsubscript{D}
  \textbar\textbar{} R\textsubscript{L} = (5,000 x 20,000) / (5,000 +
  20,000) = 100,000,000 / 25,000 = \textbf{4,000 ohm = 4 kohm}
\item
  Output voltage: V\textsubscript{out} = -g\textsubscript{m} x
  V\textsubscript{gs} x R\textsubscript{eff} (negative because drain
  current flows into the resistor, creating an inverting gain)
\end{enumerate}

A\textsubscript{v} = V\textsubscript{out} / V\textsubscript{gs} =
-g\textsubscript{m} x R\textsubscript{eff} = -0.010 x 4,000 =
\textbf{-40 V/V}

\begin{enumerate}
\def\labelenumi{(\alph{enumi})}
\setcounter{enumi}{2}
\tightlist
\item
  V\textsubscript{out} = A\textsubscript{v} x V\textsubscript{gs} = -40
  x 0.1 = \textbf{-4.0 V}
\end{enumerate}

The minus sign indicates the amplifier is inverting -- a positive input
produces a negative output.

\begin{center}\rule{0.5\linewidth}{0.5pt}\end{center}

\chapter{Chapter 7 --- Section 7.6: Circuit
Theorems}\label{chapter-7-section-7.6-circuit-theorems}

Practice problems covering Thevenin's theorem, Norton's theorem, and
maximum power transfer.

\begin{center}\rule{0.5\linewidth}{0.5pt}\end{center}

\section{Problem 7.6.1}\label{problem-7.6.1}

\textbf{Given:} A linear network consists of a 50 V source in series
with R₁ = 10 ohm, connected to a node. From that node, R₂ = 15 ohm
connects to ground, and two output terminals are taken across R₂.

\textbf{Find:} (a) The Thevenin equivalent voltage V\textsubscript{Th}.
(b) The Thevenin equivalent resistance R\textsubscript{Th}. (c) The
current delivered to a 5 ohm load connected across the output terminals.

\textbf{Solution:}

\begin{enumerate}
\def\labelenumi{(\alph{enumi})}
\tightlist
\item
  V\textsubscript{Th} (open-circuit voltage across R₂): No load current
  flows through any additional path.
\end{enumerate}

V\textsubscript{Th} = V\textsubscript{source} x R₂ / (R₁ + R₂) = 50 x 15
/ (10 + 15) = 750 / 25 = \textbf{30 V}

\begin{enumerate}
\def\labelenumi{(\alph{enumi})}
\setcounter{enumi}{1}
\tightlist
\item
  R\textsubscript{Th} (deactivate 50 V source by shorting): R₁ and R₂
  are in parallel.
\end{enumerate}

R\textsubscript{Th} = (R₁ x R₂) / (R₁ + R₂) = (10 x 15) / (10 + 15) =
150 / 25 = \textbf{6 ohm}

\begin{enumerate}
\def\labelenumi{(\alph{enumi})}
\setcounter{enumi}{2}
\tightlist
\item
  With a 5 ohm load: I\textsubscript{load} = V\textsubscript{Th} /
  (R\textsubscript{Th} + R\textsubscript{load}) = 30 / (6 + 5) = 30 / 11
  = \textbf{2.727 A}
\end{enumerate}

\begin{center}\rule{0.5\linewidth}{0.5pt}\end{center}

\section{Problem 7.6.2}\label{problem-7.6.2}

\textbf{Given:} A circuit has a 2 A current source in parallel with a 12
ohm resistor, which is in series with an 8 ohm resistor. The output
terminals are across the 8 ohm resistor.

\textbf{Find:} (a) The Norton equivalent current I\textsubscript{N}. (b)
The Norton equivalent resistance R\textsubscript{N}. (c) The Thevenin
equivalent circuit. (d) The load current for a 24 ohm load.

\textbf{Solution:} Topology: Node A at top. Current source (2 A upward)
from ground to node A. R₁ = 12 ohm from node A to ground (parallel with
the current source). R₂ = 8 ohm from node A to node B. Output terminals:
node B to ground.

\begin{enumerate}
\def\labelenumi{(\alph{enumi})}
\tightlist
\item
  Finding I\textsubscript{N} via the Thevenin approach.
\end{enumerate}

Open-circuit voltage V\textsubscript{Th}: With the output open, no
current flows through R₂. KCL at node A: 2 = V\textsubscript{A}/12, so
V\textsubscript{A} = 24 V. Since no current flows through R₂,
V\textsubscript{B} = V\textsubscript{A} = 24 V.

V\textsubscript{Th} = \textbf{24 V}

R\textsubscript{Th}: Open the current source. Looking from B to ground:
R₂ is in series with R₁ (from A to ground).

R\textsubscript{Th} = R₂ + R₁ = 8 + 12 = \textbf{20 ohm}

I\textsubscript{N} = V\textsubscript{Th} / R\textsubscript{Th} = 24 / 20
= \textbf{1.2 A}

\begin{enumerate}
\def\labelenumi{(\alph{enumi})}
\setcounter{enumi}{1}
\item
  R\textsubscript{N} = R\textsubscript{Th} = \textbf{20 ohm}
\item
  Thevenin equivalent: \textbf{24 V source in series with 20 ohm}.
\item
  Load current with R\textsubscript{L} = 24 ohm:
\end{enumerate}

I\textsubscript{load} = V\textsubscript{Th} / (R\textsubscript{Th} +
R\textsubscript{L}) = 24 / (20 + 24) = 24 / 44 = \textbf{0.545 A}

\begin{center}\rule{0.5\linewidth}{0.5pt}\end{center}

\section{Problem 7.6.3}\label{problem-7.6.3}

\textbf{Given:} A signal generator has a Thevenin equivalent of
V\textsubscript{Th} = 2 V and R\textsubscript{Th} = 50 ohm. It drives a
transmission line terminated in a load.

\textbf{Find:} (a) The load resistance for maximum power transfer. (b)
The maximum power delivered to the load. (c) The power delivered if the
load is 75 ohm (a common mismatch). (d) The percentage of maximum power
lost due to the mismatch.

\textbf{Solution:}

\begin{enumerate}
\def\labelenumi{(\alph{enumi})}
\item
  For maximum power transfer: R\textsubscript{load} =
  R\textsubscript{Th} = \textbf{50 ohm}
\item
  Maximum power: P\textsubscript{max} =
  V\textsubscript{Th}\textsuperscript{2} / (4 x R\textsubscript{Th}) =
  2\textsuperscript{2} / (4 x 50) = 4 / 200 = \textbf{20 mW}
\item
  With R\textsubscript{load} = 75 ohm:
\end{enumerate}

I = V\textsubscript{Th} / (R\textsubscript{Th} + R\textsubscript{load})
= 2 / (50 + 75) = 2 / 125 = 16 mA

P\textsubscript{load} = I\textsuperscript{2} x R\textsubscript{load} =
(0.016)\textsuperscript{2} x 75 = 2.56 x 10\textsuperscript{-4} x 75 =
\textbf{19.2 mW}

\begin{enumerate}
\def\labelenumi{(\alph{enumi})}
\setcounter{enumi}{3}
\tightlist
\item
  Power loss relative to maximum: (20 - 19.2) / 20 x 100 = \textbf{4\%}
\end{enumerate}

A 50-to-75 ohm mismatch only loses 4\% of the maximum deliverable power,
which explains why moderate impedance mismatches are often tolerable.

\begin{center}\rule{0.5\linewidth}{0.5pt}\end{center}

\section{Problem 7.6.4}\label{problem-7.6.4}

\textbf{Given:} A bridge circuit is powered by a 100 V source. The four
bridge arms are: R₁ = 100 ohm (top-left), R₂ = 200 ohm (top-right), R₃ =
150 ohm (bottom-left), R₄ = 300 ohm (bottom-right). A load
R\textsubscript{L} is connected across the bridge diagonal (between the
left midpoint and right midpoint).

\textbf{Find:} (a) The Thevenin equivalent seen by R\textsubscript{L}.
(b) The current through R\textsubscript{L} if R\textsubscript{L} = 50
ohm. (c) Whether the bridge is balanced.

\textbf{Solution:}

\begin{enumerate}
\def\labelenumi{(\alph{enumi})}
\tightlist
\item
  V\textsubscript{Th}: Open-circuit voltage across the diagonal.
\end{enumerate}

Left midpoint voltage: V\textsubscript{L} = 100 x R₃ / (R₁ + R₃) = 100 x
150 / (100 + 150) = 100 x 150/250 = \textbf{60 V}

Right midpoint voltage: V\textsubscript{R} = 100 x R₄ / (R₂ + R₄) = 100
x 300 / (200 + 300) = 100 x 300/500 = \textbf{60 V}

V\textsubscript{Th} = V\textsubscript{L} - V\textsubscript{R} = 60 - 60
= \textbf{0 V}

R\textsubscript{Th}: Short the voltage source. Looking into the
diagonal:

R\textsubscript{Th} = (R₁ \textbar\textbar{} R₃) + (R₂
\textbar\textbar{} R₄) = (100 x 150)/(100 + 150) + (200 x 300)/(200 +
300) = 60 + 120 = \textbf{180 ohm}

\begin{enumerate}
\def\labelenumi{(\alph{enumi})}
\setcounter{enumi}{1}
\item
  Since V\textsubscript{Th} = 0 V, the current through any load
  R\textsubscript{L} is: I\textsubscript{L} = 0 / (180 + 50) = \textbf{0
  A}
\item
  The bridge \textbf{is balanced} because R₁/R₃ = 100/150 = 2/3 and
  R₂/R₄ = 200/300 = 2/3. When R₁/R₃ = R₂/R₄, no current flows through
  the bridge diagonal regardless of load resistance.
\end{enumerate}

\begin{center}\rule{0.5\linewidth}{0.5pt}\end{center}

\section{Problem 7.6.5}\label{problem-7.6.5}

\textbf{Given:} A battery-powered sensor system has a battery with
V\textsubscript{oc} = 3.0 V and internal resistance R\textsubscript{int}
= 2 ohm. The sensor draws a constant 50 mA.

\textbf{Find:} (a) The terminal voltage under load. (b) The power
delivered to the sensor. (c) The power lost in the internal resistance.
(d) The efficiency (ratio of load power to total power). (e) Compare to
the maximum power transfer case.

\textbf{Solution:}

\begin{enumerate}
\def\labelenumi{(\alph{enumi})}
\item
  Terminal voltage: V\textsubscript{t} = V\textsubscript{oc} - I x
  R\textsubscript{int} = 3.0 - 0.050 x 2 = 3.0 - 0.1 = \textbf{2.9 V}
\item
  Power to sensor: P\textsubscript{load} = V\textsubscript{t} x I = 2.9
  x 0.050 = \textbf{145 mW}
\item
  Power lost internally: P\textsubscript{int} = I\textsuperscript{2} x
  R\textsubscript{int} = (0.050)\textsuperscript{2} x 2 = 0.0025 x 2 =
  \textbf{5 mW}
\item
  Efficiency: eta = P\textsubscript{load} / (P\textsubscript{load} +
  P\textsubscript{int}) = 145 / (145 + 5) = 145 / 150 = \textbf{96.7\%}
\item
  At maximum power transfer, R\textsubscript{load} =
  R\textsubscript{int} = 2 ohm:
\end{enumerate}

I\textsubscript{max} = 3.0 / (2 + 2) = 0.75 A

P\textsubscript{max} = I\textsubscript{max}\textsuperscript{2} x
R\textsubscript{load} = 0.5625 x 2 = 1.125 W

But efficiency would be only 50\%, and the battery would drain 30x
faster. The sensor's operating point (R\textsubscript{load} = 2.9/0.050
= 58 ohm \textgreater\textgreater{} R\textsubscript{int}) achieves much
higher efficiency at the cost of lower power extraction -- appropriate
for a battery-powered device.

\begin{center}\rule{0.5\linewidth}{0.5pt}\end{center}

\chapter{Chapter 7 --- Section 7.7: AC Circuit
Analysis}\label{chapter-7-section-7.7-ac-circuit-analysis}

Practice problems covering impedance, resonance, AC power, and mutual
inductance/coupled circuits.

\begin{center}\rule{0.5\linewidth}{0.5pt}\end{center}

\section{Problem 7.7.1}\label{problem-7.7.1}

\textbf{Given:} A series RLC circuit has R = 47 ohm, L = 100 mH, and C =
10 uF, driven by a 120 V\textsubscript{rms}, 50 Hz source.

\textbf{Find:} (a) The inductive reactance. (b) The capacitive
reactance. (c) The total impedance (magnitude and angle). (d) The
current magnitude. (e) The phase angle between voltage and current.

\textbf{Solution:}

\begin{enumerate}
\def\labelenumi{(\alph{enumi})}
\tightlist
\item
  omega = 2 x pi x 50 = 314.16 rad/s
\end{enumerate}

X\textsubscript{L} = omega x L = 314.16 x 0.100 = \textbf{31.42 ohm}

\begin{enumerate}
\def\labelenumi{(\alph{enumi})}
\setcounter{enumi}{1}
\item
  X\textsubscript{C} = 1 / (omega x C) = 1 / (314.16 x 10 x
  10\textsuperscript{-6}) = 1 / 0.003142 = \textbf{318.3 ohm}
\item
  Net reactance: X = X\textsubscript{L} - X\textsubscript{C} = 31.42 -
  318.3 = -286.9 ohm (capacitive)
\end{enumerate}

\textbar Z\textbar{} = sqrt(R\textsuperscript{2} + X\textsuperscript{2})
= sqrt(47\textsuperscript{2} + 286.9\textsuperscript{2}) = sqrt(2,209 +
82,312) = sqrt(84,521) = \textbf{290.7 ohm}

theta = arctan(X/R) = arctan(-286.9/47) = arctan(-6.105) = \textbf{-80.7
deg}

Z = 47 - j286.9 ohm

\begin{enumerate}
\def\labelenumi{(\alph{enumi})}
\setcounter{enumi}{3}
\item
  Current: I = V / \textbar Z\textbar{} = 120 / 290.7 = \textbf{0.413
  A\textsubscript{rms}}
\item
  The current \textbf{leads} the voltage by \textbf{80.7 deg}
  (capacitive circuit, since X\textsubscript{C} \textgreater{}
  X\textsubscript{L}).
\end{enumerate}

\begin{center}\rule{0.5\linewidth}{0.5pt}\end{center}

\section{Problem 7.7.2}\label{problem-7.7.2}

\textbf{Given:} A parallel RLC circuit has R = 1 kohm, L = 5 mH, and C =
20 nF. It is driven by a current source.

\textbf{Find:} (a) The resonant frequency. (b) The quality factor. (c)
The bandwidth. (d) The impedance at resonance.

\textbf{Solution:}

\begin{enumerate}
\def\labelenumi{(\alph{enumi})}
\tightlist
\item
  Resonant frequency: f₀ = 1 / (2 x pi x sqrt(LC))
\end{enumerate}

f₀ = 1 / (2 x pi x sqrt(5 x 10\textsuperscript{-3} x 20 x
10\textsuperscript{-9}))

f₀ = 1 / (2 x pi x sqrt(10\textsuperscript{-10}))

f₀ = 1 / (2 x pi x 10\textsuperscript{-5}) = 1 / (6.283 x
10\textsuperscript{-5}) = \textbf{15,915 Hz = 15.92 kHz}

\begin{enumerate}
\def\labelenumi{(\alph{enumi})}
\setcounter{enumi}{1}
\tightlist
\item
  For a parallel RLC circuit: Q = R x sqrt(C/L)
\end{enumerate}

Q = 1000 x sqrt(20 x 10\textsuperscript{-9} / 5 x
10\textsuperscript{-3}) = 1000 x sqrt(4 x 10\textsuperscript{-6}) = 1000
x 2 x 10\textsuperscript{-3} = \textbf{2.0}

Or equivalently: Q = R / (2 x pi x f₀ x L) \ldots{} Let's verify:

Q = R / (omega₀ x L) = 1000 / (2 x pi x 15915 x 0.005) = 1000 / 500 =
2.0 (verified)

\begin{enumerate}
\def\labelenumi{(\alph{enumi})}
\setcounter{enumi}{2}
\item
  Bandwidth: BW = f₀ / Q = 15,915 / 2.0 = \textbf{7,958 Hz = 7.96 kHz}
\item
  At resonance, the impedance of a parallel RLC circuit equals R (the
  reactive branches cancel):
\end{enumerate}

Z\textsubscript{resonance} = \textbf{1,000 ohm = 1 kohm}

\begin{center}\rule{0.5\linewidth}{0.5pt}\end{center}

\section{Problem 7.7.3}\label{problem-7.7.3}

\textbf{Given:} A single-phase motor draws 8 A\textsubscript{rms} from a
240 V\textsubscript{rms}, 60 Hz supply with a power factor of 0.65
lagging. A capacitor bank is to be added in parallel to correct the
power factor to 0.95 lagging.

\textbf{Find:} (a) The real power P, reactive power Q, and apparent
power S before correction. (b) The reactive power after correction. (c)
The required capacitor reactive power. (d) The capacitor value needed.
(e) The new supply current after correction.

\textbf{Solution:}

\begin{enumerate}
\def\labelenumi{(\alph{enumi})}
\tightlist
\item
  Before correction:
\end{enumerate}

S = V x I = 240 x 8 = \textbf{1,920 VA}

P = S x cos(phi) = 1,920 x 0.65 = \textbf{1,248 W}

phi₁ = arccos(0.65) = 49.46 deg

Q₁ = S x sin(phi₁) = 1,920 x sin(49.46) = 1,920 x 0.7599 = \textbf{1,459
VAR (inductive)}

\begin{enumerate}
\def\labelenumi{(\alph{enumi})}
\setcounter{enumi}{1}
\tightlist
\item
  After correction (pf = 0.95), the real power remains the same:
\end{enumerate}

phi₂ = arccos(0.95) = 18.19 deg

Q₂ = P x tan(phi₂) = 1,248 x tan(18.19) = 1,248 x 0.3287 = \textbf{410.3
VAR}

\begin{enumerate}
\def\labelenumi{(\alph{enumi})}
\setcounter{enumi}{2}
\item
  Capacitor reactive power: Q\textsubscript{C} = Q₁ - Q₂ = 1,459 - 410.3
  = \textbf{1,048.7 VAR}
\item
  Capacitor value: Q\textsubscript{C} = V\textsuperscript{2} x omega x C
\end{enumerate}

C = Q\textsubscript{C} / (V\textsuperscript{2} x omega) = 1,048.7 /
(240\textsuperscript{2} x 2 x pi x 60)

C = 1,048.7 / (57,600 x 376.99) = 1,048.7 / 21,714,624 = \textbf{48.3
uF}

\begin{enumerate}
\def\labelenumi{(\alph{enumi})}
\setcounter{enumi}{4}
\tightlist
\item
  New apparent power: S₂ = P / cos(phi₂) = 1,248 / 0.95 = 1,313.7 VA
\end{enumerate}

New current: I₂ = S₂ / V = 1,313.7 / 240 = \textbf{5.47
A\textsubscript{rms}}

The supply current drops from 8 A to 5.47 A -- a \textbf{31.6\%
reduction} -- reducing I\textsuperscript{2}R losses in the supply
wiring.

\begin{center}\rule{0.5\linewidth}{0.5pt}\end{center}

\section{Problem 7.7.4}\label{problem-7.7.4}

\textbf{Given:} An audio transformer has a turns ratio of N₁/N₂ = 10:1,
primary inductance L₁ = 500 mH, and coupling coefficient k = 0.97. The
transformer connects a 600 ohm microphone line to an 8 ohm preamplifier
input.

\textbf{Find:} (a) The secondary inductance L₂. (b) The mutual
inductance M. (c) The reflected impedance seen by the microphone line.
(d) Whether this transformer provides a good impedance match.

\textbf{Solution:}

\begin{enumerate}
\def\labelenumi{(\alph{enumi})}
\item
  L₂ = L₁ / (N₁/N₂)\textsuperscript{2} = 500 mH / 10\textsuperscript{2}
  = 500 / 100 = \textbf{5 mH}
\item
  M = k x sqrt(L₁ x L₂) = 0.97 x sqrt(0.500 x 0.005) = 0.97 x
  sqrt(0.0025) = 0.97 x 0.05 = \textbf{48.5 mH}
\item
  Reflected impedance: Z\textsubscript{reflected} =
  (N₁/N₂)\textsuperscript{2} x Z\textsubscript{load} =
  10\textsuperscript{2} x 8 = \textbf{800 ohm}
\item
  The microphone line sees 800 ohm instead of the desired 600 ohm -- a
  mismatch ratio of 800/600 = 1.33:1.
\end{enumerate}

For a perfect match: N₁/N₂ =
sqrt(Z\textsubscript{source}/Z\textsubscript{load}) = sqrt(600/8) =
sqrt(75) = \textbf{8.66:1}

The 10:1 transformer is reasonably close but not optimal. A custom
8.66:1 ratio would provide exact impedance matching.

\begin{center}\rule{0.5\linewidth}{0.5pt}\end{center}

\section{Problem 7.7.5}\label{problem-7.7.5}

\textbf{Given:} A series RL circuit with R = 200 ohm and L = 0.5 H
carries a current of 3 A\textsubscript{rms} at 60 Hz.

\textbf{Find:} (a) The impedance. (b) The voltage across the resistor.
(c) The voltage across the inductor. (d) The total source voltage. (e)
The real, reactive, and apparent power.

\textbf{Solution:}

\begin{enumerate}
\def\labelenumi{(\alph{enumi})}
\tightlist
\item
  omega = 2 x pi x 60 = 376.99 rad/s
\end{enumerate}

X\textsubscript{L} = omega x L = 376.99 x 0.5 = 188.5 ohm

Z = R + jX\textsubscript{L} = 200 + j188.5 ohm

\textbar Z\textbar{} = sqrt(200\textsuperscript{2} +
188.5\textsuperscript{2}) = sqrt(40,000 + 35,532) = sqrt(75,532) =
\textbf{274.8 ohm}

\begin{enumerate}
\def\labelenumi{(\alph{enumi})}
\setcounter{enumi}{1}
\item
  V\textsubscript{R} = I x R = 3 x 200 = \textbf{600
  V\textsubscript{rms}}
\item
  V\textsubscript{L} = I x X\textsubscript{L} = 3 x 188.5 =
  \textbf{565.5 V\textsubscript{rms}}
\item
  V\textsubscript{source} = I x \textbar Z\textbar{} = 3 x 274.8 =
  \textbf{824.4 V\textsubscript{rms}}
\end{enumerate}

Note: V\textsubscript{source} is not equal to V\textsubscript{R} +
V\textsubscript{L} arithmetically (600 + 565.5 = 1165.5) because the
voltages are 90 deg out of phase. The phasor sum gives
sqrt(600\textsuperscript{2} + 565.5\textsuperscript{2}) = sqrt(360,000 +
319,790) = sqrt(679,790) = 824.5 V (verified).

\begin{enumerate}
\def\labelenumi{(\alph{enumi})}
\setcounter{enumi}{4}
\tightlist
\item
  Phase angle: phi = arctan(188.5/200) = arctan(0.9425) = 43.3 deg
\end{enumerate}

P = V x I x cos(phi) = 824.4 x 3 x cos(43.3) = 2473.2 x 0.7275 =
\textbf{1,800 W}

Or: P = I\textsuperscript{2} x R = 9 x 200 = 1,800 W (verified)

Q = I\textsuperscript{2} x X\textsubscript{L} = 9 x 188.5 =
\textbf{1,696.5 VAR}

S = V x I = 824.4 x 3 = \textbf{2,473.2 VA}

Check: S = sqrt(P\textsuperscript{2} + Q\textsuperscript{2}) =
sqrt(3,240,000 + 2,878,140) = sqrt(6,118,140) = 2,473.5 VA (verified).

\begin{center}\rule{0.5\linewidth}{0.5pt}\end{center}

\section{Problem 7.7.6}\label{problem-7.7.6}

\textbf{Given:} A series RLC tuning circuit in an AM radio receiver must
select a station at 1,000 kHz. The inductor is L = 250 uH with a quality
factor Q = 50 at the resonant frequency.

\textbf{Find:} (a) The required capacitance. (b) The bandwidth of the
tuned circuit. (c) The resistance of the inductor coil. (d) Whether the
circuit can reject an adjacent channel at 1,010 kHz (10 kHz spacing).

\textbf{Solution:}

\begin{enumerate}
\def\labelenumi{(\alph{enumi})}
\tightlist
\item
  f₀ = 1 / (2 x pi x sqrt(LC)), solving for C:
\end{enumerate}

C = 1 / (4 x pi\textsuperscript{2} x f₀\textsuperscript{2} x L) = 1 / (4
x pi\textsuperscript{2} x (10\textsuperscript{6})\textsuperscript{2} x
250 x 10\textsuperscript{-6})

C = 1 / (4 x 9.8696 x 10\textsuperscript{12} x 2.5 x
10\textsuperscript{-4}) = 1 / (9.8696 x 10\textsuperscript{9})

C = \textbf{101.3 pF}

\begin{enumerate}
\def\labelenumi{(\alph{enumi})}
\setcounter{enumi}{1}
\item
  Bandwidth: BW = f₀ / Q = 1,000,000 / 50 = \textbf{20,000 Hz = 20 kHz}
\item
  At resonance: Q = omega₀ x L / R
\end{enumerate}

R = omega₀ x L / Q = 2 x pi x 10\textsuperscript{6} x 250 x
10\textsuperscript{-6} / 50 = 6,283,185 x 2.5 x 10\textsuperscript{-4} /
50 = 1,570.8 / 50 = \textbf{31.4 ohm}

\begin{enumerate}
\def\labelenumi{(\alph{enumi})}
\setcounter{enumi}{3}
\tightlist
\item
  The adjacent channel at 1,010 kHz is 10 kHz away from the center
  frequency. The 3 dB bandwidth is 20 kHz, so the 10 kHz offset is at
  the half-bandwidth point.
\end{enumerate}

At f = 1,010 kHz, the attenuation relative to the peak is approximately:

Attenuation = 1 / sqrt(1 + (2 x delta\_f / BW)\textsuperscript{2}) where
delta\_f = 10 kHz

= 1 / sqrt(1 + (20/20)\textsuperscript{2}) = 1 / sqrt(2) = 0.707 =
\textbf{-3 dB}

Only 3 dB of adjacent channel rejection is \textbf{insufficient} for
good selectivity. A higher-Q circuit or a multi-stage filter would be
needed. Doubling Q to 100 would narrow the bandwidth to 10 kHz,
providing better selectivity.

\begin{center}\rule{0.5\linewidth}{0.5pt}\end{center}

\chapter{Chapter 7 --- Section 7.8: Transient
Analysis}\label{chapter-7-section-7.8-transient-analysis}

Practice problems covering RC circuits, RL circuits, and RLC circuits.

\begin{center}\rule{0.5\linewidth}{0.5pt}\end{center}

\section{Problem 7.8.1}\label{problem-7.8.1}

\textbf{Given:} A 220 uF capacitor charged to 50 V is discharged through
a 10 kohm resistor at t = 0.

\textbf{Find:} (a) The time constant. (b) The capacitor voltage at t = 1
s, 2 s, and 5 s. (c) The initial discharge current. (d) The time at
which the voltage drops to 10 V.

\textbf{Solution:} (a) tau = R x C = 10,000 x 220 x
10\textsuperscript{-6} = \textbf{2.2 s}

\begin{enumerate}
\def\labelenumi{(\alph{enumi})}
\setcounter{enumi}{1}
\tightlist
\item
  Discharge: V\textsubscript{c}(t) = V₀ x e\textsuperscript{-t/tau}
\end{enumerate}

At t = 1 s: V\textsubscript{c} = 50 x e\textsuperscript{-1/2.2} = 50 x
e\textsuperscript{-0.4545} = 50 x 0.6346 = \textbf{31.73 V}

At t = 2 s: V\textsubscript{c} = 50 x e\textsuperscript{-2/2.2} = 50 x
e\textsuperscript{-0.9091} = 50 x 0.4029 = \textbf{20.14 V}

At t = 5 s: V\textsubscript{c} = 50 x e\textsuperscript{-5/2.2} = 50 x
e\textsuperscript{-2.2727} = 50 x 0.1030 = \textbf{5.15 V}

\begin{enumerate}
\def\labelenumi{(\alph{enumi})}
\setcounter{enumi}{2}
\item
  Initial current: I₀ = V₀ / R = 50 / 10,000 = \textbf{5 mA}
\item
  Time to reach 10 V: 10 = 50 x e\textsuperscript{-t/2.2}
\end{enumerate}

e\textsuperscript{-t/2.2} = 0.2

-t/2.2 = ln(0.2) = -1.6094

t = 2.2 x 1.6094 = \textbf{3.54 s}

\begin{center}\rule{0.5\linewidth}{0.5pt}\end{center}

\section{Problem 7.8.2}\label{problem-7.8.2}

\textbf{Given:} A relay coil has an inductance of 50 mH and a resistance
of 25 ohm. It is suddenly connected to a 12 V DC supply.

\textbf{Find:} (a) The time constant. (b) The steady-state current. (c)
The current at t = 1 ms. (d) The time to reach 90\% of the steady-state
current. (e) The initial rate of change of current (di/dt at t = 0).

\textbf{Solution:} (a) tau = L / R = 0.050 / 25 = \textbf{2 ms}

\begin{enumerate}
\def\labelenumi{(\alph{enumi})}
\setcounter{enumi}{1}
\item
  Steady-state current: I\textsubscript{ss} = V / R = 12 / 25 =
  \textbf{480 mA}
\item
  At t = 1 ms: I(t) = I\textsubscript{ss} x (1 -
  e\textsuperscript{-t/tau})
\end{enumerate}

I(0.001) = 0.480 x (1 - e\textsuperscript{-1/2}) = 0.480 x (1 - 0.6065)
= 0.480 x 0.3935 = \textbf{188.9 mA}

\begin{enumerate}
\def\labelenumi{(\alph{enumi})}
\setcounter{enumi}{3}
\tightlist
\item
  For 90\% of I\textsubscript{ss}: 0.9 = 1 - e\textsuperscript{-t/tau}
\end{enumerate}

e\textsuperscript{-t/tau} = 0.1

-t/tau = ln(0.1) = -2.3026

t = 2 x 10\textsuperscript{-3} x 2.3026 = \textbf{4.61 ms}

\begin{enumerate}
\def\labelenumi{(\alph{enumi})}
\setcounter{enumi}{4}
\tightlist
\item
  Initial di/dt: di/dt\textbar{}\textsubscript{t=0} = V / L = 12 / 0.050
  = \textbf{240 A/s}
\end{enumerate}

This is the maximum rate of current rise. It decreases exponentially as
the back-EMF decreases.

\begin{center}\rule{0.5\linewidth}{0.5pt}\end{center}

\section{Problem 7.8.3}\label{problem-7.8.3}

\textbf{Given:} A series RLC circuit has R = 200 ohm, L = 50 mH, and C =
0.5 uF. A 20 V DC step is applied at t = 0 with zero initial conditions.

\textbf{Find:} (a) The natural frequency omega₀. (b) The damping ratio
zeta. (c) The type of response (underdamped, critically damped, or
overdamped). (d) The damped natural frequency omega\textsubscript{d} (if
applicable). (e) The final (steady-state) capacitor voltage.

\textbf{Solution:} (a) omega₀ = 1 / sqrt(LC) = 1 / sqrt(0.050 x 0.5 x
10\textsuperscript{-6})

= 1 / sqrt(25 x 10\textsuperscript{-9}) = 1 / (1.581 x
10\textsuperscript{-4}) = \textbf{6,325 rad/s}

\begin{enumerate}
\def\labelenumi{(\alph{enumi})}
\setcounter{enumi}{1}
\tightlist
\item
  zeta = R / (2 x sqrt(L/C)) = 200 / (2 x sqrt(0.050 / 0.5 x
  10\textsuperscript{-6}))
\end{enumerate}

= 200 / (2 x sqrt(100,000)) = 200 / (2 x 316.2) = 200 / 632.5 =
\textbf{0.316}

\begin{enumerate}
\def\labelenumi{(\alph{enumi})}
\setcounter{enumi}{2}
\item
  Since zeta = 0.316 \textless{} 1, the response is \textbf{underdamped}
  (oscillatory with exponentially decaying envelope).
\item
  omega\textsubscript{d} = omega₀ x sqrt(1 - zeta\textsuperscript{2}) =
  6,325 x sqrt(1 - 0.1) = 6,325 x sqrt(0.9) = 6,325 x 0.9487 =
  \textbf{5,999 rad/s}
\end{enumerate}

f\textsubscript{d} = omega\textsubscript{d} / (2 x pi) = 5,999 / 6.283 =
\textbf{955 Hz}

\begin{enumerate}
\def\labelenumi{(\alph{enumi})}
\setcounter{enumi}{4}
\tightlist
\item
  At DC steady state, the capacitor is fully charged and no current
  flows:
\end{enumerate}

V\textsubscript{C,ss} = V\textsubscript{source} = \textbf{20 V} (all
voltage appears across the capacitor since the inductor acts as a short
and the resistor has no current through it)

\begin{center}\rule{0.5\linewidth}{0.5pt}\end{center}

\section{Problem 7.8.4}\label{problem-7.8.4}

\textbf{Given:} An RC timing circuit in a 555 timer uses R = 100 kohm
and C = 4.7 uF. The capacitor charges from 1/3 V\textsubscript{CC} to
2/3 V\textsubscript{CC} (where V\textsubscript{CC} = 12 V) and then
discharges back to 1/3 V\textsubscript{CC} to set the oscillation
frequency.

\textbf{Find:} (a) The time constant. (b) The charge time from 4 V to 8
V (1/3 to 2/3 of 12 V). (c) The approximate oscillation frequency (using
the 555 formula f = 1.44 / (R x C) for a basic astable with
R\textsubscript{A} = R\textsubscript{B} = R).

\textbf{Solution:} (a) tau = R x C = 100,000 x 4.7 x
10\textsuperscript{-6} = \textbf{0.47 s}

\begin{enumerate}
\def\labelenumi{(\alph{enumi})}
\setcounter{enumi}{1}
\tightlist
\item
  The capacitor charges from V₁ = 4 V toward V\textsubscript{CC} = 12 V.
  Using the charging equation:
\end{enumerate}

V(t) = V\textsubscript{CC} - (V\textsubscript{CC} - V₁) x
e\textsuperscript{-t/tau} = 12 - 8 x e\textsuperscript{-t/0.47}

At V(t) = 8 V: 8 = 12 - 8 x e\textsuperscript{-t/0.47}

8 x e\textsuperscript{-t/0.47} = 4

e\textsuperscript{-t/0.47} = 0.5

-t/0.47 = ln(0.5) = -0.6931

t\textsubscript{charge} = 0.47 x 0.6931 = \textbf{0.326 s}

\begin{enumerate}
\def\labelenumi{(\alph{enumi})}
\setcounter{enumi}{2}
\tightlist
\item
  For a basic 555 astable with R\textsubscript{A} = R\textsubscript{B} =
  R:
\end{enumerate}

f = 1.44 / ((R\textsubscript{A} + 2R\textsubscript{B}) x C) = 1.44 /
((100,000 + 200,000) x 4.7 x 10\textsuperscript{-6})

= 1.44 / (300,000 x 4.7 x 10\textsuperscript{-6}) = 1.44 / 1.41 =
\textbf{1.02 Hz}

Period: T = 1/f = \textbf{0.98 s}

\begin{center}\rule{0.5\linewidth}{0.5pt}\end{center}

\section{Problem 7.8.5}\label{problem-7.8.5}

\textbf{Given:} A series RLC circuit has R = 400 ohm, L = 25 mH, and C =
1 uF. The circuit is initially at rest and a 10 V step is applied.

\textbf{Find:} (a) omega₀ and zeta. (b) Classify the response. (c) If
overdamped, find the two natural frequencies s₁ and s₂.

\textbf{Solution:} (a) omega₀ = 1 / sqrt(LC) = 1 / sqrt(25 x
10\textsuperscript{-3} x 10\textsuperscript{-6}) = 1 / sqrt(25 x
10\textsuperscript{-9}) = 1 / (1.581 x 10\textsuperscript{-4}) =
\textbf{6,325 rad/s}

zeta = R / (2 x sqrt(L/C)) = 400 / (2 x sqrt(0.025 /
10\textsuperscript{-6})) = 400 / (2 x sqrt(25,000))

= 400 / (2 x 158.1) = 400 / 316.2 = \textbf{1.265}

\begin{enumerate}
\def\labelenumi{(\alph{enumi})}
\setcounter{enumi}{1}
\item
  Since zeta = 1.265 \textgreater{} 1, the response is
  \textbf{overdamped} (no oscillation, exponential decay).
\item
  The characteristic equation roots:
\end{enumerate}

s₁,₂ = -zeta x omega₀ +/- omega₀ x sqrt(zeta\textsuperscript{2} - 1)

= -1.265 x 6,325 +/- 6,325 x sqrt(1.6002 - 1)

= -8,001 +/- 6,325 x sqrt(0.6002)

= -8,001 +/- 6,325 x 0.7748

= -8,001 +/- 4,901

s₁ = -8,001 + 4,901 = \textbf{-3,100 rad/s} (slower mode, tau₁ = 1/3100
= 0.323 ms)

s₂ = -8,001 - 4,901 = \textbf{-12,902 rad/s} (faster mode, tau₂ =
1/12902 = 0.0775 ms)

The response is a sum of two decaying exponentials:
V\textsubscript{C}(t) = 10 + A₁e\textsuperscript{s₁t} +
A₂e\textsuperscript{s₂t}, where A₁ and A₂ are determined by initial
conditions.

\begin{center}\rule{0.5\linewidth}{0.5pt}\end{center}

\section{Problem 7.8.6}\label{problem-7.8.6}

\textbf{Given:} An RL circuit consists of a 500 mH inductor in series
with a 100 ohm resistor, carrying a steady-state current of 2 A from a
DC source. At t = 0, the source is suddenly disconnected and the
inductor discharges through the resistor.

\textbf{Find:} (a) The time constant. (b) The initial energy stored in
the inductor. (c) The voltage across the resistor at t =
0\textsuperscript{+}. (d) The time at which 95\% of the stored energy
has been dissipated.

\textbf{Solution:} (a) tau = L / R = 0.500 / 100 = \textbf{5 ms}

\begin{enumerate}
\def\labelenumi{(\alph{enumi})}
\setcounter{enumi}{1}
\item
  Initial energy: W = 1/2 x L x I\textsuperscript{2} = 0.5 x 0.500 x
  2\textsuperscript{2} = 0.5 x 0.500 x 4 = \textbf{1.0 J}
\item
  At t = 0\textsuperscript{+}, the inductor current cannot change
  instantaneously and equals 2 A. The voltage across the resistor:
\end{enumerate}

V\textsubscript{R}(0\textsuperscript{+}) = I₀ x R = 2 x 100 =
\textbf{200 V}

This demonstrates why disconnecting an inductive circuit can create
dangerous voltage spikes.

\begin{enumerate}
\def\labelenumi{(\alph{enumi})}
\setcounter{enumi}{3}
\tightlist
\item
  The energy remaining at time t: W(t) = 1/2 x L x
  I(t)\textsuperscript{2} = 1/2 x L x
  (I₀e\textsuperscript{-t/tau})\textsuperscript{2} = W₀ x
  e\textsuperscript{-2t/tau}
\end{enumerate}

For 95\% dissipated (5\% remaining): 0.05 = e\textsuperscript{-2t/tau}

-2t/tau = ln(0.05) = -2.996

t = tau x 2.996 / 2 = 0.005 x 1.498 = \textbf{7.49 ms}

This is approximately 1.5 time constants, at which point 95\% of the
magnetic energy has been converted to heat in the resistor.

\begin{center}\rule{0.5\linewidth}{0.5pt}\end{center}

\chapter{Chapter 7 --- Section 7.9: Two-Port
Networks}\label{chapter-7-section-7.9-two-port-networks}

Practice problems covering two-port parameters (Z, Y, h, ABCD) and
two-port interconnections.

\begin{center}\rule{0.5\linewidth}{0.5pt}\end{center}

\section{Problem 7.9.1}\label{problem-7.9.1}

\textbf{Given:} A pi-network has Z\textsubscript{a} = 100 ohm as a shunt
element from port 1 to ground, Z\textsubscript{b} = 50 ohm as the series
element between port 1 and port 2, and Z\textsubscript{c} = 200 ohm as a
shunt element from port 2 to ground.

\textbf{Find:} The Y-parameters of the pi-network.

\textbf{Solution:} For a pi-network, the Y-parameters are:

Y₁₁ = 1/Z\textsubscript{a} + 1/Z\textsubscript{b} (short-circuit input
admittance with V₂ = 0)

Y₁₁ = 1/100 + 1/50 = 0.01 + 0.02 = \textbf{0.03 S (30 mS)}

Y₁₂ = -1/Z\textsubscript{b} (mutual admittance)

Y₁₂ = -1/50 = \textbf{-0.02 S (-20 mS)}

Y₂₁ = -1/Z\textsubscript{b} = \textbf{-0.02 S (-20 mS)}

Y₂₂ = 1/Z\textsubscript{b} + 1/Z\textsubscript{c} = 1/50 + 1/200 = 0.02
+ 0.005 = \textbf{0.025 S (25 mS)}

Since Y₁₂ = Y₂₁, the network is reciprocal, as expected for a passive
network.

\begin{center}\rule{0.5\linewidth}{0.5pt}\end{center}

\section{Problem 7.9.2}\label{problem-7.9.2}

\textbf{Given:} A transistor amplifier is modeled as a two-port with
h-parameters: h₁₁ = 2.5 kohm (input impedance), h₁₂ = 2 x
10\textsuperscript{-4} (reverse voltage ratio), h₂₁ = 150 (current
gain), h₂₂ = 25 uS (output admittance). The amplifier is driven by a 1
kohm source and loaded with a 10 kohm collector resistor.

\textbf{Find:} (a) The voltage gain A\textsubscript{v} = V₂/V₁. (b) The
current gain A\textsubscript{i} = I₂/I₁. (c) The input impedance
Z\textsubscript{in}. (d) The output impedance Z\textsubscript{out}.

\textbf{Solution:} (a) Voltage gain for the h-parameter model with a
load R\textsubscript{L}:

A\textsubscript{v} = -h₂₁ x R\textsubscript{L} / (h₁₁ x (1 + h₂₂ x
R\textsubscript{L}) - h₁₂ x h₂₁ x R\textsubscript{L})

Denominator = h₁₁ + (h₁₁h₂₂ - h₁₂h₂₁) x R\textsubscript{L}

Delta\_h = h₁₁ x h₂₂ - h₁₂ x h₂₁ = 2,500 x 25 x 10\textsuperscript{-6} -
2 x 10\textsuperscript{-4} x 150 = 0.0625 - 0.03 = 0.0325

A\textsubscript{v} = -h₂₁ x R\textsubscript{L} / (h₁₁ + Delta\_h x
R\textsubscript{L}) = -150 x 10,000 / (2,500 + 0.0325 x 10,000)

= -1,500,000 / (2,500 + 325) = -1,500,000 / 2,825 = \textbf{-531 V/V}

\begin{enumerate}
\def\labelenumi{(\alph{enumi})}
\setcounter{enumi}{1}
\tightlist
\item
  Current gain: A\textsubscript{i} = h₂₁ / (1 + h₂₂ x
  R\textsubscript{L}) = 150 / (1 + 25 x 10\textsuperscript{-6} x 10,000)
\end{enumerate}

= 150 / (1 + 0.25) = 150 / 1.25 = \textbf{120 A/A}

\begin{enumerate}
\def\labelenumi{(\alph{enumi})}
\setcounter{enumi}{2}
\tightlist
\item
  Input impedance: Z\textsubscript{in} = h₁₁ - h₁₂ x h₂₁ x
  R\textsubscript{L} / (1 + h₂₂ x R\textsubscript{L})
\end{enumerate}

= 2,500 - (2 x 10\textsuperscript{-4} x 150 x 10,000) / (1.25) = 2,500 -
300/1.25 = 2,500 - 240 = \textbf{2,260 ohm}

\begin{enumerate}
\def\labelenumi{(\alph{enumi})}
\setcounter{enumi}{3}
\tightlist
\item
  Output impedance: Z\textsubscript{out} = 1 / (h₂₂ - h₁₂ x h₂₁ / (h₁₁ +
  R\textsubscript{S}))
\end{enumerate}

= 1 / (25 x 10\textsuperscript{-6} - 2 x 10\textsuperscript{-4} x 150 /
(2,500 + 1,000))

= 1 / (25 x 10\textsuperscript{-6} - 0.03/3,500) = 1 / (25 x
10\textsuperscript{-6} - 8.571 x 10\textsuperscript{-6})

= 1 / (16.43 x 10\textsuperscript{-6}) = \textbf{60.9 kohm}

\begin{center}\rule{0.5\linewidth}{0.5pt}\end{center}

\section{Problem 7.9.3}\label{problem-7.9.3}

\textbf{Given:} A transmission line segment is modeled as an ABCD
network with parameters: A = cosh(gamma x l), B = Z₀ x sinh(gamma x l),
C = sinh(gamma x l)/Z₀, D = cosh(gamma x l). For a lossless 50 ohm line
that is lambda/4 long at the operating frequency, gamma x l = j x pi/2.

\textbf{Find:} (a) The ABCD parameters for the quarter-wave line. (b)
The input impedance when terminated in a 200 ohm load. (c) The voltage
gain V₂/V₁.

\textbf{Solution:} (a) cosh(j x pi/2) = cos(pi/2) = 0

sinh(j x pi/2) = j x sin(pi/2) = j

A = \textbf{0}, B = Z₀ x j = \textbf{j50 ohm}, C = j/Z₀ = \textbf{j0.02
S}, D = \textbf{0}

Check reciprocity: AD - BC = 0 - (j50)(j0.02) = -j\textsuperscript{2} x
1 = 1. Verified.

\begin{enumerate}
\def\labelenumi{(\alph{enumi})}
\setcounter{enumi}{1}
\tightlist
\item
  Input impedance: Z\textsubscript{in} = (AZ\textsubscript{L} + B) /
  (CZ\textsubscript{L} + D)
\end{enumerate}

= (0 x 200 + j50) / (j0.02 x 200 + 0) = j50 / j4 = \textbf{12.5 ohm}

This demonstrates the quarter-wave transformer property:
Z\textsubscript{in} = Z₀\textsuperscript{2} / Z\textsubscript{L} =
50\textsuperscript{2}/200 = 2,500/200 = 12.5 ohm.

\begin{enumerate}
\def\labelenumi{(\alph{enumi})}
\setcounter{enumi}{2}
\tightlist
\item
  Voltage gain: V₂/V₁ = Z\textsubscript{L} / (AZ\textsubscript{L} + B) =
  200 / (0 + j50) = 200 / (j50)
\end{enumerate}

\textbar V₂/V₁\textbar{} = 200/50 = \textbf{4.0} (with a phase shift of
-90 deg)

The quarter-wave transformer produces a 4:1 voltage step-up from 12.5
ohm to 200 ohm.

\begin{center}\rule{0.5\linewidth}{0.5pt}\end{center}

\section{Problem 7.9.4}\label{problem-7.9.4}

\textbf{Given:} Three identical filter sections are cascaded. Each
section has ABCD parameters A = 1, B = 100 ohm, C = 0.01 S, D = 1.

\textbf{Find:} (a) The overall ABCD parameters for the cascade of three
sections. (b) The insertion loss (in dB) when the cascade is terminated
in Z\textsubscript{L} = 600 ohm and driven by a Z\textsubscript{S} = 600
ohm source.

\textbf{Solution:} (a) For a cascade, {[}ABCD{]}\textsubscript{total} =
{[}ABCD{]}₁ x {[}ABCD{]}₂ x {[}ABCD{]}₃.

First, cascade two sections:

A₁₂ = 1 x 1 + 100 x 0.01 = 1 + 1 = 2

B₁₂ = 1 x 100 + 100 x 1 = 200 ohm

C₁₂ = 0.01 x 1 + 1 x 0.01 = 0.02 S

D₁₂ = 0.01 x 100 + 1 x 1 = 2

Now cascade the two-section result with the third section:

A\textsubscript{T} = 2 x 1 + 200 x 0.01 = 2 + 2 = \textbf{4}

B\textsubscript{T} = 2 x 100 + 200 x 1 = 200 + 200 = \textbf{400 ohm}

C\textsubscript{T} = 0.02 x 1 + 2 x 0.01 = 0.02 + 0.02 = \textbf{0.04 S}

D\textsubscript{T} = 0.02 x 100 + 2 x 1 = 2 + 2 = \textbf{4}

Check: AD - BC = 4 x 4 - 400 x 0.04 = 16 - 16 = 0.

Note: Each section has AD - BC = 1 x 1 - 100 x 0.01 = 0 (degenerate case
--- the network is not reciprocal). The cascade of three such sections
also yields AD - BC = 0. Consistent.

\begin{enumerate}
\def\labelenumi{(\alph{enumi})}
\setcounter{enumi}{1}
\tightlist
\item
  Insertion loss: The voltage across the load with the cascade inserted:
\end{enumerate}

V₂/V\textsubscript{source} = Z\textsubscript{L} / ((A\textsubscript{T} +
B\textsubscript{T}/Z\textsubscript{L}) x Z\textsubscript{L} + \ldots{} )

Using the full formula: V₂ = V\textsubscript{S} x Z\textsubscript{L} /
(A\textsubscript{T}Z\textsubscript{L} + B\textsubscript{T} +
C\textsubscript{T}Z\textsubscript{S}Z\textsubscript{L} +
D\textsubscript{T}Z\textsubscript{S})

= V\textsubscript{S} x 600 / (4 x 600 + 400 + 0.04 x 600 x 600 + 4 x
600)

= V\textsubscript{S} x 600 / (2,400 + 400 + 14,400 + 2,400)

= V\textsubscript{S} x 600 / 19,600 = V\textsubscript{S} x 0.03061

Without the cascade (direct connection): V₂ = V\textsubscript{S} x
Z\textsubscript{L} / (Z\textsubscript{S} + Z\textsubscript{L}) =
V\textsubscript{S} x 600/1200 = 0.5 x V\textsubscript{S}

Insertion loss = 20 x log₁₀(0.5 / 0.03061) = 20 x log₁₀(16.33) = 20 x
1.213 = \textbf{24.3 dB}

\begin{center}\rule{0.5\linewidth}{0.5pt}\end{center}

\section{Problem 7.9.5}\label{problem-7.9.5}

\textbf{Given:} Two two-port networks are connected in parallel (port
voltages shared). Network A has Y-parameters: Y₁₁\textsubscript{A} =
0.05 S, Y₁₂\textsubscript{A} = -0.02 S, Y₂₁\textsubscript{A} = -0.02 S,
Y₂₂\textsubscript{A} = 0.04 S. Network B has Y-parameters:
Y₁₁\textsubscript{B} = 0.03 S, Y₁₂\textsubscript{B} = -0.01 S,
Y₂₁\textsubscript{B} = -0.01 S, Y₂₂\textsubscript{B} = 0.02 S.

\textbf{Find:} (a) The overall Y-parameters. (b) The voltage gain V₂/V₁
with port 2 terminated in a 20 ohm load.

\textbf{Solution:} (a) For parallel-connected two-ports, Y-parameters
add:

Y₁₁ = Y₁₁\textsubscript{A} + Y₁₁\textsubscript{B} = 0.05 + 0.03 =
\textbf{0.08 S}

Y₁₂ = Y₁₂\textsubscript{A} + Y₁₂\textsubscript{B} = -0.02 + (-0.01) =
\textbf{-0.03 S}

Y₂₁ = Y₂₁\textsubscript{A} + Y₂₁\textsubscript{B} = -0.02 + (-0.01) =
\textbf{-0.03 S}

Y₂₂ = Y₂₂\textsubscript{A} + Y₂₂\textsubscript{B} = 0.04 + 0.02 =
\textbf{0.06 S}

\begin{enumerate}
\def\labelenumi{(\alph{enumi})}
\setcounter{enumi}{1}
\tightlist
\item
  With a 20 ohm load at port 2: Y\textsubscript{L} = 1/20 = 0.05 S
\end{enumerate}

KCL at port 2 with the load: I₂ = Y₂₁V₁ + Y₂₂V₂, and I₂ =
-Y\textsubscript{L}V₂ (current into the load)

-0.05 V₂ = -0.03 V₁ + 0.06 V₂

-0.03 V₁ = -0.05 V₂ - 0.06 V₂ = -0.11 V₂

V₂/V₁ = 0.03/0.11 = \textbf{0.273 V/V} (-11.3 dB)

\begin{center}\rule{0.5\linewidth}{0.5pt}\end{center}

\chapter{Chapter 8 --- Section 8.1: Signals and
Systems}\label{chapter-8-section-8.1-signals-and-systems}

Practice problems covering signal classification, LTI systems,
convolution, correlation, the sampling theorem, and aliasing.

\begin{center}\rule{0.5\linewidth}{0.5pt}\end{center}

\section{Problem 8.1.1}\label{problem-8.1.1}

\textbf{Given:} A signal x(t) = 3sin(2π × 500t) + 7cos(2π × 1500t) is
measured across a 1 Ω resistor.

\textbf{Find:} (a) Whether the signal is periodic, and if so its
fundamental period, (b) the average power of each component, and (c) the
total average power.

\textbf{Solution:}

\begin{enumerate}
\def\labelenumi{(\alph{enumi})}
\item
  The first component has frequency f₁ = 500 Hz (period T₁ = 2 ms). The
  second has f₂ = 1500 Hz (period T₂ = 0.667 ms). Since f₂/f₁ = 1500/500
  = 3, a rational ratio, the composite signal is \textbf{periodic} with
  fundamental period T = T₁ = \textbf{2 ms} and fundamental frequency f₀
  = 500 Hz.
\item
  For sinusoidal signals into 1 Ω, P = A²/2: P₁ = 3²/2 = \textbf{4.5 W}
  P₂ = 7²/2 = \textbf{24.5 W}
\item
  Since the two components are at different frequencies (orthogonal),
  total power is the sum: P\textsubscript{total} = P₁ + P₂ = 4.5 + 24.5
  = \textbf{29.0 W}
\end{enumerate}

\begin{center}\rule{0.5\linewidth}{0.5pt}\end{center}

\section{Problem 8.1.2}\label{problem-8.1.2}

\textbf{Given:} An LTI system has impulse response h(t) =
10e\textsuperscript{-500t}u(t), where u(t) is the unit step function.

\textbf{Find:} (a) The frequency response H(ω), (b) the magnitude
\textbar H(ω)\textbar{} and phase ∠H(ω) at ω = 500 rad/s, and (c) the 3
dB bandwidth of the system.

\textbf{Solution:}

\begin{enumerate}
\def\labelenumi{(\alph{enumi})}
\item
  The Fourier transform of h(t) = ae\textsuperscript{-at}u(t) is H(ω) =
  a/(a + jω). With a = 500 and the leading coefficient of 10: H(ω) = 10
  × 500 / (500 + jω) = \textbf{5000 / (500 + jω)}
\item
  At ω = 500 rad/s: \textbar H(500)\textbar{} = 5000 / √(500² + 500²) =
  5000 / √(500,000) = 5000 / 707.1 = \textbf{7.07} ∠H(500) =
  -arctan(500/500) = -arctan(1) = \textbf{-45°}
\item
  The DC gain is \textbar H(0)\textbar{} = 5000/500 = 10. The 3 dB point
  occurs where \textbar H(ω)\textbar{} = 10/√2 = 7.07: 5000 / √(500² +
  ω²) = 7.07 √(250,000 + ω²) = 707.1 250,000 + ω² = 500,000
  ω\textsubscript{3dB} = √250,000 = 500 rad/s → f\textsubscript{3dB} =
  500/(2π) = \textbf{79.6 Hz}
\end{enumerate}

\begin{center}\rule{0.5\linewidth}{0.5pt}\end{center}

\section{Problem 8.1.3}\label{problem-8.1.3}

\textbf{Given:} Compute the continuous-time convolution y(t) = x(t) *
h(t) where x(t) = u(t) - u(t - 3) (a rectangular pulse of width 3 s and
height 1) and h(t) = 2e\textsuperscript{-t}u(t).

\textbf{Find:} The output y(t) for all values of t.

\textbf{Solution:}

y(t) = ∫₀\textsuperscript{∞} x(τ) × h(t - τ) dτ

For t \textless{} 0: x(τ) and h(t - τ) do not overlap, so y(t) = 0.

For 0 ≤ t \textless{} 3: x(τ) = 1 for 0 ≤ τ ≤ t, and h(t - τ) =
2e\textsuperscript{-(t-τ)}: y(t) = ∫₀\textsuperscript{t}
2e\textsuperscript{-(t-τ)} dτ = 2e\textsuperscript{-t}
∫₀\textsuperscript{t} e\textsuperscript{τ} dτ =
2e\textsuperscript{-t}(e\textsuperscript{t} - 1) = \textbf{2(1 -
e\textsuperscript{-t})}

For t ≥ 3: x(τ) = 1 for 0 ≤ τ ≤ 3: y(t) = ∫₀\textsuperscript{3}
2e\textsuperscript{-(t-τ)} dτ = 2e\textsuperscript{-t}
∫₀\textsuperscript{3} e\textsuperscript{τ} dτ =
2e\textsuperscript{-t}(e³ - 1) = \textbf{2(e³ - 1)e\textsuperscript{-t}
≈ 38.17e\textsuperscript{-t}}

The output rises exponentially toward a steady value of 2 during the
pulse, then decays exponentially after the pulse ends.

\begin{center}\rule{0.5\linewidth}{0.5pt}\end{center}

\section{Problem 8.1.4}\label{problem-8.1.4}

\textbf{Given:} Two discrete-time sequences are x{[}n{]} = \{2, -1, 3,
1\} for n = 0, 1, 2, 3 and h{[}n{]} = \{1, 2, 1\} for n = 0, 1, 2.

\textbf{Find:} (a) The convolution y{[}n{]} = x{[}n{]} * h{[}n{]}, and
(b) the total energy of the output signal.

\textbf{Solution:}

\begin{enumerate}
\def\labelenumi{(\alph{enumi})}
\tightlist
\item
  The output length is 4 + 3 - 1 = 6 samples. Apply y{[}n{]} = Σ
  x{[}k{]}h{[}n - k{]}:
\end{enumerate}

y{[}0{]} = x{[}0{]}h{[}0{]} = 2(1) = \textbf{2} y{[}1{]} =
x{[}0{]}h{[}1{]} + x{[}1{]}h{[}0{]} = 2(2) + (-1)(1) = 4 - 1 =
\textbf{3} y{[}2{]} = x{[}0{]}h{[}2{]} + x{[}1{]}h{[}1{]} +
x{[}2{]}h{[}0{]} = 2(1) + (-1)(2) + 3(1) = 2 - 2 + 3 = \textbf{3}
y{[}3{]} = x{[}1{]}h{[}2{]} + x{[}2{]}h{[}1{]} + x{[}3{]}h{[}0{]} =
(-1)(1) + 3(2) + 1(1) = -1 + 6 + 1 = \textbf{6} y{[}4{]} =
x{[}2{]}h{[}2{]} + x{[}3{]}h{[}1{]} = 3(1) + 1(2) = 3 + 2 = \textbf{5}
y{[}5{]} = x{[}3{]}h{[}2{]} = 1(1) = \textbf{1}

Therefore y{[}n{]} = \textbf{\{2, 3, 3, 6, 5, 1\}} for n = 0, 1, 2, 3,
4, 5.

\begin{enumerate}
\def\labelenumi{(\alph{enumi})}
\setcounter{enumi}{1}
\tightlist
\item
  Total energy: E = Σ \textbar y{[}n{]}\textbar² = 4 + 9 + 9 + 36 + 25 +
  1 = \textbf{84}
\end{enumerate}

\begin{center}\rule{0.5\linewidth}{0.5pt}\end{center}

\section{Problem 8.1.5}\label{problem-8.1.5}

\textbf{Given:} A sonar system transmits a pulse and receives an echo.
The cross-correlation R\textsubscript{xy}(τ) between the transmitted and
received signals peaks at a lag of τ = 12 ms. The speed of sound in
water is c = 1500 m/s.

\textbf{Find:} (a) The round-trip distance, (b) the one-way range to the
target, and (c) the minimum pulse duration required to distinguish two
targets separated by 5 m.

\textbf{Solution:}

\begin{enumerate}
\def\labelenumi{(\alph{enumi})}
\item
  Round-trip distance: d\textsubscript{round-trip} = c × τ = 1500 ×
  0.012 = \textbf{18 m}
\item
  One-way range: R = d\textsubscript{round-trip} / 2 = 18 / 2 =
  \textbf{9 m}
\item
  Two targets separated by ΔR = 5 m have a round-trip time difference
  of: Δτ = 2ΔR / c = 2 × 5 / 1500 = 6.667 ms
\end{enumerate}

To resolve them, the pulse duration T\textsubscript{p} must satisfy
T\textsubscript{p} ≤ Δτ: T\textsubscript{p} ≤ \textbf{6.67 ms}

A shorter pulse provides finer range resolution. The range resolution is
δR = cT\textsubscript{p}/2.

\begin{center}\rule{0.5\linewidth}{0.5pt}\end{center}

\section{Problem 8.1.6}\label{problem-8.1.6}

\textbf{Given:} A biomedical signal has a maximum frequency content of
150 Hz. The signal is sampled at f\textsubscript{s} = 500 Hz. Due to
electromagnetic interference, a 260 Hz component is also present in the
signal path.

\textbf{Find:} (a) The Nyquist rate for the 150 Hz signal, (b) the
Nyquist frequency at the chosen sampling rate, (c) whether the 260 Hz
interference will alias, and (d) the aliased frequency if it does.

\textbf{Solution:}

\begin{enumerate}
\def\labelenumi{(\alph{enumi})}
\item
  Nyquist rate: f\textsubscript{N} = 2 × f\textsubscript{max} = 2 × 150
  = \textbf{300 Hz}
\item
  Nyquist frequency at f\textsubscript{s} = 500 Hz:
  f\textsubscript{Nyquist} = f\textsubscript{s} / 2 = 500 / 2 =
  \textbf{250 Hz}
\item
  The 260 Hz interference exceeds the Nyquist frequency of 250 Hz, so
  \textbf{aliasing will occur} if not filtered before sampling.
\item
  Aliased frequency: f\textsubscript{alias} = f\textsubscript{s} -
  f\textsubscript{interference} = 500 - 260 = \textbf{240 Hz}
\end{enumerate}

This alias falls within the 0-250 Hz passband and will corrupt the
biomedical signal. An anti-aliasing lowpass filter with cutoff at 250 Hz
(or preferably 150 Hz to reject all content above the signal bandwidth)
must be placed before the ADC.

\begin{center}\rule{0.5\linewidth}{0.5pt}\end{center}

\section{Problem 8.1.7}\label{problem-8.1.7}

\textbf{Given:} A signal x(t) = 4cos(2π × 200t) + cos(2π × 800t) is
sampled at f\textsubscript{s} = 1000 Hz without an anti-aliasing filter.

\textbf{Find:} (a) The Nyquist frequency, (b) whether each component
aliases, (c) the frequencies that appear in the sampled signal, and (d)
a suitable sampling rate to avoid aliasing with a practical guard band
of 25\%.

\textbf{Solution:}

\begin{enumerate}
\def\labelenumi{(\alph{enumi})}
\item
  Nyquist frequency: f\textsubscript{Nyquist} = f\textsubscript{s} / 2 =
  1000 / 2 = \textbf{500 Hz}
\item
  The 200 Hz component is below 500 Hz, so it \textbf{does not alias}.
  The 800 Hz component exceeds 500 Hz, so it \textbf{aliases}.
\item
  The 200 Hz component appears at \textbf{200 Hz} in the sampled signal.
  The 800 Hz component aliases to: f\textsubscript{alias} =
  f\textsubscript{s} - 800 = 1000 - 800 = \textbf{200 Hz}. Both
  components appear at 200 Hz and \textbf{cannot be separated}. The
  sampled signal appears as a single 200 Hz tone with amplitude between
  3 and 5 (depending on phase alignment).
\item
  To sample both components without aliasing: f\textsubscript{s} ≥ 2 ×
  f\textsubscript{max} × 1.25 = 2 × 800 × 1.25 = \textbf{2000 Hz}
\end{enumerate}

\begin{center}\rule{0.5\linewidth}{0.5pt}\end{center}

\section{Problem 8.1.8}\label{problem-8.1.8}

\textbf{Given:} An LTI system has impulse response h{[}n{]} = \{1,
-0.5\} for n = 0, 1. The input is x{[}n{]} = cos(πn/4) for n = 0, 1, 2,
\ldots, which is a discrete-time sinusoid at normalized frequency ω₀ =
π/4 rad/sample.

\textbf{Find:} (a) The frequency response H(e\textsuperscript{jω}) of
the system, (b) the magnitude and phase at ω = π/4, and (c) the
steady-state output.

\textbf{Solution:}

\begin{enumerate}
\def\labelenumi{(\alph{enumi})}
\item
  The frequency response of an FIR filter with h{[}n{]} = \{h₀, h₁\} is:
  H(e\textsuperscript{jω}) = h₀ + h₁e\textsuperscript{-jω} = 1 -
  0.5e\textsuperscript{-jω}
\item
  At ω = π/4: H(e\textsuperscript{jπ/4}) = 1 -
  0.5e\textsuperscript{-jπ/4} = 1 - 0.5(cos(π/4) - jsin(π/4)) = 1 -
  0.5(0.7071 - j0.7071) = 1 - 0.3536 + j0.3536 = 0.6464 + j0.3536
\end{enumerate}

\textbar H(e\textsuperscript{jπ/4})\textbar{} = √(0.6464² + 0.3536²) =
√(0.4178 + 0.1250) = √0.5428 = \textbf{0.737}
∠H(e\textsuperscript{jπ/4}) = arctan(0.3536 / 0.6464) = arctan(0.5472) =
\textbf{28.7°}

\begin{enumerate}
\def\labelenumi{(\alph{enumi})}
\setcounter{enumi}{2}
\tightlist
\item
  For an LTI system with sinusoidal input, the steady-state output is:
  y{[}n{]} = \textbar H\textbar{} × cos(ω₀n + ∠H) = \textbf{0.737
  cos(πn/4 + 28.7°)}
\end{enumerate}

\begin{center}\rule{0.5\linewidth}{0.5pt}\end{center}

\section{Problem 8.1.9}\label{problem-8.1.9}

\textbf{Given:} A discrete-time signal x{[}n{]} has total energy
E\textsubscript{x} = 50. Its autocorrelation R\textsubscript{xx}{[}0{]}
is measured, and the autocorrelation at lag 1 is
R\textsubscript{xx}{[}1{]} = 30.

\textbf{Find:} (a) The value of R\textsubscript{xx}{[}0{]}, (b) the
normalized autocorrelation coefficient at lag 1, and (c) whether the
signal has significant correlation between adjacent samples.

\textbf{Solution:}

\begin{enumerate}
\def\labelenumi{(\alph{enumi})}
\item
  By definition, the autocorrelation at zero lag equals the signal
  energy: R\textsubscript{xx}{[}0{]} = E\textsubscript{x} = \textbf{50}
\item
  The normalized autocorrelation coefficient: ρ{[}1{]} =
  R\textsubscript{xx}{[}1{]} / R\textsubscript{xx}{[}0{]} = 30 / 50 =
  \textbf{0.6}
\item
  The normalized autocorrelation coefficient of 0.6 indicates
  \textbf{significant positive correlation} between adjacent samples.
  This means the signal changes slowly relative to the sampling rate,
  with each sample being moderately predictable from its neighbor. A
  value of 0.6 is typical of oversampled signals or signals with most
  energy at low frequencies. This correlation can be exploited for data
  compression (predictive coding) or noise reduction (averaging).
\end{enumerate}

\begin{center}\rule{0.5\linewidth}{0.5pt}\end{center}

\section{Problem 8.1.10}\label{problem-8.1.10}

\textbf{Given:} A bandpass signal is centered at a carrier frequency
f\textsubscript{c} = 70 MHz with a bandwidth of B = 5 MHz (occupying
67.5 to 72.5 MHz). Instead of sampling at the Nyquist rate of 2 × 72.5 =
145 MHz, bandpass sampling is used.

\textbf{Find:} (a) The minimum bandpass sampling rate, (b) a valid
sampling rate that avoids spectral overlap, and (c) the frequency shift
of the signal in the sampled spectrum.

\textbf{Solution:}

\begin{enumerate}
\def\labelenumi{(\alph{enumi})}
\item
  The minimum bandpass sampling rate is: f\textsubscript{s,min} = 2B = 2
  × 5 = \textbf{10 MHz}
\item
  For bandpass sampling, the sampling rate must satisfy
  f\textsubscript{s} = 2f\textsubscript{upper}/m for integer m, where
  the spectral replicas do not overlap. With f\textsubscript{upper} =
  72.5 MHz:
\end{enumerate}

Check m = 5: f\textsubscript{s} = 2 × 72.5 / 5 = 29.0 MHz Verify: the
signal at 67.5-72.5 MHz replicates. The baseband image is at
f\textsubscript{s} - f\textsubscript{upper} = 29.0 - 72.5 = \ldots{}
Using the formula, the folded band: 72.5 mod 29.0 = 72.5 - 2(29.0) =
14.5 MHz (upper edge), and 67.5 mod 29.0 = 67.5 - 2(29.0) = 9.5 MHz
(lower edge). The band 9.5-14.5 MHz fits within 0 to
f\textsubscript{s}/2 = 14.5 MHz with no overlap. A valid sampling rate
is \textbf{f\textsubscript{s} = 29.0 MHz}.

\begin{enumerate}
\def\labelenumi{(\alph{enumi})}
\setcounter{enumi}{2}
\tightlist
\item
  In the sampled spectrum, the signal occupies 9.5 to 14.5 MHz
  (bandwidth still 5 MHz). The center frequency shifts from 70 MHz to
  (9.5 + 14.5)/2 = \textbf{12.0 MHz}. This is a reduction from 145 MHz
  Nyquist rate to 29 MHz, a factor of \textbf{5× reduction} in sampling
  rate.
\end{enumerate}

\chapter{Chapter 8 --- Section 8.2: Fourier
Analysis}\label{chapter-8-section-8.2-fourier-analysis}

Practice problems covering Fourier series, Fourier transform, DFT, FFT,
DCT, Hilbert transform, and the Goertzel algorithm.

\begin{center}\rule{0.5\linewidth}{0.5pt}\end{center}

\section{Problem 8.2.1}\label{problem-8.2.1}

\textbf{Given:} A periodic sawtooth wave has a peak-to-peak amplitude of
10 V (ranging from -5 V to +5 V) and a fundamental frequency of 200 Hz.

\textbf{Find:} (a) The Fourier series coefficients for the first four
harmonics, (b) the DC component, and (c) the fraction of total power
contained in the fundamental and first three harmonics.

\textbf{Solution:}

\begin{enumerate}
\def\labelenumi{(\alph{enumi})}
\tightlist
\item
  A sawtooth wave with amplitude A = 5 V has the Fourier series: x(t) =
  Σ (2A/(nπ)) × (-1)\textsuperscript{n+1} × sin(2πnf₀t). The
  coefficients bₙ = 2A/(nπ) × (-1)\textsuperscript{n+1}:
\end{enumerate}

b₁ = 2(5)/(1π) × 1 = 10/π = \textbf{3.183 V} (at 200 Hz) b₂ = 2(5)/(2π)
× (-1) = -10/(2π) = \textbf{-1.592 V} (at 400 Hz) b₃ = 2(5)/(3π) × 1 =
10/(3π) = \textbf{1.061 V} (at 600 Hz) b₄ = 2(5)/(4π) × (-1) = -10/(4π)
= \textbf{-0.796 V} (at 800 Hz)

\begin{enumerate}
\def\labelenumi{(\alph{enumi})}
\setcounter{enumi}{1}
\item
  The sawtooth wave is an odd function symmetric about zero, so the DC
  component is: a₀ = \textbf{0 V}
\item
  Power of each harmonic (into 1 Ω): P\textsubscript{n} = bₙ²/2. P₁ =
  3.183²/2 = 5.066 W P₂ = 1.592²/2 = 1.267 W P₃ = 1.061²/2 = 0.563 W P₄
  = 0.796²/2 = 0.317 W Sum = 7.213 W
\end{enumerate}

Total power: P\textsubscript{total} = A²/3 = 25/3 = 8.333 W (for a
sawtooth). Fraction: 7.213 / 8.333 = \textbf{86.6\%}

\begin{center}\rule{0.5\linewidth}{0.5pt}\end{center}

\section{Problem 8.2.2}\label{problem-8.2.2}

\textbf{Given:} A Gaussian pulse is defined as x(t) =
e\textsuperscript{-πt²/τ²} with τ = 2 ms.

\textbf{Find:} (a) The Fourier transform X(f), (b) the 3 dB bandwidth
(half-power bandwidth), and (c) the time-bandwidth product.

\textbf{Solution:}

\begin{enumerate}
\def\labelenumi{(\alph{enumi})}
\item
  The Fourier transform of a Gaussian is also a Gaussian: X(f) = τ ×
  e\textsuperscript{-πτ²f²} = 0.002 × e\textsuperscript{-π(0.002)²f²} =
  \textbf{0.002 × e\textsuperscript{-π × 4 × 10⁻⁶ × f²}}
\item
  The 3 dB bandwidth is where \textbar X(f)\textbar² =
  \textbar X(0)\textbar²/2: e\textsuperscript{-2πτ²f²} = 0.5 -2πτ²f² =
  ln(0.5) = -0.6931 f\textsubscript{3dB} = √(0.6931 / (2π × (0.002)²)) =
  √(0.6931 / (2.513 × 10⁻⁵)) f\textsubscript{3dB} = √(27,573) =
  \textbf{166.1 Hz}
\end{enumerate}

The two-sided 3 dB bandwidth is 2 × 166.1 = 332.2 Hz.

\begin{enumerate}
\def\labelenumi{(\alph{enumi})}
\setcounter{enumi}{2}
\tightlist
\item
  Time-bandwidth product: The pulse duration at the 3 dB level: setting
  e\textsuperscript{-πt²/τ²} = 1/√2 gives t\textsubscript{3dB} =
  τ√(ln2/π) = 0.002 × √(0.2206) = 0.002 × 0.4697 = 0.939 ms. Two-sided
  pulse width = 2 × 0.939 = 1.879 ms. TBP = 1.879 × 10⁻³ × 332.2 =
  \textbf{0.624}
\end{enumerate}

This is close to the theoretical minimum of 0.44 (for a Gaussian),
confirming that Gaussian pulses have near-optimal time-bandwidth
product.

\begin{center}\rule{0.5\linewidth}{0.5pt}\end{center}

\section{Problem 8.2.3}\label{problem-8.2.3}

\textbf{Given:} A signal consists of two tones at 1000 Hz and 1100 Hz,
sampled at f\textsubscript{s} = 10,000 Hz. A 200-point DFT is computed.

\textbf{Find:} (a) The frequency resolution, (b) the DFT bin indices for
each tone, (c) whether the two tones can be resolved, and (d) the
minimum DFT length to resolve the two tones with a Hanning window.

\textbf{Solution:}

\begin{enumerate}
\def\labelenumi{(\alph{enumi})}
\item
  Frequency resolution: Δf = f\textsubscript{s} / N = 10,000 / 200 =
  \textbf{50 Hz}
\item
  Bin indices: k₁ = f₁ / Δf = 1000 / 50 = \textbf{20} k₂ = f₂ / Δf =
  1100 / 50 = \textbf{22}
\item
  The two tones are separated by 100 Hz = 2 bins. With a rectangular
  window (main lobe width = 2 bins), the tones are \textbf{barely
  resolvable} --- they fall in adjacent main lobes with a visible dip
  between them.
\item
  A Hanning window has a main lobe width of approximately 4 bins. To
  resolve two tones separated by Δf\textsubscript{sep} = 100 Hz, we need
  4 × (f\textsubscript{s}/N) ≤ 100: N ≥ 4 × f\textsubscript{s} / 100 = 4
  × 10,000 / 100 = \textbf{400 points}
\end{enumerate}

With N = 400: Δf = 25 Hz, and the Hanning main lobe width = 4 × 25 = 100
Hz, just sufficient to resolve the two tones.

\begin{center}\rule{0.5\linewidth}{0.5pt}\end{center}

\section{Problem 8.2.4}\label{problem-8.2.4}

\textbf{Given:} A real-time spectrum analyzer processes audio data at
f\textsubscript{s} = 48 kHz using overlapping FFT frames. Each frame is
2048 samples with a Hanning window and 75\% overlap.

\textbf{Find:} (a) The number of complex multiplications per FFT frame,
(b) the hop size in samples and milliseconds, (c) the number of FFT
frames per second, and (d) the total computational rate in millions of
multiplications per second (MMPS).

\textbf{Solution:}

\begin{enumerate}
\def\labelenumi{(\alph{enumi})}
\item
  Radix-2 FFT multiplications: Mults = (N/2) × log₂(N) = (2048/2) ×
  log₂(2048) = 1024 × 11 = \textbf{11,264 complex multiplications per
  frame}
\item
  Hop size with 75\% overlap: Hop = N × (1 - 0.75) = 2048 × 0.25 =
  \textbf{512 samples} = 512 / 48,000 = \textbf{10.67 ms}
\item
  Frames per second: Frames/s = f\textsubscript{s} / hop = 48,000 / 512
  = \textbf{93.75 frames/s}
\item
  Total computational rate: Rate = 11,264 × 93.75 = \textbf{1,056,000
  complex multiplications/s} Each complex multiplication requires 4 real
  multiplications and 2 real additions, so: Real multiplications =
  1,056,000 × 4 = 4,224,000 = \textbf{4.22 MMPS}
\end{enumerate}

This is well within the capability of even low-cost embedded processors.

\begin{center}\rule{0.5\linewidth}{0.5pt}\end{center}

\section{Problem 8.2.5}\label{problem-8.2.5}

\textbf{Given:} An 8×8 block of pixel values from a grayscale image is
transformed using the 2-D DCT. The resulting DCT coefficient matrix has
these magnitudes: DC coefficient = 1200, the next 5 largest coefficients
have magnitudes of 85, 62, 41, 28, and 15, and the remaining 58
coefficients all have magnitudes below 10.

\textbf{Find:} (a) The total number of DCT coefficients, (b) the
compression ratio if coefficients below magnitude 10 are quantized to
zero, (c) the percentage of energy retained, and (d) the JPEG quality
implication.

\textbf{Solution:}

\begin{enumerate}
\def\labelenumi{(\alph{enumi})}
\item
  Total DCT coefficients in an 8×8 block: N = 8 × 8 = \textbf{64
  coefficients}
\item
  Coefficients retained: DC (1) + 5 significant AC + 0 below threshold =
  \textbf{6 nonzero coefficients}. Compression ratio = 64 / 6 =
  \textbf{10.7:1}
\item
  Energy retained: E\textsubscript{retained} = 1200² + 85² + 62² + 41² +
  28² + 15² = 1,440,000 + 7,225 + 3,844 + 1,681 + 784 + 225 = 1,453,759
  E\textsubscript{discarded} ≤ 58 × 10² = 5,800 (upper bound, assuming
  all are at magnitude 10) E\textsubscript{total} ≈ 1,453,759 + 5,800 =
  1,459,559 Fraction retained = 1,453,759 / 1,459,559 = \textbf{99.6\%}
\item
  Retaining 99.6\% of energy with 10.7:1 compression corresponds to a
  \textbf{high-quality JPEG} setting (approximately quality 85-90). The
  dominant DC coefficient carries 98.7\% of the energy alone, which is
  typical for smooth image regions. Textured regions would have more
  energy spread across AC coefficients.
\end{enumerate}

\begin{center}\rule{0.5\linewidth}{0.5pt}\end{center}

\section{Problem 8.2.6}\label{problem-8.2.6}

\textbf{Given:} A signal x(t) = cos(2π × 100t) × cos(2π × 5000t)
represents a double-sideband suppressed carrier (DSB-SC) AM signal with
carrier f\textsubscript{c} = 5000 Hz and modulating frequency
f\textsubscript{m} = 100 Hz.

\textbf{Find:} (a) The frequency components present in x(t), (b) the
Hilbert transform x̂(t), (c) the analytic signal z(t), and (d) the
instantaneous envelope A(t).

\textbf{Solution:}

\begin{enumerate}
\def\labelenumi{(\alph{enumi})}
\item
  Using the product-to-sum identity: cos(A)cos(B) = 0.5{[}cos(A-B) +
  cos(A+B){]}: x(t) = 0.5cos(2π × 4900t) + 0.5cos(2π × 5100t) The signal
  contains frequencies at \textbf{4900 Hz} and \textbf{5100 Hz}.
\item
  The Hilbert transform shifts each frequency component by -90°: x̂(t) =
  0.5sin(2π × 4900t) + 0.5sin(2π × 5100t) Using the sum-to-product
  identity: x̂(t) = cos(2π × 100t) × sin(2π × 5000t) = \textbf{sin(2π ×
  5000t) × cos(2π × 100t)}
\item
  The analytic signal: z(t) = x(t) + jx̂(t) = cos(2π × 100t){[}cos(2π ×
  5000t) + jsin(2π × 5000t){]} z(t) = \textbf{cos(2π × 100t) ×
  e\textsuperscript{j2π×5000t}}
\item
  The instantaneous envelope: A(t) = \textbar z(t)\textbar{} =
  \textbar cos(2π × 100t)\textbar{} ×
  \textbar e\textsuperscript{j2π×5000t}\textbar{} =
  \textbf{\textbar cos(2π × 100t)\textbar{}}
\end{enumerate}

The envelope varies between 0 and 1 at twice the modulation frequency
(200 Hz) due to the absolute value, which is characteristic of DSB-SC
(the envelope goes through zero, unlike standard AM).

\begin{center}\rule{0.5\linewidth}{0.5pt}\end{center}

\section{Problem 8.2.7}\label{problem-8.2.7}

\textbf{Given:} A DTMF decoder must detect the digit ``9,'' which
produces tones at 852 Hz and 1477 Hz. The system samples at
f\textsubscript{s} = 8000 Hz using an N = 205 sample block.

\textbf{Find:} (a) The Goertzel bin indices for 852 Hz and 1477 Hz, (b)
the actual frequencies corresponding to those bins, (c) the frequency
error for each tone, and (d) the number of real multiplications for the
Goertzel algorithm versus a 256-point FFT.

\textbf{Solution:}

\begin{enumerate}
\def\labelenumi{(\alph{enumi})}
\item
  Bin indices: k₁ = round(852 × 205 / 8000) = round(21.83) = \textbf{22}
  k₂ = round(1477 × 205 / 8000) = round(37.85) = \textbf{38}
\item
  Actual frequencies at those bins: f₁ = 22 × 8000 / 205 = \textbf{858.5
  Hz} f₂ = 38 × 8000 / 205 = \textbf{1482.9 Hz}
\item
  Frequency errors: Error₁ = 858.5 - 852 = 6.5 Hz → 6.5/852 × 100 =
  \textbf{0.76\%} Error₂ = 1482.9 - 1477 = 5.9 Hz → 5.9/1477 × 100 =
  \textbf{0.40\%}
\end{enumerate}

Both errors are well within the DTMF detection tolerance of ±1.5\%.

\begin{enumerate}
\def\labelenumi{(\alph{enumi})}
\setcounter{enumi}{3}
\tightlist
\item
  Computational cost: Goertzel for 2 bins: 2 × N = 2 × 205 = \textbf{410
  real multiplications} (plus 2 complex multiplications for the final
  output = 8 more → 418 total) In the full DTMF detector (8 bins): 8 ×
  205 + 32 = \textbf{1,672 real multiplications} 256-point FFT: (N/2) ×
  log₂(N) = 128 × 8 = 1,024 complex multiplications = \textbf{4,096 real
  multiplications} Goertzel is \textbf{2.45× more efficient} for
  detecting 8 specific frequencies.
\end{enumerate}

\begin{center}\rule{0.5\linewidth}{0.5pt}\end{center}

\section{Problem 8.2.8}\label{problem-8.2.8}

\textbf{Given:} A periodic triangular wave has amplitude A = 3 V and
period T = 5 ms. The Fourier series contains only odd harmonics with
coefficients aₙ = 8A/(n²π²) for odd n.

\textbf{Find:} (a) The fundamental frequency, (b) the amplitudes of the
first three nonzero harmonics, (c) the total harmonic distortion (THD)
considering only these three harmonics, and (d) the Fourier series
approximation at t = T/4.

\textbf{Solution:}

\begin{enumerate}
\def\labelenumi{(\alph{enumi})}
\item
  Fundamental frequency: f₀ = 1/T = 1/0.005 = \textbf{200 Hz}
\item
  Harmonic amplitudes (odd n only): a₁ = 8(3)/(1² × π²) = 24/π² =
  24/9.8696 = \textbf{2.432 V} (at 200 Hz) a₃ = 8(3)/(9 × π²) = 24/(9 ×
  9.8696) = \textbf{0.270 V} (at 600 Hz) a₅ = 8(3)/(25 × π²) = 24/(25 ×
  9.8696) = \textbf{0.097 V} (at 1000 Hz)
\item
  THD considering these harmonics: THD = √(a₃² + a₅²) / a₁ = √(0.0730 +
  0.00944) / 2.432 = √0.0824 / 2.432 THD = 0.287 / 2.432 =
  \textbf{11.8\%}
\item
  At t = T/4 = 1.25 ms, the triangular wave reaches its peak of A = 3 V.
  Fourier approximation: x(T/4) = a₁sin(π/2) + a₃sin(3π/2) + a₅sin(5π/2)
  = 2.432(1) + 0.270(-1) + 0.097(1) = 2.432 - 0.270 + 0.097 =
  \textbf{2.259 V}
\end{enumerate}

The 3-term approximation gives 2.259 V versus the exact value of 3.0 V,
showing 24.7\% error at the peak --- more terms are needed for accurate
peak representation.

\begin{center}\rule{0.5\linewidth}{0.5pt}\end{center}

\section{Problem 8.2.9}\label{problem-8.2.9}

\textbf{Given:} A rectangular pulse x(t) of width τ = 0.5 ms and
amplitude A = 2 V is centered at t = 0.

\textbf{Find:} (a) The Fourier transform X(f), (b) the spectral null
frequencies, (c) the 3 dB bandwidth of the main lobe, and (d) the
fraction of total energy in the main lobe.

\textbf{Solution:}

\begin{enumerate}
\def\labelenumi{(\alph{enumi})}
\item
  The Fourier transform of a rectangular pulse of width τ and amplitude
  A: X(f) = Aτ × sinc(fτ) = 2 × 0.0005 × sinc(0.0005f) = \textbf{0.001 ×
  sinc(0.0005f) V·s}
\item
  Spectral nulls occur where sinc(fτ) = 0, i.e., fτ = n for integer n ≠
  0: f\textsubscript{null,n} = n/τ = n/0.0005 = 2000n Hz First null:
  \textbf{2000 Hz}, second null: \textbf{4000 Hz}, third null:
  \textbf{6000 Hz}
\item
  The 3 dB bandwidth of the main lobe: sinc(fτ) = 1/√2 occurs at
  approximately fτ = 0.4429: f\textsubscript{3dB} = 0.4429/τ =
  0.4429/0.0005 = \textbf{885.8 Hz} Two-sided 3 dB bandwidth = 2 × 885.8
  = \textbf{1,771.6 Hz}
\item
  The fraction of total energy in the main lobe (between
  ±f\textsubscript{null,1} = ±2000 Hz): For a sinc spectrum, the main
  lobe contains approximately \textbf{90.3\%} of the total energy (this
  is a well-known property of the sinc function --- the integral of
  sinc²(x) from -1 to 1 divided by the integral from -∞ to ∞ equals
  0.903).
\end{enumerate}

\begin{center}\rule{0.5\linewidth}{0.5pt}\end{center}

\section{Problem 8.2.10}\label{problem-8.2.10}

\textbf{Given:} A 512-point FFT is used to analyze a signal sampled at
f\textsubscript{s} = 20 kHz. The signal contains a 3 kHz tone at 0 dBFS
and a 3.5 kHz tone at -40 dBFS. A rectangular window is applied.

\textbf{Find:} (a) The frequency resolution, (b) the bin indices for
both tones, (c) whether the weaker tone is detectable given the
rectangular window's sidelobe level of -13 dB, and (d) a window choice
that would allow detection.

\textbf{Solution:}

\begin{enumerate}
\def\labelenumi{(\alph{enumi})}
\item
  Frequency resolution: Δf = f\textsubscript{s} / N = 20,000 / 512 =
  \textbf{39.06 Hz}
\item
  Bin indices: k₁ = 3000 / 39.06 = 76.8 → nearest bin \textbf{77}
  (actual f = 3007.8 Hz) k₂ = 3500 / 39.06 = 89.6 → nearest bin
  \textbf{90} (actual f = 3515.6 Hz)
\end{enumerate}

The two tones are separated by 90 - 77 = 13 bins.

\begin{enumerate}
\def\labelenumi{(\alph{enumi})}
\setcounter{enumi}{2}
\item
  The rectangular window has its highest sidelobe at -13 dB. The
  sidelobes decay at approximately 6 dB/octave. At 13 bins from the
  strong tone's peak: Sidelobe level ≈ -13 - 20log₁₀(13) ≈ -13 - 22.3 =
  -35.3 dB Since the weak tone is at -40 dB relative to the strong tone,
  and the sidelobe leakage at that bin is only -35.3 dB, the
  \textbf{weak tone is masked} by the spectral leakage from the strong
  tone (it falls within the sidelobe floor).
\item
  A \textbf{Blackman} window has sidelobes at -58 dB, which is well
  below the -40 dB weak tone level, allowing detection. A
  \textbf{Hamming} window (-43 dB sidelobes) would marginally work. A
  Blackman-Harris window (-92 dB sidelobes) provides the most margin.
\end{enumerate}

The trade-off: the Blackman window widens the main lobe to approximately
6 bins (234 Hz), but since the tones are 13 bins apart, resolution is
not an issue.

\chapter{Chapter 8 --- Section 8.3: Laplace
Transform}\label{chapter-8-section-8.3-laplace-transform}

Practice problems covering the Laplace transform definition, properties,
transfer functions, inverse Laplace transform, and s-domain circuit
analysis.

\begin{center}\rule{0.5\linewidth}{0.5pt}\end{center}

\section{Problem 8.3.1}\label{problem-8.3.1}

\textbf{Given:} A signal x(t) = 5te\textsuperscript{-4t}u(t), where u(t)
is the unit step function.

\textbf{Find:} (a) The Laplace transform X(s), (b) the region of
convergence, and (c) the initial value x(0⁺) and the final value x(∞)
using the value theorems.

\textbf{Solution:}

\begin{enumerate}
\def\labelenumi{(\alph{enumi})}
\item
  Using the transform pair L\{te\textsuperscript{-at}u(t)\} = 1/(s +
  a)²: X(s) = 5 × 1/(s + 4)² = \textbf{5 / (s + 4)²}
\item
  The region of convergence is: \textbf{Re\{s\} \textgreater{} -4}
\item
  Initial value theorem: x(0⁺) = lim\textsubscript{s→∞} sX(s) =
  lim\textsubscript{s→∞} 5s/(s + 4)² = lim\textsubscript{s→∞} 5s/(s² +
  8s + 16) = \textbf{0} (Confirmed: x(0) = 5 × 0 × e⁰ = 0.)
\end{enumerate}

Final value theorem: x(∞) = lim\textsubscript{s→0} sX(s) =
lim\textsubscript{s→0} 5s/(s + 4)² = 0/16 = \textbf{0} (Confirmed: as t
→ ∞, te\textsuperscript{-4t} → 0 since the exponential decay dominates.)

\begin{center}\rule{0.5\linewidth}{0.5pt}\end{center}

\section{Problem 8.3.2}\label{problem-8.3.2}

\textbf{Given:} A system has the transfer function H(s) = 50(s + 2) /
(s² + 6s + 25).

\textbf{Find:} (a) The poles and zeros, (b) whether the system is
stable, (c) the natural frequency ω\textsubscript{n} and damping ratio
ζ, (d) the DC gain, and (e) the peak resonant frequency.

\textbf{Solution:}

\begin{enumerate}
\def\labelenumi{(\alph{enumi})}
\item
  Zeros: s + 2 = 0 → z₁ = \textbf{-2} Poles: s² + 6s + 25 = 0 → s = (-6
  ± √(36 - 100))/2 = (-6 ± √(-64))/2 = (-6 ± j8)/2 p₁ = \textbf{-3 +
  j4}, p₂ = \textbf{-3 - j4}
\item
  Both poles have negative real parts (Re = -3 \textless{} 0), so the
  system is \textbf{stable}.
\item
  From the standard form s² + 2ζω\textsubscript{n}s +
  ω\textsubscript{n}²: ω\textsubscript{n}² = 25 → ω\textsubscript{n} =
  \textbf{5 rad/s} 2ζω\textsubscript{n} = 6 → ζ = 6/(2 × 5) =
  \textbf{0.6}
\item
  DC gain: H(0) = 50(2)/25 = 100/25 = \textbf{4.0}
\item
  The peak resonant frequency for an underdamped second-order system:
  ω\textsubscript{r} = ω\textsubscript{n}√(1 - 2ζ²) = 5√(1 - 2(0.36)) =
  5√(0.28) = 5 × 0.5292 = \textbf{2.65 rad/s} (f\textsubscript{r} =
  0.421 Hz)
\end{enumerate}

Since 1 - 2ζ² = 0.28 \textgreater{} 0, a resonant peak exists.

\begin{center}\rule{0.5\linewidth}{0.5pt}\end{center}

\section{Problem 8.3.3}\label{problem-8.3.3}

\textbf{Given:} Find the inverse Laplace transform of X(s) = (7s + 11) /
(s² + 5s + 6).

\textbf{Find:} The time-domain signal x(t) for t ≥ 0.

\textbf{Solution:}

Factor the denominator: s² + 5s + 6 = (s + 2)(s + 3).

Partial fraction expansion: X(s) = A/(s + 2) + B/(s + 3). 7s + 11 = A(s
+ 3) + B(s + 2)

Setting s = -2: 7(-2) + 11 = A(1) → A = -14 + 11 = \textbf{-3} Setting s
= -3: 7(-3) + 11 = B(-1) → -21 + 11 = -B → B = \textbf{10}

X(s) = -3/(s + 2) + 10/(s + 3)

Inverse transform: x(t) = \textbf{-3e\textsuperscript{-2t} +
10e\textsuperscript{-3t}} for t ≥ 0

Verification: x(0) = -3 + 10 = 7. From the initial value theorem: sX(s)
as s → ∞: s(7s + 11)/(s² + 5s + 6) → 7. Confirmed.

\begin{center}\rule{0.5\linewidth}{0.5pt}\end{center}

\section{Problem 8.3.4}\label{problem-8.3.4}

\textbf{Given:} A series RL circuit has R = 200 Ω and L = 50 mH. A step
voltage V\textsubscript{in}(t) = 24u(t) V is applied with zero initial
current.

\textbf{Find:} (a) The transfer function H(s) =
I(s)/V\textsubscript{in}(s), (b) the current I(s) in the s-domain, (c)
the time-domain current i(t), and (d) the time constant and the time to
reach 95\% of steady-state current.

\textbf{Solution:}

\begin{enumerate}
\def\labelenumi{(\alph{enumi})}
\item
  The s-domain impedance is Z(s) = R + sL = 200 + 0.05s. H(s) =
  I(s)/V\textsubscript{in}(s) = 1/Z(s) = 1/(0.05s + 200) = \textbf{20/(s
  + 4000)}
\item
  With V\textsubscript{in}(s) = 24/s: I(s) = H(s) ×
  V\textsubscript{in}(s) = 20 × 24 / {[}s(s + 4000){]} = 480 / {[}s(s +
  4000){]}
\end{enumerate}

Partial fractions: 480/{[}s(s + 4000){]} = A/s + B/(s + 4000) A =
480/4000 = 0.12, B = 480/(-4000) = -0.12 I(s) = 0.12/s - 0.12/(s + 4000)

\begin{enumerate}
\def\labelenumi{(\alph{enumi})}
\setcounter{enumi}{2}
\item
  Inverse transform: i(t) = 0.12(1 - e\textsuperscript{-4000t}) A =
  \textbf{120(1 - e\textsuperscript{-4000t}) mA} for t ≥ 0
\item
  Time constant: τ = L/R = 0.05/200 = \textbf{250 μs} Time to 95\% of
  steady state: t = 3τ = 3 × 250 = \textbf{750 μs} Steady-state current:
  I\textsubscript{ss} = V/R = 24/200 = \textbf{120 mA}
\end{enumerate}

\begin{center}\rule{0.5\linewidth}{0.5pt}\end{center}

\section{Problem 8.3.5}\label{problem-8.3.5}

\textbf{Given:} A second-order system has the transfer function H(s) =
900 / (s² + 12s + 900). A unit step input is applied.

\textbf{Find:} (a) The natural frequency and damping ratio, (b) the
damped natural frequency, (c) the percent overshoot of the step
response, and (d) the settling time (2\% criterion).

\textbf{Solution:}

\begin{enumerate}
\def\labelenumi{(\alph{enumi})}
\item
  Comparing with the standard form ω\textsubscript{n}²/(s² +
  2ζω\textsubscript{n}s + ω\textsubscript{n}²): ω\textsubscript{n}² =
  900 → ω\textsubscript{n} = \textbf{30 rad/s} (f\textsubscript{n} =
  4.77 Hz) 2ζω\textsubscript{n} = 12 → ζ = 12/60 = \textbf{0.2}
\item
  Damped natural frequency: ω\textsubscript{d} = ω\textsubscript{n}√(1 -
  ζ²) = 30√(1 - 0.04) = 30√0.96 = 30 × 0.9798 = \textbf{29.4 rad/s}
\item
  Percent overshoot: \%OS = 100 × e\textsuperscript{-πζ/√(1-ζ²)} = 100 ×
  e\textsuperscript{-π(0.2)/√(0.96)} = 100 ×
  e\textsuperscript{-0.6283/0.9798} = 100 × e\textsuperscript{-0.6413} =
  100 × 0.5267 = \textbf{52.7\%}
\item
  Settling time (2\% criterion): t\textsubscript{s} =
  4/(ζω\textsubscript{n}) = 4/(0.2 × 30) = 4/6 = \textbf{0.667 s}
\end{enumerate}

The low damping ratio (ζ = 0.2) results in significant overshoot and
oscillation before settling.

\begin{center}\rule{0.5\linewidth}{0.5pt}\end{center}

\section{Problem 8.3.6}\label{problem-8.3.6}

\textbf{Given:} A parallel RLC circuit has R = 10 kΩ, L = 100 mH, and C
= 100 nF. The input is a current step I\textsubscript{in}(t) = 1 mA ×
u(t) with zero initial conditions.

\textbf{Find:} (a) The transfer function H(s) =
V(s)/I\textsubscript{in}(s), (b) the resonant frequency, (c) the quality
factor Q, and (d) the bandwidth.

\textbf{Solution:}

\begin{enumerate}
\def\labelenumi{(\alph{enumi})}
\tightlist
\item
  The parallel impedance: Z(s) = 1/(1/R + 1/sL + sC) = 1/(sC + 1/R +
  1/(sL)) = s/(s²C + s/R + 1/L) = (s/C)/(s² + s/(RC) + 1/(LC))
\end{enumerate}

H(s) = V(s)/I\textsubscript{in}(s) = Z(s) = \textbf{(s/C) / (s² + s/(RC)
+ 1/(LC))}

Substituting values: 1/(RC) = 1/(10⁴ × 10⁻⁷) = 1000, 1/(LC) = 1/(0.1 ×
10⁻⁷) = 10⁸ H(s) = \textbf{(10⁷s) / (s² + 1000s + 10⁸)}

\begin{enumerate}
\def\labelenumi{(\alph{enumi})}
\setcounter{enumi}{1}
\item
  Resonant frequency: ω₀ = 1/√(LC) = 1/√(0.1 × 10⁻⁷) = 1/√(10⁻⁸) =
  \textbf{10⁴ rad/s} (f₀ = 1,592 Hz)
\item
  Quality factor: Q = R√(C/L) = 10,000 × √(10⁻⁷/0.1) = 10,000 × √(10⁻⁶)
  = 10,000 × 10⁻³ = \textbf{10}
\item
  Bandwidth: BW = ω₀/Q = 10⁴/10 = \textbf{1,000 rad/s}
  (f\textsubscript{BW} = 159.2 Hz)
\end{enumerate}

Alternatively: BW = 1/(RC) = 1/(10⁴ × 10⁻⁷) = 1000 rad/s. Confirmed.

\begin{center}\rule{0.5\linewidth}{0.5pt}\end{center}

\section{Problem 8.3.7}\label{problem-8.3.7}

\textbf{Given:} A system has transfer function H(s) = (s + 10) / (s³ +
6s² + 11s + 6).

\textbf{Find:} (a) The poles and their locations, (b) whether the system
is stable, (c) the partial fraction expansion, and (d) the impulse
response h(t).

\textbf{Solution:}

\begin{enumerate}
\def\labelenumi{(\alph{enumi})}
\item
  Factor the denominator. Testing s = -1: (-1)³ + 6(1) - 11 + 6 = -1 + 6
  - 11 + 6 = 0. So (s + 1) is a factor. Dividing: s³ + 6s² + 11s + 6 =
  (s + 1)(s² + 5s + 6) = (s + 1)(s + 2)(s + 3) Poles: p₁ = \textbf{-1},
  p₂ = \textbf{-2}, p₃ = \textbf{-3}
\item
  All poles are in the left half-plane (negative real parts), so the
  system is \textbf{stable}.
\item
  Partial fraction expansion: H(s) = A/(s + 1) + B/(s + 2) + C/(s + 3) s
  + 10 = A(s + 2)(s + 3) + B(s + 1)(s + 3) + C(s + 1)(s + 2)
\end{enumerate}

s = -1: -1 + 10 = A(1)(2) → 9 = 2A → A = \textbf{4.5} s = -2: -2 + 10 =
B(-1)(1) → 8 = -B → B = \textbf{-8} s = -3: -3 + 10 = C(-2)(-1) → 7 = 2C
→ C = \textbf{3.5}

H(s) = 4.5/(s + 1) - 8/(s + 2) + 3.5/(s + 3)

\begin{enumerate}
\def\labelenumi{(\alph{enumi})}
\setcounter{enumi}{3}
\tightlist
\item
  Impulse response: h(t) = \textbf{4.5e\textsuperscript{-t} -
  8e\textsuperscript{-2t} + 3.5e\textsuperscript{-3t}} for t ≥ 0
\end{enumerate}

\begin{center}\rule{0.5\linewidth}{0.5pt}\end{center}

\section{Problem 8.3.8}\label{problem-8.3.8}

\textbf{Given:} A series RLC circuit has R = 50 Ω, L = 20 mH, and C = 2
μF. A 10 V step voltage is applied. The output is taken across the
capacitor.

\textbf{Find:} (a) The transfer function H(s) =
V\textsubscript{C}(s)/V\textsubscript{in}(s), (b) whether the response
is underdamped, critically damped, or overdamped, (c) the complete step
response v\textsubscript{C}(t), and (d) the peak capacitor voltage.

\textbf{Solution:}

\begin{enumerate}
\def\labelenumi{(\alph{enumi})}
\tightlist
\item
  H(s) = (1/sC)/(R + sL + 1/sC) = 1/(s²LC + sRC + 1) = (1/LC)/(s² +
  (R/L)s + 1/(LC))
\end{enumerate}

R/L = 50/0.02 = 2500, 1/(LC) = 1/(0.02 × 2 × 10⁻⁶) = 1/(4 × 10⁻⁸) = 2.5
× 10⁷ H(s) = \textbf{2.5 × 10⁷ / (s² + 2500s + 2.5 × 10⁷)}

\begin{enumerate}
\def\labelenumi{(\alph{enumi})}
\setcounter{enumi}{1}
\item
  ω\textsubscript{n} = √(2.5 × 10⁷) = 5000 rad/s 2ζω\textsubscript{n} =
  2500 → ζ = 2500/10,000 = \textbf{0.25} (underdamped, since ζ
  \textless{} 1)
\item
  ω\textsubscript{d} = ω\textsubscript{n}√(1 - ζ²) = 5000√(1 - 0.0625) =
  5000 × 0.9682 = 4841 rad/s σ = ζω\textsubscript{n} = 0.25 × 5000 =
  1250
\end{enumerate}

The step response is: v\textsubscript{C}(t) = 10{[}1 -
(e\textsuperscript{-1250t}/√(1 - ζ²)) × sin(ω\textsubscript{d}t + φ){]}
where φ = arccos(ζ) = arccos(0.25) = 75.5° v\textsubscript{C}(t) =
\textbf{10{[}1 - 1.033e\textsuperscript{-1250t}sin(4841t + 75.5°){]}} V
for t ≥ 0

\begin{enumerate}
\def\labelenumi{(\alph{enumi})}
\setcounter{enumi}{3}
\tightlist
\item
  The percent overshoot is: \%OS = e\textsuperscript{-πζ/√(1-ζ²)} =
  e\textsuperscript{-π(0.25)/0.9682} = e\textsuperscript{-0.8109} =
  0.4443 → 44.4\%
\end{enumerate}

Peak voltage: V\textsubscript{peak} = 10 × (1 + 0.4443) = \textbf{14.44
V} This occurs at t\textsubscript{peak} = π/ω\textsubscript{d} = π/4841
= \textbf{0.649 ms}

\begin{center}\rule{0.5\linewidth}{0.5pt}\end{center}

\section{Problem 8.3.9}\label{problem-8.3.9}

\textbf{Given:} Use the final value theorem to find the steady-state
output of a system with transfer function H(s) = 20/(s + 5) when the
input is (a) a unit step, (b) a unit ramp r(t) = tu(t), and (c) a
sinusoidal input x(t) = sin(3t)u(t).

\textbf{Find:} The steady-state value for each input.

\textbf{Solution:}

\begin{enumerate}
\def\labelenumi{(\alph{enumi})}
\item
  Unit step input: X(s) = 1/s, Y(s) = 20/{[}s(s + 5){]} y(∞) =
  lim\textsubscript{s→0} sY(s) = lim\textsubscript{s→0} 20/(s + 5) =
  20/5 = \textbf{4.0}
\item
  Unit ramp input: X(s) = 1/s², Y(s) = 20/{[}s²(s + 5){]} y(∞) =
  lim\textsubscript{s→0} sY(s) = lim\textsubscript{s→0} 20/{[}s(s +
  5){]} = ∞ The final value theorem indicates the output \textbf{grows
  without bound} (the system has zero steady-state error for a step but
  infinite error for a ramp when there is no integrator in the loop).
\item
  Sinusoidal input: X(s) = 3/(s² + 9) The \textbf{final value theorem
  does not apply} because the output Y(s) has poles on the jω-axis (at s
  = ±j3), violating the requirement that sY(s) have all poles in the
  left half-plane. The steady-state output is found from the frequency
  response: y\textsubscript{ss}(t) = \textbar H(j3)\textbar{} × sin(3t +
  ∠H(j3)) \textbar H(j3)\textbar{} = 20/\textbar j3 + 5\textbar{} =
  20/√(25 + 9) = 20/√34 = \textbf{3.43} ∠H(j3) = -arctan(3/5) =
  \textbf{-30.96°} y\textsubscript{ss}(t) = \textbf{3.43 sin(3t -
  30.96°)}
\end{enumerate}

\begin{center}\rule{0.5\linewidth}{0.5pt}\end{center}

\section{Problem 8.3.10}\label{problem-8.3.10}

\textbf{Given:} An active lowpass filter has transfer function H(s) = K
/ {[}(s + 100)(s + 5000){]}, where K is an adjustable gain constant. The
filter must have a DC gain of exactly 20 dB.

\textbf{Find:} (a) The value of K, (b) the magnitude response at f = 500
Hz, (c) the phase at f = 500 Hz, and (d) the approximate high-frequency
roll-off rate.

\textbf{Solution:}

\begin{enumerate}
\def\labelenumi{(\alph{enumi})}
\item
  DC gain = 20 dB = 10 (linear). H(0) = K/(100 × 5000) = K/500,000 = 10
  K = \textbf{5,000,000} (or 5 × 10⁶)
\item
  At f = 500 Hz, ω = 2π × 500 = 3141.6 rad/s: H(jω) = 5 × 10⁶ / {[}(jω +
  100)(jω + 5000){]} \textbar jω + 100\textbar{} = √(3141.6² + 100²) =
  √(9,869,645 + 10,000) = √9,879,645 = 3143.2 \textbar jω +
  5000\textbar{} = √(3141.6² + 5000²) = √(9,869,645 + 25,000,000) =
  √34,869,645 = 5905.1 \textbar H\textbar{} = 5 × 10⁶ / (3143.2 ×
  5905.1) = 5 × 10⁶ / 18,560,586 = 0.2694 In dB: 20 log₁₀(0.2694) =
  \textbf{-11.39 dB}
\item
  Phase: ∠H = -arctan(3141.6/100) - arctan(3141.6/5000) = -arctan(31.42)
  - arctan(0.6283) = -88.18° - 32.14° = \textbf{-120.3°}
\item
  The system has two real poles, so the high-frequency roll-off is:
  \textbf{-40 dB/decade} (second-order system, -20 dB/decade per pole)
\end{enumerate}

The two corner frequencies are at f₁ = 100/(2π) = 15.9 Hz and f₂ =
5000/(2π) = 795.8 Hz. Between these frequencies, the roll-off is -20
dB/decade; above 795.8 Hz, it steepens to -40 dB/decade.

\chapter{Chapter 8 --- Section 8.4:
Z-Transform}\label{chapter-8-section-8.4-z-transform}

Practice problems covering Z-transform definition and properties,
discrete-time transfer functions, inverse Z-transform, bilinear
transform, and stability analysis in the z-plane.

\begin{center}\rule{0.5\linewidth}{0.5pt}\end{center}

\section{Problem 8.4.1}\label{problem-8.4.1}

\textbf{Given:} A discrete-time sequence x{[}n{]} = 3(0.6)ⁿu{[}n{]} +
2(-0.4)ⁿu{[}n{]}, where u{[}n{]} is the unit step.

\textbf{Find:} (a) The Z-transform X(z), (b) the region of convergence,
and (c) the value of x{[}0{]} and x{[}1{]} (verify using the initial
value theorem).

\textbf{Solution:}

\begin{enumerate}
\def\labelenumi{(\alph{enumi})}
\tightlist
\item
  Using the Z-transform pair aⁿu{[}n{]} ↔ z/(z - a): X(z) = 3z/(z - 0.6)
  + 2z/(z - (-0.4)) = 3z/(z - 0.6) + 2z/(z + 0.4)
\end{enumerate}

Combining over a common denominator: X(z) = {[}3z(z + 0.4) + 2z(z -
0.6){]} / {[}(z - 0.6)(z + 0.4){]} = {[}3z² + 1.2z + 2z² - 1.2z{]} /
{[}(z - 0.6)(z + 0.4){]} = \textbf{5z² / {[}(z - 0.6)(z + 0.4){]}} =
\textbf{5z² / (z² - 0.2z - 0.24)}

\begin{enumerate}
\def\labelenumi{(\alph{enumi})}
\setcounter{enumi}{1}
\item
  The ROC is the intersection of \textbar z\textbar{} \textgreater{} 0.6
  and \textbar z\textbar{} \textgreater{} 0.4: ROC:
  \textbf{\textbar z\textbar{} \textgreater{} 0.6} Since the ROC
  includes the unit circle, the signal has a valid DTFT.
\item
  x{[}0{]} = 3(0.6)⁰ + 2(-0.4)⁰ = 3 + 2 = \textbf{5} x{[}1{]} = 3(0.6)¹
  + 2(-0.4)¹ = 1.8 - 0.8 = \textbf{1.0}
\end{enumerate}

Initial value theorem: x{[}0{]} = lim\textsubscript{z→∞} X(z) =
lim\textsubscript{z→∞} 5z²/(z² - 0.2z - 0.24) = 5. Confirmed.

\begin{center}\rule{0.5\linewidth}{0.5pt}\end{center}

\section{Problem 8.4.2}\label{problem-8.4.2}

\textbf{Given:} A digital filter is described by the difference equation
y{[}n{]} = x{[}n{]} - 0.5x{[}n-1{]} + 0.8y{[}n-1{]} - 0.25y{[}n-2{]}.

\textbf{Find:} (a) The transfer function H(z), (b) the poles and zeros,
(c) whether the system is stable, and (d) the magnitude response at DC
and at the Nyquist frequency.

\textbf{Solution:}

\begin{enumerate}
\def\labelenumi{(\alph{enumi})}
\item
  Taking the Z-transform: Y(z) = X(z) - 0.5z⁻¹X(z) + 0.8z⁻¹Y(z) -
  0.25z⁻²Y(z) Y(z){[}1 - 0.8z⁻¹ + 0.25z⁻²{]} = X(z){[}1 - 0.5z⁻¹{]} H(z)
  = (1 - 0.5z⁻¹)/(1 - 0.8z⁻¹ + 0.25z⁻²) = \textbf{z(z - 0.5) / (z² -
  0.8z + 0.25)}
\item
  Zeros: z = 0 and z = \textbf{0.5} Poles: z = (0.8 ± √(0.64 - 1.0))/2 =
  (0.8 ± √(-0.36))/2 = (0.8 ± j0.6)/2 p₁ = \textbf{0.4 + j0.3}, p₂ =
  \textbf{0.4 - j0.3}
\item
  Pole magnitude: \textbar p\textbar{} = √(0.4² + 0.3²) = √(0.16 + 0.09)
  = √0.25 = \textbf{0.5} Since \textbar p\textbar{} = 0.5 \textless{} 1,
  the system is \textbf{stable}.
\item
  At DC (z = 1): \textbar H(1)\textbar{} = \textbar1 -
  0.5\textbar/\textbar1 - 0.8 + 0.25\textbar{} = 0.5/0.45 =
  \textbf{1.111} (0.92 dB) At Nyquist (z = -1): \textbar H(-1)\textbar{}
  = \textbar-1 - 0.5\textbar/\textbar1 + 0.8 + 0.25\textbar{} = 1.5/2.05
  = \textbf{0.732} (-2.71 dB)
\end{enumerate}

\begin{center}\rule{0.5\linewidth}{0.5pt}\end{center}

\section{Problem 8.4.3}\label{problem-8.4.3}

\textbf{Given:} Find the inverse Z-transform of X(z) = (4z² + 2z) / (z²
- 0.5z - 0.5).

\textbf{Find:} (a) The partial fraction expansion, (b) the time-domain
sequence x{[}n{]}, and (c) the first four samples x{[}0{]} through
x{[}3{]}.

\textbf{Solution:}

\begin{enumerate}
\def\labelenumi{(\alph{enumi})}
\tightlist
\item
  Factor the denominator: z² - 0.5z - 0.5 = 0 z = (0.5 ± √(0.25 + 2))/2
  = (0.5 ± √2.25)/2 = (0.5 ± 1.5)/2 z₁ = 1.0, z₂ = -0.5
\end{enumerate}

So z² - 0.5z - 0.5 = (z - 1)(z + 0.5).

First, perform polynomial long division since the degree of the
numerator equals the denominator: (4z² + 2z) / (z² - 0.5z - 0.5) = 4 +
(4z + 2)/(z² - 0.5z - 0.5)

Partial fraction expansion of (4z + 2)/{[}(z - 1)(z + 0.5){]}: X(z)/z
for the remainder: (4z + 2)/{[}z(z - 1)(z + 0.5){]} = A/z + B/(z - 1) +
C/(z + 0.5)

Actually, let's expand (4z + 2)/{[}(z - 1)(z + 0.5){]} = B/(z - 1) +
C/(z + 0.5): 4z + 2 = B(z + 0.5) + C(z - 1) z = 1: 6 = 1.5B → B = 4 z =
-0.5: 0 = -1.5C → C = 0

So X(z) = 4 + 4z/(z - 1) + 0/(z + 0.5) = 4δ{[}n{]} + 4(1)ⁿu{[}n{]}

Wait, let me redo this more carefully. We need X(z)/z: X(z)/z = (4z +
2)/{[}(z - 1)(z + 0.5){]} = A/(z - 1) + B/(z + 0.5) 4z + 2 = A(z + 0.5)
+ B(z - 1) z = 1: 6 = 1.5A → A = 4 z = -0.5: 0 = -1.5B → B = 0

X(z) = 4z/(z - 1) = 4 × z/(z - 1)

\begin{enumerate}
\def\labelenumi{(\alph{enumi})}
\setcounter{enumi}{1}
\item
  x{[}n{]} = \textbf{4u{[}n{]}} (a constant sequence of value 4 for n ≥
  0)
\item
  Verification by long division of X(z) = (4z² + 2z)/(z² - 0.5z - 0.5)
  in powers of z⁻¹: Rewrite as (4 + 2z⁻¹)/(1 - 0.5z⁻¹ - 0.5z⁻²)
\end{enumerate}

x{[}0{]} = \textbf{4}, x{[}1{]} = 2 + 0.5(4) = 2 + 2 = \textbf{4},
x{[}2{]} = 0.5(4) + 0.5(4) = 2 + 2 = \textbf{4}, x{[}3{]} = 0.5(4) +
0.5(4) = \textbf{4}

The sequence is indeed a constant value of 4 for all n ≥ 0.

\begin{center}\rule{0.5\linewidth}{0.5pt}\end{center}

\section{Problem 8.4.4}\label{problem-8.4.4}

\textbf{Given:} Design a first-order digital highpass filter using the
bilinear transform applied to the analog prototype H\textsubscript{a}(s)
= s/(s + Ω\textsubscript{c}). The desired -3 dB cutoff is 1 kHz and the
sampling rate is f\textsubscript{s} = 8 kHz.

\textbf{Find:} (a) The pre-warped analog cutoff frequency, (b) the
digital transfer function H(z), (c) the difference equation, and (d) the
magnitude response at DC and at 4 kHz (Nyquist).

\textbf{Solution:}

\begin{enumerate}
\def\labelenumi{(\alph{enumi})}
\item
  Digital cutoff frequency: ω\textsubscript{c} = 2π × 1000/8000 = 0.25π
  rad/sample Pre-warped analog frequency: Ω\textsubscript{c} =
  (2/T)tan(ω\textsubscript{c}T/2) = 2 × 8000 × tan(0.25π/2) = 16,000 ×
  tan(0.3927) = 16,000 × 0.4142 = \textbf{6,627 rad/s}
\item
  Analog prototype: H\textsubscript{a}(s) = s/(s + 6627) Bilinear
  substitution s = (2/T)(z - 1)/(z + 1) = 16,000(z - 1)/(z + 1):
\end{enumerate}

H(z) = {[}16,000(z - 1)/(z + 1){]} / {[}16,000(z - 1)/(z + 1) + 6627{]}
= 16,000(z - 1) / {[}16,000(z - 1) + 6627(z + 1){]} = 16,000(z - 1) /
{[}16,000z - 16,000 + 6627z + 6627{]} = 16,000(z - 1) / {[}22,627z -
9,373{]}

Dividing by 22,627: H(z) = 0.7071(z - 1)/(z - 0.4142) = \textbf{0.7071(1
- z⁻¹)/(1 - 0.4142z⁻¹)}

\begin{enumerate}
\def\labelenumi{(\alph{enumi})}
\setcounter{enumi}{2}
\item
  Difference equation: y{[}n{]} = 0.4142y{[}n-1{]} + 0.7071x{[}n{]} -
  0.7071x{[}n-1{]}
\item
  At DC (z = 1): H(1) = 0.7071(1 - 1)/(1 - 0.4142) = \textbf{0} (blocks
  DC, as expected for a highpass filter)
\end{enumerate}

At Nyquist (z = -1): \textbar H(-1)\textbar{} = 0.7071\textbar-1 -
1\textbar/\textbar-1 - 0.4142\textbar{} = 0.7071 × 2/1.4142 =
\textbf{1.0} (unity gain at Nyquist)

\begin{center}\rule{0.5\linewidth}{0.5pt}\end{center}

\section{Problem 8.4.5}\label{problem-8.4.5}

\textbf{Given:} A second-order IIR filter has transfer function H(z) =
1/(1 - 1.2z⁻¹ + 0.8z⁻²). The coefficients are a₁ = -1.2 and a₂ = 0.8.

\textbf{Find:} (a) Whether the filter is stable using the stability
triangle conditions, (b) the pole locations and their magnitudes, (c)
the resonant frequency if the sampling rate is f\textsubscript{s} = 44.1
kHz, and (d) the maximum value of a₂ that would maintain stability.

\textbf{Solution:}

\begin{enumerate}
\def\labelenumi{(\alph{enumi})}
\item
  Stability triangle check (a₁ = -1.2, a₂ = 0.8): Condition 1: 1 + a₁ +
  a₂ = 1 - 1.2 + 0.8 = 0.6 \textgreater{} 0 ✓ Condition 2: 1 - a₁ + a₂ =
  1 + 1.2 + 0.8 = 3.0 \textgreater{} 0 ✓ Condition 3:
  \textbar a₂\textbar{} = 0.8 \textless{} 1 ✓ All conditions met →
  \textbf{Stable}
\item
  Poles from z² - 1.2z + 0.8 = 0: z = (1.2 ± √(1.44 - 3.2))/2 = (1.2 ±
  √(-1.76))/2 = (1.2 ± j1.3266)/2 z = 0.6 ± j0.6633
\end{enumerate}

\textbar z\textbar{} = √(0.36 + 0.44) = √0.80 = \textbf{0.8944}

Both poles have magnitude 0.8944, confirming they are inside the unit
circle.

\begin{enumerate}
\def\labelenumi{(\alph{enumi})}
\setcounter{enumi}{2}
\item
  Pole angle: θ = arctan(0.6633/0.6) = arctan(1.1055) = 47.87° Resonant
  frequency: f\textsubscript{r} = θ/(360°) × f\textsubscript{s} =
  (47.87/360) × 44,100 = \textbf{5,865 Hz}
\item
  From the stability triangle: Condition 1: a₂ \textgreater{} -1 - a₁ =
  -1 + 1.2 = 0.2 Condition 3: a₂ \textless{} 1 Therefore the maximum a₂
  for stability (with a₁ = -1.2) is a₂ \textless{} \textbf{1.0}. At a₂ =
  1.0, the poles would be exactly on the unit circle
  (\textbar z\textbar{} = √a₂ = 1), making the system marginally stable.
\end{enumerate}

\begin{center}\rule{0.5\linewidth}{0.5pt}\end{center}

\section{Problem 8.4.6}\label{problem-8.4.6}

\textbf{Given:} A causal system has the Z-transform H(z) = (z² - 1) /
(z² - 0.5z + 0.06).

\textbf{Find:} (a) The poles and zeros, (b) the partial fraction
expansion of H(z), (c) the impulse response h{[}n{]}, and (d) the
steady-state response to a unit step.

\textbf{Solution:}

\begin{enumerate}
\def\labelenumi{(\alph{enumi})}
\item
  Zeros: z² - 1 = 0 → z = \textbf{+1} and z = \textbf{-1} Poles: z² -
  0.5z + 0.06 = 0 → z = (0.5 ± √(0.25 - 0.24))/2 = (0.5 ± 0.1)/2 p₁ =
  \textbf{0.3}, p₂ = \textbf{0.2}
\item
  Since numerator and denominator have the same degree, perform long
  division first: (z² - 1)/(z² - 0.5z + 0.06) = 1 + (0.5z - 1.06)/(z² -
  0.5z + 0.06)
\end{enumerate}

Partial fraction of remainder R(z)/z = (0.5z - 1.06)/{[}z(z - 0.3)(z -
0.2){]}: Wait, let's use H(z)/z = (z² - 1)/{[}z(z - 0.3)(z - 0.2){]} for
the full expression.

Better approach --- expand the remainder term: (0.5z - 1.06)/{[}(z -
0.3)(z - 0.2){]} = A/(z - 0.3) + B/(z - 0.2) 0.5z - 1.06 = A(z - 0.2) +
B(z - 0.3) z = 0.3: 0.15 - 1.06 = 0.1A → A = -9.1 z = 0.2: 0.10 - 1.06 =
-0.1B → B = 9.6

H(z) = 1 + (-9.1z)/(z - 0.3) + (9.6z)/(z - 0.2) \ldots{} actually this
doesn't look right. Let me redo:

(0.5z - 1.06)/{[}(z - 0.3)(z - 0.2){]}: multiply by z/z to get partial
fractions involving z/(z - a): = {[}-9.1(z - 0.3) + \ldots{} {]} ---
let's just do it directly for the time-domain sequence.

H(z)/z = (z² - 1)/{[}z(z - 0.3)(z - 0.2){]} = A/z + B/(z - 0.3) + C/(z -
0.2) z² - 1 = A(z - 0.3)(z - 0.2) + Bz(z - 0.2) + Cz(z - 0.3) z = 0: -1
= A(0.06) → A = -16.667 z = 0.3: 0.09 - 1 = B(0.3)(0.1) → -0.91 = 0.03B
→ B = -30.333 z = 0.2: 0.04 - 1 = C(0.2)(-0.1) → -0.96 = -0.02C → C =
48.0

H(z) = -16.667 + (-30.333)z/(z - 0.3) + 48.0z/(z - 0.2)

\begin{enumerate}
\def\labelenumi{(\alph{enumi})}
\setcounter{enumi}{2}
\tightlist
\item
  h{[}n{]} = -16.667δ{[}n{]} - 30.333(0.3)ⁿu{[}n{]} + 48.0(0.2)ⁿu{[}n{]}
  h{[}0{]} = -16.667 - 30.333 + 48.0 = \textbf{1.0} h{[}1{]} =
  -30.333(0.3) + 48.0(0.2) = -9.1 + 9.6 = \textbf{0.5}
\end{enumerate}

Verify: h{[}0{]} should equal the leading coefficient, which from long
division is 1. Confirmed.

\begin{enumerate}
\def\labelenumi{(\alph{enumi})}
\setcounter{enumi}{3}
\tightlist
\item
  For a unit step (z/(z-1)), using the final value theorem: y(∞) =
  lim\textsubscript{z→1} (z-1) × H(z) × z/(z-1) = H(1) = (1 - 1)/(1 -
  0.5 + 0.06) = 0/0.56 = \textbf{0}
\end{enumerate}

The zero at z = 1 blocks the DC component of the step, so the
steady-state output is zero.

\begin{center}\rule{0.5\linewidth}{0.5pt}\end{center}

\section{Problem 8.4.7}\label{problem-8.4.7}

\textbf{Given:} Design a second-order digital bandpass filter using the
bilinear transform. The center frequency is 2 kHz, the 3 dB bandwidth is
400 Hz, and the sampling rate is f\textsubscript{s} = 16 kHz. The analog
prototype is H\textsubscript{a}(s) = (BW × s)/(s² + BW × s + ω₀²).

\textbf{Find:} (a) The pre-warped center frequency and bandwidth, (b)
the analog prototype transfer function, and (c) the digital transfer
function coefficients.

\textbf{Solution:}

\begin{enumerate}
\def\labelenumi{(\alph{enumi})}
\tightlist
\item
  Digital frequencies: ω₀ = 2π × 2000/16,000 = 0.25π rad/sample
  ω\textsubscript{L} = 2π × 1800/16,000 = 0.225π (lower edge)
  ω\textsubscript{H} = 2π × 2200/16,000 = 0.275π (upper edge)
\end{enumerate}

Pre-warped frequencies: Ω₀ = (2/T)tan(ω₀/2) = 2 × 16,000 × tan(0.125π) =
32,000 × 0.4142 = \textbf{13,255 rad/s} Ω\textsubscript{L} = 32,000 ×
tan(0.1125π) = 32,000 × 0.3640 = \textbf{11,647 rad/s}
Ω\textsubscript{H} = 32,000 × tan(0.1375π) = 32,000 × 0.4680 =
\textbf{14,977 rad/s} Pre-warped BW = Ω\textsubscript{H} -
Ω\textsubscript{L} = 14,977 - 11,647 = \textbf{3,330 rad/s}

\begin{enumerate}
\def\labelenumi{(\alph{enumi})}
\setcounter{enumi}{1}
\item
  Analog prototype: H\textsubscript{a}(s) = 3,330s / (s² + 3,330s +
  13,255²) = \textbf{3,330s / (s² + 3,330s + 1.757 × 10⁸)}
\item
  Applying the bilinear transform s = 32,000(z - 1)/(z + 1) and
  simplifying (normalizing by the coefficient of z² in the denominator):
\end{enumerate}

The denominator becomes: {[}32,000(z-1)/(z+1){]}² +
3,330{[}32,000(z-1)/(z+1){]} + 1.757 × 10⁸

After algebraic simplification (expanding and collecting terms in z², z,
and constant): Numerator ∝ (z² - 1) (bandpass: zero at DC and Nyquist)
Denominator coefficients yield:

H(z) ≈ \textbf{0.0940(1 - z⁻²) / (1 - 1.6629z⁻¹ + 0.8120z⁻²)}

The pole magnitude is √0.8120 = 0.9011 (stable, inside unit circle). The
Q factor is f₀/BW = 2000/400 = 5, producing sharp resonance at 2 kHz.

\begin{center}\rule{0.5\linewidth}{0.5pt}\end{center}

\section{Problem 8.4.8}\label{problem-8.4.8}

\textbf{Given:} A third-order IIR filter has denominator polynomial D(z)
= z³ - 1.8z² + 1.2z - 0.3.

\textbf{Find:} (a) Apply the Jury stability test necessary conditions,
(b) determine if the filter is stable, and (c) find the approximate pole
locations.

\textbf{Solution:}

\begin{enumerate}
\def\labelenumi{(\alph{enumi})}
\tightlist
\item
  The polynomial is D(z) = z³ + a₁z² + a₂z + a₃ where a₁ = -1.8, a₂ =
  1.2, a₃ = -0.3, and the leading coefficient a₀ = 1.
\end{enumerate}

Necessary conditions for the Jury test (n = 3): Condition 1: D(1)
\textgreater{} 0 → 1 - 1.8 + 1.2 - 0.3 = \textbf{0.1 \textgreater{} 0} ✓
Condition 2: (-1)ⁿD(-1) \textgreater{} 0 → (-1)³(-1 - 1.8 - 1.2 + 0.3) =
(-1)(-2.7) = \textbf{2.7 \textgreater{} 0} ✓ Condition 3:
\textbar a₃\textbar{} \textless{} \textbar a₀\textbar{} →
\textbar-0.3\textbar{} \textless{} \textbar1\textbar{} → \textbf{0.3
\textless{} 1} ✓

\begin{enumerate}
\def\labelenumi{(\alph{enumi})}
\setcounter{enumi}{1}
\tightlist
\item
  Construct the Jury array: Row 1: 1, -1.8, 1.2, -0.3 Row 2 (reversed):
  -0.3, 1.2, -1.8, 1
\end{enumerate}

Compute b\textsubscript{k} = \textbar a₀, a₃₋ₖ; a₃,
a\textsubscript{k}\textbar{} (determinant): b₀ = 1(1) - (-0.3)(-0.3) = 1
- 0.09 = 0.91 b₁ = 1(-1.8) - (-0.3)(1.2) = -1.8 + 0.36 = -1.44 b₂ =
1(1.2) - (-0.3)(-1.8) = 1.2 - 0.54 = 0.66

Check \textbar b₀\textbar{} \textgreater{} \textbar b₂\textbar:
\textbar0.91\textbar{} \textgreater{} \textbar0.66\textbar{} ✓

All conditions satisfied → \textbf{Stable}

\begin{enumerate}
\def\labelenumi{(\alph{enumi})}
\setcounter{enumi}{2}
\tightlist
\item
  Trying z = 0.5: D(0.5) = 0.125 - 0.45 + 0.6 - 0.3 = -0.025 ≈ 0
  Refining, one real pole near z ≈ \textbf{0.48}. The remaining
  quadratic: z² - 1.32z + 0.625 = 0 → z = (1.32 ± √(1.7424 - 2.5))/2 =
  (1.32 ± j0.870)/2 z = \textbf{0.66 ± j0.435}, \textbar z\textbar{} =
  √(0.4356 + 0.1892) = √0.6248 = \textbf{0.791} All poles inside the
  unit circle --- confirmed stable.
\end{enumerate}

\begin{center}\rule{0.5\linewidth}{0.5pt}\end{center}

\section{Problem 8.4.9}\label{problem-8.4.9}

\textbf{Given:} A digital system has transfer function H(z) = (1 + 2z⁻¹
+ z⁻²) / (1 - 0.5z⁻¹). The sampling rate is f\textsubscript{s} = 10 kHz.

\textbf{Find:} (a) The poles and zeros, (b) whether this is FIR or IIR,
(c) the impulse response h{[}n{]} for n = 0 to 4, and (d) the frequency
response magnitude at 1 kHz and 5 kHz.

\textbf{Solution:}

\begin{enumerate}
\def\labelenumi{(\alph{enumi})}
\item
  Numerator: 1 + 2z⁻¹ + z⁻² = (1 + z⁻¹)² → zeros at z = \textbf{-1}
  (double zero) Denominator: 1 - 0.5z⁻¹ → pole at z = \textbf{0.5} This
  is an \textbf{IIR} filter (has a non-trivial pole at z = 0.5).
\item
  This is \textbf{IIR} --- the feedback pole at z = 0.5 creates an
  infinite impulse response.
\item
  Difference equation: y{[}n{]} = 0.5y{[}n-1{]} + x{[}n{]} + 2x{[}n-1{]}
  + x{[}n-2{]} With impulse input (x{[}0{]} = 1, x{[}n{]} = 0 for n ≠
  0): h{[}0{]} = 0 + 1 + 0 + 0 = \textbf{1} h{[}1{]} = 0.5(1) + 0 + 2(1)
  + 0 = \textbf{2.5} h{[}2{]} = 0.5(2.5) + 0 + 0 + 1 = \textbf{2.25}
  h{[}3{]} = 0.5(2.25) + 0 + 0 + 0 = \textbf{1.125} h{[}4{]} =
  0.5(1.125) = \textbf{0.5625}
\item
  At f = 1 kHz: ω = 2π × 1000/10,000 = 0.2π rad/sample Numerator:
  \textbar1 + 2e\textsuperscript{-j0.2π} +
  e\textsuperscript{-j0.4π}\textbar{} = \textbar(1 +
  e\textsuperscript{-j0.2π})²\textbar{} = \textbar1 + cos(0.2π) -
  jsin(0.2π)\textbar² \ldots{} let me compute directly: \textbar1 +
  e\textsuperscript{-jω}\textbar² = (1 + cos ω)² + sin²ω = 2 + 2cos ω =
  2(1 + cos(0.2π)) = 2(1 + 0.809) = 3.618 So
  \textbar numerator\textbar{} = 3.618
\end{enumerate}

Denominator: \textbar1 - 0.5e\textsuperscript{-j0.2π}\textbar{} = √((1 -
0.5cos(0.2π))² + (0.5sin(0.2π))²) = √((1 - 0.4045)² + (0.2939)²) =
√(0.3544 + 0.0864) = √0.4408 = 0.6639

\textbar H\textbar{} = 3.618/0.6639 = \textbf{5.45}

At f = 5 kHz (Nyquist): ω = π, z = -1: \textbar Numerator\textbar{} =
\textbar1 - 2 + 1\textbar{} = 0 (double zero at z = -1)
\textbar H\textbar{} = \textbf{0} (complete null at Nyquist, as expected
from the double zero at z = -1)

\begin{center}\rule{0.5\linewidth}{0.5pt}\end{center}

\section{Problem 8.4.10}\label{problem-8.4.10}

\textbf{Given:} A digital control system has open-loop transfer function
G(z) = K/(z - 0.7) in a unity feedback configuration. The closed-loop
transfer function is T(z) = G(z)/(1 + G(z)).

\textbf{Find:} (a) The closed-loop transfer function in terms of K, (b)
the closed-loop pole location, (c) the maximum value of K for stability,
and (d) the value of K that places the closed-loop pole at z = 0
(deadbeat response).

\textbf{Solution:}

\begin{enumerate}
\def\labelenumi{(\alph{enumi})}
\item
  T(z) = {[}K/(z - 0.7){]} / {[}1 + K/(z - 0.7){]} = K / (z - 0.7 + K) =
  \textbf{K / (z - 0.7 + K)}
\item
  The closed-loop pole is at: z = 0.7 - K → z\textsubscript{pole} =
  \textbf{0.7 - K}
\item
  For stability, the pole must be inside the unit circle:
  \textbar z\textsubscript{pole}\textbar{} \textless{} 1 -1 \textless{}
  0.7 - K \textless{} 1 From the left inequality: K \textless{} 1.7 From
  the right inequality: K \textgreater{} -0.3 (always true for positive
  K) Maximum K for stability: \textbf{K \textless{} 1.7}
\item
  For deadbeat response (pole at z = 0): 0.7 - K = 0 → \textbf{K = 0.7}
\end{enumerate}

With K = 0.7, T(z) = 0.7/z = 0.7z⁻¹. The impulse response is h{[}0{]} =
0, h{[}1{]} = 0.7, h{[}n{]} = 0 for n ≥ 2 --- the system settles in
exactly one sample period (deadbeat response).

\chapter{Chapter 8 --- Section 8.5: Digital
Filters}\label{chapter-8-section-8.5-digital-filters}

Practice problems covering FIR filters, IIR filters, filter design
specifications, multirate signal processing, fixed-point implementation,
polyphase filters, and allpass group delay equalization.

\begin{center}\rule{0.5\linewidth}{0.5pt}\end{center}

\section{Problem 8.5.1}\label{problem-8.5.1}

\textbf{Given:} A 5-tap FIR filter has symmetric coefficients b = \{0.1,
0.25, 0.3, 0.25, 0.1\}. The filter is sampled at f\textsubscript{s} = 20
kHz.

\textbf{Find:} (a) Whether the filter has linear phase, (b) the group
delay in samples and milliseconds, (c) the DC gain, (d) the gain at the
Nyquist frequency, and (e) the type of frequency response (lowpass,
highpass, bandpass).

\textbf{Solution:}

\begin{enumerate}
\def\labelenumi{(\alph{enumi})}
\item
  The coefficients are symmetric: b{[}0{]} = b{[}4{]} = 0.1 and b{[}1{]}
  = b{[}3{]} = 0.25. A symmetric FIR filter with odd length has
  \textbf{Type I linear phase} --- yes, the filter has \textbf{exactly
  linear phase}.
\item
  For a Type I FIR filter with N = 5 taps: Group delay = (N - 1)/2 = (5
  - 1)/2 = \textbf{2 samples} = 2/20,000 = \textbf{0.1 ms}
\item
  DC gain (z = 1): H(1) = 0.1 + 0.25 + 0.3 + 0.25 + 0.1 = \textbf{1.0}
  (0 dB)
\item
  Gain at Nyquist (z = -1): H(-1) = 0.1 - 0.25 + 0.3 - 0.25 + 0.1 =
  \textbf{0.0} (-∞ dB)
\item
  The filter passes DC with unity gain and completely rejects the
  Nyquist frequency. This is a \textbf{lowpass} filter. The null at
  Nyquist confirms the lowpass characteristic.
\end{enumerate}

\begin{center}\rule{0.5\linewidth}{0.5pt}\end{center}

\section{Problem 8.5.2}\label{problem-8.5.2}

\textbf{Given:} A second-order IIR bandpass filter is defined by:
y{[}n{]} = 1.2728y{[}n-1{]} - 0.81y{[}n-2{]} + 0.095x{[}n{]} -
0.095x{[}n-2{]}

The sampling rate is f\textsubscript{s} = 8000 Hz.

\textbf{Find:} (a) The transfer function H(z), (b) the pole and zero
locations, (c) the center frequency, and (d) the 3 dB bandwidth.

\textbf{Solution:}

\begin{enumerate}
\def\labelenumi{(\alph{enumi})}
\item
  Taking the Z-transform: H(z) = (0.095 - 0.095z⁻²)/(1 - 1.2728z⁻¹ +
  0.81z⁻²) = \textbf{0.095(z² - 1)/(z² - 1.2728z + 0.81)}
\item
  Zeros: z² - 1 = 0 → z = \textbf{+1} and z = \textbf{-1} (zeros at DC
  and Nyquist --- bandpass characteristic) Poles: z² - 1.2728z + 0.81 =
  0 z = (1.2728 ± √(1.6200 - 3.24))/2 = (1.2728 ± √(-1.62))/2 Wait:
  1.2728² = 1.6200, 4 × 0.81 = 3.24 z = (1.2728 ± j√(3.24 - 1.62))/2 =
  (1.2728 ± j√1.62)/2 = (1.2728 ± j1.2728)/2
\end{enumerate}

Hmm, let me recalculate: discriminant = 1.2728² - 4(0.81) = 1.6200 -
3.24 = -1.62 z = (1.2728 ± j1.2728)/2 = 0.6364 ± j0.6364

Pole magnitude: \textbar z\textbar{} = √(0.6364² + 0.6364²) = 0.6364√2 =
\textbf{0.9} (= √0.81)

\begin{enumerate}
\def\labelenumi{(\alph{enumi})}
\setcounter{enumi}{2}
\item
  Pole angle: θ = arctan(0.6364/0.6364) = arctan(1) = 45° = π/4 rad
  Center frequency: f₀ = (θ/2π) × f\textsubscript{s} = (45/360) × 8000 =
  \textbf{1000 Hz}
\item
  The 3 dB bandwidth for a second-order bandpass filter: BW ≈ (1 - r²) ×
  f\textsubscript{s} / (2π) where r = pole radius = 0.9 BW ≈ (1 - 0.81)
  × 8000 / (2π) = 0.19 × 8000 / 6.283 = \textbf{241.9 Hz}
\end{enumerate}

The Q factor is Q = f₀/BW = 1000/241.9 ≈ 4.1.

\begin{center}\rule{0.5\linewidth}{0.5pt}\end{center}

\section{Problem 8.5.3}\label{problem-8.5.3}

\textbf{Given:} A digital lowpass filter is required with passband edge
f\textsubscript{p} = 3 kHz (1 dB ripple), stopband edge
f\textsubscript{s} = 4.5 kHz (50 dB attenuation), and sampling rate
f\textsubscript{s} = 20 kHz. Compare the required orders for Butterworth
and Chebyshev Type I IIR designs.

\textbf{Find:} (a) The normalized frequencies, (b) the pre-warped analog
frequencies, (c) the Butterworth order, and (d) the Chebyshev Type I
order.

\textbf{Solution:}

\begin{enumerate}
\def\labelenumi{(\alph{enumi})}
\item
  Normalized digital frequencies: Ω\textsubscript{p} = 2π × 3000/20,000
  = 0.3π rad/sample Ω\textsubscript{s} = 2π × 4500/20,000 = 0.45π
  rad/sample
\item
  Pre-warped analog frequencies: ω\textsubscript{p} =
  tan(Ω\textsubscript{p}/2) = tan(0.15π) = tan(27°) = \textbf{0.5095}
  ω\textsubscript{s} = tan(Ω\textsubscript{s}/2) = tan(0.225π) =
  tan(40.5°) = \textbf{0.8541} Selectivity ratio: k =
  ω\textsubscript{p}/ω\textsubscript{s} = 0.5095/0.8541 = 0.5965
\item
  Butterworth order: n ≥ log{[}(10\textsuperscript{As/10} -
  1)/(10\textsuperscript{Ap/10} - 1){]} / (2 × log(1/k)) = log{[}(10⁵ -
  1)/(10⁰·¹ - 1){]} / (2 × log(1.676)) = log{[}99,999/0.2589{]} / (2 ×
  0.2245) = log(386,247) / 0.4490 = 5.587 / 0.4490 = \textbf{12.4 → n =
  13}
\item
  Chebyshev Type I order: n ≥ cosh⁻¹(√{[}(10\textsuperscript{As/10} -
  1)/(10\textsuperscript{Ap/10} - 1){]}) / cosh⁻¹(1/k) =
  cosh⁻¹(√(386,247)) / cosh⁻¹(1.676) = cosh⁻¹(621.5) / cosh⁻¹(1.676) =
  7.125 / 1.114 = \textbf{6.4 → n = 7}
\end{enumerate}

The Chebyshev filter requires roughly half the order of the Butterworth
(7 vs 13) at the cost of passband ripple. This translates to
approximately half the computational cost for the IIR implementation.

\begin{center}\rule{0.5\linewidth}{0.5pt}\end{center}

\section{Problem 8.5.4}\label{problem-8.5.4}

\textbf{Given:} A digital audio signal at f\textsubscript{s} = 44.1 kHz
must be converted to 48 kHz for a professional audio interface.

\textbf{Find:} (a) The rational sample rate conversion factor L/M, (b)
the required anti-aliasing/anti-imaging filter cutoff, (c) the number of
output samples for 1 second of input, and (d) the computational
implications.

\textbf{Solution:}

\begin{enumerate}
\def\labelenumi{(\alph{enumi})}
\item
  Sample rate ratio: 48,000/44,100 = 480/441 = \textbf{160/147} (after
  reducing by GCD = 3). L = 160 (interpolation factor), M = 147
  (decimation factor).
\item
  The anti-aliasing filter must prevent aliasing from the lower sampling
  rate: f\textsubscript{cutoff} = min(f\textsubscript{s,in},
  f\textsubscript{s,out})/2 = min(44,100, 48,000)/2 = 44,100/2 =
  \textbf{22,050 Hz} In the intermediate (upsampled) rate of 44,100 ×
  160 = 7,056,000 Hz, the cutoff is 22,050 Hz.
\item
  Input samples for 1 second: 44,100 Output samples: 44,100 × 160/147 =
  7,056,000/147 = \textbf{48,000 samples} ✓
\item
  Direct implementation at the intermediate rate of 7.056 MHz is
  impractical. A polyphase implementation operates the filter at the
  output rate of 48 kHz. If the anti-aliasing filter has N = 1440 taps,
  the polyphase structure uses 160 phases of 9 taps each, requiring only
  \textbf{9 multiplications per output sample} --- a dramatic reduction
  from 1440 at the intermediate rate.
\end{enumerate}

\begin{center}\rule{0.5\linewidth}{0.5pt}\end{center}

\section{Problem 8.5.5}\label{problem-8.5.5}

\textbf{Given:} A second-order IIR notch filter centered at 60 Hz (for
power line rejection) is designed for f\textsubscript{s} = 1000 Hz. The
ideal coefficients are a₁ = -1.61803 and a₂ = 0.95. The filter is
implemented in Q15 fixed-point (16-bit signed).

\textbf{Find:} (a) The Q15 representations of a₁ and a₂, (b) the
quantized coefficient values, (c) the quantization error for each
coefficient, and (d) the shift in notch frequency due to quantization.

\textbf{Solution:}

\begin{enumerate}
\def\labelenumi{(\alph{enumi})}
\tightlist
\item
  Q15 represents values from -1.0 to +0.99997 with resolution 2⁻¹⁵ =
  3.052 × 10⁻⁵. However, \textbar a₁\textbar{} = 1.61803 \textgreater{}
  1.0, which \textbf{exceeds Q15 range}.
\end{enumerate}

Use Q14 scaling (divide by 2): a₁\textsubscript{Q14} = round(-1.61803 ×
2¹⁴) = round(-26,510.4) = -26,510 a₂\textsubscript{Q15} = round(0.95 ×
2¹⁵) = round(31,129.6) = 31,130

\begin{enumerate}
\def\labelenumi{(\alph{enumi})}
\setcounter{enumi}{1}
\item
  Quantized values: a₁\textsubscript{quant} = -26,510 / 2¹⁴ = -26,510 /
  16,384 = \textbf{-1.61798} a₂\textsubscript{quant} = 31,130 / 2¹⁵ =
  31,130 / 32,768 = \textbf{0.94998}
\item
  Quantization errors: Δa₁ = -1.61798 - (-1.61803) = \textbf{+5 × 10⁻⁵}
  (0.003\%) Δa₂ = 0.94998 - 0.95 = \textbf{-2 × 10⁻⁵} (0.002\%)
\item
  The notch frequency depends on a₁: for a second-order notch, ω₀ =
  arccos(-a₁/2). Ideal: ω₀ = arccos(1.61803/2) = arccos(0.80902) =
  0.37699 rad/sample → f₀ = 60.000 Hz Quantized: ω₀ = arccos(1.61798/2)
  = arccos(0.80899) = 0.37703 rad/sample → f₀ = \textbf{60.005 Hz} Shift
  = \textbf{0.005 Hz}, which is negligible for a 60 Hz notch
  application.
\end{enumerate}

\begin{center}\rule{0.5\linewidth}{0.5pt}\end{center}

\section{Problem 8.5.6}\label{problem-8.5.6}

\textbf{Given:} A 120-tap FIR lowpass filter is used for decimation by M
= 4 from 40 kS/s to 10 kS/s. Compare direct and polyphase
implementations.

\textbf{Find:} (a) The polyphase subfilter length, (b) multiplications
per output sample for each approach, (c) total multiplications per
second, and (d) the memory requirement for the polyphase structure.

\textbf{Solution:}

\begin{enumerate}
\def\labelenumi{(\alph{enumi})}
\item
  Polyphase subfilter length: Each phase has N/M = 120/4 = \textbf{30
  taps}
\item
  Multiplications per output sample: Direct: the filter runs at the
  input rate. For each output sample, M = 4 input samples are processed,
  each requiring 120 multiplications. But only 1 output per 4 inputs is
  kept: Direct = 120 × 4 = \textbf{480 multiplications per output
  sample}
\end{enumerate}

Polyphase: M subfilters of 30 taps each run at the output rate:
Polyphase = 4 × 30 = \textbf{120 multiplications per output sample}

\begin{enumerate}
\def\labelenumi{(\alph{enumi})}
\setcounter{enumi}{2}
\item
  Multiplications per second: Direct: 120 × 40,000 =
  \textbf{4,800,000/s} (filter at input rate) Polyphase: 120 × 10,000 =
  \textbf{1,200,000/s} (filter at output rate) Savings: 4× reduction =
  factor of M, as expected.
\item
  Memory for polyphase: Coefficients: 120 values (same as direct ---
  just rearranged into 4 phases) Delay line: 4 delay lines of 30 samples
  each = 120 samples State variable for commutator: 1 variable Total:
  \textbf{120 coefficient + 120 delay line = 240 values}
\end{enumerate}

\begin{center}\rule{0.5\linewidth}{0.5pt}\end{center}

\section{Problem 8.5.7}\label{problem-8.5.7}

\textbf{Given:} A 6th-order Elliptic lowpass IIR filter (3 biquad
sections) has group delay that varies from 5.8 samples at DC to 31.2
samples at the passband edge. A cascade of two second-order allpass
sections is designed to equalize the group delay to approximately 32
samples.

\textbf{Find:} (a) The group delay variation before equalization, (b)
the total system order after adding the equalizer, (c) the added latency
in milliseconds at f\textsubscript{s} = 48 kHz, and (d) the total number
of multiply-accumulate operations per sample.

\textbf{Solution:}

\begin{enumerate}
\def\labelenumi{(\alph{enumi})}
\item
  Group delay variation: Δτ = τ\textsubscript{max} -
  τ\textsubscript{min} = 31.2 - 5.8 = \textbf{25.4 samples} At
  f\textsubscript{s} = 48 kHz: Δτ = 25.4/48,000 = \textbf{0.529 ms}
\item
  Total system order: Original Elliptic filter: 6th order (3 biquad
  sections) Allpass equalizer: 2 × 2nd order = 4th order (2 biquad
  sections) Total: \textbf{10th order} (5 biquad sections)
\item
  Added latency: Before: minimum group delay = 5.8 samples After:
  approximately flat at 32 samples Added latency = 32 - 5.8 = 26.2
  samples = 26.2/48,000 = \textbf{0.546 ms}
\item
  Each biquad section requires 5 multiply-accumulate operations (2
  feedforward + 2 feedback + 1 input scaling, using Direct Form II
  Transposed): Total MACs = 5 × 5 sections = \textbf{25
  multiply-accumulate operations per sample}
\end{enumerate}

At f\textsubscript{s} = 48 kHz: 25 × 48,000 = 1,200,000 MACs/s =
\textbf{1.2 MMAC/s} --- easily handled by any modern DSP.

\begin{center}\rule{0.5\linewidth}{0.5pt}\end{center}

\section{Problem 8.5.8}\label{problem-8.5.8}

\textbf{Given:} A Parks-McClellan (equiripple) FIR lowpass filter is
designed with passband edge 4 kHz, stopband edge 5 kHz, passband ripple
δ\textsubscript{p} = 0.01 (0.087 dB), and stopband attenuation
δ\textsubscript{s} = 0.001 (-60 dB). The sampling rate is
f\textsubscript{s} = 44.1 kHz.

\textbf{Find:} (a) The transition bandwidth, (b) an estimate of the
required filter order using the Bellanger formula, (c) the group delay,
and (d) the computational cost in MMAC/s.

\textbf{Solution:}

\begin{enumerate}
\def\labelenumi{(\alph{enumi})}
\item
  Transition bandwidth: Δf = f\textsubscript{stop} -
  f\textsubscript{pass} = 5000 - 4000 = \textbf{1000 Hz}
\item
  Bellanger formula: N ≈
  (-2/3)log₁₀(10δ\textsubscript{p}δ\textsubscript{s}) /
  (Δf/f\textsubscript{s}) = (-2/3)log₁₀(10 × 0.01 × 0.001) /
  (1000/44,100) = (-2/3)log₁₀(10⁻⁴) / 0.02268 = (-2/3)(-4) / 0.02268 =
  2.667 / 0.02268 = \textbf{117.6 → N ≈ 118 taps}
\item
  Group delay for linear-phase FIR: τ = (N - 1)/2 = 117/2 = \textbf{58.5
  samples} = 58.5/44,100 = \textbf{1.33 ms}
\item
  Computational cost: MACs per sample = N = 118 (or 59 using symmetry)
  At f\textsubscript{s} = 44.1 kHz: 59 × 44,100 = 2,601,900 =
  \textbf{2.60 MMAC/s}
\end{enumerate}

\begin{center}\rule{0.5\linewidth}{0.5pt}\end{center}

\section{Problem 8.5.9}\label{problem-8.5.9}

\textbf{Given:} A half-band FIR filter has the property that every other
coefficient (except the center tap) is exactly zero. A 15-tap half-band
filter has nonzero coefficients at positions 0, 2, 4, 6, 7, 8, 10, 12,
14 with values b = \{-0.025, 0, 0.075, 0, -0.30, 0.5, -0.30, 0, 0.075,
0, -0.025, 0, 0, 0, 0\}.

Wait --- let me reconsider. A proper 15-tap half-band has nonzero taps
at indices 0, 2, 4, 6, 7, 8, 10, 12, 14. Actually, a half-band filter
with N = 15 has (N+1)/2 = 8 nonzero coefficients.

\textbf{Find:} (a) How many multiplications per output sample are needed
(exploiting zero coefficients and symmetry), (b) the cutoff frequency
relative to f\textsubscript{s}, (c) why half-band filters are ideal for
decimation by 2, and (d) the magnitude response at f\textsubscript{s}/4.

\textbf{Solution:}

Let the actual half-band coefficients be: h = \{-0.0215, 0, 0.0718, 0,
-0.3066, 0.5, -0.3066, 0, 0.0718, 0, -0.0215, 0, 0, 0, 0\} --- but this
has 15 entries. A standard 11-tap half-band: h = \{c₀, 0, c₂, 0, c₄,
0.5, c₄, 0, c₂, 0, c₀\}.

Using an 11-tap half-band with coefficients h = \{-0.025, 0, 0.15, 0,
-0.375, 0.5, -0.375, 0, 0.15, 0, -0.025\}:

\begin{enumerate}
\def\labelenumi{(\alph{enumi})}
\item
  Total taps = 11. Zero coefficients: 4 (at odd positions except
  center). Nonzero: 7. Using symmetry (h{[}0{]}=h{[}10{]},
  h{[}2{]}=h{[}8{]}, h{[}4{]}=h{[}6{]}): 3 symmetric pairs + 1 center
  tap = \textbf{4 multiplications per output sample} (instead of 11).
\item
  A half-band filter has its -6 dB point at exactly:
  f\textsubscript{-6dB} = \textbf{f\textsubscript{s}/4}
\end{enumerate}

This is the defining property --- the transition band is centered at
f\textsubscript{s}/4, and the passband and stopband ripples are equal
(δ\textsubscript{p} = δ\textsubscript{s}).

\begin{enumerate}
\def\labelenumi{(\alph{enumi})}
\setcounter{enumi}{2}
\tightlist
\item
  Half-band filters are ideal for decimation/interpolation by 2 because:
\end{enumerate}

\begin{itemize}
\tightlist
\item
  The cutoff is naturally at f\textsubscript{s}/4 =
  f\textsubscript{new}/2 (the new Nyquist frequency after decimation)
\item
  The zero coefficients reduce computation by nearly 50\%
\item
  Combined with polyphase decomposition, each output sample requires
  only (N+1)/4 multiplications
\end{itemize}

\begin{enumerate}
\def\labelenumi{(\alph{enumi})}
\setcounter{enumi}{3}
\tightlist
\item
  At f = f\textsubscript{s}/4, by the half-band property:
  \textbar H(e\textsuperscript{jπ/2})\textbar{} = \textbf{0.5} (-6.02
  dB)
\end{enumerate}

This is exact by construction, not an approximation.

\begin{center}\rule{0.5\linewidth}{0.5pt}\end{center}

\section{Problem 8.5.10}\label{problem-8.5.10}

\textbf{Given:} A 3-stage cascaded integrator-comb (CIC) decimation
filter reduces the sampling rate from 10 MHz to 100 kHz (decimation
factor M = 100). The CIC filter has N = 3 stages and differential delay
D = 1.

\textbf{Find:} (a) The transfer function of the CIC filter, (b) the
first null frequency, (c) the passband droop at 40 kHz, and (d) whether
a compensation filter is needed.

\textbf{Solution:}

\begin{enumerate}
\def\labelenumi{(\alph{enumi})}
\tightlist
\item
  The CIC decimation filter transfer function (before downsampling):
  H(z) = {[}(1 - z⁻ᴹ)/(1 - z⁻¹){]}ᴺ = \textbf{{[}(1 - z⁻¹⁰⁰)/(1 -
  z⁻¹){]}³}
\end{enumerate}

The magnitude response: \textbar H(f)\textbar{} =
\textbar sin(πMf/f\textsubscript{s})/sin(πf/f\textsubscript{s})\textbar ᴺ

\begin{enumerate}
\def\labelenumi{(\alph{enumi})}
\setcounter{enumi}{1}
\tightlist
\item
  First null frequency: The first null occurs when
  sin(πMf/f\textsubscript{s}) = 0 and sin(πf/f\textsubscript{s}) ≠ 0:
  πMf/f\textsubscript{s} = π → f = f\textsubscript{s}/M = 10,000,000/100
  = \textbf{100 kHz}
\end{enumerate}

This conveniently falls at the decimated sampling rate, so the first
null acts as the anti-aliasing boundary.

\begin{enumerate}
\def\labelenumi{(\alph{enumi})}
\setcounter{enumi}{2}
\tightlist
\item
  Passband droop at 40 kHz (the edge of the desired passband, which is
  f\textsubscript{out}/2 = 50 kHz, so 40 kHz is within the passband):
  \textbar H(40 kHz)\textbar{} = \textbar sin(π × 100 ×
  40,000/10⁷)/sin(π × 40,000/10⁷)\textbar³ =
  \textbar sin(0.4π)/sin(0.004π)\textbar³ =
  \textbar sin(72°)/sin(0.72°)\textbar³ =
  \textbar0.9511/0.01257\textbar³ = \textbar75.66\textbar³ \ldots{}
\end{enumerate}

This is the unnormalized gain. The DC gain is M\^{}N = 100³ = 10⁶.
Normalized: \textbar H(40 kHz)\textbar{}\textsubscript{norm} =
\textbar sin(0.4π)/(100 × sin(0.004π))\textbar³ = (0.9511/(100 ×
0.01257))³ = (0.9511/1.257)³ = (0.7565)³ = \textbf{0.4328}

Droop in dB: 20log₁₀(0.4328) = \textbf{-7.28 dB}

\begin{enumerate}
\def\labelenumi{(\alph{enumi})}
\setcounter{enumi}{3}
\tightlist
\item
  A passband droop of 7.28 dB at 40 kHz is \textbf{severe} --- a
  compensation (droop correction) FIR filter operating at the decimated
  rate of 100 kHz is \textbf{required}. The compensator has an
  inverse-sinc response that boosts frequencies near the passband edge
  to flatten the overall response. Typically a short (5-15 tap) FIR
  filter with a mild highpass tilt suffices, adding negligible
  computational cost at the low output rate.
\end{enumerate}

\chapter{Chapter 8 --- Section 8.6: Spectral
Analysis}\label{chapter-8-section-8.6-spectral-analysis}

Practice problems covering power spectral density, Welch's method,
windowing trade-offs, time-frequency analysis (STFT), parametric
spectral estimation (AR models, MUSIC), cepstral analysis, MFCCs, and
wavelet transforms with denoising.

\begin{center}\rule{0.5\linewidth}{0.5pt}\end{center}

\section{Problem 8.6.1}\label{problem-8.6.1}

\textbf{Given:} A thermal noise source has a flat (white) power spectral
density of S(f) = 4.0 × 10⁻⁹ V²/Hz. The signal is measured through a
system with a noise bandwidth of 200 kHz.

\textbf{Find:} (a) The total noise power, (b) the RMS noise voltage, and
(c) the RMS noise voltage if a bandpass filter restricts the measurement
to 50 kHz bandwidth centered at 1 MHz.

\textbf{Solution:}

\begin{enumerate}
\def\labelenumi{(\alph{enumi})}
\item
  Total noise power: P = S(f) × BW = 4.0 × 10⁻⁹ × 200,000 = \textbf{8.0
  × 10⁻⁴ V² = 0.8 mV²}
\item
  RMS noise voltage: V\textsubscript{rms} = √P = √(8.0 × 10⁻⁴) =
  \textbf{0.02828 V = 28.28 mV}
\item
  With the 50 kHz bandpass filter: P\textsubscript{BP} = S(f) × 50,000 =
  4.0 × 10⁻⁹ × 50,000 = 2.0 × 10⁻⁴ V² V\textsubscript{rms,BP} = √(2.0 ×
  10⁻⁴) = \textbf{0.01414 V = 14.14 mV}
\end{enumerate}

The bandpass filter reduces the noise power by a factor of
200,000/50,000 = 4, and the RMS voltage by √4 = 2. This demonstrates
that narrowing the measurement bandwidth is the primary technique for
reducing noise in precision measurements.

\begin{center}\rule{0.5\linewidth}{0.5pt}\end{center}

\section{Problem 8.6.2}\label{problem-8.6.2}

\textbf{Given:} A signal is sampled at f\textsubscript{s} = 48 kHz, and
a 2048-point FFT is used for spectral analysis. The signal contains two
tones at 5,000 Hz and 5,080 Hz with equal amplitudes.

\textbf{Find:} (a) The frequency resolution with a rectangular window,
(b) whether the two tones can be resolved with a rectangular window, (c)
the frequency resolution with a Hanning window and whether the tones can
be resolved, and (d) the minimum FFT length required to resolve the
tones using a Hanning window.

\textbf{Solution:}

\begin{enumerate}
\def\labelenumi{(\alph{enumi})}
\tightlist
\item
  Rectangular window frequency resolution: Δf = f\textsubscript{s} / N =
  48,000 / 2,048 = \textbf{23.44 Hz}
\end{enumerate}

Main lobe width (rectangular) = 2 × Δf = 46.88 Hz.

\begin{enumerate}
\def\labelenumi{(\alph{enumi})}
\setcounter{enumi}{1}
\item
  The tone separation is 5,080 - 5,000 = 80 Hz. Since 80 Hz
  \textgreater{} 46.88 Hz (main lobe width), the two tones \textbf{can
  be resolved} with the rectangular window. However, sidelobes at -13 dB
  may cause artifacts.
\item
  Hanning window main lobe width = 4 × Δf = 4 × 23.44 = 93.75 Hz.
\end{enumerate}

Since 80 Hz \textless{} 93.75 Hz, the two tones \textbf{cannot be
resolved} with a Hanning window at N = 2048.

\begin{enumerate}
\def\labelenumi{(\alph{enumi})}
\setcounter{enumi}{3}
\tightlist
\item
  For the Hanning window to resolve the tones, the main lobe width must
  be less than the separation: 4 × (f\textsubscript{s}/N) \textless{} 80
  → N \textgreater{} 4 × 48,000/80 = 2,400
\end{enumerate}

The minimum FFT length is \textbf{N = 2,400} (next power of 2 would be N
= 4,096 for computational efficiency).

With N = 4,096: Δf = 48,000/4,096 = 11.72 Hz; Hanning main lobe = 46.88
Hz \textless{} 80 Hz → \textbf{resolved}.

\begin{center}\rule{0.5\linewidth}{0.5pt}\end{center}

\section{Problem 8.6.3}\label{problem-8.6.3}

\textbf{Given:} Welch's method is applied to estimate the PSD of a
10-second recording sampled at f\textsubscript{s} = 8 kHz. The data is
divided into segments of L = 1024 samples with 50\% overlap, and each
segment is windowed with a Hamming window.

\textbf{Find:} (a) The total number of samples, (b) the number of
overlapping segments, (c) the frequency resolution, (d) the variance
reduction factor compared to a single periodogram, and (e) the number of
frequency bins in the output PSD.

\textbf{Solution:}

\begin{enumerate}
\def\labelenumi{(\alph{enumi})}
\item
  Total samples: N = 10 × 8,000 = \textbf{80,000 samples}
\item
  With 50\% overlap, the hop size is L/2 = 512 samples. Number of
  segments: K = (N - L)/(L/2) + 1 = (80,000 - 1,024)/512 + 1 =
  78,976/512 + 1 = 154.25 + 1 ≈ \textbf{155 segments}
\item
  Frequency resolution: Δf = f\textsubscript{s} / L = 8,000 / 1,024 =
  \textbf{7.8125 Hz}
\item
  For Welch's method with 50\% overlap and Hamming window, the effective
  number of independent segments is approximately K × (1 -
  overlap\_correlation). For Hamming with 50\% overlap, the correlation
  between adjacent segments is about 0.5, giving an effective
  independent segment count of approximately K × 0.667 ≈ 155 × 0.667 ≈
  103.
\end{enumerate}

Variance reduction factor ≈ \textbf{103×} compared to a single
periodogram (the PSD estimate variance is reduced by a factor of
\textasciitilde103).

In the simplified case (assuming independent segments): variance
reduction ≈ K = \textbf{155×}.

\begin{enumerate}
\def\labelenumi{(\alph{enumi})}
\setcounter{enumi}{4}
\tightlist
\item
  Number of frequency bins: N\textsubscript{bins} = L/2 + 1 = 1,024/2 +
  1 = \textbf{513 bins} (covering 0 to 4,000 Hz)
\end{enumerate}

\begin{center}\rule{0.5\linewidth}{0.5pt}\end{center}

\section{Problem 8.6.4}\label{problem-8.6.4}

\textbf{Given:} An STFT is performed on a 5-second speech signal sampled
at f\textsubscript{s} = 22.05 kHz. A 1024-sample Hamming window with
75\% overlap is used.

\textbf{Find:} (a) The window duration in milliseconds, (b) the
frequency resolution, (c) the time resolution (hop size), (d) the total
number of STFT frames, and (e) the spectrogram dimensions (time frames ×
frequency bins).

\textbf{Solution:}

\begin{enumerate}
\def\labelenumi{(\alph{enumi})}
\item
  Window duration: T\textsubscript{w} = 1024 / 22,050 = \textbf{0.04644
  s = 46.44 ms}
\item
  Frequency resolution: Δf = f\textsubscript{s} / N = 22,050 / 1,024 =
  \textbf{21.53 Hz}
\item
  Hop size with 75\% overlap: Hop = 1024 × (1 - 0.75) = 256 samples Δt =
  256 / 22,050 = \textbf{0.01161 s = 11.61 ms}
\item
  Total samples: 5 × 22,050 = 110,250 Number of frames: (110,250 -
  1,024) / 256 + 1 = 109,226 / 256 + 1 = 426.7 + 1 ≈ \textbf{427 frames}
\item
  Frequency bins: N/2 + 1 = 513 Spectrogram dimensions: \textbf{427 ×
  513} (427 time frames by 513 frequency bins, covering 0 to 11.025 kHz)
\end{enumerate}

The spectrogram contains 427 × 513 = 219,051 complex-valued entries. At
75\% overlap, the time resolution is approximately 4× finer than with no
overlap, at the cost of 4× more computation.

\begin{center}\rule{0.5\linewidth}{0.5pt}\end{center}

\section{Problem 8.6.5}\label{problem-8.6.5}

\textbf{Given:} Two sinusoids at f₁ = 2,000 Hz and f₂ = 2,030 Hz (30 Hz
apart) are present in a signal sampled at f\textsubscript{s} = 8 kHz.
Only N = 128 samples are available. Both sinusoids have equal amplitude
(A = 1.0) with additive white Gaussian noise at SNR = 20 dB.

\textbf{Find:} (a) The FFT frequency resolution and whether classical
FFT can resolve the tones, (b) the required AR model order for
parametric estimation, and (c) the required data length for FFT-based
resolution of the two tones.

\textbf{Solution:}

\begin{enumerate}
\def\labelenumi{(\alph{enumi})}
\tightlist
\item
  FFT frequency resolution: Δf = f\textsubscript{s} / N = 8,000 / 128 =
  \textbf{62.5 Hz}
\end{enumerate}

Since the tone separation is 30 Hz \textless{} 62.5 Hz, the FFT
\textbf{cannot resolve} the two sinusoids. They will appear as a single
peak at approximately 2,015 Hz.

Even with zero-padding to 1,024 points (interpolated Δf = 7.8 Hz), the
fundamental resolution remains 62.5 Hz --- zero-padding smooths the
spectral shape but does not improve resolving power.

\begin{enumerate}
\def\labelenumi{(\alph{enumi})}
\setcounter{enumi}{1}
\tightlist
\item
  For two sinusoids, the signal can be modeled as an AR process with 2
  poles per sinusoid (representing the complex conjugate pair). The
  minimum AR model order is:
\end{enumerate}

p = 2 × (number of sinusoids) = 2 × 2 = \textbf{4}

In practice, p = 6 to 8 is chosen to account for noise. The Burg method
with p = 6 applied to the 128 samples will place spectral peaks near
2,000 Hz and 2,030 Hz.

Alternatively, the MUSIC algorithm with a 32 × 32 correlation matrix and
2 assumed signal components can achieve super-resolution, resolving the
30 Hz separation from only 128 samples at 20 dB SNR.

\begin{enumerate}
\def\labelenumi{(\alph{enumi})}
\setcounter{enumi}{2}
\tightlist
\item
  For FFT-based resolution with a rectangular window: N ≥
  f\textsubscript{s} / Δf\textsubscript{required} = 8,000 / 30 =
  \textbf{267 samples} (minimum)
\end{enumerate}

With a Hanning window (main lobe = 4 × Δf): N ≥ 4 × 8,000 / 30 =
\textbf{1,067 samples}

At f\textsubscript{s} = 8 kHz, this corresponds to 267/8,000 = 33 ms
(rectangular) or 1,067/8,000 = 133 ms (Hanning) of data.

\begin{center}\rule{0.5\linewidth}{0.5pt}\end{center}

\section{Problem 8.6.6}\label{problem-8.6.6}

\textbf{Given:} A speech signal is sampled at f\textsubscript{s} = 16
kHz. A 20 ms Hamming window is used for frame analysis (N = 320 samples,
zero-padded to 512). The speaker's fundamental frequency is f₀ = 200 Hz
(male speaker). A 24-channel mel filter bank is applied from 0 to 8 kHz.

\textbf{Find:} (a) The mel frequency range, (b) the pitch period in
samples, (c) the quefrency index of the pitch peak in the cepstrum, and
(d) the number of MFCC coefficients typically retained and their
significance.

\textbf{Solution:}

\begin{enumerate}
\def\labelenumi{(\alph{enumi})}
\tightlist
\item
  Mel frequency at 8 kHz: f\textsubscript{mel} = 2595 × log₁₀(1 +
  8000/700) = 2595 × log₁₀(12.43) = 2595 × 1.0943 = \textbf{2840 mel}
\end{enumerate}

The mel range spans 0 to 2840 mel, with the 24 triangular filters
uniformly spaced across this range.

\begin{enumerate}
\def\labelenumi{(\alph{enumi})}
\setcounter{enumi}{1}
\tightlist
\item
  Pitch period: T₀ = 1/f₀ = 1/200 = 5 ms
\end{enumerate}

In samples: T₀ = 5 × 10⁻³ × 16,000 = \textbf{80 samples}

\begin{enumerate}
\def\labelenumi{(\alph{enumi})}
\setcounter{enumi}{2}
\tightlist
\item
  In the 512-point cepstrum, the pitch appears as a peak at quefrency
  index: n = T₀ × f\textsubscript{s} / 1 = \textbf{80} (corresponding to
  80/16,000 = 5 ms)
\end{enumerate}

This peak is clearly separated from the vocal tract information, which
is concentrated in cepstral indices 0 to about 50 (0 to 3.125 ms).

\begin{enumerate}
\def\labelenumi{(\alph{enumi})}
\setcounter{enumi}{3}
\tightlist
\item
  Typically \textbf{13 MFCCs} (c₀ through c₁₂) are retained. These
  coefficients capture the smooth spectral envelope that encodes phoneme
  identity:
\end{enumerate}

\begin{itemize}
\tightlist
\item
  c₀ represents the overall log energy of the frame
\item
  c₁ through c₁₂ capture the spectral shape (formant locations and
  bandwidths)
\item
  Higher-order coefficients (c₁₃ and above) represent fine spectral
  detail (pitch harmonics) and are discarded
\end{itemize}

In practice, delta (velocity) and delta-delta (acceleration)
coefficients are appended, giving a total of \textbf{39 features per
frame} (13 static + 13 delta + 13 delta-delta) for speech recognition
systems.

\begin{center}\rule{0.5\linewidth}{0.5pt}\end{center}

\section{Problem 8.6.7}\label{problem-8.6.7}

\textbf{Given:} A vibration signal from a rotating machine is sampled at
f\textsubscript{s} = 4 kHz for 2 seconds. The signal contains a known
shaft frequency at f\textsubscript{shaft} = 25 Hz and harmonics, plus
broadband bearing noise. The PSD is estimated using Welch's method with
L = 512 segments and 50\% overlap.

\textbf{Find:} (a) The frequency resolution, (b) the number of segments,
(c) the total noise power if the broadband PSD level is
S\textsubscript{n} = 5.0 × 10⁻⁶ V²/Hz from 0 to 2 kHz, (d) the signal
power in a 10 Hz band centered on the shaft frequency if the peak PSD is
S\textsubscript{peak} = 2.0 × 10⁻³ V²/Hz, and (e) the SNR in that band.

\textbf{Solution:}

\begin{enumerate}
\def\labelenumi{(\alph{enumi})}
\item
  Frequency resolution: Δf = f\textsubscript{s} / L = 4,000 / 512 =
  \textbf{7.8125 Hz}
\item
  Total samples: N = 2 × 4,000 = 8,000 Hop size = L/2 = 256 Number of
  segments: K = (8,000 - 512)/256 + 1 = 7,488/256 + 1 = 29.25 + 1 ≈
  \textbf{30 segments}
\item
  Total broadband noise power: P\textsubscript{n} = S\textsubscript{n} ×
  BW = 5.0 × 10⁻⁶ × 2,000 = \textbf{0.01 V² = 10 mV²}
\end{enumerate}

V\textsubscript{rms,noise} = √0.01 = \textbf{0.1 V = 100 mV}

\begin{enumerate}
\def\labelenumi{(\alph{enumi})}
\setcounter{enumi}{3}
\item
  Signal power in 10 Hz band around shaft frequency:
  P\textsubscript{signal} = S\textsubscript{peak} × 10 = 2.0 × 10⁻³ × 10
  = \textbf{0.02 V² = 20 mV²}
\item
  Noise power in the same 10 Hz band: P\textsubscript{n,band} =
  S\textsubscript{n} × 10 = 5.0 × 10⁻⁶ × 10 = 5.0 × 10⁻⁵ V²
\end{enumerate}

SNR = P\textsubscript{signal} / P\textsubscript{n,band} = 0.02 / 5.0 ×
10⁻⁵ = 400

SNR\textsubscript{dB} = 10 × log₁₀(400) = \textbf{26.0 dB}

The shaft frequency is clearly detectable above the broadband noise
floor.

\begin{center}\rule{0.5\linewidth}{0.5pt}\end{center}

\section{Problem 8.6.8}\label{problem-8.6.8}

\textbf{Given:} A noisy ECG signal (1024 samples, f\textsubscript{s} =
500 Hz) is decomposed using a 4-level DWT with Daubechies db6 wavelets.
The noise standard deviation is estimated at σ = 0.15 mV.

\textbf{Find:} (a) The frequency bands at each decomposition level, (b)
the number of coefficients at each level, (c) the universal threshold λ,
and (d) the expected effect of soft thresholding on the ECG signal.

\textbf{Solution:}

\begin{enumerate}
\def\labelenumi{(\alph{enumi})}
\item
  At each DWT level, the bandwidth is halved: Level 1 detail (d₁):
  \textbf{125--250 Hz} (highest frequency noise, muscle artifacts) Level
  2 detail (d₂): \textbf{62.5--125 Hz} (high-frequency noise) Level 3
  detail (d₃): \textbf{31.25--62.5 Hz} (includes some high-frequency ECG
  components) Level 4 detail (d₄): \textbf{15.625--31.25 Hz} (QRS
  complex energy band) Level 4 approximation (a₄): \textbf{0--15.625 Hz}
  (baseline, P and T waves)
\item
  Coefficients at each level (downsampled by 2 at each level): d₁:
  1024/2 = \textbf{512 coefficients} d₂: 512/2 = \textbf{256
  coefficients} d₃: 256/2 = \textbf{128 coefficients} d₄: 128/2 =
  \textbf{64 coefficients} a₄: \textbf{64 coefficients}
\end{enumerate}

Total wavelet coefficients = 512 + 256 + 128 + 64 + 64 = 1024 (same as
original signal length --- DWT preserves information).

\begin{enumerate}
\def\labelenumi{(\alph{enumi})}
\setcounter{enumi}{2}
\item
  Universal threshold: λ = σ√(2 ln N) = 0.15 × √(2 × ln(1024)) = 0.15 ×
  √(2 × 6.931) = 0.15 × √13.863 = 0.15 × 3.724 = \textbf{0.559 mV}
\item
  Soft thresholding is applied to the detail coefficients d₁ through d₄.
  The approximation coefficients a₄ are left unchanged (they contain the
  baseline ECG waveform). Detail coefficients with magnitudes below
  0.559 mV are set to zero (removing noise), while larger coefficients
  are shrunk by λ toward zero. The QRS complex, which produces large
  coefficients in d₃ and d₄, is preserved because its amplitude
  (typically 1--3 mV) exceeds the threshold. High-frequency muscle noise
  in d₁ and d₂ is largely removed. This method preserves sharp QRS edges
  better than a conventional lowpass filter.
\end{enumerate}

\begin{center}\rule{0.5\linewidth}{0.5pt}\end{center}

\section{Problem 8.6.9}\label{problem-8.6.9}

\textbf{Given:} A signal of length N = 512 samples at f\textsubscript{s}
= 10 kHz contains 3 sinusoidal components in white Gaussian noise. The
MUSIC algorithm is applied using a correlation matrix of size M × M,
where M = 64.

\textbf{Find:} (a) The number of signal and noise subspace eigenvectors,
(b) the minimum SNR typically required for MUSIC to resolve two closely
spaced sinusoids, (c) the theoretical frequency resolution advantage
over FFT, and (d) the frequency resolution of a standard 512-point FFT
for comparison.

\textbf{Solution:}

\begin{enumerate}
\def\labelenumi{(\alph{enumi})}
\tightlist
\item
  With 3 sinusoidal components (each producing a complex conjugate pair
  in the correlation matrix but counted as one signal dimension for real
  sinusoids observed via real-valued data --- for the general complex
  case, each sinusoid uses one eigenvector):
\end{enumerate}

Signal subspace dimension: \textbf{3 eigenvectors} (one per sinusoidal
component, for the complex data model) Noise subspace dimension: M - 3 =
64 - 3 = \textbf{61 eigenvectors}

MUSIC forms the pseudospectrum by projecting the frequency steering
vector onto the noise subspace. The pseudospectrum peaks sharply where
the steering vector is orthogonal to the noise subspace, indicating the
signal frequencies.

\begin{enumerate}
\def\labelenumi{(\alph{enumi})}
\setcounter{enumi}{1}
\item
  MUSIC typically requires SNR \textgreater{} \textbf{0 to 10 dB} for
  reliable super-resolution, with performance degrading rapidly below 0
  dB. For closely spaced sinusoids (separation \textless{}
  Δf\textsubscript{FFT}), SNR \textgreater{} \textbf{10 dB} is generally
  needed.
\item
  MUSIC's resolution is not limited by the data record length N or the
  FFT bin width. It is determined by the correlation matrix size M and
  the SNR. For M = 64 and adequate SNR, MUSIC can resolve sinusoids
  separated by as little as:
\end{enumerate}

Δf\textsubscript{MUSIC} ≈ f\textsubscript{s} / (M ×
√SNR\textsubscript{linear})

At SNR = 20 dB (linear = 100): Δf ≈ 10,000 / (64 × 10) = \textbf{15.6
Hz}

\begin{enumerate}
\def\labelenumi{(\alph{enumi})}
\setcounter{enumi}{3}
\tightlist
\item
  Standard FFT resolution: Δf\textsubscript{FFT} = f\textsubscript{s} /
  N = 10,000 / 512 = \textbf{19.53 Hz}
\end{enumerate}

MUSIC achieves approximately 15.6 Hz resolution from only 64-sample
windows, slightly better than the 19.53 Hz FFT resolution from the full
512 samples. The real advantage of MUSIC is when N is small --- for N =
64 samples, FFT resolution would be 156.25 Hz while MUSIC still achieves
\textasciitilde15.6 Hz.

\begin{center}\rule{0.5\linewidth}{0.5pt}\end{center}

\section{Problem 8.6.10}\label{problem-8.6.10}

\textbf{Given:} A 3-level DWT is performed on a 2048-sample signal at
f\textsubscript{s} = 16 kHz using Haar wavelets. The signal contains a
transient event (a 1 ms pulse) occurring at sample 1000 and a 200 Hz
sinusoidal component.

\textbf{Find:} (a) The frequency bands at each DWT level, (b) which
decomposition level best captures the transient event and why, (c) the
time resolution at each level, and (d) which level captures the 200 Hz
sinusoid.

\textbf{Solution:}

\begin{enumerate}
\def\labelenumi{(\alph{enumi})}
\item
  Frequency bands: Level 1 detail (d₁): \textbf{4--8 kHz} (1024
  coefficients) Level 2 detail (d₂): \textbf{2--4 kHz} (512
  coefficients) Level 3 detail (d₃): \textbf{1--2 kHz} (256
  coefficients) Level 3 approximation (a₃): \textbf{0--1 kHz} (256
  coefficients)
\item
  The 1 ms transient pulse has a bandwidth of approximately 1/0.001 = 1
  kHz, meaning its spectral content extends from 0 to several kHz.
\end{enumerate}

The transient will appear in \textbf{all detail levels} (d₁, d₂, d₃)
because its broadband energy spans all frequency bands. However, the
transient is best localized in \textbf{level 1 (d₁)} because: - d₁ has
the finest time resolution (1 sample = 0.0625 ms per Haar wavelet
coefficient at this level) - The transient's energy in the 4--8 kHz band
produces a sharp spike in d₁ near coefficient index 500 (corresponding
to sample 1000)

\begin{enumerate}
\def\labelenumi{(\alph{enumi})}
\setcounter{enumi}{2}
\item
  Time resolution at each level (Haar wavelet width): Level 1: 2 samples
  = 2/16,000 = \textbf{0.125 ms} Level 2: 4 samples = 4/16,000 =
  \textbf{0.25 ms} Level 3: 8 samples = 8/16,000 = \textbf{0.5 ms}
\item
  The 200 Hz sinusoid falls in the 0--1 kHz band, which is captured by
  the \textbf{level 3 approximation coefficients (a₃)}. The a₃
  coefficients contain all energy from 0 to 1 kHz, including the 200 Hz
  component. With 256 coefficients representing this band, the effective
  frequency resolution is 1,000/256 = 3.91 Hz --- sufficient to
  represent the 200 Hz sinusoid.
\end{enumerate}

This illustrates the DWT's multi-resolution property: the transient is
well-localized in time at level 1, while the sinusoid is well-localized
in frequency at level 3.

\chapter{Chapter 8 --- Section 8.7: Adaptive
Filtering}\label{chapter-8-section-8.7-adaptive-filtering}

Practice problems covering the LMS algorithm, Normalized LMS, adaptive
noise cancellation, echo cancellation, channel equalization, Recursive
Least Squares (RLS), Kalman filtering, Wiener filter design, and
compressive sensing.

\begin{center}\rule{0.5\linewidth}{0.5pt}\end{center}

\section{Problem 8.7.1}\label{problem-8.7.1}

\textbf{Given:} An LMS adaptive filter of order M = 8 is used for system
identification. The input signal power is P\textsubscript{x} = 2.0 W.

\textbf{Find:} (a) The maximum step size μ\textsubscript{max} for
convergence, (b) a practical step size at 5\% of the maximum, (c) the
approximate number of iterations to converge within 10\% of the optimal
solution using the time constant formula τ = 1/(4μMP\textsubscript{x}),
and (d) the steady-state excess mean squared error (MSE) relative to the
minimum MSE, given by the misadjustment factor M\textsubscript{adj} =
μMP\textsubscript{x}.

\textbf{Solution:}

\begin{enumerate}
\def\labelenumi{(\alph{enumi})}
\item
  Maximum step size: μ\textsubscript{max} = 1/(M × P\textsubscript{x}) =
  1/(8 × 2.0) = \textbf{0.0625}
\item
  Practical step size: μ = 0.05 × μ\textsubscript{max} = 0.05 × 0.0625 =
  \textbf{0.003125}
\item
  Convergence time constant: τ = 1/(4μMP\textsubscript{x}) = 1/(4 ×
  0.003125 × 8 × 2.0) = 1/0.2 = \textbf{5 iterations} (time constant)
\end{enumerate}

To converge within 10\% of optimal (2.3 time constants): 2.3 × 5 =
\textbf{\textasciitilde12 iterations}

To converge within 1\% of optimal (4.6 time constants): 4.6 × 5 =
\textbf{\textasciitilde23 iterations}

\begin{enumerate}
\def\labelenumi{(\alph{enumi})}
\setcounter{enumi}{3}
\tightlist
\item
  Misadjustment factor: M\textsubscript{adj} = μMP\textsubscript{x} =
  0.003125 × 8 × 2.0 = \textbf{0.05 = 5\%}
\end{enumerate}

This means the steady-state MSE is 5\% above the minimum achievable MSE
(Wiener solution). The excess MSE is the penalty for using a stochastic
gradient instead of the true gradient.

\begin{center}\rule{0.5\linewidth}{0.5pt}\end{center}

\section{Problem 8.7.2}\label{problem-8.7.2}

\textbf{Given:} An LMS adaptive filter has M = 6 taps, step size μ =
0.01, current weight vector w = {[}0.5, -0.3, 0.2, 0.1, -0.1, 0.4{]},
input vector x = {[}0.8, -0.6, 0.4, 1.0, -0.2, 0.5{]}, and desired
signal d = 1.2.

\textbf{Find:} (a) The filter output y, (b) the error signal e, (c) the
updated weight vector after one LMS iteration, and (d) the squared error
before and expected after the update.

\textbf{Solution:}

\begin{enumerate}
\def\labelenumi{(\alph{enumi})}
\item
  Filter output: y = w\textsuperscript{T}x = 0.5(0.8) + (-0.3)(-0.6) +
  0.2(0.4) + 0.1(1.0) + (-0.1)(-0.2) + 0.4(0.5) = 0.40 + 0.18 + 0.08 +
  0.10 + 0.02 + 0.20 = \textbf{0.98}
\item
  Error: e = d - y = 1.2 - 0.98 = \textbf{0.22}
\item
  Weight update: w{[}n+1{]} = w{[}n{]} + 2μe × x
\end{enumerate}

2μe = 2 × 0.01 × 0.22 = 0.0044

Update vector = 0.0044 × {[}0.8, -0.6, 0.4, 1.0, -0.2, 0.5{]} =
{[}0.00352, -0.00264, 0.00176, 0.00440, -0.00088, 0.00220{]}

w{[}n+1{]} = {[}0.50352, -0.30264, 0.20176, 0.10440, -0.10088,
0.40220{]}

Rounded: \textbf{w = {[}0.5035, -0.3026, 0.2018, 0.1044, -0.1009,
0.4022{]}}

\begin{enumerate}
\def\labelenumi{(\alph{enumi})}
\setcounter{enumi}{3}
\tightlist
\item
  Squared error before update: e² = 0.22² = \textbf{0.0484}
\end{enumerate}

After the update, the new output would be approximately: y' =
w{[}n+1{]}\textsuperscript{T}x ≈ 0.98 + 2μe × x\textsuperscript{T}x =
0.98 + 0.0044 × (0.64 + 0.36 + 0.16 + 1.00 + 0.04 + 0.25) = 0.98 +
0.0044 × 2.45 = 0.98 + 0.01078 = 0.9908

Expected new error: 1.2 - 0.9908 = 0.2092, e² ≈ \textbf{0.0438} (a 9.5\%
reduction in squared error)

\begin{center}\rule{0.5\linewidth}{0.5pt}\end{center}

\section{Problem 8.7.3}\label{problem-8.7.3}

\textbf{Given:} An NLMS (Normalized LMS) adaptive filter has M = 10
taps, normalized step size μ̃ = 0.5, and regularization parameter δ =
0.001. The input signal has instantaneous power
\textbar\textbar x{[}n{]}\textbar\textbar² = 3.2.

\textbf{Find:} (a) The effective step size for this sample, (b)
comparison with a standard LMS step size of μ = 0.02, and (c) the
effective step size when \textbar\textbar x{[}n{]}\textbar\textbar²
drops to 0.1 (low-power segment).

\textbf{Solution:}

\begin{enumerate}
\def\labelenumi{(\alph{enumi})}
\tightlist
\item
  NLMS update: w{[}n+1{]} = w{[}n{]} + (μ̃ /
  (\textbar\textbar x{[}n{]}\textbar\textbar² + δ)) × e{[}n{]} ×
  x{[}n{]}
\end{enumerate}

Effective step size: μ\textsubscript{eff} = μ̃ /
(\textbar\textbar x{[}n{]}\textbar\textbar² + δ) = 0.5 / (3.2 + 0.001) =
0.5 / 3.201 = \textbf{0.1562}

\begin{enumerate}
\def\labelenumi{(\alph{enumi})}
\setcounter{enumi}{1}
\tightlist
\item
  The standard LMS with μ = 0.02 applies the same step size regardless
  of input power. The NLMS effective step size of 0.1562 is much larger,
  meaning NLMS converges faster when the input power is high.
\end{enumerate}

Maximum stable LMS step size: μ\textsubscript{max} =
1/(MP\textsubscript{x}) = 1/(10 × 0.32) = 0.3125 (using average power =
\textbar\textbar x\textbar\textbar²/M = 3.2/10 = 0.32).

The NLMS effective step size of 0.1562 is 50\% of μ\textsubscript{max},
which is within the stable range.

\begin{enumerate}
\def\labelenumi{(\alph{enumi})}
\setcounter{enumi}{2}
\tightlist
\item
  When \textbar\textbar x{[}n{]}\textbar\textbar² = 0.1 (quiet segment):
  μ\textsubscript{eff} = 0.5 / (0.1 + 0.001) = 0.5 / 0.101 =
  \textbf{4.95}
\end{enumerate}

Without the regularization parameter, μ\textsubscript{eff} = 0.5/0.1 =
5.0, which would cause instability. The regularization δ = 0.001 limits
the effective step size during very low-power segments but does not
prevent the large value here. A more robust choice would be δ = 0.01 or
using a power-estimated NLMS variant.

In practice, the NLMS step size μ̃ should satisfy 0 \textless{} μ̃
\textless{} 2 for stability, and the normalization automatically
prevents the update from becoming too large relative to the input energy
--- the weight correction magnitude is bounded by μ̃ ×
\textbar e{[}n{]}\textbar{} /
\textbar\textbar x{[}n{]}\textbar\textbar{} regardless of the input
power level.

\begin{center}\rule{0.5\linewidth}{0.5pt}\end{center}

\section{Problem 8.7.4}\label{problem-8.7.4}

\textbf{Given:} A telephone echo canceller uses a 512-tap LMS adaptive
filter at f\textsubscript{s} = 8 kHz. The echo path has a flat-delay
component of 30 ms followed by an exponentially decaying room impulse
response with a time constant of 20 ms. The near-end speech power is
P\textsubscript{s} = 0.01 W and the far-end speech power is
P\textsubscript{x} = 0.05 W.

\textbf{Find:} (a) The duration of the echo path that can be modeled,
(b) whether the filter length is sufficient, (c) the maximum LMS step
size for convergence, (d) the computational cost per sample in
multiply-accumulate (MAC) operations, and (e) the echo return loss
enhancement (ERLE) target and its relation to the misadjustment.

\textbf{Solution:}

\begin{enumerate}
\def\labelenumi{(\alph{enumi})}
\item
  Echo path duration covered by the filter: T\textsubscript{filter} = M
  / f\textsubscript{s} = 512 / 8,000 = \textbf{64 ms}
\item
  The echo path requires: flat delay (30 ms) + decay time (at least 3τ =
  60 ms for 95\% decay) = 90 ms total.
\end{enumerate}

Since 64 ms \textless{} 90 ms, the 512-tap filter is \textbf{not
sufficient}. Minimum required taps: 90 × 10⁻³ × 8,000 = 720 taps. A
practical choice would be \textbf{1024 taps} (128 ms), providing margin
for the echo tail.

\begin{enumerate}
\def\labelenumi{(\alph{enumi})}
\setcounter{enumi}{2}
\tightlist
\item
  Maximum step size: μ\textsubscript{max} = 1/(M × P\textsubscript{x}) =
  1/(512 × 0.05) = 1/25.6 = \textbf{0.0391}
\end{enumerate}

A practical step size: μ = 0.1 × μ\textsubscript{max} = \textbf{0.00391}

\begin{enumerate}
\def\labelenumi{(\alph{enumi})}
\setcounter{enumi}{3}
\tightlist
\item
  Computational cost per sample: LMS requires 2M + 1 MAC operations per
  sample = 2 × 512 + 1 = \textbf{1,025 MACs} (512 for filter output, 512
  for weight update, 1 for error computation)
\end{enumerate}

At f\textsubscript{s} = 8 kHz: 1,025 × 8,000 = \textbf{8.2 MMAC/s}
(easily handled by any modern DSP)

\begin{enumerate}
\def\labelenumi{(\alph{enumi})}
\setcounter{enumi}{4}
\tightlist
\item
  The ERLE measures how well the adaptive filter cancels the echo: ERLE
  = 10 × log₁₀(P\textsubscript{echo} / P\textsubscript{residual})
\end{enumerate}

For LMS with misadjustment M\textsubscript{adj} = μMP\textsubscript{x} =
0.00391 × 512 × 0.05 = 0.1: ERLE ≈ -10 × log₁₀(M\textsubscript{adj}) =
-10 × log₁₀(0.1) = \textbf{10 dB}

Typical targets are 20--30 dB ERLE, requiring a smaller step size (μ ≈
0.0004) or a normalized algorithm.

\begin{center}\rule{0.5\linewidth}{0.5pt}\end{center}

\section{Problem 8.7.5}\label{problem-8.7.5}

\textbf{Given:} An RLS adaptive equalizer has M = 4 taps and forgetting
factor λ = 0.98. The initial inverse correlation matrix is P{[}0{]} =
100 × I (4 × 4 identity scaled by 100). At time n = 1, the input vector
is x = {[}1.0, 0.5, -0.3, 0.2{]}\textsuperscript{T}, the desired signal
is d = 0.8, and the current weight vector is w = {[}0, 0, 0,
0{]}\textsuperscript{T}.

\textbf{Find:} (a) The a priori error, (b) the gain vector k, (c) the
updated weight vector, and (d) the updated P matrix.

\textbf{Solution:}

\begin{enumerate}
\def\labelenumi{(\alph{enumi})}
\item
  A priori error: y = w\textsuperscript{T}x = {[}0,0,0,0{]} × {[}1.0,
  0.5, -0.3, 0.2{]}\textsuperscript{T} = 0 e = d - y = 0.8 - 0 =
  \textbf{0.8}
\item
  Gain vector: k = P{[}0{]}x / (λ + x\textsuperscript{T}P{[}0{]}x)
\end{enumerate}

P{[}0{]}x = 100I × x = 100 × {[}1.0, 0.5, -0.3,
0.2{]}\textsuperscript{T} = {[}100, 50, -30, 20{]}\textsuperscript{T}

x\textsuperscript{T}P{[}0{]}x = x\textsuperscript{T} × 100x = 100 ×
(1.0² + 0.5² + 0.3² + 0.2²) = 100 × (1.0 + 0.25 + 0.09 + 0.04) = 100 ×
1.38 = 138

k = {[}100, 50, -30, 20{]}\textsuperscript{T} / (0.98 + 138) = {[}100,
50, -30, 20{]}\textsuperscript{T} / 138.98

k = \textbf{{[}0.7195, 0.3598, -0.2159, 0.1439{]}\textsuperscript{T}}

\begin{enumerate}
\def\labelenumi{(\alph{enumi})}
\setcounter{enumi}{2}
\tightlist
\item
  Updated weights: w{[}1{]} = w{[}0{]} + k × e = {[}0,0,0,0{]} + 0.8 ×
  {[}0.7195, 0.3598, -0.2159, 0.1439{]}
\end{enumerate}

w{[}1{]} = \textbf{{[}0.5756, 0.2878, -0.1727,
0.1152{]}\textsuperscript{T}}

\begin{enumerate}
\def\labelenumi{(\alph{enumi})}
\setcounter{enumi}{3}
\tightlist
\item
  Updated P matrix: P{[}1{]} = (P{[}0{]} - k × x\textsuperscript{T} ×
  P{[}0{]}) / λ
\end{enumerate}

k × x\textsuperscript{T} × P{[}0{]} = k × {[}100, 50, -30,
20{]}\textsuperscript{T}

This is the outer product of k (4×1) with P{[}0{]}x\textsuperscript{T}
(1×4):

k × (P{[}0{]}x)\textsuperscript{T} = {[}0.7195; 0.3598; -0.2159;
0.1439{]} × {[}100, 50, -30, 20{]}

The diagonal entries of P{[}0{]} - k(P{[}0{]}x)\textsuperscript{T}: P₁₁
= 100 - 0.7195 × 100 = 100 - 71.95 = 28.05 P₂₂ = 100 - 0.3598 × 50 = 100
- 17.99 = 82.01 P₃₃ = 100 - (-0.2159)(-30) = 100 - 6.477 = 93.52 P₄₄ =
100 - 0.1439 × 20 = 100 - 2.878 = 97.12

Dividing by λ = 0.98: P₁₁ = 28.62, P₂₂ = 83.68, P₃₃ = 95.43, P₄₄ = 99.10

The diagonal entries show that uncertainty has decreased most for the
first tap (which had the largest input component), demonstrating how RLS
allocates learning proportional to the input signal strength.

\begin{center}\rule{0.5\linewidth}{0.5pt}\end{center}

\section{Problem 8.7.6}\label{problem-8.7.6}

\textbf{Given:} A 1-D Kalman filter tracks the voltage of a slowly
drifting battery. The state is x = battery voltage (V). The state
transition model is x{[}n{]} = x{[}n-1{]} + w{[}n{]} (random walk), the
measurement model is z{[}n{]} = x{[}n{]} + v{[}n{]}, the process noise
variance is Q = 0.001 V², and the measurement noise variance is R = 0.25
V². The initial estimate is x̂{[}0{]} = 12.0 V with error variance
P{[}0{]} = 1.0 V². Three measurements arrive: z{[}1{]} = 11.8 V,
z{[}2{]} = 12.1 V, z{[}3{]} = 11.9 V.

\textbf{Find:} The Kalman gain, updated state estimate, and error
variance after each measurement.

\textbf{Solution:}

\textbf{Measurement 1 (z{[}1{]} = 11.8 V):} Prediction:
x̂{[}1\textbar0{]} = x̂{[}0{]} = 12.0 V, P{[}1\textbar0{]} = P{[}0{]} + Q
= 1.0 + 0.001 = 1.001

Kalman gain: K{[}1{]} = P{[}1\textbar0{]}/(P{[}1\textbar0{]} + R) =
1.001/(1.001 + 0.25) = 1.001/1.251 = \textbf{0.800}

Update: x̂{[}1{]} = 12.0 + 0.800 × (11.8 - 12.0) = 12.0 + 0.800 × (-0.2)
= 12.0 - 0.16 = \textbf{11.84 V} P{[}1{]} = (1 - 0.800) × 1.001 = 0.200
× 1.001 = \textbf{0.2002 V²}

\textbf{Measurement 2 (z{[}2{]} = 12.1 V):} Prediction:
x̂{[}2\textbar1{]} = 11.84 V, P{[}2\textbar1{]} = 0.2002 + 0.001 = 0.2012

K{[}2{]} = 0.2012/(0.2012 + 0.25) = 0.2012/0.4512 = \textbf{0.446}

Update: x̂{[}2{]} = 11.84 + 0.446 × (12.1 - 11.84) = 11.84 + 0.446 × 0.26
= 11.84 + 0.116 = \textbf{11.956 V} P{[}2{]} = (1 - 0.446) × 0.2012 =
0.554 × 0.2012 = \textbf{0.1115 V²}

\textbf{Measurement 3 (z{[}3{]} = 11.9 V):} Prediction:
x̂{[}3\textbar2{]} = 11.956 V, P{[}3\textbar2{]} = 0.1115 + 0.001 =
0.1125

K{[}3{]} = 0.1125/(0.1125 + 0.25) = 0.1125/0.3625 = \textbf{0.310}

Update: x̂{[}3{]} = 11.956 + 0.310 × (11.9 - 11.956) = 11.956 + 0.310 ×
(-0.056) = 11.956 - 0.017 = \textbf{11.939 V} P{[}3{]} = (1 - 0.310) ×
0.1125 = 0.690 × 0.1125 = \textbf{0.0776 V²}

The Kalman gain decreases (0.800 → 0.446 → 0.310) as the estimate
becomes more confident. The error variance drops from 1.0 to 0.0776 V²
(standard deviation from 1.0 V to 0.279 V). The filter converges toward
a steady-state gain of K\textsubscript{ss} ≈ Q/(Q + Q) × (something),
which can be found by solving P\textsubscript{ss} = (P\textsubscript{ss}
+ Q)R / (P\textsubscript{ss} + Q + R). Eventually, the Kalman gain will
settle at approximately \textbf{K\textsubscript{ss} ≈ 0.06}, balancing
the slow process drift against measurement noise.

\begin{center}\rule{0.5\linewidth}{0.5pt}\end{center}

\section{Problem 8.7.7}\label{problem-8.7.7}

\textbf{Given:} A Kalman filter tracks a falling object under gravity.
The state vector is x = {[}height (m), velocity
(m/s){]}\textsuperscript{T}. The discrete state transition (Δt = 0.1 s)
is A = {[}{[}1, 0.1{]}, {[}0, 1{]}{]}, the control input is B =
{[}-0.5(0.1)², -0.1{]}\textsuperscript{T} × g = {[}-0.049,
-0.981{]}\textsuperscript{T} (free fall with g = 9.81 m/s²), H = {[}1,
0{]}, Q = {[}{[}0.01, 0{]}, {[}0, 0.1{]}{]}, and R = 4 m². At n = 0: x̂ =
{[}1000, 0{]}\textsuperscript{T}, P = {[}{[}10, 0{]}, {[}0, 1{]}{]}. A
radar measurement z = 994 m arrives at n = 1.

\textbf{Find:} (a) The predicted state, (b) the predicted covariance,
(c) the Kalman gain, and (d) the updated state estimate.

\textbf{Solution:}

\begin{enumerate}
\def\labelenumi{(\alph{enumi})}
\tightlist
\item
  Predicted state (free fall, no control input u --- gravity is
  incorporated in B): x̂{[}1\textbar0{]} = Ax̂{[}0{]} + B
\end{enumerate}

x̂{[}1\textbar0{]} = {[}{[}1, 0.1{]}, {[}0, 1{]}{]} × {[}1000,
0{]}\textsuperscript{T} + {[}-0.049, -0.981{]}\textsuperscript{T}

= {[}1000 + 0, 0{]}\textsuperscript{T} + {[}-0.049,
-0.981{]}\textsuperscript{T} = \textbf{{[}999.951,
-0.981{]}\textsuperscript{T}}

The object has fallen 0.049 m and has velocity -0.981 m/s (downward)
after 0.1 s.

\begin{enumerate}
\def\labelenumi{(\alph{enumi})}
\setcounter{enumi}{1}
\tightlist
\item
  Predicted covariance: P{[}1\textbar0{]} = APA\textsuperscript{T} + Q
\end{enumerate}

APA\textsuperscript{T} = {[}{[}1, 0.1{]}, {[}0, 1{]}{]} × {[}{[}10,
0{]}, {[}0, 1{]}{]} × {[}{[}1, 0{]}, {[}0.1, 1{]}{]}

First: AP = {[}{[}10, 0.1{]}, {[}0, 1{]}{]}

APA\textsuperscript{T} = {[}{[}10, 0.1{]}, {[}0, 1{]}{]} × {[}{[}1,
0{]}, {[}0.1, 1{]}{]} = {[}{[}10 + 0.01, 0 + 0.1{]}, {[}0 + 0.1, 0 +
1{]}{]} = {[}{[}10.01, 0.1{]}, {[}0.1, 1{]}{]}

P{[}1\textbar0{]} = {[}{[}10.01, 0.1{]}, {[}0.1, 1{]}{]} + {[}{[}0.01,
0{]}, {[}0, 0.1{]}{]} = \textbf{{[}{[}10.02, 0.1{]}, {[}0.1, 1.1{]}{]}}

\begin{enumerate}
\def\labelenumi{(\alph{enumi})}
\setcounter{enumi}{2}
\tightlist
\item
  Kalman gain: S = HP{[}1\textbar0{]}H\textsuperscript{T} + R = {[}1,
  0{]} × {[}{[}10.02, 0.1{]}, {[}0.1, 1.1{]}{]} × {[}1,
  0{]}\textsuperscript{T} + 4 = 10.02 + 4 = 14.02
\end{enumerate}

K = P{[}1\textbar0{]}H\textsuperscript{T} / S = {[}10.02,
0.1{]}\textsuperscript{T} / 14.02 = \textbf{{[}0.715,
0.00713{]}\textsuperscript{T}}

\begin{enumerate}
\def\labelenumi{(\alph{enumi})}
\setcounter{enumi}{3}
\tightlist
\item
  Innovation: z - Hx̂{[}1\textbar0{]} = 994 - 999.951 = -5.951 m
\end{enumerate}

Updated state: x̂{[}1{]} = {[}999.951, -0.981{]}\textsuperscript{T} +
{[}0.715, 0.00713{]}\textsuperscript{T} × (-5.951)

= {[}999.951 - 4.255, -0.981 - 0.0424{]}\textsuperscript{T} =
\textbf{{[}995.696, -1.023{]}\textsuperscript{T}}

The Kalman filter corrects the predicted height (999.95 m) significantly
toward the measurement (994 m), pulling it to 995.7 m. The velocity
estimate also increases slightly (from -0.981 to -1.023 m/s) because the
lower-than-expected altitude implies the object is falling faster.

\begin{center}\rule{0.5\linewidth}{0.5pt}\end{center}

\section{Problem 8.7.8}\label{problem-8.7.8}

\textbf{Given:} A signal d(t) with PSD S\textsubscript{dd}(f) = 10⁻³ /
(1 + (f/500)²) V²/Hz (Lorentzian spectrum centered at DC with 3 dB
bandwidth of 500 Hz) is observed in additive white noise with
S\textsubscript{nn}(f) = 5 × 10⁻⁶ V²/Hz. The measurement bandwidth
extends from 0 to 5 kHz.

\textbf{Find:} (a) The Wiener filter frequency response H(f) at f = 0,
500, 1000, and 5000 Hz, (b) the input SNR at each of these frequencies,
and (c) the overall noise reduction in dB.

\textbf{Solution:}

\begin{enumerate}
\def\labelenumi{(\alph{enumi})}
\tightlist
\item
  Wiener filter: H(f) = S\textsubscript{dd}(f) / (S\textsubscript{dd}(f)
  + S\textsubscript{nn}(f))
\end{enumerate}

At f = 0: S\textsubscript{dd}(0) = 10⁻³ V²/Hz H(0) = 10⁻³ / (10⁻³ + 5 ×
10⁻⁶) = 10⁻³ / 1.005 × 10⁻³ = \textbf{0.995}

At f = 500 Hz: S\textsubscript{dd}(500) = 10⁻³ / (1 + 1) = 5 × 10⁻⁴
V²/Hz H(500) = 5 × 10⁻⁴ / (5 × 10⁻⁴ + 5 × 10⁻⁶) = 5 × 10⁻⁴ / 5.05 × 10⁻⁴
= \textbf{0.990}

At f = 1000 Hz: S\textsubscript{dd}(1000) = 10⁻³ / (1 + 4) = 2 × 10⁻⁴
V²/Hz H(1000) = 2 × 10⁻⁴ / (2 × 10⁻⁴ + 5 × 10⁻⁶) = 2 × 10⁻⁴ / 2.05 ×
10⁻⁴ = \textbf{0.976}

At f = 5000 Hz: S\textsubscript{dd}(5000) = 10⁻³ / (1 + 100) = 9.9 ×
10⁻⁶ V²/Hz H(5000) = 9.9 × 10⁻⁶ / (9.9 × 10⁻⁶ + 5 × 10⁻⁶) = 9.9 × 10⁻⁶ /
14.9 × 10⁻⁶ = \textbf{0.664}

\begin{enumerate}
\def\labelenumi{(\alph{enumi})}
\setcounter{enumi}{1}
\tightlist
\item
  Input SNR at each frequency: SNR(0) = 10⁻³ / 5 × 10⁻⁶ = 200 =
  \textbf{23.0 dB} SNR(500) = 5 × 10⁻⁴ / 5 × 10⁻⁶ = 100 = \textbf{20.0
  dB} SNR(1000) = 2 × 10⁻⁴ / 5 × 10⁻⁶ = 40 = \textbf{16.0 dB} SNR(5000)
  = 9.9 × 10⁻⁶ / 5 × 10⁻⁶ = 1.98 = \textbf{3.0 dB}
\end{enumerate}

The Wiener filter passes frequencies with high SNR nearly unchanged and
attenuates frequencies with low SNR.

\begin{enumerate}
\def\labelenumi{(\alph{enumi})}
\setcounter{enumi}{2}
\tightlist
\item
  Total input noise power: P\textsubscript{n,in} = S\textsubscript{nn} ×
  BW = 5 × 10⁻⁶ × 5000 = 0.025 V²
\end{enumerate}

Total output noise power: P\textsubscript{n,out} = ∫₀⁵⁰⁰⁰ H²(f) ×
S\textsubscript{nn} df

Since H(f) varies, we approximate by integrating numerically. For the
Lorentzian signal spectrum, H(f) ≈ 1 for f \textless\textless{} 5000 Hz
and drops toward 0.664 near 5 kHz. The signal energy is concentrated
below 1 kHz, so H²(f) × S\textsubscript{nn} ≈ S\textsubscript{nn} for f
\textless{} 1 kHz and progressively attenuated beyond.

Approximate: P\textsubscript{n,out} ≈ S\textsubscript{nn} × {[}1000 ×
0.99² + 2000 × 0.95² + 2000 × 0.75²{]} = 5 × 10⁻⁶ × {[}980 + 1805 +
1125{]} = 5 × 10⁻⁶ × 3910 = 0.01955 V²

Noise reduction = 10 × log₁₀(0.025 / 0.01955) = 10 × log₁₀(1.279) =
\textbf{1.1 dB}

The modest noise reduction reflects that the signal spectrum is
broadband (500 Hz bandwidth) relative to the measurement bandwidth (5
kHz), leaving the Wiener filter unable to fully separate signal from
noise at most frequencies.

\begin{center}\rule{0.5\linewidth}{0.5pt}\end{center}

\section{Problem 8.7.9}\label{problem-8.7.9}

\textbf{Given:} A signal of length N = 512 is K = 10 sparse in the
wavelet domain. A random Gaussian measurement matrix of size M × 512 is
used for compressive sensing. The practical constant C = 3 is used for
the measurement bound.

\textbf{Find:} (a) The minimum number of measurements M, (b) the
compression ratio, (c) the percentage of measurements saved compared to
Nyquist, (d) the measurement matrix dimensions, and (e) the expected
number of OMP iterations for recovery.

\textbf{Solution:}

\begin{enumerate}
\def\labelenumi{(\alph{enumi})}
\tightlist
\item
  Minimum measurements: M ≥ C × K × log(N/K) = 3 × 10 × log(512/10) = 30
  × log(51.2) = 30 × 3.934
\end{enumerate}

Using natural logarithm: M ≥ 30 × 3.934 = 118.0

M = \textbf{118 measurements} (minimum)

\begin{enumerate}
\def\labelenumi{(\alph{enumi})}
\setcounter{enumi}{1}
\item
  Compression ratio: CR = N/M = 512/118 = \textbf{4.34:1}
\item
  Percentage saved: Savings = (1 - M/N) × 100 = (1 - 118/512) × 100 = (1
  - 0.2305) × 100 = \textbf{76.9\%}
\item
  Measurement matrix Φ is \textbf{118 × 512}. Each of the 118
  measurements is a random linear combination of all 512 signal values,
  with entries drawn from a Gaussian distribution N(0, 1/M).
\item
  OMP (Orthogonal Matching Pursuit) iteratively identifies one support
  element per iteration. Since the signal has K = 10 nonzero
  coefficients, OMP requires exactly \textbf{K = 10 iterations} to
  identify the full support. At each iteration, OMP:
\end{enumerate}

\begin{enumerate}
\def\labelenumi{\arabic{enumi}.}
\tightlist
\item
  Correlates the residual with all 512 columns of Φ (512 inner products)
\item
  Selects the column with maximum correlation
\item
  Solves a least-squares problem over the selected columns (growing from
  1 to 10 columns)
\end{enumerate}

Total computation: approximately 10 × 512 × 118 ≈ 604,160
multiply-accumulate operations, plus the growing least-squares
solutions.

\begin{center}\rule{0.5\linewidth}{0.5pt}\end{center}

\section{Problem 8.7.10}\label{problem-8.7.10}

\textbf{Given:} An MRI scan acquires a 256 × 256 image with sparsity K =
3,000 in the wavelet domain. The current full-scan acquisition takes 8
minutes. Compressive sensing is applied with a practical constant C = 4.

\textbf{Find:} (a) The total number of pixels (Nyquist samples), (b) the
minimum CS measurements required, (c) the scan time reduction factor,
(d) the new scan time, and (e) the trade-off considerations for choosing
the actual number of measurements.

\textbf{Solution:}

\begin{enumerate}
\def\labelenumi{(\alph{enumi})}
\item
  Total pixels (Nyquist): N = 256 × 256 = \textbf{65,536 samples}
\item
  Minimum CS measurements: M ≥ C × K × log(N/K) = 4 × 3,000 ×
  log(65,536/3,000) = 12,000 × log(21.85)
\end{enumerate}

log(21.85) = 3.085 (natural logarithm)

M ≥ 12,000 × 3.085 = \textbf{37,016 measurements}

\begin{enumerate}
\def\labelenumi{(\alph{enumi})}
\setcounter{enumi}{2}
\item
  Scan time reduction: Reduction factor = N/M = 65,536/37,016 =
  \textbf{1.77×}
\item
  New scan time: t\textsubscript{CS} = 8 / 1.77 = \textbf{4.52 minutes}
\end{enumerate}

This is a modest reduction because the sparsity ratio K/N = 3,000/65,536
= 4.6\% is not extremely sparse.

\begin{enumerate}
\def\labelenumi{(\alph{enumi})}
\setcounter{enumi}{4}
\tightlist
\item
  Trade-off considerations:
\end{enumerate}

\begin{itemize}
\tightlist
\item
  \textbf{Minimum M = 37,016}: fastest scan but highest reconstruction
  error, potential artifacts near edges and fine structures
\item
  \textbf{Practical M = 45,000--50,000}: 20--35\% safety margin, scan
  time ≈ 5.5--6.1 minutes, reliable reconstruction at moderate SNR
\item
  \textbf{M = 55,000}: only 16\% scan time reduction but near-Nyquist
  quality --- diminishing returns
\end{itemize}

For clinical MRI, the typical choice is M ≈ 1.5 × M\textsubscript{min} =
\textbf{55,500 measurements} (scan time ≈ 6.8 minutes, 1.2 minutes
saved), providing robust image quality. Research applications may accept
M closer to the theoretical minimum for faster scanning of dynamic
processes (cardiac MRI, functional MRI).

The actual k-space sampling pattern also matters: random undersampling
of k-space phase-encode lines (maintaining full readout) produces
incoherent aliasing artifacts that CS algorithms can remove, whereas
structured undersampling produces coherent artifacts that violate the
RIP and degrade reconstruction.

\chapter{Chapter 9 --- Section 9.1:
Electrostatics}\label{chapter-9-section-9.1-electrostatics}

Practice problems covering electric charge, Coulomb's law, electric
fields, potential, capacitance, dielectric materials, boundary
conditions, and Laplace's/Poisson's equations.

\begin{center}\rule{0.5\linewidth}{0.5pt}\end{center}

\section{Problem 9.1.1}\label{problem-9.1.1}

\textbf{Given:} Three point charges are arranged along the x-axis: q₁ =
+6 μC at x = 0, q₂ = -3 μC at x = 20 cm, and q₃ = +2 μC at x = 50 cm.
The Coulomb constant k = 8.99 × 10⁹ N·m²/C².

\textbf{Find:} The net electrostatic force on q₂ (magnitude and
direction).

\textbf{Solution:}

Force on q₂ due to q₁: F₂₁ = k\textbar q₁\textbar\textbar q₂\textbar{} /
r₁₂² = (8.99 × 10⁹)(6 × 10⁻⁶)(3 × 10⁻⁶) / (0.20)² = (8.99 × 10⁹)(18 ×
10⁻¹²) / 0.04 = 0.16182 / 0.04 = 4.046 N

Since q₁ is positive and q₂ is negative, the force is attractive,
pulling q₂ toward q₁ (in the -x direction).

Force on q₂ due to q₃: r₂₃ = 0.50 - 0.20 = 0.30 m F₂₃ =
k\textbar q₂\textbar\textbar q₃\textbar{} / r₂₃² = (8.99 × 10⁹)(3 ×
10⁻⁶)(2 × 10⁻⁶) / (0.30)² = (8.99 × 10⁹)(6 × 10⁻¹²) / 0.09 = 0.05394 /
0.09 = 0.5993 N

Since q₂ is negative and q₃ is positive, the force is attractive,
pulling q₂ toward q₃ (in the +x direction).

Net force on q₂: F\textsubscript{net} = -4.046 + 0.5993 = -3.447 N (in
the -x direction).

\textbf{The net force on q₂ is 3.45 N directed toward q₁ (in the -x
direction).}

\begin{center}\rule{0.5\linewidth}{0.5pt}\end{center}

\section{Problem 9.1.2}\label{problem-9.1.2}

\textbf{Given:} A uniformly charged thin spherical shell of radius R =
15 cm carries a total charge Q = +10 nC. The permittivity of free space
is ε₀ = 8.854 × 10⁻¹² F/m.

\textbf{Find:} (a) The electric field at r = 10 cm (inside the shell).
(b) The electric field at r = 25 cm (outside the shell). (c) The
electric potential at r = 25 cm and at r = 10 cm.

\textbf{Solution:}

\begin{enumerate}
\def\labelenumi{(\alph{enumi})}
\item
  By Gauss's Law, a spherical Gaussian surface at r = 10 cm \textless{}
  R encloses no charge. \textbf{E(r = 10 cm) = 0 V/m.}
\item
  At r = 25 cm \textgreater{} R, the shell appears as a point charge: E
  = kQ / r² = (8.99 × 10⁹)(10 × 10⁻⁹) / (0.25)² = 89.9 / 0.0625 =
  \textbf{1,438 V/m directed radially outward.}
\item
  Potential at r = 25 cm: V = kQ / r = (8.99 × 10⁹)(10 × 10⁻⁹) / 0.25 =
  89.9 / 0.25 = \textbf{359.6 V}
\end{enumerate}

Potential at r = 10 cm: Inside a conducting shell, the potential is
constant and equal to the potential at the surface. V = kQ / R = (8.99 ×
10⁹)(10 × 10⁻⁹) / 0.15 = 89.9 / 0.15 = \textbf{599.3 V}

\begin{center}\rule{0.5\linewidth}{0.5pt}\end{center}

\section{Problem 9.1.3}\label{problem-9.1.3}

\textbf{Given:} Two charges q₁ = +8 μC and q₂ = -5 μC are located at
positions (0, 0) and (0.6 m, 0), respectively. The Coulomb constant k =
8.99 × 10⁹ N·m²/C².

\textbf{Find:} The electric potential at point P located at (0.3 m, 0.4
m).

\textbf{Solution:}

Distance from q₁ to P: r₁ = √(0.3² + 0.4²) = √(0.09 + 0.16) = √0.25 =
0.50 m

Distance from q₂ to P: r₂ = √((0.3 - 0.6)² + 0.4²) = √(0.09 + 0.16) =
√0.25 = 0.50 m

V\textsubscript{P} = kq₁/r₁ + kq₂/r₂ = (8.99 × 10⁹)(8 × 10⁻⁶)/0.50 +
(8.99 × 10⁹)(-5 × 10⁻⁶)/0.50 V\textsubscript{P} = 143,840 + (-89,900) =
\textbf{53,940 V ≈ 53.9 kV}

\begin{center}\rule{0.5\linewidth}{0.5pt}\end{center}

\section{Problem 9.1.4}\label{problem-9.1.4}

\textbf{Given:} A parallel-plate capacitor with plate area A = 50 cm²
and plate separation d = 1 mm uses a ceramic dielectric with
ε\textsubscript{r} = 200. The capacitor is charged to 50 V and then
disconnected from the source.

\textbf{Find:} (a) The capacitance. (b) The stored energy. (c) The
charge on the plates. (d) If the dielectric is removed while the
capacitor remains disconnected, find the new voltage and stored energy.

\textbf{Solution:}

\begin{enumerate}
\def\labelenumi{(\alph{enumi})}
\item
  C = ε₀ε\textsubscript{r}A / d = (8.854 × 10⁻¹² × 200 × 50 × 10⁻⁴) / (1
  × 10⁻³) = (8.854 × 10⁻⁹) / 10⁻³ = \textbf{8.854 nF}
\item
  W = ½CV² = 0.5 × 8.854 × 10⁻⁹ × 50² = 0.5 × 8.854 × 10⁻⁹ × 2500 =
  \textbf{11.07 μJ}
\item
  Q = CV = 8.854 × 10⁻⁹ × 50 = \textbf{442.7 nC}
\item
  With the dielectric removed, Q remains constant (disconnected). New
  capacitance: C\textsubscript{new} = ε₀A / d = 8.854 × 10⁻¹² × 50 ×
  10⁻⁴ / 10⁻³ = 44.27 pF
\end{enumerate}

New voltage: V\textsubscript{new} = Q / C\textsubscript{new} = 442.7 ×
10⁻⁹ / 44.27 × 10⁻¹² = \textbf{10,000 V = 10 kV}

New energy: W\textsubscript{new} =
½C\textsubscript{new}V\textsubscript{new}² = 0.5 × 44.27 × 10⁻¹² ×
(10,000)² = \textbf{2.214 mJ}

The energy increased by a factor of 200 (= ε\textsubscript{r}). The work
to remove the dielectric against the attractive electric force was
converted to additional stored energy.

\begin{center}\rule{0.5\linewidth}{0.5pt}\end{center}

\section{Problem 9.1.5}\label{problem-9.1.5}

\textbf{Given:} A coaxial capacitor has inner radius a = 2 mm, outer
radius b = 6 mm, and length l = 10 cm. The space between the conductors
is filled with two concentric dielectric layers: the inner half (2 mm to
4 mm) has ε\textsubscript{r1} = 3.0, and the outer half (4 mm to 6 mm)
has ε\textsubscript{r2} = 6.0.

\textbf{Find:} The total capacitance of the structure.

\textbf{Solution:}

For a coaxial geometry, the two concentric dielectric layers act as two
capacitors in series (since the voltage adds radially).

C₁ = 2πε₀ε\textsubscript{r1}l / ln(r\textsubscript{mid}/a) = 2π(8.854 ×
10⁻¹²)(3.0)(0.10) / ln(4/2) = 2π × 2.656 × 10⁻¹² / 0.6931 = 16.69 ×
10⁻¹² / 0.6931 = 24.08 pF

C₂ = 2πε₀ε\textsubscript{r2}l / ln(b/r\textsubscript{mid}) = 2π(8.854 ×
10⁻¹²)(6.0)(0.10) / ln(6/4) = 2π × 5.313 × 10⁻¹² / 0.4055 = 33.38 ×
10⁻¹² / 0.4055 = 82.31 pF

Series combination: 1/C = 1/C₁ + 1/C₂ = 1/24.08 + 1/82.31 = 0.04153 +
0.01215 = 0.05368 pF⁻¹

\textbf{C = 18.63 pF}

\begin{center}\rule{0.5\linewidth}{0.5pt}\end{center}

\section{Problem 9.1.6}\label{problem-9.1.6}

\textbf{Given:} A parallel-plate capacitor has plate separation d = 4
mm. The bottom plate is grounded (V = 0) and the top plate is at V = 200
V. A 1 mm thick conductor (floating, unconnected) is inserted parallel
to the plates with its bottom surface 1.5 mm above the grounded plate.
The medium is air (ε\textsubscript{r} = 1) everywhere except within the
conductor.

\textbf{Find:} Using Laplace's equation, find the electric field in each
air gap and the surface charge densities on the conductor surfaces.

\textbf{Solution:}

The conductor is an equipotential body, so V is constant throughout it.
The air gaps are: lower gap d₁ = 1.5 mm, upper gap d₂ = 4 - 1.5 - 1 =
1.5 mm.

In each air region, ∇²V = 0 reduces to d²V/dz² = 0 (1D problem), so the
potential varies linearly.

Since both air gaps have the same thickness (1.5 mm each) and the same
dielectric (air), the electric field is the same in both gaps: E =
V\textsubscript{total} / (d₁ + d₂) = 200 / (1.5 × 10⁻³ + 1.5 × 10⁻³) =
200 / 3.0 × 10⁻³ = \textbf{66,667 V/m = 66.7 kV/m in both air gaps}

The conductor floats at V\textsubscript{cond} = E × d₁ = 66,667 × 1.5 ×
10⁻³ = 100 V (exactly midway, as expected from symmetry).

Surface charge density on the bottom surface of the conductor:
σ\textsubscript{bottom} = ε₀E = 8.854 × 10⁻¹² × 66,667 = \textbf{590.3
nC/m²} (positive, facing the grounded plate)

Surface charge density on the top surface: σ\textsubscript{top} = -ε₀E =
\textbf{-590.3 nC/m²} (negative, facing the +200 V plate)

The conductor redistributes the field but does not change its magnitude
in this symmetric geometry. The conducting slab reduces the effective
capacitor gap from 4 mm to 3 mm, increasing capacitance by a factor of
4/3.

\begin{center}\rule{0.5\linewidth}{0.5pt}\end{center}

\section{Problem 9.1.7}\label{problem-9.1.7}

\textbf{Given:} A point charge Q = +20 nC is located at the center of a
dielectric sphere of radius R = 5 cm with relative permittivity
ε\textsubscript{r} = 4.

\textbf{Find:} (a) The electric displacement D, electric field E, and
polarization P at r = 3 cm (inside the dielectric). (b) The electric
field just outside the sphere at r = 5.01 cm (in free space).

\textbf{Solution:}

\begin{enumerate}
\def\labelenumi{(\alph{enumi})}
\tightlist
\item
  By Gauss's Law for D (unaffected by dielectrics): D = Q / (4πr²) At r
  = 3 cm: D = 20 × 10⁻⁹ / (4π × (0.03)²) = 20 × 10⁻⁹ / (4π × 9 × 10⁻⁴) =
  20 × 10⁻⁹ / 1.131 × 10⁻² = \textbf{1.769 μC/m²}
\end{enumerate}

E = D / (ε₀ε\textsubscript{r}) = 1.769 × 10⁻⁶ / (8.854 × 10⁻¹² × 4) =
1.769 × 10⁻⁶ / 3.542 × 10⁻¹¹ = \textbf{49,950 V/m ≈ 50.0 kV/m}

P = D - ε₀E = D(1 - 1/ε\textsubscript{r}) = 1.769 × 10⁻⁶ × (1 - 0.25) =
\textbf{1.327 μC/m²}

\begin{enumerate}
\def\labelenumi{(\alph{enumi})}
\setcounter{enumi}{1}
\tightlist
\item
  Just outside the sphere (r = 5.01 cm, free space): D = Q / (4πr²) = 20
  × 10⁻⁹ / (4π × (0.0501)²) = 20 × 10⁻⁹ / 3.155 × 10⁻² = 0.6339 μC/m² E
  = D / ε₀ = 0.6339 × 10⁻⁶ / 8.854 × 10⁻¹² = \textbf{71,590 V/m ≈ 71.6
  kV/m}
\end{enumerate}

The field is 1.43× stronger just outside the sphere than at the same
radius inside, because the dielectric reduces the internal field by a
factor of ε\textsubscript{r}. At the boundary, the normal D is
continuous (no free surface charge), but E jumps by a factor of
ε\textsubscript{r}.

\begin{center}\rule{0.5\linewidth}{0.5pt}\end{center}

\section{Problem 9.1.8}\label{problem-9.1.8}

\textbf{Given:} Two capacitors, C₁ = 10 μF charged to 100 V and C₂ = 22
μF initially uncharged, are connected in parallel (positive to
positive).

\textbf{Find:} (a) The final voltage across the combination. (b) The
energy stored before and after connection. (c) The energy lost and
explain where it went.

\textbf{Solution:}

\begin{enumerate}
\def\labelenumi{(\alph{enumi})}
\item
  Total charge: Q = C₁V₁ + C₂V₂ = 10 × 10⁻⁶ × 100 + 0 = 1.0 mC Final
  voltage: V\textsubscript{f} = Q / (C₁ + C₂) = 1.0 × 10⁻³ / (32 × 10⁻⁶)
  = \textbf{31.25 V}
\item
  Energy before: W\textsubscript{before} = ½C₁V₁² = 0.5 × 10 × 10⁻⁶ ×
  100² = \textbf{50.0 mJ} Energy after: W\textsubscript{after} = ½(C₁ +
  C₂)V\textsubscript{f}² = 0.5 × 32 × 10⁻⁶ × 31.25² = 0.5 × 32 × 10⁻⁶ ×
  976.6 = \textbf{15.63 mJ}
\item
  Energy lost: ΔW = 50.0 - 15.63 = \textbf{34.37 mJ}
\end{enumerate}

This energy was dissipated as heat in the resistance of the connecting
wires and contact resistance during the transient equalization current.
Even with ideal (zero resistance) wires, the energy is radiated as an
electromagnetic pulse. Energy is always lost when capacitors at
different voltages are connected in parallel; the fraction lost is
C₁C₂(V₁ - V₂)² / {[}2(C₁ + C₂){]}.

\begin{center}\rule{0.5\linewidth}{0.5pt}\end{center}

\section{Problem 9.1.9}\label{problem-9.1.9}

\textbf{Given:} A parallel-plate capacitor has plates of area A = 200
cm² separated by d = 3 mm of air. A voltage of V = 1,000 V is applied.
The dielectric strength of air is 3 MV/m.

\textbf{Find:} (a) The capacitance. (b) The electric field between the
plates. (c) The surface charge density on the plates. (d) Whether the
air will break down. (e) The maximum voltage that can be applied without
breakdown.

\textbf{Solution:}

\begin{enumerate}
\def\labelenumi{(\alph{enumi})}
\item
  C = ε₀A / d = 8.854 × 10⁻¹² × 200 × 10⁻⁴ / 3 × 10⁻³ = 8.854 × 10⁻¹² ×
  6.667 = \textbf{59.0 pF}
\item
  E = V / d = 1,000 / 3 × 10⁻³ = \textbf{333,333 V/m = 333 kV/m}
\item
  σ = ε₀E = 8.854 × 10⁻¹² × 333,333 = \textbf{2.951 μC/m²}
\item
  The dielectric strength of air is 3 MV/m = 3,000 kV/m. Since E = 333
  kV/m \textless{} 3,000 kV/m, \textbf{the air will not break down.} The
  field is only 11\% of the breakdown value.
\item
  V\textsubscript{max} = E\textsubscript{breakdown} × d = 3 × 10⁶ × 3 ×
  10⁻³ = \textbf{9,000 V = 9 kV}
\end{enumerate}

\begin{center}\rule{0.5\linewidth}{0.5pt}\end{center}

\section{Problem 9.1.10}\label{problem-9.1.10}

\textbf{Given:} A spherical capacitor has inner radius a = 10 cm and
outer radius b = 15 cm. The space between the shells is filled with oil
having ε\textsubscript{r} = 2.5.

\textbf{Find:} (a) The capacitance. (b) If the capacitor stores 5 μJ,
find the voltage between the shells and the maximum electric field
(which occurs at r = a).

\textbf{Solution:}

\begin{enumerate}
\def\labelenumi{(\alph{enumi})}
\item
  C = 4πε₀ε\textsubscript{r}ab / (b - a) = 4π(8.854 ×
  10⁻¹²)(2.5)(0.10)(0.15) / (0.05) = 4π × 8.854 × 10⁻¹² × 2.5 × 0.015 /
  0.05 = 4π × 3.320 × 10⁻¹³ / 0.05 = 4.175 × 10⁻¹² / 0.05 = \textbf{83.5
  pF}
\item
  From W = ½CV²: V = √(2W/C) = √(2 × 5 × 10⁻⁶ / 83.5 × 10⁻¹²) = √(1.198
  × 10⁵) = \textbf{346.1 V}
\end{enumerate}

Maximum electric field at r = a: E\textsubscript{max} = V × b / {[}a(b -
a){]} = 346.1 × 0.15 / {[}0.10 × 0.05{]} = 51.92 / 0.005 =
\textbf{10,383 V/m ≈ 10.4 kV/m}

Alternatively: E\textsubscript{max} = Q / (4πε₀ε\textsubscript{r}a²)
where Q = CV = 83.5 × 10⁻¹² × 346.1 = 28.9 nC. E\textsubscript{max} =
28.9 × 10⁻⁹ / (4π × 8.854 × 10⁻¹² × 2.5 × 0.01) = 28.9 × 10⁻⁹ / 2.78 ×
10⁻¹² = \textbf{10,396 V/m}, confirming the result.

\chapter{Chapter 9 --- Section 9.2:
Magnetostatics}\label{chapter-9-section-9.2-magnetostatics}

Practice problems covering magnetic fields, Ampere's law, magnetic
force, inductance, magnetic materials, hysteresis, magnetic circuits,
and eddy currents.

\begin{center}\rule{0.5\linewidth}{0.5pt}\end{center}

\section{Problem 9.2.1}\label{problem-9.2.1}

\textbf{Given:} Two long, parallel conductors are separated by 8 cm.
Conductor A carries 20 A and conductor B carries 30 A, both in the same
direction. The permeability of free space is μ₀ = 4π × 10⁻⁷ H/m.

\textbf{Find:} (a) The magnetic field at the midpoint between the
conductors. (b) The force per unit length between the conductors and
whether it is attractive or repulsive.

\textbf{Solution:}

\begin{enumerate}
\def\labelenumi{(\alph{enumi})}
\tightlist
\item
  At the midpoint (r = 0.04 m from each conductor): B\textsubscript{A} =
  μ₀I\textsubscript{A} / (2πr) = (4π × 10⁻⁷ × 20) / (2π × 0.04) = (8π ×
  10⁻⁶) / (0.08π) = 100 μT B\textsubscript{B} = μ₀I\textsubscript{B} /
  (2πr) = (4π × 10⁻⁷ × 30) / (2π × 0.04) = (12π × 10⁻⁶) / (0.08π) = 150
  μT
\end{enumerate}

By the right-hand rule, at the midpoint, B\textsubscript{A} and
B\textsubscript{B} point in opposite directions (each field circles its
conductor, and at the midpoint between same-direction currents, the
fields oppose). B\textsubscript{net} = \textbar B\textsubscript{B} -
B\textsubscript{A}\textbar{} = 150 - 100 = \textbf{50 μT}

\begin{enumerate}
\def\labelenumi{(\alph{enumi})}
\setcounter{enumi}{1}
\tightlist
\item
  Force per unit length: F/l = μ₀I\textsubscript{A}I\textsubscript{B} /
  (2πd) = (4π × 10⁻⁷ × 20 × 30) / (2π × 0.08) = (4π × 10⁻⁷ × 600) /
  (0.16π) = (2400π × 10⁻⁷) / (0.16π) = 2400 × 10⁻⁷ / 0.16 = \textbf{1.5
  × 10⁻³ N/m = 1.5 mN/m}
\end{enumerate}

Since the currents flow in the same direction, the force is
\textbf{attractive}.

\begin{center}\rule{0.5\linewidth}{0.5pt}\end{center}

\section{Problem 9.2.2}\label{problem-9.2.2}

\textbf{Given:} A toroidal coil has N = 400 turns, a mean radius of 12
cm (mean path length l = 2π × 0.12 m), and a cross-sectional area A = 6
cm². The core material has relative permeability μ\textsubscript{r} =
500. The current is I = 1.5 A.

\textbf{Find:} (a) The magnetic field intensity H. (b) The magnetic flux
density B. (c) The total magnetic flux. (d) The inductance of the
toroid.

\textbf{Solution:}

Mean path length: l = 2π × 0.12 = 0.7540 m

\begin{enumerate}
\def\labelenumi{(\alph{enumi})}
\item
  H = NI / l = 400 × 1.5 / 0.7540 = \textbf{795.8 A/m}
\item
  B = μ₀μ\textsubscript{r}H = 4π × 10⁻⁷ × 500 × 795.8 = 6.283 × 10⁻⁴ ×
  795.8 = \textbf{0.500 T}
\item
  Φ = BA = 0.500 × 6 × 10⁻⁴ = \textbf{3.0 × 10⁻⁴ Wb = 0.30 mWb}
\item
  L = NΦ / I = 400 × 3.0 × 10⁻⁴ / 1.5 = 0.12 / 1.5 = \textbf{80.0 mH}
\end{enumerate}

Alternatively: L = μ₀μ\textsubscript{r}N²A / l = 4π × 10⁻⁷ × 500 × 400²
× 6 × 10⁻⁴ / 0.7540 = 6.283 × 10⁻⁴ × 160,000 × 6 × 10⁻⁴ / 0.7540 =
0.06032 / 0.7540 = 80.0 mH, confirming the result.

\begin{center}\rule{0.5\linewidth}{0.5pt}\end{center}

\section{Problem 9.2.3}\label{problem-9.2.3}

\textbf{Given:} A rectangular current loop (dimensions 15 cm × 10 cm)
carries a current of 5 A and is placed in a uniform magnetic field B =
0.4 T. The plane of the loop makes an angle of 60° with the magnetic
field direction.

\textbf{Find:} (a) The magnetic dipole moment. (b) The torque on the
loop.

\textbf{Solution:}

\begin{enumerate}
\def\labelenumi{(\alph{enumi})}
\item
  m = NIA = 1 × 5 × (0.15 × 0.10) = 5 × 0.015 = \textbf{0.075 A·m²}
\item
  The torque is τ = mB sin θ, where θ is the angle between the magnetic
  moment (normal to the loop plane) and the field. If the loop plane
  makes 60° with B, then the normal to the plane makes 90° - 60° = 30°
  with B.
\end{enumerate}

τ = mB sin(30°) = 0.075 × 0.4 × 0.5 = \textbf{0.015 N·m = 15 mN·m}

\begin{center}\rule{0.5\linewidth}{0.5pt}\end{center}

\section{Problem 9.2.4}\label{problem-9.2.4}

\textbf{Given:} Two coils wound on a common ferrite core have N₁ = 200
turns and N₂ = 50 turns. The self-inductances are L₁ = 40 mH and L₂ =
2.5 mH. The coupling coefficient k = 0.95. The current in coil 1 changes
at a rate of dI₁/dt = 500 A/s.

\textbf{Find:} (a) The mutual inductance M. (b) The voltage induced in
coil 2. (c) The energy stored in the system when I₁ = 3 A and I₂ = 0 A.

\textbf{Solution:}

\begin{enumerate}
\def\labelenumi{(\alph{enumi})}
\item
  M = k√(L₁L₂) = 0.95 × √(40 × 10⁻³ × 2.5 × 10⁻³) = 0.95 × √(100 × 10⁻⁶)
  = 0.95 × 10 × 10⁻³ = \textbf{9.5 mH}
\item
  V₂ = M × dI₁/dt = 9.5 × 10⁻³ × 500 = \textbf{4.75 V}
\item
  W = ½L₁I₁² + ½L₂I₂² + MI₁I₂ = ½ × 40 × 10⁻³ × 9 + 0 + 0 = \textbf{180
  mJ}
\end{enumerate}

\begin{center}\rule{0.5\linewidth}{0.5pt}\end{center}

\section{Problem 9.2.5}\label{problem-9.2.5}

\textbf{Given:} A ferrite E-core inductor for a flyback converter has a
mean magnetic path length l\textsubscript{e} = 10 cm, effective
cross-sectional area A\textsubscript{e} = 2 cm², and relative
permeability μ\textsubscript{r} = 2,000. The core saturates at
B\textsubscript{sat} = 0.38 T. The winding has N = 30 turns.

\textbf{Find:} (a) The core reluctance. (b) The inductance without an
air gap. (c) The maximum current before saturation. (d) If a 0.5 mm air
gap is introduced, find the new inductance and the new saturation
current.

\textbf{Solution:}

\begin{enumerate}
\def\labelenumi{(\alph{enumi})}
\item
  ℛ\textsubscript{core} = l\textsubscript{e} /
  (μ₀μ\textsubscript{r}A\textsubscript{e}) = 0.10 / (4π × 10⁻⁷ × 2000 ×
  2 × 10⁻⁴) = 0.10 / (5.027 × 10⁻⁷) = \textbf{1.989 × 10⁵ A-turns/Wb}
\item
  L = N² / ℛ = 30² / 1.989 × 10⁵ = 900 / 1.989 × 10⁵ = \textbf{4.525 mH}
\item
  B\textsubscript{max} = μ₀μ\textsubscript{r}NI / l\textsubscript{e}, so
  I\textsubscript{max} = B\textsubscript{sat}l\textsubscript{e} /
  (μ₀μ\textsubscript{r}N) = 0.38 × 0.10 / (4π × 10⁻⁷ × 2000 × 30) =
  0.038 / 0.07540 = \textbf{0.504 A}
\item
  Air gap reluctance: ℛ\textsubscript{gap} = l\textsubscript{g} /
  (μ₀A\textsubscript{e}) = 0.5 × 10⁻³ / (4π × 10⁻⁷ × 2 × 10⁻⁴) = 5 ×
  10⁻⁴ / 2.513 × 10⁻¹⁰ = \textbf{1.989 × 10⁶ A-turns/Wb}
\end{enumerate}

Total reluctance: ℛ\textsubscript{total} = ℛ\textsubscript{core} +
ℛ\textsubscript{gap} = 1.989 × 10⁵ + 1.989 × 10⁶ = 2.188 × 10⁶
A-turns/Wb

New inductance: L = N² / ℛ\textsubscript{total} = 900 / 2.188 × 10⁶ =
\textbf{411 μH}

New saturation current: The gap dominates. B\textsubscript{max} = LI /
(NA\textsubscript{e}), so I\textsubscript{sat} =
B\textsubscript{sat}NA\textsubscript{e} / L = 0.38 × 30 × 2 × 10⁻⁴ / 411
× 10⁻⁶ = 2.28 × 10⁻³ / 4.11 × 10⁻⁴ = \textbf{5.55 A}

The air gap reduced the inductance by a factor of 11 but increased the
saturation current by a factor of 11, which is exactly the purpose of
gapping inductor cores.

\begin{center}\rule{0.5\linewidth}{0.5pt}\end{center}

\section{Problem 9.2.6}\label{problem-9.2.6}

\textbf{Given:} A transformer core has two paths: a center leg (length
10 cm, area 4 cm², μ\textsubscript{r} = 3,000) and two outer legs (each
20 cm, area 4 cm², μ\textsubscript{r} = 3,000). The center leg has an N
= 100-turn coil carrying 2 A. The two outer legs are in parallel.

\textbf{Find:} (a) The total reluctance of the magnetic circuit. (b) The
flux in the center leg and in each outer leg.

\textbf{Solution:}

\begin{enumerate}
\def\labelenumi{(\alph{enumi})}
\tightlist
\item
  Center leg reluctance: ℛ\textsubscript{center} = 0.10 / (4π × 10⁻⁷ ×
  3000 × 4 × 10⁻⁴) = 0.10 / (1.508 × 10⁻⁶) = 6.631 × 10⁴ A-turns/Wb
\end{enumerate}

Each outer leg reluctance: ℛ\textsubscript{outer} = 0.20 / (4π × 10⁻⁷ ×
3000 × 4 × 10⁻⁴) = 0.20 / 1.508 × 10⁻⁶ = 1.326 × 10⁵ A-turns/Wb

The two outer legs are in parallel: ℛ\textsubscript{parallel} =
ℛ\textsubscript{outer} / 2 = 1.326 × 10⁵ / 2 = 6.631 × 10⁴ A-turns/Wb

Total reluctance: ℛ\textsubscript{total} = ℛ\textsubscript{center} +
ℛ\textsubscript{parallel} = 6.631 × 10⁴ + 6.631 × 10⁴ = \textbf{1.326 ×
10⁵ A-turns/Wb}

\begin{enumerate}
\def\labelenumi{(\alph{enumi})}
\setcounter{enumi}{1}
\tightlist
\item
  MMF = NI = 100 × 2 = 200 A-turns. Φ\textsubscript{center} = ℱ /
  ℛ\textsubscript{total} = 200 / 1.326 × 10⁵ = \textbf{1.508 mWb}
\end{enumerate}

By symmetry, the flux splits equally between the two outer legs:
Φ\textsubscript{each outer} = Φ\textsubscript{center} / 2 =
\textbf{0.754 mWb}

\begin{center}\rule{0.5\linewidth}{0.5pt}\end{center}

\section{Problem 9.2.7}\label{problem-9.2.7}

\textbf{Given:} An induction heating system operates at f = 50 kHz to
heat a steel cylindrical workpiece (ρ = 1.2 × 10⁻⁶ Ω·m,
μ\textsubscript{r} = 200, density = 7,800 kg/m³, specific heat = 460
J/(kg·°C)). The workpiece has diameter 25 mm and length 100 mm. The
system delivers 5 kW to the workpiece.

\textbf{Find:} (a) The skin depth. (b) The time to raise the surface
temperature by 500°C (assuming uniform heating for a simplified
estimate). (c) Whether through-heating or surface-only heating occurs.

\textbf{Solution:}

\begin{enumerate}
\def\labelenumi{(\alph{enumi})}
\item
  δ = √(2ρ / (ωμ)) = √(2 × 1.2 × 10⁻⁶ / (2π × 50,000 × 200 × 4π × 10⁻⁷))
  = √(2.4 × 10⁻⁶ / (2π × 5 × 10⁴ × 2.513 × 10⁻⁴)) = √(2.4 × 10⁻⁶ /
  78.96) = √(3.039 × 10⁻⁸) = \textbf{0.174 mm}
\item
  Workpiece volume: V = π(0.0125)² × 0.10 = π × 1.5625 × 10⁻⁴ × 0.10 =
  4.909 × 10⁻⁵ m³ Mass: m = 7800 × 4.909 × 10⁻⁵ = 0.3829 kg Energy for
  ΔT = 500°C: Q = mcΔT = 0.3829 × 460 × 500 = 88,067 J Time: t = Q / P =
  88,067 / 5,000 = \textbf{17.6 seconds}
\item
  The skin depth is 0.174 mm while the workpiece radius is 12.5 mm. The
  ratio radius/δ = 12.5/0.174 = 71.8. Since the skin depth is only 1.4\%
  of the radius, \textbf{this is surface-only heating}. The surface
  heats first and heat conducts inward. At 50 kHz, this setup is
  suitable for surface hardening (case hardening) of steel, not
  through-heating. For through-heating, a much lower frequency (1-3 kHz,
  giving δ of several mm) would be needed.
\end{enumerate}

\begin{center}\rule{0.5\linewidth}{0.5pt}\end{center}

\section{Problem 9.2.8}\label{problem-9.2.8}

\textbf{Given:} A Helmholtz coil pair consists of two identical circular
coils of radius R = 20 cm, each with N = 50 turns, separated by a
distance equal to R (20 cm). The current in each coil is I = 4 A,
flowing in the same direction.

\textbf{Find:} The magnetic field at the midpoint between the coils.

\textbf{Solution:}

The field at the center of a single coil on its axis at distance x is: B
= μ₀NIR² / {[}2(R² + x²)\textsuperscript{3/2}{]}

At the midpoint, each coil is at x = R/2 from the center point.
B\textsubscript{each} = μ₀NIR² / {[}2(R² + R²/4)\textsuperscript{3/2}{]}
= μ₀NIR² / {[}2(5R²/4)\textsuperscript{3/2}{]}

(5R²/4)\textsuperscript{3/2} = (5/4)\textsuperscript{3/2} × R³ =
(1.25)\textsuperscript{1.5} × R³ = 1.3975 × R³

B\textsubscript{each} = μ₀NI / (2 × 1.3975 × R) = μ₀NI / (2.795R)

Total field (both coils contribute equally): B = 2 ×
B\textsubscript{each} = μ₀NI / (1.3975R)

B = (4π × 10⁻⁷ × 50 × 4) / (1.3975 × 0.20) = (2.513 × 10⁻⁴) / 0.2795 =
\textbf{8.99 × 10⁻⁴ T ≈ 0.899 mT}

The Helmholtz configuration produces a highly uniform field in the
region between the coils, with the first and second spatial derivatives
of B vanishing at the midpoint.

\chapter{Chapter 9 --- Section 9.3: Maxwell's
Equations}\label{chapter-9-section-9.3-maxwells-equations}

Practice problems covering Gauss's law for electricity, Gauss's law for
magnetism, Faraday's law, the Ampere-Maxwell law, electromagnetic
boundary conditions, and electromagnetic potentials.

\begin{center}\rule{0.5\linewidth}{0.5pt}\end{center}

\section{Problem 9.3.1}\label{problem-9.3.1}

\textbf{Given:} A solid insulating sphere of radius R = 10 cm carries a
uniformly distributed total charge of Q = +50 nC. The permittivity of
free space is ε₀ = 8.854 × 10⁻¹² F/m.

\textbf{Find:} Using Gauss's Law, find the electric field at (a) r = 5
cm (inside) and (b) r = 30 cm (outside). (c) Find the radius at which E
is maximum.

\textbf{Solution:}

\begin{enumerate}
\def\labelenumi{(\alph{enumi})}
\tightlist
\item
  Inside (r = 5 cm): The enclosed charge is Q\textsubscript{enc} =
  Q(r/R)³ = 50 × 10⁻⁹ × (0.05/0.10)³ = 50 × 10⁻⁹ × 0.125 = 6.25 nC. By
  Gauss's Law: E × 4πr² = Q\textsubscript{enc} / ε₀ E =
  Q\textsubscript{enc} / (4πε₀r²) = (8.99 × 10⁹)(6.25 × 10⁻⁹) / (0.05)²
  = 56.19 / 0.0025 = \textbf{22,475 V/m ≈ 22.5 kV/m}
\end{enumerate}

Alternatively: E = Qr / (4πε₀R³) = (8.99 × 10⁹)(50 × 10⁻⁹)(0.05) /
(0.10)³ = 22.475 / 0.001 = 22,475 V/m, confirming the result.

\begin{enumerate}
\def\labelenumi{(\alph{enumi})}
\setcounter{enumi}{1}
\item
  Outside (r = 30 cm): E = kQ / r² = (8.99 × 10⁹)(50 × 10⁻⁹) / (0.30)² =
  449.5 / 0.09 = \textbf{4,994 V/m ≈ 5.0 kV/m}
\item
  Inside the sphere, E increases linearly with r: E = Qr/(4πε₀R³).
  Outside, E decreases as 1/r². \textbf{E is maximum at the surface (r =
  R):} E\textsubscript{max} = kQ / R² = (8.99 × 10⁹)(50 × 10⁻⁹) / 0.01 =
  \textbf{44,950 V/m ≈ 45.0 kV/m}
\end{enumerate}

\begin{center}\rule{0.5\linewidth}{0.5pt}\end{center}

\section{Problem 9.3.2}\label{problem-9.3.2}

\textbf{Given:} A closed cylindrical surface (radius 5 cm, height 10 cm)
is placed in a non-uniform magnetic field. The flux entering through the
top circular face is Φ\textsubscript{top} = 2.5 mWb (inward). The flux
through the curved side is Φ\textsubscript{side} = -1.8 mWb (outward).

\textbf{Find:} The magnetic flux through the bottom circular face
(magnitude and direction).

\textbf{Solution:}

By Gauss's Law for magnetism, the total flux through any closed surface
is zero: Φ\textsubscript{top} + Φ\textsubscript{side} +
Φ\textsubscript{bottom} = 0

Taking inward as positive: 2.5 + (-1.8) + Φ\textsubscript{bottom} = 0
Φ\textsubscript{bottom} = -0.7 mWb

\textbf{The flux through the bottom face is 0.7 mWb directed outward
(leaving the surface).}

The total outward flux (1.8 + 0.7 = 2.5 mWb) equals the inward flux (2.5
mWb), as required by the absence of magnetic monopoles.

\begin{center}\rule{0.5\linewidth}{0.5pt}\end{center}

\section{Problem 9.3.3}\label{problem-9.3.3}

\textbf{Given:} A square loop of wire (side length 25 cm, resistance R =
5 Ω) lies in a uniform magnetic field. The field is perpendicular to the
loop and varies as B(t) = 0.8 - 0.3t (in tesla), where t is in seconds.

\textbf{Find:} (a) The induced EMF. (b) The induced current. (c) The
power dissipated in the loop at t = 1 s.

\textbf{Solution:}

\begin{enumerate}
\def\labelenumi{(\alph{enumi})}
\tightlist
\item
  Area: A = (0.25)² = 0.0625 m² Φ = BA = (0.8 - 0.3t) × 0.0625
\end{enumerate}

EMF = -dΦ/dt = -A × dB/dt = -0.0625 × (-0.3) = \textbf{0.01875 V = 18.75
mV}

The EMF is constant because dB/dt = -0.3 T/s is constant.

\begin{enumerate}
\def\labelenumi{(\alph{enumi})}
\setcounter{enumi}{1}
\tightlist
\item
  I = EMF / R = 0.01875 / 5 = \textbf{3.75 mA}
\end{enumerate}

By Lenz's law, the current opposes the decrease in flux, so it flows in
a direction that creates a field in the same direction as the original
field.

\begin{enumerate}
\def\labelenumi{(\alph{enumi})}
\setcounter{enumi}{2}
\tightlist
\item
  P = I²R = (3.75 × 10⁻³)² × 5 = 1.406 × 10⁻⁵ × 5 = \textbf{70.3 μW}
\end{enumerate}

This is constant and independent of time since the rate of change of B
is constant.

\begin{center}\rule{0.5\linewidth}{0.5pt}\end{center}

\section{Problem 9.3.4}\label{problem-9.3.4}

\textbf{Given:} A parallel-plate capacitor with circular plates of
radius 10 cm and gap d = 2 mm is being charged. The electric field
between the plates is increasing at a rate of dE/dt = 2 × 10¹³ V/(m·s).

\textbf{Find:} (a) The displacement current between the plates. (b) The
magnetic field at the edge of the plates (r = 10 cm) due to the
displacement current. (c) The conduction current flowing in the external
circuit.

\textbf{Solution:}

\begin{enumerate}
\def\labelenumi{(\alph{enumi})}
\item
  Plate area: A = π(0.10)² = 0.03142 m² I\textsubscript{d} = ε₀A(dE/dt)
  = 8.854 × 10⁻¹² × 0.03142 × 2 × 10¹³ = 8.854 × 10⁻¹² × 6.283 × 10¹¹ =
  \textbf{5.564 A}
\item
  Using Ampere-Maxwell law with a circular Amperian loop of radius r =
  10 cm around the axis: ∮B·dl = μ₀I\textsubscript{d(enclosed)} B × 2πr
  = μ₀I\textsubscript{d} (the full displacement current is enclosed at r
  = plate radius) B = μ₀I\textsubscript{d} / (2πr) = (4π × 10⁻⁷ × 5.564)
  / (2π × 0.10) = 6.991 × 10⁻⁶ / 0.6283 = \textbf{11.13 μT}
\item
  By Kirchhoff's current law (continuity of current), the conduction
  current in the external circuit equals the displacement current
  between the plates: \textbf{I\textsubscript{c} = 5.564 A}
\end{enumerate}

\begin{center}\rule{0.5\linewidth}{0.5pt}\end{center}

\section{Problem 9.3.5}\label{problem-9.3.5}

\textbf{Given:} An electromagnetic wave in free space (μ₁ = μ₀, ε₁ = ε₀)
strikes a flat glass surface (μ₂ = μ₀, ε₂ = 4ε₀, non-magnetic, lossless)
at normal incidence. The incident electric field amplitude is
E\textsubscript{i} = 50 V/m.

\textbf{Find:} (a) The intrinsic impedances of both media. (b) The
reflection and transmission coefficients. (c) The reflected and
transmitted electric field amplitudes. (d) The fraction of incident
power transmitted.

\textbf{Solution:}

\begin{enumerate}
\def\labelenumi{(\alph{enumi})}
\item
  η₁ = √(μ₀/ε₀) = 377 Ω η₂ = √(μ₀/4ε₀) = 377/√4 = 377/2 = \textbf{188.5
  Ω}
\item
  Γ = (η₂ - η₁)/(η₂ + η₁) = (188.5 - 377)/(188.5 + 377) = -188.5/565.5 =
  \textbf{-0.3333} τ = 2η₂/(η₁ + η₂) = 2 × 188.5/565.5 = 377/565.5 =
  \textbf{0.6667}
\item
  E\textsubscript{r} = ΓE\textsubscript{i} = -0.3333 × 50 =
  \textbf{-16.67 V/m} (180° phase reversal) E\textsubscript{t} =
  τE\textsubscript{i} = 0.6667 × 50 = \textbf{33.33 V/m}
\item
  Power transmitted: 1 - \textbar Γ\textbar² = 1 - (1/3)² = 1 - 1/9 =
  \textbf{8/9 ≈ 88.9\%}
\end{enumerate}

Verification: (η₁/η₂)\textbar τ\textbar² = (377/188.5)(0.6667)² = 2 ×
0.4444 = 0.889 = 88.9\%.

\begin{center}\rule{0.5\linewidth}{0.5pt}\end{center}

\section{Problem 9.3.6}\label{problem-9.3.6}

\textbf{Given:} A generator coil has 200 turns and area A = 0.05 m². It
rotates at 3,600 RPM in a uniform magnetic field B = 1.2 T. The axis of
rotation is perpendicular to the field.

\textbf{Find:} (a) The angular frequency of rotation. (b) The peak EMF.
(c) The RMS EMF. (d) The frequency of the generated voltage.

\textbf{Solution:}

\begin{enumerate}
\def\labelenumi{(\alph{enumi})}
\item
  ω = 2π × (RPM/60) = 2π × (3600/60) = 2π × 60 = \textbf{376.99 rad/s}
\item
  The flux is Φ = NBAcos(ωt). The EMF = -dΦ/dt = NBAω sin(ωt).
  EMF\textsubscript{peak} = NBAω = 200 × 1.2 × 0.05 × 377.0 =
  \textbf{4,523 V}
\item
  EMF\textsubscript{rms} = EMF\textsubscript{peak} / √2 = 4,523 / 1.414
  = \textbf{3,199 V ≈ 3.2 kV}
\item
  f = RPM / 60 = 3600 / 60 = \textbf{60 Hz}
\end{enumerate}

\begin{center}\rule{0.5\linewidth}{0.5pt}\end{center}

\section{Problem 9.3.7}\label{problem-9.3.7}

\textbf{Given:} A Hertzian dipole (short current element) of length dl =
0.02λ carries a peak current I₀ = 5 A at frequency f = 1 GHz.

\textbf{Find:} (a) The radiation resistance. (b) The total radiated
power. (c) The directivity in dBi.

\textbf{Solution:}

\begin{enumerate}
\def\labelenumi{(\alph{enumi})}
\item
  R\textsubscript{rad} = 80π²(dl/λ)² = 80 × 9.8696 × (0.02)² = 80 ×
  9.8696 × 4 × 10⁻⁴ = \textbf{0.3158 Ω}
\item
  P\textsubscript{rad} = ½I₀²R\textsubscript{rad} = 0.5 × 25 × 0.3158 =
  \textbf{3.95 W}
\item
  The directivity of a Hertzian dipole is always D = 1.5 (3/2),
  regardless of length. \textbf{D = 10 log₁₀(1.5) = 1.76 dBi}
\end{enumerate}

\begin{center}\rule{0.5\linewidth}{0.5pt}\end{center}

\section{Problem 9.3.8}\label{problem-9.3.8}

\textbf{Given:} A transformer has a primary coil (N₁ = 500 turns) and a
secondary coil (N₂ = 100 turns) wound on a common iron core. The primary
is connected to a 120 V\textsubscript{rms}, 60 Hz source. The peak
magnetic flux in the core is 0.5 mWb.

\textbf{Find:} (a) The secondary voltage. (b) Whether the core flux is
consistent with the applied primary voltage (using Faraday's law). (c)
The magnetizing current if the core reluctance is ℛ = 5 × 10⁵
A-turns/Wb.

\textbf{Solution:}

\begin{enumerate}
\def\labelenumi{(\alph{enumi})}
\item
  V₂ = V₁ × N₂/N₁ = 120 × 100/500 = \textbf{24 V\textsubscript{rms}}
\item
  By Faraday's law: V₁(rms) = 4.44 × f × N₁ × Φ\textsubscript{max} =
  4.44 × 60 × 500 × 0.5 × 10⁻³ = 4.44 × 60 × 0.25 = \textbf{66.6 V}
\end{enumerate}

This is less than 120 V, which means the actual peak flux must be
higher: Φ\textsubscript{max} = V₁ / (4.44 × f × N₁) = 120 / (4.44 × 60 ×
500) = 120 / 133,200 = \textbf{0.9009 mWb}

The stated 0.5 mWb is inconsistent; the actual peak flux for 120 V at 60
Hz with 500 turns is 0.90 mWb.

\begin{enumerate}
\def\labelenumi{(\alph{enumi})}
\setcounter{enumi}{2}
\tightlist
\item
  Peak MMF = ℛ × Φ\textsubscript{max} = 5 × 10⁵ × 0.9009 × 10⁻³ = 450.5
  A-turns Peak magnetizing current: I\textsubscript{m(peak)} = MMF / N₁
  = 450.5 / 500 = 0.901 A RMS magnetizing current:
  I\textsubscript{m(rms)} = 0.901 / √2 = \textbf{0.637 A}
\end{enumerate}

\chapter{Chapter 9 --- Section 9.4: Electromagnetic
Waves}\label{chapter-9-section-9.4-electromagnetic-waves}

Practice problems covering the wave equation, the electromagnetic
spectrum, polarization, skin effect, wave reflection at interfaces, the
Poynting vector, and wave propagation in lossy media.

\begin{center}\rule{0.5\linewidth}{0.5pt}\end{center}

\section{Problem 9.4.1}\label{problem-9.4.1}

\textbf{Given:} An electromagnetic wave propagates through a lossless,
non-magnetic dielectric medium with relative permittivity
ε\textsubscript{r} = 9.0 and μ\textsubscript{r} = 1.

\textbf{Find:} (a) The wave velocity. (b) The intrinsic impedance. (c)
The wavelength and period at 2.4 GHz. (d) If the electric field
amplitude is E₀ = 20 V/m, find the magnetic field amplitude.

\textbf{Solution:}

\begin{enumerate}
\def\labelenumi{(\alph{enumi})}
\item
  v = c / √(ε\textsubscript{r}) = 3 × 10⁸ / √9 = 3 × 10⁸ / 3 =
  \textbf{1.0 × 10⁸ m/s}
\item
  η = η₀ / √(ε\textsubscript{r}) = 377 / 3 = \textbf{125.7 Ω}
\item
  λ = v / f = 1.0 × 10⁸ / 2.4 × 10⁹ = \textbf{0.04167 m = 41.7 mm} T = 1
  / f = 1 / 2.4 × 10⁹ = \textbf{4.167 × 10⁻¹⁰ s = 0.417 ns}
\item
  H₀ = E₀ / η = 20 / 125.7 = \textbf{0.159 A/m = 159 mA/m}
\end{enumerate}

\begin{center}\rule{0.5\linewidth}{0.5pt}\end{center}

\section{Problem 9.4.2}\label{problem-9.4.2}

\textbf{Given:} A 77 GHz automotive radar system (FMCW radar for
adaptive cruise control) operates in the millimeter-wave band.

\textbf{Find:} (a) The free-space wavelength. (b) The region of the
electromagnetic spectrum. (c) If the radar antenna is a circular
aperture of diameter D = 25 mm, estimate the far-field boundary
distance.

\textbf{Solution:}

\begin{enumerate}
\def\labelenumi{(\alph{enumi})}
\item
  λ = c / f = 3 × 10⁸ / 77 × 10⁹ = \textbf{3.896 × 10⁻³ m = 3.9 mm}
\item
  This falls in the \textbf{millimeter-wave band} (30-300 GHz,
  wavelengths 1-10 mm), at the boundary of EHF (Extremely High
  Frequency) in the radio spectrum.
\item
  Far-field distance: r\textsubscript{ff} = 2D² / λ = 2 × (0.025)² /
  3.896 × 10⁻³ = 2 × 6.25 × 10⁻⁴ / 3.896 × 10⁻³ = 1.25 × 10⁻³ / 3.896 ×
  10⁻³ = \textbf{0.321 m ≈ 32 cm}
\end{enumerate}

Since automotive targets are at distances \textgreater\textgreater{} 32
cm, the far-field assumption is always valid for this radar.

\begin{center}\rule{0.5\linewidth}{0.5pt}\end{center}

\section{Problem 9.4.3}\label{problem-9.4.3}

\textbf{Given:} A satellite downlink transmitter sends a right-hand
circularly polarized (RHCP) signal. The receiving ground station antenna
is left-hand circularly polarized (LHCP).

\textbf{Find:} (a) The polarization loss factor. (b) The received power
if the incident power density is 10 pW/m² and the receive antenna has an
effective aperture of 5 m². (c) If the ground antenna is switched to
RHCP, find the new received power.

\textbf{Solution:}

\begin{enumerate}
\def\labelenumi{(\alph{enumi})}
\item
  RHCP and LHCP are orthogonal polarizations. The polarization loss
  factor (PLF) = 0. In dB: \textbf{PLF = -∞ dB (complete polarization
  mismatch; no power is received).}
\item
  P\textsubscript{r} = S × A\textsubscript{e} × PLF = 10 × 10⁻¹² × 5 × 0
  = \textbf{0 W (no signal received)}
\item
  With matched RHCP polarization: PLF = 1 (0 dB). P\textsubscript{r} = S
  × A\textsubscript{e} × 1 = 10 × 10⁻¹² × 5 = \textbf{50 pW = 50 × 10⁻¹²
  W} In dBm: P\textsubscript{r} = 10 log₁₀(50 × 10⁻¹²/10⁻³) = 10 log₁₀(5
  × 10⁻⁸) = \textbf{-73.0 dBm}
\end{enumerate}

This illustrates why polarization matching is critical in satellite
communications.

\begin{center}\rule{0.5\linewidth}{0.5pt}\end{center}

\section{Problem 9.4.4}\label{problem-9.4.4}

\textbf{Given:} Calculate the skin depth and AC resistance increase for
the following metals at 1 GHz. All are non-magnetic (μ\textsubscript{r}
= 1). - Copper: σ = 5.8 × 10⁷ S/m - Aluminum: σ = 3.5 × 10⁷ S/m - Gold:
σ = 4.1 × 10⁷ S/m

A copper PCB trace is 35 μm thick (1 oz copper) and 0.5 mm wide.

\textbf{Find:} (a) The skin depth for each metal at 1 GHz. (b) Whether
the 35 μm copper trace is thick enough for effective RF conduction at 1
GHz. (c) The sheet resistance of the copper trace at 1 GHz.

\textbf{Solution:}

\begin{enumerate}
\def\labelenumi{(\alph{enumi})}
\tightlist
\item
  δ = 1 / √(πfμ₀σ)
\end{enumerate}

Copper: δ = 1 / √(π × 10⁹ × 4π × 10⁻⁷ × 5.8 × 10⁷) = 1 / √(π × 10⁹ ×
72.88) = 1 / √(2.289 × 10¹¹) = \textbf{2.09 μm}

Aluminum: δ = 1 / √(π × 10⁹ × 4π × 10⁻⁷ × 3.5 × 10⁷) = 1 / √(1.382 ×
10¹¹) = \textbf{2.69 μm}

Gold: δ = 1 / √(π × 10⁹ × 4π × 10⁻⁷ × 4.1 × 10⁷) = 1 / √(1.619 × 10¹¹) =
\textbf{2.49 μm}

\begin{enumerate}
\def\labelenumi{(\alph{enumi})}
\setcounter{enumi}{1}
\item
  The copper trace is 35 μm thick = 35/2.09 = \textbf{16.7 skin depths}
  thick. Since current penetrates approximately 3δ = 6.3 μm from each
  side, the trace is far thicker than needed. \textbf{The trace is
  effectively infinitely thick for RF at 1 GHz.}
\item
  Sheet resistance: R\textsubscript{s} = 1/(σδ) = 1/(5.8 × 10⁷ × 2.09 ×
  10⁻⁶) = 1/121.2 = \textbf{8.25 mΩ/square}
\end{enumerate}

For the 0.5 mm wide trace, current flows on both top and bottom
surfaces, so the effective resistance per unit length is approximately
R\textsubscript{s}/(2w) = 8.25 × 10⁻³ / (2 × 0.5 × 10⁻³) = \textbf{8.25
Ω/m}.

\begin{center}\rule{0.5\linewidth}{0.5pt}\end{center}

\section{Problem 9.4.5}\label{problem-9.4.5}

\textbf{Given:} A 900 MHz cellular signal in free space (η₁ = 377 Ω)
strikes a brick wall (ε\textsubscript{r} = 4.0, μ\textsubscript{r} = 1,
lossless approximation) at normal incidence.

\textbf{Find:} (a) The intrinsic impedance of the brick. (b) The
reflection coefficient. (c) The percentage of power reflected and
transmitted. (d) If 1 W/m² is incident, find the power density
transmitted into the brick.

\textbf{Solution:}

\begin{enumerate}
\def\labelenumi{(\alph{enumi})}
\item
  η₂ = η₀ / √(ε\textsubscript{r}) = 377 / √4 = 377 / 2 = \textbf{188.5
  Ω}
\item
  Γ = (η₂ - η₁) / (η₂ + η₁) = (188.5 - 377) / (188.5 + 377) = -188.5 /
  565.5 = \textbf{-0.3333}
\item
  Power reflected: \textbar Γ\textbar² = (1/3)² = \textbf{11.1\%} Power
  transmitted: 1 - \textbar Γ\textbar² = \textbf{88.9\%}
\item
  Transmitted power density: S\textsubscript{t} = 0.889 × 1 =
  \textbf{0.889 W/m²}
\end{enumerate}

Note: This calculation neglects the absorption within the brick and the
second reflection at the exit surface. A real brick wall (with loss
tangent and finite thickness) would have additional absorption loss,
making the total wall attenuation significantly higher.

\begin{center}\rule{0.5\linewidth}{0.5pt}\end{center}

\section{Problem 9.4.6}\label{problem-9.4.6}

\textbf{Given:} A 2.4 GHz Wi-Fi access point transmits 200 mW (23 dBm)
with an antenna gain of 6 dBi. An isotropic receiver is located 50 m
away.

\textbf{Find:} (a) The EIRP (Effective Isotropic Radiated Power). (b)
The power density at the receiver using the Poynting vector. (c) The
electric field strength at the receiver. (d) Compare the power density
to the ICNIRP general public exposure limit of 10 W/m² at 2.4 GHz.

\textbf{Solution:}

\begin{enumerate}
\def\labelenumi{(\alph{enumi})}
\item
  EIRP = P\textsubscript{t} × G\textsubscript{t}(linear) = 200 × 10⁻³ ×
  10\textsuperscript{6/10} = 0.2 × 3.981 = \textbf{796 mW} In dBm: EIRP
  = 23 + 6 = \textbf{29 dBm}
\item
  S = EIRP / (4πr²) = 0.796 / (4π × 50²) = 0.796 / 31,416 =
  \textbf{2.533 × 10⁻⁵ W/m² = 25.3 μW/m²}
\item
  S = E² / (2η₀), so E = √(2η₀S) = √(2 × 377 × 2.533 × 10⁻⁵) =
  √(0.01910) = \textbf{0.138 V/m = 138 mV/m}
\item
  The power density of 25.3 μW/m² is far below the 10 W/m² limit --- by
  a factor of 395,000 (56 dB). \textbf{The exposure is completely safe},
  even accounting for multiple access points in the environment.
\end{enumerate}

\begin{center}\rule{0.5\linewidth}{0.5pt}\end{center}

\section{Problem 9.4.7}\label{problem-9.4.7}

\textbf{Given:} A 5 GHz Wi-Fi signal must pass through a concrete floor
slab. The concrete has ε\textsubscript{r} = 5.5, tan δ = 0.03,
μ\textsubscript{r} = 1, and the slab is 150 mm thick.

\textbf{Find:} (a) The attenuation constant α. (b) The total absorption
loss through the slab. (c) The reflection loss at each air-concrete
interface. (d) The total signal attenuation through the slab.

\textbf{Solution:}

\begin{enumerate}
\def\labelenumi{(\alph{enumi})}
\item
  For low-loss dielectric (tan δ \textless\textless{} 1): α = (πf tan δ
  √ε\textsubscript{r}) / c = (π × 5 × 10⁹ × 0.03 × √5.5) / (3 × 10⁸) =
  (π × 5 × 10⁹ × 0.03 × 2.345) / (3 × 10⁸) = (1.104 × 10⁹) / (3 × 10⁸) =
  \textbf{3.68 Np/m}
\item
  Absorption loss: A = α × d = 3.68 × 0.15 = 0.552 Np = 0.552 × 8.686 =
  \textbf{4.79 dB}
\item
  Reflection coefficient at each surface: η₂ = 377 / √5.5 = 377 / 2.345
  = 160.8 Ω Γ = (160.8 - 377) / (160.8 + 377) = -216.2 / 537.8 = -0.402
\end{enumerate}

Reflection loss per surface: -10 log₁₀(1 - \textbar Γ\textbar²) = -10
log₁₀(1 - 0.1616) = -10 log₁₀(0.8384) = \textbf{0.77 dB} Total
reflection loss (two surfaces): 2 × 0.77 = \textbf{1.53 dB}

\begin{enumerate}
\def\labelenumi{(\alph{enumi})}
\setcounter{enumi}{3}
\tightlist
\item
  Total attenuation: A\textsubscript{total} = absorption + reflection =
  4.79 + 1.53 = \textbf{6.32 dB}
\end{enumerate}

A single concrete floor attenuates the 5 GHz signal by approximately 6.3
dB (factor of 4.3× in power). Multiple floors would multiply this loss,
which is why multi-story buildings often need access points on every
floor.

\begin{center}\rule{0.5\linewidth}{0.5pt}\end{center}

\section{Problem 9.4.8}\label{problem-9.4.8}

\textbf{Given:} An electromagnetic shield made of 0.1 mm thick copper
foil (σ = 5.8 × 10⁷ S/m, μ\textsubscript{r} = 1) is used to shield a
sensitive receiver from interference at 100 kHz.

\textbf{Find:} (a) The skin depth at 100 kHz. (b) The absorption loss.
(c) Whether the shield provides adequate attenuation (target:
\textgreater40 dB).

\textbf{Solution:}

\begin{enumerate}
\def\labelenumi{(\alph{enumi})}
\item
  δ = 1 / √(πfμ₀σ) = 1 / √(π × 10⁵ × 4π × 10⁻⁷ × 5.8 × 10⁷) = 1 / √(π ×
  10⁵ × 72.88) = 1 / √(2.289 × 10⁷) = \textbf{0.209 mm = 209 μm}
\item
  A = 8.686 × (t/δ) = 8.686 × (0.1/0.209) = 8.686 × 0.4785 =
  \textbf{4.16 dB}
\end{enumerate}

The absorption loss alone (4.16 dB) is far below the 40 dB target. At
this low frequency, the copper foil is less than one skin depth thick,
so absorption is minimal.

However, the reflection loss for a good conductor at low frequency is
very high: R ≈ 168 - 10 log₁₀(f × μ\textsubscript{r}/σ\textsubscript{r})
where σ\textsubscript{r} = σ/σ\textsubscript{Cu} = 1.0 R = 168 - 10
log₁₀(10⁵ × 1/1) = 168 - 50 = \textbf{118 dB}

Total SE = A + R = 4.16 + 118 = \textbf{122 dB
\textgreater\textgreater{} 40 dB}

\begin{enumerate}
\def\labelenumi{(\alph{enumi})}
\setcounter{enumi}{2}
\tightlist
\item
  \textbf{Yes, the shield is more than adequate.} At low frequencies,
  the shielding is dominated by reflection (impedance mismatch between
  the low-impedance conductor and the high-impedance wave in free
  space), not absorption. Even at 100 kHz where the foil is thin
  relative to the skin depth, the copper reflects virtually all incident
  energy.
\end{enumerate}

\chapter{Chapter 9 --- Section 9.5: Transmission
Lines}\label{chapter-9-section-9.5-transmission-lines}

Practice problems covering characteristic impedance, reflection and
standing waves, the Smith Chart, waveguides, microstrip and stripline,
transmission line transients and TDR, impedance matching, and coupled
lines/crosstalk.

\begin{center}\rule{0.5\linewidth}{0.5pt}\end{center}

\section{Problem 9.5.1}\label{problem-9.5.1}

\textbf{Given:} A coaxial cable has inner conductor diameter 2a = 1.02
mm and outer conductor inner diameter 2b = 3.66 mm. The dielectric is
solid polyethylene with ε\textsubscript{r} = 2.25 and μ\textsubscript{r}
= 1.

\textbf{Find:} (a) The characteristic impedance. (b) The propagation
velocity. (c) The propagation delay per meter. (d) The electrical length
in wavelengths for a 10 m cable at 100 MHz.

\textbf{Solution:}

\begin{enumerate}
\def\labelenumi{(\alph{enumi})}
\tightlist
\item
  Z₀ = (1/2π) × (η₀/√ε\textsubscript{r}) × ln(b/a) = (1/2π) × (377/1.5)
  × ln(1.83/0.51) = (1/6.283) × 251.3 × ln(3.588) = (1/6.283) × 251.3 ×
  1.278 = (1/6.283) × 321.2 = \textbf{51.1 Ω}
\end{enumerate}

This is a standard 50 Ω coaxial cable (RG-58 type); the slight deviation
from 50 Ω is within manufacturing tolerance.

\begin{enumerate}
\def\labelenumi{(\alph{enumi})}
\setcounter{enumi}{1}
\item
  v = c / √ε\textsubscript{r} = 3 × 10⁸ / 1.5 = \textbf{2.0 × 10⁸ m/s}
  (velocity factor = 0.667 = 66.7\%)
\item
  Delay = 1/v = 1/(2.0 × 10⁸) = \textbf{5.0 ns/m}
\item
  λ = v/f = 2.0 × 10⁸ / 10⁸ = 2.0 m. Electrical length = 10 / 2.0 =
  \textbf{5.0 wavelengths}
\end{enumerate}

\begin{center}\rule{0.5\linewidth}{0.5pt}\end{center}

\section{Problem 9.5.2}\label{problem-9.5.2}

\textbf{Given:} A 75 Ω coaxial cable is terminated with an antenna
having impedance Z\textsubscript{L} = 150 + j100 Ω.

\textbf{Find:} (a) The complex reflection coefficient (magnitude and
phase). (b) The VSWR. (c) The return loss. (d) The fraction of incident
power delivered to the load.

\textbf{Solution:}

\begin{enumerate}
\def\labelenumi{(\alph{enumi})}
\tightlist
\item
  Γ = (Z\textsubscript{L} - Z₀) / (Z\textsubscript{L} + Z₀) = (150 +
  j100 - 75) / (150 + j100 + 75) = (75 + j100) / (225 + j100)
\end{enumerate}

\textbar75 + j100\textbar{} = √(75² + 100²) = √(5625 + 10000) = √15625 =
125 \textbar225 + j100\textbar{} = √(225² + 100²) = √(50625 + 10000) =
√60625 = 246.2

\textbar Γ\textbar{} = 125 / 246.2 = \textbf{0.5077}

Phase: ∠Γ = arctan(100/75) - arctan(100/225) = 53.13° - 23.96° =
\textbf{29.17°}

So Γ = 0.508 ∠29.2°

\begin{enumerate}
\def\labelenumi{(\alph{enumi})}
\setcounter{enumi}{1}
\item
  VSWR = (1 + \textbar Γ\textbar) / (1 - \textbar Γ\textbar) = 1.508 /
  0.492 = \textbf{3.06:1}
\item
  Return loss = -20 log₁₀(\textbar Γ\textbar) = -20 log₁₀(0.508) = -20 ×
  (-0.294) = \textbf{5.88 dB}
\item
  Power delivered: 1 - \textbar Γ\textbar² = 1 - 0.258 = \textbf{0.742 =
  74.2\%}
\end{enumerate}

\begin{center}\rule{0.5\linewidth}{0.5pt}\end{center}

\section{Problem 9.5.3}\label{problem-9.5.3}

\textbf{Given:} An antenna with impedance Z\textsubscript{L} = 36 + j20
Ω must be matched to a 50 Ω system at 2.4 GHz.

\textbf{Find:} (a) The normalized impedance. (b) The reflection
coefficient magnitude. (c) The VSWR. (d) Design a quarter-wave
transformer match (assuming the reactive part is first tuned out with a
series capacitor).

\textbf{Solution:}

\begin{enumerate}
\def\labelenumi{(\alph{enumi})}
\item
  z = Z\textsubscript{L} / Z₀ = (36 + j20) / 50 = \textbf{0.72 + j0.40}
\item
  \textbar Γ\textbar{} = \textbar z - 1\textbar{} / \textbar z +
  1\textbar{} = \textbar(−0.28 + j0.40)\textbar{} / \textbar(1.72 +
  j0.40)\textbar{} = √(0.0784 + 0.16) / √(2.9584 + 0.16) = √0.2384 /
  √3.1184 = 0.4883 / 1.7659 = \textbf{0.2766}
\item
  VSWR = (1 + 0.277) / (1 - 0.277) = 1.277 / 0.723 = \textbf{1.77:1}
\item
  First, tune out the +j20 Ω reactance with a series capacitor: C = 1 /
  (2πfX) = 1 / (2π × 2.4 × 10⁹ × 20) = 1 / (3.016 × 10¹¹) = \textbf{3.32
  pF}
\end{enumerate}

Now the load is purely resistive at 36 Ω. Quarter-wave transformer:
Z\textsubscript{T} = √(Z₀ × R\textsubscript{L}) = √(50 × 36) = √1800 =
\textbf{42.43 Ω}

Transformer length: λ/4 at 2.4 GHz. λ = c/f = 3 × 10⁸ / 2.4 × 10⁹ =
0.125 m in free space. On a PCB (ε\textsubscript{eff} ≈ 3.0):
λ\textsubscript{eff} = 0.125 / √3 = 0.0722 m. Length =
λ\textsubscript{eff}/4 = \textbf{18.0 mm}

\begin{center}\rule{0.5\linewidth}{0.5pt}\end{center}

\section{Problem 9.5.4}\label{problem-9.5.4}

\textbf{Given:} A WR-62 rectangular waveguide has interior dimensions a
= 15.80 mm and b = 7.90 mm.

\textbf{Find:} (a) The cutoff frequency of the dominant TE₁₀ mode. (b)
The cutoff frequency of the next mode (TE₂₀). (c) The single-mode
operating range. (d) The guide wavelength at 15 GHz.

\textbf{Solution:}

\begin{enumerate}
\def\labelenumi{(\alph{enumi})}
\item
  f\textsubscript{c(TE₁₀)} = c / (2a) = 3 × 10⁸ / (2 × 0.01580) = 3 ×
  10⁸ / 0.03160 = \textbf{9.494 GHz}
\item
  f\textsubscript{c(TE₂₀)} = c / a = 3 × 10⁸ / 0.01580 = \textbf{18.99
  GHz}
\end{enumerate}

Also check TE₀₁: f\textsubscript{c(TE₀₁)} = c / (2b) = 3 × 10⁸ / (2 ×
0.00790) = 18.99 GHz --- same as TE₂₀.

\begin{enumerate}
\def\labelenumi{(\alph{enumi})}
\setcounter{enumi}{2}
\item
  Single-mode range: from \textasciitilde1.25 × f\textsubscript{c(TE₁₀)}
  to f\textsubscript{c(TE₂₀)}: \textbf{11.9 GHz to 19.0 GHz} (the
  Ku-band). Practical usable range is 12.4-18.0 GHz.
\item
  At 15 GHz: λ₀ = c/f = 3 × 10⁸ / 15 × 10⁹ = 20.0 mm λ\textsubscript{g}
  = λ₀ / √(1 - (f\textsubscript{c}/f)²) = 20.0 / √(1 - (9.494/15)²) =
  20.0 / √(1 - 0.4007) = 20.0 / √0.5993 = 20.0 / 0.7742 = \textbf{25.83
  mm}
\end{enumerate}

\begin{center}\rule{0.5\linewidth}{0.5pt}\end{center}

\section{Problem 9.5.5}\label{problem-9.5.5}

\textbf{Given:} A 50 Ω microstrip line is designed on Rogers RO4003C
substrate (ε\textsubscript{r} = 3.55, h = 0.508 mm, t = 0.035 mm
copper).

\textbf{Find:} (a) The approximate trace width for 50 Ω impedance. (b)
The effective dielectric constant. (c) The propagation delay per unit
length. (d) The maximum frequency for which a 100 mm trace length is
less than λ/10 (electrically short).

\textbf{Solution:}

\begin{enumerate}
\def\labelenumi{(\alph{enumi})}
\item
  Using the microstrip impedance approximation Z₀ ≈
  (87/√(ε\textsubscript{r} + 1.41)) × ln(5.98h/(0.8w + t)): Solving
  iteratively for Z₀ = 50 Ω: w ≈ \textbf{1.12 mm} (w/h ≈ 2.2)
\item
  ε\textsubscript{eff} ≈ (ε\textsubscript{r} + 1)/2 +
  (ε\textsubscript{r} - 1)/(2√(1 + 12h/w)) = (3.55 + 1)/2 + (3.55 -
  1)/(2√(1 + 12 × 0.508/1.12)) = 2.275 + 2.55/(2√(1 + 5.44)) = 2.275 +
  2.55/(2 × 2.539) = 2.275 + 0.502 = \textbf{2.777}
\item
  v = c/√ε\textsubscript{eff} = 3 × 10⁸ / √2.777 = 3 × 10⁸ / 1.666 =
  1.801 × 10⁸ m/s Delay = 1/v = \textbf{5.55 ns/m = 5.55 ps/mm}
\item
  For a 100 mm trace to be \textless{} λ/10: λ \textgreater{} 10 × 0.1 =
  1.0 m. f\textsubscript{max} = v / λ = 1.801 × 10⁸ / 1.0 = \textbf{180
  MHz}
\end{enumerate}

Above 180 MHz, the trace must be treated as a transmission line and
impedance-controlled routing is required.

\begin{center}\rule{0.5\linewidth}{0.5pt}\end{center}

\section{Problem 9.5.6}\label{problem-9.5.6}

\textbf{Given:} A 50 Ω transmission line (velocity factor 0.66, length 2
m) connects a 50 Ω source (V\textsubscript{S} = 2 V step) to a load
Z\textsubscript{L} = 200 Ω.

\textbf{Find:} Using a lattice diagram, find (a) the initial voltage
wave, (b) the voltage at the load after the first transit, (c) whether
any re-reflection occurs, and (d) the final steady-state voltage at the
load.

\textbf{Solution:}

\begin{enumerate}
\def\labelenumi{(\alph{enumi})}
\tightlist
\item
  Propagation velocity: v = 0.66 × 3 × 10⁸ = 1.98 × 10⁸ m/s One-way
  delay: t\textsubscript{d} = l/v = 2 / 1.98 × 10⁸ = 10.1 ns
\end{enumerate}

Initial voltage wave: V⁺ = V\textsubscript{S} × Z₀ / (Z\textsubscript{S}
+ Z₀) = 2 × 50 / (50 + 50) = \textbf{1.0 V}

\begin{enumerate}
\def\labelenumi{(\alph{enumi})}
\setcounter{enumi}{1}
\tightlist
\item
  Load reflection coefficient: Γ\textsubscript{L} = (200 - 50)/(200 +
  50) = 150/250 = 0.6 At t = 10.1 ns: V\textsubscript{L} = V⁺(1 +
  Γ\textsubscript{L}) = 1.0 × (1 + 0.6) = \textbf{1.6 V}
\end{enumerate}

Source reflection coefficient: Γ\textsubscript{S} = (50 - 50)/(50 + 50)
= \textbf{0} (matched source)

\begin{enumerate}
\def\labelenumi{(\alph{enumi})}
\setcounter{enumi}{2}
\item
  The reflected wave V⁻ = Γ\textsubscript{L}V⁺ = 0.6 × 1.0 = 0.6 V
  travels back. Since Γ\textsubscript{S} = 0, the source absorbs this
  wave entirely. \textbf{No re-reflection occurs.}
\item
  Steady-state: V\textsubscript{L} = V\textsubscript{S} ×
  Z\textsubscript{L}/(Z\textsubscript{S} + Z\textsubscript{L}) = 2 ×
  200/(50 + 200) = 400/250 = \textbf{1.6 V}
\end{enumerate}

The line reaches steady state after just one round trip (20.2 ns)
because the source is matched to Z₀.

\begin{center}\rule{0.5\linewidth}{0.5pt}\end{center}

\section{Problem 9.5.7}\label{problem-9.5.7}

\textbf{Given:} A 433 MHz ISM-band transmitter has a 50 Ω output and
must drive a loop antenna with impedance Z\textsubscript{L} = 10 Ω
(purely resistive at the design frequency).

\textbf{Find:} (a) Design an L-network match. (b) Calculate component
values. (c) Estimate the matching bandwidth.

\textbf{Solution:}

\begin{enumerate}
\def\labelenumi{(\alph{enumi})}
\tightlist
\item
  Since R\textsubscript{L} = 10 Ω \textless{} R\textsubscript{S} = 50 Ω,
  the shunt element goes across the high-impedance (source) side. Q =
  √(R\textsubscript{high}/R\textsubscript{low} - 1) = √(50/10 - 1) = √4
  = \textbf{2.0}
\end{enumerate}

Shunt reactance: X\textsubscript{shunt} = R\textsubscript{high}/Q = 50/2
= \textbf{25 Ω} Series reactance: X\textsubscript{series} = Q ×
R\textsubscript{low} = 2 × 10 = \textbf{20 Ω}

\begin{enumerate}
\def\labelenumi{(\alph{enumi})}
\setcounter{enumi}{1}
\tightlist
\item
  \textbf{Low-pass solution (preferred for harmonic rejection):} shunt
  inductor + series capacitor. L\textsubscript{shunt} =
  X\textsubscript{shunt}/(2πf) = 25/(2π × 433 × 10⁶) = \textbf{9.19 nH}
  C\textsubscript{series} = 1/(2πfX\textsubscript{series}) = 1/(2π × 433
  × 10⁶ × 20) = \textbf{18.4 pF}
\end{enumerate}

\textbf{High-pass solution:} shunt capacitor + series inductor.
C\textsubscript{shunt} = 1/(2πf × 25) = 14.7 pF L\textsubscript{series}
= 20/(2πf) = 7.35 nH

\begin{enumerate}
\def\labelenumi{(\alph{enumi})}
\setcounter{enumi}{2}
\tightlist
\item
  Bandwidth: BW ≈ f₀/Q = 433/2.0 = \textbf{216.5 MHz}
\end{enumerate}

The VSWR \textless{} 2:1 bandwidth spans approximately ±108 MHz around
433 MHz, which is far wider than the ISM band (433.05-434.79 MHz). The
match easily covers the full band.

\begin{center}\rule{0.5\linewidth}{0.5pt}\end{center}

\section{Problem 9.5.8}\label{problem-9.5.8}

\textbf{Given:} Two 50 Ω microstrip traces on a 4-layer PCB (h = 0.15
mm, ε\textsubscript{eff} = 3.0) run parallel with edge-to-edge spacing s
= 0.15 mm for a coupled length of 50 mm. The aggressor carries a 1.8 V
LVCMOS signal with 0.3 ns rise time. The coupling parameters are:
C\textsubscript{m} = 35 pF/m, L\textsubscript{m} = 80 nH/m, C₀ = 110
pF/m, L₀ = 275 nH/m.

\textbf{Find:} (a) The NEXT coefficient K\textsubscript{b}. (b) The
critical length. (c) The NEXT voltage. (d) The FEXT coefficient and
voltage.

\textbf{Solution:}

\begin{enumerate}
\def\labelenumi{(\alph{enumi})}
\item
  K\textsubscript{b} = (C\textsubscript{m}/C₀ + L\textsubscript{m}/L₀) /
  4 = (35/110 + 80/275) / 4 = (0.3182 + 0.2909) / 4 = 0.6091 / 4 =
  \textbf{0.1523}
\item
  v = 1/√(L₀C₀) = 1/√(275 × 10⁻⁹ × 110 × 10⁻¹²) = 1/√(3.025 × 10⁻¹⁷) =
  1.818 × 10⁸ m/s Critical length: l\textsubscript{crit} =
  t\textsubscript{r} × v / 2 = 0.3 × 10⁻⁹ × 1.818 × 10⁸ / 2 =
  \textbf{27.3 mm}
\item
  Coupled length (50 mm) \textgreater{} l\textsubscript{crit} (27.3 mm),
  so NEXT saturates: V\textsubscript{NEXT} = K\textsubscript{b} ×
  V\textsubscript{step} = 0.1523 × 1.8 = \textbf{274 mV} (15.2\% of
  signal amplitude)
\end{enumerate}

This is dangerously high for LVCMOS (noise margin \textasciitilde0.4 V),
and could cause false switching.

\begin{enumerate}
\def\labelenumi{(\alph{enumi})}
\setcounter{enumi}{3}
\tightlist
\item
  K\textsubscript{f} = (C\textsubscript{m}/C₀ - L\textsubscript{m}/L₀) /
  2 = (0.3182 - 0.2909) / 2 = 0.02727 / 2 = \textbf{0.01364}
\end{enumerate}

Propagation delay for 50 mm: t\textsubscript{d} = 0.05 / 1.818 × 10⁸ =
0.275 ns V\textsubscript{FEXT} = K\textsubscript{f} ×
V\textsubscript{step} × (t\textsubscript{d}/t\textsubscript{r}) =
0.01364 × 1.8 × (0.275/0.3) = 0.01364 × 1.8 × 0.917 = \textbf{22.5 mV}

FEXT is much smaller than NEXT because C\textsubscript{m}/C₀ and
L\textsubscript{m}/L₀ are nearly equal. The design should increase
spacing to at least 3h = 0.45 mm (the ``3W'' rule) to reduce NEXT to
acceptable levels.

\begin{center}\rule{0.5\linewidth}{0.5pt}\end{center}

\section{Problem 9.5.9}\label{problem-9.5.9}

\textbf{Given:} A quarter-wave transformer must match a 100 Ω load to a
50 Ω line at 5.8 GHz. The transformer is fabricated as a microstrip on
FR-4 (ε\textsubscript{eff} = 3.1).

\textbf{Find:} (a) The required transformer impedance. (b) The physical
length of the transformer. (c) The bandwidth over which VSWR remains
below 1.5:1.

\textbf{Solution:}

\begin{enumerate}
\def\labelenumi{(\alph{enumi})}
\item
  Z\textsubscript{T} = √(Z₀ × Z\textsubscript{L}) = √(50 × 100) = √5000
  = \textbf{70.71 Ω}
\item
  λ\textsubscript{eff} = c / (f × √ε\textsubscript{eff}) = 3 × 10⁸ /
  (5.8 × 10⁹ × √3.1) = 3 × 10⁸ / (5.8 × 10⁹ × 1.761) = 3 × 10⁸ / 1.021 ×
  10¹⁰ = 29.38 mm Length = λ\textsubscript{eff} / 4 = \textbf{7.35 mm}
\item
  For a single-section quarter-wave transformer matching
  Z\textsubscript{L}/Z₀ = 2:1, the fractional bandwidth for VSWR
  \textless{} 1.5:1 is approximately:
\end{enumerate}

For VSWR = 1.5, \textbar Γ\textsubscript{max}\textbar{} = (1.5 - 1)/(1.5
+ 1) = 0.2 At center frequency: \textbar Γ\textsubscript{0}\textbar{} =
(Z\textsubscript{L} - Z₀)/(Z\textsubscript{L} + Z₀) = (100 - 50)/(100 +
50) = 1/3

The bandwidth where \textbar Γ\textbar{} \textless{} 0.2: BW/f₀ ≈ (2/π)
×
arccos(\textbar Γ\textsubscript{max}\textbar/\textbar Γ\textsubscript{0}\textbar{}
× 1/√(1 - \textbar Γ\textsubscript{max}\textbar²))

A simplified estimate: fractional BW ≈ 2 - 4f\textsubscript{c}/f₀,
giving approximately \textbf{40-50\% bandwidth}, or roughly 2.3-3.5 GHz
around 5.8 GHz. The quarter-wave transformer provides reasonably
broadband matching for a 2:1 impedance ratio.

\chapter{Chapter 9 --- Section 9.6:
Antennas}\label{chapter-9-section-9.6-antennas}

Practice problems covering antenna fundamentals, dipole antennas,
antenna arrays, and near-field/far-field regions.

\begin{center}\rule{0.5\linewidth}{0.5pt}\end{center}

\section{Problem 9.6.1}\label{problem-9.6.1}

\textbf{Given:} A satellite transmitter radiates 20 W at 12 GHz
(Ku-band) through a parabolic dish antenna with a gain of 35 dBi. A
ground station 36,000 km away has a receive antenna with gain 42 dBi.

\textbf{Find:} (a) The free-space path loss. (b) The received power
using the Friis equation. (c) The effective aperture of the receive
antenna.

\textbf{Solution:}

\begin{enumerate}
\def\labelenumi{(\alph{enumi})}
\tightlist
\item
  λ = c/f = 3 × 10⁸ / 12 × 10⁹ = 0.025 m FSPL = (4πd/λ)² = (4π × 3.6 ×
  10⁷ / 0.025)² = (1.8096 × 10¹⁰)²
\end{enumerate}

In dB: FSPL = 20 log₁₀(4πd/λ) = 20 log₁₀(1.8096 × 10¹⁰) = 20 × 10.258 =
\textbf{205.2 dB}

\begin{enumerate}
\def\labelenumi{(\alph{enumi})}
\setcounter{enumi}{1}
\tightlist
\item
  P\textsubscript{r}(dBW) = P\textsubscript{t}(dBW) + G\textsubscript{t}
  + G\textsubscript{r} - FSPL = 10 log₁₀(20) + 35 + 42 - 205.2 = 13.0 +
  35 + 42 - 205.2 = \textbf{-115.2 dBW}
\end{enumerate}

P\textsubscript{r} = 10\textsuperscript{-115.2/10} = 3.02 × 10⁻¹² W =
\textbf{3.02 pW}

\begin{enumerate}
\def\labelenumi{(\alph{enumi})}
\setcounter{enumi}{2}
\tightlist
\item
  A\textsubscript{e} = G\textsubscript{r}λ² / (4π) where
  G\textsubscript{r} = 10\textsuperscript{42/10} = 15,849
  A\textsubscript{e} = 15,849 × (0.025)² / (4π) = 15,849 × 6.25 × 10⁻⁴ /
  12.566 = 9.906 / 12.566 = \textbf{0.788 m²}
\end{enumerate}

\begin{center}\rule{0.5\linewidth}{0.5pt}\end{center}

\section{Problem 9.6.2}\label{problem-9.6.2}

\textbf{Given:} A half-wave dipole antenna is designed for the 2-meter
amateur radio band at 146 MHz.

\textbf{Find:} (a) The theoretical full length. (b) The practical length
with 5\% shortening. (c) The radiation resistance. (d) The gain in dBi
and dBd. (e) The beamwidth.

\textbf{Solution:}

\begin{enumerate}
\def\labelenumi{(\alph{enumi})}
\item
  λ = c/f = 3 × 10⁸ / 146 × 10⁶ = 2.055 m L = λ/2 = \textbf{1.027 m}
  (each arm = 0.514 m)
\item
  L\textsubscript{practical} = 0.95 × 1.027 = \textbf{0.976 m} (each arm
  = 0.488 m)
\item
  The radiation resistance of a half-wave dipole is
  \textbf{R\textsubscript{rad} = 73.1 Ω}
\item
  Gain: In dBi: \textbf{G = 2.15 dBi} (linear gain = 1.64) In dBd:
  \textbf{G = 0 dBd} (by definition, since dBd uses the half-wave dipole
  as reference)
\item
  The E-plane (containing the dipole axis) half-power beamwidth of a
  half-wave dipole is \textbf{78°}. The H-plane pattern is
  omnidirectional (360°).
\end{enumerate}

\begin{center}\rule{0.5\linewidth}{0.5pt}\end{center}

\section{Problem 9.6.3}\label{problem-9.6.3}

\textbf{Given:} A broadside uniform linear array of N = 16 isotropic
elements is spaced at d = λ/2 and operates at 10 GHz.

\textbf{Find:} (a) The wavelength and element spacing. (b) The
half-power beamwidth. (c) The array directivity. (d) The total array
length.

\textbf{Solution:}

\begin{enumerate}
\def\labelenumi{(\alph{enumi})}
\item
  λ = c/f = 3 × 10⁸ / 10 × 10⁹ = 0.03 m = 30 mm d = λ/2 = \textbf{15 mm}
\item
  θ\textsubscript{3dB} ≈ 0.886λ / (Nd) = 0.886 × 0.03 / (16 × 0.015) =
  0.02658 / 0.24 = 0.1108 rad = \textbf{6.35°}
\item
  For a uniform linear array of isotropic elements with λ/2 spacing: D ≈
  N = 16 = \textbf{12.04 dBi}
\item
  L = (N - 1) × d = 15 × 0.015 = \textbf{0.225 m = 225 mm}
\end{enumerate}

\begin{center}\rule{0.5\linewidth}{0.5pt}\end{center}

\section{Problem 9.6.4}\label{problem-9.6.4}

\textbf{Given:} A parabolic reflector antenna has a diameter D = 1.2 m
and operates at 6 GHz (C-band satellite uplink). The aperture efficiency
is η\textsubscript{ap} = 0.55.

\textbf{Find:} (a) The wavelength. (b) The physical aperture area. (c)
The effective aperture. (d) The antenna gain. (e) The far-field boundary
distance. (f) The half-power beamwidth.

\textbf{Solution:}

\begin{enumerate}
\def\labelenumi{(\alph{enumi})}
\item
  λ = c/f = 3 × 10⁸ / 6 × 10⁹ = \textbf{0.05 m = 50 mm}
\item
  A\textsubscript{physical} = π(D/2)² = π(0.6)² = \textbf{1.131 m²}
\item
  A\textsubscript{e} = η\textsubscript{ap} × A\textsubscript{physical} =
  0.55 × 1.131 = \textbf{0.622 m²}
\item
  G = 4πA\textsubscript{e}/λ² = 4π × 0.622 / (0.05)² = 7.817 / 0.0025 =
  3127 = \textbf{34.95 dBi}
\end{enumerate}

Alternatively: G = η\textsubscript{ap}(πD/λ)² = 0.55 × (π × 1.2/0.05)² =
0.55 × (75.4)² = 0.55 × 5685 = 3127, confirming.

\begin{enumerate}
\def\labelenumi{(\alph{enumi})}
\setcounter{enumi}{4}
\item
  Far-field boundary: r\textsubscript{ff} = 2D²/λ = 2 × (1.2)² / 0.05 =
  2.88 / 0.05 = \textbf{57.6 m}
\item
  Half-power beamwidth: θ\textsubscript{3dB} ≈ 70λ/D = 70 × 0.05 / 1.2 =
  3.5 / 1.2 = \textbf{2.92°}
\end{enumerate}

\begin{center}\rule{0.5\linewidth}{0.5pt}\end{center}

\section{Problem 9.6.5}\label{problem-9.6.5}

\textbf{Given:} A 5G millimeter-wave base station uses a planar phased
array antenna panel of dimensions 0.3 m × 0.3 m operating at 39 GHz. An
NFC reader at 13.56 MHz uses a 40 mm × 40 mm loop antenna.

\textbf{Find:} For each system: (a) The far-field boundary. (b) The
reactive near-field extent. (c) Classify the typical operating range as
near-field or far-field.

\textbf{Solution:}

\textbf{5G mmWave (39 GHz):} λ = 3 × 10⁸ / 39 × 10⁹ = 7.69 mm, D = 0.3 m

\begin{enumerate}
\def\labelenumi{(\alph{enumi})}
\item
  r\textsubscript{ff} = 2D²/λ = 2 × 0.09 / 0.00769 = \textbf{23.4 m}
\item
  r\textsubscript{reactive} = 0.62√(D³/λ) = 0.62√(0.027/0.00769) =
  0.62√3.512 = 0.62 × 1.874 = \textbf{1.16 m}
\item
  A 5G base station communicates at typical distances of 50-300 m, which
  is well into the \textbf{far-field region}. Antenna measurements
  require at least 23.4 m of clear range.
\end{enumerate}

\textbf{NFC (13.56 MHz):} λ = 3 × 10⁸ / 13.56 × 10⁶ = 22.12 m, D = 0.04
m

\begin{enumerate}
\def\labelenumi{(\alph{enumi})}
\item
  r\textsubscript{ff} = 2D²/λ = 2 × 0.0016 / 22.12 = \textbf{0.145 mm}
  --- essentially zero.
\item
  r\textsubscript{reactive} = 0.62√(D³/λ) = 0.62√(6.4 × 10⁻⁵ / 22.12) =
  0.62√(2.89 × 10⁻⁶) = 0.62 × 1.70 × 10⁻³ = \textbf{1.05 mm}
\item
  NFC operates at ranges up to \textasciitilde100 mm, which is deep in
  the \textbf{reactive near-field}. At 100 mm, the field decays as
  approximately (1.05/100)³ ≈ 10⁻⁶ relative to the reactive boundary,
  providing the security advantage of NFC's extremely limited range.
\end{enumerate}

\begin{center}\rule{0.5\linewidth}{0.5pt}\end{center}

\section{Problem 9.6.6}\label{problem-9.6.6}

\textbf{Given:} A Yagi-Uda antenna has 6 elements (1 driven dipole, 1
reflector, 4 directors) and operates at 432 MHz (70 cm amateur band).
The measured gain is 11.5 dBi and the front-to-back ratio is 20 dB. The
antenna input impedance is 28 Ω.

\textbf{Find:} (a) The wavelength. (b) The VSWR when fed with 50 Ω coax.
(c) The effective aperture. (d) If the transmitter power is 50 W, find
the EIRP and the power density at 10 km.

\textbf{Solution:}

\begin{enumerate}
\def\labelenumi{(\alph{enumi})}
\item
  λ = c/f = 3 × 10⁸ / 432 × 10⁶ = \textbf{0.694 m = 69.4 cm}
\item
  Γ = (28 - 50)/(28 + 50) = -22/78 = -0.282 VSWR = (1 + 0.282)/(1 -
  0.282) = 1.282/0.718 = \textbf{1.79:1}
\item
  G = 10\textsuperscript{11.5/10} = 14.13 A\textsubscript{e} = Gλ²/(4π)
  = 14.13 × (0.694)² / (4π) = 14.13 × 0.4816 / 12.566 = 6.805 / 12.566 =
  \textbf{0.541 m²}
\item
  EIRP = P\textsubscript{t} × G = 50 × 14.13 = \textbf{706.5 W} = 28.49
  dBW = \textbf{58.49 dBm}
\end{enumerate}

Power density at 10 km: S = EIRP/(4πr²) = 706.5/(4π × (10⁴)²) = 706.5 /
(1.257 × 10⁹) = \textbf{5.62 × 10⁻⁷ W/m² = 0.562 μW/m²}

\chapter{Chapter 9 --- Section 9.7: Electromagnetic
Compatibility}\label{chapter-9-section-9.7-electromagnetic-compatibility}

Practice problems covering shielding effectiveness, EMI filtering,
conducted emissions, PCB layout for EMC, ESD protection, and
grounding/bonding.

\begin{center}\rule{0.5\linewidth}{0.5pt}\end{center}

\section{Problem 9.7.1}\label{problem-9.7.1}

\textbf{Given:} A steel enclosure (σ = 6.0 × 10⁶ S/m, μ\textsubscript{r}
= 200) with wall thickness t = 2 mm must shield against a 1 MHz
interference source. The target shielding effectiveness is 80 dB.

\textbf{Find:} (a) The skin depth at 1 MHz. (b) The absorption loss. (c)
The approximate reflection loss. (d) Whether the target SE is met. (e)
The maximum permissible ventilation slot length.

\textbf{Solution:}

\begin{enumerate}
\def\labelenumi{(\alph{enumi})}
\item
  δ = 1/√(πfμ₀μ\textsubscript{r}σ) = 1/√(π × 10⁶ × 4π × 10⁻⁷ × 200 × 6.0
  × 10⁶) = 1/√(π × 10⁶ × 1.508 × 10⁻³) = 1/√(4,736) = \textbf{14.5 μm}
\item
  A = 8.686 × (t/δ) = 8.686 × (2 × 10⁻³ / 14.5 × 10⁻⁶) = 8.686 × 137.9 =
  \textbf{1,198 dB}
\item
  σ\textsubscript{r} = σ/σ\textsubscript{Cu} = 6.0 × 10⁶ / 5.8 × 10⁷ =
  0.1034 R ≈ 168 - 10 log₁₀(f × μ\textsubscript{r}/σ\textsubscript{r}) =
  168 - 10 log₁₀(10⁶ × 200/0.1034) = 168 - 10 log₁₀(1.934 × 10⁹) = 168 -
  92.9 = \textbf{75.1 dB}
\item
  Total SE = A + R = 1,198 + 75.1 = 1,273 dB \textgreater\textgreater{}
  80 dB. \textbf{The target is easily met} by the solid walls.
\end{enumerate}

However, any aperture in the enclosure is the limiting factor.

\begin{enumerate}
\def\labelenumi{(\alph{enumi})}
\setcounter{enumi}{4}
\tightlist
\item
  At 1 MHz: λ = c/f = 3 × 10⁸ / 10⁶ = 300 m. For SE \textgreater{} 80
  dB, the maximum slot length should be less than λ/20 for good
  practice, but the actual limit depends on the required attenuation. A
  slot of length l resonates at λ = 2l and degrades SE dramatically.
\end{enumerate}

Maximum slot: l \textless{} λ/20 = 300/20 = \textbf{15 m} --- at 1 MHz,
even large ventilation openings are not a concern.

At higher frequencies, slots become the bottleneck. At 300 MHz (λ = 1
m), l\textsubscript{max} = 50 mm; at 1 GHz (λ = 0.3 m),
l\textsubscript{max} = \textbf{15 mm}.

\begin{center}\rule{0.5\linewidth}{0.5pt}\end{center}

\section{Problem 9.7.2}\label{problem-9.7.2}

\textbf{Given:} A switching power supply produces the following
conducted emissions measured with a LISN: - At 150 kHz: 82 dBμV
(quasi-peak) - At 500 kHz: 70 dBμV - At 5 MHz: 55 dBμV

The CISPR 32 Class B quasi-peak limits are: - 150 kHz -- 500 kHz: 66-56
dBμV (linearly decreasing in log scale) - 500 kHz -- 5 MHz: 56 dBμV - 5
MHz -- 30 MHz: 60 dBμV

A 6 dB design margin is required.

\textbf{Find:} (a) The amount by which each frequency exceeds the limit.
(b) The required filter attenuation at each frequency. (c) Design an EMI
filter (choose L and C values) to achieve compliance.

\textbf{Solution:}

\begin{enumerate}
\def\labelenumi{(\alph{enumi})}
\item
  At 150 kHz: limit = 66 dBμV, emission = 82 dBμV → \textbf{exceeds by
  16 dB} At 500 kHz: limit = 56 dBμV, emission = 70 dBμV →
  \textbf{exceeds by 14 dB} At 5 MHz: limit = 60 dBμV, emission = 55
  dBμV → \textbf{within limit by 5 dB} (but need 6 dB margin, so need 1
  dB reduction)
\item
  Required attenuation (including 6 dB margin): At 150 kHz: 16 + 6 =
  \textbf{22 dB} At 500 kHz: 14 + 6 = \textbf{20 dB} At 5 MHz: 1 dB
  (already close to margin)
\item
  The critical requirement is 22 dB at 150 kHz. A single-stage LC filter
  with -40 dB/decade roll-off: Attenuation at f: A(f) = (f/f₀)² above
  corner frequency. For 22 dB at 150 kHz: 10\textsuperscript{22/20} =
  12.59 = (150,000/f₀)² f₀ = 150,000/√12.59 = 150,000/3.548 =
  \textbf{42.3 kHz}
\end{enumerate}

Choose L\textsubscript{DM} = 470 μH. Then: C = 1/(4π²f₀²L) = 1/(4π² ×
(42,300)² × 470 × 10⁻⁶) = 1/(4 × 9.87 × 1.789 × 10⁹ × 4.7 × 10⁻⁴) =
1/(3.333 × 10⁷) = \textbf{30.0 nF}

Use a 33 nF X2 capacitor (standard value).

Verify at 500 kHz: A = (500/42.3)² = (11.82)² = 139.7 = 21.45 dB (needs
20 dB --- passes). At 5 MHz: A = (5000/42.3)² = (118.2)² = 13,971 = 41.5
dB (far exceeds the 1 dB needed).

Add a 10 mH common-mode choke and 2.2 nF Y-capacitors for common-mode
filtering.

\textbf{Final filter: 10 mH CM choke + 33 nF X2 cap + 470 μH DM inductor
+ 33 nF X2 cap.}

\begin{center}\rule{0.5\linewidth}{0.5pt}\end{center}

\section{Problem 9.7.3}\label{problem-9.7.3}

\textbf{Given:} A 6-layer PCB stackup has the following layer
assignments: Sig1 -- GND -- Sig2 -- PWR -- GND -- Sig3. The substrate
height between Sig1 and GND is h₁ = 0.1 mm. A 3.3 V, 100 MHz clock trace
runs 80 mm on Sig1. The clock signal has a 50\% duty cycle and the
current is approximately 15 mA for the fundamental.

\textbf{Find:} (a) The loop area formed by the clock trace and its
return current. (b) Compare to the same trace on Sig2 (h₂ = 0.2 mm to
GND). (c) Estimate whether the radiated emission from the Sig1 trace is
likely to pass FCC Class B at 3 m (limit \textasciitilde100 μV/m at 100
MHz).

\textbf{Solution:}

\begin{enumerate}
\def\labelenumi{(\alph{enumi})}
\item
  On Sig1 (0.1 mm above GND): Loop area: A₁ = l × h₁ = 80 × 0.1 = 8.0
  mm² = \textbf{8.0 × 10⁻⁶ m²}
\item
  On Sig2 (0.2 mm above GND): A₂ = 80 × 0.2 = 16.0 mm² = \textbf{16.0 ×
  10⁻⁶ m²} Ratio: A₂/A₁ = 2.0 --- the Sig2 trace radiates \textbf{6 dB
  more} than the Sig1 trace.
\item
  For an electrically small loop antenna, the radiated E-field at
  distance r: E ≈ 1.316 × 10⁻¹⁴ × f² × I × A / r
\end{enumerate}

At 100 MHz from Sig1: E ≈ 1.316 × 10⁻¹⁴ × (10⁸)² × 0.015 × 8 × 10⁻⁶ / 3
= 1.316 × 10⁻¹⁴ × 10¹⁶ × 1.2 × 10⁻⁷ / 3 = 131.6 × 1.2 × 10⁻⁷ / 3 = 5.26
× 10⁻⁶ V/m = \textbf{5.26 μV/m}

This is well below the FCC Class B limit of \textasciitilde100 μV/m at 3
m. \textbf{The design is likely to pass.}

However, harmonics (300 MHz, 500 MHz) will have higher radiation due to
the f² factor, and actual PCB emissions include contributions from all
signal traces, vias, and decoupling loops. A margin of 10-20 dB above
the single-trace estimate is prudent.

\begin{center}\rule{0.5\linewidth}{0.5pt}\end{center}

\section{Problem 9.7.4}\label{problem-9.7.4}

\textbf{Given:} A USB-C port on a consumer device must survive ±8 kV
contact discharge and ±15 kV air discharge per IEC 61000-4-2. The USB
3.2 data lines (10 Gbps) must be protected. A TVS diode array is
selected with: V\textsubscript{BR} = 5.5 V, V\textsubscript{C} = 9 V at
8 A, C\textsubscript{J} = 0.25 pF per line.

\textbf{Find:} (a) The peak current during an 8 kV contact discharge.
(b) The signal integrity impact on USB 3.2 (5 GHz fundamental
frequency). (c) Whether a series resistor can be used and what value is
acceptable.

\textbf{Solution:}

\begin{enumerate}
\def\labelenumi{(\alph{enumi})}
\tightlist
\item
  Per IEC 61000-4-2, an 8 kV contact discharge produces a peak current
  of approximately \textbf{30 A} (with 0.7-1 ns rise time and
  characteristic double-exponential waveform).
\end{enumerate}

The 8 kV voltage is applied through a 330 Ω + 150 pF network defined by
the standard. The initial peak is I\textsubscript{peak} = V /
Z\textsubscript{contact} ≈ 8000/330 ≈ 24 A for the first pulse, followed
by a 30 A peak from the discharge network resonance.

\begin{enumerate}
\def\labelenumi{(\alph{enumi})}
\setcounter{enumi}{1}
\tightlist
\item
  The TVS capacitance of 0.25 pF per line is critical for USB 3.2. With
  a 45 Ω differential line impedance: the 3 dB bandwidth due to the TVS
  capacitance: f\textsubscript{3dB} = 1/(π × Z₀ × C\textsubscript{J}) =
  1/(π × 45 × 0.25 × 10⁻¹²) = \textbf{28.3 GHz}
\end{enumerate}

Since USB 3.2 Gen 2 operates at 10 Gbps (5 GHz fundamental), the TVS
capacitance causes: Insertion loss at 5 GHz ≈ (f/f\textsubscript{3dB})²
= (5/28.3)² = 0.031 = 0.14 dB --- \textbf{negligible impact}.

The S11 (return loss) is more important: \textbar S11\textbar{} ≈
2πfC\textsubscript{J}Z₀ = 2π × 5 × 10⁹ × 0.25 × 10⁻¹² × 45 = 0.354 →
return loss = 9.0 dB. This is marginal but acceptable for ESD
protection.

\begin{enumerate}
\def\labelenumi{(\alph{enumi})}
\setcounter{enumi}{2}
\tightlist
\item
  A series resistor increases insertion loss: at 10 Gbps, even 5 Ω of
  series resistance degrades the eye diagram significantly due to the
  high data rate and the equalizer margin required.
\end{enumerate}

For USB 3.2: \textbf{no series resistor is recommended}. The TVS alone
must handle the ESD event. The low 0.25 pF capacitance TVS was
specifically chosen to avoid needing a series resistor.

For lower-speed USB 2.0 (480 Mbps): a 10-22 Ω series resistor is
acceptable and improves clamping voltage seen by the IC.

\begin{center}\rule{0.5\linewidth}{0.5pt}\end{center}

\section{Problem 9.7.5}\label{problem-9.7.5}

\textbf{Given:} An audio measurement system has a preamplifier
(sensitivity 50 μV full scale, bandwidth 20 Hz - 20 kHz) connected to a
microphone 30 m away via a shielded cable. The building has a 3 V, 60 Hz
ground potential difference between the microphone location and the
preamplifier location. The cable shield has 0.2 Ω resistance.

\textbf{Find:} (a) The ground loop current if the shield is grounded at
both ends. (b) The noise voltage induced in the signal conductor (assume
transfer impedance Z\textsubscript{T} = 15 mΩ/m). (c) The SNR. (d)
Propose a practical solution.

\textbf{Solution:}

\begin{enumerate}
\def\labelenumi{(\alph{enumi})}
\item
  I\textsubscript{loop} = V\textsubscript{ground} /
  R\textsubscript{shield} = 3 / 0.2 = \textbf{15 A}
\item
  V\textsubscript{noise} = I\textsubscript{loop} × Z\textsubscript{T} ×
  length = 15 × 0.015 × 30 = \textbf{6.75 V}
\item
  SNR = 20 log₁₀(V\textsubscript{signal}/V\textsubscript{noise}) = 20
  log₁₀(50 × 10⁻⁶ / 6.75) = 20 log₁₀(7.41 × 10⁻⁶) = 20 × (-5.13) =
  \textbf{-102.6 dB}
\end{enumerate}

The noise completely buries the signal. The measurement is impossible in
this configuration.

\begin{enumerate}
\def\labelenumi{(\alph{enumi})}
\setcounter{enumi}{3}
\tightlist
\item
  \textbf{Solution:} Use a combination of techniques:
\end{enumerate}

\begin{enumerate}
\def\labelenumi{\arabic{enumi}.}
\item
  \textbf{Ground the shield at one end only} (receiver end): This breaks
  the ground loop. The 3 V appears as common-mode voltage on the signal
  pair.
\item
  \textbf{Use a balanced (differential) input} with an instrumentation
  amplifier having CMRR ≥ 100 dB at 60 Hz: V\textsubscript{noise} = 3 V
  / 10\textsuperscript{100/20} = 3 / 100,000 = \textbf{30 μV}
\item
  New SNR = 20 log₁₀(50 × 10⁻⁶ / 30 × 10⁻⁶) = 20 log₁₀(1.67) =
  \textbf{4.4 dB} --- barely usable.
\item
  Add \textbf{galvanic isolation} (audio isolation transformer or
  isolated preamp at the microphone): This completely eliminates the
  ground loop, reducing noise to the preamp's intrinsic noise (typically
  1-5 μV\textsubscript{rms} in a 20 kHz bandwidth).
\end{enumerate}

With isolation: SNR = 20 log₁₀(50 × 10⁻⁶ / 3 × 10⁻⁶) = 20 log₁₀(16.7) =
\textbf{24.4 dB} --- acceptable for the measurement.

\begin{center}\rule{0.5\linewidth}{0.5pt}\end{center}

\section{Problem 9.7.6}\label{problem-9.7.6}

\textbf{Given:} A 4-layer PCB design routes a 25 MHz clock signal (10
mA, 1 ns rise time) across a 15 mm gap (split) in the ground plane on
layer 2. The trace is 40 mm long on the top signal layer. The substrate
height is 0.15 mm.

\textbf{Find:} (a) The loop area with an intact ground plane. (b) The
effective loop area when the return current detours around the 15 mm
split. (c) The increase in radiated emissions (in dB). (d) A design
remedy.

\textbf{Solution:}

\begin{enumerate}
\def\labelenumi{(\alph{enumi})}
\item
  Intact ground plane: A₁ = trace length × substrate height = 40 × 0.15
  = \textbf{6.0 mm²}
\item
  With the 15 mm split, the return current must detour around the gap.
  The detour path adds roughly the split width to the effective loop
  height. Effective loop area: A₂ ≈ 40 × 15 = \textbf{600 mm²}
  (conservative estimate --- the actual loop is the area enclosed by the
  signal trace going forward and the return current going around the
  split)
\end{enumerate}

More precisely, the detour creates a slot antenna of dimensions
approximately 40 mm × 15 mm. A₂ = \textbf{600 mm²}

\begin{enumerate}
\def\labelenumi{(\alph{enumi})}
\setcounter{enumi}{2}
\tightlist
\item
  Ratio: A₂/A₁ = 600/6 = 100. Since radiated emissions scale with loop
  area: ΔE = 20 log₁₀(100) = \textbf{40 dB increase in radiated
  emissions}
\end{enumerate}

A 40 dB increase can easily cause an EMC failure. If the intact design
had 10 dB margin, the split ground plane would exceed the limit by 30
dB.

\begin{enumerate}
\def\labelenumi{(\alph{enumi})}
\setcounter{enumi}{3}
\tightlist
\item
  \textbf{Design remedies:}
\end{enumerate}

\begin{itemize}
\tightlist
\item
  \textbf{Never route high-speed signals across ground plane splits.}
  Route around the split or use a different signal layer.
\item
  If a split is unavoidable, place \textbf{stitching capacitors} (100
  nF) across the split every 5-10 mm to provide a low-impedance AC
  return path.
\item
  Redesign the stackup so that the ground plane is continuous under all
  high-speed signals, or move the split to a power plane layer with
  decoupling capacitors bridging it.
\end{itemize}

\chapter{Chapter 10 --- Section 10.1: Switching
Devices}\label{chapter-10-section-10.1-switching-devices}

Practice problems covering power diodes, MOSFETs, IGBTs, thyristors,
wide-bandgap semiconductors (SiC and GaN), and gate driver design for
power electronic converters.

\begin{center}\rule{0.5\linewidth}{0.5pt}\end{center}

\section{Problem 10.1.1}\label{problem-10.1.1}

\textbf{Given:} A fast-recovery silicon power diode has a forward
voltage drop V\textsubscript{F} = 1.1 V and carries a pulsed forward
current with an RMS value of 18 A and an average value of 12 A. The
reverse recovery charge is Q\textsubscript{rr} = 800 nC, the DC bus
voltage is V\textsubscript{R} = 600 V, and the switching frequency is
f\textsubscript{sw} = 50 kHz.

\textbf{Find:} (a) The conduction power loss, (b) the reverse recovery
energy per switching event, (c) the reverse recovery power loss, and (d)
the total diode power loss.

\textbf{Solution:}

\begin{enumerate}
\def\labelenumi{(\alph{enumi})}
\item
  Conduction power loss (using average current for the
  V\textsubscript{F} drop model): P\textsubscript{cond} =
  V\textsubscript{F} × I\textsubscript{avg} = 1.1 × 12 = \textbf{13.2 W}
\item
  Reverse recovery energy per event: E\textsubscript{rr} = ½ ×
  Q\textsubscript{rr} × V\textsubscript{R} = 0.5 × 800 × 10⁻⁹ × 600 =
  \textbf{240 μJ}
\item
  Reverse recovery power loss: P\textsubscript{rr} = E\textsubscript{rr}
  × f\textsubscript{sw} = 240 × 10⁻⁶ × 50 × 10³ = \textbf{12.0 W}
\item
  Total diode power loss: P\textsubscript{total} = P\textsubscript{cond}
  + P\textsubscript{rr} = 13.2 + 12.0 = \textbf{25.2 W}
\end{enumerate}

\begin{center}\rule{0.5\linewidth}{0.5pt}\end{center}

\section{Problem 10.1.2}\label{problem-10.1.2}

\textbf{Given:} A power MOSFET with R\textsubscript{DS(on)} = 80 mΩ at
25°C has a temperature coefficient of +0.5\%/°C. It operates in a
synchronous buck converter at f\textsubscript{sw} = 400 kHz with a drain
current I\textsubscript{D} = 15 A, bus voltage V\textsubscript{DS} = 48
V, turn-on time t\textsubscript{on} = 18 ns, and turn-off time
t\textsubscript{off} = 25 ns. The junction temperature is 110°C and the
duty cycle is D = 0.35.

\textbf{Find:} (a) R\textsubscript{DS(on)} at 110°C, (b) the RMS current
through the MOSFET, (c) the conduction loss, (d) the switching loss, and
(e) the total MOSFET power loss.

\textbf{Solution:}

\begin{enumerate}
\def\labelenumi{(\alph{enumi})}
\item
  R\textsubscript{DS(on)} at 110°C: R\textsubscript{DS(on)} = 80 × {[}1
  + 0.005 × (110 − 25){]} = 80 × {[}1 + 0.425{]} = 80 × 1.425 =
  \textbf{114 mΩ}
\item
  RMS current through the high-side MOSFET: I\textsubscript{rms} =
  I\textsubscript{D} × √D = 15 × √0.35 = 15 × 0.5916 = \textbf{8.87 A}
\item
  Conduction loss: P\textsubscript{cond} = I\textsubscript{rms}² ×
  R\textsubscript{DS(on)} = 8.87² × 0.114 = 78.7 × 0.114 = \textbf{8.97
  W}
\item
  Switching loss: P\textsubscript{sw} = ½ × V\textsubscript{DS} ×
  I\textsubscript{D} × (t\textsubscript{on} + t\textsubscript{off}) ×
  f\textsubscript{sw} P\textsubscript{sw} = 0.5 × 48 × 15 × (18 + 25) ×
  10⁻⁹ × 400 × 10³ P\textsubscript{sw} = 0.5 × 48 × 15 × 43 × 10⁻⁹ × 4 ×
  10⁵ = \textbf{6.19 W}
\item
  Total MOSFET loss: P\textsubscript{total} = 8.97 + 6.19 =
  \textbf{15.16 W}
\end{enumerate}

\begin{center}\rule{0.5\linewidth}{0.5pt}\end{center}

\section{Problem 10.1.3}\label{problem-10.1.3}

\textbf{Given:} An IGBT module rated for 1,200 V / 300 A has
V\textsubscript{CE(sat)} = 2.4 V at rated current, turn-on energy
E\textsubscript{on} = 45 mJ, and turn-off energy E\textsubscript{off} =
35 mJ (at V\textsubscript{DC} = 600 V, I\textsubscript{C} = 300 A). The
module drives a three-phase motor at f\textsubscript{sw} = 8 kHz. The
modulation index is m\textsubscript{a} = 0.9 and the load power factor
angle is φ = 30°.

\textbf{Find:} (a) The average conduction loss per IGBT (using the
approximation for sinusoidal PWM), (b) the switching loss per IGBT, (c)
the total loss per IGBT, and (d) the total loss for all six IGBTs in the
three-phase bridge.

\textbf{Solution:}

\begin{enumerate}
\def\labelenumi{(\alph{enumi})}
\tightlist
\item
  Average conduction loss per IGBT (sinusoidal PWM approximation):
  P\textsubscript{cond} = V\textsubscript{CE(sat)} ×
  I\textsubscript{peak}/(2π) × (π/4 + m\textsubscript{a} × cos φ / 3)
\end{enumerate}

Using I\textsubscript{peak} = 300 A: P\textsubscript{cond} = 2.4 × 300 /
(2π) × (π/4 + 0.9 × cos 30° / 3) P\textsubscript{cond} = 2.4 × 300 /
6.283 × (0.7854 + 0.9 × 0.866 / 3) P\textsubscript{cond} = 114.6 ×
(0.7854 + 0.2598) P\textsubscript{cond} = 114.6 × 1.0452 = \textbf{119.8
W}

\begin{enumerate}
\def\labelenumi{(\alph{enumi})}
\setcounter{enumi}{1}
\item
  Switching loss per IGBT: P\textsubscript{sw} = (E\textsubscript{on} +
  E\textsubscript{off}) × f\textsubscript{sw} / π P\textsubscript{sw} =
  (45 + 35) × 10⁻³ × 8,000 / π = 80 × 10⁻³ × 8,000 / 3.1416
  P\textsubscript{sw} = 640 / 3.1416 = \textbf{203.7 W}
\item
  Total loss per IGBT: P\textsubscript{total} = 119.8 + 203.7 =
  \textbf{323.5 W}
\item
  Total loss for six IGBTs: P\textsubscript{6-IGBT} = 6 × 323.5 =
  \textbf{1,941 W}
\end{enumerate}

Note: This does not include the antiparallel diode losses, which
typically add 20--40\% to the total module loss.

\begin{center}\rule{0.5\linewidth}{0.5pt}\end{center}

\section{Problem 10.1.4}\label{problem-10.1.4}

\textbf{Given:} A thyristor-controlled single-phase full-wave bridge
rectifier is fed from a 240 V\textsubscript{rms}, 50 Hz source. The load
is a DC motor with back-EMF E = 180 V and armature resistance
R\textsubscript{a} = 0.5 Ω. The thyristor forward voltage drop is
V\textsubscript{T} = 1.5 V (two thyristors conduct at any time).

\textbf{Find:} (a) The firing angle α required to produce an average
motor current of 50 A, (b) the average output voltage, (c) the power
delivered to the motor, and (d) the total thyristor conduction losses.

\textbf{Solution:}

\begin{enumerate}
\def\labelenumi{(\alph{enumi})}
\tightlist
\item
  The average output voltage of a fully controlled single-phase bridge:
  V\textsubscript{dc} = (2V\textsubscript{peak}/π) × cos α
\end{enumerate}

The required V\textsubscript{dc} must overcome the back-EMF, armature
resistance drop, and thyristor drops: V\textsubscript{dc} = E +
I\textsubscript{a} × R\textsubscript{a} + 2V\textsubscript{T} = 180 + 50
× 0.5 + 2 × 1.5 = 180 + 25 + 3 = 208 V

V\textsubscript{peak} = 240 × √2 = 339.4 V 2V\textsubscript{peak}/π = 2
× 339.4 / π = 216.1 V

cos α = V\textsubscript{dc} / (2V\textsubscript{peak}/π) = 208 / 216.1 =
0.9625 α = cos⁻¹(0.9625) = \textbf{15.7°}

\begin{enumerate}
\def\labelenumi{(\alph{enumi})}
\setcounter{enumi}{1}
\item
  Average output voltage: V\textsubscript{dc} = \textbf{208 V}
\item
  Power delivered to the motor (mechanical plus armature heating):
  P\textsubscript{motor} = E × I\textsubscript{a} + I\textsubscript{a}²
  × R\textsubscript{a} = 180 × 50 + 50² × 0.5 = 9,000 + 1,250 =
  \textbf{10,250 W}
\item
  Total thyristor conduction losses (two thyristors in series):
  P\textsubscript{thyristor} = 2 × V\textsubscript{T} ×
  I\textsubscript{avg} = 2 × 1.5 × 50 = \textbf{150 W}
\end{enumerate}

\begin{center}\rule{0.5\linewidth}{0.5pt}\end{center}

\section{Problem 10.1.5}\label{problem-10.1.5}

\textbf{Given:} A 1,200 V silicon IGBT module has
V\textsubscript{CE(sat)} = 2.1 V and switching energies
E\textsubscript{on} = 12 mJ, E\textsubscript{off} = 8 mJ at 100 A. A
1,200 V SiC MOSFET has R\textsubscript{DS(on)} = 13 mΩ and switching
energies E\textsubscript{on} = 1.5 mJ, E\textsubscript{off} = 0.8 mJ at
100 A. Both devices operate at I\textsubscript{D} = 100 A with a duty
cycle of 0.5 in a DC-DC converter with V\textsubscript{bus} = 800 V.

\textbf{Find:} (a) The conduction loss for each device, (b) the
switching loss for each at f\textsubscript{sw} = 20 kHz, (c) the total
losses and efficiency improvement of SiC, and (d) the maximum switching
frequency at which the SiC device dissipates the same total power as the
Si IGBT at 20 kHz.

\textbf{Solution:}

\begin{enumerate}
\def\labelenumi{(\alph{enumi})}
\item
  Conduction losses (D = 0.5): Si IGBT: P\textsubscript{cond} =
  V\textsubscript{CE(sat)} × I\textsubscript{D} × D = 2.1 × 100 × 0.5 =
  \textbf{105 W} SiC MOSFET: I\textsubscript{rms} = I\textsubscript{D} ×
  √D = 100 × √0.5 = 70.7 A P\textsubscript{cond} = I\textsubscript{rms}²
  × R\textsubscript{DS(on)} = 70.7² × 0.013 = 5,000 × 0.013 =
  \textbf{65.0 W}
\item
  Switching losses at 20 kHz: Si IGBT: P\textsubscript{sw} = (12 + 8) ×
  10⁻³ × 20,000 = \textbf{400 W} SiC MOSFET: P\textsubscript{sw} = (1.5
  + 0.8) × 10⁻³ × 20,000 = \textbf{46.0 W}
\item
  Total losses: Si IGBT: P\textsubscript{total} = 105 + 400 =
  \textbf{505 W} SiC MOSFET: P\textsubscript{total} = 65.0 + 46.0 =
  \textbf{111.0 W}
\end{enumerate}

At P\textsubscript{out} = V\textsubscript{bus} × I\textsubscript{D} × D
= 800 × 100 × 0.5 = 40,000 W: Si IGBT efficiency: η = 40,000 / (40,000 +
505) = 98.75\% SiC efficiency: η = 40,000 / (40,000 + 111) = 99.72\%
\textbf{Efficiency improvement: 0.97 percentage points}

\begin{enumerate}
\def\labelenumi{(\alph{enumi})}
\setcounter{enumi}{3}
\tightlist
\item
  For SiC total loss = 505 W (same as Si IGBT at 20 kHz): 505 = 65.0 +
  (1.5 + 0.8) × 10⁻³ × f\textsubscript{sw} 440 = 2.3 × 10⁻³ ×
  f\textsubscript{sw} f\textsubscript{sw} = 440 / 0.0023 = \textbf{191.3
  kHz}
\end{enumerate}

The SiC device can switch at nearly 10× the frequency of the Si IGBT at
equal total loss.

\begin{center}\rule{0.5\linewidth}{0.5pt}\end{center}

\section{Problem 10.1.6}\label{problem-10.1.6}

\textbf{Given:} A bootstrap gate driver circuit drives the high-side
MOSFET in a half-bridge. The MOSFET has total gate charge
Q\textsubscript{g} = 85 nC and requires V\textsubscript{GS} ≥ 10 V. The
driver IC quiescent current is I\textsubscript{Q} = 3 mA, and the level
shifter leakage is I\textsubscript{leak} = 100 μA. The switching
frequency is f\textsubscript{sw} = 200 kHz with a maximum duty cycle of
95\%. The supply voltage is V\textsubscript{CC} = 12 V and the bootstrap
diode has V\textsubscript{F} = 0.7 V. The allowable bootstrap voltage
droop is ΔV = 0.8 V.

\textbf{Find:} (a) The maximum high-side on-time, (b) the total charge
drawn from the bootstrap capacitor per cycle, (c) the minimum bootstrap
capacitance, (d) the effective gate drive voltage at end of the
high-side on-time, and (e) whether the design provides adequate gate
drive voltage.

\textbf{Solution:}

\begin{enumerate}
\def\labelenumi{(\alph{enumi})}
\item
  Maximum high-side on-time: T\textsubscript{sw} = 1/f\textsubscript{sw}
  = 1/200,000 = 5 μs t\textsubscript{on(max)} = D\textsubscript{max} ×
  T\textsubscript{sw} = 0.95 × 5 = \textbf{4.75 μs}
\item
  Total charge per cycle: Q\textsubscript{total} = Q\textsubscript{g} +
  (I\textsubscript{Q} + I\textsubscript{leak}) ×
  t\textsubscript{on(max)} Q\textsubscript{total} = 85 × 10⁻⁹ + (3 ×
  10⁻³ + 100 × 10⁻⁶) × 4.75 × 10⁻⁶ Q\textsubscript{total} = 85 nC + 3.1
  × 10⁻³ × 4.75 × 10⁻⁶ = 85 nC + 14.7 nC = \textbf{99.7 nC}
\item
  Minimum bootstrap capacitance: C\textsubscript{boot} =
  Q\textsubscript{total} / ΔV = 99.7 × 10⁻⁹ / 0.8 = \textbf{124.6 nF}
\end{enumerate}

Use a standard 220 nF ceramic capacitor for margin.

\begin{enumerate}
\def\labelenumi{(\alph{enumi})}
\setcounter{enumi}{3}
\tightlist
\item
  Bootstrap voltage when fully charged: V\textsubscript{boot} =
  V\textsubscript{CC} − V\textsubscript{F} = 12 − 0.7 = 11.3 V
\end{enumerate}

Voltage at end of on-time (using 220 nF): V\textsubscript{boot,end} =
11.3 − Q\textsubscript{total}/C = 11.3 − 99.7 × 10⁻⁹ / 220 × 10⁻⁹ = 11.3
− 0.45 = \textbf{10.85 V}

\begin{enumerate}
\def\labelenumi{(\alph{enumi})}
\setcounter{enumi}{4}
\tightlist
\item
  Since V\textsubscript{boot,end} = 10.85 V \textgreater{}
  V\textsubscript{GS(min)} = 10 V, the design provides adequate gate
  drive voltage with a margin of \textbf{0.85 V}. This is a slim margin;
  using a 470 nF capacitor would give V\textsubscript{boot,end} = 11.3 −
  0.21 = 11.09 V with a more comfortable 1.09 V margin.
\end{enumerate}

\begin{center}\rule{0.5\linewidth}{0.5pt}\end{center}

\section{Problem 10.1.7}\label{problem-10.1.7}

\textbf{Given:} A SiC Schottky diode (V\textsubscript{F} = 1.35 V,
Q\textsubscript{rr} ≈ 0) and a silicon fast-recovery diode
(V\textsubscript{F} = 0.95 V, t\textsubscript{rr} = 35 ns) both operate
as the freewheeling diode in a boost converter with V\textsubscript{bus}
= 400 V, I\textsubscript{avg} = 8 A, I\textsubscript{peak} = 10 A, and
f\textsubscript{sw} = 200 kHz. The silicon diode's reverse recovery
current peak is estimated as I\textsubscript{rr} ≈ 0.5 ×
I\textsubscript{peak} = 5 A.

\textbf{Find:} (a) The conduction loss for each diode, (b) the reverse
recovery loss for the silicon diode, (c) the total loss for each diode,
and (d) the percentage reduction in total diode loss using SiC.

\textbf{Solution:}

\begin{enumerate}
\def\labelenumi{(\alph{enumi})}
\item
  Conduction losses: SiC: P\textsubscript{cond} = V\textsubscript{F} ×
  I\textsubscript{avg} = 1.35 × 8 = \textbf{10.8 W} Si:
  P\textsubscript{cond} = V\textsubscript{F} × I\textsubscript{avg} =
  0.95 × 8 = \textbf{7.6 W}
\item
  Silicon diode reverse recovery loss: Q\textsubscript{rr} = ½ ×
  I\textsubscript{rr} × t\textsubscript{rr} = 0.5 × 5 × 35 × 10⁻⁹ = 87.5
  nC E\textsubscript{rr} = ½ × Q\textsubscript{rr} × V\textsubscript{R}
  = 0.5 × 87.5 × 10⁻⁹ × 400 = 17.5 μJ P\textsubscript{rr} =
  E\textsubscript{rr} × f\textsubscript{sw} = 17.5 × 10⁻⁶ × 200 × 10³ =
  \textbf{3.5 W}
\item
  Total losses: SiC: P\textsubscript{total} = 10.8 + 0 = \textbf{10.8 W}
  Si: P\textsubscript{total} = 7.6 + 3.5 = \textbf{11.1 W}
\item
  Reduction: (11.1 − 10.8) / 11.1 × 100 = \textbf{2.7\%}
\end{enumerate}

At this frequency, the SiC advantage is modest. At 500 kHz, the Si
recovery loss would be 8.75 W, making total Si loss 16.35 W versus 10.8
W for SiC --- a 34\% reduction, illustrating the growing SiC advantage
at higher frequencies.

\begin{center}\rule{0.5\linewidth}{0.5pt}\end{center}

\section{Problem 10.1.8}\label{problem-10.1.8}

\textbf{Given:} A half-bridge gate driver must provide dead time to
prevent shoot-through. The high-side MOSFET has turn-off delay
t\textsubscript{d(off)} = 40 ns and fall time t\textsubscript{f} = 18
ns. The low-side MOSFET has turn-on delay t\textsubscript{d(on)} = 25 ns
and rise time t\textsubscript{r} = 12 ns. The gate driver propagation
delay mismatch between channels is ±8 ns. The switching frequency is 500
kHz and V\textsubscript{bus} = 48 V.

\textbf{Find:} (a) The minimum dead time required to prevent
shoot-through, (b) the recommended dead time with a 20\% safety margin,
(c) the body diode conduction time during dead time, (d) the body diode
conduction loss if I\textsubscript{load} = 20 A and V\textsubscript{SD}
= 0.8 V, and (e) the dead time loss as a percentage of output power at
24 V output.

\textbf{Solution:}

\begin{enumerate}
\def\labelenumi{(\alph{enumi})}
\tightlist
\item
  Minimum dead time must ensure the turning-off device is fully off
  before the turning-on device starts conducting. Worst case: the
  turn-off command arrives late (−8 ns mismatch) and the turn-on command
  arrives early (+8 ns mismatch).
\end{enumerate}

t\textsubscript{dead(min)} = t\textsubscript{d(off)} +
t\textsubscript{f} + 2 × Δt\textsubscript{prop} − t\textsubscript{d(on)}
t\textsubscript{dead(min)} = 40 + 18 + 16 − 25 = \textbf{49 ns}

\begin{enumerate}
\def\labelenumi{(\alph{enumi})}
\setcounter{enumi}{1}
\item
  Recommended dead time with 20\% margin: t\textsubscript{dead} = 49 ×
  1.20 = \textbf{58.8 ns} → use 60 ns
\item
  Body diode conduction time ≈ dead time: t\textsubscript{BD} =
  \textbf{60 ns} (per transition, two transitions per switching cycle)
\item
  Body diode conduction loss: P\textsubscript{BD} = V\textsubscript{SD}
  × I\textsubscript{load} × 2 × t\textsubscript{BD} ×
  f\textsubscript{sw} P\textsubscript{BD} = 0.8 × 20 × 2 × 60 × 10⁻⁹ ×
  500 × 10³ = \textbf{0.96 W}
\item
  Output power: P\textsubscript{out} = V\textsubscript{out} ×
  I\textsubscript{load} = 24 × 20 = 480 W Dead time loss as percentage:
  0.96 / 480 × 100 = \textbf{0.20\%}
\end{enumerate}

\begin{center}\rule{0.5\linewidth}{0.5pt}\end{center}

\section{Problem 10.1.9}\label{problem-10.1.9}

\textbf{Given:} A GaN HEMT (650 V, R\textsubscript{DS(on)} = 55 mΩ,
C\textsubscript{oss} = 45 pF, Q\textsubscript{g} = 6.2 nC) and a silicon
superjunction MOSFET (650 V, R\textsubscript{DS(on)} = 120 mΩ,
C\textsubscript{oss} = 80 pF, Q\textsubscript{g} = 52 nC) are compared
for a 400 V, 600 W totem-pole PFC application at f\textsubscript{sw} =
500 kHz. The RMS switch current is 2.5 A. Gate drive voltage is 6 V for
GaN and 12 V for Si.

\textbf{Find:} (a) The conduction loss for each device, (b) the output
capacitance (C\textsubscript{oss}) switching loss for each, (c) the gate
drive loss for each, (d) the total device loss for each, and (e) the
efficiency impact at 600 W output.

\textbf{Solution:}

\begin{enumerate}
\def\labelenumi{(\alph{enumi})}
\item
  Conduction losses: GaN: P\textsubscript{cond} = I\textsubscript{rms}²
  × R\textsubscript{DS(on)} = 2.5² × 0.055 = 6.25 × 0.055 =
  \textbf{0.344 W} Si: P\textsubscript{cond} = 2.5² × 0.120 = 6.25 ×
  0.120 = \textbf{0.750 W}
\item
  C\textsubscript{oss} switching loss: GaN: P\textsubscript{oss} = ½ ×
  C\textsubscript{oss} × V² × f\textsubscript{sw} = 0.5 × 45 × 10⁻¹² ×
  400² × 500 × 10³ P\textsubscript{oss} = 0.5 × 45 × 10⁻¹² × 160,000 ×
  500,000 = \textbf{1.80 W} Si: P\textsubscript{oss} = 0.5 × 80 × 10⁻¹²
  × 160,000 × 500,000 = \textbf{3.20 W}
\item
  Gate drive loss: GaN: P\textsubscript{gate} = Q\textsubscript{g} ×
  V\textsubscript{GS} × f\textsubscript{sw} = 6.2 × 10⁻⁹ × 6 × 500 × 10³
  = \textbf{0.019 W} Si: P\textsubscript{gate} = 52 × 10⁻⁹ × 12 × 500 ×
  10³ = \textbf{0.312 W}
\item
  Total device losses: GaN: P\textsubscript{total} = 0.344 + 1.80 +
  0.019 = \textbf{2.16 W} Si: P\textsubscript{total} = 0.750 + 3.20 +
  0.312 = \textbf{4.26 W}
\item
  Efficiency impact: GaN: η = 600 / (600 + 2.16) = 99.64\% Si: η = 600 /
  (600 + 4.26) = 99.29\% \textbf{Efficiency improvement with GaN: 0.35
  percentage points}
\end{enumerate}

The GaN device reduces total losses by 49\%, enabling higher power
density or elimination of the heat sink in a compact charger design.

\begin{center}\rule{0.5\linewidth}{0.5pt}\end{center}

\section{Problem 10.1.10}\label{problem-10.1.10}

\textbf{Given:} An IGBT-based three-phase inverter drives a 150 kW
induction motor. The DC bus is 650 V. Each IGBT has
V\textsubscript{CE(sat)} = 1.8 V at 200 A, E\textsubscript{on} = 18 mJ,
and E\textsubscript{off} = 14 mJ. The antiparallel diode has
V\textsubscript{F} = 1.2 V and E\textsubscript{rr} = 8 mJ. The motor
power factor is cos φ = 0.87 (lagging) and the modulation index is
m\textsubscript{a} = 0.92. The switching frequency is
f\textsubscript{sw} = 10 kHz and the peak phase current is
I\textsubscript{peak} = 200 A.

\textbf{Find:} (a) The conduction loss per IGBT, (b) the switching loss
per IGBT, (c) the diode conduction loss per antiparallel diode, (d) the
diode reverse recovery loss, and (e) the total inverter loss for all six
IGBT/diode pairs.

\textbf{Solution:}

\begin{enumerate}
\def\labelenumi{(\alph{enumi})}
\item
  IGBT conduction loss (sinusoidal PWM approximation per IGBT):
  P\textsubscript{cond,IGBT} = V\textsubscript{CE(sat)} ×
  I\textsubscript{peak} / (2π) × (π/4 + m\textsubscript{a} × cos φ / 3)
  P\textsubscript{cond,IGBT} = 1.8 × 200 / (2π) × (0.7854 + 0.92 × 0.87
  / 3) P\textsubscript{cond,IGBT} = 57.30 × (0.7854 + 0.2668)
  P\textsubscript{cond,IGBT} = 57.30 × 1.0522 = \textbf{60.3 W}
\item
  IGBT switching loss: P\textsubscript{sw,IGBT} = (E\textsubscript{on} +
  E\textsubscript{off}) × f\textsubscript{sw} / π
  P\textsubscript{sw,IGBT} = (18 + 14) × 10⁻³ × 10,000 / π = 320 / π =
  \textbf{101.9 W}
\item
  Diode conduction loss (sinusoidal PWM, antiparallel diode conducts
  during the complementary interval): P\textsubscript{cond,D} =
  V\textsubscript{F} × I\textsubscript{peak} / (2π) × (π/4 −
  m\textsubscript{a} × cos φ / 3) P\textsubscript{cond,D} = 1.2 × 200 /
  (2π) × (0.7854 − 0.2668) P\textsubscript{cond,D} = 38.20 × 0.5186 =
  \textbf{19.8 W}
\item
  Diode reverse recovery loss: P\textsubscript{rr} = E\textsubscript{rr}
  × f\textsubscript{sw} / π = 8 × 10⁻³ × 10,000 / π = 80 / π =
  \textbf{25.5 W}
\item
  Total inverter loss (6 IGBTs + 6 diodes): P\textsubscript{total} = 6 ×
  (P\textsubscript{cond,IGBT} + P\textsubscript{sw,IGBT} +
  P\textsubscript{cond,D} + P\textsubscript{rr}) P\textsubscript{total}
  = 6 × (60.3 + 101.9 + 19.8 + 25.5) = 6 × 207.5 = \textbf{1,245 W}
\end{enumerate}

Inverter efficiency: η = 150,000 / (150,000 + 1,245) = \textbf{99.18\%}

\chapter{Chapter 10 --- Section 10.2:
Rectifiers}\label{chapter-10-section-10.2-rectifiers}

Practice problems covering single-phase rectifiers, three-phase
rectifiers, controlled rectifiers with thyristors, rectifier harmonics,
multi-pulse configurations, and input filtering.

\begin{center}\rule{0.5\linewidth}{0.5pt}\end{center}

\section{Problem 10.2.1}\label{problem-10.2.1}

\textbf{Given:} A single-phase full-wave bridge rectifier with a
capacitor filter is fed from a 230 V\textsubscript{rms}, 50 Hz source.
The load draws 2 A DC. The filter capacitor is C = 1,000 μF, and the
diode forward voltage drop is V\textsubscript{F} = 0.8 V per diode.

\textbf{Find:} (a) The peak rectified voltage (accounting for diode
drops), (b) the ideal no-load DC output voltage, (c) the peak-to-peak
ripple voltage, (d) the ripple factor, and (e) the minimum instantaneous
output voltage.

\textbf{Solution:}

\begin{enumerate}
\def\labelenumi{(\alph{enumi})}
\item
  Peak rectified voltage: V\textsubscript{peak} =
  V\textsubscript{in,peak} − 2V\textsubscript{F} = 230 × √2 − 2 × 0.8 =
  325.3 − 1.6 = \textbf{323.7 V}
\item
  Ideal no-load DC output (with capacitor, output ≈ peak):
  V\textsubscript{dc} ≈ V\textsubscript{peak} = \textbf{323.7 V}
\item
  Peak-to-peak ripple voltage (for full-wave rectifier):
  V\textsubscript{ripple} = I\textsubscript{dc} / (2 × f × C) = 2 / (2 ×
  50 × 1,000 × 10⁻⁶) = 2 / 0.1 = \textbf{20.0 V}
\item
  Ripple factor (for a sawtooth approximation,
  V\textsubscript{ripple(rms)} = V\textsubscript{ripple(pp)} / 2√3):
  V\textsubscript{ripple(rms)} = 20.0 / 3.464 = 5.77 V r =
  V\textsubscript{ripple(rms)} / V\textsubscript{dc} × 100 = 5.77 /
  (323.7 − 10.0) × 100 = 5.77 / 313.7 × 100 = \textbf{1.84\%}
\item
  Minimum instantaneous output voltage: V\textsubscript{min} =
  V\textsubscript{peak} − V\textsubscript{ripple} = 323.7 − 20.0 =
  \textbf{303.7 V}
\end{enumerate}

\begin{center}\rule{0.5\linewidth}{0.5pt}\end{center}

\section{Problem 10.2.2}\label{problem-10.2.2}

\textbf{Given:} A three-phase six-pulse diode bridge rectifier is fed
from a 400 V\textsubscript{rms} line-to-line, 60 Hz source. The DC
output feeds a 50 Ω resistive load through an LC filter (L = 10 mH, C =
500 μF).

\textbf{Find:} (a) The average DC output voltage, (b) the ripple
frequency, (c) the DC load current, (d) the approximate peak-to-peak
ripple voltage without the LC filter, and (e) the attenuation provided
by the LC filter at the ripple frequency.

\textbf{Solution:}

\begin{enumerate}
\def\labelenumi{(\alph{enumi})}
\item
  Average DC output voltage for a six-pulse rectifier:
  V\textsubscript{dc} = 1.35 × V\textsubscript{rms(L-L)} = 1.35 × 400 =
  \textbf{540 V}
\item
  Ripple frequency: f\textsubscript{ripple} = 6 × f\textsubscript{line}
  = 6 × 60 = \textbf{360 Hz}
\item
  DC load current: I\textsubscript{dc} = V\textsubscript{dc} / R = 540 /
  50 = \textbf{10.8 A}
\item
  Peak-to-peak ripple without filter (approximately 4.2\% of
  V\textsubscript{dc} for six-pulse): V\textsubscript{ripple(pp)} =
  0.042 × 540 = \textbf{22.7 V}
\item
  LC filter attenuation at 360 Hz: The resonant frequency: f₀ =
  1/(2π√(LC)) = 1/(2π√(10 × 10⁻³ × 500 × 10⁻⁶)) = 1/(2π√(5 × 10⁻⁶)) =
  1/(2π × 2.236 × 10⁻³) = 71.2 Hz
\end{enumerate}

Attenuation at 360 Hz: A = (f\textsubscript{ripple}/f₀)² = (360/71.2)² =
5.056² = 25.6 In dB: 20 × log₁₀(25.6) = \textbf{28.2 dB}

Filtered ripple: V\textsubscript{ripple,filtered} = 22.7 / 25.6 =
\textbf{0.89 V} peak-to-peak

\begin{center}\rule{0.5\linewidth}{0.5pt}\end{center}

\section{Problem 10.2.3}\label{problem-10.2.3}

\textbf{Given:} A single-phase full-wave diode bridge rectifier with a
capacitor filter draws 5 A\textsubscript{rms} from a 120
V\textsubscript{rms}, 60 Hz source. The fundamental current component is
3.2 A\textsubscript{rms}. The harmonic currents are: I₃ = 2.8 A, I₅ =
2.1 A, I₇ = 1.4 A, I₉ = 0.9 A, I₁₁ = 0.5 A.

\textbf{Find:} (a) The current THD, (b) the displacement power factor
(assuming current fundamental is in phase with voltage), (c) the true
power factor, (d) the apparent power, and (e) the real power.

\textbf{Solution:}

\begin{enumerate}
\def\labelenumi{(\alph{enumi})}
\item
  Current THD: I\textsubscript{harm} = √(2.8² + 2.1² + 1.4² + 0.9² +
  0.5²) = √(7.84 + 4.41 + 1.96 + 0.81 + 0.25) = √15.27 = 3.91 A THD =
  I\textsubscript{harm} / I₁ × 100 = 3.91 / 3.2 × 100 = \textbf{122.2\%}
\item
  Displacement power factor: Since the fundamental current is in phase
  with the voltage: DPF = cos φ₁ = \textbf{1.00}
\item
  True power factor: PF = DPF × I₁ / I\textsubscript{rms} = 1.00 × 3.2 /
  5.0 = \textbf{0.640}
\end{enumerate}

Alternatively: PF = 1 / √(1 + THD²) = 1 / √(1 + 1.222²) = 1 / √(2.493) =
1 / 1.579 = 0.633 Using the measured RMS current: PF = I₁ × DPF /
I\textsubscript{rms} = 3.2 × 1.0 / 5.0 = \textbf{0.640}

\begin{enumerate}
\def\labelenumi{(\alph{enumi})}
\setcounter{enumi}{3}
\item
  Apparent power: S = V\textsubscript{rms} × I\textsubscript{rms} = 120
  × 5.0 = \textbf{600 VA}
\item
  Real power: P = S × PF = 600 × 0.640 = \textbf{384 W}
\end{enumerate}

\begin{center}\rule{0.5\linewidth}{0.5pt}\end{center}

\section{Problem 10.2.4}\label{problem-10.2.4}

\textbf{Given:} A three-phase fully controlled thyristor bridge
rectifier is fed from a 480 V\textsubscript{rms} line-to-line, 60 Hz
source. The DC output feeds an electroplating load that requires a
regulated 450 V DC at 200 A.

\textbf{Find:} (a) The firing angle α required, (b) the power delivered
to the load, (c) the reactive power drawn from the source, (d) the power
factor, and (e) the thyristor voltage rating with a 2.0× safety factor.

\textbf{Solution:}

\begin{enumerate}
\def\labelenumi{(\alph{enumi})}
\item
  For a six-pulse controlled rectifier: V\textsubscript{dc} = 1.35 ×
  V\textsubscript{LL} × cos α 450 = 1.35 × 480 × cos α = 648 × cos α cos
  α = 450 / 648 = 0.6944 α = cos⁻¹(0.6944) = \textbf{46.1°}
\item
  Power delivered to the load: P\textsubscript{dc} = V\textsubscript{dc}
  × I\textsubscript{dc} = 450 × 200 = \textbf{90,000 W = 90 kW}
\item
  Reactive power (fundamental displacement): The displacement power
  factor for a controlled rectifier: DPF ≈ cos α = 0.6944 Q = P ×
  tan(cos⁻¹(DPF)) = 90,000 × tan(46.1°) = 90,000 × 1.038 =
  \textbf{93,420 VAR = 93.4 kVAR}
\item
  Power factor (including harmonics): For a six-pulse bridge, the
  fundamental current ratio: I₁/I\textsubscript{rms} ≈ 3/π = 0.955 PF =
  (I₁/I\textsubscript{rms}) × DPF = 0.955 × 0.6944 = \textbf{0.663}
\item
  Peak reverse voltage on thyristor = √2 × V\textsubscript{LL} = √2 ×
  480 = 678.8 V With 2.0× safety factor: V\textsubscript{rating} = 2.0 ×
  678.8 = \textbf{1,357.6 V} → select 1,400 V or 1,600 V rated
  thyristors
\end{enumerate}

\begin{center}\rule{0.5\linewidth}{0.5pt}\end{center}

\section{Problem 10.2.5}\label{problem-10.2.5}

\textbf{Given:} A 12-pulse rectifier uses two six-pulse bridges with a
delta-wye / delta-delta transformer arrangement. The input supply is
13.8 kV\textsubscript{rms} line-to-line, 60 Hz. The total DC output is
10,000 V at 500 A. Each bridge carries half the load current.

\textbf{Find:} (a) The total DC output power, (b) the transformer
secondary voltages required, (c) the harmonic orders present at the
12-pulse input, (d) the expected THD compared to a single six-pulse
bridge, and (e) the fundamental input current from the 13.8 kV supply.

\textbf{Solution:}

\begin{enumerate}
\def\labelenumi{(\alph{enumi})}
\item
  Total DC output power: P\textsubscript{dc} = V\textsubscript{dc} ×
  I\textsubscript{dc} = 10,000 × 500 = \textbf{5,000,000 W = 5 MW}
\item
  Each bridge produces half the total voltage:
  V\textsubscript{dc,bridge} = 10,000 / 2 = 5,000 V For a six-pulse
  diode bridge: V\textsubscript{dc} = 1.35 × V\textsubscript{LL,sec}
  V\textsubscript{LL,sec} = 5,000 / 1.35 = \textbf{3,704 V} per
  secondary winding
\item
  For a 12-pulse rectifier, the 5th and 7th harmonics cancel. Remaining
  characteristic harmonics: h = 12k ± 1 → \textbf{11th, 13th, 23rd,
  25th, 35th, 37th, \ldots{}}
\item
  Six-pulse THD ≈ √{[}(1/5)² + (1/7)² + (1/11)² + (1/13)²{]} × 100 =
  √{[}0.04 + 0.0204 + 0.00826 + 0.00592{]} × 100 = √0.0746 × 100 =
  27.3\%
\end{enumerate}

12-pulse THD ≈ √{[}(1/11)² + (1/13)² + (1/23)² + (1/25)²{]} × 100 =
√{[}0.00826 + 0.00592 + 0.00189 + 0.0016{]} × 100 = √0.01767 × 100 =
\textbf{13.3\%}

THD reduction: (27.3 − 13.3) / 27.3 × 100 = \textbf{51\% reduction}

\begin{enumerate}
\def\labelenumi{(\alph{enumi})}
\setcounter{enumi}{4}
\tightlist
\item
  Fundamental input current (assuming 95\% transformer efficiency):
  P\textsubscript{in} = P\textsubscript{dc} / η = 5,000,000 / 0.95 =
  5,263,158 W I₁ = P\textsubscript{in} / (√3 × V\textsubscript{LL} ×
  DPF) = 5,263,158 / (√3 × 13,800 × 1.0) = 5,263,158 / 23,901 =
  \textbf{220.2 A}
\end{enumerate}

\begin{center}\rule{0.5\linewidth}{0.5pt}\end{center}

\section{Problem 10.2.6}\label{problem-10.2.6}

\textbf{Given:} A single-phase half-wave controlled rectifier (one
thyristor, one freewheeling diode) supplies an RL load with R = 5 Ω and
L = 50 mH from a 120 V\textsubscript{rms}, 60 Hz source. The firing
angle is α = 60°.

\textbf{Find:} (a) The peak source voltage, (b) the average output
voltage, (c) the average load current, (d) the load time constant, and
(e) whether the current is continuous or discontinuous.

\textbf{Solution:}

\begin{enumerate}
\def\labelenumi{(\alph{enumi})}
\item
  Peak source voltage: V\textsubscript{peak} = 120 × √2 = \textbf{169.7
  V}
\item
  For a half-wave controlled rectifier with a freewheeling diode (R-L
  load, continuous conduction assumed): V\textsubscript{dc} =
  (V\textsubscript{peak}/(2π)) × (1 + cos α) = (169.7 / 6.283) × (1 +
  cos 60°) V\textsubscript{dc} = 27.01 × (1 + 0.5) = 27.01 × 1.5 =
  \textbf{40.5 V}
\item
  Average load current: I\textsubscript{dc} = V\textsubscript{dc} / R =
  40.5 / 5 = \textbf{8.1 A}
\item
  Load time constant: τ = L / R = 50 × 10⁻³ / 5 = \textbf{10 ms}
\item
  The source period is T = 1/60 = 16.67 ms. The time constant τ = 10 ms
  is comparable to the half-period (8.33 ms). Since τ \textgreater{}
  T/2, the inductor maintains significant current during the
  freewheeling interval. The current is \textbf{continuous} --- the
  freewheeling diode conducts during the interval when the thyristor
  blocks, maintaining inductor current flow.
\end{enumerate}

\begin{center}\rule{0.5\linewidth}{0.5pt}\end{center}

\section{Problem 10.2.7}\label{problem-10.2.7}

\textbf{Given:} A passive 5th harmonic tuned LC filter is designed for a
six-pulse rectifier drawing 400 A fundamental at 480 V, 60 Hz. The 5th
harmonic current is I₅ = 80 A (≈ I₁/5). The tuned frequency is f₅ = 300
Hz. The filter quality factor is Q = 50.

\textbf{Find:} (a) The filter resonant impedance, (b) the required
capacitor value if the filter inductor is L = 0.5 mH, (c) the capacitor
voltage rating, (d) the reactive power supplied by the capacitor at 60
Hz, and (e) the 5th harmonic voltage at the filter terminals.

\textbf{Solution:}

\begin{enumerate}
\def\labelenumi{(\alph{enumi})}
\item
  Characteristic impedance and resonant impedance: X\textsubscript{L} at
  300 Hz: X\textsubscript{L} = 2π × 300 × 0.5 × 10⁻³ = 0.9425 Ω At
  resonance, X\textsubscript{L} = X\textsubscript{C} = 0.9425 Ω Resonant
  impedance: Z\textsubscript{res} = X\textsubscript{L} / Q = 0.9425 / 50
  = \textbf{0.01885 Ω}
\item
  Required capacitance: X\textsubscript{C} at 300 Hz = 0.9425 Ω C =
  1/(2π × f₅ × X\textsubscript{C}) = 1/(2π × 300 × 0.9425) = 1/1,776.3 =
  \textbf{562.9 μF}
\item
  Capacitor voltage at 60 Hz: X\textsubscript{C} at 60 Hz = 1/(2π × 60 ×
  562.9 × 10⁻⁶) = 1/0.21225 = 4.711 Ω Fundamental current through filter
  ≈ V\textsubscript{source} / (X\textsubscript{L,60} −
  X\textsubscript{C,60}) X\textsubscript{L} at 60 Hz = 2π × 60 × 0.5 ×
  10⁻³ = 0.1885 Ω Net impedance at 60 Hz: X\textsubscript{net} = 0.1885
  − 4.711 = −4.522 Ω (capacitive) Filter current at 60 Hz:
  I\textsubscript{f,60} = V\textsubscript{phase} /
  \textbar X\textsubscript{net}\textbar{} = (480/√3) / 4.522 = 277.1 /
  4.522 = 61.3 A V\textsubscript{cap} = I\textsubscript{f,60} ×
  X\textsubscript{C,60} = 61.3 × 4.711 = 288.8 V\textsubscript{rms}
  Peak: V\textsubscript{cap,peak} = 288.8 × √2 = \textbf{408.3 V} →
  select capacitor rated at \textbf{500 V} minimum
\item
  Reactive power supplied by the capacitor at 60 Hz:
  Q\textsubscript{var} = V\textsubscript{cap} × I\textsubscript{f,60} =
  288.8 × 61.3 = \textbf{17,703 VAR = 17.7 kVAR} per phase
\item
  5th harmonic voltage at filter terminals: V₅ = I₅ ×
  Z\textsubscript{res} = 80 × 0.01885 = \textbf{1.51 V}
\end{enumerate}

The tuned filter provides a near short-circuit path for the 5th
harmonic, reducing the 5th harmonic voltage to a negligible level.

\begin{center}\rule{0.5\linewidth}{0.5pt}\end{center}

\section{Problem 10.2.8}\label{problem-10.2.8}

\textbf{Given:} A three-phase six-pulse controlled rectifier operates
from 575 V\textsubscript{rms} line-to-line, 60 Hz. The rectifier
supplies a DC bus at 600 V for an industrial motor drive. The DC current
is 120 A. Each thyristor has V\textsubscript{T} = 1.5 V.

\textbf{Find:} (a) The required firing angle, (b) the total thyristor
losses (two thyristors conduct at any instant), (c) the AC line current
(RMS), (d) the apparent power drawn from the source, and (e) the power
factor.

\textbf{Solution:}

\begin{enumerate}
\def\labelenumi{(\alph{enumi})}
\item
  Firing angle: V\textsubscript{dc} = 1.35 × V\textsubscript{LL} × cos α
  600 = 1.35 × 575 × cos α = 776.3 × cos α cos α = 600 / 776.3 = 0.7729
  α = cos⁻¹(0.7729) = \textbf{39.4°}
\item
  Total thyristor losses: P\textsubscript{thyristor} = 2 ×
  V\textsubscript{T} × I\textsubscript{dc} = 2 × 1.5 × 120 = \textbf{360
  W}
\item
  AC line current (for a six-pulse bridge, assuming rectangular current
  blocks): I\textsubscript{ac(rms)} = I\textsubscript{dc} × √(2/3) = 120
  × 0.8165 = \textbf{98.0 A}
\item
  Apparent power: S = √3 × V\textsubscript{LL} × I\textsubscript{ac} =
  √3 × 575 × 98.0 = 97,575 VA = \textbf{97.6 kVA}
\item
  Real power: P = V\textsubscript{dc} × I\textsubscript{dc} +
  P\textsubscript{thyristor} = 600 × 120 + 360 = 72,360 W = 72.4 kW PF =
  P / S = 72,360 / 97,575 = \textbf{0.742}
\end{enumerate}

\begin{center}\rule{0.5\linewidth}{0.5pt}\end{center}

\section{Problem 10.2.9}\label{problem-10.2.9}

\textbf{Given:} An IEEE 519 analysis requires evaluating a facility's
harmonic compliance at the point of common coupling (PCC). The facility
has a six-pulse rectifier load drawing 500 A fundamental at 480 V. The
utility short-circuit current at the PCC is I\textsubscript{SC} = 15,000
A. IEEE 519 limits for I\textsubscript{SC}/I\textsubscript{L}
\textgreater{} 20 are: I₅ \textless{} 12\%, I₇ \textless{} 8.5\%, I₁₁
\textless{} 5.5\%, TDD \textless{} 15\%.

\textbf{Find:} (a) The I\textsubscript{SC}/I\textsubscript{L} ratio, (b)
the actual harmonic currents (using I\textsubscript{h} ≈ I₁/h), (c) the
harmonic currents as percentages of I\textsubscript{L}, (d) whether the
facility is compliant with IEEE 519, and (e) the total demand distortion
(TDD).

\textbf{Solution:}

\begin{enumerate}
\def\labelenumi{(\alph{enumi})}
\item
  I\textsubscript{SC}/I\textsubscript{L} ratio:
  I\textsubscript{SC}/I\textsubscript{L} = 15,000 / 500 = \textbf{30}
  (use the \textgreater{} 20 column in IEEE 519 Table 2)
\item
  Actual harmonic currents (I\textsubscript{h} ≈ I₁/h): I₅ = 500/5 = 100
  A I₇ = 500/7 = 71.4 A I₁₁ = 500/11 = 45.5 A I₁₃ = 500/13 = 38.5 A
\item
  As percentages of I\textsubscript{L} (= I₁ = 500 A):
  I₅/I\textsubscript{L} = 100/500 × 100 = \textbf{20.0\%}
  I₇/I\textsubscript{L} = 71.4/500 × 100 = \textbf{14.3\%}
  I₁₁/I\textsubscript{L} = 45.5/500 × 100 = \textbf{9.1\%}
\item
  Compliance check: I₅ = 20.0\% \textgreater{} 12\% limit →
  \textbf{NON-COMPLIANT} I₇ = 14.3\% \textgreater{} 8.5\% limit →
  \textbf{NON-COMPLIANT} I₁₁ = 9.1\% \textgreater{} 5.5\% limit →
  \textbf{NON-COMPLIANT}
\item
  TDD: TDD = √(I₅² + I₇² + I₁₁² + I₁₃² + \ldots) / I\textsubscript{L} ×
  100 TDD = √(100² + 71.4² + 45.5² + 38.5²) / 500 × 100 TDD = √(10,000 +
  5,098 + 2,070 + 1,482) / 500 × 100 TDD = √18,650 / 500 × 100 = 136.6 /
  500 × 100 = \textbf{27.3\%} \textgreater{} 15\% limit →
  \textbf{NON-COMPLIANT}
\end{enumerate}

The facility requires a 12-pulse upgrade or active harmonic filter to
achieve compliance.

\begin{center}\rule{0.5\linewidth}{0.5pt}\end{center}

\section{Problem 10.2.10}\label{problem-10.2.10}

\textbf{Given:} A three-phase diode bridge rectifier with capacitor
filter draws current from a 208 V\textsubscript{rms}, 60 Hz source. The
load is 2 kW. The bridge uses Schottky diodes with V\textsubscript{F} =
0.45 V each. The DC bus capacitor is 4,700 μF.

\textbf{Find:} (a) The DC output voltage, (b) the DC load current, (c)
the peak-to-peak ripple voltage on the DC bus, (d) the conduction loss
in the diode bridge (two diodes in series at any time), and (e) the
bridge rectifier efficiency.

\textbf{Solution:}

\begin{enumerate}
\def\labelenumi{(\alph{enumi})}
\item
  DC output voltage (with capacitor, output ≈ peak line-to-line minus
  two diode drops): V\textsubscript{dc} = V\textsubscript{peak(L-L)} −
  2V\textsubscript{F} = 208 × √2 − 2 × 0.45 = 294.2 − 0.9 =
  \textbf{293.3 V}
\item
  DC load current: I\textsubscript{dc} = P / V\textsubscript{dc} = 2,000
  / 293.3 = \textbf{6.82 A}
\item
  Peak-to-peak ripple voltage for a six-pulse rectifier with capacitor
  filter: The ripple period is 1/(6 × 60) = 2.778 ms
  V\textsubscript{ripple} = I\textsubscript{dc} × Δt / C = 6.82 × 2.778
  × 10⁻³ / 4,700 × 10⁻⁶ = 0.01894 / 0.0047 = \textbf{4.03 V}
\end{enumerate}

This is 4.03/293.3 = 1.37\% ripple, which is acceptable for most
applications.

\begin{enumerate}
\def\labelenumi{(\alph{enumi})}
\setcounter{enumi}{3}
\item
  Conduction loss (two diodes in series): P\textsubscript{diode} = 2 ×
  V\textsubscript{F} × I\textsubscript{dc} = 2 × 0.45 × 6.82 =
  \textbf{6.14 W}
\item
  Bridge rectifier efficiency: η = P\textsubscript{load} /
  (P\textsubscript{load} + P\textsubscript{diode}) = 2,000 / (2,000 +
  6.14) = 2,000 / 2,006.14 = \textbf{99.69\%}
\end{enumerate}

The use of Schottky diodes reduces conduction losses to a negligible
level compared to silicon diodes (which would dissipate 2 × 0.9 × 6.82 =
12.3 W).

\chapter{Chapter 10 --- Section 10.3: DC-DC
Converters}\label{chapter-10-section-10.3-dc-dc-converters}

Practice problems covering buck converters, boost converters, buck-boost
converters, isolated converters (flyback, forward, full-bridge),
resonant converters, and voltage/current mode control.

\begin{center}\rule{0.5\linewidth}{0.5pt}\end{center}

\section{Problem 10.3.1}\label{problem-10.3.1}

\textbf{Given:} A synchronous buck converter steps down
V\textsubscript{in} = 48 V to V\textsubscript{out} = 12 V at
I\textsubscript{out} = 8 A. The switching frequency is
f\textsubscript{sw} = 250 kHz. The inductor has L = 22 μH and DCR = 25
mΩ. The high-side MOSFET has R\textsubscript{DS(on)} = 18 mΩ and the
low-side MOSFET has R\textsubscript{DS(on)} = 12 mΩ.

\textbf{Find:} (a) The duty cycle, (b) the inductor ripple current, (c)
the peak and valley inductor current, (d) the conduction loss in each
MOSFET, and (e) the inductor copper loss.

\textbf{Solution:}

\begin{enumerate}
\def\labelenumi{(\alph{enumi})}
\item
  Duty cycle: D = V\textsubscript{out} / V\textsubscript{in} = 12 / 48 =
  \textbf{0.25 (25\%)}
\item
  Inductor ripple current: ΔI\textsubscript{L} = (V\textsubscript{in} −
  V\textsubscript{out}) × D / (L × f\textsubscript{sw})
  ΔI\textsubscript{L} = (48 − 12) × 0.25 / (22 × 10⁻⁶ × 250 × 10³)
  ΔI\textsubscript{L} = 36 × 0.25 / 5.5 = 9.0 / 5.5 = \textbf{1.636 A}
\item
  Peak and valley inductor current: I\textsubscript{peak} =
  I\textsubscript{out} + ΔI\textsubscript{L}/2 = 8.0 + 0.818 =
  \textbf{8.818 A} I\textsubscript{valley} = I\textsubscript{out} −
  ΔI\textsubscript{L}/2 = 8.0 − 0.818 = \textbf{7.182 A}
\end{enumerate}

Since I\textsubscript{valley} \textgreater{} 0, the converter operates
in CCM.

\begin{enumerate}
\def\labelenumi{(\alph{enumi})}
\setcounter{enumi}{3}
\tightlist
\item
  MOSFET conduction losses: High-side RMS current:
  I\textsubscript{HS,rms} = I\textsubscript{out} × √D = 8.0 × √0.25 =
  8.0 × 0.5 = 4.0 A P\textsubscript{HS} = I\textsubscript{HS,rms}² ×
  R\textsubscript{DS(on)} = 4.0² × 0.018 = \textbf{0.288 W}
\end{enumerate}

Low-side RMS current: I\textsubscript{LS,rms} = I\textsubscript{out} ×
√(1 − D) = 8.0 × √0.75 = 8.0 × 0.866 = 6.928 A P\textsubscript{LS} =
I\textsubscript{LS,rms}² × R\textsubscript{DS(on)} = 6.928² × 0.012 =
48.0 × 0.012 = \textbf{0.576 W}

\begin{enumerate}
\def\labelenumi{(\alph{enumi})}
\setcounter{enumi}{4}
\tightlist
\item
  Inductor copper loss: I\textsubscript{L,rms} ≈ I\textsubscript{out} =
  8.0 A (ripple contribution is small) P\textsubscript{DCR} =
  I\textsubscript{L,rms}² × DCR = 8.0² × 0.025 = \textbf{1.60 W}
\end{enumerate}

Total conduction losses = 0.288 + 0.576 + 1.60 = \textbf{2.46 W}

\begin{center}\rule{0.5\linewidth}{0.5pt}\end{center}

\section{Problem 10.3.2}\label{problem-10.3.2}

\textbf{Given:} A boost converter with V\textsubscript{in} = 5 V
produces V\textsubscript{out} = 12 V at I\textsubscript{out} = 1.5 A.
The switching frequency is f\textsubscript{sw} = 1 MHz. The inductor is
L = 4.7 μH. The MOSFET has R\textsubscript{DS(on)} = 35 mΩ and the
Schottky diode has V\textsubscript{F} = 0.4 V.

\textbf{Find:} (a) The duty cycle, (b) the input current, (c) the
inductor ripple current, (d) the MOSFET conduction loss, and (e) the
diode conduction loss.

\textbf{Solution:}

\begin{enumerate}
\def\labelenumi{(\alph{enumi})}
\item
  Duty cycle: D = 1 − V\textsubscript{in}/V\textsubscript{out} = 1 −
  5/12 = 1 − 0.4167 = \textbf{0.5833 (58.3\%)}
\item
  Input current (equals inductor current average): I\textsubscript{in} =
  I\textsubscript{out} / (1 − D) = 1.5 / 0.4167 = \textbf{3.60 A}
\item
  Inductor ripple current: ΔI\textsubscript{L} = V\textsubscript{in} × D
  / (L × f\textsubscript{sw}) = 5 × 0.5833 / (4.7 × 10⁻⁶ × 1 × 10⁶)
  ΔI\textsubscript{L} = 2.917 / 4.7 = \textbf{0.620 A}
\end{enumerate}

Ripple ratio: ΔI\textsubscript{L}/I\textsubscript{in} = 0.620/3.60 =
17.2\% --- well within CCM.

\begin{enumerate}
\def\labelenumi{(\alph{enumi})}
\setcounter{enumi}{3}
\item
  MOSFET conduction loss: The MOSFET conducts during the on-time with
  the full inductor current. I\textsubscript{Q,rms} =
  I\textsubscript{in} × √D = 3.60 × √0.5833 = 3.60 × 0.7638 = 2.75 A
  P\textsubscript{Q} = I\textsubscript{Q,rms}² × R\textsubscript{DS(on)}
  = 2.75² × 0.035 = 7.5625 × 0.035 = \textbf{0.265 W}
\item
  Diode conduction loss: The diode conducts during the off-time.
  I\textsubscript{D,avg} = I\textsubscript{out} = 1.5 A (average diode
  current equals the output current) P\textsubscript{D} =
  V\textsubscript{F} × I\textsubscript{D,avg} = 0.4 × 1.5 = \textbf{0.60
  W}
\end{enumerate}

Total output power: P\textsubscript{out} = 12 × 1.5 = 18 W Total
estimated conduction losses: 0.265 + 0.60 = 0.865 W → η ≈ 18/(18 +
0.865) = \textbf{95.4\%} (conduction only)

\begin{center}\rule{0.5\linewidth}{0.5pt}\end{center}

\section{Problem 10.3.3}\label{problem-10.3.3}

\textbf{Given:} A SEPIC (Single-Ended Primary-Inductor Converter)
operates from V\textsubscript{in} = 12 V and must produce
V\textsubscript{out} = 24 V at I\textsubscript{out} = 0.5 A. The
switching frequency is f\textsubscript{sw} = 300 kHz. Both inductors are
L₁ = L₂ = 47 μH (uncoupled). The coupling capacitor is
C\textsubscript{s} = 10 μF.

\textbf{Find:} (a) The duty cycle, (b) the input current, (c) the
inductor ripple currents, (d) the RMS current in the coupling capacitor,
and (e) the voltage across the coupling capacitor.

\textbf{Solution:}

\begin{enumerate}
\def\labelenumi{(\alph{enumi})}
\item
  Duty cycle (SEPIC has the same transfer function as a buck-boost
  without inversion): V\textsubscript{out}/V\textsubscript{in} = D/(1 −
  D) 24/12 = D/(1 − D) → 2(1 − D) = D → 2 − 2D = D → 2 = 3D D =
  \textbf{0.667 (66.7\%)}
\item
  Input current: P\textsubscript{out} = V\textsubscript{out} ×
  I\textsubscript{out} = 24 × 0.5 = 12 W Assuming 100\% efficiency:
  I\textsubscript{in} = P\textsubscript{out}/V\textsubscript{in} = 12/12
  = \textbf{1.0 A}
\item
  Inductor ripple currents: ΔI\textsubscript{L1} = V\textsubscript{in} ×
  D / (L₁ × f\textsubscript{sw}) = 12 × 0.667 / (47 × 10⁻⁶ × 300 × 10³)
  = 8.0 / 14.1 = \textbf{0.567 A} ΔI\textsubscript{L2} =
  V\textsubscript{out} × (1 − D) / (L₂ × f\textsubscript{sw})
\end{enumerate}

During the switch on-time, the coupling capacitor voltage (≈
V\textsubscript{in}) is applied across L₂: ΔI\textsubscript{L2} =
V\textsubscript{in} × D / (L₂ × f\textsubscript{sw}) = 12 × 0.667 / 14.1
= \textbf{0.567 A} (same as L₁ since L₁ = L₂)

\begin{enumerate}
\def\labelenumi{(\alph{enumi})}
\setcounter{enumi}{3}
\tightlist
\item
  RMS coupling capacitor current: The coupling capacitor carries the AC
  component of the L₁ current during the off-time and the L₂ current
  during the on-time. The RMS current is approximately:
  I\textsubscript{Cs,rms} ≈ √(D × I\textsubscript{L2,avg}² + (1 − D) ×
  I\textsubscript{L1,avg}²)
\end{enumerate}

Average L₂ current = I\textsubscript{out} = 0.5 A; Average L₁ current =
I\textsubscript{in} = 1.0 A I\textsubscript{Cs,rms} ≈ √(0.667 × 0.5² +
0.333 × 1.0²) = √(0.167 + 0.333) = √0.5 = \textbf{0.707 A}

\begin{enumerate}
\def\labelenumi{(\alph{enumi})}
\setcounter{enumi}{4}
\tightlist
\item
  Voltage across the coupling capacitor: In a SEPIC, the coupling
  capacitor charges to V\textsubscript{in}: V\textsubscript{Cs} =
  V\textsubscript{in} = \textbf{12 V}
\end{enumerate}

\begin{center}\rule{0.5\linewidth}{0.5pt}\end{center}

\section{Problem 10.3.4}\label{problem-10.3.4}

\textbf{Given:} A forward converter operates from V\textsubscript{in} =
48 V and produces V\textsubscript{out} = 3.3 V at I\textsubscript{out} =
20 A. The transformer turns ratio is
N\textsubscript{p}:N\textsubscript{s} = 8:1. The switching frequency is
f\textsubscript{sw} = 350 kHz. The output inductor is L\textsubscript{o}
= 3.3 μH. Maximum duty cycle is D\textsubscript{max} = 0.45.

\textbf{Find:} (a) The reflected secondary voltage, (b) the required
duty cycle, (c) the output inductor ripple current, (d) the transformer
primary current at full load, and (e) the maximum output voltage
achievable at D\textsubscript{max}.

\textbf{Solution:}

\begin{enumerate}
\def\labelenumi{(\alph{enumi})}
\item
  Reflected secondary voltage: V\textsubscript{sec} =
  V\textsubscript{in} × N\textsubscript{s}/N\textsubscript{p} = 48 × 1/8
  = \textbf{6.0 V}
\item
  Required duty cycle: V\textsubscript{out} = V\textsubscript{sec} × D =
  6.0 × D D = V\textsubscript{out}/V\textsubscript{sec} = 3.3/6.0 =
  \textbf{0.55 (55\%)}
\end{enumerate}

This exceeds D\textsubscript{max} = 0.45, indicating the turns ratio is
not optimal. With D\textsubscript{max} = 0.45: V\textsubscript{out,max}
= 6.0 × 0.45 = 2.7 V \textless{} 3.3 V.

Revised: The turns ratio needs adjustment. For V\textsubscript{out} =
3.3 V at D = 0.40 (leaving margin):
N\textsubscript{s}/N\textsubscript{p} =
V\textsubscript{out}/(V\textsubscript{in} × D) = 3.3/(48 × 0.40) =
3.3/19.2 = 0.172 A practical ratio of
N\textsubscript{p}:N\textsubscript{s} = 6:1 gives V\textsubscript{sec} =
48/6 = 8.0 V. D = 3.3/8.0 = \textbf{0.4125 (41.25\%)} --- within
D\textsubscript{max} with margin.

\begin{enumerate}
\def\labelenumi{(\alph{enumi})}
\setcounter{enumi}{2}
\tightlist
\item
  Output inductor ripple current (with 6:1 turns ratio):
  ΔI\textsubscript{L} = (V\textsubscript{sec} − V\textsubscript{out}) ×
  D / (L\textsubscript{o} × f\textsubscript{sw}) ΔI\textsubscript{L} =
  (8.0 − 3.3) × 0.4125 / (3.3 × 10⁻⁶ × 350 × 10³) ΔI\textsubscript{L} =
  4.7 × 0.4125 / 1.155 = 1.939 / 1.155 = \textbf{1.68 A}
\end{enumerate}

Ripple ratio = 1.68/20 = 8.4\% --- acceptable.

\begin{enumerate}
\def\labelenumi{(\alph{enumi})}
\setcounter{enumi}{3}
\item
  Transformer primary current at full load: I\textsubscript{pri} =
  I\textsubscript{out} × N\textsubscript{s}/N\textsubscript{p} = 20 ×
  1/6 = \textbf{3.33 A} (during the on-time)
\item
  Maximum output voltage at D\textsubscript{max} (6:1 ratio):
  V\textsubscript{out,max} = V\textsubscript{sec} × D\textsubscript{max}
  = 8.0 × 0.45 = \textbf{3.6 V}
\end{enumerate}

\begin{center}\rule{0.5\linewidth}{0.5pt}\end{center}

\section{Problem 10.3.5}\label{problem-10.3.5}

\textbf{Given:} An LLC resonant converter has L\textsubscript{r} = 30
μH, C\textsubscript{r} = 22 nF, and L\textsubscript{m} = 180 μH. The
transformer turns ratio is 20:1 (half-bridge primary).
V\textsubscript{in} = 390 V (from PFC) and the target is
V\textsubscript{out} = 12 V at 25 A.

\textbf{Find:} (a) The series resonant frequency f\textsubscript{r}, (b)
the secondary resonant frequency f\textsubscript{p} (involving
L\textsubscript{m} + L\textsubscript{r}), (c) the
L\textsubscript{m}/L\textsubscript{r} ratio, (d) the output voltage at
f\textsubscript{sw} = f\textsubscript{r}, and (e) the required switching
frequency to produce exactly 12 V output (qualitative direction).

\textbf{Solution:}

\begin{enumerate}
\def\labelenumi{(\alph{enumi})}
\item
  Series resonant frequency: f\textsubscript{r} =
  1/(2π√(L\textsubscript{r} × C\textsubscript{r})) = 1/(2π√(30 × 10⁻⁶ ×
  22 × 10⁻⁹)) = 1/(2π√(6.6 × 10⁻¹³)) = 1/(2π × 8.124 × 10⁻⁷) = 1/(5.105
  × 10⁻⁶) = \textbf{195.9 kHz}
\item
  Secondary resonant frequency (all inductance in series with
  C\textsubscript{r}): f\textsubscript{p} = 1/(2π√((L\textsubscript{m} +
  L\textsubscript{r}) × C\textsubscript{r})) = 1/(2π√((180 + 30) × 10⁻⁶
  × 22 × 10⁻⁹)) = 1/(2π√(210 × 10⁻⁶ × 22 × 10⁻⁹)) = 1/(2π√(4.62 ×
  10⁻¹²)) = 1/(2π × 2.149 × 10⁻⁶) = 1/(1.3505 × 10⁻⁵) = \textbf{74.0
  kHz}
\item
  Inductance ratio: L\textsubscript{m}/L\textsubscript{r} = 180/30 =
  \textbf{6.0}
\item
  Output voltage at f\textsubscript{sw} = f\textsubscript{r}
  (half-bridge, gain ≈ 1.0): V\textsubscript{out} =
  (V\textsubscript{in}/2) × (N\textsubscript{s}/N\textsubscript{p}) ×
  gain = (390/2) × (1/20) × 1.0 V\textsubscript{out} = 195 × 0.05 =
  \textbf{9.75 V}
\end{enumerate}

This is below the 12 V target.

\begin{enumerate}
\def\labelenumi{(\alph{enumi})}
\setcounter{enumi}{4}
\tightlist
\item
  To increase the output to 12 V, the gain must be: G = 12/9.75 = 1.231.
  For an LLC converter, gain \textgreater{} 1 occurs when
  f\textsubscript{sw} \textless{} f\textsubscript{r} (below resonance).
  The switching frequency must be \textbf{reduced below 195.9 kHz} (into
  the inductive region between f\textsubscript{p} and
  f\textsubscript{r}) to boost the gain to 1.23.
\end{enumerate}

At light load, the converter operates near f\textsubscript{r}; at full
load, it operates below f\textsubscript{r} to provide the higher gain
needed to compensate for voltage drops.

\begin{center}\rule{0.5\linewidth}{0.5pt}\end{center}

\section{Problem 10.3.6}\label{problem-10.3.6}

\textbf{Given:} A buck converter with peak current mode control has
V\textsubscript{in} = 24 V, V\textsubscript{out} = 5 V, L = 10 μH, C =
220 μF (ESR = 10 mΩ), and f\textsubscript{sw} = 500 kHz. The current
sense resistor is R\textsubscript{sense} = 25 mΩ.

\textbf{Find:} (a) The duty cycle, (b) the inductor current up-slope and
down-slope, (c) the minimum slope compensation ramp to ensure stability
at all duty cycles, (d) the sense voltage at the current limit if
I\textsubscript{peak,max} = 12 A, and (e) the output voltage ripple.

\textbf{Solution:}

\begin{enumerate}
\def\labelenumi{(\alph{enumi})}
\item
  Duty cycle: D = V\textsubscript{out}/V\textsubscript{in} = 5/24 =
  \textbf{0.2083 (20.8\%)}
\item
  Inductor current slopes: Up-slope: m₁ = (V\textsubscript{in} −
  V\textsubscript{out})/L = (24 − 5)/(10 × 10⁻⁶) = 19/10⁻⁵ = \textbf{1.9
  A/μs} Down-slope: m₂ = V\textsubscript{out}/L = 5/(10 × 10⁻⁶) =
  \textbf{0.5 A/μs}
\item
  Minimum slope compensation: S\textsubscript{e} ≥ m₂/2 = 0.5/2 =
  \textbf{0.25 A/μs}
\end{enumerate}

In terms of sense voltage: S\textsubscript{e,v} = S\textsubscript{e} ×
R\textsubscript{sense} = 0.25 × 0.025 = 6.25 mV/μs Per switching period
(2 μs): ramp amplitude = 6.25 × 2 = \textbf{12.5 mV}

\begin{enumerate}
\def\labelenumi{(\alph{enumi})}
\setcounter{enumi}{3}
\item
  Sense voltage at current limit: V\textsubscript{sense} =
  I\textsubscript{peak,max} × R\textsubscript{sense} = 12 × 0.025 =
  \textbf{300 mV}
\item
  Output voltage ripple: Inductor ripple current: ΔI\textsubscript{L} =
  (V\textsubscript{in} − V\textsubscript{out}) × D / (L ×
  f\textsubscript{sw}) = 19 × 0.2083 / (10 × 10⁻⁶ × 500 × 10³) = 3.958 /
  5.0 = 0.792 A
\end{enumerate}

ESR ripple dominates at this frequency: V\textsubscript{ripple,ESR} =
ΔI\textsubscript{L} × ESR = 0.792 × 0.010 = 7.92 mV

Capacitive ripple: V\textsubscript{ripple,C} = ΔI\textsubscript{L} / (8
× C × f\textsubscript{sw}) = 0.792 / (8 × 220 × 10⁻⁶ × 500 × 10³) =
0.792 / 880 = 0.90 mV

Total ripple (RSS): V\textsubscript{ripple} = √(7.92² + 0.90²) = √(62.7
+ 0.81) = √63.5 = \textbf{7.97 mV} ≈ \textbf{8.0 mV peak-to-peak}

\begin{center}\rule{0.5\linewidth}{0.5pt}\end{center}

\section{Problem 10.3.7}\label{problem-10.3.7}

\textbf{Given:} A buck converter must maintain V\textsubscript{out} =
3.3 V from an input that varies between V\textsubscript{in} = 8 V and
V\textsubscript{in} = 16 V. The load current ranges from 0.5 A to 5 A.
The switching frequency is f\textsubscript{sw} = 600 kHz. The design
target is a maximum ripple current of 30\% of full-load current.

\textbf{Find:} (a) The duty cycle range, (b) the minimum inductance to
meet the ripple specification at worst case, (c) the minimum output
capacitance for 1\% output voltage ripple (ESR = 0), (d) the minimum
load for CCM at V\textsubscript{in} = 16 V, and (e) the input current
range at full load.

\textbf{Solution:}

\begin{enumerate}
\def\labelenumi{(\alph{enumi})}
\item
  Duty cycle range: D\textsubscript{max} =
  V\textsubscript{out}/V\textsubscript{in,min} = 3.3/8 = \textbf{0.4125}
  D\textsubscript{min} = V\textsubscript{out}/V\textsubscript{in,max} =
  3.3/16 = \textbf{0.2063}
\item
  Inductor ripple current is maximum when (V\textsubscript{in} −
  V\textsubscript{out}) × D is maximum. For a buck converter, this
  occurs at maximum V\textsubscript{in}:
\end{enumerate}

ΔI\textsubscript{L,max} = 0.30 × 5.0 = 1.5 A

L\textsubscript{min} = (V\textsubscript{in,max} − V\textsubscript{out})
× D\textsubscript{min} / (ΔI\textsubscript{L} × f\textsubscript{sw})
L\textsubscript{min} = (16 − 3.3) × 0.2063 / (1.5 × 600 × 10³)
L\textsubscript{min} = 12.7 × 0.2063 / 900,000 = 2.62 / 900,000 =
\textbf{2.91 μH} → use 3.3 μH

\begin{enumerate}
\def\labelenumi{(\alph{enumi})}
\setcounter{enumi}{2}
\item
  Minimum output capacitance for 1\% ripple (ESR = 0):
  V\textsubscript{ripple} = 0.01 × 3.3 = 33 mV C\textsubscript{min} =
  ΔI\textsubscript{L} / (8 × f\textsubscript{sw} ×
  V\textsubscript{ripple}) = 1.5 / (8 × 600 × 10³ × 0.033)
  C\textsubscript{min} = 1.5 / 158,400 = \textbf{9.47 μF} → use 10 μF
  ceramic
\item
  Minimum load current for CCM (at V\textsubscript{in} = 16 V with L =
  3.3 μH): ΔI\textsubscript{L} = (16 − 3.3) × 0.2063 / (3.3 × 10⁻⁶ × 600
  × 10³) = 2.62 / 1.98 = 1.323 A I\textsubscript{load,min} =
  ΔI\textsubscript{L}/2 = 1.323/2 = \textbf{0.662 A}
\end{enumerate}

Since the minimum load is 0.5 A \textless{} 0.662 A, the converter will
enter DCM at light load and V\textsubscript{in} = 16 V.

\begin{enumerate}
\def\labelenumi{(\alph{enumi})}
\setcounter{enumi}{4}
\tightlist
\item
  Input current range at full load (5 A): I\textsubscript{in} =
  I\textsubscript{out} × V\textsubscript{out}/V\textsubscript{in}
  (ideal) At V\textsubscript{in} = 8 V: I\textsubscript{in} = 5 × 3.3/8
  = \textbf{2.06 A} At V\textsubscript{in} = 16 V: I\textsubscript{in} =
  5 × 3.3/16 = \textbf{1.03 A}
\end{enumerate}

\begin{center}\rule{0.5\linewidth}{0.5pt}\end{center}

\section{Problem 10.3.8}\label{problem-10.3.8}

\textbf{Given:} A flyback converter operates from V\textsubscript{in} =
170 V DC (from a 120 V\textsubscript{rms} rectified input without PFC)
and produces two outputs: V\textsubscript{out1} = 5 V at 3 A and
V\textsubscript{out2} = 12 V at 1 A. The transformer turns ratio is
N\textsubscript{p}:N\textsubscript{s1}:N\textsubscript{s2} = 34:1:2.4.
The switching frequency is f\textsubscript{sw} = 100 kHz and the
converter operates in DCM.

\textbf{Find:} (a) The total output power, (b) the secondary voltage
reflected to the primary during the off-time (using the 5 V output), (c)
the peak voltage stress on the primary MOSFET (assuming
V\textsubscript{clamp} = 1.4 × reflected voltage), (d) the primary peak
current for the 5 V output, and (e) the MOSFET voltage rating required.

\textbf{Solution:}

\begin{enumerate}
\def\labelenumi{(\alph{enumi})}
\item
  Total output power: P\textsubscript{out} = V\textsubscript{out1} ×
  I\textsubscript{out1} + V\textsubscript{out2} × I\textsubscript{out2}
  = 5 × 3 + 12 × 1 = 15 + 12 = \textbf{27 W}
\item
  Reflected voltage from 5 V secondary to primary during off-time:
  V\textsubscript{reflected} = (V\textsubscript{out1} +
  V\textsubscript{F}) × N\textsubscript{p}/N\textsubscript{s1} = (5 +
  0.7) × 34/1 = 5.7 × 34 = \textbf{193.8 V}
\item
  Peak voltage on MOSFET: V\textsubscript{DS,peak} = V\textsubscript{in}
  + V\textsubscript{clamp} = 170 + 1.4 × 193.8 = 170 + 271.3 =
  \textbf{441.3 V}
\item
  Primary peak current (in DCM, assuming 85\% efficiency):
  P\textsubscript{in} = P\textsubscript{out}/η = 27/0.85 = 31.8 W In
  DCM: P\textsubscript{in} = ½ × L\textsubscript{p} ×
  I\textsubscript{pk}² × f\textsubscript{sw}
\end{enumerate}

The duty cycle for the 5 V output: V\textsubscript{out1} =
V\textsubscript{in} × (N\textsubscript{s1}/N\textsubscript{p}) × D/(1 −
D) doesn't directly apply in DCM. Using the power relationship:

Assuming L\textsubscript{p} = 1.0 mH (typical for this power level):
I\textsubscript{pk} = √(2 × P\textsubscript{in} / (L\textsubscript{p} ×
f\textsubscript{sw})) = √(2 × 31.8 / (1.0 × 10⁻³ × 100 × 10³))
I\textsubscript{pk} = √(63.6 / 100) = √0.636 = \textbf{0.798 A}

\begin{enumerate}
\def\labelenumi{(\alph{enumi})}
\setcounter{enumi}{4}
\tightlist
\item
  MOSFET voltage rating required (with safety margin):
  V\textsubscript{rating} ≥ V\textsubscript{DS,peak} × 1.2 = 441.3 × 1.2
  = 529.6 V → select \textbf{600 V} rated MOSFET
\end{enumerate}

\begin{center}\rule{0.5\linewidth}{0.5pt}\end{center}

\section{Problem 10.3.9}\label{problem-10.3.9}

\textbf{Given:} A phase-shifted full-bridge converter operates from
V\textsubscript{in} = 400 V and produces V\textsubscript{out} = 48 V at
I\textsubscript{out} = 50 A. The transformer turns ratio is
N\textsubscript{p}:N\textsubscript{s} = 4:1. The switching frequency is
f\textsubscript{sw} = 100 kHz. The converter achieves zero-voltage
switching (ZVS) with a dead time of t\textsubscript{dead} = 200 ns. Each
primary MOSFET has C\textsubscript{oss} = 200 pF and the transformer
leakage inductance is L\textsubscript{lk} = 5 μH.

\textbf{Find:} (a) The effective duty cycle, (b) the minimum current
required for ZVS, (c) whether the converter achieves ZVS at full load,
(d) the duty cycle loss due to leakage inductance, and (e) the converter
efficiency estimate from conduction losses alone
(R\textsubscript{DS(on)} = 100 mΩ per MOSFET, output rectifier
V\textsubscript{F} = 0.5 V).

\textbf{Solution:}

\begin{enumerate}
\def\labelenumi{(\alph{enumi})}
\item
  Effective duty cycle: V\textsubscript{out} = V\textsubscript{in} ×
  (N\textsubscript{s}/N\textsubscript{p}) × D\textsubscript{eff} 48 =
  400 × (1/4) × D\textsubscript{eff} = 100 × D\textsubscript{eff}
  D\textsubscript{eff} = 48/100 = \textbf{0.48 (48\%)}
\item
  Minimum current for ZVS (must charge/discharge 4 ×
  C\textsubscript{oss} during dead time): Energy required: E = 4 × ½ ×
  C\textsubscript{oss} × V\textsubscript{in}² = 4 × 0.5 × 200 × 10⁻¹² ×
  400² = 4 × 16 × 10⁻⁶ = 64 μJ
\end{enumerate}

This energy comes from the leakage inductance: ½ × L\textsubscript{lk} ×
I\textsubscript{min}² = 64 × 10⁻⁶ I\textsubscript{min} = √(2 × 64 × 10⁻⁶
/ 5 × 10⁻⁶) = √(128/5) = √25.6 = \textbf{5.06 A} (primary current)

\begin{enumerate}
\def\labelenumi{(\alph{enumi})}
\setcounter{enumi}{2}
\item
  Full-load primary current: I\textsubscript{pri} = I\textsubscript{out}
  × N\textsubscript{s}/N\textsubscript{p} = 50 × 1/4 = 12.5 A Since 12.5
  A \textgreater{} 5.06 A, \textbf{ZVS is achieved} at full load with
  ample margin.
\item
  Duty cycle loss: Δt\textsubscript{loss} = L\textsubscript{lk} ×
  I\textsubscript{pri} / V\textsubscript{in} = 5 × 10⁻⁶ × 12.5 / 400 =
  156.3 ns ΔD = Δt\textsubscript{loss} × f\textsubscript{sw} = 156.3 ×
  10⁻⁹ × 100 × 10³ = \textbf{0.0156 (1.56\%)}
\end{enumerate}

The actual duty cycle must be D = D\textsubscript{eff} + ΔD = 0.48 +
0.016 = \textbf{0.496}

\begin{enumerate}
\def\labelenumi{(\alph{enumi})}
\setcounter{enumi}{4}
\tightlist
\item
  Conduction losses: Primary MOSFETs (two in series at any time):
  I\textsubscript{pri,rms} ≈ I\textsubscript{pri} × √D = 12.5 × √0.496 =
  12.5 × 0.704 = 8.80 A per MOSFET pair P\textsubscript{pri} = 2 ×
  I\textsubscript{pri,rms}² × R\textsubscript{DS(on)} = 2 × 8.80² ×
  0.100 = 2 × 7.744 × 0.1 = \textbf{15.5 W}
\end{enumerate}

Output rectifiers (center-tapped): I\textsubscript{rect} =
I\textsubscript{out}/2 = 25 A average per rectifier
P\textsubscript{rect} = 2 × V\textsubscript{F} × I\textsubscript{rect} =
2 × 0.5 × 25 = \textbf{25.0 W}

Total conduction losses = 15.5 + 25.0 = 40.5 W P\textsubscript{out} = 48
× 50 = 2,400 W η ≈ 2,400/(2,400 + 40.5) = \textbf{98.3\%} (conduction
losses only; total efficiency ≈ 96--97\% including magnetics and
switching)

\begin{center}\rule{0.5\linewidth}{0.5pt}\end{center}

\section{Problem 10.3.10}\label{problem-10.3.10}

\textbf{Given:} A non-isolated buck converter must be designed to charge
a 12.6 V (3-cell Li-ion) battery from a 20 V USB-PD source. The maximum
charge current is 3 A. The switching frequency is 1 MHz. The target
inductor ripple is 20\% of full-load current.

\textbf{Find:} (a) The duty cycle at full battery voltage, (b) the duty
cycle at the start of charging (V\textsubscript{batt} = 9.0 V), (c) the
required inductance, (d) the output capacitance for 10 mV ripple (ESR =
0), and (e) the power loss in the high-side FET (R\textsubscript{DS(on)}
= 40 mΩ) at end of charge.

\textbf{Solution:}

\begin{enumerate}
\def\labelenumi{(\alph{enumi})}
\item
  Duty cycle at full battery voltage: D =
  V\textsubscript{batt}/V\textsubscript{in} = 12.6/20 = \textbf{0.63
  (63\%)}
\item
  Duty cycle at start of charging: D = 9.0/20 = \textbf{0.45 (45\%)}
\item
  Required inductance (worst case ripple at V\textsubscript{batt} = 9.0
  V, where (V\textsubscript{in} − V\textsubscript{out}) × D is largest):
  ΔI\textsubscript{L} = 0.20 × 3.0 = 0.6 A
\end{enumerate}

Check both operating points: At V\textsubscript{batt} = 9 V:
(V\textsubscript{in} − V\textsubscript{out}) × D = (20 − 9) × 0.45 = 11
× 0.45 = 4.95 At V\textsubscript{batt} = 12.6 V: (V\textsubscript{in} −
V\textsubscript{out}) × D = (20 − 12.6) × 0.63 = 7.4 × 0.63 = 4.662

Worst case is at V\textsubscript{batt} = 9 V: L = (V\textsubscript{in} −
V\textsubscript{batt}) × D / (ΔI\textsubscript{L} × f\textsubscript{sw})
= 4.95 / (0.6 × 1 × 10⁶) = \textbf{8.25 μH} → use 10 μH

\begin{enumerate}
\def\labelenumi{(\alph{enumi})}
\setcounter{enumi}{3}
\item
  Output capacitance for 10 mV ripple: C = ΔI\textsubscript{L} / (8 ×
  f\textsubscript{sw} × V\textsubscript{ripple}) = 0.6 / (8 × 1 × 10⁶ ×
  0.010) = 0.6 / 80,000 = \textbf{7.5 μF} → use 10 μF ceramic
\item
  High-side FET loss at end of charge (V\textsubscript{batt} = 12.6 V, I
  = 3 A, D = 0.63): I\textsubscript{HS,rms} = I\textsubscript{out} × √D
  = 3.0 × √0.63 = 3.0 × 0.794 = 2.38 A P\textsubscript{HS} =
  I\textsubscript{HS,rms}² × R\textsubscript{DS(on)} = 2.38² × 0.040 =
  5.664 × 0.040 = \textbf{0.227 W}
\end{enumerate}

\chapter{Chapter 10 --- Section 10.4:
Inverters}\label{chapter-10-section-10.4-inverters}

Practice problems covering single-phase inverters, three-phase
inverters, multilevel inverters, PWM modulation schemes, DC bus
utilization, harmonic analysis, and output filter design.

\begin{center}\rule{0.5\linewidth}{0.5pt}\end{center}

\section{Problem 10.4.1}\label{problem-10.4.1}

\textbf{Given:} A single-phase full-bridge (H-bridge) inverter operates
from a DC bus of V\textsubscript{dc} = 350 V using sinusoidal PWM
(SPWM). The modulation index is m\textsubscript{a} = 0.92 and the output
fundamental frequency is 60 Hz. The switching frequency is
f\textsubscript{sw} = 18 kHz.

\textbf{Find:} (a) The peak fundamental output voltage, (b) the RMS
fundamental output voltage, and (c) the frequencies of the two lowest
significant harmonic components.

\textbf{Solution:}

\begin{enumerate}
\def\labelenumi{(\alph{enumi})}
\item
  Peak fundamental output voltage: V₁\textsubscript{(peak)} =
  m\textsubscript{a} × V\textsubscript{dc} = 0.92 × 350 = \textbf{322.0
  V}
\item
  RMS fundamental output voltage: V₁\textsubscript{(rms)} =
  V₁\textsubscript{(peak)} / √2 = 322.0 / 1.414 = \textbf{227.7 V}
\item
  SPWM harmonics appear as sidebands around multiples of the switching
  frequency. The two lowest significant harmonics are at:
  f\textsubscript{sw} − f₁ = 18,000 − 60 = \textbf{17,940 Hz}
  f\textsubscript{sw} + f₁ = 18,000 + 60 = \textbf{18,060 Hz}
\end{enumerate}

These high-frequency components are easily attenuated by a small LC
output filter.

\begin{center}\rule{0.5\linewidth}{0.5pt}\end{center}

\section{Problem 10.4.2}\label{problem-10.4.2}

\textbf{Given:} A single-phase H-bridge inverter operates in square-wave
mode from V\textsubscript{dc} = 200 V, producing a 50 Hz output. The
load is a series R-L circuit with R = 5 Ω and L = 20 mH.

\textbf{Find:} (a) The RMS value of the fundamental component of the
square-wave output voltage, (b) the RMS fundamental load current, and
(c) the power delivered at the fundamental frequency.

\textbf{Solution:}

\begin{enumerate}
\def\labelenumi{(\alph{enumi})}
\item
  A square-wave ±V\textsubscript{dc} has a fundamental amplitude of:
  V₁\textsubscript{(peak)} = 4V\textsubscript{dc} / π = 4 × 200 / π =
  254.6 V V₁\textsubscript{(rms)} = 254.6 / √2 = \textbf{180.0 V}
\item
  Load impedance at 50 Hz: X\textsubscript{L} = 2π × 50 × 0.020 = 6.283
  Ω Z₁ = √(R² + X\textsubscript{L}²) = √(25 + 39.48) = √64.48 = 8.03 Ω
  I₁\textsubscript{(rms)} = V₁\textsubscript{(rms)} / Z₁ = 180.0 / 8.03
  = \textbf{22.42 A}
\item
  Power at the fundamental: P₁ = I₁² × R = 22.42² × 5 = 502.7 × 5 =
  \textbf{2,513 W}
\end{enumerate}

\begin{center}\rule{0.5\linewidth}{0.5pt}\end{center}

\section{Problem 10.4.3}\label{problem-10.4.3}

\textbf{Given:} A three-phase inverter operates from V\textsubscript{dc}
= 600 V using SPWM with a modulation index m\textsubscript{a} = 1.0. A
second identical inverter uses space vector PWM (SVPWM) from the same DC
bus.

\textbf{Find:} (a) The maximum fundamental line-to-line RMS voltage with
SPWM, (b) the maximum fundamental line-to-line RMS voltage with SVPWM,
and (c) the percentage improvement in DC bus utilization with SVPWM.

\textbf{Solution:}

\begin{enumerate}
\def\labelenumi{(\alph{enumi})}
\item
  SPWM: V\textsubscript{phase(peak)} = m\textsubscript{a} ×
  V\textsubscript{dc} / 2 = 1.0 × 600 / 2 = 300 V
  V\textsubscript{LL(rms)} = V\textsubscript{phase(peak)} × √3 / √2 =
  300 × 1.732 / 1.414 = 300 × 1.225 = \textbf{367.4 V}
\item
  SVPWM: V\textsubscript{phase(peak)} = V\textsubscript{dc} / √3 = 600 /
  1.732 = 346.4 V V\textsubscript{LL(rms)} = 346.4 × √3 / √2 = 346.4 ×
  1.225 = \textbf{424.4 V}
\item
  Improvement = (424.4 − 367.4) / 367.4 × 100 = 57.0 / 367.4 =
  \textbf{15.5\%}
\end{enumerate}

\begin{center}\rule{0.5\linewidth}{0.5pt}\end{center}

\section{Problem 10.4.4}\label{problem-10.4.4}

\textbf{Given:} A three-phase six-step (180° conduction) inverter
operates from V\textsubscript{dc} = 540 V. The output supplies a
balanced three-phase load.

\textbf{Find:} (a) The RMS value of the fundamental component of the
line-to-line output voltage, (b) the RMS values of the 5th and 7th
harmonic voltages, and (c) the voltage THD considering harmonics through
the 13th.

\textbf{Solution:}

\begin{enumerate}
\def\labelenumi{(\alph{enumi})}
\item
  For a six-step inverter, the fundamental line-to-line peak voltage is:
  V₁\textsubscript{(peak)} = (2√3/π) × V\textsubscript{dc} = (2 × 1.732
  / 3.1416) × 540 = 1.103 × 540 = 595.6 V V₁\textsubscript{(rms)} =
  595.6 / √2 = \textbf{421.2 V}
\item
  Characteristic harmonics of a six-step inverter have magnitudes
  V\textsubscript{h} = V₁/h: V₅\textsubscript{(rms)} = 421.2 / 5 =
  \textbf{84.2 V} V₇\textsubscript{(rms)} = 421.2 / 7 = \textbf{60.2 V}
\item
  Harmonics through the 13th: V₁₁\textsubscript{(rms)} = 421.2 / 11 =
  38.3 V V₁₃\textsubscript{(rms)} = 421.2 / 13 = 32.4 V THD = √(84.2² +
  60.2² + 38.3² + 32.4²) / 421.2 THD = √(7,089.6 + 3,624.0 + 1,466.9 +
  1,049.8) / 421.2 = √13,230.3 / 421.2 = 115.0 / 421.2 = \textbf{27.3\%}
\end{enumerate}

\begin{center}\rule{0.5\linewidth}{0.5pt}\end{center}

\section{Problem 10.4.5}\label{problem-10.4.5}

\textbf{Given:} A three-level NPC inverter operates from a DC bus of
V\textsubscript{dc} = 1,200 V (±600 V with a midpoint). A conventional
two-level inverter uses the same 1,200 V bus. Both use IGBT modules with
t\textsubscript{rise} = 80 ns. The switching frequency is
f\textsubscript{sw} = 3 kHz for the two-level and 3 kHz per level for
the three-level (6 kHz effective).

\textbf{Find:} (a) The voltage step size and dv/dt for each topology,
(b) the effective output switching frequency seen by the load, and (c)
the advantage in output filter size for the three-level inverter.

\textbf{Solution:}

\begin{enumerate}
\def\labelenumi{(\alph{enumi})}
\tightlist
\item
  Two-level: each transition swings between +600 V and −600 V, a step of
  1,200 V. dv/dt = 1,200 / 80 × 10⁻⁹ = \textbf{15.0 kV/μs}
\end{enumerate}

Three-level NPC: transitions between adjacent levels (e.g., 0 to +600
V), a step of 600 V. dv/dt = 600 / 80 × 10⁻⁹ = \textbf{7.5 kV/μs} (50\%
reduction)

\begin{enumerate}
\def\labelenumi{(\alph{enumi})}
\setcounter{enumi}{1}
\item
  Two-level: effective output switching frequency = f\textsubscript{sw}
  = \textbf{3 kHz} Three-level: effective output switching frequency = 2
  × f\textsubscript{sw} = 2 × 3,000 = \textbf{6 kHz} (two switching
  events per carrier cycle)
\item
  The LC filter corner frequency can be placed at \textasciitilde1/10 of
  the effective switching frequency. With the three-level inverter, the
  effective switching frequency is doubled, allowing the filter inductor
  and capacitor values to be reduced by approximately a factor of 2
  each. The filter volume scales roughly as the product L × C, so the
  three-level inverter enables an output filter approximately \textbf{4×
  smaller} in volume than the two-level inverter at the same THD
  specification.
\end{enumerate}

\begin{center}\rule{0.5\linewidth}{0.5pt}\end{center}

\section{Problem 10.4.6}\label{problem-10.4.6}

\textbf{Given:} A single-phase H-bridge inverter must supply 240
V\textsubscript{rms} at 60 Hz to an off-grid load. The inverter uses
SPWM at f\textsubscript{sw} = 24 kHz. An LC output filter must attenuate
the switching harmonics to less than 3\% THD. The rated load current is
20 A.

\textbf{Find:} (a) The minimum required DC bus voltage, (b) the LC
filter resonant frequency, and (c) suitable inductor and capacitor
values.

\textbf{Solution:}

\begin{enumerate}
\def\labelenumi{(\alph{enumi})}
\item
  For SPWM, maximum fundamental output: V₁\textsubscript{(peak)} =
  m\textsubscript{a} × V\textsubscript{dc} with m\textsubscript{a} ≤ 1.0
  Required V₁\textsubscript{(peak)} = 240 × √2 = 339.4 V
  V\textsubscript{dc(min)} = V₁\textsubscript{(peak)} /
  m\textsubscript{a} = 339.4 / 0.95 = 357.3 V (using m\textsubscript{a}
  = 0.95 for headroom) Use \textbf{V\textsubscript{dc} = 380 V} for
  adequate margin.
\item
  The first switching harmonic cluster is at f\textsubscript{sw} = 24
  kHz. The LC filter must provide sufficient attenuation at 24 kHz. A
  second-order LC filter rolls off at −40 dB/decade. For 3\% THD, the
  filter must attenuate harmonics by at least 30 dB at 24 kHz. 30 dB =
  31.6× attenuation. For a second-order filter: (f/f₀)² = 31.6, so f₀ =
  24,000 / √31.6 = 24,000 / 5.62 = \textbf{4,270 Hz}
\item
  Choose L = 1.5 mH (acceptable impedance at 60 Hz: X\textsubscript{L} =
  2π × 60 × 0.0015 = 0.565 Ω, negligible voltage drop at 20 A). C = 1 /
  (4π²f₀²L) = 1 / (4 × 9.87 × (4,270)² × 0.0015) = 1 / (4 × 9.87 × 18.23
  × 10⁶ × 0.0015) C = 1 / (1,079,000) = 0.927 μF Use \textbf{L = 1.5 mH}
  and \textbf{C = 1.0 μF} (standard value). The actual resonant
  frequency is f₀ = 1/(2π√(0.0015 × 10⁻⁶)) = 1/(2π × 1.225 × 10⁻³) ≈
  \textbf{4,107 Hz}, providing slightly more attenuation.
\end{enumerate}

\begin{center}\rule{0.5\linewidth}{0.5pt}\end{center}

\section{Problem 10.4.7}\label{problem-10.4.7}

\textbf{Given:} A three-phase inverter drives a 15 kW induction motor at
60 Hz from a 480 V\textsubscript{dc} bus using SVPWM. The motor power
factor is 0.87 lagging. The inverter efficiency is 97\%.

\textbf{Find:} (a) The maximum output line-to-line RMS voltage, (b) the
output line current, (c) the DC bus current, and (d) the power
dissipated in the inverter.

\textbf{Solution:}

\begin{enumerate}
\def\labelenumi{(\alph{enumi})}
\item
  SVPWM maximum output: V\textsubscript{phase(peak)} =
  V\textsubscript{dc} / √3 = 480 / 1.732 = 277.1 V
  V\textsubscript{LL(rms)} = 277.1 × √3 / √2 = 277.1 × 1.225 =
  \textbf{339.5 V}
\item
  Output line current: I\textsubscript{line} = P / (√3 ×
  V\textsubscript{LL} × PF) = 15,000 / (1.732 × 339.5 × 0.87) = 15,000 /
  511.7 = \textbf{29.3 A}
\item
  DC bus current (input power = output power / η): P\textsubscript{in} =
  15,000 / 0.97 = 15,464 W I\textsubscript{dc} = P\textsubscript{in} /
  V\textsubscript{dc} = 15,464 / 480 = \textbf{32.2 A}
\item
  Power dissipated in the inverter: P\textsubscript{loss} =
  P\textsubscript{in} − P\textsubscript{out} = 15,464 − 15,000 =
  \textbf{464 W}
\end{enumerate}

\begin{center}\rule{0.5\linewidth}{0.5pt}\end{center}

\section{Problem 10.4.8}\label{problem-10.4.8}

\textbf{Given:} A five-level cascaded H-bridge (CHB) inverter uses two
H-bridge modules per phase, each powered from an isolated 200 V DC
source. The inverter operates at 60 Hz with phase-shifted PWM at
f\textsubscript{sw} = 5 kHz per module.

\textbf{Find:} (a) The output voltage levels per phase, (b) the peak
fundamental phase voltage at maximum modulation, (c) the effective
output switching frequency, and (d) the voltage step size between
levels.

\textbf{Solution:}

\begin{enumerate}
\def\labelenumi{(\alph{enumi})}
\item
  Two H-bridge modules per phase, each capable of +V\textsubscript{dc},
  0, −V\textsubscript{dc}. The combined output can produce: +400 V, +200
  V, 0 V, −200 V, −400 V → \textbf{5 levels}
\item
  Peak fundamental phase voltage at maximum modulation
  (m\textsubscript{a} = 1.0): V\textsubscript{phase(peak)} = 2 ×
  V\textsubscript{dc} = 2 × 200 = \textbf{400 V}
\item
  With phase-shifted PWM, each module switches at 5 kHz with 90° carrier
  phase shift between the two modules. The effective output switching
  frequency: f\textsubscript{eff} = N × f\textsubscript{sw} = 2 × 5,000
  = \textbf{10 kHz}
\item
  Voltage step between adjacent levels: ΔV = V\textsubscript{dc} =
  \textbf{200 V} (compared to 800 V for a two-level inverter with
  V\textsubscript{dc} = 400 V, a 75\% reduction)
\end{enumerate}

\begin{center}\rule{0.5\linewidth}{0.5pt}\end{center}

\section{Problem 10.4.9}\label{problem-10.4.9}

\textbf{Given:} A grid-tied three-phase inverter rated at 250 kW
connects to a 480 V, 60 Hz utility through an LCL filter. The grid code
requires THD \textless{} 5\% at the point of common coupling. The
inverter switching frequency is f\textsubscript{sw} = 10 kHz. The LCL
filter has L₁ = 0.5 mH (inverter-side), L₂ = 0.2 mH (grid-side), and
C\textsubscript{f} = 15 μF (with a damping resistor R\textsubscript{d}).

\textbf{Find:} (a) The LCL filter resonant frequency, (b) an appropriate
damping resistor value, and (c) the attenuation at the switching
frequency.

\textbf{Solution:}

\begin{enumerate}
\def\labelenumi{(\alph{enumi})}
\tightlist
\item
  LCL resonant frequency: f\textsubscript{res} = (1/2π) × √((L₁ + L₂) /
  (L₁ × L₂ × C\textsubscript{f})) f\textsubscript{res} = (1/2π) ×
  √((0.0005 + 0.0002) / (0.0005 × 0.0002 × 15 × 10⁻⁶))
  f\textsubscript{res} = (1/2π) × √(0.0007 / (1.5 × 10⁻¹²))
  f\textsubscript{res} = (1/2π) × √(4.667 × 10⁸) = (1/6.283) × 21,602 =
  \textbf{3,439 Hz}
\end{enumerate}

This is between 10\% and 50\% of f\textsubscript{sw}, a good design
range for effective attenuation without stability issues.

\begin{enumerate}
\def\labelenumi{(\alph{enumi})}
\setcounter{enumi}{1}
\item
  Damping resistor (typically set to 1/3 of the capacitor impedance at
  resonance): X\textsubscript{C(res)} = 1 / (2π × 3,439 × 15 × 10⁻⁶) = 1
  / 0.324 = 3.09 Ω R\textsubscript{d} = X\textsubscript{C(res)} / 3 =
  3.09 / 3 = \textbf{1.0 Ω}
\item
  At f\textsubscript{sw} = 10 kHz, well above resonance, the LCL filter
  provides approximately −60 dB/decade roll-off. The frequency ratio is
  10,000 / 3,439 = 2.91. Attenuation ≈ (f/f\textsubscript{res})³ = 2.91³
  = 24.6 → 20 × log₁₀(24.6) = \textbf{27.8 dB}
\end{enumerate}

The 28 dB attenuation at 10 kHz, combined with the SVPWM spectrum
characteristics, is typically sufficient to meet the 5\% THD
requirement.

\begin{center}\rule{0.5\linewidth}{0.5pt}\end{center}

\section{Problem 10.4.10}\label{problem-10.4.10}

\textbf{Given:} A modular multilevel converter (MMC) for an HVDC
application has 200 submodules per arm, each with a submodule capacitor
voltage of 2.0 kV. The converter has six arms (three phase legs, upper
and lower arms). The DC link voltage is V\textsubscript{dc}.

\textbf{Find:} (a) The DC link voltage, (b) the maximum AC phase voltage
(peak, line-to-neutral), (c) the maximum AC line-to-line RMS voltage,
and (d) the number of output voltage levels per phase.

\textbf{Solution:}

\begin{enumerate}
\def\labelenumi{(\alph{enumi})}
\item
  In an MMC, the DC link voltage equals the sum of submodule voltages in
  one arm (upper or lower): V\textsubscript{dc} = N ×
  V\textsubscript{SM} = 200 × 2.0 = \textbf{400 kV}
\item
  The AC phase voltage peak (line-to-neutral) for an MMC:
  V\textsubscript{phase(peak)} = V\textsubscript{dc} / 2 = 400 / 2 =
  \textbf{200 kV}
\item
  Maximum line-to-line RMS: V\textsubscript{LL(rms)} =
  V\textsubscript{phase(peak)} × √3 / √2 = 200 × 1.732 / 1.414 = 200 ×
  1.225 = \textbf{245.0 kV}
\item
  Each arm has 200 submodules, each either inserted (contributes
  V\textsubscript{SM}) or bypassed (contributes 0). The number of
  distinct voltage levels in the phase output: N\textsubscript{levels} =
  N + 1 = 200 + 1 = \textbf{201 levels}
\end{enumerate}

With 201 voltage levels, the output waveform is an extremely close
approximation of a sine wave, producing negligible harmonic distortion
without output filters and voltage steps of only 2.0 kV out of 400 kV
total (0.5\% per step).

\chapter{Chapter 10 --- Section 10.5: AC-AC
Converters}\label{chapter-10-section-10.5-ac-ac-converters}

Practice problems covering AC voltage controllers, cycloconverters,
matrix converters, phase-angle control, integral cycle control, voltage
transfer ratios, and power factor effects.

\begin{center}\rule{0.5\linewidth}{0.5pt}\end{center}

\section{Problem 10.5.1}\label{problem-10.5.1}

\textbf{Given:} A single-phase AC voltage controller uses back-to-back
thyristors to supply a 15 Ω resistive heating element from a 208
V\textsubscript{rms}, 60 Hz source. The firing angle is α = 60°.

\textbf{Find:} (a) The RMS output voltage, (b) the power delivered to
the load, and (c) the power as a percentage of full (uncontrolled)
power.

\textbf{Solution:}

\begin{enumerate}
\def\labelenumi{(\alph{enumi})}
\item
  For a resistive load with phase-angle control:
  V\textsubscript{out(rms)} = V\textsubscript{in} × √{[}(π − α +
  sin(2α)/2) / π{]} With α = 60° = π/3 radians: sin(2 × 60°) = sin(120°)
  = 0.8660 V\textsubscript{out(rms)} = 208 × √{[}(π − π/3 + 0.8660/2) /
  π{]} V\textsubscript{out(rms)} = 208 × √{[}(3.1416 − 1.0472 + 0.4330)
  / 3.1416{]} V\textsubscript{out(rms)} = 208 × √{[}2.5274 / 3.1416{]} =
  208 × √0.8046 = 208 × 0.8970 = \textbf{186.6 V}
\item
  Power delivered: P = V\textsubscript{out(rms)}² / R = 186.6² / 15 =
  34,820 / 15 = \textbf{2,321 W}
\item
  Full power: P\textsubscript{full} = V\textsubscript{in}² / R = 208² /
  15 = 43,264 / 15 = 2,884 W Percentage: 2,321 / 2,884 × 100 =
  \textbf{80.5\%}
\end{enumerate}

\begin{center}\rule{0.5\linewidth}{0.5pt}\end{center}

\section{Problem 10.5.2}\label{problem-10.5.2}

\textbf{Given:} A single-phase AC voltage controller with phase-angle
control supplies a 20 Ω resistive load from a 240 V\textsubscript{rms},
50 Hz source. The desired output power is 1,200 W.

\textbf{Find:} (a) The required RMS output voltage, (b) the required
firing angle α, and (c) the input power factor.

\textbf{Solution:}

\begin{enumerate}
\def\labelenumi{(\alph{enumi})}
\item
  Required RMS output voltage: V\textsubscript{out} = √(P × R) = √(1,200
  × 20) = √24,000 = \textbf{154.9 V}
\item
  The output-to-input voltage ratio:
  V\textsubscript{out}/V\textsubscript{in} = 154.9 / 240 = 0.6454
  (V\textsubscript{out}/V\textsubscript{in})² = 0.4165 = (π − α +
  sin(2α)/2) / π
\end{enumerate}

Solving: π − α + sin(2α)/2 = 0.4165 × π = 1.3084 rad

Testing α = 105° = 1.8326 rad: π − 1.8326 + sin(210°)/2 = 1.3090 +
(−0.5)/2 = 1.3090 − 0.25 = 1.0590 (too low)

Testing α = 90° = π/2: π − π/2 + sin(180°)/2 = 1.5708 + 0 = 1.5708 (too
high)

Testing α = 100° = 1.7453 rad: π − 1.7453 + sin(200°)/2 = 1.3963 +
(−0.3420)/2 = 1.3963 − 0.1710 = 1.2253 (too low)

Testing α = 95° = 1.6581 rad: π − 1.6581 + sin(190°)/2 = 1.4835 +
(−0.1736)/2 = 1.4835 − 0.0868 = 1.3967 (close)

Testing α = 97° = 1.6930 rad: π − 1.6930 + sin(194°)/2 = 1.4486 +
(−0.2419)/2 = 1.4486 − 0.1210 = 1.3276 (close)

Testing α = 98° = 1.7104 rad: π − 1.7104 + sin(196°)/2 = 1.4312 +
(−0.2756)/2 = 1.4312 − 0.1378 = 1.2934

Interpolating between 97° and 95°: \textbf{α ≈ 96°}

\begin{enumerate}
\def\labelenumi{(\alph{enumi})}
\setcounter{enumi}{2}
\tightlist
\item
  Input power factor = P\textsubscript{out} / S\textsubscript{in}. The
  input apparent power: S\textsubscript{in} = V\textsubscript{in} ×
  I\textsubscript{in(rms)} = 240 × (154.9/20) = 240 × 7.745 = 1,858.8 VA
  (Note: I\textsubscript{in(rms)} = I\textsubscript{out(rms)} =
  V\textsubscript{out(rms)}/R for a resistive load with thyristor
  control) PF = P / S\textsubscript{in} = 1,200 / 1,858.8 =
  \textbf{0.645}
\end{enumerate}

The low power factor is a characteristic disadvantage of phase-angle
control at reduced output.

\begin{center}\rule{0.5\linewidth}{0.5pt}\end{center}

\section{Problem 10.5.3}\label{problem-10.5.3}

\textbf{Given:} An integral cycle (burst-firing) controller operates on
a 240 V\textsubscript{rms}, 50 Hz supply. It uses a duty pattern of 3
cycles ON, 5 cycles OFF to control power to a 30 Ω resistive furnace
heating element.

\textbf{Find:} (a) The fraction of power delivered, (b) the average
power to the load, (c) the RMS output voltage, and (d) the input power
factor.

\textbf{Solution:}

\begin{enumerate}
\def\labelenumi{(\alph{enumi})}
\item
  Fraction of power: Duty ratio = cycles ON / total cycles = 3 / (3 + 5)
  = 3/8 = \textbf{0.375 (37.5\%)}
\item
  Full power: P\textsubscript{full} = V\textsubscript{in}² / R = 240² /
  30 = 57,600 / 30 = 1,920 W Average power: P\textsubscript{avg} = 0.375
  × 1,920 = \textbf{720 W}
\item
  RMS output voltage: V\textsubscript{out(rms)} = V\textsubscript{in} ×
  √(duty ratio) = 240 × √0.375 = 240 × 0.6124 = \textbf{146.97 V}
\item
  Input power factor: PF = √(duty ratio) = √0.375 = \textbf{0.612}
\end{enumerate}

Integral cycle control produces no sub-line-frequency harmonics at the
switching frequency, but introduces a modulation component at the burst
rate of 50/(3+5) = 6.25 Hz, which can cause visible flicker in lighting
loads.

\begin{center}\rule{0.5\linewidth}{0.5pt}\end{center}

\section{Problem 10.5.4}\label{problem-10.5.4}

\textbf{Given:} A three-phase to single-phase cycloconverter is fed from
a 60 Hz, 460 V\textsubscript{rms} (line-to-line) three-phase supply. The
cycloconverter must produce a 12 Hz output for a low-speed grinding mill
drive rated at 200 kW.

\textbf{Find:} (a) Whether the 12 Hz output frequency is within the
cycloconverter's capability, (b) the maximum output RMS voltage, (c) the
output current at rated power, and (d) the input apparent power assuming
a displacement power factor of 0.85.

\textbf{Solution:}

\begin{enumerate}
\def\labelenumi{(\alph{enumi})}
\item
  Maximum recommended output frequency: f\textsubscript{out(max)} =
  f\textsubscript{in} / 3 = 60 / 3 = 20 Hz Since 12 Hz \textless{} 20
  Hz, the output frequency is \textbf{achievable} with acceptable
  waveform quality.
\item
  Practical maximum output voltage for a cycloconverter (accounting for
  commutation overlap): V\textsubscript{out(rms)} ≈ 0.75 ×
  V\textsubscript{in(LL)} = 0.75 × 460 = \textbf{345
  V\textsubscript{rms}}
\item
  Output current at rated power: I\textsubscript{out} = P /
  V\textsubscript{out} = 200,000 / 345 = \textbf{579.7 A}
\item
  Input apparent power: S\textsubscript{in} = P / DPF = 200,000 / 0.85 =
  \textbf{235.3 kVA} Input line current: I\textsubscript{in} =
  S\textsubscript{in} / (√3 × V\textsubscript{in}) = 235,300 / (1.732 ×
  460) = 235,300 / 796.7 = \textbf{295.4 A}
\end{enumerate}

\begin{center}\rule{0.5\linewidth}{0.5pt}\end{center}

\section{Problem 10.5.5}\label{problem-10.5.5}

\textbf{Given:} A three-phase matrix converter is supplied from a 400
V\textsubscript{rms} line-to-line, 50 Hz source and must drive a
three-phase induction motor at 35 Hz. The motor is rated at 22 kW with a
power factor of 0.85. The converter efficiency is 96\%.

\textbf{Find:} (a) The maximum output line-to-line voltage, (b) the
output line current, (c) the input line current at unity input
displacement power factor, and (d) the input power factor.

\textbf{Solution:}

\begin{enumerate}
\def\labelenumi{(\alph{enumi})}
\item
  Maximum voltage transfer ratio for a matrix converter:
  q\textsubscript{max} = √3/2 = 0.866 V\textsubscript{out(LL)} =
  q\textsubscript{max} × V\textsubscript{in(LL)} = 0.866 × 400 =
  \textbf{346.4 V}
\item
  Output line current: I\textsubscript{out} = P / (√3 ×
  V\textsubscript{out} × PF) = 22,000 / (1.732 × 346.4 × 0.85) = 22,000
  / 509.8 = \textbf{43.2 A}
\item
  Input power (accounting for efficiency): P\textsubscript{in} =
  P\textsubscript{out} / η = 22,000 / 0.96 = 22,917 W
  I\textsubscript{in} = P\textsubscript{in} / (√3 × V\textsubscript{in}
  × cos φ) = 22,917 / (1.732 × 400 × 1.0) = 22,917 / 692.8 =
  \textbf{33.1 A}
\item
  The input displacement power factor is controllable to unity. The true
  input power factor accounts for harmonic currents:
  PF\textsubscript{in} ≈ DPF × (I₁/I\textsubscript{rms}) ≈ 1.0 × 0.98 =
  \textbf{0.98} (matrix converters produce very low input current THD,
  approximately 3-5\%)
\end{enumerate}

\begin{center}\rule{0.5\linewidth}{0.5pt}\end{center}

\section{Problem 10.5.6}\label{problem-10.5.6}

\textbf{Given:} A single-phase AC voltage controller with back-to-back
thyristors drives a 5 Ω pure resistive load from a 120
V\textsubscript{rms}, 60 Hz source. The firing angle is α = 120°.

\textbf{Find:} (a) The RMS output voltage, (b) the output power, (c) the
RMS of the fundamental component of the input current, and (d) the total
harmonic distortion of the input current.

\textbf{Solution:}

\begin{enumerate}
\def\labelenumi{(\alph{enumi})}
\item
  V\textsubscript{out(rms)} = V\textsubscript{in} × √{[}(π − α +
  sin(2α)/2) / π{]} α = 120° = 2π/3 rad; sin(240°) = −0.866
  V\textsubscript{out(rms)} = 120 × √{[}(π − 2π/3 + (−0.866)/2) / π{]}
  V\textsubscript{out(rms)} = 120 × √{[}(3.1416 − 2.0944 − 0.4330) /
  3.1416{]} V\textsubscript{out(rms)} = 120 × √{[}0.6142 / 3.1416{]} =
  120 × √0.1955 = 120 × 0.4421 = \textbf{53.1 V}
\item
  Output power: P = V\textsubscript{out(rms)}² / R = 53.1² / 5 = 2,819.6
  / 5 = \textbf{564 W}
\item
  The RMS fundamental input current: I\textsubscript{rms(total)} =
  V\textsubscript{out(rms)} / R = 53.1 / 5 = 10.62 A The fundamental
  Fourier coefficient for a phase-controlled resistive load:
  I₁\textsubscript{(rms)} can be found from the displacement power
  factor relationship. For α = 120°, the fundamental component
  amplitude: I₁\textsubscript{(peak)} = (V\textsubscript{peak}/(πR)) ×
  √{[}(π − α)² + sin²(α) + 2(π − α)sin(α)cos(α){]}
\end{enumerate}

Simplified approach using known power factor relations: The total power
= V\textsubscript{in} × I₁\textsubscript{(rms)} × cos(φ₁), where φ₁ is
the displacement angle ≈ α/2 for resistive loads. P = 564 W,
V\textsubscript{in} = 120 V I₁\textsubscript{(rms)} × cos(60°) = 564/120
= 4.70 I₁\textsubscript{(rms)} = 4.70/0.5 = \textbf{9.40 A}

\begin{enumerate}
\def\labelenumi{(\alph{enumi})}
\setcounter{enumi}{3}
\tightlist
\item
  THD of input current: THD = √(I\textsubscript{rms}² − I₁² ) / I₁ =
  √(10.62² − 9.40²) / 9.40 = √(112.8 − 88.4) / 9.40 = √24.4 / 9.40 =
  4.94 / 9.40 = \textbf{52.5\%}
\end{enumerate}

\begin{center}\rule{0.5\linewidth}{0.5pt}\end{center}

\section{Problem 10.5.7}\label{problem-10.5.7}

\textbf{Given:} A three-phase matrix converter replaces a back-to-back
voltage-source converter (VSC) in a wind turbine application. The matrix
converter eliminates the DC link capacitor. The back-to-back VSC uses
4,000 μF of DC link capacitance at 700 V.

\textbf{Find:} (a) The stored energy in the DC link capacitors of the
back-to-back VSC, (b) the weight savings if electrolytic capacitors have
an energy density of 0.08 J/g, (c) the voltage transfer ratio limitation
of the matrix converter, and (d) the maximum output voltage if the grid
is 690 V\textsubscript{rms} line-to-line.

\textbf{Solution:}

\begin{enumerate}
\def\labelenumi{(\alph{enumi})}
\item
  Stored energy in DC link: E = ½CV² = 0.5 × 4,000 × 10⁻⁶ × 700² = 0.5 ×
  0.004 × 490,000 = \textbf{980 J}
\item
  Weight of capacitors: m = E / (energy density) = 980 / 0.08 =
  \textbf{12,250 g = 12.25 kg} Eliminating the DC link saves
  approximately \textbf{12.25 kg} of capacitor weight (additional
  savings from capacitor mounting hardware and bus bars).
\item
  Matrix converter maximum voltage transfer ratio: q\textsubscript{max}
  = √3/2 = \textbf{0.866} (86.6\% of the input voltage)
\end{enumerate}

This is a limitation compared to the back-to-back VSC which can achieve
unity or higher voltage ratio through the DC link.

\begin{enumerate}
\def\labelenumi{(\alph{enumi})}
\setcounter{enumi}{3}
\tightlist
\item
  Maximum output line-to-line voltage: V\textsubscript{out(LL)} = 0.866
  × 690 = \textbf{597.5 V}
\end{enumerate}

\begin{center}\rule{0.5\linewidth}{0.5pt}\end{center}

\section{Problem 10.5.8}\label{problem-10.5.8}

\textbf{Given:} An AC voltage controller uses integral cycle control to
regulate the temperature of a 5 kW, 240 V\textsubscript{rms} industrial
oven (pure resistive load, R = 11.52 Ω). The control period is 10 cycles
(200 ms at 50 Hz). The oven requires 3.2 kW to maintain the setpoint
temperature.

\textbf{Find:} (a) The number of ON cycles required per control period,
(b) the RMS output voltage, (c) the modulation frequency, and (d) the
sub-harmonic frequency components introduced.

\textbf{Solution:}

\begin{enumerate}
\def\labelenumi{(\alph{enumi})}
\tightlist
\item
  Power fraction required: P\textsubscript{required} /
  P\textsubscript{full} = 3,200 / 5,000 = 0.64 Number of ON cycles =
  0.64 × 10 = 6.4 → round to \textbf{6 cycles ON, 4 cycles OFF}
  (delivering 60\% power) or \textbf{7 cycles ON, 3 cycles OFF}
  (delivering 70\% power)
\end{enumerate}

For a setpoint of 3.2 kW, use \textbf{6 ON / 4 OFF} (3,000 W) or
\textbf{7 ON / 3 OFF} (3,500 W), alternating between the two patterns. A
common approach: use 6 ON for most periods and 7 ON periodically to
average 3,200 W.

With 6 ON: P = (6/10) × 5,000 = \textbf{3,000 W}

\begin{enumerate}
\def\labelenumi{(\alph{enumi})}
\setcounter{enumi}{1}
\item
  RMS output voltage: V\textsubscript{out(rms)} = 240 × √(6/10) = 240 ×
  0.7746 = \textbf{185.9 V}
\item
  Modulation frequency: f\textsubscript{mod} = f\textsubscript{line} /
  total cycles = 50 / 10 = \textbf{5 Hz} (the burst repetition rate)
\item
  Sub-harmonic components are introduced at multiples of the modulation
  frequency: Primary sub-harmonic at \textbf{5 Hz}, with sidebands at 50
  ± 5 = \textbf{45 Hz and 55 Hz}, and higher-order sidebands at 100 ± 5
  Hz, etc. The 5 Hz modulation is well below the flicker sensitivity
  peak of the human eye (\textasciitilde8 Hz), making this acceptable
  for heating loads but unsuitable for lighting.
\end{enumerate}

\begin{center}\rule{0.5\linewidth}{0.5pt}\end{center}

\section{Problem 10.5.9}\label{problem-10.5.9}

\textbf{Given:} A three-phase to three-phase cycloconverter uses six
three-phase thyristor bridges to produce a three-phase, 10 Hz output
from a 50 Hz, 415 V\textsubscript{rms} (line-to-line) supply. The load
is a 500 kW synchronous motor at 0.90 power factor leading.

\textbf{Find:} (a) The output frequency ratio, (b) the theoretical
maximum output voltage, (c) the practical output voltage at 75\% voltage
ratio, (d) the output current, and (e) the input kVA.

\textbf{Solution:}

\begin{enumerate}
\def\labelenumi{(\alph{enumi})}
\item
  Output frequency ratio: f\textsubscript{out} / f\textsubscript{in} =
  10 / 50 = \textbf{0.20} (well within the 1/3 limit of 0.333)
\item
  Theoretical maximum output voltage: V\textsubscript{dc(max)} of a
  three-phase bridge = (3√3/π) × V\textsubscript{peak(LL)} = 1.654 ×
  415√2 = 1.654 × 586.9 = 970.7 V V\textsubscript{out(rms,max)} =
  V\textsubscript{dc(max)} / √2 = 970.7 / 1.414 = \textbf{686.5 V}
\item
  Practical output voltage: V\textsubscript{out(rms)} = 0.75 × 415 =
  \textbf{311.3 V}
\item
  Output current: I\textsubscript{out} = P / (√3 × V\textsubscript{out}
  × PF) = 500,000 / (1.732 × 311.3 × 0.90) = 500,000 / 484.8 =
  \textbf{1,031 A}
\item
  Input kVA (assuming cycloconverter input power factor ≈ 0.70 due to
  thyristor phase control): P\textsubscript{in} ≈ P\textsubscript{out} /
  0.98 = 500,000 / 0.98 = 510,204 W (assuming 98\% converter efficiency)
  S\textsubscript{in} = P\textsubscript{in} / PF\textsubscript{in} =
  510,204 / 0.70 = \textbf{728.9 kVA} I\textsubscript{in} =
  S\textsubscript{in} / (√3 × 415) = 728,900 / 718.8 = \textbf{1,014 A}
\end{enumerate}

\begin{center}\rule{0.5\linewidth}{0.5pt}\end{center}

\section{Problem 10.5.10}\label{problem-10.5.10}

\textbf{Given:} A matrix converter feeding a three-phase motor must
perform a four-step commutation sequence when switching an output phase
from input phase A to input phase B. The output current is
I\textsubscript{out} = 30 A, V\textsubscript{A} = 310 V,
V\textsubscript{B} = 185 V at the switching instant, and the IGBT
turn-off time is t\textsubscript{off} = 200 ns. A clamp circuit with a
20 μF capacitor protects against overvoltage.

\textbf{Find:} (a) The voltage across the outgoing switch at turn-off,
(b) the time available for each commutation step if the total
commutation must complete within 2 μs, (c) the energy stored in the
clamp capacitor if it absorbs one commutation event, and (d) the clamp
capacitor voltage rise from one event.

\textbf{Solution:}

\begin{enumerate}
\def\labelenumi{(\alph{enumi})}
\item
  The voltage across the outgoing switch when it turns off is the
  difference between the two input phase voltages:
  V\textsubscript{switch} = V\textsubscript{A} − V\textsubscript{B} =
  310 − 185 = \textbf{125 V}
\item
  With a four-step commutation and total time budget of 2 μs: Time per
  step = 2,000 ns / 4 = \textbf{500 ns per step} This exceeds the 200 ns
  IGBT turn-off time, providing adequate margin.
\item
  Energy stored if the clamp absorbs the commutation event: The
  commutation occurs over approximately t\textsubscript{off} = 200 ns.
  Energy = ½ × L\textsubscript{stray} × I² (from stray inductance), but
  for a voltage-driven clamp: E\textsubscript{clamp} ≈
  V\textsubscript{switch} × I\textsubscript{out} × t\textsubscript{off}
  = 125 × 30 × 200 × 10⁻⁹ = \textbf{0.75 mJ}
\item
  Voltage rise on the clamp capacitor: ΔV = √(2 × E / C +
  V\textsubscript{initial}²) − V\textsubscript{initial}
\end{enumerate}

For a pre-charged clamp at V\textsubscript{clamp} =
V\textsubscript{peak(LL)} = 400√2 = 565.7 V: ΔE = 0.75 × 10⁻³ J. C = 20
× 10⁻⁶ F. ΔV ≈ E / (C × V\textsubscript{clamp}) = 0.75 × 10⁻³ / (20 ×
10⁻⁶ × 565.7) = 0.75 × 10⁻³ / 0.01131 = \textbf{0.066 V}

The voltage rise is negligible, confirming the 20 μF clamp capacitor is
adequately sized for commutation protection.

\chapter{Chapter 10 --- Section 10.6: Thermal
Management}\label{chapter-10-section-10.6-thermal-management}

Practice problems covering power loss calculations, thermal resistance
networks, heat sink sizing, junction temperature analysis, thermal
interface materials, liquid cooling design, and EMI filter design for
power converters.

\begin{center}\rule{0.5\linewidth}{0.5pt}\end{center}

\section{Problem 10.6.1}\label{problem-10.6.1}

\textbf{Given:} A synchronous buck converter uses a high-side MOSFET
with R\textsubscript{DS(on)} = 22 mΩ at 25°C (temperature coefficient
+0.4\%/°C), switching at f\textsubscript{sw} = 400 kHz. Operating
conditions: V\textsubscript{in} = 48 V, V\textsubscript{out} = 12 V,
I\textsubscript{out} = 15 A. The switching transitions have
t\textsubscript{rise} = 12 ns and t\textsubscript{fall} = 18 ns. The
junction temperature is 105°C. The gate charge is Q\textsubscript{g} =
45 nC at V\textsubscript{gs} = 10 V.

\textbf{Find:} (a) The duty cycle, (b) the RMS current through the
high-side MOSFET, (c) the conduction loss at operating temperature, (d)
the switching loss, (e) the gate drive loss, and (f) the total high-side
MOSFET loss.

\textbf{Solution:}

\begin{enumerate}
\def\labelenumi{(\alph{enumi})}
\item
  Duty cycle: D = V\textsubscript{out} / V\textsubscript{in} = 12 / 48 =
  \textbf{0.25 (25\%)}
\item
  RMS current through the high-side MOSFET: I\textsubscript{Q1(rms)} =
  I\textsubscript{out} × √D = 15 × √0.25 = 15 × 0.5 = \textbf{7.5 A}
\item
  R\textsubscript{DS(on)} at 105°C: R\textsubscript{DS(on)} = 22 × {[}1
  + 0.004 × (105 − 25){]} = 22 × {[}1 + 0.32{]} = 22 × 1.32 = 29.04 mΩ
  P\textsubscript{cond} = I\textsubscript{Q1(rms)}² ×
  R\textsubscript{DS(on)} = 7.5² × 0.02904 = 56.25 × 0.02904 =
  \textbf{1.63 W}
\item
  Switching loss (high-side MOSFET sees full V\textsubscript{in} and
  I\textsubscript{out} during transitions): P\textsubscript{sw} = ½ ×
  V\textsubscript{in} × I\textsubscript{out} × (t\textsubscript{rise} +
  t\textsubscript{fall}) × f\textsubscript{sw} P\textsubscript{sw} = 0.5
  × 48 × 15 × (12 + 18) × 10⁻⁹ × 400 × 10³ = 0.5 × 48 × 15 × 30 × 10⁻⁹ ×
  4 × 10⁵ P\textsubscript{sw} = 0.5 × 48 × 15 × 0.012 = \textbf{4.32 W}
\item
  Gate drive loss: P\textsubscript{gate} = Q\textsubscript{g} ×
  V\textsubscript{gs} × f\textsubscript{sw} = 45 × 10⁻⁹ × 10 × 400 × 10³
  = \textbf{0.18 W}
\item
  Total high-side MOSFET loss: P\textsubscript{total} = 1.63 + 4.32 +
  0.18 = \textbf{6.13 W}
\end{enumerate}

\begin{center}\rule{0.5\linewidth}{0.5pt}\end{center}

\section{Problem 10.6.2}\label{problem-10.6.2}

\textbf{Given:} An IGBT module dissipates 350 W total (IGBT: 220 W,
freewheeling diode: 130 W). The thermal path has:
R\textsubscript{θJC(IGBT)} = 0.12 °C/W, R\textsubscript{θJC(diode)} =
0.20 °C/W, R\textsubscript{θCS} = 0.03 °C/W (shared case-to-sink), and
R\textsubscript{θSA} = 0.10 °C/W. Ambient temperature is
T\textsubscript{A} = 45°C. Maximum T\textsubscript{J} = 150°C for both
devices.

\textbf{Find:} (a) The heat sink temperature, (b) the case temperature,
(c) the IGBT junction temperature, (d) the diode junction temperature,
and (e) whether the design meets thermal limits.

\textbf{Solution:}

\begin{enumerate}
\def\labelenumi{(\alph{enumi})}
\item
  Heat sink temperature: T\textsubscript{HS} = T\textsubscript{A} +
  P\textsubscript{total} × R\textsubscript{θSA} = 45 + 350 × 0.10 = 45 +
  35 = \textbf{80°C}
\item
  Case temperature (total power flows through shared
  R\textsubscript{θCS}): T\textsubscript{C} = T\textsubscript{HS} +
  P\textsubscript{total} × R\textsubscript{θCS} = 80 + 350 × 0.03 = 80 +
  10.5 = \textbf{90.5°C}
\item
  IGBT junction temperature: T\textsubscript{J(IGBT)} =
  T\textsubscript{C} + P\textsubscript{IGBT} ×
  R\textsubscript{θJC(IGBT)} = 90.5 + 220 × 0.12 = 90.5 + 26.4 =
  \textbf{116.9°C}
\item
  Diode junction temperature: T\textsubscript{J(diode)} =
  T\textsubscript{C} + P\textsubscript{diode} ×
  R\textsubscript{θJC(diode)} = 90.5 + 130 × 0.20 = 90.5 + 26.0 =
  \textbf{116.5°C}
\item
  Both T\textsubscript{J(IGBT)} = 116.9°C and T\textsubscript{J(diode)}
  = 116.5°C are below T\textsubscript{J(max)} = 150°C. IGBT margin: 150
  − 116.9 = \textbf{33.1°C margin} (adequate) Diode margin: 150 − 116.5
  = \textbf{33.5°C margin} (adequate) The design \textbf{meets thermal
  limits}.
\end{enumerate}

\begin{center}\rule{0.5\linewidth}{0.5pt}\end{center}

\section{Problem 10.6.3}\label{problem-10.6.3}

\textbf{Given:} A power supply designer must select a heat sink for a
MOSFET dissipating 8.5 W. The MOSFET has R\textsubscript{θJC} = 1.2 °C/W
and R\textsubscript{θCS} = 0.5 °C/W (with thermal pad). The maximum
junction temperature is T\textsubscript{J(max)} = 150°C, the ambient is
50°C, and the designer wants a 25°C safety margin.

\textbf{Find:} (a) The maximum allowable R\textsubscript{θSA}, (b)
whether a natural convection heat sink with R\textsubscript{θSA} = 6.5
°C/W is adequate, and (c) the required airflow velocity if a forced-air
heat sink has R\textsubscript{θSA} = 12 / (1 + 0.8v) °C/W where v is air
velocity in m/s.

\textbf{Solution:}

\begin{enumerate}
\def\labelenumi{(\alph{enumi})}
\item
  Target junction temperature: T\textsubscript{J(target)} = 150 − 25 =
  125°C Allowable total temperature rise: ΔT = 125 − 50 = 75°C
  R\textsubscript{θJA(max)} = ΔT / P = 75 / 8.5 = 8.82 °C/W
  R\textsubscript{θSA(max)} = R\textsubscript{θJA(max)} −
  R\textsubscript{θJC} − R\textsubscript{θCS} = 8.82 − 1.2 − 0.5 =
  \textbf{7.12 °C/W}
\item
  Natural convection heat sink: R\textsubscript{θSA} = 6.5 °C/W
  \textless{} 7.12 °C/W. T\textsubscript{J} = 50 + 8.5 × (1.2 + 0.5 +
  6.5) = 50 + 8.5 × 8.2 = 50 + 69.7 = 119.7°C This is below 125°C with
  5.3°C margin. The heat sink is \textbf{adequate}.
\item
  Required airflow if forced-air is needed (e.g., in a more constrained
  design): R\textsubscript{θSA} ≤ 7.12 °C/W: 12 / (1 + 0.8v) ≤ 7.12 1 +
  0.8v ≥ 12 / 7.12 = 1.685 0.8v ≥ 0.685 v ≥ 0.856 m/s ≈ \textbf{0.9 m/s}
  (approximately 170 ft/min, achievable with a small fan)
\end{enumerate}

\begin{center}\rule{0.5\linewidth}{0.5pt}\end{center}

\section{Problem 10.6.4}\label{problem-10.6.4}

\textbf{Given:} A SiC MOSFET module dissipates 600 W and is mounted on a
liquid-cooled cold plate. The thermal interface uses sintered silver
with thermal conductivity k\textsubscript{TIM} = 200 W/m·K and bond-line
thickness t = 30 μm over a 50 mm × 60 mm contact area. The cold plate
R\textsubscript{θ(plate-to-coolant)} = 0.015 °C/W, the module
R\textsubscript{θJC} = 0.06 °C/W, and coolant inlet temperature is 30°C.

\textbf{Find:} (a) The thermal resistance of the sintered silver
interface, (b) the total junction-to-coolant thermal resistance, (c) the
junction temperature, and (d) the comparison to thermal grease (k = 3.5
W/m·K, t = 50 μm).

\textbf{Solution:}

\begin{enumerate}
\def\labelenumi{(\alph{enumi})}
\item
  TIM contact area: A = 0.050 × 0.060 = 3.0 × 10⁻³ m²
  R\textsubscript{θCS} = t / (k × A) = 30 × 10⁻⁶ / (200 × 3.0 × 10⁻³) =
  30 × 10⁻⁶ / 0.60 = \textbf{0.00005 °C/W} (negligible)
\item
  Total thermal resistance: R\textsubscript{θ(total)} =
  R\textsubscript{θJC} + R\textsubscript{θCS} +
  R\textsubscript{θ(plate)} = 0.06 + 0.00005 + 0.015 = \textbf{0.0751
  °C/W}
\item
  Junction temperature: T\textsubscript{J} = T\textsubscript{coolant} +
  P × R\textsubscript{θ(total)} = 30 + 600 × 0.0751 = 30 + 45.0 =
  \textbf{75.0°C}
\item
  With thermal grease: R\textsubscript{θCS(grease)} = 50 × 10⁻⁶ / (3.5 ×
  3.0 × 10⁻³) = 50 × 10⁻⁶ / 0.0105 = 0.00476 °C/W
  R\textsubscript{θ(total,grease)} = 0.06 + 0.00476 + 0.015 = 0.0798
  °C/W T\textsubscript{J(grease)} = 30 + 600 × 0.0798 = 30 + 47.9 =
  \textbf{77.9°C}
\end{enumerate}

The sintered silver saves only 2.9°C compared to grease in this case
because R\textsubscript{θJC} and R\textsubscript{θ(plate)} dominate.
Sintered silver becomes more advantageous at higher power densities
where R\textsubscript{θCS} is a larger fraction of the total.

\begin{center}\rule{0.5\linewidth}{0.5pt}\end{center}

\section{Problem 10.6.5}\label{problem-10.6.5}

\textbf{Given:} A liquid cooling loop for a 100 kW power converter uses
deionized water (c\textsubscript{p} = 4,180 J/kg·°C, ρ = 998 kg/m³). The
converter efficiency is 97\%, and all losses are removed by the cooling
loop. The maximum allowable coolant temperature rise is 8°C.

\textbf{Find:} (a) The total heat to be removed, (b) the required
coolant flow rate, (c) the volumetric flow rate in L/min, and (d) the
pump power if the system pressure drop is 150 kPa and pump efficiency is
65\%.

\textbf{Solution:}

\begin{enumerate}
\def\labelenumi{(\alph{enumi})}
\item
  Total heat dissipation: P\textsubscript{loss} = P\textsubscript{out} ×
  (1/η − 1) = 100,000 × (1/0.97 − 1) = 100,000 × 0.03093 = \textbf{3,093
  W}
\item
  Required mass flow rate: ṁ = Q̇ / (c\textsubscript{p} × ΔT) = 3,093 /
  (4,180 × 8) = 3,093 / 33,440 = \textbf{0.0925 kg/s}
\item
  Volumetric flow rate: V̇ = ṁ / ρ = 0.0925 / 998 = 9.27 × 10⁻⁵ m³/s =
  9.27 × 10⁻⁵ × 60,000 = \textbf{5.56 L/min}
\item
  Pump hydraulic power: P\textsubscript{hydraulic} = ΔP × V̇ = 150,000 ×
  9.27 × 10⁻⁵ = 13.9 W P\textsubscript{pump} =
  P\textsubscript{hydraulic} / η\textsubscript{pump} = 13.9 / 0.65 =
  \textbf{21.4 W}
\end{enumerate}

The pump power is a small fraction (0.7\%) of the total converter loss,
confirming that liquid cooling is energy-efficient.

\begin{center}\rule{0.5\linewidth}{0.5pt}\end{center}

\section{Problem 10.6.6}\label{problem-10.6.6}

\textbf{Given:} A two-stage EMI filter must be designed for a 500 W PFC
converter switching at f\textsubscript{sw} = 65 kHz. Conducted emission
testing shows the converter exceeds CISPR 32 Class B quasi-peak limits
by 22 dB at 65 kHz and 15 dB at 130 kHz. A 6 dB design margin is
required.

\textbf{Find:} (a) The required attenuation at the fundamental and
second harmonic, (b) the filter corner frequency for a single-stage LC
filter, (c) suitable component values, and (d) whether a single stage is
sufficient at 130 kHz.

\textbf{Solution:}

\begin{enumerate}
\def\labelenumi{(\alph{enumi})}
\item
  Required attenuation: At 65 kHz: 22 + 6 = \textbf{28 dB} At 130 kHz:
  15 + 6 = \textbf{21 dB}
\item
  A single-stage LC (second-order) filter has −40 dB/decade roll-off.
  Required attenuation of 28 dB = 10\textsuperscript{28/20} = 25.1× at
  65 kHz. For a second-order filter: attenuation = (f/f₀)² f₀ = f /
  √(attenuation) = 65,000 / √25.1 = 65,000 / 5.01 = \textbf{12,974 Hz ≈
  13 kHz}
\item
  Choose L\textsubscript{DM} = 200 μH: C = 1 / (4π²f₀²L) = 1 / (4 × 9.87
  × (13,000)² × 200 × 10⁻⁶) C = 1 / (4 × 9.87 × 1.69 × 10⁸ × 2 × 10⁻⁴) =
  1 / (1.332 × 10⁶) = \textbf{0.75 μF} Use \textbf{L = 200 μH} and
  \textbf{C = 0.82 μF} (standard X2 capacitor value).
\item
  Verify at 130 kHz with actual component values: f₀ = 1/(2π√(200 × 10⁻⁶
  × 0.82 × 10⁻⁶)) = 1/(2π × 1.281 × 10⁻⁵) = 1/(8.05 × 10⁻⁵) = 12,422 Hz
  Attenuation at 130 kHz = (130,000/12,422)² = (10.47)² = 109.5 = 40.8
  dB Required: 21 dB. The single-stage filter provides \textbf{40.8 dB},
  which exceeds the 21 dB requirement by 19.8 dB. A single stage is
  \textbf{sufficient} at both frequencies.
\end{enumerate}

\begin{center}\rule{0.5\linewidth}{0.5pt}\end{center}

\section{Problem 10.6.7}\label{problem-10.6.7}

\textbf{Given:} An inverter has three IGBT half-bridge modules, each
dissipating 180 W. All three are mounted on a common aluminum heat sink
with forced-air cooling. The heat sink has a base area of 200 mm × 300
mm, fin height of 50 mm, and 15 fins at 2 mm thickness with 12 mm
spacing. The fan provides 2.5 m/s airflow. Ambient temperature is 40°C.

\textbf{Find:} (a) The total heat to be dissipated, (b) the required
R\textsubscript{θSA} if each module has R\textsubscript{θJC} = 0.15
°C/W, R\textsubscript{θCS} = 0.04 °C/W, and T\textsubscript{J(max)} =
150°C with 20°C margin, (c) the heat sink fin surface area, and (d)
whether the heat sink is adequate (assume convective heat transfer
coefficient h = 35 W/m²·°C for 2.5 m/s airflow).

\textbf{Solution:}

\begin{enumerate}
\def\labelenumi{(\alph{enumi})}
\item
  Total heat: P\textsubscript{total} = 3 × 180 = \textbf{540 W}
\item
  Target T\textsubscript{J} = 150 − 20 = 130°C. For the worst-case
  module: T\textsubscript{J} = T\textsubscript{HS} +
  P\textsubscript{module} × (R\textsubscript{θJC} +
  R\textsubscript{θCS}) 130 = T\textsubscript{HS} + 180 × (0.15 + 0.04)
  = T\textsubscript{HS} + 34.2 T\textsubscript{HS(max)} = 130 − 34.2 =
  95.8°C R\textsubscript{θSA} = (T\textsubscript{HS} −
  T\textsubscript{A}) / P\textsubscript{total} = (95.8 − 40) / 540 =
  55.8 / 540 = \textbf{0.103 °C/W}
\item
  Fin surface area (both sides of each fin plus the base between fins):
  Fin area per fin (both sides): 2 × 0.050 × 0.300 = 0.030 m² Total fin
  area: 15 × 0.030 = 0.45 m² Base area between fins: 14 × 0.012 × 0.300
  = 0.0504 m² Total surface area: A\textsubscript{total} = 0.45 + 0.0504
  = \textbf{0.50 m²}
\item
  Thermal resistance of the heat sink: R\textsubscript{θSA} = 1 / (h × A
  × η\textsubscript{fin}) where η\textsubscript{fin} ≈ 0.85 for aluminum
  fins of this geometry. R\textsubscript{θSA} = 1 / (35 × 0.50 × 0.85) =
  1 / 14.88 = \textbf{0.067 °C/W}
\end{enumerate}

Since 0.067 \textless{} 0.103 °C/W required, the heat sink is
\textbf{adequate} with margin. T\textsubscript{HS} = 40 + 540 × 0.067 =
40 + 36.2 = 76.2°C T\textsubscript{J(worst)} = 76.2 + 34.2 = 110.4°C
(39.6°C below the 150°C limit)

\begin{center}\rule{0.5\linewidth}{0.5pt}\end{center}

\section{Problem 10.6.8}\label{problem-10.6.8}

\textbf{Given:} A phase-change thermal interface material (PCM) has
thermal conductivity k = 4.0 W/m·K in its softened state and a thickness
of 0.15 mm. It is used between a TO-247 MOSFET package (contact area 15
mm × 20 mm) and a heat sink. Compare this to thermal grease with k = 2.5
W/m·K at 0.05 mm bond-line thickness.

\textbf{Find:} (a) The thermal resistance of the PCM interface, (b) the
thermal resistance of the thermal grease interface, (c) the temperature
difference across each TIM at 50 W dissipation, and (d) which TIM is
preferred and why.

\textbf{Solution:}

\begin{enumerate}
\def\labelenumi{(\alph{enumi})}
\item
  PCM contact area: A = 0.015 × 0.020 = 3.0 × 10⁻⁴ m²
  R\textsubscript{θ(PCM)} = t / (k × A) = 0.15 × 10⁻³ / (4.0 × 3.0 ×
  10⁻⁴) = 1.5 × 10⁻⁴ / 1.2 × 10⁻³ = \textbf{0.125 °C/W}
\item
  Thermal grease: R\textsubscript{θ(grease)} = t / (k × A) = 0.05 × 10⁻³
  / (2.5 × 3.0 × 10⁻⁴) = 5.0 × 10⁻⁵ / 7.5 × 10⁻⁴ = \textbf{0.067 °C/W}
\item
  Temperature difference at 50 W: PCM: ΔT = 50 × 0.125 = \textbf{6.25°C}
  Grease: ΔT = 50 × 0.067 = \textbf{3.33°C}
\item
  Despite higher thermal conductivity (4.0 vs 2.5 W/m·K), the PCM has
  \textbf{higher thermal resistance} because its thickness (0.15 mm) is
  three times that of the grease (0.05 mm). The thermal grease provides
  a \textbf{2.92°C lower temperature drop}. However, the PCM offers
  easier handling (solid at room temperature), no pump-out over time,
  and consistent performance --- making it preferred when long-term
  reliability outweighs the modest thermal penalty.
\end{enumerate}

\begin{center}\rule{0.5\linewidth}{0.5pt}\end{center}

\section{Problem 10.6.9}\label{problem-10.6.9}

\textbf{Given:} A 1 kW AC-DC power supply has conducted emissions
exceeding CISPR 32 Class B limits. Measurements with a LISN show:
differential-mode (DM) noise dominates below 500 kHz at 85 dBμV peak,
and common-mode (CM) noise dominates above 1 MHz at 72 dBμV peak. The
Class B quasi-peak limit is 56 dBμV at 500 kHz and 60 dBμV at 1 MHz.

\textbf{Find:} (a) The required DM attenuation at 500 kHz, (b) the
required CM attenuation at 1 MHz, (c) suitable X-capacitor and DM
inductor values for the DM filter, and (d) suitable CM choke inductance
and Y-capacitor values.

\textbf{Solution:}

\begin{enumerate}
\def\labelenumi{(\alph{enumi})}
\item
  DM attenuation required at 500 kHz: Excess = 85 − 56 = 29 dB. With 6
  dB margin: \textbf{35 dB required}
\item
  CM attenuation required at 1 MHz: Excess = 72 − 60 = 12 dB. With 6 dB
  margin: \textbf{18 dB required}
\item
  DM filter design (second-order LC, −40 dB/decade): 35 dB =
  10\textsuperscript{35/20} = 56.2× attenuation f₀ = 500,000 / √56.2 =
  500,000 / 7.50 = 66,700 Hz
\end{enumerate}

Choose C\textsubscript{X} = 0.47 μF (X2 film capacitor):
L\textsubscript{DM} = 1/(4π²f₀²C) = 1/(4 × 9.87 × (66,700)² × 0.47 ×
10⁻⁶) L\textsubscript{DM} = 1/(4 × 9.87 × 4.449 × 10⁹ × 4.7 × 10⁻⁷) =
1/(82,700) = \textbf{12.1 μH} Use \textbf{L\textsubscript{DM} = 15 μH}
and \textbf{C\textsubscript{X} = 0.47 μF} for margin.

\begin{enumerate}
\def\labelenumi{(\alph{enumi})}
\setcounter{enumi}{3}
\tightlist
\item
  CM filter design: 18 dB = 10\textsuperscript{18/20} = 7.94×
  attenuation at 1 MHz f₀\textsubscript{(CM)} = 1,000,000 / √7.94 =
  1,000,000 / 2.82 = 354,900 Hz
\end{enumerate}

Choose C\textsubscript{Y} = 2.2 nF (Y2 ceramic, limited by safety
leakage current): L\textsubscript{CM} = 1/(4π²f₀²C) = 1/(4 × 9.87 ×
(354,900)² × 2.2 × 10⁻⁹) L\textsubscript{CM} = 1/(4 × 9.87 × 1.260 ×
10¹¹ × 2.2 × 10⁻⁹) = 1/(10,940) = \textbf{91 μH} Use a \textbf{100 μH
common-mode choke} and \textbf{2.2 nF Y2 capacitors} (two, one per line
to ground).

\begin{center}\rule{0.5\linewidth}{0.5pt}\end{center}

\section{Problem 10.6.10}\label{problem-10.6.10}

\textbf{Given:} A power module with two paralleled MOSFET die has the
following thermal network: each die dissipates P₁ = P₂ = 75 W, with
R\textsubscript{θJC₁} = R\textsubscript{θJC₂} = 0.25 °C/W, thermal
coupling between die R\textsubscript{θ12} = 1.5 °C/W, a shared
R\textsubscript{θCS} = 0.02 °C/W, and heat sink R\textsubscript{θSA} =
0.08 °C/W. Ambient is 35°C.

\textbf{Find:} (a) The heat sink temperature, (b) the case temperature,
(c) the junction temperature of each die accounting for self-heating
only, (d) the additional temperature rise from thermal coupling, and (e)
the total junction temperature of each die.

\textbf{Solution:}

\begin{enumerate}
\def\labelenumi{(\alph{enumi})}
\item
  Heat sink temperature: P\textsubscript{total} = 75 + 75 = 150 W
  T\textsubscript{HS} = T\textsubscript{A} + P\textsubscript{total} ×
  R\textsubscript{θSA} = 35 + 150 × 0.08 = 35 + 12 = \textbf{47°C}
\item
  Case temperature: T\textsubscript{C} = T\textsubscript{HS} +
  P\textsubscript{total} × R\textsubscript{θCS} = 47 + 150 × 0.02 = 47 +
  3 = \textbf{50°C}
\item
  Junction temperature from self-heating only: T\textsubscript{J(self)}
  = T\textsubscript{C} + P × R\textsubscript{θJC} = 50 + 75 × 0.25 = 50
  + 18.75 = \textbf{68.75°C}
\item
  Thermal coupling from adjacent die: The die temperatures are coupled
  through the substrate. The fraction of adjacent die power that appears
  as additional temperature rise at die 1 depends on the relative
  impedances in the coupling path:
\end{enumerate}

Coupling factor = R\textsubscript{θJC} / (R\textsubscript{θJC} +
R\textsubscript{θ12}) = 0.25 / (0.25 + 1.5) = 0.25 / 1.75 = 0.143

ΔT\textsubscript{coupling} = P₂ × R\textsubscript{θJC} × coupling factor
= 75 × 0.25 × 0.143 = \textbf{2.68°C}

\begin{enumerate}
\def\labelenumi{(\alph{enumi})}
\setcounter{enumi}{4}
\tightlist
\item
  Total junction temperature: T\textsubscript{J} =
  T\textsubscript{J(self)} + ΔT\textsubscript{coupling} = 68.75 + 2.68 =
  \textbf{71.4°C}
\end{enumerate}

Both die are symmetric, so they reach the same temperature. The thermal
coupling adds only 2.68°C because the coupling resistance is much larger
than the self-heating resistance (1.5 °C/W vs 0.25 °C/W), meaning
relatively little heat transfers between die through the substrate.

\chapter{Chapter 10 --- Section 10.7: Power Factor
Correction}\label{chapter-10-section-10.7-power-factor-correction}

Practice problems covering active PFC topologies, boost PFC design,
bridgeless PFC, critical conduction mode, interleaved PFC, power supply
protection circuits, soft start, and inrush current limiting.

\begin{center}\rule{0.5\linewidth}{0.5pt}\end{center}

\section{Problem 10.7.1}\label{problem-10.7.1}

\textbf{Given:} A boost PFC converter operates from a universal input of
120 V\textsubscript{rms} (worst case) at 60 Hz and produces a 400 V DC
output. The rated output power is 350 W. The switching frequency is 100
kHz and the boost inductor is 600 μH.

\textbf{Find:} (a) The RMS input current, (b) the peak input current,
(c) the duty cycle at the peak of the AC input, (d) the inductor current
ripple at the peak of the AC input, and (e) whether the converter
remains in CCM at the peak.

\textbf{Solution:}

\begin{enumerate}
\def\labelenumi{(\alph{enumi})}
\item
  RMS input current (assuming PF ≈ 1.0): I\textsubscript{in(rms)} = P /
  V\textsubscript{in(rms)} = 350 / 120 = \textbf{2.917 A}
\item
  Peak input current: I\textsubscript{in(peak)} =
  I\textsubscript{in(rms)} × √2 = 2.917 × 1.414 = \textbf{4.124 A}
\item
  Peak input voltage: V\textsubscript{in(peak)} = 120 × √2 = 169.7 V
  Duty cycle at peak: D = 1 −
  V\textsubscript{in(peak)}/V\textsubscript{out} = 1 − 169.7/400 = 1 −
  0.4243 = \textbf{0.576 (57.6\%)}
\item
  Inductor current ripple at the peak: ΔI\textsubscript{L} =
  V\textsubscript{in(peak)} × D / (L × f\textsubscript{sw}) = 169.7 ×
  0.576 / (600 × 10⁻⁶ × 100 × 10³) ΔI\textsubscript{L} = 97.75 / 60 =
  \textbf{1.629 A\textsubscript{pp}}
\item
  Check CCM: The minimum inductor current at the peak is:
  I\textsubscript{min} = I\textsubscript{in(peak)} −
  ΔI\textsubscript{L}/2 = 4.124 − 1.629/2 = 4.124 − 0.815 =
  \textbf{3.309 A}
\end{enumerate}

Since I\textsubscript{min} = 3.309 A \textgreater{} 0, the converter
\textbf{remains in CCM} at the peak. However, near the zero crossings of
the AC input, the average current is much lower and the converter will
enter DCM, which is typical for CCM boost PFC at light instantaneous
current.

\begin{center}\rule{0.5\linewidth}{0.5pt}\end{center}

\section{Problem 10.7.2}\label{problem-10.7.2}

\textbf{Given:} A critical conduction mode (CrCM) boost PFC operates
from 230 V\textsubscript{rms}, 50 Hz and produces 385 V DC at 150 W. The
boost inductor is 1.2 mH.

\textbf{Find:} (a) The switching frequency at the peak of the AC input,
(b) the switching frequency at 30° past the zero crossing, (c) the peak
inductor current at the AC peak, and (d) the range of switching
frequencies over the AC cycle.

\textbf{Solution:}

\begin{enumerate}
\def\labelenumi{(\alph{enumi})}
\tightlist
\item
  In CrCM, the inductor current ramps from zero to a peak and back to
  zero each cycle. V\textsubscript{in(peak)} = 230 × √2 = 325.3 V
  D\textsubscript{peak} = 1 −
  V\textsubscript{in(peak)}/V\textsubscript{out} = 1 − 325.3/385 =
  0.1551
\end{enumerate}

The instantaneous average input power at the peak: p(t) = 2P/T ×
\ldots{} The average current at the peak: I\textsubscript{avg(peak)} =
(P × √2) / V\textsubscript{in(rms)} × (2/π is not needed for peak)
\ldots{}

Simpler approach: I\textsubscript{avg(peak)} = I\textsubscript{in(rms)}
× √2 = (150/230) × √2 = 0.6522 × 1.414 = 0.922 A

In CrCM, the peak current = 2 × I\textsubscript{avg} = 2 × 0.922 = 1.844
A. On-time: t\textsubscript{on} = I\textsubscript{peak} × L /
V\textsubscript{in(peak)} = 1.844 × 1.2 × 10⁻³ / 325.3 = 6.80 μs
Off-time: t\textsubscript{off} = I\textsubscript{peak} × L /
(V\textsubscript{out} − V\textsubscript{in(peak)}) = 1.844 × 1.2 × 10⁻³
/ (385 − 325.3) = 2.213 × 10⁻³ / 59.7 = 37.07 μs T\textsubscript{sw} =
t\textsubscript{on} + t\textsubscript{off} = 6.80 + 37.07 = 43.87 μs
f\textsubscript{sw(peak)} = 1 / 43.87 × 10⁻⁶ = \textbf{22.8 kHz}

\begin{enumerate}
\def\labelenumi{(\alph{enumi})}
\setcounter{enumi}{1}
\item
  At 30° past zero crossing: V\textsubscript{in}(30°) = 325.3 × sin(30°)
  = 325.3 × 0.5 = 162.65 V I\textsubscript{avg}(30°) = 0.922 × sin(30°)
  = 0.461 A (sinusoidal current envelope) I\textsubscript{peak}(30°) = 2
  × 0.461 = 0.922 A t\textsubscript{on} = 0.922 × 1.2 × 10⁻³ / 162.65 =
  6.80 μs t\textsubscript{off} = 0.922 × 1.2 × 10⁻³ / (385 − 162.65) =
  1.106 × 10⁻³ / 222.35 = 4.975 μs T\textsubscript{sw} = 6.80 + 4.975 =
  11.78 μs f\textsubscript{sw}(30°) = 1 / 11.78 × 10⁻⁶ = \textbf{84.9
  kHz}
\item
  Peak inductor current at the AC peak: I\textsubscript{L(peak)} = 2 ×
  I\textsubscript{avg(peak)} = \textbf{1.844 A} (as calculated above)
\item
  The switching frequency is lowest at the AC peak (\textasciitilde23
  kHz) and highest near the zero crossings (where it can exceed 200 kHz
  in theory but is typically clamped). Practical range:
  \textbf{\textasciitilde23 kHz to \textasciitilde150 kHz} (clamped by
  the controller's maximum frequency limit).
\end{enumerate}

\begin{center}\rule{0.5\linewidth}{0.5pt}\end{center}

\section{Problem 10.7.3}\label{problem-10.7.3}

\textbf{Given:} A 3 kW interleaved boost PFC uses two paralleled boost
stages, each operating at f\textsubscript{sw} = 70 kHz with 180° phase
shift. Input: 240 V\textsubscript{rms}, 50 Hz. Output: 400 V DC. Each
boost inductor is 400 μH.

\textbf{Find:} (a) The current per phase, (b) the inductor ripple
current per phase at the AC peak, (c) the effective input ripple
frequency, and (d) the input current ripple cancellation at D = 0.5.

\textbf{Solution:}

\begin{enumerate}
\def\labelenumi{(\alph{enumi})}
\item
  Total input current: I\textsubscript{in(rms)} = P/V\textsubscript{in}
  = 3,000/240 = 12.5 A Current per phase: I\textsubscript{phase} =
  12.5/2 = \textbf{6.25 A\textsubscript{rms}}
\item
  Peak input voltage: V\textsubscript{in(peak)} = 240√2 = 339.4 V Duty
  at peak: D = 1 − 339.4/400 = 0.1515 Ripple per phase: ΔI =
  V\textsubscript{in(peak)} × D / (L × f\textsubscript{sw}) = 339.4 ×
  0.1515 / (400 × 10⁻⁶ × 70 × 10³) ΔI = 51.42 / 28 = \textbf{1.836
  A\textsubscript{pp}} per phase
\item
  Effective input ripple frequency: f\textsubscript{ripple} = 2 ×
  f\textsubscript{sw} = 2 × 70,000 = \textbf{140 kHz} (doubled due to
  interleaving)
\item
  At D = 0.5 (which occurs when V\textsubscript{in} =
  V\textsubscript{out}/2 = 200 V during the AC cycle): With 180°
  interleaving and D = 0.5, the two phases' ripple currents are
  perfectly complementary --- when one phase's current is ramping up,
  the other is ramping down by the same amount. The net input current
  ripple is: ΔI\textsubscript{input} = \textbf{0 A} (complete
  cancellation at D = 0.5)
\end{enumerate}

This is a key advantage of interleaved PFC: the total input ripple is
dramatically reduced at and near D = 0.5, reducing EMI filter
requirements. At other duty cycles, partial cancellation still reduces
the total ripple by 50-80\% compared to a single-phase converter.

\begin{center}\rule{0.5\linewidth}{0.5pt}\end{center}

\section{Problem 10.7.4}\label{problem-10.7.4}

\textbf{Given:} A 48 V, 30 A DC power supply has an output capacitor
bank of 3,300 μF. The soft-start circuit ramps the output from 0 to 48 V
in 15 ms. The overcurrent protection (OCP) threshold is 38 A.

\textbf{Find:} (a) The voltage ramp rate, (b) the capacitor charging
current during soft start, (c) the peak total current (charging + load)
near the end of the ramp, (d) whether OCP will trip, and (e) the minimum
soft-start time to stay below OCP.

\textbf{Solution:}

\begin{enumerate}
\def\labelenumi{(\alph{enumi})}
\item
  Voltage ramp rate: dV/dt = 48 / 15 × 10⁻³ = \textbf{3,200 V/s}
\item
  Capacitor charging current: I\textsubscript{cap} = C × dV/dt = 3,300 ×
  10⁻⁶ × 3,200 = \textbf{10.56 A}
\item
  Peak total current near end of ramp (worst case, full load current
  developing as output approaches 48 V): I\textsubscript{peak} =
  I\textsubscript{cap} + I\textsubscript{load} = 10.56 + 30 =
  \textbf{40.56 A}
\item
  Since I\textsubscript{peak} = 40.56 A \textgreater{} OCP threshold of
  38 A, OCP \textbf{will trip}, causing the power supply to enter hiccup
  mode and fail to start properly.
\item
  Minimum soft-start time: Maximum allowable capacitor current:
  I\textsubscript{cap(max)} = 38 − 30 = 8 A dV/dt\textsubscript{max} =
  I\textsubscript{cap(max)} / C = 8 / 3,300 × 10⁻⁶ = 2,424 V/s
  t\textsubscript{ss(min)} = 48 / 2,424 = 0.0198 s = \textbf{19.8 ms}
\end{enumerate}

Use \textbf{25 ms} for adequate margin, giving I\textsubscript{cap} =
3,300 × 10⁻⁶ × (48/0.025) = 6.34 A, and I\textsubscript{peak} = 6.34 +
30 = 36.3 A (1.7 A below OCP).

\begin{center}\rule{0.5\linewidth}{0.5pt}\end{center}

\section{Problem 10.7.5}\label{problem-10.7.5}

\textbf{Given:} A 750 W offline power supply has a 450 μF bulk input
capacitor charged through a full-bridge diode rectifier from a 230
V\textsubscript{rms}, 50 Hz source. The inrush current is limited by an
NTC thermistor with cold resistance R\textsubscript{cold} = 10 Ω and
steady-state hot resistance R\textsubscript{hot} = 0.5 Ω.

\textbf{Find:} (a) The peak voltage across the capacitor at steady
state, (b) the worst-case peak inrush current (if power is applied at
the peak of the AC line), (c) the steady-state power dissipated in the
NTC, and (d) the NTC surface temperature rise if its thermal resistance
to ambient is 25 °C/W.

\textbf{Solution:}

\begin{enumerate}
\def\labelenumi{(\alph{enumi})}
\item
  Peak capacitor voltage: V\textsubscript{cap} = V\textsubscript{peak} =
  230 × √2 = \textbf{325.3 V}
\item
  Worst-case inrush (capacitor discharged, power applied at line peak):
  I\textsubscript{inrush} = V\textsubscript{peak} /
  R\textsubscript{cold} = 325.3 / 10 = \textbf{32.5 A}
\end{enumerate}

Without the NTC: I\textsubscript{inrush} would be limited only by source
impedance and wiring resistance (typically 0.1-0.5 Ω), yielding
650-3,253 A --- potentially destructive. The NTC reduces inrush by
approximately 100×.

\begin{enumerate}
\def\labelenumi{(\alph{enumi})}
\setcounter{enumi}{2}
\item
  Steady-state input current: I\textsubscript{in(rms)} = P /
  (V\textsubscript{in} × PF). Without PFC, PF ≈ 0.60:
  I\textsubscript{in(rms)} = 750 / (230 × 0.60) = 5.43 A
  P\textsubscript{NTC} = I\textsubscript{in(rms)}² ×
  R\textsubscript{hot} = 5.43² × 0.5 = 29.5 × 0.5 = \textbf{14.7 W}
\item
  NTC temperature rise: ΔT = P × R\textsubscript{θ} = 14.7 × 25 =
  \textbf{367.5°C}
\end{enumerate}

This is an unrealistically high temperature, indicating that the 0.5 Ω
hot resistance will actually be lower at equilibrium (NTC resistance
continues to decrease with temperature). In practice, the NTC settles
around 150-200°C surface temperature, with R\textsubscript{hot}
decreasing further. This also highlights the 14.7 W of continuous
dissipation as an efficiency penalty of 14.7/750 = 2.0\%, which is why
high-power supplies (\textgreater500 W) use an active bypass relay
instead.

\begin{center}\rule{0.5\linewidth}{0.5pt}\end{center}

\section{Problem 10.7.6}\label{problem-10.7.6}

\textbf{Given:} A totem-pole bridgeless PFC converter uses GaN HEMTs
with R\textsubscript{DS(on)} = 35 mΩ for the high-frequency switching
leg and silicon MOSFETs with R\textsubscript{DS(on)} = 6 mΩ for the
low-frequency commutation leg. Input: 230 V\textsubscript{rms}, 50 Hz.
Output: 400 V DC. Power: 2 kW. Switching frequency: 140 kHz.

\textbf{Find:} (a) The RMS input current, (b) the conduction losses in
the high-frequency GaN switches, (c) the conduction losses in the
low-frequency silicon MOSFETs, (d) the total conduction losses, and (e)
the efficiency improvement over a conventional boost PFC with a diode
bridge (V\textsubscript{F} = 0.85 V per diode) and a silicon MOSFET
(R\textsubscript{DS(on)} = 40 mΩ).

\textbf{Solution:}

\begin{enumerate}
\def\labelenumi{(\alph{enumi})}
\item
  I\textsubscript{in(rms)} = P / V\textsubscript{in} = 2,000 / 230 =
  \textbf{8.70 A}
\item
  The high-frequency GaN switches carry current with an RMS value that
  depends on the duty cycle. With average D ≈ 0.42 for 230 V / 400 V
  operation: Active switch RMS: I\textsubscript{Q(rms)} =
  I\textsubscript{in} × √(D\textsubscript{avg}) = 8.70 × √0.42 = 8.70 ×
  0.648 = 5.64 A Sync rectifier RMS: I\textsubscript{SR(rms)} =
  I\textsubscript{in} × √(1 − D\textsubscript{avg}) = 8.70 × √0.58 =
  8.70 × 0.762 = 6.63 A P\textsubscript{GaN} = (5.64² + 6.63²) × 0.035 =
  (31.8 + 44.0) × 0.035 = 75.8 × 0.035 = \textbf{2.65 W}
\item
  Low-frequency silicon MOSFETs carry the full input current (one
  conducts per half-cycle): P\textsubscript{LF} =
  I\textsubscript{in(rms)}² × R\textsubscript{DS(on)} = 8.70² × 0.006 =
  75.7 × 0.006 = \textbf{0.45 W}
\item
  Total totem-pole conduction losses: P\textsubscript{total} = 2.65 +
  0.45 = \textbf{3.10 W}
\item
  Conventional boost PFC: Diode bridge: P\textsubscript{bridge} = 2 ×
  V\textsubscript{F} × I\textsubscript{avg} = 2 × 0.85 × (8.70 × 2√2/π)
  = 2 × 0.85 × 7.82 = 13.3 W MOSFET: P\textsubscript{Q} =
  I\textsubscript{Q(rms)}² × R\textsubscript{DS(on)} = 5.64² × 0.040 =
  1.27 W Diode (boost): \textasciitilde2 W (estimated from freewheeling
  diode losses) Total conventional: 13.3 + 1.27 + 2.0 = 16.6 W
\end{enumerate}

Reduction: 16.6 − 3.10 = 13.5 W saved Efficiency improvement: 13.5/2,000
= \textbf{0.67 percentage points}

\begin{center}\rule{0.5\linewidth}{0.5pt}\end{center}

\section{Problem 10.7.7}\label{problem-10.7.7}

\textbf{Given:} A boost PFC converter must comply with IEC 61000-3-2
Class D (equipment consuming 75-600 W). The converter power is 200 W
from a 230 V\textsubscript{rms} input. The standard limits the harmonic
current per watt to: 3rd harmonic ≤ 3.4 mA/W, 5th ≤ 1.9 mA/W, 7th ≤ 1.0
mA/W, 9th ≤ 0.5 mA/W.

\textbf{Find:} (a) The absolute harmonic current limits, (b) the maximum
allowable 3rd harmonic current, (c) whether a passive valley-fill PFC
circuit with input current THD = 45\% would comply (assume the 3rd
harmonic is 40\% of the fundamental), and (d) the required THD for
active PFC compliance.

\textbf{Solution:}

\begin{enumerate}
\def\labelenumi{(\alph{enumi})}
\item
  Absolute harmonic limits at 200 W: 3rd: 3.4 × 200 = \textbf{680 mA}
  5th: 1.9 × 200 = \textbf{380 mA} 7th: 1.0 × 200 = \textbf{200 mA} 9th:
  0.5 × 200 = \textbf{100 mA}
\item
  Maximum allowable 3rd harmonic: \textbf{680 mA = 0.68 A}
\item
  Valley-fill PFC: Fundamental input current: I₁ = P /
  V\textsubscript{in} = 200 / 230 = 0.870 A (assuming displacement PF ≈
  1) 3rd harmonic at 40\% of fundamental: I₃ = 0.40 × 0.870 = 0.348 A =
  348 mA Since 348 mA \textless{} 680 mA limit, the 3rd harmonic
  \textbf{complies}.
\end{enumerate}

5th harmonic (typical for valley-fill, \textasciitilde20\% of
fundamental): I₅ = 0.20 × 0.870 = 174 mA \textless{} 380 mA.
\textbf{Complies.} 7th harmonic (\textasciitilde14\%): I₇ = 0.14 × 0.870
= 122 mA \textless{} 200 mA. \textbf{Complies.} 9th harmonic
(\textasciitilde10\%): I₉ = 0.10 × 0.870 = 87 mA \textless{} 100 mA.
\textbf{Complies.}

The passive valley-fill circuit \textbf{meets IEC 61000-3-2 Class D} at
200 W despite 45\% THD, because the per-watt limits are generous at
lower power levels.

\begin{enumerate}
\def\labelenumi{(\alph{enumi})}
\setcounter{enumi}{3}
\tightlist
\item
  For active PFC, THD is typically below \textbf{5\%}, far exceeding
  Class D requirements. Active PFC is necessary for Class C (lighting)
  or for powers above 600 W where Class A limits apply with fixed
  current maximums.
\end{enumerate}

\begin{center}\rule{0.5\linewidth}{0.5pt}\end{center}

\section{Problem 10.7.8}\label{problem-10.7.8}

\textbf{Given:} A 1.5 kW power supply uses a crowbar overvoltage
protection circuit on the 12 V output. The crowbar thyristor fires when
V\textsubscript{out} exceeds 14.4 V (120\% of nominal). The output
capacitance is 5,000 μF and a 20 A fast-blow fuse protects the circuit.

\textbf{Find:} (a) The energy stored in the output capacitor at the OVP
trip point, (b) the peak crowbar current (assuming zero thyristor
impedance and 5 mΩ total circuit resistance), (c) the time for the fuse
to blow (assuming the fuse requires 20² × t ≤ 400 A²·s for fast-blow),
and (d) the energy dissipated in the circuit resistance before the fuse
blows.

\textbf{Solution:}

\begin{enumerate}
\def\labelenumi{(\alph{enumi})}
\item
  Energy stored at OVP trip: E = ½CV² = 0.5 × 5,000 × 10⁻⁶ × 14.4² = 0.5
  × 0.005 × 207.36 = \textbf{0.518 J}
\item
  Peak crowbar current: I\textsubscript{peak} = V / R = 14.4 / 0.005 =
  \textbf{2,880 A}
\end{enumerate}

This is an extremely high pulse current, but it decays rapidly as the
capacitor discharges.

\begin{enumerate}
\def\labelenumi{(\alph{enumi})}
\setcounter{enumi}{2}
\tightlist
\item
  The discharge is exponential with τ = RC = 0.005 × 5,000 × 10⁻⁶ = 25
  μs. The fuse I²t rating: For the fuse to blow, ∫I²dt must reach 400
  A²·s. For an exponential discharge: ∫I²dt = (V²/R²) × (RC/2) =
  (14.4²/0.005²) × (25 × 10⁻⁶/2) = (207.36/2.5 × 10⁻⁵) × 12.5 × 10⁻⁶ =
  8,294,400 × 12.5 × 10⁻⁶ = \textbf{103.7 A²·s}
\end{enumerate}

Since 103.7 A²·s \textless{} 400 A²·s, the capacitor energy alone will
\textbf{not} blow the fuse. The fuse blows only if the power supply
continues to deliver current into the shorted output. With the PSU
delivering 1,500/12 = 125 A into the short: Time to accumulate remaining
I²t: (400 − 103.7) = 296.3 A²·s at 125 A: t = 296.3 / 125² = 296.3 /
15,625 = \textbf{0.019 s = 19 ms}

\begin{enumerate}
\def\labelenumi{(\alph{enumi})}
\setcounter{enumi}{3}
\tightlist
\item
  Energy dissipated in circuit resistance from capacitor discharge: E =
  ½CV² = \textbf{0.518 J} (all capacitor energy dissipates in the
  resistance) Additional energy from PSU in 19 ms: E = V × I × t ≈ 0.5 ×
  125 × 0.019 = \textbf{1.19 J} (approximate, as voltage drops during
  the event) Total: 0.518 + 1.19 ≈ \textbf{1.7 J}
\end{enumerate}

\begin{center}\rule{0.5\linewidth}{0.5pt}\end{center}

\section{Problem 10.7.9}\label{problem-10.7.9}

\textbf{Given:} A 600 W AC-DC converter has undervoltage lockout (UVLO)
with a turn-on threshold of 80 V\textsubscript{dc} (after the input
rectifier) and a turn-off threshold of 70 V\textsubscript{dc}, providing
10 V hysteresis. The input is rectified from a 120 V\textsubscript{rms}
AC source with a 100 μF input capacitor.

\textbf{Find:} (a) The DC voltage at steady state (no PFC,
capacitor-filtered), (b) the voltage during a brownout at 60
V\textsubscript{rms} input, (c) whether the converter stays on during
the brownout, and (d) the minimum input voltage for the converter to
start.

\textbf{Solution:}

\begin{enumerate}
\def\labelenumi{(\alph{enumi})}
\tightlist
\item
  Steady-state DC voltage (peak of rectified 120 V\textsubscript{rms}):
  V\textsubscript{dc} = V\textsubscript{peak} − ΔV\textsubscript{ripple}
  V\textsubscript{peak} = 120 × √2 = 169.7 V For a rough ripple
  estimate: ΔV = P / (2 × f × C × V\textsubscript{dc}) ≈ 600 / (2 × 60 ×
  100 × 10⁻⁶ × 169.7) = 600 / 2.036 = 295 V
\end{enumerate}

This ripple estimate exceeds the peak voltage, indicating the 100 μF
capacitor is too small for 600 W at 120 V --- the ripple model breaks
down. More realistically, the minimum voltage: V\textsubscript{min} ≈
V\textsubscript{peak} × √(1 − P/(π × f × C × V\textsubscript{peak}²))

Using the conduction angle approach: with such a small capacitor at 600
W, the output is heavily rippled. The average DC is approximately:
V\textsubscript{dc(avg)} ≈ 2V\textsubscript{peak}/π = 2 × 169.7/3.14 =
\textbf{108 V} (approaching the rectified average without much
filtering)

In practice with the small capacitor: V\textsubscript{dc} varies between
\textasciitilde80 V and \textasciitilde170 V. The average is
approximately \textbf{120 V} (close to the RMS input due to heavy
loading).

\begin{enumerate}
\def\labelenumi{(\alph{enumi})}
\setcounter{enumi}{1}
\item
  At 60 V\textsubscript{rms} brownout: V\textsubscript{peak} = 60 × √2 =
  84.9 V The heavily-loaded rectified voltage drops to approximately
  V\textsubscript{dc(avg)} ≈ \textbf{60 V} (similar to input RMS)
\item
  The UVLO turn-off threshold is 70 V\textsubscript{dc}. The minimum
  instantaneous voltage with 60 V\textsubscript{rms} input will drop
  well below 70 V between rectified pulses. The converter \textbf{will
  turn off} during the brownout as the DC voltage drops below 70 V
  between peaks, then may briefly restart as V rises above 80 V at each
  peak, causing oscillation. The UVLO hysteresis prevents sustained
  operation.
\item
  Minimum input voltage for start (converter must reach 80
  V\textsubscript{dc}): V\textsubscript{in(min,rms)} = 80 / √2 =
  \textbf{56.6 V\textsubscript{rms}} (at zero load, the peak must reach
  the 80 V threshold) Under load, the minimum is higher due to voltage
  drops. Practically: V\textsubscript{in(min)} ≈ \textbf{65-70
  V\textsubscript{rms}} depending on load.
\end{enumerate}

\begin{center}\rule{0.5\linewidth}{0.5pt}\end{center}

\section{Problem 10.7.10}\label{problem-10.7.10}

\textbf{Given:} A server power supply rated at 2.4 kW, 230
V\textsubscript{rms} input uses an active PFC stage achieving 99.2\%
power factor and 2.8\% current THD. Without PFC, the same supply would
have a power factor of 0.62 and current THD of 130\%.

\textbf{Find:} (a) The RMS input current with and without PFC, (b) the
apparent power with and without PFC, (c) the reactive/distortion power
saved by PFC, (d) the upstream conductor I²R savings (assume 0.2 Ω line
impedance), and (e) the harmonic current reduction at the 3rd harmonic.

\textbf{Solution:}

\begin{enumerate}
\def\labelenumi{(\alph{enumi})}
\tightlist
\item
  With PFC: I\textsubscript{in(rms)} = P / (V × PF) = 2,400 / (230 ×
  0.992) = 2,400 / 228.2 = \textbf{10.52 A}
\end{enumerate}

Without PFC: I\textsubscript{in(rms)} = P / (V × PF) = 2,400 / (230 ×
0.62) = 2,400 / 142.6 = \textbf{16.83 A}

\begin{enumerate}
\def\labelenumi{(\alph{enumi})}
\setcounter{enumi}{1}
\item
  Apparent power: With PFC: S = V × I = 230 × 10.52 = \textbf{2,420 VA}
  Without PFC: S = V × I = 230 × 16.83 = \textbf{3,871 VA}
\item
  Reactive/distortion power: With PFC: Q = √(S² − P²) = √(2,420² −
  2,400²) = √(5,856,400 − 5,760,000) = √96,400 = \textbf{310 VAR}
  Without PFC: Q = √(3,871² − 2,400²) = √(14,984,641 − 5,760,000) =
  √9,224,641 = \textbf{3,037 VAR} Savings: 3,037 − 310 = \textbf{2,727
  VAR}
\item
  I²R losses in line impedance: With PFC: P\textsubscript{line} = 10.52²
  × 0.2 = 110.7 × 0.2 = \textbf{22.1 W} Without PFC:
  P\textsubscript{line} = 16.83² × 0.2 = 283.2 × 0.2 = \textbf{56.6 W}
  Savings: 56.6 − 22.1 = \textbf{34.5 W} (61\% reduction in line losses)
\item
  3rd harmonic current: Without PFC (130\% THD, 3rd harmonic is
  dominant, typically \textasciitilde90\% of fundamental): I₁ =
  I\textsubscript{rms} / √(1 + THD²) = 16.83 / √(1 + 1.30²) = 16.83 /
  √2.69 = 16.83 / 1.640 = 10.26 A I₃ ≈ 0.90 × 10.26 = \textbf{9.23 A}
  (without PFC)
\end{enumerate}

With PFC (2.8\% THD, 3rd harmonic \textasciitilde2\% of fundamental): I₁
≈ 10.52 A, I₃ = 0.02 × 10.52 = \textbf{0.21 A} (with PFC) Reduction:
9.23 → 0.21 A, a factor of \textbf{44×} reduction.

\chapter{Chapter 10 --- Section 10.8: Battery Management
Systems}\label{chapter-10-section-10.8-battery-management-systems}

Practice problems covering battery cell characteristics, pack design,
cell balancing, state of charge estimation, coulomb counting, BMS
thermal management, battery protection, and fault detection.

\begin{center}\rule{0.5\linewidth}{0.5pt}\end{center}

\section{Problem 10.8.1}\label{problem-10.8.1}

\textbf{Given:} An electric bus battery pack uses 120 series-connected
LFP (LiFePO₄) cells with a nominal voltage of 3.2 V and a capacity of
100 Ah each. The cells have an internal resistance of 1.5 mΩ each. The
maximum discharge rate is 2C.

\textbf{Find:} (a) The nominal pack voltage, (b) the energy capacity in
kWh, (c) the maximum continuous discharge current, (d) the total pack
internal resistance, and (e) the terminal voltage and power at maximum
discharge.

\textbf{Solution:}

\begin{enumerate}
\def\labelenumi{(\alph{enumi})}
\item
  Nominal pack voltage: V\textsubscript{pack} = 120 × 3.2 =
  \textbf{384.0 V}
\item
  Energy capacity: E = V\textsubscript{pack} × C = 384.0 × 100 = 38,400
  Wh = \textbf{38.4 kWh}
\item
  Maximum discharge current at 2C: I\textsubscript{max} = 2 × 100 =
  \textbf{200 A}
\item
  Total pack internal resistance (series cells): R\textsubscript{pack} =
  120 × 0.0015 = \textbf{0.180 Ω}
\item
  Voltage drop at maximum discharge: ΔV = I\textsubscript{max} ×
  R\textsubscript{pack} = 200 × 0.180 = 36.0 V V\textsubscript{terminal}
  = V\textsubscript{pack} − ΔV = 384.0 − 36.0 = \textbf{348.0 V}
  P\textsubscript{max} = V\textsubscript{terminal} ×
  I\textsubscript{max} = 348.0 × 200 = \textbf{69.6 kW} Power lost to
  internal resistance: P\textsubscript{loss} = I² × R = 200² × 0.180 =
  \textbf{7.2 kW} (9.4\% of total)
\end{enumerate}

\begin{center}\rule{0.5\linewidth}{0.5pt}\end{center}

\section{Problem 10.8.2}\label{problem-10.8.2}

\textbf{Given:} A 16-series lithium NMC battery pack uses passive cell
balancing with 47 Ω bleed resistors. At end of charge, cell voltages
range from 4.05 V (lowest) to 4.20 V (highest). The target is to balance
all cells to 4.05 V. The highest cell is 80 mAh above the target.

\textbf{Find:} (a) The balancing current for the highest-voltage cell,
(b) the power dissipated in its balancing resistor, (c) the time to
remove 80 mAh, (d) the total energy wasted across all cells needing
balancing (assume 8 cells require balancing with an average excess of 40
mAh), and (e) the maximum heat generated per resistor.

\textbf{Solution:}

\begin{enumerate}
\def\labelenumi{(\alph{enumi})}
\item
  Balancing current for the highest cell: I\textsubscript{bal} =
  V\textsubscript{cell} / R = 4.20 / 47 = \textbf{89.4 mA}
\item
  Power per resistor: P = V² / R = 4.20² / 47 = 17.64 / 47 =
  \textbf{0.375 W}
\item
  Time to balance 80 mAh: t = Q / I = 80 / 89.4 = 0.895 hours =
  \textbf{53.7 minutes}
\item
  Energy wasted across 8 cells with average 40 mAh excess: Average
  balancing time per cell: t\textsubscript{avg} = 40 / 89.4 = 0.4474
  hours Average power per cell: P\textsubscript{avg} ≈ (4.12)² / 47 =
  0.361 W (using average cell voltage of \textasciitilde4.12 V) Total
  energy: E = 8 × 0.361 × 0.4474 = \textbf{1.29 Wh}
\item
  Maximum heat per resistor: P\textsubscript{max} = \textbf{0.375 W}
  (occurs at the highest cell voltage of 4.20 V) A 47 Ω resistor rated
  at 0.5 W or higher is required, with adequate PCB copper area for heat
  dissipation.
\end{enumerate}

\begin{center}\rule{0.5\linewidth}{0.5pt}\end{center}

\section{Problem 10.8.3}\label{problem-10.8.3}

\textbf{Given:} An active cell balancing system uses a
switched-capacitor circuit to transfer charge between adjacent cells.
The balancing capacitor is 100 μF, the switching frequency is 50 kHz,
and the switch resistance is 30 mΩ per MOSFET (two in series per
transfer path). The voltage difference between two adjacent cells is 50
mV.

\textbf{Find:} (a) The charge transferred per switching cycle, (b) the
theoretical maximum balancing current, (c) the actual balancing current
accounting for switch resistance, (d) the balancing efficiency, and (e)
the time to transfer 100 mAh.

\textbf{Solution:}

\begin{enumerate}
\def\labelenumi{(\alph{enumi})}
\item
  Charge per cycle: Q\textsubscript{cycle} = C × ΔV = 100 × 10⁻⁶ × 0.050
  = \textbf{5.0 × 10⁻⁶ C = 5.0 μC}
\item
  Theoretical maximum balancing current: I\textsubscript{bal(max)} =
  Q\textsubscript{cycle} × f\textsubscript{sw} = 5.0 × 10⁻⁶ × 50,000 =
  \textbf{0.25 A = 250 mA}
\item
  With switch resistance (two MOSFETs per path, R\textsubscript{total} =
  2 × 0.030 = 0.060 Ω): The RC time constant: τ = R × C = 0.060 × 100 ×
  10⁻⁶ = 6.0 μs Half-period: T/2 = 1/(2 × 50,000) = 10 μs Ratio T/(2τ) =
  10/6.0 = 1.67, so the capacitor charges/discharges approximately 81\%
  per half-cycle (1 − e\textsuperscript{−1.67} = 0.812). Actual
  balancing current: I\textsubscript{bal} = 0.812 × 250 = \textbf{203
  mA}
\item
  Efficiency: Energy transferred per cycle: E\textsubscript{transfer} =
  C × ΔV² / 2 × (charge fraction) Power delivered to lower cell:
  P\textsubscript{out} = I\textsubscript{bal} × V\textsubscript{lower} ×
  ΔV / (ΔV + I\textsubscript{bal} × R) Simplified: efficiency ≈
  V\textsubscript{lower} / (V\textsubscript{lower} +
  I\textsubscript{bal} × R) = 3.95 / (3.95 + 0.203 × 0.060) = 3.95 /
  3.962 = \textbf{99.7\%}
\end{enumerate}

Active balancing is far more efficient than passive balancing (which
wastes 100\% of the transferred energy as heat).

\begin{enumerate}
\def\labelenumi{(\alph{enumi})}
\setcounter{enumi}{4}
\tightlist
\item
  Time to transfer 100 mAh: t = Q / I\textsubscript{bal} = 100 / 203 =
  0.493 hours = \textbf{29.6 minutes}
\end{enumerate}

\begin{center}\rule{0.5\linewidth}{0.5pt}\end{center}

\section{Problem 10.8.4}\label{problem-10.8.4}

\textbf{Given:} A 100 Ah battery pack starts at SOC = 92\%. A coulomb
counting BMS integrates the measured current over a 4-hour drive cycle.
The current sensor has a gain error of ±0.3\% and an offset of ±15 mA.
The pack delivers a total of 72 Ah during the drive cycle.

\textbf{Find:} (a) The estimated SOC after the drive cycle, (b) the
worst-case coulomb counting error from gain error, (c) the worst-case
error from offset drift, (d) the total SOC uncertainty, and (e) the
recommended recalibration method.

\textbf{Solution:}

\begin{enumerate}
\def\labelenumi{(\alph{enumi})}
\item
  Estimated SOC: SOC = 92\% − (72/100) × 100\% = 92\% − 72\% =
  \textbf{20\%}
\item
  Gain error over 72 Ah: ΔQ\textsubscript{gain} = 0.003 × 72 =
  \textbf{0.216 Ah} SOC error from gain: 0.216/100 × 100\% =
  \textbf{0.216\%}
\item
  Offset error over 4 hours: ΔQ\textsubscript{offset} = 0.015 × 4 =
  \textbf{0.060 Ah} SOC error from offset: 0.060/100 × 100\% =
  \textbf{0.060\%}
\item
  Total worst-case SOC uncertainty (errors add): ΔQ\textsubscript{total}
  = 0.216 + 0.060 = 0.276 Ah SOC uncertainty: \textbf{±0.276\%}, giving
  SOC range of \textbf{19.7\% to 20.3\%}
\end{enumerate}

This is a relatively small error for a single drive cycle. Over many
cycles without recalibration, the offset error accumulates linearly
(0.06 Ah per 4-hour cycle), reaching 1\% SOC error after approximately:
n = (1.0/0.060) = 16.7 cycles ≈ \textbf{17 drive cycles}

\begin{enumerate}
\def\labelenumi{(\alph{enumi})}
\setcounter{enumi}{4}
\tightlist
\item
  Recalibration method: When the pack is fully charged (charger
  terminates at the cell voltage limit), reset SOC to 100\%. When the
  pack reaches a known rest state, use the \textbf{OCV-SOC lookup table}
  to correct the coulomb-counted SOC. A \textbf{Kalman filter} combining
  coulomb counting with voltage measurement provides continuous
  correction, reducing accumulated drift to near zero.
\end{enumerate}

\begin{center}\rule{0.5\linewidth}{0.5pt}\end{center}

\section{Problem 10.8.5}\label{problem-10.8.5}

\textbf{Given:} An EV battery pack generates 1,200 W of heat at
sustained highway driving. The liquid cooling system uses a 50/50
water-glycol mixture with c\textsubscript{p} = 3,350 J/kg·°C and ρ =
1.06 kg/L. The flow rate is 8 L/min. The maximum cell surface
temperature is 35°C with a 4°C thermal resistance between coolant and
cell.

\textbf{Find:} (a) The coolant temperature rise through the pack, (b)
the required coolant inlet temperature, (c) the mass flow rate, (d) the
thermal power capacity of the cooling system at 15°C ΔT, and (e) the
maximum C-rate before the cooling system is exceeded (assume heat
generation scales with I²R where R\textsubscript{pack} = 0.15 Ω and the
pack is 80 Ah).

\textbf{Solution:}

\begin{enumerate}
\def\labelenumi{(\alph{enumi})}
\item
  Mass flow rate: ṁ = 8 × 1.06 = 8.48 kg/min = 0.1413 kg/s Coolant
  temperature rise: ΔT\textsubscript{coolant} = Q̇ / (ṁ ×
  c\textsubscript{p}) = 1,200 / (0.1413 × 3,350) = 1,200 / 473.4 =
  \textbf{2.54°C}
\item
  Maximum coolant temperature at cell interface:
  T\textsubscript{coolant,max} = 35 − 4 = 31°C Inlet temperature:
  T\textsubscript{inlet} = T\textsubscript{coolant,max} −
  ΔT\textsubscript{coolant} = 31 − 2.54 = \textbf{28.5°C}
\item
  Mass flow rate: \textbf{0.1413 kg/s} (as calculated above, =
  \textbf{8.48 kg/min})
\item
  Maximum cooling capacity at 15°C ΔT (if the coolant inlet were at
  16°C): Q̇\textsubscript{max} = ṁ × c\textsubscript{p} ×
  ΔT\textsubscript{max} = 0.1413 × 3,350 × 15 = \textbf{7,100 W = 7.1
  kW}
\item
  Heat generation: Q = I² × R\textsubscript{pack} = I² × 0.15 Cooling
  capacity Q̇\textsubscript{max} = 7,100 W: I\textsubscript{max} =
  √(7,100/0.15) = √47,333 = \textbf{217.6 A} C-rate = 217.6/80 =
  \textbf{2.72C}
\end{enumerate}

At this C-rate, the cooling system reaches its limit. Higher discharge
rates require either increased flow rate, lower inlet temperature, or
temporary operation with rising cell temperature.

\begin{center}\rule{0.5\linewidth}{0.5pt}\end{center}

\section{Problem 10.8.6}\label{problem-10.8.6}

\textbf{Given:} A BMS monitors a 96-series NMC pack using a
daisy-chained analog front end (AFE) IC. Each AFE measures 12 cells with
a voltage accuracy of ±2.5 mV and a temperature measurement accuracy of
±1°C using NTC thermistors. The OVP threshold is 4.20 V and UVP
threshold is 2.80 V per cell.

\textbf{Find:} (a) The number of AFE ICs required, (b) the minimum and
maximum cell voltages the BMS must distinguish, (c) the pack voltage
measurement accuracy, (d) the OVP and UVP thresholds with measurement
uncertainty, and (e) the required bits for the cell voltage ADC to
achieve 1 mV resolution over a 0-5 V range.

\textbf{Solution:}

\begin{enumerate}
\def\labelenumi{(\alph{enumi})}
\item
  Number of AFE ICs: N = 96 / 12 = \textbf{8 AFE ICs}
\item
  Cell voltage range: Minimum: 2.50 V (deep discharge, below UVP) to
  maximum: 4.25 V (above OVP, transient) Useful range: approximately
  \textbf{2.50 V to 4.25 V} (1.75 V span)
\item
  Pack voltage measurement accuracy: Each cell has ±2.5 mV error. For 96
  cells in series, the worst-case pack accuracy: ΔV\textsubscript{pack}
  = 96 × 2.5 = \textbf{240 mV = ±0.24 V} At nominal pack voltage (96 ×
  3.7 = 355.2 V): accuracy = 0.24/355.2 = \textbf{±0.068\%}
\item
  OVP with measurement uncertainty: The BMS must set its OVP threshold
  to: 4.20 − 2.5 × 10⁻³ = \textbf{4.1975 V} to ensure no cell exceeds
  4.20 V. UVP threshold: 2.80 + 2.5 × 10⁻³ = \textbf{2.8025 V} to ensure
  no cell drops below 2.80 V. This narrows the usable voltage window by
  5 mV total (2 × 2.5 mV), reducing usable capacity by approximately
  0.5\%.
\item
  Required ADC resolution for 1 mV over 0-5 V: Bits =
  log₂(V\textsubscript{range} / resolution) = log₂(5.0 / 0.001) =
  log₂(5,000) = \textbf{12.3 bits → 13 bits minimum}
\end{enumerate}

Most BMS AFE ICs use 14-bit or 16-bit ADCs to provide sub-millivolt
resolution with headroom.

\begin{center}\rule{0.5\linewidth}{0.5pt}\end{center}

\section{Problem 10.8.7}\label{problem-10.8.7}

\textbf{Given:} A battery pack experiences thermal runaway propagation
testing per UL 9540A. A single cell with 50 Wh energy content releases
100\% of its energy as heat in 3 seconds during thermal runaway. The
cell mass is 0.9 kg and specific heat is 1,100 J/kg·°C. The adjacent
cell is separated by 2 mm of aerogel insulation (k = 0.02 W/m·K) with a
contact area of 100 mm × 200 mm.

\textbf{Find:} (a) The peak temperature of the failing cell, (b) the
heat flux through the insulation, (c) the temperature rise of the
adjacent cell after 30 seconds (assuming the adjacent cell's thermal
mass absorbs all conducted heat), and (d) whether the adjacent cell is
at risk of cascading thermal runaway (onset at 150°C).

\textbf{Solution:}

\begin{enumerate}
\def\labelenumi{(\alph{enumi})}
\item
  Energy released: E = 50 × 3,600 = 180,000 J Temperature rise of
  failing cell: ΔT = E / (m × c) = 180,000 / (0.9 × 1,100) = 180,000 /
  990 = 181.8°C Peak temperature: T = 25 + 181.8 = \textbf{206.8°C} (in
  practice, much higher due to non-uniform heating and gas venting)
\item
  Heat flux through insulation (steady-state maximum): Contact area: A =
  0.100 × 0.200 = 0.020 m² ΔT across insulation ≈ 206.8 − 25 = 181.8°C
  (worst case, adjacent cell initially at 25°C) Q̇ = k × A × ΔT / d =
  0.02 × 0.020 × 181.8 / 0.002 = 0.02 × 0.020 × 90,900 = \textbf{36.4 W}
\item
  Adjacent cell temperature rise in 30 seconds:
  E\textsubscript{conducted} = Q̇ × t = 36.4 × 30 = 1,092 J (using peak
  heat flux as upper bound) ΔT\textsubscript{adjacent} = E / (m × c) =
  1,092 / (0.9 × 1,100) = 1,092 / 990 = \textbf{1.1°C} Adjacent cell
  temperature: 25 + 1.1 = \textbf{26.1°C}
\item
  At 26.1°C, the adjacent cell is far below the 150°C thermal runaway
  onset. The aerogel insulation provides effective propagation
  resistance. Even after 10 minutes (600 s): E\textsubscript{conducted}
  = 36.4 × 600 = 21,840 J → ΔT = 21,840/990 = 22.1°C → T = 47.1°C The
  adjacent cell \textbf{is not at risk} of cascading thermal runaway
  with the 2 mm aerogel barrier.
\end{enumerate}

\begin{center}\rule{0.5\linewidth}{0.5pt}\end{center}

\section{Problem 10.8.8}\label{problem-10.8.8}

\textbf{Given:} A Kalman filter SOC estimator uses a first-order
equivalent circuit model: V\textsubscript{terminal} = OCV(SOC) − I × R₀
− V\textsubscript{RC}, where R₀ = 3.0 mΩ (ohmic resistance), R₁ = 1.5 mΩ
(polarization resistance), C₁ = 5,000 F (polarization capacitance), and
the OCV-SOC slope is dOCV/dSOC = 0.8 V per unit SOC (= 8 mV per 1\% SOC)
in the mid-SOC range. The current sensor noise is σ\textsubscript{I} =
0.1 A and voltage sensor noise is σ\textsubscript{V} = 2 mV.

\textbf{Find:} (a) The RC time constant of the polarization circuit, (b)
the steady-state voltage across the RC network at I = 50 A, (c) the
voltage change for a 1\% SOC change, (d) the process noise and
measurement noise for the Kalman filter, and (e) the expected
steady-state SOC estimation accuracy.

\textbf{Solution:}

\begin{enumerate}
\def\labelenumi{(\alph{enumi})}
\tightlist
\item
  RC time constant: τ = R₁ × C₁ = 0.0015 × 5,000 = \textbf{7.5 seconds}
\end{enumerate}

This represents the polarization relaxation time --- the voltage
response delay when current changes.

\begin{enumerate}
\def\labelenumi{(\alph{enumi})}
\setcounter{enumi}{1}
\tightlist
\item
  Steady-state RC voltage at 50 A: V\textsubscript{RC(ss)} = I × R₁ = 50
  × 0.0015 = \textbf{75 mV}
\end{enumerate}

The total voltage drop under load: V\textsubscript{drop} = I × R₀ +
V\textsubscript{RC} = 50 × 0.003 + 0.075 = \textbf{225 mV}

\begin{enumerate}
\def\labelenumi{(\alph{enumi})}
\setcounter{enumi}{2}
\item
  Voltage change per 1\% SOC: ΔV\textsubscript{OCV} = 0.008 V =
  \textbf{8 mV per 1\% SOC}
\item
  Process noise (SOC drift per time step Δt, from current integration):
  For Δt = 1 s and Q\textsubscript{cap} = 100 Ah = 360,000 C:
  σ\textsubscript{SOC} = σ\textsubscript{I} × Δt / Q\textsubscript{cap}
  = 0.1 × 1 / 360,000 = 2.78 × 10⁻⁷ = \textbf{0.0000278\% per second}
\end{enumerate}

Measurement noise (voltage, converted to SOC equivalent):
σ\textsubscript{SOC(meas)} = σ\textsubscript{V} / (dOCV/dSOC) = 0.002 /
0.8 = \textbf{0.0025 = 0.25\% SOC equivalent}

\begin{enumerate}
\def\labelenumi{(\alph{enumi})}
\setcounter{enumi}{4}
\tightlist
\item
  The Kalman filter optimally fuses the process model (coulomb counting)
  with the voltage measurement. The steady-state estimation accuracy is
  bounded by: σ\textsubscript{SOC(KF)} ≈ √(σ\textsubscript{process} ×
  σ\textsubscript{measurement}) (geometric mean, simplified)
\end{enumerate}

Over a 1-hour window (3,600 s): accumulated process noise = 2.78 × 10⁻⁷
× √3,600 = 1.67 × 10⁻⁵ The Kalman filter corrects continuously using
voltage, achieving steady-state accuracy of approximately \textbf{±0.5\%
SOC} --- significantly better than coulomb counting alone (which drifts)
or voltage-based estimation alone (±0.25\% but noisy).

\begin{center}\rule{0.5\linewidth}{0.5pt}\end{center}

\section{Problem 10.8.9}\label{problem-10.8.9}

\textbf{Given:} A 400 V, 150 Ah EV battery pack uses a contactor
(high-voltage relay) with a 2 ms opening time for fault disconnection.
The pack has 0.5 mΩ current-sense resistor and the BMS detects a short
circuit when dI/dt exceeds 500 A/ms. Total circuit inductance is 20 μH.

\textbf{Find:} (a) The maximum short-circuit current assuming 0.5 Ω
total fault impedance, (b) the current at the time of fault detection
(assuming initial current is 100 A and the short occurs
instantaneously), (c) the current at contactor opening (2 ms after
detection), (d) the energy in the circuit inductance at opening, and (e)
the arc energy during contactor opening.

\textbf{Solution:}

\begin{enumerate}
\def\labelenumi{(\alph{enumi})}
\item
  Maximum steady-state short-circuit current: I\textsubscript{SC(max)} =
  V / R\textsubscript{fault} = 400 / 0.5 = \textbf{800 A}
\item
  The rate of current rise is limited by circuit inductance: dI/dt = V /
  L = 400 / 20 × 10⁻⁶ = 20 × 10⁶ A/s = 20,000 A/ms
\end{enumerate}

At dI/dt = 500 A/ms (detection threshold), the time from fault to
detection: The actual dI/dt starts at 20,000 A/ms and decreases
exponentially. With L/R = 20 × 10⁻⁶/0.5 = 40 μs time constant:

After \textasciitilde40 μs, the current reaches: I =
I\textsubscript{SC}(1 − e\textsuperscript{−t/τ}) + I₀ ×
e\textsuperscript{−t/τ} At 40 μs: I = 800(1 − 0.368) + 100 × 0.368 = 800
× 0.632 + 36.8 = 505.6 + 36.8 = 542 A

The dI/dt at this point: dI/dt = (V − IR)/L = (400 − 542 × 0.5)/20 ×
10⁻⁶ = (400 − 271)/20 × 10⁻⁶ = 6.45 × 10⁶ A/s = 6,450 A/ms

dI/dt drops below 500 A/ms when: 500 × 10³ = (400 − I × 0.5)/20 × 10⁻⁶
400 − 0.5I = 500 × 10³ × 20 × 10⁻⁶ = 10 I = (400 − 10)/0.5 = \textbf{780
A} at detection

\begin{enumerate}
\def\labelenumi{(\alph{enumi})}
\setcounter{enumi}{2}
\item
  Current 2 ms after detection (circuit is approaching steady state
  since τ = 40 μs ≪ 2 ms): I at opening ≈ I\textsubscript{SC(max)} =
  \textbf{800 A} (essentially at steady state)
\item
  Energy in circuit inductance: E = ½LI² = 0.5 × 20 × 10⁻⁶ × 800² = 0.5
  × 20 × 10⁻⁶ × 640,000 = \textbf{6.4 J}
\item
  Arc energy during opening (arc sustains for \textasciitilde1 ms at
  \textasciitilde30 V arc voltage, worst case): E\textsubscript{arc} =
  V\textsubscript{arc} × I × t\textsubscript{arc} = 30 × 800 × 0.001 =
  \textbf{24 J}
\end{enumerate}

The contactor must be rated for this arc energy. Pre-charge resistors
and arc suppression (varistors) across the contactor reduce the arc
energy and extend contactor life.

\begin{center}\rule{0.5\linewidth}{0.5pt}\end{center}

\section{Problem 10.8.10}\label{problem-10.8.10}

\textbf{Given:} An EV battery pack with 96 series NMC cells (3.7 V nom,
60 Ah) has been in service for 3 years with an average of 1 full cycle
per day. The cells have a degradation rate of 0.02\% capacity loss per
full equivalent cycle and 1.5\% calendar aging per year. The initial
capacity is 60 Ah.

\textbf{Find:} (a) The total number of full equivalent cycles after 3
years, (b) the cycling capacity loss, (c) the calendar aging loss, (d)
the total capacity loss and current SOH, and (e) the remaining usable
energy in kWh if the pack is considered end-of-life at 80\% SOH.

\textbf{Solution:}

\begin{enumerate}
\def\labelenumi{(\alph{enumi})}
\item
  Total cycles: N = 365 × 3 = \textbf{1,095 full equivalent cycles}
\item
  Cycling capacity loss: Loss\textsubscript{cycling} = N × 0.02\% =
  1,095 × 0.0002 = 0.219 = \textbf{21.9\%}
\item
  Calendar aging loss: Loss\textsubscript{calendar} = 3 × 1.5\% =
  \textbf{4.5\%}
\item
  Total capacity loss (these mechanisms are roughly additive for this
  model): Loss\textsubscript{total} = 21.9\% + 4.5\% = \textbf{26.4\%}
  Current capacity: Q = 60 × (1 − 0.264) = 60 × 0.736 = \textbf{44.16
  Ah} SOH = 44.16/60 × 100\% = \textbf{73.6\%}
\item
  Since SOH = 73.6\% \textless{} 80\%, the pack has already
  \textbf{reached end-of-life} for its intended EV application. The
  remaining usable energy: E = 96 × 3.7 × 44.16 = 355.2 × 44.16 =
  \textbf{15,685 Wh = 15.7 kWh} (down from 21.3 kWh when new)
\end{enumerate}

The pack may be suitable for second-life stationary storage where the
lower power demands and relaxed cycle life requirements can extract
additional value. The high cycling degradation rate (0.02\% per cycle)
suggests the cells are being stressed --- reducing DOD to 70\% or
lowering the C-rate could significantly extend life in future designs.

\chapter{Chapter 10 --- Section 10.9: Battery Energy Storage
Systems}\label{chapter-10-section-10.9-battery-energy-storage-systems}

Practice problems covering BESS architecture, power conversion systems,
round-trip efficiency, grid services, frequency regulation, peak
shaving, BESS sizing, levelized cost of storage, and degradation
analysis.

\begin{center}\rule{0.5\linewidth}{0.5pt}\end{center}

\section{Problem 10.9.1}\label{problem-10.9.1}

\textbf{Given:} A utility-scale BESS project requires 100 MW / 400 MWh
(4-hour duration) using LFP cells rated at 3.2 V nominal, 280 Ah. Each
battery module has 16 series cells (51.2 V nominal). Each rack has 12
series modules (614.4 V DC bus). The PCS inverters are rated at 5 MW
each.

\textbf{Find:} (a) The energy per rack, (b) the number of racks
required, (c) the total number of cells, (d) the number of PCS
inverters, and (e) the number of 40-foot containers if each container
holds 30 racks.

\textbf{Solution:}

\begin{enumerate}
\def\labelenumi{(\alph{enumi})}
\item
  Energy per rack: E\textsubscript{rack} = V\textsubscript{rack} × C =
  614.4 × 280 = 172,032 Wh = \textbf{172.0 kWh}
\item
  Number of racks: N\textsubscript{racks} = 400,000 / 172.0 = 2,325.6 →
  \textbf{2,326 racks} (round up)
\item
  Cells per rack = 16 × 12 = 192 Total cells = 2,326 × 192 =
  \textbf{446,592 cells}
\item
  Number of PCS inverters: N\textsubscript{PCS} = 100 / 5 = \textbf{20
  inverters}
\item
  Number of containers: N\textsubscript{containers} = 2,326 / 30 = 77.5
  → \textbf{78 battery containers} (plus additional containers for PCS,
  switchgear, HVAC, and controls --- typically 20-25 more, for
  approximately 100 total)
\end{enumerate}

\begin{center}\rule{0.5\linewidth}{0.5pt}\end{center}

\section{Problem 10.9.2}\label{problem-10.9.2}

\textbf{Given:} A 25 MW / 100 MWh BESS uses a PCS with one-way
(single-direction) efficiency of 97.5\% and auxiliary power consumption
of 1.5\% of throughput. The system completes 350 cycles per year at 85\%
depth of discharge.

\textbf{Find:} (a) The round-trip efficiency, (b) the annual energy
discharged, (c) the annual energy charged (input), (d) the annual energy
losses, and (e) the annual auxiliary consumption.

\textbf{Solution:}

\begin{enumerate}
\def\labelenumi{(\alph{enumi})}
\item
  Round-trip efficiency: η\textsubscript{RT} = η\textsubscript{inv}² ×
  (1 − p\textsubscript{aux})² = 0.975² × (1 − 0.015)²
  η\textsubscript{RT} = 0.9506 × 0.9702 = \textbf{92.2\%}
\item
  Annual energy discharged: E\textsubscript{discharge} = 100 × 0.85 ×
  350 = \textbf{29,750 MWh/year}
\item
  Annual energy charged: E\textsubscript{charge} =
  E\textsubscript{discharge} / η\textsubscript{RT} = 29,750 / 0.922 =
  \textbf{32,268 MWh/year}
\item
  Annual energy losses: E\textsubscript{loss} = E\textsubscript{charge}
  − E\textsubscript{discharge} = 32,268 − 29,750 = \textbf{2,518
  MWh/year}
\item
  Annual auxiliary consumption (applied to both charge and discharge):
  E\textsubscript{aux} = (E\textsubscript{charge} +
  E\textsubscript{discharge}) × 0.015 = (32,268 + 29,750) × 0.015 =
  62,018 × 0.015 = \textbf{930 MWh/year}
\end{enumerate}

The auxiliary consumption (930 MWh) is included within the total losses
(2,518 MWh). The remaining 1,588 MWh are inverter conversion losses.

\begin{center}\rule{0.5\linewidth}{0.5pt}\end{center}

\section{Problem 10.9.3}\label{problem-10.9.3}

\textbf{Given:} A 30 MW BESS provides primary frequency regulation with
a 5\% droop setting. The system has a ±0.025 Hz deadband around the 60
Hz nominal frequency. During a grid event, the frequency drops to 59.70
Hz and remains there for 8 minutes before recovering.

\textbf{Find:} (a) The effective frequency deviation outside the
deadband, (b) the BESS power output during the event, (c) the energy
dispatched during the 8-minute event, (d) the SOC change if the BESS has
120 MWh capacity, and (e) the frequency support if two identical BESS
units provide the same service.

\textbf{Solution:}

\begin{enumerate}
\def\labelenumi{(\alph{enumi})}
\item
  Frequency deviation: Δf = 60 − 59.70 = 0.30 Hz Effective deviation
  (subtract deadband): Δf\textsubscript{eff} = 0.30 − 0.025 =
  \textbf{0.275 Hz}
\item
  Power output using droop equation: ΔP = P\textsubscript{rated} ×
  Δf\textsubscript{eff} / (f\textsubscript{nom} × R) = 30 × 0.275 / (60
  × 0.05) = 30 × 0.275 / 3.0 = \textbf{2.75 MW} (discharging)
\item
  Energy dispatched: E = P × t = 2.75 × (8/60) = 2.75 × 0.1333 =
  \textbf{0.367 MWh}
\item
  SOC change: ΔSOC = E / E\textsubscript{total} × 100 = 0.367 / 120 ×
  100 = \textbf{0.306\%}
\end{enumerate}

The small SOC impact confirms that frequency regulation is an
energy-light, power-heavy application ideally suited for BESS.

\begin{enumerate}
\def\labelenumi{(\alph{enumi})}
\setcounter{enumi}{4}
\tightlist
\item
  With two 30 MW units (60 MW total), each provides 2.75 MW: Total
  response = 2 × 2.75 = \textbf{5.50 MW}
\end{enumerate}

Alternatively, if each operates independently with the same droop, the
combined response at the grid level would help arrest the frequency
deviation faster. In a system with sufficient BESS, the frequency would
not drop as far due to the faster response.

\begin{center}\rule{0.5\linewidth}{0.5pt}\end{center}

\section{Problem 10.9.4}\label{problem-10.9.4}

\textbf{Given:} A BESS provides voltage support (reactive power) at a
13.8 kV distribution bus. The PCS inverter is rated at 10 MVA with a
maximum active power output of 8 MW. The grid needs 5 MVAR of reactive
power support while the BESS simultaneously discharges at 6 MW.

\textbf{Find:} (a) The total apparent power required, (b) whether the
inverter can simultaneously provide both P and Q, (c) the inverter
current, (d) the power factor, and (e) the maximum reactive power
available if the BESS is discharging at full 8 MW.

\textbf{Solution:}

\begin{enumerate}
\def\labelenumi{(\alph{enumi})}
\item
  Total apparent power: S = √(P² + Q²) = √(6² + 5²) = √(36 + 25) = √61 =
  \textbf{7.81 MVA}
\item
  Since S = 7.81 MVA \textless{} S\textsubscript{rated} = 10 MVA, the
  inverter \textbf{can provide both} simultaneously.
\item
  Inverter current: I = S / (√3 × V) = 7,810,000 / (1.732 × 13,800) =
  7,810,000 / 23,902 = \textbf{327 A}
\item
  Power factor: PF = P / S = 6 / 7.81 = \textbf{0.768 lagging} (since Q
  is inductive/lagging)
\item
  Maximum Q at full active power: Q\textsubscript{max} =
  √(S\textsubscript{rated}² − P\textsubscript{max}²) = √(10² − 8²) =
  √(100 − 64) = √36 = \textbf{6.0 MVAR}
\end{enumerate}

Even at full 8 MW discharge, the inverter can still provide 6.0 MVAR of
reactive power support --- a significant advantage of BESS over
conventional generators for combined active/reactive power support.

\begin{center}\rule{0.5\linewidth}{0.5pt}\end{center}

\section{Problem 10.9.5}\label{problem-10.9.5}

\textbf{Given:} A commercial facility has a load profile where demand
exceeds 4.0 MW for 3 hours per day, peaking at 5.5 MW. Below 4.0 MW, the
demand averages 2.8 MW for the remaining 21 hours. The utility demand
charge is \$22/kW-month. The BESS round-trip efficiency is 90\%.

\textbf{Find:} (a) The BESS power rating to shave the peak to 4.0 MW,
(b) the energy capacity required (model the peak as triangular), (c) the
monthly demand charge savings, (d) the annual savings, and (e) the
energy needed to recharge the BESS daily.

\textbf{Solution:}

\begin{enumerate}
\def\labelenumi{(\alph{enumi})}
\item
  BESS power rating: P\textsubscript{BESS} = P\textsubscript{peak} −
  P\textsubscript{target} = 5.5 − 4.0 = \textbf{1.5 MW}
\item
  Energy capacity (triangular peak profile over 3 hours with 1.5 MW
  maximum excess): E = ½ × P\textsubscript{BESS} × duration = ½ × 1.5 ×
  3 = \textbf{2.25 MWh}
\end{enumerate}

Add 15\% margin for degradation: E\textsubscript{nameplate} = 2.25 /
0.85 = \textbf{2.65 MWh} (specify 3.0 MWh standard module)

\begin{enumerate}
\def\labelenumi{(\alph{enumi})}
\setcounter{enumi}{2}
\item
  Monthly demand charge savings: ΔD = (5.5 − 4.0) × 1,000 × \$22 = 1,500
  × \$22 = \textbf{\$33,000/month}
\item
  Annual savings: Savings = \$33,000 × 12 = \textbf{\$396,000/year}
\item
  Daily recharge energy: E\textsubscript{charge} =
  E\textsubscript{discharge} / η\textsubscript{RT} = 2.25 / 0.90 =
  \textbf{2.50 MWh/day}
\end{enumerate}

The additional grid energy for recharging costs approximately 2.50 ×
\$50/MWh = \$125/day or \$3,750/month, which is subtracted from the
demand charge savings for a net monthly benefit of \$33,000 − \$3,750 =
\$29,250.

\begin{center}\rule{0.5\linewidth}{0.5pt}\end{center}

\section{Problem 10.9.6}\label{problem-10.9.6}

\textbf{Given:} A BESS performs energy arbitrage, charging during
off-peak hours at \$25/MWh and discharging during on-peak hours at
\$85/MWh. The system is 50 MW / 200 MWh with 91\% round-trip efficiency.
It operates 330 days per year at 80\% DOD.

\textbf{Find:} (a) The daily energy discharged, (b) the daily energy
required for charging, (c) the daily revenue from arbitrage, (d) the
daily cost of charging, and (e) the annual net arbitrage revenue.

\textbf{Solution:}

\begin{enumerate}
\def\labelenumi{(\alph{enumi})}
\item
  Daily energy discharged: E\textsubscript{discharge} = 200 × 0.80 =
  \textbf{160 MWh/day}
\item
  Daily energy for charging: E\textsubscript{charge} =
  E\textsubscript{discharge} / η\textsubscript{RT} = 160 / 0.91 =
  \textbf{175.8 MWh/day}
\item
  Daily revenue: Revenue = E\textsubscript{discharge} ×
  price\textsubscript{on-peak} = 160 × \$85 = \textbf{\$13,600/day}
\item
  Daily charging cost: Cost = E\textsubscript{charge} ×
  price\textsubscript{off-peak} = 175.8 × \$25 = \textbf{\$4,395/day}
\item
  Annual net arbitrage revenue: Daily profit = \$13,600 − \$4,395 =
  \$9,205/day Annual = \$9,205 × 330 = \textbf{\$3,037,650/year}
\end{enumerate}

The \$60/MWh spread yields \$3.04M/year. With a typical BESS installed
cost of \$250/kWh × 200,000 = \$50M, the simple payback from arbitrage
alone is 50M/3.04M = 16.4 years --- too long for a viable standalone
business case. Revenue stacking with frequency regulation, capacity
payments, and demand charge reduction is essential for economic
viability.

\begin{center}\rule{0.5\linewidth}{0.5pt}\end{center}

\section{Problem 10.9.7}\label{problem-10.9.7}

\textbf{Given:} A 200 MWh LFP BESS has the following degradation
parameters: 0.015\% capacity loss per full equivalent cycle (FEC) and
1.2\% calendar fade per year. The system operates 1.2 cycles per day at
75\% DOD. The performance guarantee requires 80\% capacity retention.

\textbf{Find:} (a) The annual full equivalent cycles, (b) the annual
cycling degradation, (c) the total annual degradation (cycling +
calendar), (d) the number of years to reach 80\% capacity, and (e) the
total energy throughput over the system's warranted life.

\textbf{Solution:}

\begin{enumerate}
\def\labelenumi{(\alph{enumi})}
\item
  Annual FEC: FEC\textsubscript{annual} = 1.2 cycles/day × 0.75 DOD ×
  365 = \textbf{328.5 FEC/year}
\item
  Annual cycling degradation: Loss\textsubscript{cycling} = 328.5 ×
  0.015\% = 328.5 × 0.00015 = \textbf{4.93\%/year}
\item
  Total annual degradation: Loss\textsubscript{total} = 4.93\% + 1.2\% =
  \textbf{6.13\%/year}
\item
  Years to reach 80\% retention: Capacity loss to reach 80\%: 20\% Years
  = 20\% / 6.13\% = \textbf{3.26 years}
\end{enumerate}

This is an unrealistically short warranty period, indicating the
degradation parameters are aggressive. More typical LFP degradation
(0.005\% per FEC) would yield: Loss\textsubscript{cycling} = 328.5 ×
0.005\% = 1.64\%/year Total = 1.64\% + 1.2\% = 2.84\%/year Years = 20\%
/ 2.84\% = \textbf{7.0 years} (still below the 15-20 year target for
utility BESS)

Augmentation (adding modules to replace lost capacity) is planned at
years 7-8 to extend the system to 15+ years.

\begin{enumerate}
\def\labelenumi{(\alph{enumi})}
\setcounter{enumi}{4}
\tightlist
\item
  Total energy throughput over 3.26 years (with original degradation
  rate): Annual throughput = 200 × 0.75 × 1.2 × 365 = \textbf{65,700
  MWh/year} (year 1) Average annual throughput (accounting for
  degradation): ≈ 65,700 × (1 − 0.0613 × 1.63) = 65,700 × 0.90 = 59,130
  MWh avg Total throughput ≈ 59,130 × 3.26 = \textbf{192,764 MWh}
\end{enumerate}

\begin{center}\rule{0.5\linewidth}{0.5pt}\end{center}

\section{Problem 10.9.8}\label{problem-10.9.8}

\textbf{Given:} A BESS provides renewable smoothing for a 50 MW solar
farm. The maximum allowable ramp rate at the point of interconnection
(POI) is 10\% of plant capacity per minute. The solar irradiance causes
a cloud-induced ramp of 35 MW in 2 minutes.

\textbf{Find:} (a) The maximum allowable power ramp rate, (b) the
maximum power change over 2 minutes, (c) the excess ramp that must be
compensated by the BESS, (d) the energy the BESS must supply or absorb
during the event, and (e) the minimum BESS power and energy rating for
this service.

\textbf{Solution:}

\begin{enumerate}
\def\labelenumi{(\alph{enumi})}
\item
  Maximum allowable ramp rate: ΔP/Δt = 10\% × 50 MW/min = \textbf{5
  MW/min}
\item
  Maximum allowable power change in 2 minutes: ΔP\textsubscript{max} = 5
  × 2 = \textbf{10 MW}
\item
  The solar ramp is 35 MW in 2 minutes (17.5 MW/min), but only 10 MW is
  allowed. Excess ramp: ΔP\textsubscript{excess} = 35 − 10 = \textbf{25
  MW} that the BESS must compensate
\item
  The BESS must inject power to replace the lost solar generation and
  smooth the ramp. Over the 2-minute event, the BESS ramps from 0 to 25
  MW linearly: E = ½ × 25 × (2/60) = ½ × 25 × 0.0333 = \textbf{0.417
  MWh}
\item
  Minimum BESS power rating: P\textsubscript{BESS} = \textbf{25 MW}
  (must cover the full excess ramp)
\end{enumerate}

Minimum energy: 0.417 MWh, but the BESS must handle repeated events.
With a design margin for 10 consecutive events: E\textsubscript{BESS} =
10 × 0.417 = 4.17 MWh → \textbf{5 MWh} (with margin)

A \textbf{25 MW / 5 MWh BESS} (12-minute duration) is sufficient for
ramp rate compliance. This is a short-duration, high-power application.

\begin{center}\rule{0.5\linewidth}{0.5pt}\end{center}

\section{Problem 10.9.9}\label{problem-10.9.9}

\textbf{Given:} A BESS project has the following cost structure: battery
modules at \$130/kWh, PCS at \$80/kW, balance of plant at \$40/kWh, EPC
and soft costs at \$25/kWh. The system is 50 MW / 200 MWh with a 20-year
life, 8\% discount rate, 365 cycles/year at 80\% DOD, 91\% round-trip
efficiency, and O\&M of \$5/MWh throughput. Augmentation of 15\%
capacity is required at year 10 at 50\% of original battery module cost.

\textbf{Find:} (a) The total installed capital cost, (b) the annual
energy throughput, (c) the present worth of lifetime O\&M, (d) the
present worth of augmentation, and (e) the LCOS.

\textbf{Solution:}

\begin{enumerate}
\def\labelenumi{(\alph{enumi})}
\item
  Total capital cost: Battery: 200,000 × \$130 = \$26,000,000 PCS:
  50,000 × \$80 = \$4,000,000 BOP: 200,000 × \$40 = \$8,000,000 EPC:
  200,000 × \$25 = \$5,000,000 Total: \$26,000,000 + \$4,000,000 +
  \$8,000,000 + \$5,000,000 = \textbf{\$43,000,000}
\item
  Annual energy throughput (discharge side): E\textsubscript{annual} =
  200 × 0.80 × 365 = \textbf{58,400 MWh/year} (year 1, declining with
  degradation)
\item
  O\&M costs: Annual O\&M = 58,400 × \$5 = \$292,000/year Present worth
  factor (P/A, 8\%, 20) = (1 − 1.08⁻²⁰)/0.08 = (1 − 0.2145)/0.08 = 9.818
  PW of O\&M = \$292,000 × 9.818 = \textbf{\$2,866,856}
\item
  Augmentation at year 10: Cost = 0.15 × 200,000 × \$130 × 0.50 = 0.15 ×
  \$13,000,000 = \textbf{\$1,950,000} (in year-10 dollars) PW =
  \$1,950,000 × (1.08)⁻¹⁰ = \$1,950,000 × 0.4632 = \textbf{\$903,240}
\item
  LCOS: Total PW of costs = \$43,000,000 + \$2,866,856 + \$903,240 =
  \$46,770,096
\end{enumerate}

PW of lifetime energy discharged (assuming flat throughput for
simplicity): PW of energy = 58,400 × 9.818 = \textbf{573,371 MWh}
(present worth of annual throughput stream)

LCOS = \$46,770,096 / 573,371 = \textbf{\$81.6/MWh}

This LCOS of \$81.6/MWh is competitive with peaking gas turbines
(\$100-150/MWh) and viable in markets with stacked revenue streams
exceeding \$100/MWh.

\begin{center}\rule{0.5\linewidth}{0.5pt}\end{center}

\section{Problem 10.9.10}\label{problem-10.9.10}

\textbf{Given:} A 100 MWh BESS uses LFP cells with a beginning-of-life
capacity of 100 MWh. The degradation model is: capacity retention = 1 −
(0.00008 × FEC) − (0.012 × years), where FEC is the cumulative full
equivalent cycles. The system operates 1 cycle/day at 80\% DOD. A
capacity augmentation event adds 20 MWh of new modules when capacity
drops to 85 MWh.

\textbf{Find:} (a) The annual FEC, (b) the capacity at end of year 1,
year 3, and year 5, (c) the year when augmentation is triggered, (d) the
capacity after augmentation, and (e) the effective capacity at year 10
(second degradation period includes the augmented modules aging from
their install date).

\textbf{Solution:}

\begin{enumerate}
\def\labelenumi{(\alph{enumi})}
\item
  Annual FEC: FEC\textsubscript{annual} = 1 × 0.80 × 365 = \textbf{292
  FEC/year}
\item
  Capacity at end of each year:
\end{enumerate}

Year 1: FEC = 292, years = 1 Retention = 1 − (0.00008 × 292) − (0.012 ×
1) = 1 − 0.02336 − 0.012 = 0.9646 Capacity = 100 × 0.9646 =
\textbf{96.46 MWh}

Year 3: FEC = 876, years = 3 Retention = 1 − (0.00008 × 876) − (0.012 ×
3) = 1 − 0.07008 − 0.036 = 0.8939 Capacity = 100 × 0.8939 =
\textbf{89.39 MWh}

Year 5: FEC = 1,460, years = 5 Retention = 1 − (0.00008 × 1,460) −
(0.012 × 5) = 1 − 0.1168 − 0.060 = 0.8232 Capacity = 100 × 0.8232 =
\textbf{82.32 MWh}

\begin{enumerate}
\def\labelenumi{(\alph{enumi})}
\setcounter{enumi}{2}
\item
  Augmentation triggered when capacity = 85 MWh: 0.85 = 1 − 0.00008 ×
  (292t) − 0.012t = 1 − 0.02336t − 0.012t = 1 − 0.03536t 0.03536t = 0.15
  t = 0.15/0.03536 = \textbf{4.24 years} (augmentation triggered during
  year 5, approximately month 3)
\item
  After augmentation at year 4.24: Existing modules: 85 MWh (at trigger
  point) New modules: 20 MWh (fresh, 100\% SOH) Total capacity = 85 + 20
  = \textbf{105 MWh}
\item
  At year 10 (5.76 years after augmentation): Original modules (10 years
  old, FEC = 2,920): Retention = 1 − (0.00008 × 2,920) − (0.012 × 10) =
  1 − 0.2336 − 0.120 = 0.6464 Capacity\textsubscript{original} = 100 ×
  0.6464 = 64.64 MWh
\end{enumerate}

New modules (5.76 years old, FEC = 5.76 × 292 = 1,682): Retention = 1 −
(0.00008 × 1,682) − (0.012 × 5.76) = 1 − 0.1346 − 0.0691 = 0.7963
Capacity\textsubscript{new} = 20 × 0.7963 = 15.93 MWh

Total at year 10 = 64.64 + 15.93 = \textbf{80.57 MWh}

The system is at \textasciitilde80.6\% of original capacity at year 10,
approaching the typical end-of-warranty threshold of 80\%. A second
augmentation may be needed around year 11 to extend operation to 15-20
years.

\chapter{Chapter 10 --- Section 10.10: Battery
Charging}\label{chapter-10-section-10.10-battery-charging}

Practice problems covering CC/CV charging profiles, charge termination,
charger topologies and efficiency, EV charging levels and standards,
fast charging thermal analysis, lithium plating risk, and wireless
inductive charging systems.

\begin{center}\rule{0.5\linewidth}{0.5pt}\end{center}

\section{Problem 10.10.1}\label{problem-10.10.1}

\textbf{Given:} A cylindrical 18650 NMC lithium-ion cell has a rated
capacity of 3.5 Ah and is charged using a CC/CV algorithm. The CC rate
is 0.7C, and the charge cutoff voltage is 4.20 V. The cell reaches 4.20
V after 1.3 hours. During the CV phase, the current decays exponentially
with a time constant τ = 0.35 hours, and charge terminates at C/25 (0.14
A).

\textbf{Find:} (a) The CC charging current, (b) the charge delivered
during the CC phase, (c) the time from CV phase start to charge
termination, (d) the charge delivered during the CV phase, and (e) the
total charge time.

\textbf{Solution:}

\begin{enumerate}
\def\labelenumi{(\alph{enumi})}
\item
  CC current: I\textsubscript{CC} = 0.7 × 3.5 = \textbf{2.45 A}
\item
  CC phase charge: Q\textsubscript{CC} = 2.45 × 1.3 = \textbf{3.185 Ah}
\item
  CV current decays as I(t) = 2.45 × e\textsuperscript{−t/0.35}.
  Termination at I = 0.14 A: 0.14 = 2.45 × e\textsuperscript{−t/0.35} →
  e\textsuperscript{−t/0.35} = 0.0571 → t = −0.35 × ln(0.0571) = 0.35 ×
  2.864 = \textbf{1.00 hours}
\item
  CV phase charge: Q\textsubscript{CV} = ∫₀\textsuperscript{1.00} 2.45 ×
  e\textsuperscript{−t/0.35} dt = 2.45 × 0.35 × (1 −
  e\textsuperscript{−1.00/0.35}) = 0.8575 × (1 − 0.0571) = 0.8575 ×
  0.9429 = \textbf{0.809 Ah}
\item
  Total charge time = 1.3 + 1.00 = \textbf{2.30 hours}
\end{enumerate}

Total charge delivered = 3.185 + 0.809 = 3.994 Ah (slightly above rated
capacity due to coulombic efficiency \textless{} 100\%).

\begin{center}\rule{0.5\linewidth}{0.5pt}\end{center}

\section{Problem 10.10.2}\label{problem-10.10.2}

\textbf{Given:} A lead-acid battery bank (48 V nominal, 200 Ah) uses
three-stage charging. The bulk (CC) stage supplies 40 A until the
voltage reaches 57.6 V (2.40 V/cell × 24 cells). The absorption (CV)
stage holds 57.6 V until the current drops to 4 A. The float stage
maintains 54.0 V. The battery is at 30\% SOC (70\% depleted) when
charging begins.

\textbf{Find:} (a) The energy depleted from the battery (assuming 48 V
average discharge voltage), (b) the approximate bulk stage duration, (c)
the charge delivered in the bulk stage, and (d) the percentage of
capacity restored during the bulk stage.

\textbf{Solution:}

\begin{enumerate}
\def\labelenumi{(\alph{enumi})}
\item
  Energy depleted: E = V\textsubscript{avg} × C × DOD = 48 × 200 × 0.70
  = \textbf{6,720 Wh} = 6.72 kWh
\item
  Charge depleted = 200 × 0.70 = 140 Ah. The bulk stage typically
  restores \textasciitilde80\% of the depleted charge (accounting for
  the increasing voltage during bulk). Bulk stage duration ≈ (140 ×
  0.80) / 40 = 112 / 40 = \textbf{2.8 hours}
\item
  Charge delivered in bulk: Q\textsubscript{bulk} = 40 × 2.8 =
  \textbf{112 Ah}
\item
  Percentage of depleted capacity restored: 112 / 140 = \textbf{80\%}
  (the remaining 20\% is restored during the slower absorption stage)
\end{enumerate}

\begin{center}\rule{0.5\linewidth}{0.5pt}\end{center}

\section{Problem 10.10.3}\label{problem-10.10.3}

\textbf{Given:} A 6.6 kW on-board charger has a bridgeless totem-pole
PFC stage (98.5\% efficiency) and a CLLC resonant DC-DC stage (97.0\%
efficiency). It converts 240 V\textsubscript{rms} single-phase AC to a
350--450 V battery pack. The charger operates with a power factor of
0.99.

\textbf{Find:} (a) The overall charger efficiency, (b) the AC input
current at full load, (c) the power dissipated as heat, and (d) the
apparent power drawn from the grid.

\textbf{Solution:}

\begin{enumerate}
\def\labelenumi{(\alph{enumi})}
\item
  η\textsubscript{total} = η\textsubscript{PFC} × η\textsubscript{DC-DC}
  = 0.985 × 0.970 = \textbf{95.5\%}
\item
  Input power: P\textsubscript{in} = 6,600 / 0.955 = 6,911 W.
  I\textsubscript{AC} = P\textsubscript{in} / V\textsubscript{AC} =
  6,911 / 240 = \textbf{28.8 A}
\item
  Heat dissipated: P\textsubscript{loss} = 6,911 − 6,600 = \textbf{311
  W}
\item
  Apparent power: S = P\textsubscript{in} / PF = 6,911 / 0.99 =
  \textbf{6,981 VA}
\end{enumerate}

\begin{center}\rule{0.5\linewidth}{0.5pt}\end{center}

\section{Problem 10.10.4}\label{problem-10.10.4}

\textbf{Given:} A 350 kW off-board DC fast charger uses a three-phase
480 V AC input with a Vienna rectifier (PFC) stage and a phase-shifted
full-bridge DC-DC stage. The overall efficiency is 94\%. The charger
supplies 800 V DC to the vehicle.

\textbf{Find:} (a) The AC input power at full load, (b) the three-phase
input current (assuming unity power factor), (c) the DC output current
at 800 V, and (d) the total heat rejected by the cooling system.

\textbf{Solution:}

\begin{enumerate}
\def\labelenumi{(\alph{enumi})}
\item
  Input power: P\textsubscript{in} = 350,000 / 0.94 = \textbf{372,340 W}
  = 372.3 kW
\item
  Three-phase current: I\textsubscript{AC} = P\textsubscript{in} / (√3 ×
  V\textsubscript{LL}) = 372,340 / (1.732 × 480) = 372,340 / 831.4 =
  \textbf{447.8 A}
\item
  DC output current: I\textsubscript{DC} = P\textsubscript{out} /
  V\textsubscript{DC} = 350,000 / 800 = \textbf{437.5 A}
\item
  Heat rejected: P\textsubscript{loss} = 372,340 − 350,000 =
  \textbf{22,340 W} = 22.3 kW (requires liquid cooling with substantial
  radiator capacity)
\end{enumerate}

\begin{center}\rule{0.5\linewidth}{0.5pt}\end{center}

\section{Problem 10.10.5}\label{problem-10.10.5}

\textbf{Given:} Three EVs with identical 82 kWh usable battery packs
arrive at charging stations at 15\% SOC. Vehicle A uses Level 1 (1.44
kW), Vehicle B uses Level 2 (11.5 kW), and Vehicle C uses DCFC (250 kW,
constant power up to 80\% SOC, then tapering to 50 kW average from
80--90\%). Assume 93\% charger efficiency for all levels.

\textbf{Find:} (a) The energy required to charge each vehicle from 15\%
to 90\% SOC, (b) the charge time for Vehicle A, (c) the charge time for
Vehicle B, (d) the charge time for Vehicle C, and (e) the total energy
drawn from the grid for Vehicle B.

\textbf{Solution:}

\begin{enumerate}
\def\labelenumi{(\alph{enumi})}
\item
  Energy required: E = 82 × (0.90 − 0.15) = 82 × 0.75 = \textbf{61.5
  kWh} (all three vehicles require the same energy)
\item
  Vehicle A (Level 1): t = 61.5 / 1.44 = \textbf{42.7 hours} (nearly two
  full days)
\item
  Vehicle B (Level 2): t = 61.5 / 11.5 = \textbf{5.35 hours}
\item
  Vehicle C (DCFC): Energy from 15\% to 80\% = 82 × 0.65 = 53.3 kWh.
  Time = 53.3 / 250 = 0.213 h = 12.8 min. Energy from 80\% to 90\% = 82
  × 0.10 = 8.2 kWh. Time = 8.2 / 50 = 0.164 h = 9.8 min. Total DCFC time
  = 12.8 + 9.8 = \textbf{22.6 minutes}
\item
  Grid energy for Vehicle B: E\textsubscript{grid} =
  E\textsubscript{battery} / η = 61.5 / 0.93 = \textbf{66.1 kWh} (the
  4.6 kWh difference is dissipated as heat in the charger)
\end{enumerate}

\begin{center}\rule{0.5\linewidth}{0.5pt}\end{center}

\section{Problem 10.10.6}\label{problem-10.10.6}

\textbf{Given:} A pouch NMC cell has a capacity of 60 Ah, internal
resistance of 1.2 mΩ at 25°C and 3.0 mΩ at 0°C, mass of 0.9 kg, and
specific heat capacity of 1,050 J/(kg·°C). The cell is charged at 2C
(120 A).

\textbf{Find:} (a) The I²R heat generation at 25°C, (b) the I²R heat
generation at 0°C, (c) the adiabatic temperature rise per minute at each
temperature, and (d) the time to reach 45°C from a 0°C start (adiabatic,
assuming constant resistance).

\textbf{Solution:}

\begin{enumerate}
\def\labelenumi{(\alph{enumi})}
\item
  Q̇\textsubscript{25°C} = I² × R = 120² × 0.0012 = 14,400 × 0.0012 =
  \textbf{17.3 W}
\item
  Q̇\textsubscript{0°C} = 120² × 0.0030 = 14,400 × 0.0030 = \textbf{43.2
  W}
\item
  At 25°C: ΔT/min = Q̇ × 60 / (m × c\textsubscript{p}) = 17.3 × 60 / (0.9
  × 1,050) = 1,038 / 945 = \textbf{1.10°C/min} At 0°C: ΔT/min = 43.2 ×
  60 / 945 = 2,592 / 945 = \textbf{2.74°C/min}
\item
  ΔT = 45 − 0 = 45°C. E\textsubscript{max} = 0.9 × 1,050 × 45 = 42,525
  J. t = 42,525 / 43.2 = 984 s = \textbf{16.4 minutes} --- well before
  the CC phase would complete (30 min for 2C), indicating that active
  cooling or reduced charge rate is essential for cold-weather fast
  charging.
\end{enumerate}

\begin{center}\rule{0.5\linewidth}{0.5pt}\end{center}

\section{Problem 10.10.7}\label{problem-10.10.7}

\textbf{Given:} A pulse charging protocol for a 100 Ah LFP cell uses
5-second pulses at 200 A followed by 2-second rest periods. The cell's
internal resistance is 0.8 mΩ.

\textbf{Find:} (a) The heat generated per pulse, (b) the average power
dissipation over a pulse-rest cycle, (c) the average charging current
over one cycle, and (d) the time to deliver 80 Ah using this protocol.

\textbf{Solution:}

\begin{enumerate}
\def\labelenumi{(\alph{enumi})}
\item
  Power during pulse: P = I² × R = 200² × 0.0008 = 32 W. Energy per
  pulse: E = P × t = 32 × 5 = \textbf{160 J}
\item
  Cycle duration = 5 + 2 = 7 seconds. Average power:
  P\textsubscript{avg} = 160 / 7 = \textbf{22.9 W}
\item
  Charge per pulse: Q = 200 × 5 = 1,000 As = 0.2778 Ah. Average current:
  I\textsubscript{avg} = 0.2778 / (7/3600) = 0.2778 / 0.001944 =
  \textbf{142.9 A} (equivalent to 1.43C)
\item
  Time to deliver 80 Ah: t = 80 / 142.9 = 0.56 hours = \textbf{33.6
  minutes} (or equivalently, 80/0.2778 = 288 cycles × 7 s = 2,016 s =
  33.6 min)
\end{enumerate}

\begin{center}\rule{0.5\linewidth}{0.5pt}\end{center}

\section{Problem 10.10.8}\label{problem-10.10.8}

\textbf{Given:} An SAE J1772 Level 2 EVSE uses a 1 kHz pilot signal with
a duty cycle of 50\%. Per the J1772 specification, the maximum available
current = duty cycle × 60 A (for duty cycles between 10\% and 85\%).

\textbf{Find:} (a) The maximum available current communicated by the
50\% duty cycle, (b) the maximum charging power at 240 V, (c) the NEC
minimum circuit breaker size (125\% continuous load rule), and (d) the
minimum copper conductor size from NEC Table 310.16 at 75°C.

\textbf{Solution:}

\begin{enumerate}
\def\labelenumi{(\alph{enumi})}
\item
  Maximum available current: I\textsubscript{max} = 0.50 × 60 =
  \textbf{30 A}
\item
  Maximum power: P = 240 × 30 = \textbf{7,200 W} = 7.2 kW
\item
  Circuit breaker: I\textsubscript{CB} = 30 × 1.25 = 37.5 A → \textbf{40
  A} (next standard size per NEC 240.6(A))
\item
  From NEC Table 310.16 at 75°C: 8 AWG copper has 50 A ampacity, which
  exceeds 40 A → \textbf{8 AWG}
\end{enumerate}

\begin{center}\rule{0.5\linewidth}{0.5pt}\end{center}

\section{Problem 10.10.9}\label{problem-10.10.9}

\textbf{Given:} A wireless EV charging system (SAE J2954 WPT2, 7.7 kW)
operates at 85 kHz with transmitter coil L₁ = 180 μH, receiver coil L₂ =
220 μH, coupling coefficient k = 0.18, and coil AC resistance R₁ = R₂ =
0.12 Ω. SS compensation is used.

\textbf{Find:} (a) The mutual inductance, (b) the compensation
capacitors, (c) the coil quality factors, and (d) the maximum
theoretical efficiency.

\textbf{Solution:}

\begin{enumerate}
\def\labelenumi{(\alph{enumi})}
\item
  M = k × √(L₁ × L₂) = 0.18 × √(180 × 10⁻⁶ × 220 × 10⁻⁶) = 0.18 × √(3.96
  × 10⁻⁸) = 0.18 × 1.99 × 10⁻⁴ = \textbf{35.8 μH}
\item
  ω = 2π × 85,000 = 534,071 rad/s. C₁ = 1/(ω²L₁) = 1/(2.852 × 10¹¹ ×
  1.80 × 10⁻⁴) = 1/(5.134 × 10⁷) = \textbf{19.5 nF} C₂ = 1/(ω²L₂) =
  1/(2.852 × 10¹¹ × 2.20 × 10⁻⁴) = 1/(6.274 × 10⁷) = \textbf{15.9 nF}
\item
  Q₁ = ωL₁/R₁ = 534,071 × 180 × 10⁻⁶ / 0.12 = 96.1 / 0.12 = \textbf{801}
  Q₂ = ωL₂/R₂ = 534,071 × 220 × 10⁻⁶ / 0.12 = 117.5 / 0.12 =
  \textbf{979}
\item
  k²Q₁Q₂ = 0.0324 × 801 × 979 = 0.0324 × 784,179 = 25,407.
  η\textsubscript{max} = 25,407 / (1 + √(1 + 25,407))² = 25,407 / (1 +
  159.4)² = 25,407 / 25,729 = \textbf{98.7\%} (theoretical; practical
  DC-to-DC efficiency ≈ 88--92\%)
\end{enumerate}

\begin{center}\rule{0.5\linewidth}{0.5pt}\end{center}

\section{Problem 10.10.10}\label{problem-10.10.10}

\textbf{Given:} A Qi wireless charging system for a smartphone operates
at 150 kHz with transmitter coil L₁ = 10 μH, receiver coil L₂ = 6 μH,
coupling coefficient k = 0.55, and coil resistances R₁ = 0.3 Ω, R₂ = 0.5
Ω. The system delivers 10 W to the battery.

\textbf{Find:} (a) The mutual inductance, (b) the quality factor of each
coil, (c) the maximum theoretical coil-to-coil efficiency, and (d) the
estimated total system efficiency if the inverter and rectifier each
have 95\% efficiency.

\textbf{Solution:}

\begin{enumerate}
\def\labelenumi{(\alph{enumi})}
\item
  M = 0.55 × √(10 × 10⁻⁶ × 6 × 10⁻⁶) = 0.55 × √(6.0 × 10⁻¹¹) = 0.55 ×
  7.746 × 10⁻⁶ = \textbf{4.26 μH}
\item
  ω = 2π × 150,000 = 942,478 rad/s. Q₁ = 942,478 × 10 × 10⁻⁶ / 0.3 =
  9.425 / 0.3 = \textbf{31.4} Q₂ = 942,478 × 6 × 10⁻⁶ / 0.5 = 5.655 /
  0.5 = \textbf{11.3}
\item
  k²Q₁Q₂ = 0.3025 × 31.4 × 11.3 = 0.3025 × 354.8 = 107.3.
  η\textsubscript{max} = 107.3 / (1 + √(1 + 107.3))² = 107.3 / (1 +
  10.41)² = 107.3 / 130.2 = \textbf{82.4\%}
\item
  Total system efficiency: η\textsubscript{total} = η\textsubscript{inv}
  × η\textsubscript{coils} × η\textsubscript{rect} = 0.95 × 0.824 × 0.95
  = \textbf{74.4\%} DC input power required: P\textsubscript{in} = 10 /
  0.744 = 13.4 W
\end{enumerate}

\chapter{Chapter 10 --- Section 10.11:
Supercapacitors}\label{chapter-10-section-10.11-supercapacitors}

Practice problems covering EDLC fundamentals, pseudocapacitor and hybrid
supercapacitor energy comparison, ESR and constant-current discharge
analysis, series-parallel bank design, and applications including
regenerative braking and hybrid battery-supercapacitor systems.

\begin{center}\rule{0.5\linewidth}{0.5pt}\end{center}

\section{Problem 10.11.1}\label{problem-10.11.1}

\textbf{Given:} An EDLC cell uses two symmetric activated carbon
electrodes. Each electrode has a specific surface area of 2,000 m²/g, a
mass of 8 g, and a double-layer capacitance per unit area of 18 μF/cm².

\textbf{Find:} (a) The capacitance of one electrode. (b) The cell
capacitance (two electrodes in series). (c) The energy stored when
charged to V\textsubscript{max} = 2.5 V, in joules and watt-hours. (d)
The percentage of stored energy extracted when discharged to
V\textsubscript{max}/2.

\textbf{Solution:}

\begin{enumerate}
\def\labelenumi{(\alph{enumi})}
\tightlist
\item
  Total electrode area = 2,000 m²/g × 8 g = 16,000 m² = 16,000 × 10⁴ cm²
  = 1.6 × 10⁸ cm²
\end{enumerate}

C\textsubscript{electrode} = 18 × 10⁻⁶ F/cm² × 1.6 × 10⁸ cm² =
\textbf{2,880 F}

\begin{enumerate}
\def\labelenumi{(\alph{enumi})}
\setcounter{enumi}{1}
\item
  Two double layers in series (symmetric cell): C\textsubscript{cell} =
  C\textsubscript{electrode}/2 = 2,880/2 = \textbf{1,440 F}
\item
  E = ½ × C\textsubscript{cell} × V\textsubscript{max}² = ½ × 1,440 ×
  (2.5)² = ½ × 1,440 × 6.25 = \textbf{4,500 J (1.25 Wh)}
\item
  E\textsubscript{remaining} = ½ × 1,440 × (1.25)² = ½ × 1,440 × 1.5625
  = 1,125 J
\end{enumerate}

Fraction extracted = (4,500 − 1,125) / 4,500 = 3,375 / 4,500 =
\textbf{75\%} --- discharging to half voltage always recovers exactly
75\% of stored energy.

\begin{center}\rule{0.5\linewidth}{0.5pt}\end{center}

\section{Problem 10.11.2}\label{problem-10.11.2}

\textbf{Given:} A 600 F EDLC cell (V\textsubscript{max} = 2.7 V, mass =
85 g) is compared to a lithium-ion capacitor (LIC) cell of identical
mass with a specific energy of 22 Wh/kg.

\textbf{Find:} (a) The energy stored in the EDLC. (b) The energy stored
in the LIC. (c) The specific energy of the EDLC in Wh/kg. (d) The ratio
of LIC to EDLC specific energy.

\textbf{Solution:}

\begin{enumerate}
\def\labelenumi{(\alph{enumi})}
\item
  E\textsubscript{EDLC} = ½ × 600 × (2.7)² = ½ × 600 × 7.29 =
  \textbf{2,187 J (0.607 Wh)}
\item
  E\textsubscript{LIC} = 22 Wh/kg × 0.085 kg = \textbf{1.87 Wh}
\item
  Specific energy\textsubscript{EDLC} = 0.607 Wh / 0.085 kg =
  \textbf{7.14 Wh/kg}
\item
  Ratio = 22 / 7.14 = \textbf{3.08×} --- the LIC stores approximately 3×
  more energy per kilogram, consistent with its hybrid battery-EDLC
  architecture that raises the effective voltage window.
\end{enumerate}

\begin{center}\rule{0.5\linewidth}{0.5pt}\end{center}

\section{Problem 10.11.3}\label{problem-10.11.3}

\textbf{Given:} A 25 F supercapacitor module has ESR = 30 mΩ and is
charged to V₀ = 48 V. It is discharged at a constant current of 50 A.

\textbf{Find:} (a) The initial terminal voltage. (b) The terminal
voltage after 10 seconds. (c) The time for the capacitor voltage to
reach 24 V. (d) The maximum power deliverable to a matched load.

\textbf{Solution:}

\begin{enumerate}
\def\labelenumi{(\alph{enumi})}
\item
  V\textsubscript{terminal}(0) = V₀ − I × ESR = 48 − 50 × 0.030 =
  \textbf{46.5 V}
\item
  V\textsubscript{C}(10) = 48 − (I/C) × t = 48 − (50/25) × 10 = 48 − 20
  = 28 V
\end{enumerate}

V\textsubscript{terminal}(10) = 28 − 50 × 0.030 = 28 − 1.5 =
\textbf{26.5 V}

\begin{enumerate}
\def\labelenumi{(\alph{enumi})}
\setcounter{enumi}{2}
\item
  Setting V\textsubscript{C} = 24 V: 24 = 48 − 2t → t = 24/2 =
  \textbf{12 s}
\item
  P\textsubscript{max} = V₀² / (4 × ESR) = (48)² / (4 × 0.030) = 2,304 /
  0.120 = \textbf{19,200 W (19.2 kW)}
\end{enumerate}

\begin{center}\rule{0.5\linewidth}{0.5pt}\end{center}

\section{Problem 10.11.4}\label{problem-10.11.4}

\textbf{Given:} An industrial energy storage module must store at least
200 Wh at a 72 V nominal bus voltage using cells rated at 2.7 V, 5,000
F, ESR = 0.3 mΩ each.

\textbf{Find:} (a) The minimum cells per series string. (b) The string
capacitance and energy. (c) The number of parallel strings and total
capacitance. (d) The total module ESR.

\textbf{Solution:}

\begin{enumerate}
\def\labelenumi{(\alph{enumi})}
\item
  n\textsubscript{s} = ⌈72 / 2.7⌉ = ⌈26.67⌉ = \textbf{27 cells};
  V\textsubscript{string} = 27 × 2.7 = 72.9 V
\item
  C\textsubscript{string} = 5,000 / 27 = \textbf{185.2 F}
\end{enumerate}

E\textsubscript{string} = ½ × 185.2 × (72.9)² = ½ × 185.2 × 5,314.4 =
\textbf{492,114 J (136.7 Wh)}

\begin{enumerate}
\def\labelenumi{(\alph{enumi})}
\setcounter{enumi}{2}
\tightlist
\item
  n\textsubscript{p} = ⌈200 / 136.7⌉ = ⌈1.46⌉ = \textbf{2 strings}
\end{enumerate}

C\textsubscript{total} = 2 × 185.2 = \textbf{370.4 F};
E\textsubscript{total} = 2 × 136.7 = \textbf{273.4 Wh} ✓

\begin{enumerate}
\def\labelenumi{(\alph{enumi})}
\setcounter{enumi}{3}
\tightlist
\item
  ESR per string = n\textsubscript{s} × ESR\textsubscript{cell} = 27 ×
  0.3 × 10⁻³ = 8.1 mΩ
\end{enumerate}

Two strings in parallel: ESR\textsubscript{module} = 8.1 / 2 =
\textbf{4.05 mΩ}

\begin{center}\rule{0.5\linewidth}{0.5pt}\end{center}

\section{Problem 10.11.5}\label{problem-10.11.5}

\textbf{Given:} A crane lowers a 2,000 kg load through 5 m in 8 seconds.
During lowering, 70\% of the potential energy is recovered
regeneratively into a supercapacitor bank operating between
V\textsubscript{min} = 50 V and V\textsubscript{max} = 100 V. The motor
efficiency is 85\%.

\textbf{Find:} (a) The potential energy of the load. (b) The energy
captured in the supercapacitor. (c) The minimum bank capacitance. (d) If
the recovered energy assists the next identical lift (same 85\% motor
efficiency), the net energy drawn from the supply for the second lift.

\textbf{Solution:}

\begin{enumerate}
\def\labelenumi{(\alph{enumi})}
\item
  E\textsubscript{PE} = m × g × h = 2,000 × 9.81 × 5 = \textbf{98,100 J
  (98.1 kJ)}
\item
  E\textsubscript{regen} = 0.70 × 98,100 = \textbf{68,670 J (68.7 kJ)}
\item
  E\textsubscript{usable} = ½ × C × (V\textsubscript{max}² −
  V\textsubscript{min}²) = ½ × C × (100² − 50²) = 3,750 × C
\end{enumerate}

C = 68,670 / 3,750 = \textbf{18.3 F}

\begin{enumerate}
\def\labelenumi{(\alph{enumi})}
\setcounter{enumi}{3}
\tightlist
\item
  Energy to lift: E\textsubscript{lift} = E\textsubscript{PE} / η =
  98,100 / 0.85 = 115,412 J
\end{enumerate}

Energy delivered to motor from SC: E\textsubscript{assist} =
E\textsubscript{regen} × η = 68,670 × 0.85 = 58,370 J

Net supply energy = 115,412 − 58,370 = \textbf{57,042 J (57.0 kJ)} --- a
50.6\% reduction versus lifting without regeneration.

\begin{center}\rule{0.5\linewidth}{0.5pt}\end{center}

\section{Problem 10.11.6}\label{problem-10.11.6}

\textbf{Given:} A hybrid energy storage system on an electric bus
operates at a 400 V DC bus. The battery is limited to 30 kW continuous
output; the supercapacitor bank handles all power above 30 kW. During
acceleration, the total demand is 120 kW for 6 seconds.

\textbf{Find:} (a) The total energy required during acceleration. (b)
The energy supplied by the battery. (c) The energy supplied by the
supercapacitor. (d) The peak battery current with the supercapacitor.
(e) The peak battery current without the supercapacitor. (f) The battery
peak-current reduction factor.

\textbf{Solution:}

\begin{enumerate}
\def\labelenumi{(\alph{enumi})}
\item
  E\textsubscript{total} = 120,000 × 6 = \textbf{720,000 J (720 kJ)}
\item
  E\textsubscript{battery} = 30,000 × 6 = \textbf{180,000 J (180 kJ)}
\item
  E\textsubscript{SC} = 720,000 − 180,000 = \textbf{540,000 J (540 kJ)}
\item
  I\textsubscript{battery} (with SC) = P\textsubscript{battery} /
  V\textsubscript{bus} = 30,000 / 400 = \textbf{75 A}
\item
  I\textsubscript{battery} (without SC) = P\textsubscript{total} /
  V\textsubscript{bus} = 120,000 / 400 = \textbf{300 A}
\item
  Reduction factor = 300 / 75 = \textbf{4×} --- the supercapacitor
  reduces peak battery current by a factor of four, significantly
  reducing battery stress and extending cycle life.
\end{enumerate}

\begin{center}\rule{0.5\linewidth}{0.5pt}\end{center}

\chapter{Chapter 10 --- Section 10.12: Solar Photovoltaic
Systems}\label{chapter-10-section-10.12-solar-photovoltaic-systems}

Practice problems covering PV cell I-V characteristics and fill factor,
maximum power point tracking with the P\&O algorithm, grid-tied inverter
sizing and anti-islanding, off-grid battery and array sizing, string
voltage compliance and shading analysis, and whole-system energy
economics.

\begin{center}\rule{0.5\linewidth}{0.5pt}\end{center}

\section{Problem 10.12.1}\label{problem-10.12.1}

\textbf{Given:} A 125 mm × 125 mm monocrystalline silicon PV cell (area
= 0.0156 m²) has the following STC parameters: I\textsubscript{sc} = 7.5
A, V\textsubscript{oc} = 0.620 V, I\textsubscript{mp} = 7.1 A,
V\textsubscript{mp} = 0.530 V.

\textbf{Find:} (a) The fill factor. (b) P\textsubscript{max}. (c) Cell
efficiency at STC (G = 1,000 W/m²). (d) At 60 °C, V\textsubscript{oc}
decreases at −2.2 mV/°C from the 25 °C baseline. Calculate the new
V\textsubscript{oc} and the percentage reduction.

\textbf{Solution:}

\begin{enumerate}
\def\labelenumi{(\alph{enumi})}
\item
  FF = (V\textsubscript{mp} × I\textsubscript{mp}) /
  (V\textsubscript{oc} × I\textsubscript{sc}) = (0.530 × 7.1) / (0.620 ×
  7.5) = 3.763 / 4.650 = \textbf{0.809 (80.9\%)}
\item
  P\textsubscript{max} = V\textsubscript{mp} × I\textsubscript{mp} =
  0.530 × 7.1 = \textbf{3.763 W}
\item
  η = P\textsubscript{max} / (G × A) = 3.763 / (1,000 × 0.0156) = 3.763
  / 15.6 = \textbf{24.1\%}
\item
  ΔT = 60 − 25 = 35 °C; ΔV\textsubscript{oc} = −2.2 × 10⁻³ × 35 = −0.077
  V
\end{enumerate}

V\textsubscript{oc}(60 °C) = 0.620 − 0.077 = \textbf{0.543 V} --- a
12.4\% reduction, consistent with the −0.3 \%/°C temperature coefficient
typical of silicon.

\begin{center}\rule{0.5\linewidth}{0.5pt}\end{center}

\section{Problem 10.12.2}\label{problem-10.12.2}

\textbf{Given:} A PV module has V\textsubscript{oc} = 48 V,
I\textsubscript{sc} = 10.5 A, V\textsubscript{mp} = 38.5 V,
I\textsubscript{mp} = 9.8 A. A boost converter MPPT stage steps up to a
600 V DC bus with 97\% efficiency.

\textbf{Find:} (a) Module P\textsubscript{max}. (b) Boost converter duty
cycle at MPP (ideal). (c) DC bus power delivered. (d) A P\&O algorithm
at V = 37.0 V measures P = 362 W, then perturbs to V = 37.5 V and
measures P = 368 W. State the direction of the next perturbation and the
reason.

\textbf{Solution:}

\begin{enumerate}
\def\labelenumi{(\alph{enumi})}
\item
  P\textsubscript{mp} = V\textsubscript{mp} × I\textsubscript{mp} = 38.5
  × 9.8 = \textbf{377.3 W}
\item
  Ideal boost: 1 − D = V\textsubscript{in} / V\textsubscript{out} = 38.5
  / 600 = 0.0642 → D = \textbf{0.936 (93.6\%)}
\item
  P\textsubscript{dc} = P\textsubscript{mp} × η = 377.3 × 0.97 =
  \textbf{365.9 W}
\item
  ΔP = 368 − 362 = +6 W \textgreater{} 0 with ΔV = +0.5 V → power
  increased with a voltage increase → \textbf{perturb in the same
  direction (increase V further)}, as the operating point is still on
  the left side of the MPP.
\end{enumerate}

\begin{center}\rule{0.5\linewidth}{0.5pt}\end{center}

\section{Problem 10.12.3}\label{problem-10.12.3}

\textbf{Given:} A three-phase grid-tied PV inverter connects to a 480
V\textsubscript{L-L} / 60 Hz grid. The DC bus voltage is 750 V. The
inverter delivers 15 kW at unity power factor with a 20 kHz switching
frequency.

\textbf{Find:} (a) Phase voltage (V\textsubscript{rms} per phase). (b)
RMS line current. (c) SPWM modulation index m = V̂\textsubscript{phase} /
V\textsubscript{DC}. (d) Number of switching cycles in the 2-second
anti-islanding detection window required by IEEE 1547.

\textbf{Solution:}

\begin{enumerate}
\def\labelenumi{(\alph{enumi})}
\item
  V\textsubscript{phase} = V\textsubscript{L-L} / √3 = 480 / √3 =
  \textbf{277.1 V\textsubscript{rms}}
\item
  I\textsubscript{line} = P / (√3 × V\textsubscript{L-L}) = 15,000 / (√3
  × 480) = 15,000 / 831.4 = \textbf{18.0 A\textsubscript{rms}}
\item
  V̂\textsubscript{phase} = 277.1 × √2 = 391.9 V; m = 391.9 / 750 =
  \textbf{0.523}
\item
  N\textsubscript{cycles} = f\textsubscript{sw} × t = 20,000 × 2 =
  \textbf{40,000 switching cycles}
\end{enumerate}

\begin{center}\rule{0.5\linewidth}{0.5pt}\end{center}

\section{Problem 10.12.4}\label{problem-10.12.4}

\textbf{Given:} An off-grid remote weather station has a daily load of
400 Wh at a 12 V bus. The system must provide 5 days of autonomy with
flooded lead-acid batteries (50\% maximum DoD). Peak sun hours = 4.5
h/day; system efficiency from PV to load = 75\%. Modules available have
V\textsubscript{mp} = 22.5 V and I\textsubscript{mp} = 5.6 A with an
MPPT charge controller that steps down to the 12 V bus.

\textbf{Find:} (a) Minimum battery bank capacity in Ah. (b) Minimum PV
array peak power. (c) Power per module and number of modules required.
(d) At the charge controller output (12 V bus), calculate the charge
current delivered by one module at MPP (assume 92\% controller
efficiency).

\textbf{Solution:}

\begin{enumerate}
\def\labelenumi{(\alph{enumi})}
\tightlist
\item
  Autonomy energy = 5 × 400 = 2,000 Wh. At 50\% DoD, total battery
  energy = 2,000 / 0.50 = 4,000 Wh.
\end{enumerate}

Capacity = 4,000 / 12 = \textbf{333 Ah}

\begin{enumerate}
\def\labelenumi{(\alph{enumi})}
\setcounter{enumi}{1}
\item
  E\textsubscript{pv} = 400 / 0.75 = 533.3 Wh/day; P\textsubscript{pv} =
  533.3 / 4.5 = \textbf{118.5 W} (peak array power)
\item
  P\textsubscript{module} = V\textsubscript{mp} × I\textsubscript{mp} =
  22.5 × 5.6 = 126 W; N = ⌈118 / 126⌉ = \textbf{1 module} (126 W
  \textgreater{} 118 W required) ✓
\item
  P\textsubscript{controller out} = 126 × 0.92 = 115.9 W;
  I\textsubscript{charge} = 115.9 / 12 = \textbf{9.66 A} at the 12 V
  bus.
\end{enumerate}

\begin{center}\rule{0.5\linewidth}{0.5pt}\end{center}

\section{Problem 10.12.5}\label{problem-10.12.5}

\textbf{Given:} A PV array uses strings of 12 series-connected 60-cell
modules. Each module: V\textsubscript{oc} = 45.5 V, I\textsubscript{sc}
= 9.2 A, V\textsubscript{mp} = 37.0 V, I\textsubscript{mp} = 8.7 A. The
inverter MPPT window is 250--600 V DC; maximum DC input voltage is 650
V. At low temperature (−10 °C), V\textsubscript{oc} increases at +0.34
\%/°C from the 25 °C baseline.

\textbf{Find:} (a) String V\textsubscript{oc} and V\textsubscript{mp} at
STC. (b) Verify the string fits the inverter MPPT window. (c) Calculate
worst-case (cold) string V\textsubscript{oc} at −10 °C and verify it
does not exceed 650 V. (d) One module falls completely into shade and
its bypass diodes activate. Calculate the shaded string power and the
percentage power loss.

\textbf{Solution:}

\begin{enumerate}
\def\labelenumi{(\alph{enumi})}
\item
  String V\textsubscript{oc} = 12 × 45.5 = \textbf{546 V}; String
  V\textsubscript{mp} = 12 × 37.0 = \textbf{444 V}
\item
  250 V ≤ 444 V ≤ 600 V ✓; V\textsubscript{oc} = 546 V \textless{} 650 V
  max ✓ --- \textbf{within MPPT window and below maximum input voltage}
\item
  ΔT = 25 − (−10) = 35 °C below STC; V\textsubscript{oc} rise = 0.0034 ×
  35 × 546 = 65.0 V
\end{enumerate}

V\textsubscript{oc}(−10 °C) = 546 + 65 = \textbf{611 V} --- below the
650 V maximum ✓ (6.0\% margin)

\begin{enumerate}
\def\labelenumi{(\alph{enumi})}
\setcounter{enumi}{3}
\tightlist
\item
  Shaded module: all 3 bypass diodes active; module contributes ≈ 0 V
  and 0 W.
\end{enumerate}

V\textsubscript{string,shaded} = 11 × 37.0 = 407 V; I\textsubscript{mp}
unchanged at 8.7 A

P\textsubscript{shaded} = 407 × 8.7 = 3,540.9 W; P\textsubscript{full} =
444 × 8.7 = 3,862.8 W

Loss = 3,862.8 − 3,540.9 = 321.9 W → \textbf{(8.3\%)} --- equal to 1/12
of string power, as expected for one bypassed module.

\begin{center}\rule{0.5\linewidth}{0.5pt}\end{center}

\section{Problem 10.12.6}\label{problem-10.12.6}

\textbf{Given:} A 10 kW nominal grid-tied rooftop PV system is built
from 25 modules (V\textsubscript{mp} = 40 V, I\textsubscript{mp} = 10 A
each) arranged in 5 strings × 5 modules per string. The inverter has
96\% efficiency. The installation site receives 1,400 peak sun hours per
year (equivalent to 1,400 kWh/year per kW of installed capacity). The
installed system cost is \$25,000 and the local electricity rate is
\$0.15/kWh.

\textbf{Find:} (a) Array V\textsubscript{mp}, total I\textsubscript{mp},
and array P\textsubscript{max}. (b) Inverter AC output power at MPP. (c)
Annual AC energy production. (d) Annual electricity cost savings and
simple payback period.

\textbf{Solution:}

\begin{enumerate}
\def\labelenumi{(\alph{enumi})}
\tightlist
\item
  String V\textsubscript{mp} = 5 × 40 = 200 V; total I\textsubscript{mp}
  = 5 strings × 10 A = 50 A
\end{enumerate}

P\textsubscript{max} = 200 × 50 = \textbf{10,000 W} --- consistent with
25 modules × (40 × 10) = 25 × 400 = 10,000 W ✓

\begin{enumerate}
\def\labelenumi{(\alph{enumi})}
\setcounter{enumi}{1}
\item
  P\textsubscript{AC} = P\textsubscript{max} × η\textsubscript{inv} =
  10,000 × 0.96 = \textbf{9,600 W (9.6 kW)}
\item
  Annual energy = 10 kW × 1,400 kWh/kW/yr = \textbf{14,000 kWh/year}
\item
  Annual savings = 14,000 × \$0.15 = \$2,100/yr
\end{enumerate}

Simple payback = \$25,000 / \$2,100 = \textbf{11.9 years} --- a typical
result for residential PV without incentives; tax credits and feed-in
tariffs reduce payback to 5--8 years in most markets.

\begin{center}\rule{0.5\linewidth}{0.5pt}\end{center}

\chapter{Chapter 11 -- Section 11.1: Measurement
Fundamentals}\label{chapter-11-section-11.1-measurement-fundamentals}

Practice problems covering accuracy, precision, resolution, error
analysis, calibration, signal-to-noise ratio, measurement uncertainty,
and bridge circuits.

\begin{center}\rule{0.5\linewidth}{0.5pt}\end{center}

\section{Problem 11.1.1}\label{problem-11.1.1}

\textbf{Given:} A 12-bit ADC has an input range of 0-3.3 V. A pressure
sensor output is measured 8 times, yielding readings (in volts): 1.4523,
1.4519, 1.4526, 1.4520, 1.4524, 1.4521, 1.4518, 1.4525. The true sensor
voltage (from a precision calibrator) is 1.4550 V.

\textbf{Find:} The ADC resolution, the measurement accuracy (systematic
error), and the measurement precision (standard deviation).

\textbf{Solution:} Resolution = full-scale range / 2\textsuperscript{N}
= 3.3 / 2\textsuperscript{12} = 3.3 / 4,096 = 0.806 mV per count.

Mean of measurements: x̄ = (1.4523 + 1.4519 + 1.4526 + 1.4520 + 1.4524 +
1.4521 + 1.4518 + 1.4525) / 8 = 11.6176 / 8 = 1.45220 V.

Accuracy (systematic error): \textbar1.4550 - 1.4522\textbar{} = 2.8 mV,
or (2.8 / 1.4550) x 100 = 0.192\% of reading.

Deviations from mean (mV): +0.1, -0.3, +0.4, -0.2, +0.2, -0.1, -0.4,
+0.3.

Precision (standard deviation): s = sqrt{[}sum(x\textsubscript{i} -
x̄)\^{}2 / (N-1){]} = sqrt{[}(0.01 + 0.09 + 0.16 + 0.04 + 0.04 + 0.01 +
0.16 + 0.09) x 10\^{}-6 / 7{]} = sqrt{[}0.60 x 10\^{}-6 / 7{]} =
sqrt{[}8.571 x 10\^{}-8{]} = \textbf{0.293 mV}.

\textbf{The instrument has good precision (0.293 mV spread) but a 2.8 mV
systematic offset indicating a calibration correction is needed.}

\begin{center}\rule{0.5\linewidth}{0.5pt}\end{center}

\section{Problem 11.1.2}\label{problem-11.1.2}

\textbf{Given:} A DMM is calibrated at four points against a
NIST-traceable reference. The reference values and meter readings are:
0.500 V reads 0.502 V, 1.500 V reads 1.505 V, 2.500 V reads 2.508 V,
3.500 V reads 3.511 V.

\textbf{Find:} The gain error, offset error, and the corrected reading
when the DMM displays 2.000 V.

\textbf{Solution:} Using a linear error model: V\textsubscript{reading}
= G x V\textsubscript{true} + V\textsubscript{offset}.

Using the first and last calibration points: G = (3.511 - 0.502) /
(3.500 - 0.500) = 3.009 / 3.000 = 1.00300.

V\textsubscript{offset} = V\textsubscript{reading} - G x
V\textsubscript{true} = 0.502 - 1.00300 x 0.500 = 0.502 - 0.50150 = 0.50
mV.

Gain error = (G - 1) x 100\% = \textbf{0.300\%}. Offset error =
\textbf{0.50 mV}.

Verification at 2.500 V: predicted = 1.00300 x 2.500 + 0.00050 = 2.50800
V, actual = 2.508 V -- confirmed.

Corrected reading when DMM displays 2.000 V: V\textsubscript{true} =
(V\textsubscript{reading} - V\textsubscript{offset}) / G = (2.000 -
0.00050) / 1.00300 = 1.99950 / 1.00300 = \textbf{1.9935 V}.

\begin{center}\rule{0.5\linewidth}{0.5pt}\end{center}

\section{Problem 11.1.3}\label{problem-11.1.3}

\textbf{Given:} A strain gauge amplifier has a signal output of 12
mV\textsubscript{rms}. The sensor source resistance is 120 kOhm at a
temperature of 30 degrees C. The measurement system has a bandwidth of
200 kHz. Boltzmann constant k = 1.381 x 10\^{}-23 J/K.

\textbf{Find:} The thermal noise voltage, the SNR in dB, and the number
of signal averages needed to achieve 70 dB SNR.

\textbf{Solution:} Temperature in Kelvin: T = 30 + 273.15 = 303.15 K.

Thermal noise voltage: V\textsubscript{n} = sqrt(4kTR delta-f) = sqrt(4
x 1.381 x 10\^{}-23 x 303.15 x 120,000 x 200,000) = sqrt(4 x 1.381 x
10\^{}-23 x 7.276 x 10\^{}9) = sqrt(4.021 x 10\^{}-13) = \textbf{20.05
uV\textsubscript{rms}}.

SNR = 20 log10(V\textsubscript{signal} / V\textsubscript{noise}) = 20
log10(12 x 10\^{}-3 / 20.05 x 10\^{}-6) = 20 log10(598.5) = 20 x 2.777 =
\textbf{55.5 dB}.

Improvement needed for 70 dB: 70 - 55.5 = 14.5 dB. Voltage ratio
improvement = 10\^{}(14.5/20) = 5.31. Since averaging improves SNR by
sqrt(N): sqrt(N) = 5.31, so N = 28.2.

\textbf{Round up to N = 29 averages to achieve 70 dB SNR.}

\begin{center}\rule{0.5\linewidth}{0.5pt}\end{center}

\section{Problem 11.1.4}\label{problem-11.1.4}

\textbf{Given:} A pressure transmitter reads 250.5 kPa. The uncertainty
budget includes: (1) repeatability from 10 measurements with standard
deviation s = 0.35 kPa, (2) calibration certificate uncertainty of
+/-0.20 kPa at k = 2, (3) display resolution of 0.1 kPa, (4) temperature
effect estimated as +/-0.25 kPa (rectangular distribution), and (5)
hysteresis estimated as +/-0.10 kPa (rectangular distribution).

\textbf{Find:} The combined standard uncertainty and the expanded
uncertainty at 95\% confidence (k = 2).

\textbf{Solution:} Type A: u1 = s / sqrt(N) = 0.35 / sqrt(10) = 0.35 /
3.162 = \textbf{0.1107 kPa}.

Type B components: Calibration: u2 = 0.20 / 2 = \textbf{0.100 kPa}
(convert from k = 2). Resolution: u3 = 0.1 / (2 sqrt(3)) = 0.1 / 3.464 =
\textbf{0.0289 kPa} (half-width, rectangular). Temperature: u4 = 0.25 /
sqrt(3) = \textbf{0.1443 kPa} (rectangular). Hysteresis: u5 = 0.10 /
sqrt(3) = \textbf{0.0577 kPa} (rectangular).

Combined standard uncertainty: u\textsubscript{c} = sqrt(0.1107\^{}2 +
0.100\^{}2 + 0.0289\^{}2 + 0.1443\^{}2 + 0.0577\^{}2) = sqrt(0.01225 +
0.01000 + 0.000835 + 0.02082 + 0.00333) = sqrt(0.04724) = \textbf{0.217
kPa}.

Expanded uncertainty at 95\% (k = 2): U = 2 x 0.217 = \textbf{0.43 kPa}.

The result is reported as \textbf{250.5 +/- 0.4 kPa (k = 2, 95\%
confidence)}. The temperature effect is the dominant contributor (44\%
of variance).

\begin{center}\rule{0.5\linewidth}{0.5pt}\end{center}

\section{Problem 11.1.5}\label{problem-11.1.5}

\textbf{Given:} A Wheatstone bridge measures a platinum RTD. The bridge
has R1 = 100 Ohm, R2 = 100 Ohm, and R3 (decade box) is adjusted to 138.5
Ohm for a null reading. The excitation voltage is 5 V. The galvanometer
sensitivity is 8 uA/mV and the minimum detectable current is 0.05 uA.

\textbf{Find:} The unknown RTD resistance, the corresponding temperature
(using Pt100: R(T) = 100(1 + 0.00385T)), and the measurement resolution
in ohms and degrees C.

\textbf{Solution:} At balance: R\textsubscript{x} = R2 x R3 / R1 = 100 x
138.5 / 100 = \textbf{138.5 Ohm}.

Temperature from Pt100 equation: 138.5 = 100(1 + 0.00385T) 1.385 = 1 +
0.00385T T = 0.385 / 0.00385 = \textbf{100.0 degrees C}.

Minimum detectable imbalance voltage: V\textsubscript{min} =
I\textsubscript{min} / sensitivity = 0.05 uA / 8 uA/mV = 0.00625 mV.

Bridge sensitivity near balance (all arms \textasciitilde{} R): R
\textasciitilde{} sqrt(R1 x R2) = sqrt(100 x 100) = 100 Ohm (geometric
mean of fixed arms). dV/dR \textasciitilde{} V\textsubscript{ex} / (4R)
= 5 / (4 x 100) = 12.5 mV/Ohm.

Minimum detectable resistance change: delta-R = V\textsubscript{min} /
(dV/dR) = 0.00625 / 12.5 = \textbf{0.0005 Ohm}.

Temperature resolution: dR/dT = R0 x alpha = 100 x 0.00385 = 0.385
Ohm/degrees C. delta-T = delta-R / (dR/dT) = 0.0005 / 0.385 =
\textbf{0.0013 degrees C}.

The bridge achieves millidegree temperature resolution -- far superior
to direct resistance measurement with a DMM.

\begin{center}\rule{0.5\linewidth}{0.5pt}\end{center}

\section{Problem 11.1.6}\label{problem-11.1.6}

\textbf{Given:} A Kelvin double bridge measures a 0.001 Ohm shunt
resistor used for current sensing. The bridge uses a standard resistor
R\textsubscript{s} = 0.001000 Ohm (certified), ratio arms R1 = R3 = 10
Ohm, and auxiliary ratio arms r1 = r3 = 10 Ohm. The link resistance
between the standard and unknown is r\textsubscript{link} = 0.005 Ohm.
At balance, R2 = 10.045 Ohm and r2 = 10.045 Ohm.

\textbf{Find:} The unknown shunt resistance and the error that would
result from using a simple Wheatstone bridge (ignoring lead resistance
effects).

\textbf{Solution:} For a Kelvin double bridge at balance:
R\textsubscript{x} = R\textsubscript{s} x (R2/R1) +
r\textsubscript{link} x {[}(R2/R1) - (r2/r1){]} / {[}1 + (r1 + r2 +
r\textsubscript{link})/r3\ldots{} {]}

With matched ratio arms (R2/R1 = r2/r1): R\textsubscript{x} =
R\textsubscript{s} x (R2/R1) = 0.001000 x (10.045/10) = 0.001000 x
1.0045 = \textbf{0.0010045 Ohm = 1.0045 mOhm}.

The matched ratio condition (R2/R1 = r2/r1 = 1.0045) eliminates the link
resistance term entirely.

In a simple Wheatstone bridge, the lead resistances (say 0.005 Ohm per
lead, two leads = 0.010 Ohm total) add directly to the measured
resistance: R\textsubscript{measured} = R\textsubscript{x} +
R\textsubscript{leads} = 0.0010045 + 0.010 = 0.0110 Ohm.

Error = (0.0110 - 0.0010045) / 0.0010045 x 100\% = \textbf{995\% error}.

\textbf{The Kelvin bridge eliminates the 10 mOhm lead resistance that
would completely swamp the 1 mOhm measurement in a simple Wheatstone
bridge.}

\begin{center}\rule{0.5\linewidth}{0.5pt}\end{center}

\section{Problem 11.1.7}\label{problem-11.1.7}

\textbf{Given:} An instrumentation system measures a 1 mV thermocouple
signal using a 24-bit sigma-delta ADC with a 2.5 V reference. The system
has a 10 Hz measurement bandwidth. The ADC datasheet specifies an
effective number of bits (ENOB) of 20.5 at 10 Hz output rate. The
thermal noise from the thermocouple's 50 Ohm source resistance at 25
degrees C contributes additional noise.

\textbf{Find:} The ADC quantization noise, the thermal noise, the total
system SNR, and whether the system can resolve a 0.1 uV signal change.

\textbf{Solution:} ADC resolution with full 24 bits: LSB = 2.5 / 2\^{}24
= 2.5 / 16,777,216 = 0.149 uV.

Effective resolution at ENOB = 20.5: effective LSB = 2.5 / 2\^{}20.5 =
2.5 / 1,488,828 = 1.679 uV.

ADC RMS noise: V\textsubscript{n,ADC} = effective LSB / sqrt(12) = 1.679
/ 3.464 = \textbf{0.485 uV\textsubscript{rms}}.

Thermal noise: V\textsubscript{n,th} = sqrt(4 x 1.381 x 10⁻²³ x 298 x 50
x 10) = sqrt(8.23 x 10⁻¹⁸) = \textbf{0.00287 uV\textsubscript{rms}}
(2.87 nV).

Total noise (RSS): V\textsubscript{n,total} = sqrt(0.485² + 0.00287²) =
sqrt(0.2352 + 0.0000082) = sqrt(0.2352) = \textbf{0.485
uV\textsubscript{rms}}.

SNR for 1 mV signal: SNR = 20 log10(1000 / 0.486) = 20 log10(2058) = 20
x 3.313 = \textbf{66.3 dB}.

To resolve a 0.1 uV change, the signal change must exceed the
peak-to-peak noise (\textasciitilde{} 6.6 x V\textsubscript{n} for
99.9\% confidence): 6.6 x 0.486 = 3.21 uV \textgreater\textgreater{} 0.1
uV.

\textbf{The system cannot resolve a 0.1 uV change. Averaging N =
(3.21/0.1)\^{}2 = 1,031 samples would be needed, requiring
\textasciitilde103 seconds at 10 Hz output rate.}

\begin{center}\rule{0.5\linewidth}{0.5pt}\end{center}

\section{Problem 11.1.8}\label{problem-11.1.8}

\textbf{Given:} A Maxwell bridge is used to measure an unknown inductor
at 1 kHz. The bridge components are: R1 = 1,000 Ohm, R2 = 2,200 Ohm, C3
= 47 nF (in parallel with R3), and R3 = 4,700 Ohm. At balance, the
detector reads null.

\textbf{Find:} The unknown inductance L\textsubscript{x} and its series
resistance R\textsubscript{x}, and the Q factor of the inductor at 1
kHz.

\textbf{Solution:} For a Maxwell bridge at balance: L\textsubscript{x} =
R1 x R2 x C3 = 1,000 x 2,200 x 47 x 10\^{}-9 = \textbf{103.4 mH}.

R\textsubscript{x} = R1 x R2 / R3 = 1,000 x 2,200 / 4,700 =
\textbf{468.1 Ohm}.

Inductive reactance at 1 kHz: X\textsubscript{L} = 2 pi f L = 2 pi x
1,000 x 0.1034 = \textbf{649.8 Ohm}.

Q factor: Q = X\textsubscript{L} / R\textsubscript{x} = 649.8 / 468.1 =
\textbf{1.39}.

Equivalently: Q = omega x C3 x R3 = 2 pi x 1,000 x 47 x 10\^{}-9 x 4,700
= 1.39 -- confirmed.

\textbf{The inductor has L = 103.4 mH, R\textsubscript{x} = 468.1 Ohm,
and Q = 1.39 at 1 kHz, indicating a lossy inductor typical of a small
iron-core choke.}

\chapter{Chapter 11 -- Section 11.2: Sensors and
Transducers}\label{chapter-11-section-11.2-sensors-and-transducers}

Practice problems covering temperature sensors, strain gauges, pressure
sensors, proximity/position sensors, accelerometers, magnetic field
sensors, current sensors, optical sensors, and flow sensors.

\begin{center}\rule{0.5\linewidth}{0.5pt}\end{center}

\section{Problem 11.2.1}\label{problem-11.2.1}

\textbf{Given:} A Pt1000 RTD (R0 = 1,000 Ohm at 0 degrees C, alpha =
0.00385 Ohm/Ohm/degrees C) is used in a cooling system. The measured
resistance is 1,154.0 Ohm. A Type T thermocouple at the same location
produces a voltage of 1.612 mV (reference junction at 0 degrees C). The
Type T sensitivity is approximately 43 uV/degrees C near the measurement
point.

\textbf{Find:} The RTD temperature, the thermocouple temperature
estimate, and the error in the linear thermocouple approximation.

\textbf{Solution:} RTD temperature: R(T) = R0(1 + alpha T) 1,154.0 =
1,000(1 + 0.00385T) 1.154 = 1 + 0.00385T T = 0.154 / 0.00385 =
\textbf{40.0 degrees C}.

Thermocouple linear estimate: T = V / sensitivity = 1.612 x 10\^{}-3 /
43 x 10\^{}-6 = \textbf{37.5 degrees C}.

Error = 40.0 - 37.5 = 2.5 degrees C (6.3\% error).

From Type T tables, 1.612 mV corresponds to approximately 39.8 degrees
C, very close to the RTD reading.

\textbf{The RTD reads 40.0 degrees C. The linear thermocouple
approximation gives 37.5 degrees C (2.5 degrees C error), demonstrating
the need for thermocouple reference tables or polynomial corrections.}

\begin{center}\rule{0.5\linewidth}{0.5pt}\end{center}

\section{Problem 11.2.2}\label{problem-11.2.2}

\textbf{Given:} A full-bridge strain gauge configuration uses four 120
Ohm gauges (GF = 2.0) bonded to an aluminum cantilever beam (Young's
modulus E = 70 GPa). Two gauges are in tension (top) and two in
compression (bottom). The excitation voltage is V\textsubscript{ex} = 10
V, and the measured bridge output is V\textsubscript{out} = 8.5 mV.

\textbf{Find:} The strain, the stress, and the resistance change in each
gauge.

\textbf{Solution:} For a full-bridge (4 active arms):
V\textsubscript{out} = GF x epsilon x V\textsubscript{ex}.

Solving for strain: epsilon = V\textsubscript{out} / (GF x
V\textsubscript{ex}) = 8.5 x 10\^{}-3 / (2.0 x 10) = 8.5 x 10\^{}-3 / 20
= \textbf{425 x 10\^{}-6 = 425 microstrain}.

Stress: sigma = E x epsilon = 70 x 10\^{}9 x 425 x 10\^{}-6 =
\textbf{29.75 MPa}.

Resistance change per gauge: delta-R = GF x epsilon x R = 2.0 x 425 x
10\^{}-6 x 120 = \textbf{0.102 Ohm} (a change of 0.085\%).

\textbf{The full-bridge provides 4x the sensitivity of a quarter-bridge,
giving V\textsubscript{out} = 8.5 mV compared to \textasciitilde2.1 mV
for a quarter-bridge at the same strain.}

\begin{center}\rule{0.5\linewidth}{0.5pt}\end{center}

\section{Problem 11.2.3}\label{problem-11.2.3}

\textbf{Given:} A piezoresistive MEMS pressure sensor has a full-scale
range of 0-500 kPa (gauge), a sensitivity of 20 mV/V/500 kPa, and an
excitation voltage of 5 V. The sensor nonlinearity is +/-0.25\% of full
scale, and the temperature coefficient of offset is 0.02\% FS/degrees C.
The sensor operates in an environment that varies from 10 degrees C to
50 degrees C (calibrated at 25 degrees C).

\textbf{Find:} The full-scale output voltage, the output voltage at 350
kPa, the nonlinearity error in mV and kPa, and the worst-case
temperature-induced offset error.

\textbf{Solution:} Full-scale output: V\textsubscript{FS} = sensitivity
x V\textsubscript{ex} = 20 mV/V x 5 V = \textbf{100 mV}.

Output at 350 kPa: V\textsubscript{out} = (350/500) x 100 = \textbf{70.0
mV}.

Nonlinearity error: +/-0.25\% of FS = +/-0.0025 x 100 mV =
\textbf{+/-0.25 mV}. In pressure units: +/-0.0025 x 500 =
\textbf{+/-1.25 kPa}.

Worst-case temperature deviation from calibration: max(\textbar50 -
25\textbar, \textbar10 - 25\textbar) = 25 degrees C. Temperature offset
error: 0.02\% FS/degrees C x 25 degrees C = 0.50\% FS = 0.005 x 100 mV =
\textbf{0.50 mV}. In pressure units: 0.005 x 500 = \textbf{2.5 kPa}.

\textbf{Total worst-case error (offset + nonlinearity): +/-(0.25 + 0.50)
= +/-0.75 mV or +/-3.75 kPa (0.75\% FS).}

\begin{center}\rule{0.5\linewidth}{0.5pt}\end{center}

\section{Problem 11.2.4}\label{problem-11.2.4}

\textbf{Given:} An incremental rotary encoder has 1,024 lines and is
used with quadrature decoding on a motor shaft. The encoder drives a
ball screw with 10 mm pitch through a 5:1 gear reducer. The encoder
produces an index pulse once per revolution.

\textbf{Find:} The angular resolution at the motor shaft, the linear
resolution at the ball screw output, and the maximum positional error
that can accumulate before the index pulse resets it.

\textbf{Solution:} Quadrature decoding: counts/rev = 1,024 x 4 =
\textbf{4,096 counts/rev} at the motor shaft.

Angular resolution: theta = 360 degrees / 4,096 = \textbf{0.0879 degrees
per count} = 0.001534 rad.

With 5:1 gear reduction, counts per revolution at ball screw: 4,096 x 5
= \textbf{20,480 counts/rev} at the output.

Linear resolution: d = ball screw pitch / counts per screw rev = 10 mm /
20,480 = \textbf{0.000488 mm = 0.488 um per count}.

Maximum positional error accumulation: The index pulse occurs once per
motor revolution (every 4,096 counts). Linear travel per motor
revolution = 10 mm / 5 = 2 mm. If one count is missed, the error is
0.488 um and persists until the next index pulse resets the count.

\textbf{Maximum error before index reset: +/-1 count = +/-0.488 um,
accumulated over at most 2 mm of travel. The index pulse provides
absolute reference every 2 mm of linear travel.}

\begin{center}\rule{0.5\linewidth}{0.5pt}\end{center}

\section{Problem 11.2.5}\label{problem-11.2.5}

\textbf{Given:} A 3-axis MEMS accelerometer has a sensitivity of 16,384
LSB/g, a +/-2 g measurement range, a noise density of 80 ug/sqrt(Hz),
and a -3 dB bandwidth of 1,000 Hz. The accelerometer is used to measure
tilt angle by sensing the gravity vector.

\textbf{Find:} The RMS noise floor in mg, the minimum detectable tilt
angle, and the dynamic range in dB.

\textbf{Solution:} Total RMS noise: a\textsubscript{noise} = noise
density x sqrt(BW) = 80 x 10\^{}-6 x sqrt(1,000) = 80 x 10\^{}-6 x 31.62
= \textbf{2.53 mg\textsubscript{rms}}.

For tilt sensing, 1 degree of tilt produces: a = sin(1 degree) x 1 g =
0.01745 g = 17.45 mg.

Minimum detectable tilt (at 3-sigma for 99.7\% confidence): 3 x 2.53 =
7.59 mg. Tilt angle = arcsin(7.59 x 10\^{}-3) = \textbf{0.435 degrees}.

Full-scale range: 4 g (from -2 g to +2 g). Dynamic range: DR = 20 x
log10(full-scale / noise) = 20 x log10(4 / 2.53 x 10\^{}-3) = 20 x
log10(1,581) = 20 x 3.199 = \textbf{64.0 dB}.

\textbf{The accelerometer can resolve tilt angles down to approximately
0.44 degrees with 99.7\% confidence, adequate for platform leveling but
insufficient for precision inclinometry (which requires 0.01 degree
resolution).}

\begin{center}\rule{0.5\linewidth}{0.5pt}\end{center}

\section{Problem 11.2.6}\label{problem-11.2.6}

\textbf{Given:} A Hall effect current sensor (open-loop type) has a
turns ratio of N = 1,000 (equivalent turns), a core with a cross-section
of 60 mm\^{}2 and an air gap of 1.0 mm. The Hall element sensitivity is
1.4 mV/mT with a 5 V supply. A primary current of 200 A flows through a
single-turn conductor.

\textbf{Find:} The magnetic flux density in the air gap, the Hall sensor
output voltage, and the overall current-to-voltage sensitivity.

\textbf{Solution:} Magnetic field in the air gap from the primary
current: Using Ampere's law with the air gap dominating reluctance: B =
mu0 x N\textsubscript{primary} x I / l\textsubscript{gap} = 4 pi x
10\^{}-7 x 1 x 200 / 1.0 x 10\^{}-3

B = 2.513 x 10\^{}-4 x 200 / 10\^{}-3 = \textbf{0.2513 T = 251.3 mT}.

Wait -- this assumes no core. With a magnetic core and air gap:
H\textsubscript{gap} x l\textsubscript{gap} = N x I (neglecting core
reluctance since mu\_r \textgreater\textgreater{} 1).
H\textsubscript{gap} = N x I / l\textsubscript{gap} = 1 x 200 / 0.001 =
200,000 A/m. B\textsubscript{gap} = mu0 x H\textsubscript{gap} = 4 pi x
10\^{}-7 x 200,000 = \textbf{0.2513 T = 251.3 mT}.

Hall sensor output: V\textsubscript{H} = sensitivity x B = 1.4 mV/mT x
251.3 mT = \textbf{351.8 mV}.

Current-to-voltage sensitivity: V\textsubscript{H} / I = 351.8 / 200 =
\textbf{1.759 mV/A}.

\textbf{At 200 A, the Hall sensor produces 351.8 mV. The core would
saturate well before this point (typical saturation at
\textasciitilde1.5 T), so in practice the core uses multiple turns or
the turns ratio is adjusted so that the air gap field stays within the
linear range (\textless{} 100 mT).}

\begin{center}\rule{0.5\linewidth}{0.5pt}\end{center}

\section{Problem 11.2.7}\label{problem-11.2.7}

\textbf{Given:} A Rogowski coil current sensor has a mutual inductance
of M = 0.5 uH and is connected to an integrator circuit with an
integrator time constant RC = 100 us. The coil monitors a power
electronic switching waveform where the current rises linearly from 0 to
500 A in 10 us, then remains constant at 500 A for 40 us.

\textbf{Find:} The Rogowski coil output voltage during the current ramp,
the integrator output voltage at the end of the ramp (representing the
actual current), and the integrator output during the constant-current
phase.

\textbf{Solution:} During the linear current ramp: di/dt = 500 / (10 x
10\^{}-6) = 5 x 10\^{}7 A/s.

Rogowski coil output (proportional to di/dt): V\textsubscript{coil} = M
x di/dt = 0.5 x 10\^{}-6 x 5 x 10\^{}7 = \textbf{25 V}.

Integrator output (represents the actual current waveform):
V\textsubscript{int}(t) = (1/RC) x integral(V\textsubscript{coil} dt) =
(M/RC) x I(t). At end of ramp (I = 500 A): V\textsubscript{int} = (0.5 x
10\^{}-6 / 100 x 10\^{}-6) x 500 = 0.005 x 500 = \textbf{2.5 V}.

During the constant-current phase (di/dt = 0): V\textsubscript{coil} = M
x 0 = \textbf{0 V}. The integrator output remains at \textbf{2.5 V}
(representing 500 A), since the integral of zero is constant.

\textbf{Sensitivity: 2.5 V / 500 A = 5 mV/A. The Rogowski coil naturally
differentiates the current; the integrator restores the original
waveform.}

\begin{center}\rule{0.5\linewidth}{0.5pt}\end{center}

\section{Problem 11.2.8}\label{problem-11.2.8}

\textbf{Given:} A fiber Bragg grating (FBG) strain sensor has a Bragg
wavelength of lambda\textsubscript{B} = 1550.000 nm at 25 degrees C and
zero strain. The strain sensitivity is 1.2 pm/microstrain and the
temperature sensitivity is 10 pm/degrees C. The FBG is bonded to a steel
bridge girder. The interrogator measures a wavelength shift to 1550.385
nm. The temperature at the sensor location is 35 degrees C.

\textbf{Find:} The total wavelength shift, the temperature-induced
shift, the strain-induced shift, and the actual mechanical strain.

\textbf{Solution:} Total wavelength shift: delta-lambda = 1550.385 -
1550.000 = \textbf{0.385 nm = 385 pm}.

Temperature-induced shift: delta-lambda\textsubscript{T} = 10 pm/degrees
C x (35 - 25) = 10 x 10 = \textbf{100 pm}.

Strain-induced shift: delta-lambda\textsubscript{epsilon} =
delta-lambda\textsubscript{total} - delta-lambda\textsubscript{T} = 385
- 100 = \textbf{285 pm}.

Mechanical strain: epsilon = delta-lambda\textsubscript{epsilon} /
strain sensitivity = 285 / 1.2 = \textbf{237.5 microstrain}.

Stress in steel (E = 200 GPa): sigma = E x epsilon = 200 x 10\^{}9 x
237.5 x 10\^{}-6 = \textbf{47.5 MPa}.

\textbf{The FBG sensor reads 237.5 microstrain (47.5 MPa stress) after
compensating for the 10 degrees C temperature change. Without
temperature compensation, the apparent strain would be 385/1.2 = 320.8
microstrain -- a 35\% overestimate.}

\begin{center}\rule{0.5\linewidth}{0.5pt}\end{center}

\section{Problem 11.2.9}\label{problem-11.2.9}

\textbf{Given:} An electromagnetic flow meter is installed on a 200 mm
(8-inch) diameter pipe carrying process water. The magnetic field
strength is B = 0.04 T, and the measured electrode voltage is 1.88 mV.
The pipe operates 24 hours/day.

\textbf{Find:} The average flow velocity, the volumetric flow rate in
m\^{}3/h, and the daily volume throughput.

\textbf{Solution:} Average velocity: v = V / (B x D) = 1.88 x 10\^{}-3 /
(0.04 x 0.200) = 1.88 x 10\^{}-3 / 8.0 x 10\^{}-3 = \textbf{0.235 m/s}.

Pipe cross-sectional area: A = pi/4 x D\^{}2 = pi/4 x (0.200)\^{}2 =
\textbf{0.03142 m\^{}2}.

Volumetric flow rate: Q = v x A = 0.235 x 0.03142 = 7.384 x 10\^{}-3
m\^{}3/s. In m\^{}3/h: Q = 7.384 x 10\^{}-3 x 3,600 = \textbf{26.6
m\^{}3/h}.

Daily volume: V\textsubscript{day} = 26.6 x 24 = \textbf{638.4
m\^{}3/day}.

\textbf{The flow meter measures 26.6 m\^{}3/h (117 GPM). At this
moderate velocity of 0.235 m/s, the flow is well within the mag meter's
optimal range of 0.3-10 m/s, though slightly below the minimum
recommended velocity for best accuracy.}

\begin{center}\rule{0.5\linewidth}{0.5pt}\end{center}

\section{Problem 11.2.10}\label{problem-11.2.10}

\textbf{Given:} A Coriolis mass flow meter on a chemical feed line reads
a mass flow rate of 2,450 kg/h, a fluid density of 1,185 kg/m\^{}3, and
a fluid temperature of 42 degrees C. The meter accuracy is +/-0.1\% of
reading for mass flow and +/-0.5 kg/m\^{}3 for density.

\textbf{Find:} The volumetric flow rate, the mass flow uncertainty, and
the density-derived verification (compare measured density to the known
value of 1,182 kg/m\^{}3 for the chemical at 42 degrees C).

\textbf{Solution:} Volumetric flow rate: Q = mass flow / density = 2,450
/ 1,185 = \textbf{2.068 m\^{}3/h}.

In liters/min: Q = 2,068 / 60 = \textbf{34.5 L/min}.

Mass flow uncertainty: +/-0.1\% x 2,450 = \textbf{+/-2.45 kg/h}.

Density measurement: 1,185 +/- 0.5 kg/m\^{}3. Known density: 1,182
kg/m\^{}3. Deviation: 1,185 - 1,182 = 3 kg/m\^{}3, which exceeds the
+/-0.5 kg/m\^{}3 accuracy spec.

\textbf{This 3 kg/m\^{}3 density discrepancy suggests either the process
fluid concentration has changed, the temperature measurement is slightly
off, or the Coriolis meter needs recalibration. The Coriolis meter's
ability to simultaneously measure mass flow and density provides a
built-in diagnostic capability.}

\chapter{Chapter 11 -- Section 11.3: Signal
Conditioning}\label{chapter-11-section-11.3-signal-conditioning}

Practice problems covering amplification, filtering, isolation,
linearization, and ADC selection.

\begin{center}\rule{0.5\linewidth}{0.5pt}\end{center}

\section{Problem 11.3.1}\label{problem-11.3.1}

\textbf{Given:} An instrumentation amplifier with gain G = 200 and CMRR
= 110 dB amplifies a 500 uV signal from a strain gauge bridge. The
bridge wires run alongside power cables, picking up a 60 Hz common-mode
voltage of V\textsubscript{cm} = 3.0 V. The amplifier has an input
offset voltage of 15 uV and offset drift of 0.5 uV/degrees C over a 30
degrees C temperature range.

\textbf{Find:} The desired output signal, the common-mode error at the
output, the offset error at the output, and the total signal-to-error
ratio.

\textbf{Solution:} Desired output: V\textsubscript{out(signal)} = G x
V\textsubscript{diff} = 200 x 500 uV = \textbf{100 mV}.

CMRR = 110 dB = 10\^{}(110/20) = 316,228.

Common-mode error at input: V\textsubscript{cm,error} =
V\textsubscript{cm} / CMRR = 3.0 / 316,228 = 9.49 uV. At output:
V\textsubscript{cm,error,out} = 200 x 9.49 = \textbf{1.90 mV}.

Offset error at input: V\textsubscript{os} = 15 + 0.5 x 30 = 30 uV. At
output: V\textsubscript{os,out} = 200 x 30 = \textbf{6.0 mV}.

Total error (RSS): V\textsubscript{error} = sqrt(1.90\^{}2 + 6.0\^{}2) =
sqrt(3.61 + 36.0) = sqrt(39.61) = \textbf{6.29 mV}.

Signal-to-error ratio: 100 / 6.29 = 15.9, or 20 log10(15.9) =
\textbf{24.0 dB}.

\textbf{The offset drift dominates the error budget. Chopper-stabilized
instrumentation amplifiers (with \textless{} 1 uV offset and \textless{}
10 nV/degrees C drift) would reduce the offset error by 30x, improving
the signal-to-error ratio to approximately 44 dB.}

\begin{center}\rule{0.5\linewidth}{0.5pt}\end{center}

\section{Problem 11.3.2}\label{problem-11.3.2}

\textbf{Given:} A vibration measurement system has a signal bandwidth of
2 kHz. The ADC samples at 20 kS/s. A 4th-order Butterworth lowpass
anti-aliasing filter is used with a cutoff frequency of 4 kHz. A strong
noise source exists at 15 kHz.

\textbf{Find:} The Nyquist frequency, the attenuation of the
anti-aliasing filter at the Nyquist frequency, the attenuation at 15
kHz, and whether aliased noise at 15 kHz will corrupt the measurement.

\textbf{Solution:} Nyquist frequency: f\textsubscript{N} =
f\textsubscript{s} / 2 = 20,000 / 2 = \textbf{10 kHz}.

4th-order Butterworth magnitude at frequency f: \textbar H(f)\textbar{}
= 1 / sqrt(1 + (f/f\textsubscript{c})\^{}(2n)) where n = 4.

At Nyquist (10 kHz): \textbar H\textbar{} = 1 / sqrt(1 + (10/4)\^{}8) =
1 / sqrt(1 + 2.5\^{}8) = 1 / sqrt(1 + 1,526) = 1 / sqrt(1,527) = 1 /
39.08 = 0.0256. Attenuation = 20 log10(0.0256) = \textbf{-31.8 dB}.

At 15 kHz: \textbar H\textbar{} = 1 / sqrt(1 + (15/4)\^{}8) = 1 / sqrt(1
+ 3.75⁸) = 1 / sqrt(1 + 39,107) = 1 / sqrt(39,108) = 1 / 197.8 =
0.00506. Attenuation = 20 log10(0.00506) = \textbf{-45.9 dB}.

The 15 kHz noise aliases to: f\textsubscript{alias} = f\textsubscript{s}
- 15,000 = 20,000 - 15,000 = \textbf{5 kHz}, which falls within the 0-10
kHz measurement band.

At -45.9 dB, the noise is reduced by a factor of 197.8 and is
well-suppressed in the stopband.

\textbf{For applications requiring \textgreater{} 60 dB rejection,
either increase the filter order to 8th-order (giving -79.6 dB at 15
kHz) or increase the sample rate to 50 kS/s (pushing Nyquist to 25 kHz
and placing the 15 kHz signal well within the passband for direct
measurement rather than aliasing).}

\begin{center}\rule{0.5\linewidth}{0.5pt}\end{center}

\section{Problem 11.3.3}\label{problem-11.3.3}

\textbf{Given:} A thermocouple measurement system monitors a 480 V,
3-phase motor winding temperature. The thermocouples are embedded in the
windings and produce 0-20 mV signals. An isolated amplifier with 3,750
V\textsubscript{rms} isolation, gain of 100, gain error of 0.02\%, input
offset of 5 uV, offset drift of 0.3 uV/degrees C, and 50 kHz bandwidth
is used. The ambient temperature varies by 40 degrees C from
calibration.

\textbf{Find:} The output voltage range, the gain error contribution,
the offset drift error, and the total measurement error as a percentage
of a 10 mV input signal.

\textbf{Solution:} Output voltage range: V\textsubscript{out} = gain x
V\textsubscript{in} = 100 x (0 to 20 mV) = \textbf{0 to 2.0 V}.

At V\textsubscript{in} = 10 mV, V\textsubscript{out} = 100 x 10 = 1,000
mV = 1.000 V.

Gain error: 0.02\% x 1.000 V = \textbf{0.20 mV}.

Offset drift: 0.3 uV/degrees C x 40 degrees C = 12 uV at input. At
output: 100 x 12 = \textbf{1.20 mV}.

Input offset (referred to output): 100 x 5 uV = \textbf{0.50 mV}.

Total error (RSS): sqrt(0.20\^{}2 + 1.20\^{}2 + 0.50\^{}2) = sqrt(0.04 +
1.44 + 0.25) = sqrt(1.73) = \textbf{1.32 mV}.

As percentage of output: 1.32 / 1,000 x 100 = \textbf{0.132\%}.

For a Type K thermocouple at \textasciitilde250 degrees C (where 10 mV
is produced): 0.132\% of the reading corresponds to approximately 0.33
degrees C temperature error.

\textbf{The isolation amplifier maintains 0.13\% accuracy while
providing 3,750 V galvanic isolation between the 480 V motor windings
and the ground-referenced DAQ system.}

\begin{center}\rule{0.5\linewidth}{0.5pt}\end{center}

\section{Problem 11.3.4}\label{problem-11.3.4}

\textbf{Given:} An NTC thermistor has Steinhart-Hart coefficients A =
1.125 x 10\^{}-3, B = 2.347 x 10\^{}-4, C = 0.855 x 10\^{}-7. The
thermistor is read by a 12-bit ADC via a voltage divider with a series
resistor R\textsubscript{ref} = 10,000 Ohm and V\textsubscript{supply} =
3.3 V.

\textbf{Find:} The thermistor resistance and temperature when the ADC
reads a digital code of 2,048 (midscale), and the temperature resolution
limited by the ADC.

\textbf{Solution:} ADC voltage: V\textsubscript{ADC} = (code / 2\^{}12)
x V\textsubscript{ref} = (2,048 / 4,096) x 3.3 = 0.500 x 3.3 =
\textbf{1.650 V}.

For a voltage divider (thermistor on bottom): V\textsubscript{ADC} =
V\textsubscript{supply} x R\textsubscript{th} / (R\textsubscript{ref} +
R\textsubscript{th}). 1.650 = 3.3 x R\textsubscript{th} / (10,000 +
R\textsubscript{th}). 1.650 x (10,000 + R\textsubscript{th}) = 3.3 x
R\textsubscript{th}. 16,500 + 1.650 R\textsubscript{th} = 3.3
R\textsubscript{th}. 16,500 = 1.650 R\textsubscript{th}.
R\textsubscript{th} = \textbf{10,000 Ohm}.

At R\textsubscript{th} = R\textsubscript{ref} = 10,000 Ohm (midscale):
this is by design the operating midpoint.

Steinhart-Hart: 1/T = A + B ln(R) + C (ln(R))\^{}3 ln(10,000) = 9.2103.
1/T = 1.125 x 10\^{}-3 + 2.347 x 10\^{}-4 x 9.2103 + 0.855 x 10\^{}-7 x
(9.2103)\^{}3 = 1.125 x 10\^{}-3 + 2.162 x 10\^{}-3 + 0.855 x 10\^{}-7 x
781.3 = 1.125 x 10\^{}-3 + 2.162 x 10\^{}-3 + 0.06681 x 10\^{}-3 = 3.354
x 10\^{}-3. T = 1 / 3.354 x 10\^{}-3 = 298.17 K = \textbf{25.0 degrees
C}.

Temperature resolution: one ADC code change at midscale: delta-V = 3.3 /
4,096 = 0.806 mV. Near midscale, dR/dV is found by differentiating the
divider equation: dR/dV = (R\textsubscript{ref} +
R\textsubscript{th})\^{}2 / (V\textsubscript{supply} x
R\textsubscript{ref}) = (20,000)\^{}2 / (3.3 x 10,000) = 12,121 Ohm/V.
delta-R = 12,121 x 0.000806 = 9.77 Ohm.

NTC sensitivity at 25 degrees C (beta \textasciitilde{} 3,950 K): dR/dT
\textasciitilde{} -R x beta/T\^{}2 = -10,000 x 3,950 / 298\^{}2 = -444.4
Ohm/degrees C. delta-T = delta-R / \textbar dR/dT\textbar{} = 9.77 /
444.4 = \textbf{0.022 degrees C per ADC count}.

\textbf{The thermistor + voltage divider combination with a 12-bit ADC
provides 0.022 degrees C resolution at 25 degrees C -- excellent for
environmental monitoring.}

\begin{center}\rule{0.5\linewidth}{0.5pt}\end{center}

\section{Problem 11.3.5}\label{problem-11.3.5}

\textbf{Given:} A load cell system requires 0.01\% accuracy over a +/-10
mV full-scale range from a 350 Ohm strain gauge bridge. The measurement
bandwidth is 50 Hz. The system must reject 60 Hz power line interference
by at least 60 dB. Power supply is 5 V with a 1 mA current budget.

\textbf{Find:} The minimum ADC resolution (bits), the recommended ADC
architecture, and whether the 60 Hz rejection requirement is met.

\textbf{Solution:} Required resolution: 0.01\% of full scale = 0.0001 x
20 mV = \textbf{2.0 uV}.

Number of bits: 2\^{}N = 20 mV / 2.0 uV = 10,000. N = log2(10,000) =
13.3 bits minimum. With noise margin (typically add 2-3 bits):
\textbf{16 bits minimum}.

Nyquist rate: 2 x 50 Hz = 100 S/s minimum.

ADC architecture analysis: - SAR: 16-bit at 100 S/s is feasible but
requires external low-noise PGA and separate anti-aliasing filter. No
inherent 60 Hz rejection. - Sigma-Delta: 24-bit at 50 Hz output data
rate provides \textasciitilde19-20 noise-free bits, integrated PGA, and
built-in sinc digital filter.

For a sigma-delta with sinc3 filter at 50 Hz output rate: The sinc
filter has notches at multiples of the output data rate. Setting output
rate = 60 Hz (or 50 Hz with notch at 60 Hz): many 24-bit sigma-delta
ADCs provide a specific 50/60 Hz rejection mode.

At 50 Hz output rate with a sinc3 filter, the rejection at 60 Hz: sinc3
rejection = \textbar sin(pi x 60/f\textsubscript{mod}) / (sin(pi x 60/(N
x f\textsubscript{mod})))\textbar\^{}3, but practically these ADCs
specify \textgreater{} 80 dB rejection of 50/60 Hz.

\textbf{The 60 dB requirement is met with margin.}

\textbf{Selection: 24-bit sigma-delta ADC with integrated PGA (such as
ADS1234 or AD7190). Provides 20+ noise-free bits, built-in 60 Hz
rejection \textgreater{} 80 dB, integrated excitation current, and
\textless{} 1 mA power consumption at 3.3-5 V.}

\begin{center}\rule{0.5\linewidth}{0.5pt}\end{center}

\section{Problem 11.3.6}\label{problem-11.3.6}

\textbf{Given:} A photodiode current sensor produces a photocurrent of
0-100 nA proportional to light intensity. A transimpedance amplifier
(TIA) with feedback resistance R\textsubscript{f} = 10 MOhm and feedback
capacitance C\textsubscript{f} = 1.6 pF converts the current to voltage.
The op-amp has a gain-bandwidth product of 1 MHz and input bias current
of 1 pA.

\textbf{Find:} The output voltage range, the -3 dB bandwidth, and the
measurement error caused by input bias current.

\textbf{Solution:} Output voltage: V\textsubscript{out} =
-I\textsubscript{photo} x R\textsubscript{f}. At I = 100 nA:
V\textsubscript{out} = -100 x 10\^{}-9 x 10 x 10\^{}6 = \textbf{-1.0 V}.
Range: \textbf{0 to -1.0 V} (inverting configuration).

Transimpedance gain: Z\textsubscript{f} = R\textsubscript{f} / (1 + j 2
pi f R\textsubscript{f} C\textsubscript{f}).

-3 dB bandwidth: f\textsubscript{3dB} = 1 / (2 pi R\textsubscript{f}
C\textsubscript{f}) = 1 / (2 pi x 10 x 10\^{}6 x 1.6 x 10\^{}-12) = 1 /
(1.005 x 10\^{}-4) = \textbf{9.95 kHz}.

Check stability: the noise gain intersection with the op-amp open-loop
gain determines stability. f\textsubscript{noise} \textasciitilde{}
sqrt(GBW / (2 pi R\textsubscript{f} C\textsubscript{f})) -- with proper
C\textsubscript{f} selection, this should be stable.

Bias current error: V\textsubscript{error} = I\textsubscript{bias} x
R\textsubscript{f} = 1 x 10\^{}-12 x 10 x 10\^{}6 = \textbf{10 uV}.

At the minimum detectable current (noise limited by bias current):
I\textsubscript{min} = I\textsubscript{bias} = 1 pA, producing
V\textsubscript{out} = 10 uV.

Error as percentage of full scale: 10 uV / 1.0 V x 100 =
\textbf{0.001\%}.

\textbf{The TIA converts 0-100 nA to 0-1 V with 9.95 kHz bandwidth and
0.001\% error from bias current -- excellent performance for optical
sensing applications.}

\chapter{Chapter 11 -- Section 11.4: Measurement
Instruments}\label{chapter-11-section-11.4-measurement-instruments}

Practice problems covering digital multimeters, oscilloscopes, spectrum
analyzers, function generators, power analyzers, LCR meters, frequency
counters, and vector network analyzers.

\begin{center}\rule{0.5\linewidth}{0.5pt}\end{center}

\section{Problem 11.4.1}\label{problem-11.4.1}

\textbf{Given:} A 5-1/2 digit DMM on the 1 V range (input impedance 10
GOhm) measures the output of a high-impedance pH sensor with a source
impedance of 200 MOhm. The true sensor voltage is 0.41500 V. The DMM has
a specified accuracy of +/-(0.005\% of reading + 0.003\% of range).

\textbf{Find:} The loading error, the DMM accuracy error, and the total
measurement uncertainty.

\textbf{Solution:} Loading error: V\textsubscript{measured} =
V\textsubscript{true} x R\textsubscript{DMM} / (R\textsubscript{source}
+ R\textsubscript{DMM}) = 0.41500 x 10 x 10\^{}9 / (200 x 10\^{}6 + 10 x
10\^{}9) = 0.41500 x 10\^{}10 / 10.2 x 10\^{}9 = 0.41500 x 0.98039 =
\textbf{0.40686 V}.

Loading error: 0.41500 - 0.40686 = 8.14 mV, or 1.96\% of reading.
\textbf{This is severe.}

DMM accuracy error (on the reading): Reading error = 0.005\% x 0.40686 +
0.003\% x 1.0 = 0.0000500 x 0.40686 + 0.0000300 x 1.0 = 0.02034 mV +
0.03000 mV = \textbf{0.05034 mV}.

DMM resolution: 1 V / 199,999 counts = \textbf{5.0 uV per count}.

Total uncertainty: loading error (8.14 mV) completely dominates the DMM
accuracy error (0.05 mV).

\textbf{The DMM's 10 GOhm input impedance is inadequate for a 200 MOhm
source. An electrometer with \textgreater{} 10 TOhm input impedance
would reduce the loading error to \textless{} 0.002\%. Alternatively, a
unity-gain buffer amplifier (FET-input, \textgreater{} 10\^{}12 Ohm
input) placed between the sensor and DMM eliminates the loading error.}

\begin{center}\rule{0.5\linewidth}{0.5pt}\end{center}

\section{Problem 11.4.2}\label{problem-11.4.2}

\textbf{Given:} A digital oscilloscope has 350 MHz bandwidth, 2.5 GS/s
sample rate, 8-bit vertical resolution, and 20 Mpoints memory depth. A
100 MHz clock signal with 1.2 ns rise time and 3.3 V amplitude is being
measured. The probe has 500 MHz bandwidth and 10:1 attenuation.

\textbf{Find:} The system bandwidth, the measured rise time, the percent
error in rise time measurement, and the maximum capture duration at full
sample rate.

\textbf{Solution:} System bandwidth: BW\textsubscript{system} = 1 /
sqrt(1/BW\textsubscript{scope}\^{}2 + 1/BW\textsubscript{probe}\^{}2) =
1 / sqrt(1/350\^{}2 + 1/500\^{}2) x 10\^{}6 = 1 / sqrt(8.163 x 10\^{}-18
+ 4.0 x 10\^{}-18) = 1 / sqrt(12.163 x 10\^{}-18) = 1 / (3.488 x
10\^{}-9) = \textbf{286.7 MHz}.

System rise time: t\textsubscript{r,system} = 0.35 /
BW\textsubscript{system} = 0.35 / 286.7 x 10\^{}6 = \textbf{1.221 ns}.

Measured rise time: t\textsubscript{r,meas} =
sqrt(t\textsubscript{r,signal}\^{}2 + t\textsubscript{r,system}\^{}2) =
sqrt(1.2\^{}2 + 1.221\^{}2) = sqrt(1.44 + 1.491) = sqrt(2.931) =
\textbf{1.712 ns}.

Rise time error: (1.712 - 1.2) / 1.2 x 100 = \textbf{42.7\%}.

\textbf{This is unacceptable.} A 1 GHz oscilloscope with a 1.5 GHz probe
would give: BW\textsubscript{system} = 832 MHz,
t\textsubscript{r,system} = 0.42 ns, t\textsubscript{r,meas} = sqrt(1.44
+ 0.177) = 1.27 ns (5.8\% error).

Maximum capture at full sample rate: T = memory / sample rate = 20 x
10\^{}6 / 2.5 x 10\^{}9 = \textbf{8.0 ms}.

At 100 MHz signal, this captures 800,000 cycles -- more than adequate
for signal analysis.

\textbf{For accurate rise time measurements, the rule of thumb is
oscilloscope bandwidth \textgreater= 5x the signal bandwidth (1/rise
time). Here, 5 / 1.2 ns = 4.2 GHz bandwidth would give \textless{} 1\%
error.}

\begin{center}\rule{0.5\linewidth}{0.5pt}\end{center}

\section{Problem 11.4.3}\label{problem-11.4.3}

\textbf{Given:} A spectrum analyzer with a noise floor of -120 dBm (in
100 Hz RBW) is used to measure the third-order intermodulation (IM3)
products of an RF amplifier. Two test tones at 900 MHz and 901 MHz, each
at -10 dBm, are applied to the amplifier with 20 dB gain. The IM3
products appear at 899 MHz and 902 MHz at -25 dBm. The measurement uses
1 kHz RBW.

\textbf{Find:} The amplifier output power per tone, the IM3 level
relative to the carrier, the third-order intercept point (OIP3), and the
noise floor at the measurement RBW.

\textbf{Solution:} Amplifier output per tone: P\textsubscript{out} =
P\textsubscript{in} + Gain = -10 + 20 = \textbf{+10 dBm}.

IM3 relative to carrier: IM3 = P\textsubscript{IM3} -
P\textsubscript{out} = -25 - 10 = \textbf{-35 dBc}.

Third-order intercept point (output-referred): OIP3 =
P\textsubscript{out} + \textbar IM3\textbar{} / 2 = 10 + 35/2 = 10 +
17.5 = \textbf{+27.5 dBm}.

Input-referred: IIP3 = OIP3 - Gain = 27.5 - 20 = \textbf{+7.5 dBm}.

Noise floor at 1 kHz RBW: NF = -120 + 10 log10(1,000/100) = -120 + 10 =
\textbf{-110 dBm}.

Signal-to-noise margin for IM3 measurement: -25 - (-110) = \textbf{85
dB} above noise floor.

\textbf{The IM3 measurement is valid with excellent noise margin. The
OIP3 of +27.5 dBm is a good figure of merit for the amplifier's
linearity performance.}

\begin{center}\rule{0.5\linewidth}{0.5pt}\end{center}

\section{Problem 11.4.4}\label{problem-11.4.4}

\textbf{Given:} A function generator produces a 500 kHz square wave at 5
V\textsubscript{pp} into a 50 Ohm load (matched). The specified rise
time is 8 ns and the duty cycle is set to exactly 50\%. The output is
connected to a circuit with 1 kOhm input impedance.

\textbf{Find:} The voltage delivered to the 50 Ohm load (matched case),
the voltage delivered to the 1 kOhm load (mismatched), the approximate
bandwidth required to preserve the square wave edges, and the number of
significant harmonics.

\textbf{Solution:} Into matched 50 Ohm load: the generator is specified
at 5 V\textsubscript{pp} into 50 Ohm, so V\textsubscript{load} =
\textbf{5 V\textsubscript{pp}}. (The internal EMF is actually 10
V\textsubscript{pp}, divided equally between source and load impedance.)

Into 1 kOhm load (mismatched): V\textsubscript{load} =
V\textsubscript{emf} x R\textsubscript{load} / (R\textsubscript{source}
+ R\textsubscript{load}) = 10 x 1,000 / (50 + 1,000) = 10 x 0.952 =
\textbf{9.52 V\textsubscript{pp}}.

The amplitude nearly doubles because the high-impedance load draws
negligible current.

Bandwidth for square wave: BW \textgreater= 0.35 / t\textsubscript{rise}
= 0.35 / 8 x 10\^{}-9 = \textbf{43.75 MHz}.

Fundamental frequency: f1 = 500 kHz. Number of significant harmonics: N
= BW / f1 = 43.75 x 10\^{}6 / 500 x 10\^{}3 = \textbf{87.5}.

Approximately \textbf{88 harmonics} (odd only: 1st, 3rd, 5th, \ldots{}
up to the 175th) are needed to reconstruct the square wave with the
specified rise time.

\textbf{When connecting the generator to a high-impedance load, the 50
Ohm source impedance causes a reflection coefficient of
(1000-50)/(1000+50) = 0.905, producing signal reflections on the cable
that can distort fast edges. For the 8 ns rise time at 500 kHz, cable
effects are generally negligible for cables shorter than
\textasciitilde0.8 m (considering 8 ns round-trip at 5 ns/m propagation
delay).}

\begin{center}\rule{0.5\linewidth}{0.5pt}\end{center}

\section{Problem 11.4.5}\label{problem-11.4.5}

\textbf{Given:} A power analyzer measures a three-phase, 4-wire motor
drive system. The per-phase readings are: - Phase A: V = 231.2 V, I =
15.4 A, P = 2,845 W - Phase B: V = 229.8 V, I = 15.1 A, P = 2,790 W -
Phase C: V = 232.5 V, I = 15.8 A, P = 2,910 W

The current THD on all phases is approximately 28\%.

\textbf{Find:} Total three-phase power, total apparent power, system
power factor, displacement power factor, and voltage unbalance
percentage (NEMA definition).

\textbf{Solution:} Total real power: P\textsubscript{total} = 2,845 +
2,790 + 2,910 = \textbf{8,545 W}.

Per-phase apparent power: S\textsubscript{A} = 231.2 x 15.4 = 3,560 VA.
S\textsubscript{B} = 229.8 x 15.1 = 3,470 VA. S\textsubscript{C} = 232.5
x 15.8 = 3,674 VA. S\textsubscript{total} = 3,560 + 3,470 + 3,674 =
\textbf{10,704 VA}.

Total power factor: PF = P / S = 8,545 / 10,704 = \textbf{0.798}.

Distortion power factor: DstPF = 1 / sqrt(1 + THD\^{}2) = 1 / sqrt(1 +
0.28\^{}2) = 1 / sqrt(1.0784) = 1 / 1.0385 = \textbf{0.963}.

Displacement power factor: DPF = PF / DstPF = 0.798 / 0.963 =
\textbf{0.829} (corresponding to phi = 34.0 degrees).

Voltage unbalance (NEMA): maximum deviation from average / average
voltage. Average voltage: V\textsubscript{avg} = (231.2 + 229.8 + 232.5)
/ 3 = 231.17 V. Maximum deviation: max(\textbar231.2-231.17\textbar,
\textbar229.8-231.17\textbar, \textbar232.5-231.17\textbar) = max(0.03,
1.37, 1.33) = 1.37 V. Unbalance = 1.37 / 231.17 x 100 = \textbf{0.59\%}.

\textbf{The 0.59\% voltage unbalance is within the NEMA MG1 limit of
1\%. The 28\% current THD contributes significantly to the poor power
factor (0.798 vs.~0.829 displacement), suggesting harmonic filters would
improve the system.}

\begin{center}\rule{0.5\linewidth}{0.5pt}\end{center}

\section{Problem 11.4.6}\label{problem-11.4.6}

\textbf{Given:} A 10 uH inductor is measured on an LCR meter at 1 MHz.
The readings are \textbar Z\textbar{} = 72.5 Ohm and phase angle theta =
+78.3 degrees.

\textbf{Find:} The series resistance (ESR), the actual inductance at 1
MHz, the quality factor Q, and the self-resonant frequency if the
parasitic capacitance is estimated at 5 pF.

\textbf{Solution:} Series resistance: R\textsubscript{s} =
\textbar Z\textbar{} x cos(theta) = 72.5 x cos(78.3 degrees) = 72.5 x
0.2028 = \textbf{14.70 Ohm}.

Series reactance: X\textsubscript{s} = \textbar Z\textbar{} x sin(theta)
= 72.5 x sin(78.3 degrees) = 72.5 x 0.9793 = \textbf{71.00 Ohm}.

Inductance at 1 MHz: L = X\textsubscript{s} / (2 pi f) = 71.00 / (2 pi x
10\^{}6) = 71.00 / 6.283 x 10\^{}6 = \textbf{11.30 uH}.

The measured inductance (11.30 uH) is higher than the nominal 10 uH
because at 1 MHz, the parasitic capacitance partially resonates with the
inductance, increasing the apparent impedance.

Quality factor: Q = X\textsubscript{s} / R\textsubscript{s} = 71.00 /
14.70 = \textbf{4.83}.

Self-resonant frequency (SRF): f\textsubscript{SRF} = 1 / (2 pi sqrt(L x
C\textsubscript{p})) = 1 / (2 pi sqrt(10 x 10\^{}-6 x 5 x 10\^{}-12)) =
1 / (2 pi sqrt(5 x 10\^{}-17)) = 1 / (2 pi x 7.071 x 10\^{}-9) = 1 /
4.443 x 10\^{}-8 = \textbf{22.5 MHz}.

\textbf{Above the 22.5 MHz SRF, the inductor behaves as a capacitor. At
1 MHz (well below SRF), the inductance increase from 10 to 11.3 uH is
the early effect of the parasitic capacitance beginning to resonate with
the inductance.}

\begin{center}\rule{0.5\linewidth}{0.5pt}\end{center}

\section{Problem 11.4.7}\label{problem-11.4.7}

\textbf{Given:} A universal frequency counter with a 200 MHz timebase
measures two signals: (a) a 50 kHz clock from a crystal oscillator using
a 100 ms gate time, and (b) a 1 PPS (1.000000 Hz) GPS timing pulse using
reciprocal counting. The timebase stability is 1 x 10\^{}-8 (OCXO).

\textbf{Find:} (a) The direct count result and resolution for the 50 kHz
signal. (b) The reciprocal count resolution for the 1 PPS signal. (c)
The absolute accuracy of each measurement.

\textbf{Solution:} \textbf{(a) Direct count of 50 kHz with 100 ms gate:}
Expected count = 50,000 x 0.1 = 5,000 counts. Resolution = +/-1 count /
gate time = +/-1 / 0.1 = +/-10 Hz. Relative resolution = 10 / 50,000 = 2
x 10\^{}-4 = \textbf{0.02\% (4 digits)}.

Frequency reading: \textbf{50,000 +/- 10 Hz}.

\textbf{(b) Reciprocal count of 1 PPS:} During 1 second (one period of
the 1 Hz signal), the 200 MHz timebase counts: N = 200 x 10\^{}6 x 1 =
200,000,000 ticks. Period resolution: delta-T = 1 / 200 x 10\^{}6 = 5
ns. Frequency resolution: delta-f / f = delta-T / T = 5 x 10\^{}-9 / 1 =
\textbf{5 x 10\^{}-9} (9 digits). Absolute frequency resolution:
1.000000000 x 5 x 10\^{}-9 = \textbf{5 nHz}.

\textbf{(c) Absolute accuracy (limited by timebase stability):} 50 kHz
measurement: 50,000 x 10\^{}-8 = +/-0.0005 Hz. But direct count
resolution (+/-10 Hz) is the limiting factor. \textbf{50 kHz accuracy:
+/-10 Hz} (count-limited).

1 PPS measurement: 1.0 x 10\^{}-8 = 10 nHz. Resolution (5 nHz) and
timebase stability (10 nHz) are comparable. \textbf{1 PPS accuracy:
+/-10 nHz} (timebase-limited).

\textbf{The reciprocal method provides 50,000x better resolution than
direct counting for the 50 kHz measurement. Using reciprocal counting on
the 50 kHz signal (measuring the period of 5,000 cycles in 100 ms with a
200 MHz clock): resolution = 5 x 10\^{}-9, or delta-f = 50,000 x 5 x
10\^{}-9 = 0.25 mHz -- dramatically better than the +/-10 Hz from direct
counting.}

\begin{center}\rule{0.5\linewidth}{0.5pt}\end{center}

\section{Problem 11.4.8}\label{problem-11.4.8}

\textbf{Given:} A VNA measures a 50 Ohm coaxial cable at 2 GHz. The
measurements are: S11 = -26 dB angle -90 degrees, S21 = -3.2 dB angle
-210 degrees, S12 = -3.2 dB angle -210 degrees, and S22 = -28 dB angle
-85 degrees.

\textbf{Find:} The cable insertion loss, input VSWR, the fraction of
power delivered to the output, and the electrical length.

\textbf{Solution:} Insertion loss = -S21 = \textbf{3.2 dB}.

Input reflection coefficient: \textbar gamma\textbar{} = 10\^{}(S11/20)
= 10\^{}(-26/20) = 10\^{}(-1.3) = \textbf{0.0501}.

Input VSWR = (1 + \textbar gamma\textbar) / (1 - \textbar gamma\textbar)
= (1 + 0.0501) / (1 - 0.0501) = 1.0501 / 0.9499 = \textbf{1.106:1}.

Power reflected: \textbar gamma\textbar\^{}2 = 0.0501\^{}2 = 0.00251 =
\textbf{0.251\%}.

Power transmitted: \textbar S21\textbar\^{}2 = 10\^{}(S21(dB)/10) =
10\^{}(-3.2/10) = 10\^{}(-0.32) = \textbf{0.4786 = 47.86\%}.

Power dissipated in cable: 1 - \textbar S11\textbar\^{}2 -
\textbar S21\textbar\^{}2 = 1 - 0.00251 - 0.4786 = \textbf{0.519 =
51.9\%}.

Electrical length from S21 phase: Phase shift = -210 degrees. At 2 GHz,
one wavelength = 360 degrees. Electrical length = 210 / 360 =
\textbf{0.583 wavelengths}.

In free space at 2 GHz: lambda = c/f = 3 x 10\^{}8 / 2 x 10\^{}9 = 0.15
m. With velocity factor of 0.66 (typical coax):
lambda\textsubscript{cable} = 0.15 x 0.66 = 0.099 m. Physical length
\textasciitilde{} 0.583 x 0.099 = \textbf{0.0577 m = 5.77 cm}.

\textbf{The 3.2 dB insertion loss for a \textasciitilde6 cm cable at 2
GHz indicates either a very lossy cable, a poor connector, or the cable
is much longer than estimated (the phase wrapping may have added 360
degrees, making the actual length 1.583 wavelengths or
\textasciitilde15.7 cm).}

\chapter{Chapter 11 -- Section 11.5: Data
Acquisition}\label{chapter-11-section-11.5-data-acquisition}

Practice problems covering sampling and quantization, DAQ systems, data
logging, and automated test systems.

\begin{center}\rule{0.5\linewidth}{0.5pt}\end{center}

\section{Problem 11.5.1}\label{problem-11.5.1}

\textbf{Given:} A 16-bit SAR ADC with a +/-5 V bipolar input range
samples a vibration sensor at f\textsubscript{s} = 100 kS/s. The signal
of interest has frequencies from 10 Hz to 5 kHz, with a strong 5 kHz
component at 2 V\textsubscript{pp} and broadband noise extending to 50
kHz.

\textbf{Find:} The ADC resolution (LSB size), the ideal SINAD, the ENOB,
the Nyquist frequency, and the attenuation needed from the anti-aliasing
filter at frequencies above 45 kHz to prevent more than 1 LSB of aliased
noise.

\textbf{Solution:} Full-scale range: 10 V (from -5 V to +5 V).
Resolution (1 LSB): 10 V / 2\^{}16 = 10 / 65,536 = \textbf{0.1526 mV =
152.6 uV}.

Ideal SINAD: SINAD = 6.02N + 1.76 = 6.02 x 16 + 1.76 = 96.32 + 1.76 =
\textbf{98.08 dB}.

ENOB = (SINAD - 1.76) / 6.02 = (98.08 - 1.76) / 6.02 = \textbf{16.0
bits} (ideal case).

Nyquist frequency: f\textsubscript{N} = f\textsubscript{s} / 2 = 100,000
/ 2 = \textbf{50 kHz}.

Anti-aliasing requirement: noise above 45 kHz aliases into 5 kHz (since
f\textsubscript{s} - 45,000 = 55,000 which folds to 50,000 - 5,000 =
45,000\ldots{} more precisely, a signal at 95 kHz aliases to 5 kHz).

To limit aliased noise to \textless{} 1 LSB = 152.6 uV: If broadband
noise spectral density is approximately V\textsubscript{noise}/sqrt(BW),
we need the filter to attenuate the noise in the 50-100 kHz band (which
aliases into 0-50 kHz).

For 1 LSB at the 5 V signal level, the required attenuation is: Assuming
worst-case noise of 10 mV\textsubscript{rms} in the 45-50 kHz band:
Attenuation needed = 20 log10(0.1526 / 10) = 20 log10(0.01526) =
\textbf{-36.3 dB}.

\textbf{A 4th-order Butterworth filter with f\textsubscript{c} = 10 kHz
provides -40 dB/decade x log10(45/10) = -40 x 0.653 = -26.1 dB at 45
kHz. This is insufficient; a 6th-order filter (-39.2 dB at 45 kHz) or an
8th-order (-52.3 dB) would be needed for 1-LSB aliasing performance.}

\begin{center}\rule{0.5\linewidth}{0.5pt}\end{center}

\section{Problem 11.5.2}\label{problem-11.5.2}

\textbf{Given:} An 8-channel simultaneous-sampling DAQ system uses
18-bit ADCs with dedicated sample-and-hold circuits on each channel. The
aggregate sample rate is 500 kS/s (62.5 kS/s per channel). Channels 1-4
measure voltage (0-10 V range), channels 5-6 measure current (4-20 mA
through 250 Ohm shunt = 1-5 V), and channels 7-8 measure thermocouple
signals (0-20 mV range with 100x PGA).

\textbf{Find:} The resolution on each range, the maximum signal
frequency per channel (assuming 10x oversampling), and the inter-channel
time skew.

\textbf{Solution:} \textbf{Channels 1-4 (0-10 V range):} Resolution: 10
V / 2\^{}18 = 10 / 262,144 = \textbf{38.15 uV per count}.

\textbf{Channels 5-6 (1-5 V range):} If the ADC uses the 0-10 V range:
resolution = 38.15 uV per count. If the ADC is configured for a 0-5 V
range: 5 / 262,144 = \textbf{19.07 uV per count}. In current units
(using 250 Ohm shunt): 19.07 uV / 250 = \textbf{0.0763 uA per count}.

\textbf{Channels 7-8 (0-20 mV with 100x PGA):} After PGA: 0-2 V signal
to ADC. ADC range 0-10 V, resolution = 38.15 uV per count. Referred to
input: 38.15 / 100 = \textbf{0.3815 uV per count}. In 18 bits over 20
mV: 20 mV / 262,144 = \textbf{0.0763 uV} -- significantly finer than the
0.38 uV from the ADC-limited case. The PGA noise and ADC noise will
limit the effective resolution.

Maximum signal frequency per channel (10x oversampling):
f\textsubscript{max} = f\textsubscript{s,ch} / 10 = 62,500 / 10 =
\textbf{6.25 kHz}.

Inter-channel time skew: With simultaneous sampling, all channels are
sampled at the same instant. \textbf{Time skew = 0 ns} (by definition of
simultaneous sampling).

For a multiplexed system at the same aggregate rate, the skew would be:
1 / 500,000 = 2 us between adjacent channels, or 14 us between channel 1
and channel 8.

\textbf{Simultaneous sampling eliminates the 14 us inter-channel skew
that would cause 5.1 degrees of phase error at 1 kHz in a multiplexed
system.}

\begin{center}\rule{0.5\linewidth}{0.5pt}\end{center}

\section{Problem 11.5.3}\label{problem-11.5.3}

\textbf{Given:} A 64-channel temperature monitoring system logs Pt100
RTD data continuously. Each channel is sampled at 1 S/s with 24-bit
resolution (3 bytes per sample). Each scan includes a 64-bit (8-byte)
timestamp. Data is stored in CSV format with overhead of approximately
20 bytes per reading (channel ID, comma separators, newline). The system
must log for 1 year.

\textbf{Find:} The raw binary data rate, the CSV data rate, the storage
required for 1 year in both formats, and the recommended storage medium.

\textbf{Solution:} \textbf{Raw binary format:} Data per scan: 64
channels x 3 bytes = 192 bytes + 8 bytes timestamp = 200 bytes/scan.
Scans per second: 1. Raw data rate: \textbf{200 bytes/s = 0.200 kB/s}.

Per hour: 200 x 3,600 = 720 kB/hour. Per day: 720 x 24 = 17.28 MB/day.
Per year: 17.28 x 365 = \textbf{6.31 GB/year}.

\textbf{CSV format:} Each reading: \textasciitilde8 characters for value
(e.g., ``100.0234'') + 20 bytes overhead = \textasciitilde28 bytes per
reading. Data per scan: 64 x 28 + 20 (timestamp) = 1,792 + 20 =
\textbf{1,812 bytes/scan}. CSV data rate: \textbf{1,812 bytes/s = 1.812
kB/s}.

Per year: 1,812 x 86,400 x 365 = 57.13 x 10\^{}9 bytes = \textbf{57.1
GB/year}.

CSV-to-binary ratio: 57.1 / 6.31 = \textbf{9.05x larger} (CSV is
\textasciitilde9x the size of binary).

With lossless compression (typical 3:1 for repetitive temperature data):
Binary compressed: 6.31 / 3 = \textbf{2.1 GB/year}. CSV compressed: 57.1
/ 5 = \textbf{11.4 GB/year} (CSV compresses better, \textasciitilde5:1).

\textbf{Recommended storage: A 32 GB industrial SD card or USB flash
drive is sufficient for binary format with compression. For CSV, a 64 GB
drive provides margin. For reliability over 1 year of continuous
writing, an industrial SSD (rated for high write endurance) is
preferred.}

Write speed requirement: 1.812 kB/s (CSV) -- trivially met by any modern
storage medium.

\begin{center}\rule{0.5\linewidth}{0.5pt}\end{center}

\section{Problem 11.5.4}\label{problem-11.5.4}

\textbf{Given:} An automated test station tests PCBs using a 6-1/2 digit
DMM (SCPI-controlled over LAN), a 200 MHz oscilloscope, and a power
supply. The test sequence for each board: - 6 DC voltage measurements at
150 ms each - 4 resistance measurements at 200 ms each (includes range
switching) - 2 frequency measurements at 300 ms each - 1 oscilloscope
waveform capture at 500 ms - Board handler time: 1.2 s (load/unload)

The switching matrix adds 10 ms per relay actuation, with 1 relay per
test point.

\textbf{Find:} The total test time per board, the throughput in
boards/hour, and the throughput improvement if the handler is pipelined
(handler loads next board while testing current board).

\textbf{Solution:} Measurement time: DC voltage: 6 x (150 + 10) = 6 x
160 = 960 ms. Resistance: 4 x (200 + 10) = 4 x 210 = 840 ms. Frequency:
2 x (300 + 10) = 2 x 310 = 620 ms. Oscilloscope: 1 x (500 + 10) = 510
ms.

Total measurement time: 960 + 840 + 620 + 510 = \textbf{2,930 ms = 2.93
s}.

Communication overhead (SCPI commands, \textasciitilde5 ms per
measurement): 13 x 5 = 65 ms.

Total test time: 2.93 + 0.065 + 1.2 (handler) = \textbf{4.195 s per
board}.

Throughput: 3,600 / 4.195 = \textbf{858 boards/hour}.

With pipelined handler (handler operates during test): Test time =
max(measurement, handler) = max(2.995, 1.2) = 2.995 s. Since measurement
\textgreater{} handler, the handler time is hidden. Effective time per
board: \textbf{2.995 s}. Throughput: 3,600 / 2.995 = \textbf{1,202
boards/hour}.

\textbf{Improvement: (1,202 - 858) / 858 x 100 = 40.1\% throughput
improvement from pipelining the handler.}

To further increase throughput: use a faster DMM (100 ms measurements),
optimize SCPI communication (binary data transfer, query/response
overlapping), or implement dual-DUT fixturing.

\begin{center}\rule{0.5\linewidth}{0.5pt}\end{center}

\section{Problem 11.5.5}\label{problem-11.5.5}

\textbf{Given:} A 24-bit sigma-delta ADC operates with a master clock of
4.096 MHz and an oversampling ratio (OSR) of 256. The ADC uses a sinc3
decimation filter. The input signal is from a load cell bridge with 20
mV full-scale output.

\textbf{Find:} The output data rate, the theoretical noise-free
resolution, the noise floor in nV, and the effective resolution in bits.

\textbf{Solution:} Output data rate: f\textsubscript{out} =
f\textsubscript{clk} / OSR = 4,096,000 / 256 = \textbf{16,000 S/s = 16
kS/s}.

Theoretical oversampling benefit: Quantization noise is spread over 0 to
f\textsubscript{clk}/2 = 2.048 MHz. After decimation to 8 kHz bandwidth:
noise is reduced by the oversampling ratio.

For a sinc3 filter, the noise reduction is: SNR improvement = 10
log10(OSR) x (2n+1)/2 for nth-order noise shaping. For a first-order
sigma-delta with sinc3: SNR \textasciitilde{} 6.02N + 1.76 + 30
log10(OSR) - 10 log10(pi\^{}(2)/3) (approximately).

More practically, sigma-delta ADCs specify noise performance directly.
At 16 kS/s with 24-bit ADC, typical RMS noise is \textasciitilde1-3 uV
on a 2.5 V reference.

Assuming 2 uV\textsubscript{rms} noise: Noise referred to 20 mV full
scale: 2 uV / 20 mV = 10\^{}-4 = 0.01\%.

Effective resolution: ENOB = log2(full-scale / noise\_pp) = log2(20 mV /
(6.6 x 2 uV)) = log2(20,000/13.2) = log2(1,515) = \textbf{10.6 bits} for
peak-to-peak noise-free resolution.

This seems low because we are using only 20 mV of the 2.5 V input range.
With PGA gain of 128: Effective input range = 2.5/128 = 19.5 mV
(well-matched to the 20 mV full-scale). Noise at gain 128 referred to
input: typically 0.3 uV\textsubscript{rms}. Noise-free resolution: 20 mV
/ (6.6 x 0.3 uV) = 20,000 / 1.98 = 10,101. ENOB = log2(10,101) =
\textbf{13.3 noise-free bits}.

\textbf{At 16 kS/s with PGA gain of 128, the effective resolution is
approximately 13.3 noise-free bits on the 20 mV range, sufficient for
0.01\% load cell accuracy. Reducing the output rate to 10 S/s would
improve to approximately 18-20 noise-free bits.}

\begin{center}\rule{0.5\linewidth}{0.5pt}\end{center}

\section{Problem 11.5.6}\label{problem-11.5.6}

\textbf{Given:} A SCPI-controlled test system uses a PyVISA script to
measure 100 DUTs. Each DUT requires: power supply set to 12 V (50 ms
settling), measure supply current (DMM, 100 ms), measure 3 output
voltages (DMM, 80 ms each), and power down (20 ms). LAN communication
latency is 5 ms per command/query. Each measurement requires 2 SCPI
transactions (trigger + read).

\textbf{Find:} The total time per DUT, the total test time for 100 DUTs,
and the throughput improvement from using the DMM's internal scan list
(which eliminates per-measurement command latency for the 3 output
voltages).

\textbf{Solution:} \textbf{Standard sequential approach:} Power on: 1
command x 5 ms + 50 ms settling = 55 ms. Current measurement: 2
transactions x 5 ms + 100 ms = 110 ms. 3 voltage measurements: 3 x (2 x
5 ms + 80 ms) = 3 x 90 = 270 ms. Power down: 1 x 5 ms + 20 ms = 25 ms.

Total per DUT: 55 + 110 + 270 + 25 = \textbf{460 ms}.

For 100 DUTs: 100 x 460 = 46,000 ms = \textbf{46.0 s}. Throughput: 3,600
/ 0.46 = \textbf{7,826 DUTs/hour}.

\textbf{With DMM scan list optimization:} The 3 voltage measurements are
configured as a scan list with a single trigger command: Scan setup
(one-time): 3 commands x 5 ms = 15 ms (amortized over 100 DUTs: 0.15
ms/DUT). Trigger scan: 1 x 5 ms = 5 ms. 3 measurements execute
internally: 3 x 80 ms = 240 ms (no per-measurement latency). Read all
results: 1 x 5 ms = 5 ms.

3 voltage measurements total: 5 + 240 + 5 = 250 ms (vs.~270 ms).

Total per DUT: 55 + 110 + 250 + 25 = \textbf{440 ms}.

For 100 DUTs: 100 x 440 = 44,000 ms = \textbf{44.0 s}. Time saved: 46.0
- 44.0 = \textbf{2.0 s (4.3\% improvement)}.

\textbf{The improvement is modest because the DMM measurement time
dominates. For higher throughput, use a faster DMM (NPLC=0.1 for
\textasciitilde20 ms per reading instead of 80 ms), reducing the voltage
scan to 3 x 20 = 60 ms and cutting total per-DUT time to 55 + 50 + 70 +
25 = 200 ms = 18,000 DUTs/hour -- a 2.3x improvement.}

\chapter{Chapter 12 -- Section 12.1: DC
Motors}\label{chapter-12-section-12.1-dc-motors}

Practice problems covering brushed DC motors, brushless DC (BLDC)
motors, and universal motors.

\begin{center}\rule{0.5\linewidth}{0.5pt}\end{center}

\section{Problem 12.1.1}\label{problem-12.1.1}

\textbf{Given:} A series-wound DC motor has V\textsubscript{supply} =
120 V, armature resistance R\textsubscript{a} = 0.3 Ohm, series field
resistance R\textsubscript{f} = 0.2 Ohm, and a motor constant K = 0.05
V/(A rad/s) (back-EMF is proportional to both flux and speed:
E\textsubscript{back} = K x I\textsubscript{a} x omega). At rated load,
the motor draws 50 A.

\textbf{Find:} The back-EMF, the rated speed in RPM, the output torque,
and the mechanical output power.

\textbf{Solution:} Total resistance: R\textsubscript{total} =
R\textsubscript{a} + R\textsubscript{f} = 0.3 + 0.2 = \textbf{0.5 Ohm}.

Back-EMF: E\textsubscript{back} = V - I\textsubscript{a} x
R\textsubscript{total} = 120 - 50 x 0.5 = 120 - 25 = \textbf{95 V}.

For a series motor: E\textsubscript{back} = K x I\textsubscript{a} x
omega. omega = E\textsubscript{back} / (K x I\textsubscript{a}) = 95 /
(0.05 x 50) = 95 / 2.5 = \textbf{38.0 rad/s = 362.9 RPM}.

Torque: T = K x I\textsubscript{a}\^{}2 = 0.05 x 50\^{}2 = 0.05 x 2,500
= \textbf{125.0 N-m} (for a series motor, torque is proportional to
I\^{}2).

Wait -- let us reconsider. The electromagnetic torque equals the
electromagnetic power divided by speed: P\textsubscript{em} =
E\textsubscript{back} x I\textsubscript{a} = 95 x 50 = 4,750 W. T =
P\textsubscript{em} / omega = 4,750 / 38.0 = \textbf{125.0 N-m}.
Confirmed.

Mechanical output (assuming 5\% rotational losses): P\textsubscript{out}
= 0.95 x 4,750 = \textbf{4,512 W = 6.05 HP}.

\textbf{The series motor produces high torque (125 N-m) at low speed
(363 RPM), characteristic of series motors used in traction and starter
applications. The torque-speed curve follows an inverse-square
relationship: doubling the speed requires halving the current, which
quarters the torque.}

\begin{center}\rule{0.5\linewidth}{0.5pt}\end{center}

\section{Problem 12.1.2}\label{problem-12.1.2}

\textbf{Given:} A compound DC motor (cumulative) operates at 240 V with
armature resistance R\textsubscript{a} = 0.4 Ohm. The shunt field
produces a flux constant K\textsubscript{shunt} = 1.0 V/(rad/s), and the
series field adds 10\% additional flux at full-load current of 35 A. At
no-load, I\textsubscript{a} = 3 A and the total flux constant is
K\textsubscript{shunt} only.

\textbf{Find:} The no-load speed, the full-load speed (with the series
field contribution), and the speed regulation.

\textbf{Solution:} \textbf{No-load:} E\textsubscript{back} = V -
I\textsubscript{a} x R\textsubscript{a} = 240 - 3 x 0.4 = 240 - 1.2 =
238.8 V. omega\textsubscript{NL} = E\textsubscript{back} /
K\textsubscript{shunt} = 238.8 / 1.0 = \textbf{238.8 rad/s = 2,281 RPM}.

\textbf{Full-load:} E\textsubscript{back} = V - I\textsubscript{a} x
R\textsubscript{a} = 240 - 35 x 0.4 = 240 - 14 = 226 V. Total flux
constant at full load: K\textsubscript{total} = K\textsubscript{shunt} x
1.10 = 1.0 x 1.10 = 1.10 V/(rad/s). omega\textsubscript{FL} =
E\textsubscript{back} / K\textsubscript{total} = 226 / 1.10 =
\textbf{205.5 rad/s = 1,962 RPM}.

Speed regulation: SR = (omega\textsubscript{NL} -
omega\textsubscript{FL}) / omega\textsubscript{FL} x 100\% = (238.8 -
205.5) / 205.5 x 100\% = 33.3 / 205.5 x 100\% = \textbf{16.2\%}.

Full-load torque: T = K\textsubscript{total} x I\textsubscript{a} = 1.10
x 35 = \textbf{38.5 N-m}. Mechanical power: P = T x omega = 38.5 x 205.5
= \textbf{7,912 W = 10.6 HP}.

\textbf{The compound motor has higher speed regulation (16.2\%) than a
pure shunt motor (\textasciitilde5-10\%) because the series field
increases flux under load, further reducing speed. However, it provides
better starting torque and overload capacity than a shunt motor alone.}

\begin{center}\rule{0.5\linewidth}{0.5pt}\end{center}

\section{Problem 12.1.3}\label{problem-12.1.3}

\textbf{Given:} A BLDC motor has 12 poles, a torque constant
K\textsubscript{t} = 0.08 N-m/A, a back-EMF constant K\textsubscript{e}
= 0.08 V/(rad/s), and a phase resistance R\textsubscript{ph} = 0.35 Ohm.
The DC bus voltage is 24 V. The motor must deliver 0.5 N-m at 5,000 RPM.

\textbf{Find:} The required phase current, the back-EMF, whether the bus
voltage is sufficient, and the copper losses.

\textbf{Solution:} Speed: omega = 5,000 x 2 pi / 60 = \textbf{523.6
rad/s}.

Required current: I = T / K\textsubscript{t} = 0.5 / 0.08 = \textbf{6.25
A}.

Back-EMF: E\textsubscript{back} = K\textsubscript{e} x omega = 0.08 x
523.6 = \textbf{41.89 V}.

\textbf{The back-EMF (41.89 V) exceeds the bus voltage (24 V), so the
motor CANNOT operate at 5,000 RPM with a 24 V bus.}

Maximum speed at 24 V bus (allowing 2 V for transistor drops and I x R):
Available back-EMF: E\textsubscript{max} = 24 - 2 x 6.25 x 0.35 - 2 = 24
- 4.375 - 2 = 17.625 V. (With two phases conducting: voltage drop = 2 x
I x R\textsubscript{ph}.) omega\textsubscript{max} =
E\textsubscript{max} / K\textsubscript{e} = 17.625 / 0.08 = 220.3 rad/s
= \textbf{2,104 RPM}.

At 2,104 RPM: copper losses = 2 x I\^{}2 x R\textsubscript{ph} = 2 x
6.25\^{}2 x 0.35 = 2 x 13.67 = \textbf{27.3 W}.

Mechanical power: P = T x omega = 0.5 x 220.3 = \textbf{110.2 W}.
Efficiency (copper losses only): eta = 110.2 / (110.2 + 27.3) = 110.2 /
137.5 = \textbf{80.1\%}.

\textbf{To reach 5,000 RPM at 0.5 N-m, the bus voltage must be at least
E\textsubscript{back} + 2IR + V\textsubscript{drop} = 41.89 + 4.375 + 2
= 48.3 V. A 48 V bus would be the appropriate selection.}

\begin{center}\rule{0.5\linewidth}{0.5pt}\end{center}

\section{Problem 12.1.4}\label{problem-12.1.4}

\textbf{Given:} A universal motor in a vacuum cleaner is rated at 240 V,
6 A, 28,000 RPM. The combined armature and field resistance is
R\textsubscript{total} = 3.5 Ohm. The motor is operated at reduced speed
using a triac controller that reduces the effective voltage to 180 V.
Assume the motor's torque-speed characteristic follows T proportional to
I\^{}2 (series motor behavior).

\textbf{Find:} The back-EMF and power at rated conditions, the current
at 180 V (assuming speed drops proportionally to voltage for a lightly
loaded series motor), and the approximate new speed and power.

\textbf{Solution:} \textbf{Rated conditions (240 V, 6 A):}
E\textsubscript{back} = V - I x R\textsubscript{total} = 240 - 6 x 3.5 =
240 - 21 = \textbf{219 V}. Input power: P\textsubscript{in} = 240 x 6 =
1,440 W. Electromagnetic power: P\textsubscript{em} =
E\textsubscript{back} x I = 219 x 6 = 1,314 W. Copper losses:
P\textsubscript{Cu} = I\^{}2 x R = 36 x 3.5 = 126 W.

\textbf{At 180 V (triac-controlled):} For a lightly loaded series motor
at reduced voltage, the speed drops roughly proportionally to voltage
and the current drops similarly. Approximate ratio: V\textsubscript{new}
/ V\textsubscript{rated} = 180/240 = 0.75.

Using the relation E\textsubscript{back} proportional to I x omega
(series motor), and V = E\textsubscript{back} + I x R: Assume
I\textsubscript{new} \textasciitilde{} 0.75 x 6 = 4.5 A (first
approximation). E\textsubscript{back,new} = 180 - 4.5 x 3.5 = 180 -
15.75 = 164.25 V.

Speed ratio: omega\textsubscript{new}/omega\textsubscript{rated} =
(E\textsubscript{back,new}/E\textsubscript{back,rated}) x
(I\textsubscript{rated}/I\textsubscript{new}) = (164.25/219) x (6/4.5) =
0.750 x 1.333 = 1.00.

This suggests the speed stays nearly constant (characteristic of a
lightly loaded series motor). Let's refine: At constant load torque (T
proportional to I\^{}2): if torque is constant, I is constant at 6 A.
E\textsubscript{back} = 180 - 6 x 3.5 = 159 V. omega\textsubscript{new}
= omega\textsubscript{rated} x (159/219) x (6/6) = 28,000 x 0.726 =
\textbf{20,329 RPM}. New power: P\textsubscript{em} = 159 x 6 =
\textbf{954 W}.

\textbf{At reduced voltage with constant torque load, speed drops to
\textasciitilde20,300 RPM and power to \textasciitilde954 W. For a
variable-torque load like a vacuum (where torque scales with
speed\^{}2), the speed reduction would be less dramatic.}

\begin{center}\rule{0.5\linewidth}{0.5pt}\end{center}

\section{Problem 12.1.5}\label{problem-12.1.5}

\textbf{Given:} A shunt-wound DC motor is used to drive a hoist.
Nameplate: 200 V, R\textsubscript{a} = 0.6 Ohm, K = 0.8 V/(rad/s). The
motor operates in two modes: (1) hoisting at full load
I\textsubscript{a} = 30 A, and (2) regenerative braking while lowering
the load, where the motor acts as a generator feeding current back to
the supply at I\textsubscript{a} = -20 A (current flows in reverse).

\textbf{Find:} The motor speed in each mode and the power flow
direction.

\textbf{Solution:} \textbf{Mode 1 -- Hoisting (motoring):}
E\textsubscript{back} = V - I\textsubscript{a} x R\textsubscript{a} =
200 - 30 x 0.6 = 200 - 18 = 182 V. omega = E\textsubscript{back} / K =
182 / 0.8 = \textbf{227.5 rad/s = 2,173 RPM}.

Power from supply: P\textsubscript{in} = V x I\textsubscript{a} = 200 x
30 = 6,000 W (consumed). Mechanical power: P\textsubscript{mech} =
E\textsubscript{back} x I\textsubscript{a} = 182 x 30 = 5,460 W (to
load). Copper losses: I\^{}2 R = 900 x 0.6 = 540 W.

\textbf{Mode 2 -- Lowering (regenerative braking):}
E\textsubscript{back} = V - I\textsubscript{a} x R\textsubscript{a} =
200 - (-20) x 0.6 = 200 + 12 = \textbf{212 V}. omega = 212 / 0.8 =
\textbf{265.0 rad/s = 2,531 RPM}.

The back-EMF (212 V) exceeds the supply voltage (200 V), causing current
to flow back into the supply. Power returned to supply: P = V x
\textbar I\textsubscript{a}\textbar{} = 200 x 20 = \textbf{4,000 W}
(regenerated). Mechanical power input (from load): P\textsubscript{mech}
= E\textsubscript{back} x \textbar I\textsubscript{a}\textbar{} = 212 x
20 = 4,240 W. Copper losses: 400 x 0.6 = 240 W. Check: 4,240 = 4,000 +
240. Confirmed.

\textbf{During lowering, the motor runs faster than the motoring speed
(2,531 vs.~2,173 RPM) because the back-EMF must exceed the supply
voltage for current reversal. The load's potential energy is partially
recovered (4,000 W to supply) with 240 W dissipated as copper losses.}

\begin{center}\rule{0.5\linewidth}{0.5pt}\end{center}

\section{Problem 12.1.6}\label{problem-12.1.6}

\textbf{Given:} A BLDC motor for a drone propeller has
K\textsubscript{t} = 0.012 N-m/A, K\textsubscript{e} = 0.012 V/(rad/s),
R\textsubscript{ph} = 0.08 Ohm, and no-load current I\textsubscript{0} =
1.5 A. The motor operates from a 22.2 V (6S LiPo) battery. At hover, the
motor draws 18 A.

\textbf{Find:} The hover speed, the propeller torque, the mechanical
output power, the motor efficiency, and the thrust if the propeller
constant is K\textsubscript{thrust} = 1.1 x 10\^{}-7 N/(RPM)\^{}2.

\textbf{Solution:} Back-EMF at hover: E\textsubscript{back} = V - 2 x I
x R\textsubscript{ph} = 22.2 - 2 x 18 x 0.08 = 22.2 - 2.88 =
\textbf{19.32 V}.

Speed: omega = E\textsubscript{back} / K\textsubscript{e} = 19.32 /
0.012 = 1,610 rad/s. In RPM: n = 1,610 x 60 / (2 pi) = \textbf{15,378
RPM}.

Torque-producing current: I\textsubscript{torque} = I -
I\textsubscript{0} = 18 - 1.5 = 16.5 A. Propeller torque: T =
K\textsubscript{t} x I\textsubscript{torque} = 0.012 x 16.5 =
\textbf{0.198 N-m}.

Mechanical output: P\textsubscript{mech} = T x omega = 0.198 x 1,610 =
\textbf{318.8 W}.

Input power: P\textsubscript{in} = V x I = 22.2 x 18 = 399.6 W.
Efficiency: eta = 318.8 / 399.6 = \textbf{79.8\%}.

Losses breakdown: Copper: 2 x 18\^{}2 x 0.08 = 51.84 W. No-load
(friction, windage, iron): E\textsubscript{back} x I\textsubscript{0} =
19.32 x 1.5 = 28.98 W. Total losses: 51.84 + 28.98 = 80.82 W. Check:
399.6 - 318.8 = 80.8 W. Confirmed.

Thrust: F = K\textsubscript{thrust} x n\^{}2 = 1.1 x 10\^{}-7 x
(15,378)\^{}2 = 1.1 x 10\^{}-7 x 2.365 x 10\^{}8 = \textbf{26.0 N}
(approximately 2.65 kg of thrust).

\textbf{Each motor produces 26 N of thrust at 18 A hover current. For a
quadcopter, total thrust is 4 x 26 = 104 N, supporting an all-up weight
of \textasciitilde10.6 kg with a 2.5:1 thrust-to-weight ratio for
maneuverability.}

\chapter{Chapter 12 --- Section 12.2: AC
Motors}\label{chapter-12-section-12.2-ac-motors}

Practice problems covering induction motors, synchronous motors,
single-phase motors, reluctance motors, linear motors, and wound-rotor
induction motors.

\begin{center}\rule{0.5\linewidth}{0.5pt}\end{center}

\section{Problem 12.2.1}\label{problem-12.2.1}

\textbf{Given:} A 6-pole, three-phase induction motor is connected to a
50 Hz supply. At full load, the motor speed is 960 RPM and it delivers
30 kW of mechanical power.

\textbf{Find:} (a) The synchronous speed, (b) the slip, (c) the rotor
electrical frequency, and (d) the full-load torque.

\textbf{Solution:}

\begin{enumerate}
\def\labelenumi{(\alph{enumi})}
\item
  Synchronous speed: n\textsubscript{s} = 120f / P = 120 × 50 / 6 =
  \textbf{1,000 RPM}
\item
  Slip: s = (n\textsubscript{s} − n\textsubscript{r}) /
  n\textsubscript{s} = (1,000 − 960) / 1,000 = 0.04 = \textbf{4.0\%}
\item
  Rotor electrical frequency: f\textsubscript{r} = s × f = 0.04 × 50 =
  \textbf{2.0 Hz}
\item
  Rotor speed in rad/s: ω\textsubscript{r} = 960 × 2π/60 = 100.5 rad/s
\end{enumerate}

Full-load torque: τ = P\textsubscript{mech} / ω\textsubscript{r} =
30,000 / 100.5 = \textbf{298.5 N·m}

\begin{center}\rule{0.5\linewidth}{0.5pt}\end{center}

\section{Problem 12.2.2}\label{problem-12.2.2}

\textbf{Given:} An 8-pole, three-phase synchronous motor operates at 60
Hz. The motor is rated at 200 kW with an efficiency of 93\% and runs at
unity power factor. The terminal voltage is 4,160 V (line-to-line).

\textbf{Find:} (a) The motor speed, (b) the input power, (c) the line
current, and (d) the apparent power.

\textbf{Solution:}

\begin{enumerate}
\def\labelenumi{(\alph{enumi})}
\item
  Synchronous speed: n\textsubscript{s} = 120f / P = 120 × 60 / 8 =
  \textbf{900 RPM}
\item
  Input power: P\textsubscript{in} = P\textsubscript{out} / η = 200 /
  0.93 = \textbf{215.1 kW}
\item
  At unity power factor, S = P: I\textsubscript{L} = P\textsubscript{in}
  / (√3 × V\textsubscript{LL}) = 215,100 / (1.732 × 4,160) = 215,100 /
  7,205 = \textbf{29.9 A}
\item
  At unity power factor: S = P = \textbf{215.1 kVA}
\end{enumerate}

\begin{center}\rule{0.5\linewidth}{0.5pt}\end{center}

\section{Problem 12.2.3}\label{problem-12.2.3}

\textbf{Given:} A capacitor-start single-phase induction motor operates
at 240 V, 60 Hz. The main winding impedance is Z\textsubscript{m} = 8 +
j10 Ω and the auxiliary winding impedance is Z\textsubscript{aux} = 15 +
j6 Ω. A start capacitor is placed in series with the auxiliary winding.

\textbf{Find:} The capacitance required to place the auxiliary winding
current exactly 90° ahead of the main winding current for maximum
starting torque.

\textbf{Solution:}

Main winding impedance angle: φ\textsubscript{m} = tan⁻¹(10/8) =
tan⁻¹(1.25) = 51.3°

For the auxiliary current to lead the main current by 90°, the auxiliary
impedance angle must be: φ\textsubscript{aux} = 51.3° − 90° = −38.7°
(capacitive)

With a series capacitor of reactance X\textsubscript{C}:
Z\textsubscript{total} = 15 + j(6 − X\textsubscript{C})

For φ\textsubscript{aux} = −38.7°: tan(−38.7°) = (6 −
X\textsubscript{C}) / 15 −0.801 = (6 − X\textsubscript{C}) / 15 6 −
X\textsubscript{C} = −12.02 X\textsubscript{C} = 18.02 Ω

Capacitance: C = 1 / (2πfX\textsubscript{C}) = 1 / (2π × 60 × 18.02) = 1
/ 6,793 = \textbf{147.2 μF}

A standard \textbf{150 μF} start capacitor would be selected.

\begin{center}\rule{0.5\linewidth}{0.5pt}\end{center}

\section{Problem 12.2.4}\label{problem-12.2.4}

\textbf{Given:} A 4-pole synchronous reluctance motor operates from a 60
Hz VFD. The d-axis inductance is L\textsubscript{d} = 150 mH and the
q-axis inductance is L\textsubscript{q} = 25 mH. The rated stator
current is 8 A.

\textbf{Find:} (a) The synchronous speed, (b) the saliency ratio, and
(c) the maximum reluctance torque using τ = (3/2) × (P/2) ×
(L\textsubscript{d} − L\textsubscript{q}) × I\textsubscript{d} ×
I\textsubscript{q}, where maximum torque occurs at I\textsubscript{d} =
I\textsubscript{q} = I\textsubscript{s}/√2.

\textbf{Solution:}

\begin{enumerate}
\def\labelenumi{(\alph{enumi})}
\item
  Synchronous speed: n\textsubscript{s} = 120 × 60 / 4 = \textbf{1,800
  RPM}
\item
  Saliency ratio: ξ = L\textsubscript{d} / L\textsubscript{q} = 150 / 25
  = \textbf{6.0}
\item
  For maximum torque: I\textsubscript{d} = I\textsubscript{q} =
  I\textsubscript{s} / √2 = 8 / 1.414 = 5.657 A
\end{enumerate}

τ = (3/2) × (P/2) × (L\textsubscript{d} − L\textsubscript{q}) ×
I\textsubscript{d} × I\textsubscript{q} τ = 1.5 × 2 × (0.150 − 0.025) ×
5.657 × 5.657 τ = 3.0 × 0.125 × 32.0 = \textbf{12.0 N·m}

Mechanical power at rated speed: ω = 1,800 × 2π/60 = 188.5 rad/s
P\textsubscript{mech} = 12.0 × 188.5 = 2,262 W = \textbf{2.26 kW}

\begin{center}\rule{0.5\linewidth}{0.5pt}\end{center}

\section{Problem 12.2.5}\label{problem-12.2.5}

\textbf{Given:} A linear induction motor for a baggage handling system
has a pole pitch of τ\textsubscript{p} = 80 mm, operates at 60 Hz, and
must propel a 50 kg cart at a steady-state speed of 3 m/s with a
friction force of 120 N. The slip at operating speed is 8\%.

\textbf{Find:} (a) The synchronous speed of the traveling field, (b) the
actual cart speed from slip, (c) the thrust force required, and (d) the
mechanical power delivered.

\textbf{Solution:}

\begin{enumerate}
\def\labelenumi{(\alph{enumi})}
\item
  Synchronous speed of the traveling field: v\textsubscript{s} =
  2τ\textsubscript{p}f = 2 × 0.080 × 60 = \textbf{9.6 m/s}
\item
  Actual speed from slip: v = v\textsubscript{s}(1 − s) = 9.6 × (1 −
  0.08) = 9.6 × 0.92 = 8.83 m/s
\end{enumerate}

However, the problem states steady-state speed of 3 m/s. This means the
frequency must be adjusted. At 3 m/s with 8\% slip: v\textsubscript{s} =
v / (1 − s) = 3 / 0.92 = 3.26 m/s Required frequency: f =
v\textsubscript{s} / (2τ\textsubscript{p}) = 3.26 / (2 × 0.080) =
\textbf{20.4 Hz}

\begin{enumerate}
\def\labelenumi{(\alph{enumi})}
\setcounter{enumi}{2}
\item
  At steady state, the thrust equals friction: F = \textbf{120 N}
\item
  Mechanical power: P\textsubscript{mech} = F × v = 120 × 3 =
  \textbf{360 W}
\end{enumerate}

\begin{center}\rule{0.5\linewidth}{0.5pt}\end{center}

\section{Problem 12.2.6}\label{problem-12.2.6}

\textbf{Given:} A 6-pole, 60 Hz wound-rotor induction motor has a
standstill rotor EMF of E₂ = 250 V (line-to-line), rotor resistance R₂ =
0.4 Ω per phase, and rotor leakage reactance X₂ = 2.0 Ω per phase.

\textbf{Find:} (a) The synchronous speed, (b) the slip at maximum torque
with no external resistance, (c) the speed at maximum torque, and (d)
the external resistance per phase needed to achieve maximum torque at
standstill.

\textbf{Solution:}

\begin{enumerate}
\def\labelenumi{(\alph{enumi})}
\item
  Synchronous speed: n\textsubscript{s} = 120 × 60 / 6 = \textbf{1,200
  RPM}
\item
  Slip at maximum torque (no external resistance): s\textsubscript{max}
  = R₂ / X₂ = 0.4 / 2.0 = 0.2 = \textbf{20\%}
\item
  Speed at maximum torque: n = n\textsubscript{s}(1 −
  s\textsubscript{max}) = 1,200 × 0.8 = \textbf{960 RPM}
\item
  For maximum torque at standstill (s = 1): s\textsubscript{max} =
  R₂\textsubscript{total} / X₂ = 1, so R₂\textsubscript{total} = X₂ =
  2.0 Ω per phase R\textsubscript{ext} = R₂\textsubscript{total} − R₂ =
  2.0 − 0.4 = \textbf{1.6 Ω per phase}
\end{enumerate}

\begin{center}\rule{0.5\linewidth}{0.5pt}\end{center}

\section{Problem 12.2.7}\label{problem-12.2.7}

\textbf{Given:} A 2-pole, three-phase induction motor operates at 60 Hz.
At rated load the slip is 2.5\%. The motor delivers 75 kW of mechanical
power. The stator copper loss is 2.1 kW, core loss is 1.5 kW, and
friction/windage loss is 0.9 kW.

\textbf{Find:} (a) The synchronous speed and rotor speed, (b) the rotor
copper loss, (c) the air-gap power, (d) the total input power, and (e)
the motor efficiency.

\textbf{Solution:}

\begin{enumerate}
\def\labelenumi{(\alph{enumi})}
\item
  Synchronous speed: n\textsubscript{s} = 120 × 60 / 2 = 3,600 RPM
  n\textsubscript{r} = n\textsubscript{s}(1 − s) = 3,600 × 0.975 =
  \textbf{3,510 RPM}
\item
  The relationship P\textsubscript{mech} = P\textsubscript{ag}(1 − s)
  gives: P\textsubscript{ag} = (P\textsubscript{mech} +
  P\textsubscript{fw}) / (1 − s) = (75 + 0.9) / 0.975 = 75.9 / 0.975 =
  77.85 kW
\end{enumerate}

Rotor copper loss: P\textsubscript{Cu,rotor} = s × P\textsubscript{ag} =
0.025 × 77.85 = \textbf{1.95 kW}

\begin{enumerate}
\def\labelenumi{(\alph{enumi})}
\setcounter{enumi}{2}
\item
  Air-gap power: P\textsubscript{ag} = \textbf{77.85 kW}
\item
  Total input power: P\textsubscript{in} = P\textsubscript{ag} +
  P\textsubscript{Cu,stator} + P\textsubscript{core} = 77.85 + 2.1 + 1.5
  = \textbf{81.45 kW}
\item
  Motor efficiency: η = P\textsubscript{mech} / P\textsubscript{in} = 75
  / 81.45 = 0.921 = \textbf{92.1\%}
\end{enumerate}

\begin{center}\rule{0.5\linewidth}{0.5pt}\end{center}

\section{Problem 12.2.8}\label{problem-12.2.8}

\textbf{Given:} A permanent magnet synchronous motor (PMSM) for an
electric vehicle has 8 poles, a torque constant K\textsubscript{t} =
0.85 N·m/A, and a back-EMF constant K\textsubscript{e} = 0.85 V/(rad/s).
The DC bus voltage is 400 V, and the motor must deliver 150 N·m at 4,000
RPM.

\textbf{Find:} (a) The required phase current, (b) the back-EMF, (c) the
mechanical output power, and (d) whether the motor can operate at this
speed with the given bus voltage.

\textbf{Solution:}

\begin{enumerate}
\def\labelenumi{(\alph{enumi})}
\item
  Required phase current: I = τ / K\textsubscript{t} = 150 / 0.85 =
  \textbf{176.5 A}
\item
  Speed in rad/s: ω = 4,000 × 2π/60 = 418.9 rad/s
\end{enumerate}

Back-EMF (line-to-line): E\textsubscript{back} = K\textsubscript{e} × ω
= 0.85 × 418.9 = \textbf{356.1 V}

\begin{enumerate}
\def\labelenumi{(\alph{enumi})}
\setcounter{enumi}{2}
\item
  Mechanical output power: P\textsubscript{mech} = τ × ω = 150 × 418.9 =
  62,832 W = \textbf{62.8 kW} (84.3 HP)
\item
  The back-EMF of 356.1 V is less than the DC bus voltage of 400 V. The
  remaining voltage (400 − 356.1 = 43.9 V) must overcome the resistive
  drops and provide current regulation headroom. The motor \textbf{can
  operate} at this speed, but it is approaching the voltage limit. Above
  approximately 4,500 RPM, field weakening (reducing I\textsubscript{d})
  would be required.
\end{enumerate}

\begin{center}\rule{0.5\linewidth}{0.5pt}\end{center}

\section{Problem 12.2.9}\label{problem-12.2.9}

\textbf{Given:} A three-phase induction motor drives a centrifugal pump.
Nameplate data: 100 HP, 460 V, 60 Hz, 4-pole, FLA = 124 A, efficiency =
94.5\%, power factor = 0.87 lagging at full load. The motor operates at
85\% load.

\textbf{Find:} (a) The synchronous speed, (b) the input power at 85\%
load (assuming constant efficiency), (c) the line current at 85\% load,
and (d) the reactive power consumed.

\textbf{Solution:}

\begin{enumerate}
\def\labelenumi{(\alph{enumi})}
\item
  Synchronous speed: n\textsubscript{s} = 120 × 60 / 4 = \textbf{1,800
  RPM}
\item
  Mechanical output at 85\% load: P\textsubscript{out} = 0.85 × 100 ×
  746 = 63,410 W
\end{enumerate}

Input power: P\textsubscript{in} = P\textsubscript{out} / η = 63,410 /
0.945 = \textbf{67,101 W = 67.1 kW}

\begin{enumerate}
\def\labelenumi{(\alph{enumi})}
\setcounter{enumi}{2}
\item
  Line current at 85\% load: I = P\textsubscript{in} / (√3 × V × PF) =
  67,101 / (1.732 × 460 × 0.87) = 67,101 / 693.2 = \textbf{96.8 A}
\item
  Apparent power: S = P\textsubscript{in} / PF = 67,101 / 0.87 = 77,127
  VA
\end{enumerate}

Reactive power: Q = √(S² − P²) = √(77,127² − 67,101²) = √(5,948.6 × 10⁶
− 4,502.5 × 10⁶) = √(1,446.1 × 10⁶) = \textbf{38,027 VAR = 38.0 kVAR}

\begin{center}\rule{0.5\linewidth}{0.5pt}\end{center}

\section{Problem 12.2.10}\label{problem-12.2.10}

\textbf{Given:} A linear synchronous motor for a maglev test sled has a
pole pitch of τ\textsubscript{p} = 50 mm, a force constant
K\textsubscript{f} = 500 N/A, and drives a sled with total mass 200 kg.
The sled must accelerate at 20 m/s² to a peak velocity of 100 m/s.

\textbf{Find:} (a) The thrust force for acceleration, (b) the peak
current during acceleration, (c) the electrical frequency at peak
velocity, (d) the acceleration time and distance.

\textbf{Solution:}

\begin{enumerate}
\def\labelenumi{(\alph{enumi})}
\item
  Thrust force for acceleration: F = ma = 200 × 20 = 4,000 N Adding
  aerodynamic drag estimate of 300 N at high speed:
  F\textsubscript{total} ≈ \textbf{4,300 N}
\item
  Peak current: I = F\textsubscript{total} / K\textsubscript{f} = 4,300
  / 500 = \textbf{8.6 A}
\item
  Electrical frequency at peak velocity: f = v / (2τ\textsubscript{p}) =
  100 / (2 × 0.050) = \textbf{1,000 Hz}
\item
  Acceleration time: t = v / a = 100 / 20 = \textbf{5.0 s}
\end{enumerate}

Acceleration distance: d = ½at² = ½ × 20 × 25 = \textbf{250 m}

Peak mechanical power at the moment of reaching 100 m/s: P =
F\textsubscript{total} × v = 4,300 × 100 = 430,000 W = \textbf{430 kW}

\chapter{Chapter 12 --- Section 12.3: Stepper
Motors}\label{chapter-12-section-12.3-stepper-motors}

Practice problems covering stepper motor types and operation, drive
modes (full-step, half-step, microstepping), applications, and resonance
and torque curves.

\begin{center}\rule{0.5\linewidth}{0.5pt}\end{center}

\section{Problem 12.3.1}\label{problem-12.3.1}

\textbf{Given:} A hybrid stepper motor has 200 steps per revolution and
a rated phase current of 1.5 A. The holding torque is 1.2 N·m and the
detent torque is 0.08 N·m.

\textbf{Find:} (a) The step angle, (b) the number of full steps to
rotate exactly 72°, (c) the maximum load torque with a 50\% safety
margin, and (d) the open-loop position accuracy.

\textbf{Solution:}

\begin{enumerate}
\def\labelenumi{(\alph{enumi})}
\item
  Step angle: θ\textsubscript{step} = 360° / 200 = \textbf{1.8° per
  step}
\item
  Steps for 72°: N = 72° / 1.8° = \textbf{40 full steps}
\item
  Maximum load torque with 50\% safety margin:
  τ\textsubscript{load(max)} = 0.50 × τ\textsubscript{holding} = 0.50 ×
  1.2 = \textbf{0.60 N·m}
\item
  Open-loop position accuracy: Typical non-cumulative error = ±5\% of
  one step = ±0.05 × 1.8° = \textbf{±0.09°}
\end{enumerate}

This error is non-cumulative for a properly loaded stepper motor, so it
remains ±0.09° regardless of the number of steps taken.

\begin{center}\rule{0.5\linewidth}{0.5pt}\end{center}

\section{Problem 12.3.2}\label{problem-12.3.2}

\textbf{Given:} A stepper motor with 200 full steps/revolution is driven
with 32× microstepping. The motor drives a ball screw with a pitch of 5
mm/revolution.

\textbf{Find:} (a) The microsteps per revolution, (b) the linear
resolution per microstep, (c) the pulse frequency for a linear speed of
100 mm/s, and (d) the number of pulses for a 50 mm move.

\textbf{Solution:}

\begin{enumerate}
\def\labelenumi{(\alph{enumi})}
\item
  Microsteps per revolution: 200 × 32 = \textbf{6,400 microsteps/rev}
\item
  Linear distance per microstep: d = pitch / microsteps per rev = 5 /
  6,400 = 0.000781 mm = \textbf{0.781 μm per microstep}
\item
  Revolutions per second for 100 mm/s: n = 100 / 5 = 20 rev/s
\end{enumerate}

Pulse frequency: f\textsubscript{pulse} = 20 × 6,400 = \textbf{128,000
Hz = 128 kHz}

\begin{enumerate}
\def\labelenumi{(\alph{enumi})}
\setcounter{enumi}{3}
\tightlist
\item
  Pulses for 50 mm: N = 50 / 0.000781 = \textbf{64,000 pulses}
\end{enumerate}

Verification: 64,000 / 128,000 = 0.500 s, and 100 mm/s × 0.500 s = 50 mm
✓

\begin{center}\rule{0.5\linewidth}{0.5pt}\end{center}

\section{Problem 12.3.3}\label{problem-12.3.3}

\textbf{Given:} A CNC router uses a NEMA 23 stepper motor (1.8°/step,
holding torque 1.9 N·m) to drive the Z-axis via a lead screw with 4 mm
pitch. The spindle and carriage mass is 5 kg, and the desired maximum
vertical acceleration is 2,000 mm/s².

\textbf{Find:} (a) The linear resolution in full-step mode, (b) the
gravitational force on the axis, (c) the total force and torque required
during upward acceleration, and (d) whether the motor is adequate.

\textbf{Solution:}

\begin{enumerate}
\def\labelenumi{(\alph{enumi})}
\item
  Linear resolution per step: d = pitch / steps per rev = 4 / 200 =
  \textbf{0.020 mm = 20 μm per step}
\item
  Gravitational force: F\textsubscript{g} = mg = 5 × 9.81 = \textbf{49.1
  N}
\item
  Acceleration force: F\textsubscript{a} = ma = 5 × 2.0 = 10.0 N
\end{enumerate}

Total force during upward acceleration: F\textsubscript{total} =
F\textsubscript{g} + F\textsubscript{a} + F\textsubscript{friction} =
49.1 + 10.0 + 5.0 = 64.1 N

Torque at motor shaft: τ = F\textsubscript{total} × pitch / (2π) = 64.1
× 0.004 / (2π) = 64.1 × 6.366 × 10⁻⁴ = \textbf{0.0408 N·m = 40.8 mN·m}

\begin{enumerate}
\def\labelenumi{(\alph{enumi})}
\setcounter{enumi}{3}
\tightlist
\item
  Maximum speed: typical 50 mm/s = 50/4 = 12.5 rev/s = 2,500 steps/s. At
  this speed, available torque ≈ 50\% of holding torque = 0.50 × 1.9 =
  0.95 N·m. Since 0.041 N·m \textless\textless{} 0.95 N·m, the motor is
  \textbf{more than adequate}.
\end{enumerate}

\begin{center}\rule{0.5\linewidth}{0.5pt}\end{center}

\section{Problem 12.3.4}\label{problem-12.3.4}

\textbf{Given:} A NEMA 17 stepper motor has a rotor inertia of
J\textsubscript{rotor} = 54 g·cm² (5.4 × 10⁻⁶ kg·m²), holding torque of
0.44 N·m, and step angle of 1.8°. It drives a load with reflected
inertia J\textsubscript{load} = 30 g·cm² (3.0 × 10⁻⁶ kg·m²).

\textbf{Find:} (a) The total inertia, (b) the natural resonant
frequency, (c) the resonant step rate, and (d) the resonant speed in
RPM.

\textbf{Solution:}

\begin{enumerate}
\def\labelenumi{(\alph{enumi})}
\item
  Total inertia: J\textsubscript{total} = J\textsubscript{rotor} +
  J\textsubscript{load} = 5.4 × 10⁻⁶ + 3.0 × 10⁻⁶ = \textbf{8.4 × 10⁻⁶
  kg·m²}
\item
  Holding torque stiffness: K\textsubscript{h} = T\textsubscript{hold} /
  θ\textsubscript{step} = 0.44 / (1.8 × π/180) = 0.44 / 0.03142 = 14.0
  N·m/rad
\end{enumerate}

Natural frequency: f\textsubscript{n} = (1/2π)√(K\textsubscript{h}/J) =
(1/2π)√(14.0 / 8.4 × 10⁻⁶) = (1/2π)√(1.667 × 10⁶) = (1/2π) × 1291 =
\textbf{205 Hz}

\begin{enumerate}
\def\labelenumi{(\alph{enumi})}
\setcounter{enumi}{2}
\item
  Resonant step rate: \textbf{205 full steps/s}
\item
  Resonant speed: n = 205 × 1.8° / 360° × 60 = \textbf{61.5 RPM}
\end{enumerate}

The acceleration profile should ramp quickly through the 180--230
steps/s region to avoid resonance.

\begin{center}\rule{0.5\linewidth}{0.5pt}\end{center}

\section{Problem 12.3.5}\label{problem-12.3.5}

\textbf{Given:} A stepper motor with 200 steps/rev drives a GT2 timing
belt with a 16-tooth pulley (2 mm belt pitch) in a laser cutter X-axis.
The carriage mass is 0.8 kg, and the maximum cutting speed is 200 mm/s
with 16× microstepping.

\textbf{Find:} (a) The linear resolution per microstep, (b) the pulse
frequency at maximum speed, (c) the maximum acceleration if the motor
can provide 0.3 N·m of torque at the required speed.

\textbf{Solution:}

\begin{enumerate}
\def\labelenumi{(\alph{enumi})}
\tightlist
\item
  Pulley circumference: C = 16 × 2 = 32 mm/rev
\end{enumerate}

Microsteps per revolution: 200 × 16 = 3,200

Linear resolution per microstep: d = 32 / 3,200 = \textbf{0.010 mm = 10
μm}

\begin{enumerate}
\def\labelenumi{(\alph{enumi})}
\setcounter{enumi}{1}
\tightlist
\item
  Revolutions per second: n = 200 / 32 = 6.25 rev/s
\end{enumerate}

Pulse frequency: f = 6.25 × 3,200 = \textbf{20,000 Hz = 20 kHz}

\begin{enumerate}
\def\labelenumi{(\alph{enumi})}
\setcounter{enumi}{2}
\tightlist
\item
  Pulley radius: r = C / (2π) = 32 / (2π) = 5.093 mm
\end{enumerate}

Available force at belt: F = τ / r = 0.3 / 0.005093 = 58.9 N

Net acceleration force (subtracting 3 N friction estimate):
F\textsubscript{net} = 58.9 − 3 = 55.9 N

Maximum acceleration: a = F / m = 55.9 / 0.8 = \textbf{69.9 m/s²}

\begin{center}\rule{0.5\linewidth}{0.5pt}\end{center}

\section{Problem 12.3.6}\label{problem-12.3.6}

\textbf{Given:} A stepper-driven rotary indexing table uses a 200-step
motor with a 50:1 worm gear reducer. The table must index to 72 equally
spaced positions around 360°.

\textbf{Find:} (a) The angular resolution at the table output per full
step, (b) the number of motor steps per index position, (c) the angular
error per index if the stepper has ±3\% step accuracy, and (d) whether
the system meets a ±0.05° positioning requirement.

\textbf{Solution:}

\begin{enumerate}
\def\labelenumi{(\alph{enumi})}
\item
  Angular resolution at output per full step: θ\textsubscript{output} =
  1.8° / 50 = \textbf{0.036° per motor step}
\item
  Angle per index position: θ\textsubscript{index} = 360° / 72 = 5.0°
\end{enumerate}

Motor steps per index: N = 5.0° / 0.036° = \textbf{138.9 steps}

Since this is not an integer, the actual step count rounds to 139 steps,
giving 139 × 0.036° = 5.004°. Over 72 positions this creates a
cumulative error. A better approach is to use microstepping or select a
gear ratio that divides evenly.

\begin{enumerate}
\def\labelenumi{(\alph{enumi})}
\setcounter{enumi}{2}
\tightlist
\item
  Step accuracy error per step: Δθ = ±0.03 × 0.036° = ±0.00108°
\end{enumerate}

Since stepper error is non-cumulative: Position error =
\textbf{±0.00108°} plus the rounding error of 0.004° per position.

\begin{enumerate}
\def\labelenumi{(\alph{enumi})}
\setcounter{enumi}{3}
\tightlist
\item
  Total worst-case error = 0.004 + 0.001 = \textbf{0.005°}, which is
  well within the ±0.05° requirement. The system \textbf{meets the
  specification}.
\end{enumerate}

\begin{center}\rule{0.5\linewidth}{0.5pt}\end{center}

\section{Problem 12.3.7}\label{problem-12.3.7}

\textbf{Given:} A stepper motor datasheet shows the following pull-out
torque values: 1.8 N·m at 200 steps/s, 1.2 N·m at 500 steps/s, 0.7 N·m
at 1,000 steps/s, and 0.3 N·m at 2,000 steps/s. The motor step angle is
1.8°. A constant load torque of 0.5 N·m must be driven.

\textbf{Find:} (a) The maximum speed at which the motor can sustain the
load with a 30\% safety margin, (b) the corresponding speed in RPM, and
(c) the mechanical power output at that speed.

\textbf{Solution:}

\begin{enumerate}
\def\labelenumi{(\alph{enumi})}
\tightlist
\item
  Required pull-out torque with 30\% margin: τ\textsubscript{required} =
  0.5 / 0.70 = 0.714 N·m
\end{enumerate}

From the torque curve, 0.714 N·m falls between the 1,000 steps/s (0.7
N·m) and 500 steps/s (1.2 N·m) data points. Interpolating linearly:

Fraction = (1.2 − 0.714) / (1.2 − 0.7) = 0.486 / 0.5 = 0.972

Speed = 500 + 0.972 × (1,000 − 500) = 500 + 486 = \textbf{986 steps/s}

Rounding conservatively: \textbf{\textasciitilde980 steps/s}

\begin{enumerate}
\def\labelenumi{(\alph{enumi})}
\setcounter{enumi}{1}
\item
  Speed in RPM: n = 980 × 1.8° / 360° × 60 = \textbf{294 RPM}
\item
  Mechanical power at operating point (using actual load torque): ω =
  294 × 2π/60 = 30.8 rad/s P = τ × ω = 0.5 × 30.8 = \textbf{15.4 W}
\end{enumerate}

\begin{center}\rule{0.5\linewidth}{0.5pt}\end{center}

\section{Problem 12.3.8}\label{problem-12.3.8}

\textbf{Given:} A pick-and-place machine uses a stepper motor to rotate
a vacuum nozzle. The motor has 200 steps/rev, holding torque 0.5 N·m,
and the nozzle assembly has an inertia of J = 2.0 × 10⁻⁵ kg·m². The
nozzle must rotate 90° in 50 ms with a trapezoidal velocity profile
(equal acceleration, constant velocity, and deceleration phases).

\textbf{Find:} (a) The steps required for 90°, (b) the peak angular
velocity, (c) the acceleration torque, and (d) the total torque during
acceleration.

\textbf{Solution:}

\begin{enumerate}
\def\labelenumi{(\alph{enumi})}
\item
  Steps for 90°: N = 90° / 1.8° = \textbf{50 steps}
\item
  For a trapezoidal profile with equal phases (each 50/3 ms ≈ 16.7 ms):
  Total angle = 90° = π/2 rad = 1.571 rad
\end{enumerate}

With equal time phases: θ = ½ × ω\textsubscript{peak} ×
t\textsubscript{accel} + ω\textsubscript{peak} × t\textsubscript{const}
+ ½ × ω\textsubscript{peak} × t\textsubscript{decel} 1.571 =
ω\textsubscript{peak} × (t/2 + t/3 + t/2) × (1/3 each)\ldots{}

Simplified: for equal three phases of t/3 each: θ =
ω\textsubscript{peak} × (t/3)/2 + ω\textsubscript{peak} × t/3 +
ω\textsubscript{peak} × (t/3)/2 = ω\textsubscript{peak} × 2t/3

ω\textsubscript{peak} = θ × 3/(2t) = 1.571 × 3 / (2 × 0.050) = 4.712 /
0.100 = \textbf{47.12 rad/s}

\begin{enumerate}
\def\labelenumi{(\alph{enumi})}
\setcounter{enumi}{2}
\tightlist
\item
  Acceleration time: t\textsubscript{a} = 50/3 = 16.67 ms Angular
  acceleration: α = ω\textsubscript{peak} / t\textsubscript{a} = 47.12 /
  0.01667 = 2,827 rad/s²
\end{enumerate}

Acceleration torque: τ\textsubscript{accel} = J × α = 2.0 × 10⁻⁵ × 2,827
= \textbf{0.0565 N·m = 56.5 mN·m}

\begin{enumerate}
\def\labelenumi{(\alph{enumi})}
\setcounter{enumi}{3}
\tightlist
\item
  Adding friction estimate of 5 mN·m: τ\textsubscript{total} = 56.5 + 5
  = \textbf{61.5 mN·m}
\end{enumerate}

This is well within the 0.5 N·m holding torque, confirming the motor can
handle this motion profile.

\begin{center}\rule{0.5\linewidth}{0.5pt}\end{center}

\section{Problem 12.3.9}\label{problem-12.3.9}

\textbf{Given:} Two stepper motor driver ICs are compared for a
precision application: Driver A (A4988) with 16× max microstepping and
Driver B (TMC2209) with 256× max microstepping. Both drive a 200-step
motor coupled to a 1 mm pitch lead screw.

\textbf{Find:} (a) The linear resolution with each driver at maximum
microstepping, (b) the pulse frequency each requires for 10 mm/s travel,
and (c) the step rate improvement factor of Driver B over Driver A.

\textbf{Solution:}

\begin{enumerate}
\def\labelenumi{(\alph{enumi})}
\tightlist
\item
  Driver A (16× microstepping): Microsteps/rev = 200 × 16 = 3,200
  Resolution = 1 mm / 3,200 = 0.000313 mm = \textbf{0.313 μm}
\end{enumerate}

Driver B (256× microstepping): Microsteps/rev = 200 × 256 = 51,200
Resolution = 1 mm / 51,200 = 0.0000195 mm = \textbf{0.0195 μm = 19.5 nm}

\begin{enumerate}
\def\labelenumi{(\alph{enumi})}
\setcounter{enumi}{1}
\tightlist
\item
  At 10 mm/s: Revolutions/s = 10 / 1 = 10 rev/s
\end{enumerate}

Driver A: f = 10 × 3,200 = \textbf{32 kHz} Driver B: f = 10 × 51,200 =
\textbf{512 kHz}

\begin{enumerate}
\def\labelenumi{(\alph{enumi})}
\setcounter{enumi}{2}
\tightlist
\item
  Resolution improvement factor: 0.313 / 0.0195 = \textbf{16×} finer
  resolution
\end{enumerate}

Note: While 256× microstepping provides 16× finer position commands, the
actual positioning accuracy at this level is limited by mechanical
factors (backlash, lead screw accuracy, thermal effects) rather than the
driver resolution. Practical improvement in smoothness and noise
reduction is significant, but sub-micron accuracy requires closed-loop
feedback.

\begin{center}\rule{0.5\linewidth}{0.5pt}\end{center}

\section{Problem 12.3.10}\label{problem-12.3.10}

\textbf{Given:} A telescope mount uses two stepper motors for right
ascension (RA) and declination (Dec) tracking. The RA axis must track at
the sidereal rate (one revolution in 23 hours 56 minutes 4 seconds =
86,164 seconds). The motor has 200 steps/rev with 64× microstepping and
drives a 180:1 worm gear.

\textbf{Find:} (a) The angular resolution at the telescope axis, (b) the
step rate for sidereal tracking, (c) the tracking error per step in
arc-seconds, and (d) the time between microsteps.

\textbf{Solution:}

\begin{enumerate}
\def\labelenumi{(\alph{enumi})}
\tightlist
\item
  Microsteps per revolution at motor: 200 × 64 = 12,800 Microsteps per
  revolution at telescope axis: 12,800 × 180 = 2,304,000
\end{enumerate}

Angular resolution: θ = 360° / 2,304,000 = 0.000156° = \textbf{0.563
arc-seconds per microstep}

\begin{enumerate}
\def\labelenumi{(\alph{enumi})}
\setcounter{enumi}{1}
\item
  Sidereal rate: one revolution of telescope axis in 86,164 s. Step rate
  = 2,304,000 / 86,164 = \textbf{26.74 microsteps/s}
\item
  Tracking error per step: Each microstep moves the axis \textbf{0.563
  arc-seconds}, which is below the typical atmospheric seeing limit of
  1--2 arc-seconds.
\item
  Time between microsteps: Δt = 1 / 26.74 = \textbf{37.4 ms}
\end{enumerate}

This relatively slow step rate means the motor operates deep in the
pull-in torque region where full torque is available, and the motion
appears essentially continuous for astrophotography purposes.

\chapter{Chapter 12 --- Section 12.4: Motor
Control}\label{chapter-12-section-12.4-motor-control}

Practice problems covering variable frequency drives, servo systems,
soft starters, regenerative braking, field-oriented control, and direct
torque control.

\begin{center}\rule{0.5\linewidth}{0.5pt}\end{center}

\section{Problem 12.4.1}\label{problem-12.4.1}

\textbf{Given:} A 4-pole, 460 V, 60 Hz induction motor rated at 100 HP
drives a centrifugal fan. The motor must be operated at 40 Hz to reduce
airflow. The motor uses constant V/f control.

\textbf{Find:} (a) The new synchronous speed, (b) the voltage applied by
the VFD, (c) the approximate power savings (fan power varies as the cube
of speed), and (d) the annual energy cost savings at \$0.11/kWh running
7,000 hrs/yr.

\textbf{Solution:}

\begin{enumerate}
\def\labelenumi{(\alph{enumi})}
\item
  New synchronous speed: n\textsubscript{s(new)} = 120 × 40 / 4 =
  \textbf{1,200 RPM}
\item
  Constant V/f ratio: V/f = 460/60 = 7.667 V/Hz V\textsubscript{new} =
  7.667 × 40 = \textbf{306.7 V}
\item
  Speed ratio: n\textsubscript{new}/n\textsubscript{base} = 1,200/1,800
  = 0.667
\end{enumerate}

Power ratio (affinity law): P\textsubscript{new}/P\textsubscript{base} =
0.667³ = 0.2963

Power savings: ΔP = 100 × (1 − 0.2963) = \textbf{70.4 HP = 52.5 kW}

\begin{enumerate}
\def\labelenumi{(\alph{enumi})}
\setcounter{enumi}{3}
\tightlist
\item
  Annual energy savings: ΔE = 52.5 × 7,000 = 367,500 kWh
\end{enumerate}

Cost savings: ΔC = 367,500 × \$0.11 = \textbf{\$40,425/year}

\begin{center}\rule{0.5\linewidth}{0.5pt}\end{center}

\section{Problem 12.4.2}\label{problem-12.4.2}

\textbf{Given:} A servo motor with a 5,000-line incremental encoder
(quadrature decoded) drives a ball screw with a 10 mm pitch through a
2:1 gear reduction. The required positioning accuracy is ±10 μm.

\textbf{Find:} (a) The encoder resolution in counts per revolution, (b)
the linear resolution per count, and (c) whether the system meets the
accuracy requirement.

\textbf{Solution:}

\begin{enumerate}
\def\labelenumi{(\alph{enumi})}
\item
  Quadrature decoding: Counts/rev = 5,000 × 4 = \textbf{20,000
  counts/rev} at the motor shaft
\item
  With 2:1 gear reduction, counts per ball screw revolution: 20,000 × 2
  = 40,000 counts/rev at the output
\end{enumerate}

Linear resolution per count: d = 10 mm / 40,000 = 0.00025 mm =
\textbf{0.25 μm per count}

\begin{enumerate}
\def\labelenumi{(\alph{enumi})}
\setcounter{enumi}{2}
\tightlist
\item
  Since 0.25 μm \textless\textless{} ±10 μm, the encoder resolution is
  \textbf{40× finer} than the required accuracy. The system
  \textbf{easily meets the requirement}. Positioning accuracy will be
  limited by ball screw backlash (typically 5--20 μm), thermal
  expansion, and servo tuning rather than encoder resolution.
\end{enumerate}

\begin{center}\rule{0.5\linewidth}{0.5pt}\end{center}

\section{Problem 12.4.3}\label{problem-12.4.3}

\textbf{Given:} A 200 HP, 460 V, 3-phase induction motor has a full-load
current of I\textsubscript{FL} = 240 A and a DOL starting current of 7×
I\textsubscript{FL}. A soft starter is configured to limit starting
current to 3.5× I\textsubscript{FL} with a ramp time of 15 seconds.

\textbf{Find:} (a) The DOL inrush current, (b) the soft-start limited
current, (c) the initial voltage applied by the soft starter, and (d)
the initial starting torque as a percentage of DOL starting torque.

\textbf{Solution:}

\begin{enumerate}
\def\labelenumi{(\alph{enumi})}
\item
  DOL starting current: I\textsubscript{start(DOL)} = 7 × 240 =
  \textbf{1,680 A}
\item
  Soft-start limited current: I\textsubscript{start(soft)} = 3.5 × 240 =
  \textbf{840 A}
\item
  Initial voltage ratio: V\textsubscript{initial}/V\textsubscript{rated}
  = I\textsubscript{start(soft)}/I\textsubscript{start(DOL)} = 840/1,680
  = 0.50
\end{enumerate}

Initial voltage: V\textsubscript{initial} = 0.50 × 460 = \textbf{230 V}

\begin{enumerate}
\def\labelenumi{(\alph{enumi})}
\setcounter{enumi}{3}
\tightlist
\item
  Starting torque is proportional to voltage squared:
  T\textsubscript{initial}/T\textsubscript{DOL} = (0.50)² = 0.25 =
  \textbf{25\% of DOL starting torque}
\end{enumerate}

This is adequate for centrifugal fans and pumps but may be insufficient
for loaded conveyors or compressors.

\begin{center}\rule{0.5\linewidth}{0.5pt}\end{center}

\section{Problem 12.4.4}\label{problem-12.4.4}

\textbf{Given:} An electric bus with a mass of 12,000 kg decelerates
from 60 km/h (16.67 m/s) to a stop using regenerative braking. The
motor/generator efficiency during regeneration is 85\%, and the battery
charging efficiency is 92\%. The braking takes 12 seconds.

\textbf{Find:} (a) The kinetic energy available for recovery, (b) the
energy stored in the battery, (c) the overall recovery efficiency, and
(d) the average regenerative braking power.

\textbf{Solution:}

\begin{enumerate}
\def\labelenumi{(\alph{enumi})}
\item
  Kinetic energy: KE = ½mv² = 0.5 × 12,000 × 16.67² = 0.5 × 12,000 ×
  277.9 = \textbf{1,667,400 J = 1,667.4 kJ}
\item
  Energy stored in battery: E\textsubscript{stored} = KE ×
  η\textsubscript{motor} × η\textsubscript{battery} = 1,667.4 × 0.85 ×
  0.92 = \textbf{1,303.9 kJ}
\item
  Overall recovery efficiency: η\textsubscript{total} =
  E\textsubscript{stored} / KE = 1,303.9 / 1,667.4 = 0.782 =
  \textbf{78.2\%}
\item
  Average regenerative braking power: P\textsubscript{avg} = KE / t =
  1,667,400 / 12 = \textbf{138,950 W ≈ 139.0 kW}
\end{enumerate}

Average braking force: F = P\textsubscript{avg} / v\textsubscript{avg} =
138,950 / (16.67/2) = 138,950 / 8.33 = \textbf{16,681 N ≈ 16.7 kN}

\begin{center}\rule{0.5\linewidth}{0.5pt}\end{center}

\section{Problem 12.4.5}\label{problem-12.4.5}

\textbf{Given:} A 3-phase PMSM has: rated torque T\textsubscript{rated}
= 20 N·m, torque constant K\textsubscript{t} = 2.0 N·m/A, rated flux
linkage λ\textsubscript{m} = 0.30 Wb, stator resistance
R\textsubscript{s} = 0.3 Ω, L\textsubscript{d} = 5 mH,
L\textsubscript{q} = 10 mH, and 6 poles. The motor operates at 3,000 RPM
using FOC with I\textsubscript{d} = 0 control.

\textbf{Find:} (a) The required q-axis current for rated torque, (b) the
stator voltage magnitude, and (c) the minimum DC bus voltage for space
vector PWM.

\textbf{Solution:}

\begin{enumerate}
\def\labelenumi{(\alph{enumi})}
\tightlist
\item
  Required q-axis current: I\textsubscript{q} = T\textsubscript{rated} /
  K\textsubscript{t} = 20 / 2.0 = \textbf{10.0 A}
\end{enumerate}

With I\textsubscript{d} = 0: I\textsubscript{s} = 10.0 A

\begin{enumerate}
\def\labelenumi{(\alph{enumi})}
\setcounter{enumi}{1}
\tightlist
\item
  Electrical speed: ω\textsubscript{m} = 3,000 × 2π/60 = 314.2 rad/s
  ω\textsubscript{e} = (P/2) × ω\textsubscript{m} = 3 × 314.2 = 942.5
  rad/s
\end{enumerate}

Voltage equations in the dq frame: V\textsubscript{d} =
R\textsubscript{s} × I\textsubscript{d} − ω\textsubscript{e} ×
L\textsubscript{q} × I\textsubscript{q} = 0.3 × 0 − 942.5 × 0.010 × 10.0
= \textbf{−94.3 V}

V\textsubscript{q} = R\textsubscript{s} × I\textsubscript{q} +
ω\textsubscript{e} × L\textsubscript{d} × I\textsubscript{d} +
ω\textsubscript{e} × λ\textsubscript{m} = 0.3 × 10 + 0 + 942.5 × 0.30 =
3.0 + 282.8 = \textbf{285.8 V}

Stator voltage magnitude: V\textsubscript{s} = √(V\textsubscript{d}² +
V\textsubscript{q}²) = √(8,892 + 81,682) = √90,574 = \textbf{301.0 V}
(phase peak)

\begin{enumerate}
\def\labelenumi{(\alph{enumi})}
\setcounter{enumi}{2}
\tightlist
\item
  Minimum DC bus voltage: V\textsubscript{DC} = √3 × V\textsubscript{s}
  = 1.732 × 301.0 = \textbf{521.3 V}
\end{enumerate}

A \textbf{540 V} or higher DC bus is required.

\begin{center}\rule{0.5\linewidth}{0.5pt}\end{center}

\section{Problem 12.4.6}\label{problem-12.4.6}

\textbf{Given:} A DTC-controlled 4-pole induction motor has:
R\textsubscript{s} = 0.5 Ω, rated stator flux ψ\textsubscript{s,ref} =
0.90 Wb, DC bus voltage V\textsubscript{DC} = 600 V, flux hysteresis
band ±0.03 Wb, and torque hysteresis band ±3 N·m. At a given instant,
ψ\textsubscript{s} = 0.88 Wb and T\textsubscript{e} = 75 N·m, with
torque reference T\textsubscript{ref} = 80 N·m.

\textbf{Find:} (a) The flux error and comparator output, (b) the torque
error and comparator output, (c) the required inverter action, and (d)
the voltage vector magnitude.

\textbf{Solution:}

\begin{enumerate}
\def\labelenumi{(\alph{enumi})}
\tightlist
\item
  Flux error: Δψ = ψ\textsubscript{s,ref} − ψ\textsubscript{s} = 0.90 −
  0.88 = +0.02 Wb
\end{enumerate}

Since +0.02 Wb \textless{} +0.03 Wb (within band), the flux comparator
output is \textbf{0} (no correction needed). However, if we apply the
hysteresis logic where the output changes at the band edges: the flux is
below reference but within the band, so the output remains at its
previous state. Assuming it was previously set to +1 (increase flux):
\textbf{flux = +1}.

\begin{enumerate}
\def\labelenumi{(\alph{enumi})}
\setcounter{enumi}{1}
\tightlist
\item
  Torque error: ΔT = T\textsubscript{ref} − T\textsubscript{e} = 80 − 75
  = +5 N·m
\end{enumerate}

Since +5 N·m \textgreater{} +3 N·m (exceeds band), the torque comparator
output is \textbf{+1} (increase torque).

\begin{enumerate}
\def\labelenumi{(\alph{enumi})}
\setcounter{enumi}{2}
\item
  With flux = +1 and torque = +1, the switching table selects a voltage
  vector that advances the stator flux in the direction of rotation
  while increasing its magnitude.
\item
  Active voltage vector magnitude: V = V\textsubscript{DC} × √(2/3) =
  600 × 0.8165 = \textbf{489.9 V}
\end{enumerate}

\begin{center}\rule{0.5\linewidth}{0.5pt}\end{center}

\section{Problem 12.4.7}\label{problem-12.4.7}

\textbf{Given:} A VFD drives a 4-pole, 460 V, 60 Hz induction motor on a
constant-torque conveyor load. The motor must run at 900 RPM. The motor
rated torque at 60 Hz is 85 N·m.

\textbf{Find:} (a) The required VFD output frequency, (b) the V/f output
voltage, (c) the motor torque capability at this frequency with V/f
control, and (d) the motor output power.

\textbf{Solution:}

\begin{enumerate}
\def\labelenumi{(\alph{enumi})}
\item
  Synchronous speed at 60 Hz: n\textsubscript{s} = 1,800 RPM For 900 RPM
  synchronous speed: f = 900 × P / 120 = 900 × 4 / 120 = \textbf{30 Hz}
\item
  V/f ratio: V/f = 460/60 = 7.667 V/Hz V = 7.667 × 30 = \textbf{230 V}
\item
  With constant V/f control, the flux is maintained approximately
  constant, so the motor can deliver approximately rated torque: τ ≈
  \textbf{85 N·m} (at rated slip, the full torque is available at
  reduced speed)
\item
  Assuming the actual rotor speed is \textasciitilde870 RPM (with
  \textasciitilde3.3\% slip): ω = 870 × 2π/60 = 91.1 rad/s P = τ × ω =
  85 × 91.1 = \textbf{7,744 W = 7.74 kW} (10.4 HP)
\end{enumerate}

At half speed, the power is halved for a constant-torque load,
confirming the affinity laws do not apply here (those are for
variable-torque loads).

\begin{center}\rule{0.5\linewidth}{0.5pt}\end{center}

\section{Problem 12.4.8}\label{problem-12.4.8}

\textbf{Given:} A crane hoist uses a VFD with a braking resistor on the
DC bus. The motor is 50 HP, 460 V, 4-pole. When lowering a 5,000 kg load
at 0.5 m/s, the load drives the motor as a generator. The drum diameter
is 0.4 m and the gearbox ratio is 30:1. The DC bus voltage is 650 V and
the braking resistor activates at 720 V.

\textbf{Find:} (a) The gravitational power being regenerated, (b) the
motor speed during lowering, (c) the braking resistor power rating
required, and (d) the braking resistor value.

\textbf{Solution:}

\begin{enumerate}
\def\labelenumi{(\alph{enumi})}
\item
  Gravitational power: P\textsubscript{grav} = m × g × v = 5,000 × 9.81
  × 0.5 = \textbf{24,525 W = 24.5 kW}
\item
  Drum circumference: C = π × 0.4 = 1.257 m
\end{enumerate}

Drum speed: n\textsubscript{drum} = v / C × 60 = 0.5 / 1.257 × 60 =
23.86 RPM

Motor speed: n\textsubscript{motor} = n\textsubscript{drum} × gear ratio
= 23.86 × 30 = \textbf{715.8 RPM}

\begin{enumerate}
\def\labelenumi{(\alph{enumi})}
\setcounter{enumi}{2}
\tightlist
\item
  The braking resistor must absorb the regenerated power minus system
  losses. Assuming 15\% losses in motor and gearbox:
  P\textsubscript{brake} = P\textsubscript{grav} × 0.85 = 24,525 × 0.85
  = \textbf{20,846 W ≈ 21 kW}
\end{enumerate}

Select a braking resistor rated for at least \textbf{25 kW} with
appropriate duty cycle.

\begin{enumerate}
\def\labelenumi{(\alph{enumi})}
\setcounter{enumi}{3}
\tightlist
\item
  Braking resistor value at 720 V activation: R = V² / P = 720² / 25,000
  = 518,400 / 25,000 = \textbf{20.7 Ω}
\end{enumerate}

A standard \textbf{20 Ω} braking resistor would be selected.

\begin{center}\rule{0.5\linewidth}{0.5pt}\end{center}

\section{Problem 12.4.9}\label{problem-12.4.9}

\textbf{Given:} A sensorless FOC drive for a ceiling fan BLDC motor
estimates rotor position from back-EMF. The motor has 12 poles,
K\textsubscript{e} = 0.05 V/(rad/s), R\textsubscript{s} = 2.5 Ω, and
operates from a 24 V DC bus. The minimum detectable back-EMF for
reliable sensorless operation is 5 V.

\textbf{Find:} (a) The minimum speed for sensorless operation, (b) the
minimum speed in RPM, (c) the motor torque at this speed with 2 A phase
current and K\textsubscript{t} = 0.05 N·m/A, and (d) a strategy for
starting the motor.

\textbf{Solution:}

\begin{enumerate}
\def\labelenumi{(\alph{enumi})}
\item
  Minimum speed for detectable back-EMF: E\textsubscript{min} =
  K\textsubscript{e} × ω\textsubscript{min} ω\textsubscript{min} =
  E\textsubscript{min} / K\textsubscript{e} = 5 / 0.05 = \textbf{100
  rad/s}
\item
  Minimum speed in RPM: n = 100 × 60 / (2π) = \textbf{955 RPM}
\item
  Motor torque: τ = K\textsubscript{t} × I = 0.05 × 2 = \textbf{0.1 N·m}
\item
  Since the motor cannot detect rotor position below 955 RPM, starting
  strategies include:
\end{enumerate}

\begin{itemize}
\tightlist
\item
  \textbf{Open-loop forced commutation}: apply a rotating field at
  increasing frequency until the back-EMF is detectable
\item
  \textbf{High-frequency injection}: inject a high-frequency signal and
  detect the rotor position from saliency
\item
  \textbf{Align and go}: energize one phase to align the rotor to a
  known position, then begin the commutation sequence
\end{itemize}

For a ceiling fan, open-loop forced commutation is the simplest and most
cost-effective approach.

\begin{center}\rule{0.5\linewidth}{0.5pt}\end{center}

\section{Problem 12.4.10}\label{problem-12.4.10}

\textbf{Given:} A pump station has three identical VFD-driven pumps,
each rated at 30 kW. The system uses a cascading control strategy where
pumps are added or removed based on demand. The current demand requires
55 kW of pumping power. The VFDs have an efficiency of 97\% and the
motors have an efficiency of 93\%.

\textbf{Find:} (a) The optimal operating point (number of pumps and
individual speed), (b) the total electrical input power, (c) the
comparison with running two pumps at different speeds, and (d) the
energy savings versus throttle valve control.

\textbf{Solution:}

\begin{enumerate}
\def\labelenumi{(\alph{enumi})}
\tightlist
\item
  With 55 kW demand, two pumps each deliver 27.5 kW (91.7\% of rated).
  Each pump runs at: Speed ratio = (27.5/30)\textsuperscript{1/3} =
  0.917\textsuperscript{0.333} = 0.971 (97.1\% speed for cube-law load)
\end{enumerate}

This is close to full speed, so \textbf{two pumps at \textasciitilde97\%
speed} is optimal.

\begin{enumerate}
\def\labelenumi{(\alph{enumi})}
\setcounter{enumi}{1}
\tightlist
\item
  Electrical input per pump: P\textsubscript{elec} =
  P\textsubscript{mech} / (η\textsubscript{VFD} ×
  η\textsubscript{motor}) = 27,500 / (0.97 × 0.93) = 27,500 / 0.902 =
  30,488 W
\end{enumerate}

Total electrical input: P\textsubscript{total} = 2 × 30,488 =
\textbf{60,976 W = 61.0 kW}

\begin{enumerate}
\def\labelenumi{(\alph{enumi})}
\setcounter{enumi}{2}
\item
  Alternative: one pump at 100\% (30 kW) and one at 83.3\% (25 kW). The
  second pump speed ratio = (25/30)\textsuperscript{1/3} = 0.941
  (94.1\%). Electrical input = 30,000/0.902 + 25,000/0.902 = 33,260 +
  27,716 = 60,976 W. The result is essentially the same because the
  cube-law relationship means power scales directly.
\item
  With throttle valve control, both pumps run at full speed (30 kW each
  = 60 kW mechanical) with the excess 5 kW wasted as pressure drop
  across the valve: P\textsubscript{elec,throttle} = 60,000 / (0.93) =
  64,516 W (no VFD loss, but motor losses remain) VFD approach: 61.0 kW
  vs throttle: 64.5 kW Savings = 64.5 − 61.0 = \textbf{3.5 kW} (5.4\%
  savings)
\end{enumerate}

For a more dramatic example at lower demand (e.g., 20 kW), VFD savings
would be much larger due to the cube-law benefit.

\chapter{Chapter 12 --- Section 12.5: Motor
Specifications}\label{chapter-12-section-12.5-motor-specifications}

Practice problems covering nameplate data, motor protection, efficiency
standards, motor selection and sizing, bearings and vibration analysis,
and motor thermal modeling.

\begin{center}\rule{0.5\linewidth}{0.5pt}\end{center}

\section{Problem 12.5.1}\label{problem-12.5.1}

\textbf{Given:} A motor nameplate reads: 50 HP, 460 V, 60 Hz, 3-phase,
1,760 RPM, FLA = 62 A, SF = 1.15, Insulation Class F, TEFC, NEMA Design
B, Efficiency = 94.1\%.

\textbf{Find:} (a) The number of poles, (b) the full-load slip, (c) the
input power, (d) the power factor, and (e) the maximum continuous power
with service factor.

\textbf{Solution:}

\begin{enumerate}
\def\labelenumi{(\alph{enumi})}
\item
  Nearest synchronous speed above 1,760 is 1,800 RPM: P =
  120f/n\textsubscript{s} = 120 × 60 / 1,800 = \textbf{4 poles}
\item
  Full-load slip: s = (1,800 − 1,760) / 1,800 = 40/1,800 = 0.0222 =
  \textbf{2.22\%}
\item
  Input power: P\textsubscript{in} = P\textsubscript{out}/η = (50 × 746)
  / 0.941 = 37,300 / 0.941 = \textbf{39,639 W = 39.6 kW}
\item
  Power factor: PF = P\textsubscript{in} / (√3 × V × I) = 39,639 /
  (1.732 × 460 × 62) = 39,639 / 49,399 = \textbf{0.802 lagging}
\item
  Maximum continuous power with SF: P\textsubscript{max} = 50 × 1.15 =
  \textbf{57.5 HP}
\end{enumerate}

\begin{center}\rule{0.5\linewidth}{0.5pt}\end{center}

\section{Problem 12.5.2}\label{problem-12.5.2}

\textbf{Given:} A 75 HP, 460 V, 3-phase motor has FLA = 92 A, SF = 1.15,
and locked-rotor current of 6.5× FLA. Select protection per NEC
guidelines.

\textbf{Find:} (a) The thermal overload relay trip setting, (b) the
locked-rotor current, (c) the circuit breaker size per NEC 430.52, and
(d) the minimum conductor size.

\textbf{Solution:}

\begin{enumerate}
\def\labelenumi{(\alph{enumi})}
\item
  Overload relay trip setting at 115\% of FLA: I\textsubscript{trip} =
  1.15 × 92 = \textbf{105.8 A}
\item
  Locked-rotor current: I\textsubscript{LR} = 6.5 × 92 = \textbf{598 A}
\item
  Circuit breaker (NEC 430.52, inverse-time, Design B): Maximum = 250\%
  of FLA = 2.50 × 92 = 230 A Select next standard size: \textbf{225 A
  breaker}
\item
  Minimum conductor size (NEC 430.22): I\textsubscript{conductor} = 1.25
  × FLA = 1.25 × 92 = 115 A From NEC Table 310.16 at 75°C: \textbf{1/0
  AWG copper} (rated 150 A)
\end{enumerate}

\begin{center}\rule{0.5\linewidth}{0.5pt}\end{center}

\section{Problem 12.5.3}\label{problem-12.5.3}

\textbf{Given:} A plant operates a 55 kW, 4-pole, 400 V motor
continuously (8,760 hours/year) at 80\% load. The current IE2 motor has
90.5\% efficiency at 80\% load. A replacement IE4 motor has 95.0\%
efficiency at the same load point. Electricity costs \$0.12/kWh, and the
IE4 motor costs \$3,200 more.

\textbf{Find:} (a) The annual energy consumption of each motor, (b) the
annual savings, (c) the simple payback period, and (d) the 15-year total
savings.

\textbf{Solution:}

\begin{enumerate}
\def\labelenumi{(\alph{enumi})}
\tightlist
\item
  Mechanical output at 80\% load: P\textsubscript{mech} = 0.80 × 55 =
  44.0 kW
\end{enumerate}

IE2 input: P\textsubscript{in(IE2)} = 44.0 / 0.905 = 48.62 kW IE4 input:
P\textsubscript{in(IE4)} = 44.0 / 0.950 = 46.32 kW

Annual energy: E\textsubscript{IE2} = 48.62 × 8,760 = \textbf{425,911
kWh/year} E\textsubscript{IE4} = 46.32 × 8,760 = \textbf{405,763
kWh/year}

\begin{enumerate}
\def\labelenumi{(\alph{enumi})}
\setcounter{enumi}{1}
\item
  Annual savings: ΔE = 425,911 − 405,763 = 20,148 kWh ΔC = 20,148 ×
  \$0.12 = \textbf{\$2,418/year}
\item
  Simple payback: \$3,200 / \$2,418 = \textbf{1.32 years}
\item
  15-year total savings: 15 × \$2,418 = \textbf{\$36,270} (11.3× the
  premium cost)
\end{enumerate}

\begin{center}\rule{0.5\linewidth}{0.5pt}\end{center}

\section{Problem 12.5.4}\label{problem-12.5.4}

\textbf{Given:} A conveyor requires 15 kW of continuous mechanical power
at 900 RPM. The total reflected load inertia is J\textsubscript{load} =
1.5 kg·m², motor inertia is J\textsubscript{motor} = 0.08 kg·m², and the
system must accelerate from standstill to full speed in 5 seconds.

\textbf{Find:} (a) The continuous load torque, (b) the acceleration
torque, (c) the total torque during acceleration, and (d) whether an
18.5 kW (25 HP) motor is adequate.

\textbf{Solution:}

\begin{enumerate}
\def\labelenumi{(\alph{enumi})}
\item
  Continuous load torque: ω = 900 × 2π/60 = 94.25 rad/s
  τ\textsubscript{load} = P/ω = 15,000 / 94.25 = \textbf{159.2 N·m}
\item
  Angular acceleration: α = ω/t = 94.25 / 5 = 18.85 rad/s²
\end{enumerate}

Total inertia: J\textsubscript{total} = 0.08 + 1.50 = 1.58 kg·m²

Acceleration torque: τ\textsubscript{accel} = J\textsubscript{total} × α
= 1.58 × 18.85 = \textbf{29.8 N·m}

\begin{enumerate}
\def\labelenumi{(\alph{enumi})}
\setcounter{enumi}{2}
\item
  Total torque during acceleration: τ\textsubscript{total} = 159.2 +
  29.8 = \textbf{189.0 N·m}
\item
  An 18.5 kW, 6-pole motor at \textasciitilde940 RPM: Rated torque =
  18,500 / (940 × 2π/60) = 18,500 / 98.4 = 188.0 N·m Typical breakdown
  torque = 2.5 × 188.0 = 470 N·m
\end{enumerate}

The required acceleration torque of 189.0 N·m is approximately 100.5\%
of rated torque, essentially at the rating. The motor can deliver this,
but the \textbf{margin is very tight}. A 22 kW motor would provide a
more comfortable margin of \textasciitilde18\%.

\begin{center}\rule{0.5\linewidth}{0.5pt}\end{center}

\section{Problem 12.5.5}\label{problem-12.5.5}

\textbf{Given:} A 45 kW, 4-pole motor operates at 1,770 RPM with
6208-2RS deep groove ball bearings (9 balls, ball diameter
d\textsubscript{b} = 12.7 mm, pitch diameter d\textsubscript{p} = 48.0
mm, contact angle α = 0°). The dynamic load rating is C = 29.1 kN and
the radial load is P = 2.8 kN.

\textbf{Find:} (a) The L₁₀ bearing life in hours, (b) the BPFO (ball
pass frequency outer race), and (c) the BPFI (ball pass frequency inner
race).

\textbf{Solution:}

\begin{enumerate}
\def\labelenumi{(\alph{enumi})}
\tightlist
\item
  L₁₀ in revolutions: L₁₀ = (C/P)³ × 10⁶ = (29.1/2.8)³ × 10⁶ = (10.39)³
  × 10⁶ = 1,121 × 10⁶ rev
\end{enumerate}

L₁₀ in hours: L₁₀(hours) = 1,121 × 10⁶ / (60 × 1,770) = 1,121 × 10⁶ /
106,200 = \textbf{10,555 hours} ≈ 1.2 years continuous

\begin{enumerate}
\def\labelenumi{(\alph{enumi})}
\setcounter{enumi}{1}
\tightlist
\item
  Shaft frequency: f\textsubscript{shaft} = 1,770/60 = 29.5 Hz
\end{enumerate}

BPFO = (N\textsubscript{b}/2) × f\textsubscript{shaft} × (1 −
d\textsubscript{b}cos α / d\textsubscript{p}) BPFO = (9/2) × 29.5 × (1 −
12.7/48.0) = 4.5 × 29.5 × 0.7354 = \textbf{97.6 Hz}

\begin{enumerate}
\def\labelenumi{(\alph{enumi})}
\setcounter{enumi}{2}
\tightlist
\item
  BPFI = (N\textsubscript{b}/2) × f\textsubscript{shaft} × (1 +
  d\textsubscript{b}cos α / d\textsubscript{p}) BPFI = (9/2) × 29.5 × (1
  + 12.7/48.0) = 4.5 × 29.5 × 1.2646 = \textbf{167.9 Hz}
\end{enumerate}

A vibration peak at 97.6 Hz indicates an outer race defect; a peak at
167.9 Hz indicates an inner race defect.

\begin{center}\rule{0.5\linewidth}{0.5pt}\end{center}

\section{Problem 12.5.6}\label{problem-12.5.6}

\textbf{Given:} A Class F insulation motor (rated hot-spot 155°C, design
life 20,000 hours) operates in a 50°C ambient (10°C above standard 40°C
rating) at rated current. The rated temperature rise is 105°C (for 40°C
ambient).

\textbf{Find:} (a) The actual hot-spot temperature, (b) the temperature
above rated, and (c) the expected insulation life.

\textbf{Solution:}

\begin{enumerate}
\def\labelenumi{(\alph{enumi})}
\tightlist
\item
  At rated current, the temperature rise remains 105°C. With elevated
  ambient: T\textsubscript{actual} = T\textsubscript{ambient} + ΔT = 50
  + 105 = \textbf{155°C}
\end{enumerate}

Wait --- but the original rating assumed 40°C + 105°C rise + 10°C
hot-spot margin = 155°C. With 50°C ambient: T\textsubscript{actual} = 50
+ 105 = 155°C

The hot-spot margin is consumed by the elevated ambient. If we account
for the 10°C hot-spot allowance in the original design: Actual hot-spot
= 155°C (matching the rated limit exactly)

\begin{enumerate}
\def\labelenumi{(\alph{enumi})}
\setcounter{enumi}{1}
\tightlist
\item
  Temperature above rated: ΔT\textsubscript{excess} = 155 − 155 =
  \textbf{0°C} (at rated)
\end{enumerate}

But the original design had the 10°C margin built into the 155°C limit.
Operating at 50°C ambient consumes this margin entirely.

\begin{enumerate}
\def\labelenumi{(\alph{enumi})}
\setcounter{enumi}{2}
\tightlist
\item
  Expected life at exactly rated temperature: L = 20,000 ×
  2\textsuperscript{(155 − 155)/10} = 20,000 × 2⁰ = \textbf{20,000
  hours}
\end{enumerate}

However, this assumes perfect thermal conditions. In practice, the lost
hot-spot margin means any transient overload or temperature spike pushes
the winding above 155°C, accelerating degradation. The motor should be
\textbf{derated} for 50°C ambient to maintain the design life with
margin.

\begin{center}\rule{0.5\linewidth}{0.5pt}\end{center}

\section{Problem 12.5.7}\label{problem-12.5.7}

\textbf{Given:} A Class H insulation motor (rated hot-spot 180°C, design
life 20,000 hours) continuously carries 120\% of rated current. The
rated temperature rise at full current is 125°C in a 40°C ambient, with
15°C hot-spot margin.

\textbf{Find:} (a) The actual temperature rise (proportional to I²), (b)
the actual hot-spot temperature, (c) the expected insulation life.

\textbf{Solution:}

\begin{enumerate}
\def\labelenumi{(\alph{enumi})}
\item
  Actual temperature rise: ΔT = (I/I\textsubscript{rated})² ×
  ΔT\textsubscript{rated} = 1.20² × 125 = 1.44 × 125 = \textbf{180°C}
\item
  Actual hot-spot temperature: T\textsubscript{actual} = 40 + 180 =
  \textbf{220°C}
\item
  Temperature above rated hot-spot: ΔT\textsubscript{excess} = 220 − 180
  = 40°C
\end{enumerate}

Expected life: L = 20,000 × 2\textsuperscript{(180 − 220)/10} = 20,000 ×
2⁻⁴ = 20,000 / 16 = \textbf{1,250 hours}

A 20\% overload reduces insulation life from 20,000 hours to just 1,250
hours --- a \textbf{93.75\% reduction}. This dramatically illustrates
why sustained overloads are so destructive to motor insulation.

\begin{center}\rule{0.5\linewidth}{0.5pt}\end{center}

\section{Problem 12.5.8}\label{problem-12.5.8}

\textbf{Given:} A motor operates on an intermittent duty cycle (IEC S3):
ON for 4 minutes at 120\% rated current, OFF for 6 minutes, repeating.
The motor's continuous thermal rating is based on rated current.

\textbf{Find:} (a) The duty cycle percentage, (b) the RMS equivalent
current, (c) whether the motor is thermally adequate, and (d) the
equivalent continuous load as a percentage of rated.

\textbf{Solution:}

\begin{enumerate}
\def\labelenumi{(\alph{enumi})}
\item
  Duty cycle: DC = t\textsubscript{ON} / (t\textsubscript{ON} +
  t\textsubscript{OFF}) = 4 / (4 + 6) = 0.40 = \textbf{40\%}
\item
  RMS equivalent current: I\textsubscript{eq} = I\textsubscript{ON} ×
  √(t\textsubscript{ON} / (t\textsubscript{ON} + t\textsubscript{OFF}))
  = 1.20 × √(4/10) = 1.20 × √0.40 = 1.20 × 0.632 = \textbf{0.759 ×
  I\textsubscript{rated}}
\item
  Since I\textsubscript{eq} = 0.759 × I\textsubscript{rated} \textless{}
  1.0 × I\textsubscript{rated}, the motor is \textbf{thermally
  adequate}.
\item
  Equivalent continuous load: 75.9\% of rated current, which corresponds
  to approximately \textbf{75.9\%} of rated thermal loading.
\end{enumerate}

The motor has significant thermal margin despite the 120\% peak current
because the 60\% OFF time allows cooling.

\begin{center}\rule{0.5\linewidth}{0.5pt}\end{center}

\section{Problem 12.5.9}\label{problem-12.5.9}

\textbf{Given:} A 15 kW TEFC motor operates in two environments:
Environment A at 30°C ambient, and Environment B at 55°C ambient. The
motor has Class F insulation (155°C rated), designed for 40°C ambient
with a 105°C rise and 10°C hot-spot margin.

\textbf{Find:} (a) The available temperature rise in each environment,
(b) the derating factor for Environment B, and (c) the effective power
rating in Environment B.

\textbf{Solution:}

\begin{enumerate}
\def\labelenumi{(\alph{enumi})}
\item
  Available temperature rise: Environment A: ΔT\textsubscript{avail} =
  155 − 30 − 10 = \textbf{115°C} (more than rated 105°C --- no derating
  needed) Environment B: ΔT\textsubscript{avail} = 155 − 55 − 10 =
  \textbf{90°C}
\item
  Since temperature rise is proportional to I² and power is proportional
  to I²: Derating factor = √(ΔT\textsubscript{avail} /
  ΔT\textsubscript{rated}) = √(90 / 105) = √0.857 = \textbf{0.926}
\item
  Effective power rating in Environment B: P\textsubscript{derated} = 15
  × 0.926 = \textbf{13.9 kW}
\end{enumerate}

The motor must be derated by about 7.4\% for the high ambient
temperature to maintain the design insulation life.

\begin{center}\rule{0.5\linewidth}{0.5pt}\end{center}

\section{Problem 12.5.10}\label{problem-12.5.10}

\textbf{Given:} A motor vibration survey measures the following overall
velocity values (mm/s RMS) on a 30 kW motor (ISO 10816 Group 2): bearing
housing DE = 3.2 mm/s, bearing housing NDE = 2.1 mm/s. The previous
survey 6 months ago measured 1.8 mm/s and 1.5 mm/s respectively.

\textbf{Find:} (a) The ISO 10816 zone classification for each bearing,
(b) the rate of vibration increase, (c) the estimated time to reach the
Zone C/D boundary, and (d) the recommended action.

\textbf{Solution:}

\begin{enumerate}
\def\labelenumi{(\alph{enumi})}
\tightlist
\item
  ISO 10816 Group 2 (15--75 kW) zones:
\end{enumerate}

\begin{itemize}
\tightlist
\item
  Zone A: \textless{} 1.8 mm/s (new)
\item
  Zone B: 1.8--4.5 mm/s (acceptable)
\item
  Zone C: 4.5--11.2 mm/s (marginal)
\item
  Zone D: \textgreater{} 11.2 mm/s (danger)
\end{itemize}

DE bearing: 3.2 mm/s → \textbf{Zone B} (acceptable) NDE bearing: 2.1
mm/s → \textbf{Zone B} (acceptable)

\begin{enumerate}
\def\labelenumi{(\alph{enumi})}
\setcounter{enumi}{1}
\item
  Rate of vibration increase (DE bearing): Δv/Δt = (3.2 − 1.8) / 6
  months = 1.4 / 6 = \textbf{0.233 mm/s per month}
\item
  Time to reach Zone C/D boundary (11.2 mm/s) from current DE value: t =
  (11.2 − 3.2) / 0.233 = 8.0 / 0.233 = \textbf{34.3 months} at linear
  trend
\end{enumerate}

Time to reach Zone B/C boundary (4.5 mm/s): t = (4.5 − 3.2) / 0.233 =
1.3 / 0.233 = \textbf{5.6 months}

\begin{enumerate}
\def\labelenumi{(\alph{enumi})}
\setcounter{enumi}{3}
\tightlist
\item
  Recommended actions:
\end{enumerate}

\begin{itemize}
\tightlist
\item
  \textbf{Increase monitoring frequency} from 6 months to monthly for
  the DE bearing
\item
  \textbf{Perform spectral analysis} to identify the vibration source
  (bearing defect, misalignment, imbalance)
\item
  \textbf{Plan bearing replacement} within the next 4--5 months before
  reaching Zone C
\item
  Check for root cause: misalignment, soft foot, belt tension, or
  bearing lubrication issues
\end{itemize}

\chapter{Chapter 13 --- Section 13.1: Ideal Op-Amp
Model}\label{chapter-13-section-13.1-ideal-op-amp-model}

Practice problems covering ideal op-amp characteristics, golden rules,
open-loop vs.~closed-loop gain, gain-bandwidth product, and feedback
fraction analysis.

\begin{center}\rule{0.5\linewidth}{0.5pt}\end{center}

\section{Problem 13.1.1}\label{problem-13.1.1}

\textbf{Given:} An op-amp circuit has the non-inverting input connected
to a 1.8 V reference. A feedback network connects the output through a
47 kΩ resistor to the inverting input, with a 10 kΩ resistor from the
inverting input to ground.

\textbf{Find:} Using the ideal op-amp golden rules, determine the
voltage at the inverting input, the current into the inverting input
terminal, and the output voltage.

\textbf{Solution:} By the first golden rule (virtual short):
V\textsuperscript{-} = V\textsuperscript{+} = 1.8 V. By the second
golden rule (zero input current): I\textsuperscript{-} = 0 A.

Since V\textsuperscript{-} = 1.8 V and the 10 kΩ resistor connects to
ground: Current through R₁ = V\textsuperscript{-} / R₁ = 1.8 / 10,000 =
180 μA.

By the zero-input-current rule, this same 180 μA must flow through
R\textsubscript{f} = 47 kΩ from the output: V\textsubscript{out} =
V\textsuperscript{-} + I × R\textsubscript{f} = 1.8 + 180 × 10⁻⁶ ×
47,000 = 1.8 + 8.46 = 10.26 V.

Gain check: A\textsubscript{v} = 1 + R\textsubscript{f}/R₁ = 1 +
47,000/10,000 = 5.7. V\textsubscript{out} = 5.7 × 1.8 = \textbf{10.26
V}.

\begin{center}\rule{0.5\linewidth}{0.5pt}\end{center}

\section{Problem 13.1.2}\label{problem-13.1.2}

\textbf{Given:} An op-amp has an open-loop gain of A\textsubscript{OL} =
500,000 and a gain-bandwidth product of GBW = 8 MHz. It is configured
with a feedback fraction β = 0.02.

\textbf{Find:} The ideal closed-loop gain, the actual closed-loop gain,
the gain error, and the closed-loop bandwidth.

\textbf{Solution:} Ideal closed-loop gain: A\textsubscript{CL(ideal)} =
1/β = 1/0.02 = 50 (34 dB).

Actual closed-loop gain: A\textsubscript{CL} = A\textsubscript{OL} / (1
+ A\textsubscript{OL} × β) = 500,000 / (1 + 500,000 × 0.02) = 500,000 /
10,001 = 49.995.

Gain error: Error = (50 - 49.995) / 50 × 100\% = \textbf{0.01\%}.

Closed-loop bandwidth: f\textsubscript{3dB} = GBW / A\textsubscript{CL}
= 8,000,000 / 50 = \textbf{160 kHz}.

\begin{center}\rule{0.5\linewidth}{0.5pt}\end{center}

\section{Problem 13.1.3}\label{problem-13.1.3}

\textbf{Given:} An op-amp with GBW = 4 MHz is used in two
configurations: - Configuration A: A\textsubscript{CL} = 10 -
Configuration B: A\textsubscript{CL} = 200

\textbf{Find:} The closed-loop bandwidth for each configuration and the
frequency at which Configuration B's gain drops to unity (0 dB).

\textbf{Solution:} Configuration A: f\textsubscript{3dB} = GBW /
A\textsubscript{CL} = 4,000,000 / 10 = \textbf{400 kHz}.

Configuration B: f\textsubscript{3dB} = GBW / A\textsubscript{CL} =
4,000,000 / 200 = \textbf{20 kHz}.

The unity-gain frequency for the op-amp (where
\textbar A\textsubscript{OL}\textbar{} = 1) is approximately equal to
the GBW: f\textsubscript{unity} = \textbf{4 MHz}.

This is the same for both configurations because GBW is constant. At 4
MHz, the open-loop gain equals 1, so no closed-loop configuration can
provide gain above this frequency.

\begin{center}\rule{0.5\linewidth}{0.5pt}\end{center}

\section{Problem 13.1.4}\label{problem-13.1.4}

\textbf{Given:} A non-inverting amplifier requires a closed-loop gain of
exactly 25.00 with an error of less than 0.1\%. The feedback fraction is
β = 1/25 = 0.04.

\textbf{Find:} The minimum open-loop gain A\textsubscript{OL} required
to achieve less than 0.1\% gain error.

\textbf{Solution:} The gain error is given by: Error = 1 / (1 +
A\textsubscript{OL} × β) × 100\%.

For Error \textless{} 0.1\%: 1 / (1 + A\textsubscript{OL} × 0.04)
\textless{} 0.001.

Solving: 1 + A\textsubscript{OL} × 0.04 \textgreater{} 1000.
A\textsubscript{OL} × 0.04 \textgreater{} 999. A\textsubscript{OL}
\textgreater{} 24,975.

Minimum open-loop gain: \textbf{A\textsubscript{OL} = 25,000}
(approximately 88 dB).

Verification: A\textsubscript{CL} = 25,000 / (1 + 25,000 × 0.04) =
25,000 / 1,001 = 24.975. Error = (25 - 24.975) / 25 × 100\% = 0.10\%.

\begin{center}\rule{0.5\linewidth}{0.5pt}\end{center}

\section{Problem 13.1.5}\label{problem-13.1.5}

\textbf{Given:} An op-amp with GBW = 10 MHz and A\textsubscript{OL(DC)}
= 120 dB is configured as a non-inverting amplifier with a gain of 40.
The input signal is a 100 kHz sinusoid with an amplitude of 50 mV.

\textbf{Find:} Whether the amplifier can faithfully reproduce the signal
at 100 kHz, and the actual gain at 100 kHz.

\textbf{Solution:} Closed-loop bandwidth: f\textsubscript{3dB} = GBW /
A\textsubscript{CL} = 10,000,000 / 40 = 250 kHz.

At f = 100 kHz, the ratio f/f\textsubscript{3dB} = 100/250 = 0.4.

Gain at 100 kHz: \textbar A(f)\textbar{} = A\textsubscript{CL} / √(1 +
(f/f\textsubscript{3dB})²) = 40 / √(1 + 0.16) = 40 / √1.16 = 40 / 1.077
= \textbf{37.14}.

Gain reduction = 40 - 37.14 = 2.86 (7.15\% reduction). In dB:
20log₁₀(37.14/40) = -0.64 dB.

Output amplitude = 37.14 × 50 mV = \textbf{1.857 V}.

The signal is within the amplifier's bandwidth (100 kHz \textless{} 250
kHz), so the amplifier can reproduce it with only 0.64 dB of attenuation
relative to the passband gain.

\chapter{Chapter 13 --- Section 13.2: Inverting
Configurations}\label{chapter-13-section-13.2-inverting-configurations}

Practice problems covering inverting amplifiers, summing amplifiers,
integrators, differentiators, logarithmic amplifiers, and precision
rectifiers.

\begin{center}\rule{0.5\linewidth}{0.5pt}\end{center}

\section{Problem 13.2.1}\label{problem-13.2.1}

\textbf{Given:} An inverting amplifier must provide a gain of -50 with
an input impedance of at least 10 kΩ. The available feedback resistor is
R\textsubscript{f} = 510 kΩ. The input signal is V\textsubscript{in} =
150 mV DC.

\textbf{Find:} The required R\textsubscript{in}, the output voltage, and
the current through R\textsubscript{f}.

\textbf{Solution:} Gain: A\textsubscript{v} = -R\textsubscript{f} /
R\textsubscript{in}. R\textsubscript{in} = R\textsubscript{f} /
\textbar A\textsubscript{v}\textbar{} = 510,000 / 50 = 10,200 Ω =
\textbf{10.2 kΩ}.

This satisfies the ≥ 10 kΩ input impedance requirement
(Z\textsubscript{in} = R\textsubscript{in} = 10.2 kΩ for inverting
configuration).

Output voltage: V\textsubscript{out} = A\textsubscript{v} ×
V\textsubscript{in} = -50 × 0.150 = \textbf{-7.5 V}.

Current through R\textsubscript{in} (inverting input is at virtual
ground): I = V\textsubscript{in} / R\textsubscript{in} = 0.150 / 10,200
= 14.7 μA.

By the golden rule, the same current flows through R\textsubscript{f}:
I\textsubscript{f} = \textbf{14.7 μA}.

Verification: V\textsubscript{out} = 0 - I\textsubscript{f} ×
R\textsubscript{f} = 0 - 14.7 × 10⁻⁶ × 510,000 = -7.5 V.

\begin{center}\rule{0.5\linewidth}{0.5pt}\end{center}

\section{Problem 13.2.2}\label{problem-13.2.2}

\textbf{Given:} A summing amplifier has R\textsubscript{f} = 100 kΩ and
four inputs: V₁ = 2.0 V through R₁ = 20 kΩ, V₂ = -1.5 V through R₂ = 50
kΩ, V₃ = 0.8 V through R₃ = 10 kΩ, and V₄ = 3.0 V through R₄ = 100 kΩ.

\textbf{Find:} The output voltage and the individual contribution from
each input.

\textbf{Solution:} V\textsubscript{out} = -R\textsubscript{f} × (V₁/R₁ +
V₂/R₂ + V₃/R₃ + V₄/R₄).

Individual contributions: - V₁: -R\textsubscript{f} × V₁/R₁ = -100,000 ×
2.0/20,000 = -100,000 × 0.0001 = \textbf{-10.0 V} - V₂:
-R\textsubscript{f} × V₂/R₂ = -100,000 × (-1.5)/50,000 = -100,000 ×
(-0.00003) = \textbf{+3.0 V} - V₃: -R\textsubscript{f} × V₃/R₃ =
-100,000 × 0.8/10,000 = -100,000 × 0.00008 = \textbf{-8.0 V} - V₄:
-R\textsubscript{f} × V₄/R₄ = -100,000 × 3.0/100,000 = -100,000 ×
0.00003 = \textbf{-3.0 V}

V\textsubscript{out} = -10.0 + 3.0 + (-8.0) + (-3.0) = \textbf{-18.0 V}.

Note: This output would clip at the negative supply rail if the op-amp
is powered from ±15 V supplies (V\textsubscript{out} limited to
approximately -13 V). The circuit would need to be redesigned with lower
gain ratios or smaller input signals.

\begin{center}\rule{0.5\linewidth}{0.5pt}\end{center}

\section{Problem 13.2.3}\label{problem-13.2.3}

\textbf{Given:} An op-amp integrator has R = 22 kΩ and C = 47 nF. A
constant input of V\textsubscript{in} = +3.5 V is applied starting at t
= 0 with the capacitor initially discharged. The op-amp is powered from
±15 V supplies.

\textbf{Find:} The output voltage at t = 0.5 ms, t = 1 ms, and t = 2 ms.
Determine when the output reaches the negative supply rail.

\textbf{Solution:} Time constant: τ = RC = 22,000 × 47 × 10⁻⁹ = 1.034
ms.

For a constant input: V\textsubscript{out}(t) = -(1/RC) ×
V\textsubscript{in} × t = -(1/0.001034) × 3.5 × t = -3,385 × t.

At t = 0.5 ms: V\textsubscript{out} = -3,385 × 0.0005 = \textbf{-1.69
V}. At t = 1 ms: V\textsubscript{out} = -3,385 × 0.001 = \textbf{-3.39
V}. At t = 2 ms: V\textsubscript{out} = -3,385 × 0.002 = \textbf{-6.77
V}.

Ramp rate: dV\textsubscript{out}/dt = -3,385 V/s = -3.39 V/ms.

Output reaches -15 V (negative rail) when: -15 = -3,385 × t. t = 15 /
3,385 = \textbf{4.43 ms}.

After 4.43 ms, the output saturates at approximately -13.5 V (allowing
for the output stage saturation voltage).

\begin{center}\rule{0.5\linewidth}{0.5pt}\end{center}

\section{Problem 13.2.4}\label{problem-13.2.4}

\textbf{Given:} A differentiator circuit has C = 22 nF and
R\textsubscript{f} = 47 kΩ. The input is a sawtooth wave that ramps
linearly from 0 V to 5 V in 2 ms, then resets instantly to 0 V.

\textbf{Find:} The output voltage during the ramp portion and the time
constant R\textsubscript{f}C.

\textbf{Solution:} Time constant: R\textsubscript{f} × C = 47,000 × 22 ×
10⁻⁹ = 1.034 ms.

During the ramp (linear increase): dV\textsubscript{in}/dt = 5.0 V /
0.002 s = 2,500 V/s.

V\textsubscript{out} = -R\textsubscript{f} × C × dV\textsubscript{in}/dt
= -1.034 × 10⁻³ × 2,500 = \textbf{-2.585 V} (constant during the ramp).

During the reset (instantaneous drop from 5 V to 0 V):
dV\textsubscript{in}/dt → -∞ theoretically, producing a large positive
spike at the output. In practice, the spike amplitude is limited by the
op-amp slew rate and supply rails.

The output is a constant -2.585 V during each ramp, with a brief
positive spike at each reset.

\begin{center}\rule{0.5\linewidth}{0.5pt}\end{center}

\section{Problem 13.2.5}\label{problem-13.2.5}

\textbf{Given:} A logarithmic amplifier uses a matched transistor pair
for temperature compensation. The circuit is configured with
R\textsubscript{in} = 10 kΩ and a reference current set by
V\textsubscript{ref} = 1.0 V through R\textsubscript{ref} = 10 kΩ
(I\textsubscript{ref} = 100 μA). The operating temperature is 37°C. The
input voltage varies from 1 mV to 10 V.

\textbf{Find:} The output voltage for V\textsubscript{in} = 1 mV, 10 mV,
100 mV, 1 V, and 10 V, and the output change per decade.

\textbf{Solution:} At 37°C (310 K): kT/q = (1.381 × 10⁻²³ × 310) /
(1.602 × 10⁻¹⁹) = 26.72 mV.

With temperature-compensated configuration: V\textsubscript{out} =
-(kT/q) × ln(V\textsubscript{in} / V\textsubscript{ref}) = -0.02672 ×
ln(V\textsubscript{in} / 1.0).

For V\textsubscript{in} = 1 mV: V\textsubscript{out} = -0.02672 ×
ln(0.001) = -0.02672 × (-6.908) = \textbf{+0.1845 V}. For
V\textsubscript{in} = 10 mV: V\textsubscript{out} = -0.02672 × ln(0.01)
= -0.02672 × (-4.605) = \textbf{+0.1230 V}. For V\textsubscript{in} =
100 mV: V\textsubscript{out} = -0.02672 × ln(0.1) = -0.02672 × (-2.303)
= \textbf{+0.0615 V}. For V\textsubscript{in} = 1 V:
V\textsubscript{out} = -0.02672 × ln(1.0) = -0.02672 × 0 = \textbf{0.000
V}. For V\textsubscript{in} = 10 V: V\textsubscript{out} = -0.02672 ×
ln(10) = -0.02672 × 2.303 = \textbf{-0.0615 V}.

Change per decade: ΔV\textsubscript{out} = -(kT/q) × ln(10) = -0.02672 ×
2.303 = \textbf{-61.5 mV/decade} at 37°C.

This is slightly higher than the 59.5 mV/decade at 25°C due to the
temperature dependence of kT/q.

\begin{center}\rule{0.5\linewidth}{0.5pt}\end{center}

\section{Problem 13.2.6}\label{problem-13.2.6}

\textbf{Given:} A precision half-wave rectifier uses an op-amp with
A\textsubscript{OL} = 200,000, SR = 5 V/μs, and ±12 V supplies. The
input is a 50 mV\textsubscript{peak}, 1 kHz sine wave. The gain is set
to R\textsubscript{f}/R\textsubscript{in} = 10 (R\textsubscript{in} = 10
kΩ, R\textsubscript{f} = 100 kΩ).

\textbf{Find:} The effective diode threshold, the peak output voltage,
the DC (average) output, and the maximum operating frequency.

\textbf{Solution:} Effective diode threshold: V\textsubscript{D(eff)} =
V\textsubscript{F} / A\textsubscript{OL} = 0.6 / 200,000 = \textbf{3 μV}
(negligible compared to 50 mV input).

Peak output voltage (during positive half-cycle for inverting
configuration): \textbar V\textsubscript{out(peak)}\textbar{} =
(R\textsubscript{f}/R\textsubscript{in}) × V\textsubscript{in(peak)} =
10 × 50 mV = \textbf{500 mV}.

DC (average) value of half-wave rectified sine: V\textsubscript{dc} =
V\textsubscript{out(peak)} / π = 500 / π = \textbf{159.2 mV}.

Maximum operating frequency: Recovery time when diode turns off:
t\textsubscript{rec} = 2V\textsubscript{sat} / SR = 24 / (5 × 10⁶) = 4.8
μs. For less than 5\% distortion: t\textsubscript{rec} \textless{} 0.05
× T/2. T/2 \textgreater{} 4.8 μs / 0.05 = 96 μs. f\textsubscript{max}
\textless{} 1 / (2 × 96 μs) = \textbf{5.21 kHz}.

At 1 kHz (T/2 = 500 μs), t\textsubscript{rec}/T/2 = 4.8/500 = 0.96\%, so
distortion is negligible at this frequency.

\begin{center}\rule{0.5\linewidth}{0.5pt}\end{center}

\section{Problem 13.2.7}\label{problem-13.2.7}

\textbf{Given:} A 3-input summing amplifier is used as a simple DAC. The
inputs are digital signals (0 V or 5 V) representing a 3-bit binary
number (V₁ = MSB, V₃ = LSB). R\textsubscript{f} = 10 kΩ, R₁ = 10 kΩ
(MSB, weight 4), R₂ = 20 kΩ (weight 2), R₃ = 40 kΩ (LSB, weight 1).

\textbf{Find:} The output voltage for digital input 101 (V₁ = 5 V, V₂ =
0 V, V₃ = 5 V) and for input 110 (V₁ = 5 V, V₂ = 5 V, V₃ = 0 V).

\textbf{Solution:} For input 101 (V₁ = 5 V, V₂ = 0 V, V₃ = 5 V):
V\textsubscript{out} = -R\textsubscript{f} × (V₁/R₁ + V₂/R₂ + V₃/R₃)
V\textsubscript{out} = -10,000 × (5/10,000 + 0/20,000 + 5/40,000)
V\textsubscript{out} = -10,000 × (0.0005 + 0 + 0.000125)
V\textsubscript{out} = -10,000 × 0.000625 = \textbf{-6.25 V}.

Digital value 101 = 5 in decimal, and 5/7 × 8.75 = 6.25 V.

For input 110 (V₁ = 5 V, V₂ = 5 V, V₃ = 0 V): V\textsubscript{out} =
-10,000 × (5/10,000 + 5/20,000 + 0/40,000) V\textsubscript{out} =
-10,000 × (0.0005 + 0.00025 + 0) V\textsubscript{out} = -10,000 ×
0.00075 = \textbf{-7.50 V}.

Digital value 110 = 6 in decimal, and 6/7 × 8.75 = 7.50 V.

The LSB voltage step is: R\textsubscript{f}/R₃ × 5 = (10,000/40,000) × 5
= \textbf{1.25 V}.

\begin{center}\rule{0.5\linewidth}{0.5pt}\end{center}

\section{Problem 13.2.8}\label{problem-13.2.8}

\textbf{Given:} An integrator with R = 15 kΩ and C = 220 nF receives a 2
kHz square wave input alternating between +1 V and -1 V. The capacitor
is initially discharged.

\textbf{Find:} The peak-to-peak amplitude of the resulting triangular
wave output.

\textbf{Solution:} Time constant: RC = 15,000 × 220 × 10⁻⁹ = 3.3 ms.

The square wave has a half-period of T/2 = 1/(2 × 2,000) = 0.25 ms.

During the positive half-cycle (V\textsubscript{in} = +1 V):
V\textsubscript{out} ramps from some voltage at rate = -(1/RC) ×
V\textsubscript{in} = -(1/0.0033) × 1 = -303 V/s.

Voltage change per half-cycle: ΔV = \textbar-303\textbar{} × 0.00025 =
0.0758 V = 75.8 mV.

The output is a triangular wave with peak-to-peak amplitude:
V\textsubscript{pp} = ΔV = \textbf{75.8 mV}.

The triangular wave oscillates symmetrically around 0 V, swinging from
-37.9 mV to +37.9 mV (assuming the integrator reaches steady state and
the capacitor has zero average charge).

Since T/2 \textless\textless{} RC (0.25 ms \textless\textless{} 3.3 ms),
the ramp is very linear and the triangular wave has excellent linearity.

\chapter{Chapter 13 --- Section 13.3: Non-Inverting
Configurations}\label{chapter-13-section-13.3-non-inverting-configurations}

Practice problems covering non-inverting amplifiers, voltage followers,
transimpedance amplifiers, and programmable gain amplifiers.

\begin{center}\rule{0.5\linewidth}{0.5pt}\end{center}

\section{Problem 13.3.1}\label{problem-13.3.1}

\textbf{Given:} A non-inverting amplifier uses R₁ = 2.2 kΩ and
R\textsubscript{f} = 68 kΩ. The input signal is a 200
mV\textsubscript{peak}, 5 kHz sine wave from a 50 kΩ source impedance
sensor.

\textbf{Find:} The closed-loop gain, the output voltage amplitude, and
why this configuration is preferable over an inverting amplifier for
this application.

\textbf{Solution:} Closed-loop gain: A\textsubscript{v} = 1 +
R\textsubscript{f}/R₁ = 1 + 68,000/2,200 = 1 + 30.91 = \textbf{31.91}.

Output amplitude: V\textsubscript{out(peak)} = 31.91 × 200 mV =
\textbf{6.38 V}.

Input impedance comparison: - Non-inverting: Z\textsubscript{in} ≈
Z\textsubscript{in(op-amp)} ≈ 10⁹ Ω (negligible loading on 50 kΩ
source). - Equivalent inverting amplifier with gain of -31.9:
Z\textsubscript{in} = R\textsubscript{in}. To achieve gain = -31.9 with
R\textsubscript{f} = 68 kΩ: R\textsubscript{in} = 68,000/31.9 = 2,131 Ω
= 2.13 kΩ. The 50 kΩ source would form a voltage divider with
R\textsubscript{in}, reducing the effective input to 200 × 2,131/(50,000
+ 2,131) = \textbf{8.2 mV} (a 96\% signal loss).

The non-inverting configuration preserves the full 200 mV sensor signal,
making it the correct choice for high-impedance sources.

\begin{center}\rule{0.5\linewidth}{0.5pt}\end{center}

\section{Problem 13.3.2}\label{problem-13.3.2}

\textbf{Given:} A precision voltage reference produces 4.096 V through a
resistive divider using two 47 kΩ resistors. The divided voltage must
drive a 500 Ω load. A voltage follower buffer is inserted between the
divider and the load.

\textbf{Find:} The output voltage with and without the buffer, and the
percentage error without the buffer.

\textbf{Solution:} Nominal divider output: V\textsubscript{div} = 4.096
× 47,000 / (47,000 + 47,000) = 4.096 / 2 = \textbf{2.048 V}.

Without buffer (500 Ω load in parallel with lower 47 kΩ):
R\textsubscript{parallel} = (47,000 × 500) / (47,000 + 500) = 23,500,000
/ 47,500 = 494.7 Ω. V\textsubscript{out} = 4.096 × 494.7 / (47,000 +
494.7) = 4.096 × 494.7 / 47,494.7 = \textbf{0.04268 V}.

Percentage error = (2.048 - 0.04268) / 2.048 × 100\% = \textbf{97.9\%}
-- the divider is completely loaded down.

With buffer: V\textsubscript{out} = 2.048 V (buffer output equals input
with unity gain). The buffer's output impedance is ≈ 0 Ω, easily driving
the 500 Ω load. Error ≈ \textbf{0\%} (limited only by the buffer's input
offset voltage, typically \textless{} 1 mV).

\begin{center}\rule{0.5\linewidth}{0.5pt}\end{center}

\section{Problem 13.3.3}\label{problem-13.3.3}

\textbf{Given:} A transimpedance amplifier converts the photocurrent
from a PIN photodiode. The photodiode has a capacitance
C\textsubscript{in} = 15 pF and produces a photocurrent of 2 μA at the
operating light level. R\textsubscript{f} = 470 kΩ, and the op-amp GBW =
5 MHz.

\textbf{Find:} The output voltage, the optimal feedback capacitance
C\textsubscript{f} for stability, and the signal bandwidth.

\textbf{Solution:} Output voltage: V\textsubscript{out} =
-I\textsubscript{in} × R\textsubscript{f} = -2 × 10⁻⁶ × 470,000 =
\textbf{-0.94 V}.

Optimal feedback capacitance: C\textsubscript{f} = √(C\textsubscript{in}
/ (2π × R\textsubscript{f} × GBW)) C\textsubscript{f} = √(15 × 10⁻¹² /
(2π × 470,000 × 5 × 10⁶)) C\textsubscript{f} = √(15 × 10⁻¹² / 1.477 ×
10¹³) C\textsubscript{f} = √(1.016 × 10⁻²⁴) = 1.008 × 10⁻¹² F ≈
\textbf{1.0 pF}.

Signal bandwidth: f\textsubscript{3dB} = 1 /
(2πR\textsubscript{f}C\textsubscript{f}) = 1 / (2π × 470,000 × 1.0 ×
10⁻¹²) = 1 / (2.953 × 10⁻⁶) = \textbf{338.6 kHz}.

\begin{center}\rule{0.5\linewidth}{0.5pt}\end{center}

\section{Problem 13.3.4}\label{problem-13.3.4}

\textbf{Given:} A 14-bit SAR ADC with a ±2.5 V input range (5 V span)
has an input-referred noise of 80 μV\textsubscript{rms}. A PGA with
gains of 1, 2, 5, 10, 20, and 50 precedes the ADC. A thermocouple
produces a 2 mV signal.

\textbf{Find:} The effective number of bits (ENOB) at gains of 1, 10,
and 50, and the optimal gain setting.

\textbf{Solution:} ADC LSB = 5 V / 2¹⁴ = 5 / 16,384 = 305.2 μV.

At PGA gain = 1: Signal at ADC = 2 mV (6.6 LSBs). Noise referred to
input = 80 μV. SNR = 20log₁₀(2 mV / 80 μV) = 20log₁₀(25) = 28.0 dB. ENOB
= (28.0 - 1.76) / 6.02 = \textbf{4.4 bits}.

At PGA gain = 10: Signal at ADC = 20 mV (65.5 LSBs). Noise referred to
input = 80 μV / 10 = 8 μV. SNR = 20log₁₀(2 mV / 8 μV) = 20log₁₀(250) =
48.0 dB. ENOB = (48.0 - 1.76) / 6.02 = \textbf{7.7 bits}.

At PGA gain = 50: Signal at ADC = 100 mV (327.6 LSBs). Noise referred to
input = 80 μV / 50 = 1.6 μV. SNR = 20log₁₀(2 mV / 1.6 μV) =
20log₁₀(1,250) = 61.9 dB. ENOB = (61.9 - 1.76) / 6.02 = \textbf{10.0
bits}.

\textbf{Gain = 50} is optimal, improving from 4.4 to 10.0 effective
bits. Higher gain (if available) would push the signal closer to the
ADC's full-scale range and further reduce the impact of ADC noise, but
the PGA's own noise would eventually become the limiting factor.

\begin{center}\rule{0.5\linewidth}{0.5pt}\end{center}

\section{Problem 13.3.5}\label{problem-13.3.5}

\textbf{Given:} A transimpedance amplifier for a fiber-optic receiver
must achieve a bandwidth of 1 MHz. The photodiode capacitance is
C\textsubscript{in} = 5 pF. The op-amp has GBW = 50 MHz.

\textbf{Find:} The maximum feedback resistor R\textsubscript{f} that can
achieve 1 MHz bandwidth, the optimal C\textsubscript{f}, and the output
voltage for a 50 μA photocurrent.

\textbf{Solution:} From f\textsubscript{3dB} =
1/(2πR\textsubscript{f}C\textsubscript{f}) and the optimal
C\textsubscript{f} formula:

C\textsubscript{f} = √(C\textsubscript{in} / (2π × R\textsubscript{f} ×
GBW)).

Substituting into the bandwidth equation: f\textsubscript{3dB} = 1 /
(2πR\textsubscript{f} × √(C\textsubscript{in} / (2π × R\textsubscript{f}
× GBW))) f\textsubscript{3dB} = √(GBW /
(2πR\textsubscript{f}C\textsubscript{in})) / (2π × R\textsubscript{f} /
R\textsubscript{f})

Using the relation f\textsubscript{3dB} = √(GBW /
(2πR\textsubscript{f}C\textsubscript{in})): 1 × 10⁶ = √(50 × 10⁶ / (2π ×
R\textsubscript{f} × 5 × 10⁻¹²)) (1 × 10⁶)² = 50 × 10⁶ / (2π ×
R\textsubscript{f} × 5 × 10⁻¹²) 10¹² = 50 × 10⁶ / (31.42 × 10⁻¹² ×
R\textsubscript{f}) R\textsubscript{f} = 50 × 10⁶ / (31.42 × 10⁻¹² ×
10¹²) = 50 × 10⁶ / 31.42 = \textbf{1.59 MΩ}.

Optimal C\textsubscript{f} = √(5 × 10⁻¹² / (2π × 1.59 × 10⁶ × 50 × 10⁶))
= √(5 × 10⁻¹² / 4.995 × 10¹⁴) = √(1.0 × 10⁻²⁶) = \textbf{0.1 pF}.

Output voltage: V\textsubscript{out} = -I × R\textsubscript{f} = -50 ×
10⁻⁶ × 1.59 × 10⁶ = \textbf{-79.5 V} -- this exceeds supply rails, so
R\textsubscript{f} must be reduced.

For V\textsubscript{out} ≤ 5 V: R\textsubscript{f} = 5 / (50 × 10⁻⁶) =
\textbf{100 kΩ} maximum. With R\textsubscript{f} = 100 kΩ, bandwidth =
√(50 × 10⁶ / (2π × 100,000 × 5 × 10⁻¹²)) = √(1.592 × 10¹³) =
\textbf{3.99 MHz}, which exceeds the 1 MHz requirement.

Use R\textsubscript{f} = 100 kΩ, V\textsubscript{out} = -50 μA × 100 kΩ
= \textbf{-5.0 V}, BW = \textbf{3.99 MHz}.

\begin{center}\rule{0.5\linewidth}{0.5pt}\end{center}

\section{Problem 13.3.6}\label{problem-13.3.6}

\textbf{Given:} A voltage follower buffers a 10-bit R-2R DAC output. The
DAC has an output impedance of 10 kΩ and produces an output of 0 to 3.3
V. The follower op-amp has V\textsubscript{OS} = 1.5 mV and
I\textsubscript{B} = 50 nA.

\textbf{Find:} The total DC error at the output, expressed in LSBs of
the DAC.

\textbf{Solution:} DAC LSB = 3.3 V / 2¹⁰ = 3.3 / 1,024 = 3.223 mV.

Offset voltage error: V\textsubscript{OS} = 1.5 mV. Bias current error:
V\textsubscript{IB} = I\textsubscript{B} × R\textsubscript{source} = 50
× 10⁻⁹ × 10,000 = 0.5 mV.

Total DC error (worst case): V\textsubscript{error} = 1.5 + 0.5 = 2.0
mV.

Error in LSBs: 2.0 / 3.223 = \textbf{0.62 LSBs}.

This is less than 1 LSB, so the buffer does not degrade the DAC's
resolution. For a 12-bit DAC (LSB = 0.806 mV), this error would be 2.48
LSBs, requiring a lower-offset op-amp.

\chapter{Chapter 13 --- Section 13.4: Differential and Instrumentation
Amplifiers}\label{chapter-13-section-13.4-differential-and-instrumentation-amplifiers}

Practice problems covering difference amplifiers, instrumentation
amplifiers, and current sense amplifiers.

\begin{center}\rule{0.5\linewidth}{0.5pt}\end{center}

\section{Problem 13.4.1}\label{problem-13.4.1}

\textbf{Given:} A difference amplifier with R₁ = R₃ = 5.1 kΩ and R₂ = R₄
= 51 kΩ measures the voltage across a 50 mΩ current shunt in a 24 V DC
power bus. The load current is 12 A. R₃ has a tolerance error of 0.05\%
(R₃ = 5,102.55 Ω).

\textbf{Find:} The ideal output, the common-mode error, and the
effective CMRR.

\textbf{Solution:} Differential signal: V\textsubscript{diff} = I ×
R\textsubscript{shunt} = 12 × 0.050 = 0.600 V.

Differential gain: A\textsubscript{diff} = R₂/R₁ = 51,000/5,100 = 10.

Ideal output: V\textsubscript{out(ideal)} = 10 × 0.600 = \textbf{6.00
V}.

Common-mode gain with mismatched R₃: The mismatch ratio ΔR/R = 0.0005
(0.05\%). A\textsubscript{cm} ≈ (ΔR/R) × (R₂/R₁) / (1 + R₂/R₁) = 0.0005
× 10 / 11 = 0.000454.

Common-mode error voltage: V\textsubscript{error} = A\textsubscript{cm}
× V\textsubscript{cm} = 0.000454 × 24 = \textbf{10.9 mV}.

CMRR: CMRR = A\textsubscript{diff} / A\textsubscript{cm} = 10 / 0.000454
= 22,026. CMRR(dB) = 20log₁₀(22,026) = \textbf{86.9 dB}.

Total output: 6.00 + 0.0109 = 6.011 V (0.18\% error).

\begin{center}\rule{0.5\linewidth}{0.5pt}\end{center}

\section{Problem 13.4.2}\label{problem-13.4.2}

\textbf{Given:} An INA128 instrumentation amplifier has internal
resistors R = 25 kΩ. It is used to amplify a Wheatstone bridge output of
3.8 mV differential with 5.0 V excitation (common-mode = 2.5 V). The
required gain is 200. The INA has CMRR = 110 dB.

\textbf{Find:} The gain-setting resistor R\textsubscript{G}, the output
voltage, and the common-mode error.

\textbf{Solution:} Gain equation: A\textsubscript{v} = 1 +
2R/R\textsubscript{G}. R\textsubscript{G} = 2R / (A\textsubscript{v} -
1) = 2 × 25,000 / (200 - 1) = 50,000 / 199 = \textbf{251.3 Ω} (use 249 Ω
standard value).

Actual gain with 249 Ω: A\textsubscript{v} = 1 + 50,000/249 = 1 + 200.8
= 201.8.

Output voltage: V\textsubscript{out} = 201.8 × 3.8 mV = \textbf{766.8
mV}.

CMRR = 110 dB = 10\textsuperscript{110/20} = 316,228.

Common-mode gain: A\textsubscript{cm} = A\textsubscript{diff} / CMRR =
201.8 / 316,228 = 6.38 × 10⁻⁴.

Common-mode error at output: V\textsubscript{error} =
A\textsubscript{cm} × V\textsubscript{cm} = 6.38 × 10⁻⁴ × 2.5 =
\textbf{1.60 mV}.

Percentage error: 1.60 / 766.8 × 100\% = \textbf{0.21\%}.

\begin{center}\rule{0.5\linewidth}{0.5pt}\end{center}

\section{Problem 13.4.3}\label{problem-13.4.3}

\textbf{Given:} A high-side current sense amplifier (gain = 50 V/V)
monitors current through a 25 mΩ shunt on a 12 V automotive bus. The
amplifier specifications are: V\textsubscript{OS} = 50 μV
(input-referred), gain error = 0.5\%, CMRR = 120 dB. The load draws 8 A.

\textbf{Find:} The ideal output, the total output error, and the minimum
detectable current.

\textbf{Solution:} Shunt voltage: V\textsubscript{shunt} = I ×
R\textsubscript{shunt} = 8 × 0.025 = 200 mV.

Ideal output: V\textsubscript{out(ideal)} = V\textsubscript{shunt} ×
Gain = 0.200 × 50 = \textbf{10.0 V}.

Offset error at output: V\textsubscript{OS} × Gain = 50 × 10⁻⁶ × 50 =
2.5 mV.

Gain error: 0.5\% × 10.0 = 50 mV.

CMRR error: CMRR = 120 dB = 10⁶. CM input error = V\textsubscript{cm} /
CMRR = 12 / 10⁶ = 12 μV. CM error at output = 12 μV × 50 = 0.6 mV.

Total output error (worst case): V\textsubscript{error} = 2.5 + 50 + 0.6
= \textbf{53.1 mV} (0.53\% of 10.0 V).

Minimum detectable current (limited by offset): I\textsubscript{min} =
V\textsubscript{OS} / R\textsubscript{shunt} = 50 μV / 25 mΩ = 50 × 10⁻⁶
/ 0.025 = \textbf{2.0 mA}.

\begin{center}\rule{0.5\linewidth}{0.5pt}\end{center}

\section{Problem 13.4.4}\label{problem-13.4.4}

\textbf{Given:} A bidirectional current sense amplifier with gain = 100
V/V monitors battery charge/discharge through a 5 mΩ shunt. The
reference output voltage is V\textsubscript{ref} = 2.5 V (zero-current
midpoint). The supply is V\textsubscript{CC} = 5.0 V. Maximum charge
current is 20 A, maximum discharge current is 30 A.

\textbf{Find:} The output voltage range during charge and discharge, and
whether the output stays within the supply range.

\textbf{Solution:} During charge (positive current = 20 A):
V\textsubscript{shunt} = 20 × 0.005 = 100 mV. V\textsubscript{out} =
V\textsubscript{ref} + Gain × V\textsubscript{shunt} = 2.5 + 100 × 0.100
= 2.5 + 10.0 = \textbf{12.5 V}.

This exceeds V\textsubscript{CC} = 5.0 V! The output would clip at
approximately 4.8 V.

Maximum measurable charge current: V\textsubscript{out(max)} = 5.0 V
(approximately). V\textsubscript{shunt(max)} = (5.0 - 2.5) / 100 = 25
mV. I\textsubscript{charge(max)} = 25 mV / 5 mΩ = \textbf{5.0 A}.

During discharge (negative current = -30 A): V\textsubscript{shunt} =
-30 × 0.005 = -150 mV. V\textsubscript{out} = 2.5 + 100 × (-0.150) = 2.5
- 15.0 = \textbf{-12.5 V} (clips near 0 V).

Maximum measurable discharge current: V\textsubscript{out(min)} ≈ 0.1 V
(near ground). V\textsubscript{shunt(min)} = (0.1 - 2.5) / 100 = -24 mV.
I\textsubscript{discharge(max)} = 24 mV / 5 mΩ = \textbf{4.8 A}.

\textbf{Design correction needed:} Reduce gain to 20 V/V. Then charge:
V\textsubscript{out} = 2.5 + 20 × 0.1 = 4.5 V; discharge:
V\textsubscript{out} = 2.5 + 20 × (-0.15) = -0.5 V (still clips). Use
gain = 10 V/V: charge: 2.5 + 10 × 0.1 = \textbf{3.5 V}; discharge: 2.5 +
10 × (-0.15) = \textbf{1.0 V}. Both within 0-5 V range.

\begin{center}\rule{0.5\linewidth}{0.5pt}\end{center}

\section{Problem 13.4.5}\label{problem-13.4.5}

\textbf{Given:} An instrumentation amplifier with gain = 500 amplifies a
thermocouple signal of 1.2 mV differential. The INA has
V\textsubscript{OS} = 25 μV, CMRR = 100 dB, and the thermocouple has a
common-mode voltage of 0.5 V (from grounding differences). The circuit
bandwidth is 100 Hz.

\textbf{Find:} The desired output voltage, all error contributions, and
the total percentage error.

\textbf{Solution:} Desired output: V\textsubscript{out} = 500 × 1.2 mV =
\textbf{600 mV}.

Offset error at output: V\textsubscript{OS(out)} = 25 μV × 500 = 12.5
mV.

CMRR error: CMRR = 100 dB = 100,000. Input-referred CM error = 0.5 /
100,000 = 5 μV. Output CM error = 5 μV × 500 = 2.5 mV.

Total output error (worst case): V\textsubscript{error} = 12.5 + 2.5 =
15.0 mV.

Percentage error: 15.0 / 600 × 100\% = \textbf{2.5\%}.

To reduce error below 0.5\%, we need V\textsubscript{error} \textless{}
3.0 mV. Using a chopper-stabilized INA with V\textsubscript{OS} = 2 μV
and CMRR = 130 dB: offset error = 2 μV × 500 = 1.0 mV; CMRR error =
(0.5/3.16 × 10⁶) × 500 = 0.079 mV. Total = 1.08 mV = \textbf{0.18\%}.

\begin{center}\rule{0.5\linewidth}{0.5pt}\end{center}

\section{Problem 13.4.6}\label{problem-13.4.6}

\textbf{Given:} A difference amplifier with R₁ = R₃ = 10 kΩ and R₂ = R₄
= 10 kΩ (unity gain) measures a 120 mV signal with 60 V common mode on a
high-voltage DC bus. All resistors are 0.01\% tolerance.

\textbf{Find:} The CMRR limited by resistor matching, the common-mode
error, and whether the measurement meets 1\% accuracy.

\textbf{Solution:} Differential gain: A\textsubscript{diff} = R₂/R₁ =
10,000/10,000 = 1.

Worst-case CMRR from 0.01\% resistor matching: CMRR ≈ (1 + R₂/R₁) / (4 ×
ΔR/R) = (1 + 1) / (4 × 0.0001) = 2 / 0.0004 = 5,000. CMRR(dB) =
20log₁₀(5,000) = \textbf{74.0 dB}.

Common-mode gain: A\textsubscript{cm} = A\textsubscript{diff} / CMRR = 1
/ 5,000 = 0.0002.

Common-mode error: V\textsubscript{error} = A\textsubscript{cm} ×
V\textsubscript{cm} = 0.0002 × 60 = \textbf{12 mV}.

Output: V\textsubscript{out} = 120 + 12 = 132 mV (or 108 mV depending on
polarity). Percentage error: 12/120 × 100\% = \textbf{10\%} -- this does
NOT meet the 1\% accuracy requirement.

To achieve 1\% accuracy: V\textsubscript{error} \textless{} 1.2 mV, CMRR
\textgreater{} 60/0.0012 = 50,000 (94 dB). Required resistor matching:
ΔR/R \textless{} 2/(4 × 50,000) = 10 ppm = \textbf{0.001\%}. Use a
precision monolithic difference amplifier IC with laser-trimmed
resistors.

\chapter{Chapter 13 --- Section 13.5: Active
Filters}\label{chapter-13-section-13.5-active-filters}

Practice problems covering first-order filters, Sallen-Key filters,
state-variable filters, multiple feedback filters, and notch filters.

\begin{center}\rule{0.5\linewidth}{0.5pt}\end{center}

\section{Problem 13.5.1}\label{problem-13.5.1}

\textbf{Given:} Design a first-order active highpass filter (inverting)
with a passband gain of -8 and a cutoff frequency of 500 Hz. Use
R\textsubscript{f} = 56 kΩ.

\textbf{Find:} The input resistor R\textsubscript{in}, the input
capacitor C, and the gain at 200 Hz and 2 kHz.

\textbf{Solution:} Passband gain: A\textsubscript{v} =
-R\textsubscript{f}/R\textsubscript{in}, so R\textsubscript{in} =
R\textsubscript{f}/\textbar A\textsubscript{v}\textbar{} = 56,000/8 =
\textbf{7,000 Ω} (use 6.8 kΩ standard).

With R\textsubscript{in} = 6.8 kΩ: actual gain = -56,000/6,800 = -8.24.

Cutoff frequency: f\textsubscript{c} = 1/(2πR\textsubscript{in}C), so: C
= 1/(2π × R\textsubscript{in} × f\textsubscript{c}) = 1/(2π × 6,800 ×
500) = 1/(21,363,000) = \textbf{46.8 nF} (use 47 nF standard).

Verification: f\textsubscript{c} = 1/(2π × 6,800 × 47 × 10⁻⁹) = 1/(2.008
× 10⁻³) = \textbf{498 Hz} ≈ 500 Hz.

Gain at 200 Hz (below cutoff): \textbar A(f)\textbar{} =
\textbar A\textsubscript{v}\textbar{} × (f/f\textsubscript{c}) / √(1 +
(f/f\textsubscript{c})²) = 8.24 × (200/498) / √(1 + (200/498)²) = 8.24 ×
0.4016 / √(1 + 0.1613) = 3.309 / 1.0778 = \textbf{3.07} (−8.58 dB
relative to passband).

Gain at 2 kHz (above cutoff): \textbar A(f)\textbar{} = 8.24 ×
(2,000/498) / √(1 + (2,000/498)²) = 8.24 × 4.016 / √(1 + 16.13) = 33.09
/ 4.141 = \textbf{7.99} (essentially flat, -0.27 dB from passband).

\begin{center}\rule{0.5\linewidth}{0.5pt}\end{center}

\section{Problem 13.5.2}\label{problem-13.5.2}

\textbf{Given:} Design a unity-gain Sallen-Key Butterworth lowpass
filter with f\textsubscript{c} = 5 kHz. Use C₂ = 4.7 nF with equal
resistors.

\textbf{Find:} C₁ and R for a Butterworth (Q = 0.707) response.

\textbf{Solution:} For a Butterworth response with equal resistors: C₁ =
2 × C₂ = 2 × 4.7 nF = \textbf{9.4 nF} (use two 4.7 nF in parallel, or
use 10 nF).

Using C₁ = 10 nF and C₂ = 4.7 nF (ratio = 2.128, slightly above 2):

Cutoff frequency: f\textsubscript{c} = 1/(2πR√(C₁C₂)). √(C₁C₂) = √(10 ×
10⁻⁹ × 4.7 × 10⁻⁹) = √(47 × 10⁻¹⁸) = 6.856 × 10⁻⁹.

R = 1/(2πf\textsubscript{c}√(C₁C₂)) = 1/(2π × 5,000 × 6.856 × 10⁻⁹) =
1/(215.4 × 10⁻⁶) = \textbf{4,643 Ω} (use 4.7 kΩ).

Verification: f\textsubscript{c} = 1/(2π × 4,700 × 6.856 × 10⁻⁹) =
1/(202.3 × 10⁻⁶) = \textbf{4,943 Hz} ≈ 5 kHz.

Q = √(C₁/C₂) / 2 = √(10/4.7) / 2 = √2.128 / 2 = 1.459 / 2 =
\textbf{0.730} (close to the 0.707 Butterworth target).

\begin{center}\rule{0.5\linewidth}{0.5pt}\end{center}

\section{Problem 13.5.3}\label{problem-13.5.3}

\textbf{Given:} A state-variable bandpass filter is required for an
audio spectrum analyzer with center frequency f₀ = 1 kHz and Q = 20. The
integrator capacitors are C = 22 nF.

\textbf{Find:} The integrator resistance R, the bandwidth, and the -3 dB
frequencies.

\textbf{Solution:} Integrator resistance: R = 1/(2πf₀C) = 1/(2π × 1,000
× 22 × 10⁻⁹) = 1/(138.2 × 10⁻⁶) = \textbf{7,234 Ω} (use 7.15 kΩ or 6.8
kΩ + 470 Ω in series).

Bandwidth: BW = f₀/Q = 1,000/20 = \textbf{50 Hz}.

Lower -3 dB frequency: f\textsubscript{L} = f₀ - BW/2 = 1,000 - 25 =
\textbf{975 Hz}.

Upper -3 dB frequency: f\textsubscript{H} = f₀ + BW/2 = 1,000 + 25 =
\textbf{1,025 Hz}.

This extremely narrow 50 Hz bandwidth around 1 kHz allows the analyzer
to isolate individual frequency components in a complex audio spectrum.

\begin{center}\rule{0.5\linewidth}{0.5pt}\end{center}

\section{Problem 13.5.4}\label{problem-13.5.4}

\textbf{Given:} A multiple feedback bandpass filter requires f₀ = 10
kHz, Q = 8, and midband gain \textbar A₀\textbar{} = 5. Use equal
capacitors C = 1 nF.

\textbf{Find:} The resistor values R₁, R₂, R₃, and the bandwidth.

\textbf{Solution:} Bandwidth: BW = f₀/Q = 10,000/8 = \textbf{1,250 Hz}.

From Q equation: R₂ = Q/(π × f₀ × C) = 8/(π × 10,000 × 1 × 10⁻⁹) =
8/(31.42 × 10⁻⁶) = \textbf{254,648 Ω} (use 270 kΩ).

From gain equation: A₀ = -R₂/(2R₁), so: R₁ = R₂/(2 ×
\textbar A₀\textbar) = 270,000/(2 × 5) = \textbf{27,000 Ω} (use 27 kΩ).

From center frequency: f₀ = (1/(2πC)) × √(1/(R₁R₃)), so: R₃ =
1/((2πf₀C)² × R₁) = 1/((2π × 10,000 × 10⁻⁹)² × 27,000) = 1/((62.83 ×
10⁻⁶)² × 27,000) = 1/(3.948 × 10⁻⁹ × 27,000) = 1/(1.066 × 10⁻⁴) =
\textbf{9,381 Ω} (use 9.1 kΩ).

Verification with standard values: f₀ = (1/(2πC)) × √(1/(R₁R₃)) = (1/(2π
× 10⁻⁹)) × √(1/(27,000 × 9,100)) = 159,155 × √(4.07 × 10⁻⁹) = 159,155 ×
6.38 × 10⁻⁵ = \textbf{10,154 Hz} ≈ 10 kHz.

\begin{center}\rule{0.5\linewidth}{0.5pt}\end{center}

\section{Problem 13.5.5}\label{problem-13.5.5}

\textbf{Given:} Design an active Twin-T notch filter to reject 50 Hz
power-line hum (European mains). Use C = 68 nF. The positive feedback
fraction is k = 0.98.

\textbf{Find:} R, R/2, 2C, the notch Q, and the -3 dB bandwidth.

\textbf{Solution:} Notch frequency: f₀ = 1/(2πRC), so: R = 1/(2πf₀C) =
1/(2π × 50 × 68 × 10⁻⁹) = 1/(21.36 × 10⁻⁶) = \textbf{46,818 Ω} (use 47
kΩ).

Shunt resistor: R/2 = 47,000/2 = \textbf{23,500 Ω} (use 22 kΩ + 1.5 kΩ
in series = 23.5 kΩ). Shunt capacitor: 2C = 2 × 68 nF = \textbf{136 nF}
(use 120 nF + 15 nF in parallel = 135 nF).

Quality factor: Q = 1/(4(1 - k)) = 1/(4 × 0.02) = \textbf{12.5}.

Bandwidth: BW = f₀/Q = 50/12.5 = \textbf{4.0 Hz}.

The -3 dB frequencies are: 50 ± 2.0 Hz = \textbf{48.0 Hz and 52.0 Hz}.

This very narrow notch (4 Hz bandwidth) effectively removes the 50 Hz
fundamental while preserving frequencies even a few hertz away. Adjacent
harmonics at 100 Hz and 150 Hz are virtually unaffected.

\begin{center}\rule{0.5\linewidth}{0.5pt}\end{center}

\section{Problem 13.5.6}\label{problem-13.5.6}

\textbf{Given:} A fourth-order Butterworth lowpass filter is needed at
f\textsubscript{c} = 2 kHz. It is constructed by cascading two
second-order Sallen-Key stages. The first stage requires Q₁ = 0.541 and
the second stage Q₂ = 1.307 (Butterworth polynomial coefficients). Both
stages use R = 10 kΩ.

\textbf{Find:} The capacitor values C₁ and C₂ for each stage to achieve
the required Q values.

\textbf{Solution:} For a Sallen-Key lowpass with equal R, Q =
√(C₁/C₂)/2, so C₁/C₂ = (2Q)².

Also, f\textsubscript{c} = 1/(2πR√(C₁C₂)), so √(C₁C₂) =
1/(2πRf\textsubscript{c}) = 1/(2π × 10,000 × 2,000) = 7.958 × 10⁻⁹. C₁ ×
C₂ = (7.958 × 10⁻⁹)² = 63.33 × 10⁻¹⁸.

\textbf{Stage 1 (Q₁ = 0.541):} C₁/C₂ = (2 × 0.541)² = 1.082² = 1.171. C₁
= 1.171 × C₂. Substituting: 1.171 × C₂² = 63.33 × 10⁻¹⁸. C₂² = 54.08 ×
10⁻¹⁸. C₂ = 7.354 × 10⁻⁹ = \textbf{7.35 nF} (use 6.8 nF). C₁ = 1.171 ×
7.35 nF = \textbf{8.61 nF} (use 8.2 nF).

\textbf{Stage 2 (Q₂ = 1.307):} C₁/C₂ = (2 × 1.307)² = 2.614² = 6.833.
C₂² = 63.33 × 10⁻¹⁸ / 6.833 = 9.27 × 10⁻¹⁸. C₂ = 3.044 × 10⁻⁹ =
\textbf{3.04 nF} (use 2.7 nF). C₁ = 6.833 × 3.04 nF = \textbf{20.8 nF}
(use 22 nF).

The cascaded filter provides a fourth-order rolloff of \textbf{-80
dB/decade} above 2 kHz, with a maximally flat passband response.

\begin{center}\rule{0.5\linewidth}{0.5pt}\end{center}

\section{Problem 13.5.7}\label{problem-13.5.7}

\textbf{Given:} A state-variable filter must provide simultaneous
lowpass, highpass, and bandpass outputs at f₀ = 3 kHz with Q = 5. The
filter also needs a notch output at 3 kHz. The integrator components are
C = 4.7 nF.

\textbf{Find:} The integrator resistance R, the bandwidth of the
bandpass output, and describe how the notch output is obtained.

\textbf{Solution:} Integrator resistance: R = 1/(2πf₀C) = 1/(2π × 3,000
× 4.7 × 10⁻⁹) = 1/(88.59 × 10⁻⁶) = \textbf{11,288 Ω} (use 11 kΩ).

Verification: f₀ = 1/(2π × 11,000 × 4.7 × 10⁻⁹) = 1/(324.7 × 10⁻⁶) =
\textbf{3,080 Hz} ≈ 3 kHz.

Bandwidth of bandpass output: BW = f₀/Q = 3,000/5 = \textbf{600 Hz}.

The -3 dB frequencies are: 3,000 - 300 = \textbf{2,700 Hz} and 3,000 +
300 = \textbf{3,300 Hz}.

\textbf{Notch output:} Add a fourth op-amp configured as a summing
amplifier to combine the lowpass and highpass outputs with equal gain.
Below f₀, the lowpass output passes the signal; above f₀, the highpass
output passes the signal; at f₀, both outputs are at their -3 dB points
and sum to produce a deep null. The notch depth is limited by the
matching of the two gains and the phase accuracy of the outputs. With
precise gain matching (0.1\%), notch depths exceeding \textbf{40 dB} are
achievable.

\chapter{Chapter 13 --- Section 13.6:
Comparators}\label{chapter-13-section-13.6-comparators}

Practice problems covering basic comparators, Schmitt triggers, Wien
bridge oscillators, and relaxation oscillators.

\begin{center}\rule{0.5\linewidth}{0.5pt}\end{center}

\section{Problem 13.6.1}\label{problem-13.6.1}

\textbf{Given:} A comparator with V\textsubscript{OH} = 3.3 V and
V\textsubscript{OL} = 0 V has its inverting input connected to a 1.65 V
reference. The non-inverting input receives a 10 kHz sine wave:
V\textsubscript{in} = 1.65 + 2.0 sin(2π × 10,000t).

\textbf{Find:} The output waveform, the duty cycle, and the times during
each period when the output transitions.

\textbf{Solution:} The output is high when V\textsubscript{in}
\textgreater{} V\textsubscript{ref} = 1.65 V: 1.65 + 2.0 sin(ωt)
\textgreater{} 1.65, which simplifies to sin(ωt) \textgreater{} 0.

This occurs for 0 \textless{} ωt \textless{} π (the positive
half-cycle).

The output transitions from low to high at t = 0 (ωt = 0) and from high
to low at t = T/2 = 50 μs (ωt = π).

Duty cycle = \textbf{50\%} (symmetric about the reference).

The output is a \textbf{3.3 V square wave at 10 kHz} with 50\% duty
cycle.

If the reference is shifted to 2.65 V: sin(ωt) \textgreater{} (2.65 -
1.65)/2.0 = 0.5. ωt transitions at π/6 and 5π/6. Duty cycle = (5π/6 -
π/6)/(2π) = (4π/6)/(2π) = \textbf{33.3\%}.

\begin{center}\rule{0.5\linewidth}{0.5pt}\end{center}

\section{Problem 13.6.2}\label{problem-13.6.2}

\textbf{Given:} A Schmitt trigger must clean up a noisy signal from a
hall-effect sensor. The required thresholds are V\textsubscript{TH} =
3.5 V and V\textsubscript{TL} = 1.5 V. The comparator has
V\textsubscript{OH} = 5.0 V and V\textsubscript{OL} = 0 V. Use a
non-inverting configuration with a reference at the inverting input.

\textbf{Find:} The reference voltage V\textsubscript{ref}, the feedback
ratio R₂/(R₁ + R₂), and R₁ if R₂ = 10 kΩ. What is the maximum noise
amplitude the circuit can tolerate without false triggering?

\textbf{Solution:} Reference voltage: V\textsubscript{ref} =
(V\textsubscript{TH} × V\textsubscript{OL} - V\textsubscript{TL} ×
V\textsubscript{OH}) / (V\textsubscript{OL} - V\textsubscript{OH} +
V\textsubscript{TH} - V\textsubscript{TL}) = (3.5 × 0 - 1.5 × 5.0) / (0
- 5.0 + 3.5 - 1.5) = -7.5 / (-3.0) = \textbf{2.5 V}.

Feedback ratio: R₂/(R₁ + R₂) = (V\textsubscript{TH} -
V\textsubscript{ref}) / (V\textsubscript{OH} - V\textsubscript{ref}) =
(3.5 - 2.5) / (5.0 - 2.5) = 1.0/2.5 = \textbf{0.4}.

With R₂ = 10 kΩ: 10,000/(R₁ + 10,000) = 0.4. R₁ + 10,000 = 25,000. R₁ =
\textbf{15 kΩ}.

Hysteresis: V\textsubscript{H} = V\textsubscript{TH} -
V\textsubscript{TL} = 3.5 - 1.5 = \textbf{2.0 V}.

Maximum noise tolerance: the circuit can tolerate noise with
peak-to-peak amplitude up to \textbf{2.0 V} without false triggering,
since the input must cross the full hysteresis band to cause a state
change.

\begin{center}\rule{0.5\linewidth}{0.5pt}\end{center}

\section{Problem 13.6.3}\label{problem-13.6.3}

\textbf{Given:} Design a Wien bridge oscillator for f₀ = 1 kHz using C =
10 nF. The non-inverting amplifier uses R₁ = 10 kΩ. A 5\% gain excess
above the critical value is desired for reliable startup.

\textbf{Find:} The Wien network resistance R, the feedback resistor
R\textsubscript{f}, and the oscillation frequency with standard
component values.

\textbf{Solution:} Wien bridge frequency: R = 1/(2πf₀C) = 1/(2π × 1,000
× 10 × 10⁻⁹) = 1/(62.83 × 10⁻⁶) = \textbf{15,915 Ω} (use 15 kΩ + 910 Ω =
15.91 kΩ, or simply use 16 kΩ).

Critical gain for oscillation: A\textsubscript{v} = 3 exactly.
R\textsubscript{f(critical)} = 2 × R₁ = 2 × 10,000 = 20 kΩ.

With 5\% gain excess: A\textsubscript{v} = 3 × 1.05 = 3.15.
R\textsubscript{f} = (3.15 - 1) × R₁ = 2.15 × 10,000 = \textbf{21.5 kΩ}
(use 22 kΩ).

Actual gain: A\textsubscript{v} = 1 + 22,000/10,000 = 3.2 (6.7\%
excess).

Oscillation frequency with R = 16 kΩ: f₀ = 1/(2π × 16,000 × 10 × 10⁻⁹) =
1/(1.005 × 10⁻³) = \textbf{995 Hz} ≈ 1 kHz.

Without amplitude stabilization, the 6.7\% gain excess will cause the
output to grow until it clips at the supply rails, producing noticeable
distortion. An AGC circuit or nonlinear feedback element should be
added.

\begin{center}\rule{0.5\linewidth}{0.5pt}\end{center}

\section{Problem 13.6.4}\label{problem-13.6.4}

\textbf{Given:} An op-amp relaxation oscillator uses R = 33 kΩ, C = 4.7
nF, and feedback divider R₁ = R₂ = 22 kΩ (β = 0.5). The op-amp saturates
at ±13 V.

\textbf{Find:} The oscillation frequency, the threshold voltages, and
the capacitor voltage waveform limits.

\textbf{Solution:} Feedback fraction: β = R₂/(R₁ + R₂) = 22,000/44,000 =
\textbf{0.5}.

Threshold voltages: V\textsubscript{TH} = +β × V\textsubscript{sat} =
0.5 × 13 = \textbf{+6.5 V}. V\textsubscript{TL} = -β ×
V\textsubscript{sat} = 0.5 × (-13) = \textbf{-6.5 V}.

The capacitor swings between -6.5 V and +6.5 V.

Frequency: f = 1/(2RC × ln((1 + β)/(1 - β))) = 1/(2 × 33,000 × 4.7 ×
10⁻⁹ × ln(1.5/0.5)) = 1/(310.2 × 10⁻⁶ × ln(3)) = 1/(310.2 × 10⁻⁶ ×
1.0986) = 1/(340.8 × 10⁻⁶) = \textbf{2,934 Hz ≈ 2.93 kHz}.

Period: T = 1/2,934 = \textbf{340.8 μs}.

The output is a square wave at 2.93 kHz alternating between +13 V and
-13 V with 50\% duty cycle. The capacitor voltage is an exponential
waveform (approximating a triangle) oscillating between -6.5 V and +6.5
V.

\begin{center}\rule{0.5\linewidth}{0.5pt}\end{center}

\section{Problem 13.6.5}\label{problem-13.6.5}

\textbf{Given:} A phase-shift oscillator uses three identical RC
highpass sections with R = 10 kΩ and C = 1 nF, driving an inverting
amplifier.

\textbf{Find:} The oscillation frequency and the minimum required gain
\textbar A\textsubscript{v}\textbar.

\textbf{Solution:} For three identical RC highpass sections, the
oscillation frequency is: f₀ = 1/(2πRC√6) = 1/(2π × 10,000 × 1 × 10⁻⁹ ×
√6) = 1/(2π × 10,000 × 10⁻⁹ × 2.449) = 1/(153.9 × 10⁻⁶) = \textbf{6,498
Hz ≈ 6.5 kHz}.

Minimum required gain for oscillation:
\textbar A\textsubscript{v}\textbar{} = \textbf{29}
(R\textsubscript{f}/R\textsubscript{in} = 29).

If R\textsubscript{in} = 10 kΩ: R\textsubscript{f} = 29 × 10,000 =
\textbf{290 kΩ} (use 300 kΩ).

With R\textsubscript{f} = 300 kΩ: \textbar A\textsubscript{v}\textbar{}
= 300,000/10,000 = 30 (3.4\% excess gain for reliable startup).

The high gain requirement (29) means the phase-shift oscillator has more
noise gain than the Wien bridge oscillator (gain = 3), resulting in
higher output noise. However, it requires no amplitude stabilization for
moderate distortion levels.

\begin{center}\rule{0.5\linewidth}{0.5pt}\end{center}

\section{Problem 13.6.6}\label{problem-13.6.6}

\textbf{Given:} A triangle wave generator uses a Schmitt trigger (R₁ =
10 kΩ, R₂ = 20 kΩ, V\textsubscript{sat} = ±12 V) driving an integrator
(R = 47 kΩ, C = 10 nF). The Schmitt trigger threshold is β = R₂/(R₁ +
R₂).

\textbf{Find:} The triangle wave frequency, peak-to-peak amplitude, and
linearity advantage over the basic astable multivibrator.

\textbf{Solution:} Schmitt trigger feedback fraction: β = 20,000/(10,000
+ 20,000) = 2/3.

Threshold voltages: V\textsubscript{TH} = β × V\textsubscript{sat} =
(2/3) × 12 = \textbf{+8.0 V}. V\textsubscript{TL} = -(2/3) × 12 =
\textbf{-8.0 V}.

Triangle wave peak-to-peak amplitude = V\textsubscript{TH} -
V\textsubscript{TL} = \textbf{16.0 V}.

The integrator ramp rate when the Schmitt trigger output is at
+V\textsubscript{sat} = +12 V: dV/dt = -V\textsubscript{sat}/(RC) =
-12/(47,000 × 10 × 10⁻⁹) = -12/470 × 10⁻⁶ = -25,532 V/s.

Time for one half-cycle (ramp from +8 V to -8 V): t\textsubscript{half}
= ΔV / \textbar dV/dt\textbar{} = 16 / 25,532 = 626.3 μs.

Frequency: f = 1/(2 × t\textsubscript{half}) = 1/(2 × 626.3 × 10⁻⁶) =
\textbf{798.7 Hz ≈ 800 Hz}.

Alternative formula: f = V\textsubscript{sat} / (4RC ×
V\textsubscript{TH}) = 12 / (4 × 470 × 10⁻⁶ × 8) = 12 / (15.04 × 10⁻³) =
\textbf{798 Hz}.

\textbf{Linearity advantage:} The integrator produces a truly linear
ramp (constant dV/dt) because the op-amp maintains virtual ground at its
inverting input, forcing a constant current through R regardless of the
capacitor voltage. The basic RC astable multivibrator charges the
capacitor exponentially through a resistor, producing curved ramps that
become less linear at higher frequencies or larger voltage swings.

\chapter{Chapter 13 --- Section 13.7: Real Op-Amp
Limitations}\label{chapter-13-section-13.7-real-op-amp-limitations}

Practice problems covering input offset and bias errors, slew rate,
CMRR, PSRR, noise analysis, and input/output voltage range.

\begin{center}\rule{0.5\linewidth}{0.5pt}\end{center}

\section{Problem 13.7.1}\label{problem-13.7.1}

\textbf{Given:} An inverting amplifier with R\textsubscript{in} = 1 kΩ
and R\textsubscript{f} = 470 kΩ (gain = -470) uses a bipolar op-amp with
V\textsubscript{OS} = 3 mV, I\textsubscript{B} = 500 nA, and
I\textsubscript{OS} = 50 nA. No bias compensation resistor is used.

\textbf{Find:} The total output offset voltage, and the improvement when
a bias compensation resistor R\textsubscript{comp} = R\textsubscript{in}
\textbar\textbar{} R\textsubscript{f} is added to the non-inverting
input.

\textbf{Solution:} Noise gain: 1 +
R\textsubscript{f}/R\textsubscript{in} = 1 + 470,000/1,000 = 471.

\textbf{Without compensation:} Offset due to V\textsubscript{OS}:
V\textsubscript{out(OS)} = V\textsubscript{OS} × noise gain = 0.003 ×
471 = \textbf{1,413 mV = 1.413 V}. Offset due to I\textsubscript{B}:
V\textsubscript{out(IB)} = I\textsubscript{B} × R\textsubscript{f} = 500
× 10⁻⁹ × 470,000 = \textbf{235 mV}. Total: 1,413 + 235 = \textbf{1,648
mV = 1.65 V}.

\textbf{With compensation (R\textsubscript{comp} = R\textsubscript{in}
\textbar\textbar{} R\textsubscript{f}):} R\textsubscript{comp} = (1,000
× 470,000) / (1,000 + 470,000) = 470,000/471 = 998 Ω ≈ 1 kΩ.

The I\textsubscript{B} component is cancelled; only I\textsubscript{OS}
remains: V\textsubscript{out(IOS)} = I\textsubscript{OS} ×
R\textsubscript{f} = 50 × 10⁻⁹ × 470,000 = \textbf{23.5 mV}. Total:
1,413 + 23.5 = \textbf{1,436 mV = 1.44 V}.

Improvement: (1,648 - 1,436) / 1,648 × 100\% = \textbf{12.9\%}.

The improvement is modest because V\textsubscript{OS} is the dominant
error source at this high gain. Using a lower-offset op-amp
(V\textsubscript{OS} \textless{} 100 μV) would be more effective.

\begin{center}\rule{0.5\linewidth}{0.5pt}\end{center}

\section{Problem 13.7.2}\label{problem-13.7.2}

\textbf{Given:} An op-amp with SR = 25 V/μs produces a 5 V peak, 500 kHz
sine wave output. A separate channel amplifies a step input from 0 to 8
V.

\textbf{Find:} (a) Whether the sine wave output is slew-rate limited.
(b) The full-power bandwidth. (c) The 10-90\% rise time for the step
response.

\textbf{Solution:} \textbf{(a) Sine wave check:} Maximum rate of change:
dV/dt = 2πfV\textsubscript{p} = 2π × 500,000 × 5 = 15.71 × 10⁶ V/s =
15.71 V/μs. Since 15.71 V/μs \textless{} SR = 25 V/μs, the output is
\textbf{not slew-rate limited} at this frequency.

\textbf{(b) Full-power bandwidth:} f\textsubscript{FP} = SR /
(2πV\textsubscript{p}) = 25 × 10⁶ / (2π × 5) = 25 × 10⁶ / 31.42 =
\textbf{795.8 kHz}.

The op-amp can produce a 5 V peak sine wave without distortion up to
\textbf{796 kHz}.

\textbf{(c) Step response rise time:} Total voltage change: ΔV = 8 V.
Slew-limited rise time (0 to 100\%): t = ΔV/SR = 8/(25 × 10⁶) = 0.32 μs.
10-90\% rise time: t\textsubscript{r} = 0.8 × ΔV / SR = 6.4 / (25 × 10⁶)
= \textbf{0.256 μs = 256 ns}.

\begin{center}\rule{0.5\linewidth}{0.5pt}\end{center}

\section{Problem 13.7.3}\label{problem-13.7.3}

\textbf{Given:} An op-amp with CMRR = 80 dB at DC is used in a
non-inverting amplifier with a gain of 100. The input signal is a 500 μV
differential signal from a strain gauge bridge with 10 V excitation
(V\textsubscript{cm} = 5 V).

\textbf{Find:} The desired output, the common-mode error at the output,
and the signal-to-error ratio.

\textbf{Solution:} CMRR = 80 dB = 10\textsuperscript{80/20} = 10,000.

Desired output: V\textsubscript{out(signal)} = 100 × 500 μV = \textbf{50
mV}.

Equivalent input error from common mode: V\textsubscript{error(in)} =
V\textsubscript{cm} / CMRR = 5.0 / 10,000 = \textbf{500 μV}.

Error at output: V\textsubscript{error(out)} = 100 × 500 μV = \textbf{50
mV}.

Signal-to-error ratio: 50 mV / 50 mV = 1.0, or \textbf{0 dB}.

The common-mode error equals the signal -- the measurement is completely
unreliable! This demonstrates why an instrumentation amplifier with CMRR
\textgreater{} 100 dB is essential for high-gain measurements with
significant common-mode voltage.

For CMRR = 120 dB (10⁶): V\textsubscript{error(out)} = 100 × 5/10⁶ = 0.5
mV. SER = 50/0.5 = 100 = \textbf{40 dB}.

\begin{center}\rule{0.5\linewidth}{0.5pt}\end{center}

\section{Problem 13.7.4}\label{problem-13.7.4}

\textbf{Given:} An op-amp with PSRR = 90 dB at DC and PSRR = 45 dB at
500 kHz is powered from a 12 V supply. The supply has 50
mV\textsubscript{pp} ripple at 500 kHz from a buck converter. The
amplifier gain is 10.

\textbf{Find:} The equivalent input noise from the supply ripple and the
output ripple, with and without a 100 nF bypass capacitor (which reduces
the 500 kHz ripple to 2 mV\textsubscript{pp}).

\textbf{Solution:} PSRR at 500 kHz = 45 dB = 10\textsuperscript{45/20} =
177.8.

\textbf{Without bypass capacitor:} ΔV\textsubscript{OS} =
ΔV\textsubscript{supply} / PSRR = 50 mV / 177.8 = \textbf{281
μV\textsubscript{pp}}. Output ripple: V\textsubscript{ripple(out)} = 281
μV × 10 = \textbf{2.81 mV\textsubscript{pp}}.

\textbf{With 100 nF bypass capacitor (2 mV\textsubscript{pp} ripple):}
ΔV\textsubscript{OS} = 2 mV / 177.8 = \textbf{11.2
μV\textsubscript{pp}}. Output ripple: V\textsubscript{ripple(out)} =
11.2 μV × 10 = \textbf{112 μV\textsubscript{pp}}.

Improvement: 2,810 / 112 = 25× (\textbf{28 dB}).

For a 10-bit, 3.3 V ADC (LSB = 3.22 mV), the unbypassed ripple (2.81 mV)
is 0.87 LSB and would cause ±1 LSB jitter. The bypassed ripple (112 μV)
is only 0.035 LSB and is negligible.

\begin{center}\rule{0.5\linewidth}{0.5pt}\end{center}

\section{Problem 13.7.5}\label{problem-13.7.5}

\textbf{Given:} A non-inverting amplifier with gain = 50 uses an op-amp
with e\textsubscript{n} = 10 nV/√Hz and i\textsubscript{n} = 0.5 pA/√Hz.
The source resistance is R\textsubscript{S} = 100 kΩ. The bandwidth is 1
kHz. Assume the 1/f corner is below 10 Hz (negligible).

\textbf{Find:} The total input-referred noise, whether voltage noise or
current noise dominates, and the output noise.

\textbf{Solution:} Source resistance thermal noise: e\textsubscript{R} =
√(4kTR\textsubscript{S}) = √(1.66 × 10⁻²⁰ × 100,000) = √(1.66 × 10⁻¹⁵) =
\textbf{40.7 nV/√Hz}.

Current noise contribution: e\textsubscript{in} = i\textsubscript{n} ×
R\textsubscript{S} = 0.5 × 10⁻¹² × 100,000 = \textbf{50 nV/√Hz}.

Total input noise density: e\textsubscript{n,total} = √(10² + 50² +
40.7²) = √(100 + 2,500 + 1,656.5) = √4,256.5 = \textbf{65.2 nV/√Hz}.

\textbf{Current noise dominates} (50 nV/√Hz) over voltage noise (10
nV/√Hz) at this high source impedance. A JFET-input op-amp with lower
i\textsubscript{n} (e.g., 1 fA/√Hz) would be a much better choice.

Equivalent noise bandwidth: BW\textsubscript{n} = 1.57 × 1,000 = 1,570
Hz. Total RMS input noise: V\textsubscript{n,rms} = 65.2 × 10⁻⁹ × √1,570
= 65.2 × 10⁻⁹ × 39.62 = \textbf{2.58 μV\textsubscript{rms}}.

Output noise: V\textsubscript{n,out} = 2.58 μV × 50 = \textbf{129
μV\textsubscript{rms}}.

\begin{center}\rule{0.5\linewidth}{0.5pt}\end{center}

\section{Problem 13.7.6}\label{problem-13.7.6}

\textbf{Given:} A single-supply (V\textsubscript{CC} = 5.0 V,
V\textsubscript{EE} = 0 V) non-inverting amplifier with gain = 5 uses a
rail-to-rail output op-amp with V\textsubscript{OL} = 30 mV and
V\textsubscript{OH} = V\textsubscript{CC} - 50 mV at
I\textsubscript{load} = 2 mA, and R\textsubscript{DS(on)} = 75 Ω. The
input is biased at V\textsubscript{CC}/2 = 2.5 V, and the signal is ±0.4
V around the bias point.

\textbf{Find:} The output voltage range, whether clipping occurs at
I\textsubscript{load} = 2 mA, and the output range at
I\textsubscript{load} = 15 mA.

\textbf{Solution:} Input range: 2.5 - 0.4 = 2.1 V to 2.5 + 0.4 = 2.9 V.
Output range (ideal): 5 × 2.1 = 10.5 V to 5 × 2.9 = 14.5 V.

Wait -- the output exceeds V\textsubscript{CC} = 5.0 V. The gain of 5
centered at 2.5 V produces: V\textsubscript{out(center)} = 5 × 2.5 =
12.5 V -- far above the 5 V supply.

This circuit configuration is incorrect. For a gain of 5 with single
supply, the bias must be at V\textsubscript{CC}/(2 × gain) = 5/(2 × 5) =
0.5 V, or the input range must be limited.

\textbf{Corrected analysis} with gain = 2: Output range: 2 × 2.1 = 4.2 V
to 2 × 2.9 = 5.8 V.

At I\textsubscript{load} = 2 mA: V\textsubscript{OH} = 5.0 - 0.05 = 4.95
V. V\textsubscript{OL} = 0.03 V. Maximum output 5.8 V \textgreater{}
V\textsubscript{OH} = 4.95 V: \textbf{clipping occurs} on positive
peaks. Maximum undistorted input: V\textsubscript{in,max} =
V\textsubscript{OH}/2 = 4.95/2 = \textbf{2.475 V}. Usable input swing:
±(2.475 - 2.5) = the output clips immediately above the bias point.

\textbf{Better design:} Use gain = 2 with V\textsubscript{bias} =
V\textsubscript{CC}/4 = 1.25 V, input swing ±1.0 V. Output: 2 × 0.25 =
0.5 V to 2 × 2.25 = 4.5 V. Both within 0.03-4.95 V range.

At I\textsubscript{load} = 15 mA: V\textsubscript{sat} =
I\textsubscript{load} × R\textsubscript{DS(on)} = 0.015 × 75 = 1.125 V.
V\textsubscript{OL} ≈ 1.125 V, V\textsubscript{OH} ≈ 5.0 - 1.125 = 3.875
V. Output swing = 3.875 - 1.125 = \textbf{2.75 V\textsubscript{pp}}, a
45\% reduction from the 2 mA case.

\begin{center}\rule{0.5\linewidth}{0.5pt}\end{center}

\section{Problem 13.7.7}\label{problem-13.7.7}

\textbf{Given:} An inverting amplifier with gain = -10
(R\textsubscript{in} = 10 kΩ, R\textsubscript{f} = 100 kΩ) uses an
op-amp with V\textsubscript{OS} = 500 μV, SR = 2 V/μs, GBW = 3 MHz, and
supply rails of ±12 V. The input is a 200 mV\textsubscript{peak}, 50 kHz
sine wave.

\textbf{Find:} (a) The output offset error. (b) Whether slew-rate
limiting occurs. (c) Whether the output is within bandwidth.

\textbf{Solution:} \textbf{(a) Output offset error:}
V\textsubscript{out(OS)} = V\textsubscript{OS} × (1 +
R\textsubscript{f}/R\textsubscript{in}) = 500 × 10⁻⁶ × 11 = \textbf{5.5
mV}. The signal output = 10 × 200 mV = 2.0 V peak. Error = 5.5/2,000 ×
100\% = \textbf{0.275\%}.

\textbf{(b) Slew rate check:} Maximum output rate of change = 2π × f ×
V\textsubscript{p} = 2π × 50,000 × 2.0 = 628,318 V/s = 0.628 V/μs. Since
0.628 V/μs \textless{} SR = 2 V/μs: \textbf{no slew-rate limiting}.

Full-power bandwidth: f\textsubscript{FP} = SR/(2πV\textsubscript{p}) =
2 × 10⁶/(2π × 2.0) = \textbf{159 kHz}. 50 kHz \textless{} 159 kHz,
confirming no slew-rate distortion.

\textbf{(c) Bandwidth check:} Closed-loop bandwidth:
f\textsubscript{3dB} = GBW / \textbar A\textsubscript{CL noise
gain}\textbar{} = 3,000,000 / 11 = \textbf{273 kHz}. Since 50 kHz
\textless{} 273 kHz, the output is within bandwidth. Gain at 50 kHz:
\textbar A\textbar{} = 10/√(1 + (50/273)²) = 10/√(1.0335) = 10/1.0166 =
\textbf{9.84} (-0.14 dB from ideal).

\begin{center}\rule{0.5\linewidth}{0.5pt}\end{center}

\section{Problem 13.7.8}\label{problem-13.7.8}

\textbf{Given:} A precision instrumentation circuit requires an
input-referred noise of less than 1 μV\textsubscript{rms} in a 0.1 Hz to
10 Hz bandwidth. Two op-amps are being considered: - Op-amp A (bipolar):
e\textsubscript{n} = 3 nV/√Hz, 1/f corner f\textsubscript{c} = 2 Hz. -
Op-amp B (CMOS): e\textsubscript{n} = 20 nV/√Hz, 1/f corner
f\textsubscript{c} = 50 Hz.

\textbf{Find:} The total RMS noise for each op-amp in the 0.1--10 Hz
band, and which meets the requirement.

\textbf{Solution:} For noise in a band with 1/f component, the total
noise is: V\textsubscript{n} = e\textsubscript{n} × √(f\textsubscript{c}
× ln(f\textsubscript{H}/f\textsubscript{L}) + (f\textsubscript{H} -
f\textsubscript{L})).

\textbf{Op-amp A (bipolar):} V\textsubscript{n} = 3 × 10⁻⁹ × √(2 ×
ln(10/0.1) + (10 - 0.1)) = 3 × 10⁻⁹ × √(2 × 4.605 + 9.9) = 3 × 10⁻⁹ ×
√(9.21 + 9.9) = 3 × 10⁻⁹ × √19.11 = 3 × 10⁻⁹ × 4.372 = \textbf{13.1
nV\textsubscript{rms}} = 0.013 μV\textsubscript{rms}.

\textbf{Op-amp B (CMOS):} V\textsubscript{n} = 20 × 10⁻⁹ × √(50 ×
ln(10/0.1) + (10 - 0.1)) = 20 × 10⁻⁹ × √(50 × 4.605 + 9.9) = 20 × 10⁻⁹ ×
√(230.3 + 9.9) = 20 × 10⁻⁹ × √240.2 = 20 × 10⁻⁹ × 15.50 = \textbf{310
nV\textsubscript{rms}} = 0.31 μV\textsubscript{rms}.

Both meet the 1 μV\textsubscript{rms} requirement, but \textbf{Op-amp A
is 24× better} due to its much lower 1/f corner frequency. The CMOS
op-amp's high 1/f corner (50 Hz) causes its noise to rise significantly
in this low-frequency band.

\chapter{Chapter 14 --- Section 14.1: NEC Organization and
Structure}\label{chapter-14-section-14.1-nec-organization-and-structure}

Practice problems covering NEC code organization, chapter hierarchy, and
key table references for electrical installations.

\begin{center}\rule{0.5\linewidth}{0.5pt}\end{center}

\section{Problem 14.1.1}\label{problem-14.1.1}

\textbf{Given:} An engineer is designing a fire pump motor circuit rated
at 40 A full-load current in a healthcare facility. Article 430 (Chapter
4) requires branch circuit conductors at 125\% of FLC. Article 695
(Chapter 6) for fire pumps requires conductors to be sized at 125\% of
FLC and to be installed in a dedicated raceway. Article 517 (Chapter 5)
applies to healthcare facilities.

\textbf{Find:} The minimum conductor ampacity, identify which chapter's
wiring rules take precedence, and explain the NEC hierarchy that governs
conflicts between articles.

\textbf{Solution:} Minimum conductor ampacity per Article 430:
I\textsubscript{min} = 1.25 × 40 = \textbf{50 A}.

Article 695 (Chapter 6 --- Special Equipment) supplements and may modify
the requirements of Chapters 1 through 4. Article 517 (Chapter 5 ---
Special Occupancies) also supplements and may modify Chapters 1 through
4.

NEC hierarchy: Chapters 5, 6, and 7 can supplement or modify Chapters
1-4. When both Chapter 5 and Chapter 6 apply to the same installation,
both sets of additional requirements must be met.

Fire pump wiring per Article 695: conductors must be in a dedicated
raceway separate from other wiring, protected against physical damage,
and the overcurrent protection must not open under locked-rotor
conditions (695.7).

Healthcare facility per Article 517: the fire pump may need to be
connected to the essential electrical system (life safety branch per
517.26).

A 6 AWG THWN-2 copper conductor rated at 65 A at 75°C satisfies the
\textbf{50 A minimum}. Both Article 695 (dedicated raceway) and Article
517 (essential electrical system connection) requirements must be
applied simultaneously.

\begin{center}\rule{0.5\linewidth}{0.5pt}\end{center}

\section{Problem 14.1.2}\label{problem-14.1.2}

\textbf{Given:} An engineer must size conductors for a 150 A feeder in a
high-rise building. The conductors are copper THWN-2 (90°C) in EMT
conduit at 42°C ambient. The feeder supplies a mix of continuous and
non-continuous loads.

\textbf{Find:} Using Table 310.16 and Table 310.15(C)(1), identify the
base ampacity for 1/0 AWG, 2/0 AWG, and 3/0 AWG copper at both 75°C and
90°C column values, and apply the temperature correction factor for 42°C
ambient to find which size meets the 150 A requirement.

\textbf{Solution:} From Table 310.16 base ampacities:

{\def\LTcaptype{none} % do not increment counter
\begin{longtable}[]{@{}lll@{}}
\toprule\noalign{}
Size & 75°C column & 90°C column \\
\midrule\noalign{}
\endhead
\bottomrule\noalign{}
\endlastfoot
1/0 AWG & 150 A & 170 A \\
2/0 AWG & 175 A & 195 A \\
3/0 AWG & 200 A & 225 A \\
\end{longtable}
}

Temperature correction factor at 42°C ambient: - For 75°C insulation:
factor = 0.82. - For 90°C insulation: factor = 0.91.

Per 110.14(C), the 75°C column governs for termination purposes, but the
90°C ampacity may be used for derating calculations.

Using 90°C column for derating: - 1/0 AWG: 170 × 0.91 = 154.7 A. Check:
does not exceed 75°C termination value of 150 A. Use \textbf{150 A}
(governed by termination). - 2/0 AWG: 195 × 0.91 = 177.5 A. Check: does
not exceed 75°C value of 175 A. Use \textbf{175 A}. - 3/0 AWG: 225 ×
0.91 = 204.8 A. Check: exceeds 75°C value of 200 A, so limited to
\textbf{200 A}.

For a 150 A requirement: \textbf{1/0 AWG} copper THWN-2 is sufficient
(150 A ≥ 150 A), but has zero margin. Select \textbf{2/0 AWG} (175 A)
for engineering margin.

\begin{center}\rule{0.5\linewidth}{0.5pt}\end{center}

\section{Problem 14.1.3}\label{problem-14.1.3}

\textbf{Given:} An engineer needs the following information for a motor
circuit design: motor FLC for a 20 HP, 460 V, three-phase motor; the
equipment grounding conductor size for a 60 A breaker; and the conduit
fill percentage for three conductors.

\textbf{Find:} Identify the specific NEC table number for each piece of
information and provide the values.

\textbf{Solution:} \textbf{Motor FLC:} Table \textbf{430.250} (Full-Load
Current, Three-Phase AC Motors). For 20 HP at 460 V: FLC = \textbf{27
A}.

\textbf{Equipment grounding conductor:} Table \textbf{250.122} (Minimum
Size Equipment Grounding Conductors). For a 60 A overcurrent device: EGC
= \textbf{10 AWG} copper.

\textbf{Conduit fill percentage:} Chapter 9, Table \textbf{1} (Percent
of Cross Section of Conduit Occupied by Conductors). For three
conductors: \textbf{40\%} fill maximum.

These three tables, along with Table 310.16 (conductor ampacities) and
Table 430.52 (motor overcurrent protection), are among the most
frequently referenced tables in NEC-based electrical design.

\chapter{Chapter 14 --- Section 14.2: Conductor Sizing and
Ampacity}\label{chapter-14-section-14.2-conductor-sizing-and-ampacity}

Practice problems covering ampacity tables, temperature correction,
conduit fill adjustment, conductor resistance, conduit fill
calculations, and rooftop temperature adders.

\begin{center}\rule{0.5\linewidth}{0.5pt}\end{center}

\section{Problem 14.2.1}\label{problem-14.2.1}

\textbf{Given:} A 208 V, three-phase feeder supplies 80 A of continuous
load and 60 A of non-continuous load. The conductors are copper THWN-2
(90°C insulation) terminated on equipment rated for 75°C.

\textbf{Find:} The minimum required conductor ampacity and the conductor
size from Table 310.16.

\textbf{Solution:} Required ampacity = 1.25 × 80 (continuous) + 1.00 ×
60 (non-continuous) = 100 + 60 = \textbf{160 A}.

From Table 310.16 at 75°C column (termination limit): - 1/0 AWG copper =
150 A (insufficient, 150 \textless{} 160). - 2/0 AWG copper = 175 A
(sufficient, 175 ≥ 160).

Actual load current = 80 + 60 = 140 A. Overcurrent device must be rated
at least 160 A: next standard size per 240.6(A) = \textbf{175 A}.

Check: 2/0 AWG at 75°C = 175 A ≥ 175 A breaker rating.

Minimum conductor size: \textbf{2/0 AWG copper THWN-2}, protected by a
\textbf{175 A} circuit breaker.

\begin{center}\rule{0.5\linewidth}{0.5pt}\end{center}

\section{Problem 14.2.2}\label{problem-14.2.2}

\textbf{Given:} A set of 4 AWG THHN copper conductors (90°C rated) is
installed in a conduit in an attic space where the ambient temperature
reaches 48°C during summer.

\textbf{Find:} The corrected ampacity and compare with 4 AWG THW (75°C
rated) in the same conditions.

\textbf{Solution:} From Table 310.16 base ampacities: - 4 AWG at 90°C =
95 A. - 4 AWG at 75°C = 85 A.

Temperature correction factors at 48°C ambient: - For 90°C insulation:
factor = 0.82. - For 75°C insulation: factor = 0.65.

Corrected ampacities: - THHN (90°C): 95 × 0.82 = \textbf{77.9 A}. - THW
(75°C): 85 × 0.65 = \textbf{55.3 A}.

The 90°C insulation retains 82\% of its capacity versus only 65\% for
the 75°C insulation, providing a \textbf{41\% ampacity advantage} (77.9
vs.~55.3 A) at 48°C ambient.

\begin{center}\rule{0.5\linewidth}{0.5pt}\end{center}

\section{Problem 14.2.3}\label{problem-14.2.3}

\textbf{Given:} A conduit contains 8 current-carrying conductors: four
10 AWG THWN-2 copper conductors (two 20 A circuits) and four 6 AWG
THWN-2 copper conductors (two 40 A circuits). The ambient temperature is
40°C.

\textbf{Find:} The adjusted ampacity for both conductor sizes with
combined temperature and conduit fill derating.

\textbf{Solution:} Base ampacities from Table 310.16 at 90°C: - 10 AWG:
40 A. - 6 AWG: 75 A.

Temperature correction factor at 40°C for 90°C insulation:
\textbf{0.91}.

Conduit fill adjustment for 8 current-carrying conductors (7-9 range):
\textbf{0.70}.

Adjusted ampacities: - 10 AWG: 40 × 0.91 × 0.70 = \textbf{25.5 A}. - 6
AWG: 75 × 0.91 × 0.70 = \textbf{47.8 A}.

The 20 A circuit load is within the 10 AWG adjusted capacity of 25.5 A.
The 40 A circuit load is within the 6 AWG adjusted capacity of 47.8 A.

Both circuits are adequately sized, but with minimal margin on the 6 AWG
circuits (47.8 A vs.~40 A load = 19.5\% margin).

\begin{center}\rule{0.5\linewidth}{0.5pt}\end{center}

\section{Problem 14.2.4}\label{problem-14.2.4}

\textbf{Given:} A 250 kcmil copper conductor carries 200 A in a PVC
conduit at 60 Hz. From NEC Chapter 9, Table 9: R = 0.0541 Ω/1000 ft and
X = 0.0523 Ω/1000 ft for PVC conduit. The one-way run is 350 feet,
supplying a load at 0.80 power factor lagging on a 480 V three-phase
system.

\textbf{Find:} The per-phase voltage drop and the percent voltage drop.

\textbf{Solution:} Effective R per conductor: R = 0.0541 × 350/1000 =
0.01894 Ω. Effective X per conductor: X = 0.0523 × 350/1000 = 0.01831 Ω.

cos(θ) = 0.80, sin(θ) = 0.600.

Per-phase voltage drop: V\textsubscript{drop} = I × (R cos(θ) + X
sin(θ)) = 200 × (0.01894 × 0.80 + 0.01831 × 0.600) = 200 × (0.01515 +
0.01099) = 200 × 0.02614 = \textbf{5.23 V per phase}.

Three-phase line-to-line voltage drop: V\textsubscript{drop(LL)} = √3 ×
5.23 = \textbf{9.05 V}.

Percent voltage drop: \%V\textsubscript{drop} = 9.05 / 480 × 100 =
\textbf{1.89\%}.

This is within the NEC 3\% recommendation for feeders.

\begin{center}\rule{0.5\linewidth}{0.5pt}\end{center}

\section{Problem 14.2.5}\label{problem-14.2.5}

\textbf{Given:} The following THWN-2 conductors must be installed in a
single RMC (rigid metal conduit): six 3/0 AWG (two three-phase circuits)
and three 8 AWG (one three-phase control circuit). From Chapter 9, Table
5: 3/0 AWG THWN-2 area = 0.2679 in², 8 AWG THWN-2 area = 0.0366 in².

\textbf{Find:} The minimum RMC conduit size.

\textbf{Solution:} Total conductor area: 6 × 0.2679 + 3 × 0.0366 =
1.6074 + 0.1098 = \textbf{1.7172 in²}.

Total conductors = 9 (three or more), so 40\% fill applies. Required
conduit internal area: 1.7172 / 0.40 = \textbf{4.293 in²}.

From Chapter 9, Table 4 for RMC: - 2'' RMC: internal area = 3.408 in²
(40\% = 1.363 in² -- insufficient). - 2½'' RMC: internal area = 4.866
in² (40\% = 1.946 in² -- insufficient). - 3'' RMC: internal area = 7.499
in² (40\% = 3.000 in² -- insufficient).

Wait, let me reconsider. The total area is 1.7172 in², and 40\% fill
requires a conduit with internal area of 4.293 in².

\begin{itemize}
\tightlist
\item
  2½'' RMC: internal area = 4.866 in². 40\% fill = 1.946 in² -- this is
  larger than 1.7172 in². But wait: the allowable fill area (40\% of
  conduit) must exceed the total conductor area.
\item
  2½'' RMC: 0.40 × 4.866 = 1.946 in² \textgreater{} 1.7172 in².
\item
  2'' RMC: 0.40 × 3.408 = 1.363 in² \textless{} 1.7172 in² --
  insufficient.
\end{itemize}

Minimum conduit size: \textbf{2½'' RMC}.

Verification: fill = 1.7172 / 4.866 × 100 = \textbf{35.3\%} \textless{}
40\% limit.

\begin{center}\rule{0.5\linewidth}{0.5pt}\end{center}

\section{Problem 14.2.6}\label{problem-14.2.6}

\textbf{Given:} A solar inverter feeder conduit is mounted 3½ inches
above a flat roof surface on conduit standoffs. The outdoor ambient
temperature is 43°C. The conductors are 2 AWG THWN-2 copper (90°C
rated). Four current-carrying conductors are in the conduit.

\textbf{Find:} The effective ambient temperature, the combined derating
factor (temperature + conduit fill), and the derated ampacity.

\textbf{Solution:} Roof temperature adder at 3½ inches above roof:
\textbf{+17°C} per Table 310.15(C)(1)(a).

Effective ambient: 43 + 17 = \textbf{60°C}.

Temperature correction factor for 90°C insulation at 60°C ambient:
\textbf{0.71}.

Conduit fill adjustment for 4 current-carrying conductors (4-6 range):
\textbf{0.80}.

Base ampacity of 2 AWG THWN-2 at 90°C: \textbf{130 A}.

Derated ampacity: 130 × 0.71 × 0.80 = \textbf{73.8 A}.

Indoor comparison (30°C, 3 conductors, no derating): 130 A at 90°C,
limited to 115 A at 75°C for terminations.

The rooftop installation loses 43\% of the indoor capacity (73.8 vs.~130
A).

If the conduit were mounted directly on the roof surface (adder =
+33°C), effective ambient = 76°C, correction factor ≈ 0.37, and derated
ampacity = 130 × 0.37 × 0.80 = \textbf{38.5 A} -- barely adequate for
many solar inverter circuits.

\begin{center}\rule{0.5\linewidth}{0.5pt}\end{center}

\section{Problem 14.2.7}\label{problem-14.2.7}

\textbf{Given:} A nipple (conduit section 18 inches long) connects two
panels. It contains twelve 10 AWG THWN-2 copper conductors. The ambient
temperature is 30°C (standard).

\textbf{Find:} The adjusted ampacity, taking advantage of the 60\% fill
rule for nipples.

\textbf{Solution:} Base ampacity of 10 AWG THWN-2 at 90°C: \textbf{40
A}.

Nipples (≤ 24 inches) are permitted 60\% fill per Chapter 9, Table 1,
Note 4.

For ampacity derating, the number of current-carrying conductors still
applies: 12 conductors in the 10-20 range gives a conduit fill
adjustment factor of \textbf{0.50}.

Note: The 60\% fill rule applies to physical conduit fill calculations
(cross-sectional area), not to the ampacity derating. The ampacity
adjustment for more than 3 conductors still applies even in nipples per
310.15(C)(1). However, some jurisdictions allow relaxed derating for
nipples due to reduced thermal concern. Check with the AHJ.

Standard adjusted ampacity: 40 × 0.50 = \textbf{20 A} per conductor.

If the AHJ permits no ampacity derating for nipples (some
interpretations): the full \textbf{40 A} ampacity applies, limited to 30
A at 75°C termination temperature.

\begin{center}\rule{0.5\linewidth}{0.5pt}\end{center}

\section{Problem 14.2.8}\label{problem-14.2.8}

\textbf{Given:} A 480 V, three-phase, 300 A feeder runs 200 feet through
a steel conduit. The engineer is comparing two conductor options: -
Option A: 350 kcmil copper (R = 0.0382 Ω/1000 ft, X = 0.0441 Ω/1000 ft)
- Option B: 500 kcmil copper (R = 0.0293 Ω/1000 ft, X = 0.0439 Ω/1000
ft)

The load power factor is 0.90 lagging.

\textbf{Find:} The voltage drop for each option and determine which
meets the 2\% feeder drop recommendation.

\textbf{Solution:} cos(θ) = 0.90, sin(θ) = 0.436.

\textbf{Option A (350 kcmil):} V\textsubscript{drop} = √3 × 300 × 200 ×
(0.0382 × 0.90 + 0.0441 × 0.436) / 1000 = 519.6 × 200 × (0.03438 +
0.01923) / 1000 = 103,920 × 0.05361 / 1000 = \textbf{5.57 V}. Percent:
5.57/480 × 100 = \textbf{1.16\%}.

\textbf{Option B (500 kcmil):} V\textsubscript{drop} = √3 × 300 × 200 ×
(0.0293 × 0.90 + 0.0439 × 0.436) / 1000 = 103,920 × (0.02637 + 0.01914)
/ 1000 = 103,920 × 0.04551 / 1000 = \textbf{4.73 V}. Percent: 4.73/480 ×
100 = \textbf{0.99\%}.

Both options meet the 2\% recommendation. \textbf{Option A (350 kcmil)}
at 1.16\% is sufficient and more economical. However, Option B reduces
the voltage drop by 15\% and provides better motor starting performance.

\chapter{Chapter 14 --- Section 14.3: Overcurrent
Protection}\label{chapter-14-section-14.3-overcurrent-protection}

Practice problems covering standard OCPD ratings, tap rules,
coordination, short-circuit current calculations, and AFCI protection.

\begin{center}\rule{0.5\linewidth}{0.5pt}\end{center}

\section{Problem 14.3.1}\label{problem-14.3.1}

\textbf{Given:} A feeder has a calculated continuous load of 210 A and a
non-continuous load of 35 A. The conductors are copper THWN-2 with 75°C
terminations.

\textbf{Find:} The required conductor ampacity, the conductor size from
Table 310.16, and the overcurrent device rating from the standard sizes
in 240.6(A).

\textbf{Solution:} Required conductor ampacity: I\textsubscript{req} =
1.25 × 210 + 1.00 × 35 = 262.5 + 35 = \textbf{297.5 A}.

From Table 310.16 at 75°C: - 300 kcmil copper = 285 A (insufficient, 285
\textless{} 297.5). - 350 kcmil copper = 310 A (sufficient, 310 ≥
297.5).

Total load current = 210 + 35 = 245 A. Overcurrent device must be ≥
297.5 A.

Standard sizes per 240.6(A): 250, 300, 350 A. Next standard size at or
above 297.5 A: \textbf{300 A}.

Check per 240.4(B): for circuits ≤ 800 A, the next standard size above
the conductor ampacity is permitted. Conductor ampacity = 310 A. OCPD =
300 A ≤ 310 A.

Minimum: \textbf{350 kcmil copper THWN-2} with a \textbf{300 A} circuit
breaker.

\begin{center}\rule{0.5\linewidth}{0.5pt}\end{center}

\section{Problem 14.3.2}\label{problem-14.3.2}

\textbf{Given:} A 600 A feeder supplies a tap to a sub-panel 8 feet
away. The tap terminates in a 150 A main breaker. Copper THWN-2
conductors with 75°C terminations are used.

\textbf{Find:} Whether the 10-foot tap rule applies, the minimum tap
conductor size, and verify all conditions are met.

\textbf{Solution:} The tap length is 8 feet ≤ 10 feet, so the
\textbf{10-foot tap rule (240.21(B)(1))} applies.

Conditions for the 10-foot tap rule: 1. Tap conductors ≤ 10 feet:
\textbf{satisfied} (8 ft). 2. Tap conductor ampacity ≥ 10\% of upstream
OCPD: 0.10 × 600 = 60 A minimum. 3. Tap terminates in a single OCPD that
limits load to tap conductor ampacity: 150 A breaker. 4. Tap conductors
enclosed in raceway: must be satisfied.

The tap conductors must have an ampacity ≥ the 150 A load (since the
OCPD is 150 A).

From Table 310.16 at 75°C: 1/0 AWG copper = 150 A.

Minimum tap conductor: \textbf{1/0 AWG copper THWN-2} (150 A = 150 A
breaker rating).

Check condition 2: 150 A \textgreater\textgreater{} 60 A minimum.
\textbf{All conditions satisfied.}

\begin{center}\rule{0.5\linewidth}{0.5pt}\end{center}

\section{Problem 14.3.3}\label{problem-14.3.3}

\textbf{Given:} A distribution system has a 1200 A main breaker and a
400 A feeder breaker. At a fault current of 8,000 A, the 400 A breaker
clears in 0.03 seconds and the 1200 A breaker clears in 0.08 seconds. At
15,000 A (near the maximum available), the 400 A breaker clears in 0.017
seconds and the 1200 A breaker clears in 0.025 seconds.

\textbf{Find:} Whether the devices coordinate at both fault levels, and
identify the coordination ratio at each level.

\textbf{Solution:} At 8,000 A: Coordination ratio = upstream time /
downstream time = 0.08 / 0.03 = \textbf{2.67:1}. The upstream device is
2.67× slower -- adequate coordination (\textgreater{} 1.5:1 minimum).

At 15,000 A: Coordination ratio = 0.025 / 0.017 = \textbf{1.47:1}. The
upstream device is only 1.47× slower -- marginal, below the 1.5:1
guideline.

At the higher fault level, both devices are operating in or near their
instantaneous trip regions, where timing differences narrow.
\textbf{Coordination is lost} at very high fault currents approaching
the system maximum.

To improve coordination: (a) use a zone-selective interlocking (ZSI)
scheme that adds a restraint signal from downstream to upstream
breakers, or (b) select a main breaker with an adjustable short-time
delay that provides a definite time margin above the feeder breaker's
clearing time at all fault levels.

\begin{center}\rule{0.5\linewidth}{0.5pt}\end{center}

\section{Problem 14.3.4}\label{problem-14.3.4}

\textbf{Given:} A 750 kVA, 480Y/277 V, three-phase transformer has an
impedance of 5.0\%. The secondary conductors are 2 sets of 500 kcmil
copper in parallel per phase, running 50 feet to a main switchboard. The
``C'' constant for 500 kcmil copper is 26,706. Assume an infinite
primary bus.

\textbf{Find:} The available fault current at the transformer secondary
and at the switchboard.

\textbf{Solution:} Transformer secondary FLA: I\textsubscript{FLA} =
750,000 / (√3 × 480) = 750,000 / 831.4 = \textbf{902.1 A}.

Fault current at transformer secondary: I\textsubscript{sc(xfmr)} =
I\textsubscript{FLA} / Z\textsubscript{pu} = 902.1 / 0.05 =
\textbf{18,042 A}.

For parallel conductors, the effective ``C'' constant is multiplied by
the number of conductors per phase: C\textsubscript{eff} = 2 × 26,706 =
53,412.

Multiplier factor: f = (1.732 × L × I\textsubscript{sc}) /
(C\textsubscript{eff} × V\textsubscript{LL}) = (1.732 × 50 × 18,042) /
(53,412 × 480) = 1,562,438 / 25,637,760 = \textbf{0.06094}.

Fault current at switchboard: I\textsubscript{sc(swbd)} =
I\textsubscript{sc(xfmr)} / (1 + f) = 18,042 / 1.06094 = \textbf{17,005
A}.

All overcurrent devices at the switchboard must have an interrupting
rating ≥ \textbf{18 kAIC} (next standard rating above 17,005 A).
Standard ratings are 10, 14, 18, 22, 25, 35, 42, 50, 65, 100, and 200
kAIC.

\begin{center}\rule{0.5\linewidth}{0.5pt}\end{center}

\section{Problem 14.3.5}\label{problem-14.3.5}

\textbf{Given:} A dwelling unit has a 15 A, 120 V branch circuit serving
five duplex receptacles in a family room. The wiring is 14 AWG NM-B, 100
feet from the panel. An AFCI breaker is installed per 210.12. A series
arc develops in a damaged lamp cord with an arc current of 3 A.

\textbf{Find:} The arc power, why a standard breaker would not trip, and
the AFCI response.

\textbf{Solution:} Arc power: P\textsubscript{arc} = V × I = 120 × 3 =
\textbf{360 W}.

A standard 15 A breaker thermal trip point: approximately 135\% of
rating = 20.3 A after \textasciitilde1 hour. Magnetic (instantaneous)
trip: 5-10× rating = 75-150 A.

The 3 A arc current is only \textbf{20\%} of the 15 A breaker rating --
far below any standard trip threshold.

This 360 W arc would operate indefinitely, with surface temperatures at
the arc point easily exceeding 1,000°C -- far above the ignition
temperature of wood (\textasciitilde300°C), paper
(\textasciitilde230°C), and most fabrics (\textasciitilde250°C).

The \textbf{AFCI breaker} detects the arc signature: irregular,
sputtering current waveform with random amplitude variations (unlike the
smooth waveform of a resistive or motor load). The AFCI algorithm
identifies the arc pattern and trips within \textbf{8-60 half-cycles}
(67-500 ms at 60 Hz).

Per 210.12(A), AFCI protection is required for all 125 V, 15 A and 20 A
branch circuits in dwelling unit family rooms.

\begin{center}\rule{0.5\linewidth}{0.5pt}\end{center}

\section{Problem 14.3.6}\label{problem-14.3.6}

\textbf{Given:} A 25-foot tap from a 200 A feeder supplies a small
panelboard with a 70 A main breaker. The tap conductors are copper
THWN-2 with 75°C terminations.

\textbf{Find:} The minimum tap conductor size per the 25-foot tap rule
(240.21(B)(2)).

\textbf{Solution:} The 25-foot tap rule requires: 1. Tap conductors not
exceeding 25 feet: \textbf{satisfied} (25 ft exactly). 2. Tap conductor
ampacity ≥ 1/3 of upstream OCPD: 200/3 = \textbf{66.7 A} minimum. 3. Tap
terminates in a single OCPD: 70 A breaker -- \textbf{satisfied}. 4. Tap
conductors are protected from physical damage in a raceway.

From Table 310.16 at 75°C: - 6 AWG copper = 65 A (insufficient, 65
\textless{} 66.7). - 4 AWG copper = 85 A (sufficient, 85 ≥ 66.7).

Minimum tap conductor size: \textbf{4 AWG copper THWN-2}.

The 4 AWG conductors (85 A capacity) exceed the 70 A panelboard main
breaker, so the downstream loads are properly protected.

\begin{center}\rule{0.5\linewidth}{0.5pt}\end{center}

\section{Problem 14.3.7}\label{problem-14.3.7}

\textbf{Given:} A 1500 kVA, 13.8 kV / 480Y/277 V, three-phase
transformer has an impedance of 6.5\%. The utility provides a source
fault current of 10,000 A symmetrical at the transformer primary (13.8
kV side).

\textbf{Find:} The available fault current at the transformer secondary,
accounting for the finite (non-infinite) utility source.

\textbf{Solution:} Transformer secondary FLA: I\textsubscript{FLA(sec)}
= 1,500,000 / (√3 × 480) = 1,500,000 / 831.4 = \textbf{1,804 A}.

Transformer impedance on the secondary base: Z\textsubscript{xfmr(pu)} =
0.065.

Utility source impedance referred to secondary: First, find utility
source impedance on transformer kVA base: I\textsubscript{FLA(pri)} =
1,500,000 / (√3 × 13,800) = 1,500,000 / 23,901 = 62.76 A.
Z\textsubscript{source(pu)} = I\textsubscript{FLA(pri)} /
I\textsubscript{fault(pri)} = 62.76 / 10,000 = \textbf{0.006276 pu}.

Total impedance: Z\textsubscript{total(pu)} = Z\textsubscript{xfmr} +
Z\textsubscript{source} = 0.065 + 0.006276 = \textbf{0.07128 pu}.

Available fault current at secondary: I\textsubscript{sc(sec)} =
I\textsubscript{FLA(sec)} / Z\textsubscript{total(pu)} = 1,804 / 0.07128
= \textbf{25,308 A}.

Compare with infinite bus assumption: I\textsubscript{sc} = 1,804 /
0.065 = 27,754 A. The finite source reduces the available fault current
by \textbf{(27,754 - 25,308) / 27,754 × 100\% = 8.8\%}.

Overcurrent devices must have interrupting rating ≥ \textbf{25 kAIC} (or
30 kAIC for standard rating).

\chapter{Chapter 14 --- Section 14.4: Grounding and
Bonding}\label{chapter-14-section-14.4-grounding-and-bonding}

Practice problems covering system grounding, equipment grounding
conductors, ground fault protection, bonding, and GFCI protection.

\begin{center}\rule{0.5\linewidth}{0.5pt}\end{center}

\section{Problem 14.4.1}\label{problem-14.4.1}

\textbf{Given:} A 480Y/277 V, three-phase, 800 A service uses 2 sets of
500 kcmil copper service entrance conductors per phase. The grounding
electrode system consists of a driven ground rod and a concrete-encased
(Ufer) electrode. The total ground fault impedance on a 277 V circuit is
0.15 Ω.

\textbf{Find:} The minimum grounding electrode conductor (GEC) size per
Table 250.66, and the ground fault current magnitude.

\textbf{Solution:} From Table 250.66, for service entrance conductors
over 600 kcmil but not exceeding 1,100 kcmil (2 × 500 = 1,000 kcmil
equivalent per phase): Minimum GEC = \textbf{2/0 AWG copper}.

Ground fault current on a 277 V line-to-neutral circuit:
I\textsubscript{fault} = V / Z\textsubscript{total} = 277 / 0.15 =
\textbf{1,847 A}.

This current is well above a typical 20 A branch circuit breaker rating,
ensuring the breaker trips within the instantaneous region (typically
\textless{} 0.05 seconds).

Clearing energy: I²t = 1,847² × 0.05 = 170,570 A²·s. This low energy
minimizes damage at the fault point.

\begin{center}\rule{0.5\linewidth}{0.5pt}\end{center}

\section{Problem 14.4.2}\label{problem-14.4.2}

\textbf{Given:} A 480 V feeder is protected by a 100 A circuit breaker.
The original conductor size is 1 AWG copper (minimum per ampacity). Due
to a 500-foot run, the conductors are upsized to 4/0 AWG copper to limit
voltage drop.

\textbf{Find:} The minimum EGC size from Table 250.122 and the adjusted
EGC size per 250.122(B).

\textbf{Solution:} From Table 250.122 for a 100 A OCPD: Minimum EGC =
\textbf{8 AWG copper} (16,510 circular mils).

Per 250.122(B), proportionally increase the EGC: Size factor = area of
installed conductor / area of minimum required conductor. Installed: 4/0
AWG = 211,600 circular mils. Minimum required (1 AWG): 83,690 circular
mils. Factor = 211,600 / 83,690 = \textbf{2.528}.

Adjusted EGC area = 16,510 × 2.528 = 41,737 circular mils.

From wire gauge table: - 6 AWG = 26,240 cmil (insufficient). - 4 AWG =
41,740 cmil (sufficient, just barely).

Adjusted EGC: \textbf{4 AWG copper}.

\begin{center}\rule{0.5\linewidth}{0.5pt}\end{center}

\section{Problem 14.4.3}\label{problem-14.4.3}

\textbf{Given:} A 480Y/277 V, 1600 A service has ground fault protection
set at 1000 A pickup with a 0.3-second time delay. An arcing ground
fault develops with 1,200 A fault current that persists until cleared.

\textbf{Find:} Whether the GFPE detects the fault, the time to trip, and
the arc energy (I²t) dissipated.

\textbf{Solution:} GFPE pickup = 1,000 A. Fault current = 1,200 A.

Since 1,200 A \textgreater{} 1,000 A pickup: the GFPE \textbf{detects}
the fault.

Trip time = 0.3 seconds (the intentional time delay).

Arc energy: I²t = 1,200² × 0.3 = 1,440,000 × 0.3 = \textbf{432,000
A²·s}.

If the GFPE were set lower at 500 A pickup with 0.1 second delay: I²t =
1,200² × 0.1 = \textbf{144,000 A²·s} -- a 3× reduction in arc energy.

If the fault current were 800 A (below the 1,000 A pickup), the GFPE
would not trip, and the fault would persist until detected by other
means or until it escalated to a higher current level. This highlights
the importance of setting GFPE at the lowest practical level per NEC
230.95 (maximum 1,200 A with maximum 1-second delay).

\begin{center}\rule{0.5\linewidth}{0.5pt}\end{center}

\section{Problem 14.4.4}\label{problem-14.4.4}

\textbf{Given:} A commercial building has a 120/208 V, three-phase, 225
A service. The service entrance conductors are 4/0 AWG copper. A metal
gas pipe and a structural steel column are within 10 feet of the service
equipment.

\textbf{Find:} The bonding jumper size for both the gas pipe and the
structural steel per Table 250.66, and identify the bonding
requirements.

\textbf{Solution:} From Table 250.66 for 4/0 AWG service entrance
conductors: Minimum bonding jumper = \textbf{4 AWG copper}.

Per 250.104(A) --- Metal Piping Systems: the metal gas pipe must be
bonded to the grounding electrode system. Bond size from Table 250.66:
\textbf{4 AWG copper}.

Per 250.104(C) --- Structural Metal: if the structural steel is not
intentionally grounded and could become energized, it must be bonded.
Bond from Table 250.66: \textbf{4 AWG copper}.

Per 250.104(A), the bonding jumper for the gas pipe must be connected on
the street side of any gas meter or regulator that could be removed for
service.

The bonding connections must use listed clamps, compression connectors,
or exothermic welds. Sheet metal screws are not permitted for bonding
connections per 250.8.

\begin{center}\rule{0.5\linewidth}{0.5pt}\end{center}

\section{Problem 14.4.5}\label{problem-14.4.5}

\textbf{Given:} A commercial kitchen has a 20 A, 120 V GFCI-protected
branch circuit serving three duplex receptacles near a sink. The circuit
uses 12 AWG THHN in EMT conduit, 120 feet from the panel. Each
receptacle serves equipment with a typical leakage current of 0.8 mA.

\textbf{Find:} The total leakage current, the margin to the GFCI trip
threshold, and whether nuisance tripping is likely.

\textbf{Solution:} Total appliance leakage: 3 × 0.8 mA = \textbf{2.4
mA}.

Estimated cable leakage in EMT (lower than NM-B due to metallic
shielding): approximately 0.005 mA/ft. Cable leakage: 120 × 0.005 =
\textbf{0.6 mA}.

Total system leakage: 2.4 + 0.6 = \textbf{3.0 mA}.

GFCI Class A trip threshold: 5 mA (nominal midpoint of 4-6 mA range).
Margin: 5.0 - 3.0 = \textbf{2.0 mA} (40\% of trip threshold).

At the minimum trip point of 4 mA: margin = 4.0 - 3.0 = \textbf{1.0 mA}.

This is marginal. If a fourth appliance is added (0.8 mA) or if
equipment ages and leakage increases, total leakage could reach 4+ mA,
causing \textbf{nuisance tripping}.

Recommendation: Split the loads across two GFCI-protected circuits to
keep each circuit's leakage below 2 mA. Per 210.8(B)(2), GFCI protection
is required for kitchen receptacles within 6 feet of a sink in
non-dwelling occupancies.

\begin{center}\rule{0.5\linewidth}{0.5pt}\end{center}

\section{Problem 14.4.6}\label{problem-14.4.6}

\textbf{Given:} A 208Y/120 V, 600 A service with 500 kcmil copper
service entrance conductors requires a system grounding electrode
conductor. The grounding electrode system includes a 10-foot driven
ground rod (25 Ω to earth), a concrete-encased Ufer electrode (5 Ω to
earth), and a metal water pipe (3 Ω to earth).

\textbf{Find:} The GEC size per Table 250.66, the approximate parallel
ground resistance, and explain why the equipment grounding conductor
(not earth) carries the fault current.

\textbf{Solution:} From Table 250.66 for 500 kcmil service entrance
conductors: Minimum GEC = \textbf{1/0 AWG copper}.

Parallel ground resistance of all electrodes: 1/R\textsubscript{total} =
1/25 + 1/5 + 1/3 = 0.04 + 0.20 + 0.333 = 0.573. R\textsubscript{total} =
1/0.573 = \textbf{1.75 Ω}.

For a 120 V ground fault through the earth alone: I\textsubscript{earth}
= 120 / 1.75 = 68.6 A.

While 68.6 A might eventually trip a breaker, the clearing time would be
unacceptably long for personnel safety. More critically, on a 20 A
circuit, a 1.75 Ω earth path produces only 68.6 A, which is in the
thermal (slow) trip region of the breaker.

The \textbf{equipment grounding conductor} (metallic path) has an
impedance of typically 0.1-0.5 Ω total, producing fault currents of
240-1,200 A that trip breakers instantly. This is why NEC 250.4(A)(5)
requires an effective ground-fault current path through the EGC, not
through the earth. The earth connection provides lightning and surge
protection, not fault clearing.

\chapter{Chapter 14 --- Section 14.5: Motor
Circuits}\label{chapter-14-section-14.5-motor-circuits}

Practice problems covering motor FLC tables, branch circuit conductor
sizing, overcurrent protection, disconnecting means, VFD circuits, and
hazardous locations.

\begin{center}\rule{0.5\linewidth}{0.5pt}\end{center}

\section{Problem 14.5.1}\label{problem-14.5.1}

\textbf{Given:} A 50 HP, 460 V, three-phase motor has a nameplate FLA of
59 A. From NEC Table 430.250, the FLC for a 50 HP, 460 V motor is 65 A.
The motor is Design B with code letter H (locked-rotor kVA/HP =
6.3-7.09).

\textbf{Find:} The branch circuit conductor minimum ampacity, the
conductor size, and the locked-rotor current.

\textbf{Solution:} Branch circuit conductor ampacity per 430.22:
I\textsubscript{min} = 1.25 × FLC (from table) = 1.25 × 65 =
\textbf{81.25 A}.

From Table 310.16 at 75°C: 4 AWG copper = 85 A (sufficient, 85 ≥ 81.25).
Minimum conductor: \textbf{4 AWG copper THWN-2}.

Locked-rotor current (using mid-range code letter H = 6.7 kVA/HP):
kVA\textsubscript{LR} = 6.7 × 50 = 335 kVA. I\textsubscript{LR} =
kVA\textsubscript{LR} / (√3 × V) = 335,000 / (1.732 × 460) = 335,000 /
796.9 = \textbf{420 A}.

This is 420/65 = \textbf{6.5× the FLC}, typical for Design B motors.

Note: The NEC requires using the table FLC of 65 A (not the nameplate 59
A) for conductor and overcurrent device sizing.

\begin{center}\rule{0.5\linewidth}{0.5pt}\end{center}

\section{Problem 14.5.2}\label{problem-14.5.2}

\textbf{Given:} A motor control center (MCC) feeder supplies four motors
and a 50 A continuous heating load: - Motor A: 75 HP, 460 V, 3φ (FLC =
96 A) - Motor B: 30 HP, 460 V, 3φ (FLC = 40 A) - Motor C: 15 HP, 460 V,
3φ (FLC = 21 A) - Motor D: 5 HP, 460 V, 3φ (FLC = 7.6 A)

\textbf{Find:} The minimum feeder conductor ampacity per NEC 430.24.

\textbf{Solution:} Per 430.24: feeder ampacity = 1.25 × largest motor
FLC + sum of remaining motor FLCs + non-motor continuous loads × 1.25.

Largest motor: Motor A at 96 A.

I\textsubscript{feeder} = 1.25 × 96 + 40 + 21 + 7.6 + 1.25 × 50 = 120 +
40 + 21 + 7.6 + 62.5 = \textbf{251.1 A}.

From Table 310.16 at 75°C: - 4/0 AWG copper = 230 A (insufficient). -
250 kcmil copper = 255 A (sufficient, 255 ≥ 251.1).

Minimum feeder conductor: \textbf{250 kcmil copper}.

Total running load current: 96 + 40 + 21 + 7.6 + 50 = 214.6 A.

\begin{center}\rule{0.5\linewidth}{0.5pt}\end{center}

\section{Problem 14.5.3}\label{problem-14.5.3}

\textbf{Given:} A 25 HP, 460 V, three-phase, Design B motor has FLC = 34
A (Table 430.250) and nameplate FLA = 32 A with a service factor of 1.0.

\textbf{Find:} The overload relay setting, the maximum inverse-time
breaker size, and the maximum dual-element fuse size.

\textbf{Solution:} \textbf{Overload protection} per 430.32(A)(1): For
service factor \textless{} 1.15: trip at not more than 115\% of
nameplate FLA. I\textsubscript{OL} = 1.15 × 32 = \textbf{36.8 A} (set
relay trip to 36.8 A).

\textbf{Short-circuit/ground fault protection per Table 430.52:}

Inverse-time circuit breaker (Design B motor): maximum 250\% of FLC.
250\% × 34 = 85.0 A. Next standard size per 240.6(A): \textbf{90 A}.

Dual-element time-delay fuse (Design B motor): maximum 175\% of FLC.
175\% × 34 = 59.5 A. Next standard size: \textbf{60 A}.

Final motor branch circuit: - Conductors: 10 AWG copper (rated 35 A at
75°C ≥ 1.25 × 34 = 42.5 A -- \textbf{insufficient}). - Need 8 AWG copper
(rated 50 A at 75°C ≥ 42.5 A). - Overcurrent: 90 A inverse-time breaker
OR 60 A dual-element fuse. - Overload relay: set at 36.8 A.

\begin{center}\rule{0.5\linewidth}{0.5pt}\end{center}

\section{Problem 14.5.4}\label{problem-14.5.4}

\textbf{Given:} A 100 HP, 460 V, three-phase motor has FLC = 124 A
(Table 430.250). The motor code letter is G (kVA/HP = 5.6-6.29).

\textbf{Find:} The minimum disconnect ampere rating and the locked-rotor
current the disconnect must handle.

\textbf{Solution:} Minimum disconnect rating per 430.110(A):
I\textsubscript{disc} = 1.15 × FLC = 1.15 × 124 = \textbf{142.6 A}.

Next standard disconnect size: \textbf{200 A} (common available size).

Locked-rotor current (mid-range code letter G = 5.9 kVA/HP):
kVA\textsubscript{LR} = 5.9 × 100 = 590 kVA. I\textsubscript{LR} =
590,000 / (√3 × 460) = 590,000 / 796.9 = \textbf{740 A}.

The 200 A motor-circuit disconnect switch must have horsepower rating
adequate for 100 HP at 460 V and be capable of interrupting the
locked-rotor current of 740 A.

The disconnect must be within sight (visible and ≤ 50 feet) of the motor
per 430.102(B).

\begin{center}\rule{0.5\linewidth}{0.5pt}\end{center}

\section{Problem 14.5.5}\label{problem-14.5.5}

\textbf{Given:} A 40 HP, 460 V, three-phase motor is driven by a VFD
with rated input current of 52 A. The VFD is 150 feet from the motor.
The motor FLC from Table 430.250 is 52 A. The VFD output has a PWM
switching frequency of 8 kHz with 200 ns rise time.

\textbf{Find:} The supply-side conductor size, the overcurrent
protection, and the critical cable length for reflected voltage waves.

\textbf{Solution:} \textbf{Supply-side conductors} (to VFD input):
Ampacity = 1.25 × VFD input current = 1.25 × 52 = \textbf{65 A}. From
Table 310.16 at 75°C: 6 AWG copper = 65 A (sufficient).

\textbf{Overcurrent protection} per Table 430.52 (Design B, inverse-time
breaker): 250\% × 52 A (motor FLC from table) = 130 A. Next standard
size: \textbf{125 A} (430.52(C)(1) permits using the next standard size
up only if the standard size is insufficient for starting -- 125 A may
be adequate). If motor won't start: use \textbf{150 A}.

\textbf{Motor-side conductors:} Ampacity = 1.25 × 52 = 65 A. Use
\textbf{6 AWG copper}.

\textbf{Critical cable length} for reflected waves:
L\textsubscript{crit} = v × t\textsubscript{rise} / 2, where v ≈ 150
m/μs for typical cable. L\textsubscript{crit} = (150 × 10⁶) × (200 ×
10⁻⁹) / 2 = 30 / 2 = \textbf{15 m ≈ 49 feet}.

Since 150 feet \textgreater\textgreater{} 49 feet, reflected voltage
waves will fully develop at the motor terminals, potentially reaching 2×
DC bus voltage = 2 × 1.35 × V\textsubscript{LL} = 2 × 1.35 × 460 =
\textbf{1,242 V peak} (≈1,200 V).

An \textbf{output reactor or dv/dt filter} is recommended to limit motor
terminal voltage below 1,000 V peak. VFD-rated cable with symmetrical
ground conductors should be used.

Per 430.130(B), the EGC must be \textbf{wire-type} (conduit alone is
insufficient).

\begin{center}\rule{0.5\linewidth}{0.5pt}\end{center}

\section{Problem 14.5.6}\label{problem-14.5.6}

\textbf{Given:} A grain elevator (Class II, Division 1, Group G -- grain
dust, auto-ignition 400°C) requires a 10 HP, 460 V motor driving a
bucket elevator. The area classification extends 10 feet from any grain
handling equipment.

\textbf{Find:} The location classification, the motor enclosure type,
the wiring method, and the required equipment temperature code.

\textbf{Solution:} Classification: \textbf{Class II, Division 1, Group
G} (combustible dust present under normal operations during grain
handling).

Temperature code: Grain dust auto-ignition = 400°C. Equipment surface
temperature must be well below 400°C. A \textbf{T3} rating (200°C
maximum) provides adequate margin and is the standard rating for Class
II motors.

Motor enclosure: Must be \textbf{dust-ignitionproof} per Article 502.125
-- designed so dust cannot enter the enclosure and surface temperatures
do not ignite the surrounding dust layer.

Wiring method per Article 502.10(A)(1): - \textbf{Threaded rigid metal
conduit (RMC)} or - \textbf{Threaded intermediate metal conduit (IMC)},
or - \textbf{Type MI cable} with listed termination fittings.

All fittings must be dust-tight. Flexible connections at the motor (for
vibration isolation) require dust-tight flexible fittings listed for
Class II, Division 1.

Sealing fittings are required at boundaries between the Division 1 area
and unclassified areas to prevent dust migration through the conduit
system.

\begin{center}\rule{0.5\linewidth}{0.5pt}\end{center}

\section{Problem 14.5.7}\label{problem-14.5.7}

\textbf{Given:} A 20 HP, 208 V, three-phase motor has FLC = 59.4 A (from
Table 430.250). The motor branch circuit uses a dual-element time-delay
fuse for short-circuit protection and a thermal overload relay. The
motor nameplate shows FLA = 56 A and SF = 1.15.

\textbf{Find:} The overload relay trip setting, the maximum fuse size,
and the minimum conductor ampacity.

\textbf{Solution:} \textbf{Overload relay} per 430.32(A)(1): SF ≥ 1.15:
trip at ≤ 125\% of nameplate FLA. I\textsubscript{OL} = 1.25 × 56 =
\textbf{70 A}.

\textbf{Short-circuit protection} (dual-element fuse, Design B): Maximum
= 175\% × FLC = 1.75 × 59.4 = \textbf{103.95 A}. Next standard fuse
size: \textbf{100 A}.

If motor will not start with 100 A fuse, 430.52(C)(1) Exception 2
permits up to 225\% × FLC = 133.65 A. Next standard: \textbf{125 A}.

\textbf{Conductor ampacity:} 1.25 × FLC = 1.25 × 59.4 = \textbf{74.25
A}. From Table 310.16 at 75°C: 4 AWG copper = 85 A (sufficient).

The circuit: \textbf{4 AWG copper} conductors, \textbf{100 A}
dual-element fuse, overload relay at \textbf{70 A}.

\chapter{Chapter 14 --- Section 14.6: Transformer
Connections}\label{chapter-14-section-14.6-transformer-connections}

Practice problems covering primary-only protection, dual protection,
secondary conductors, separately derived systems, neutral sizing, and
K-factor transformers.

\begin{center}\rule{0.5\linewidth}{0.5pt}\end{center}

\section{Problem 14.6.1}\label{problem-14.6.1}

\textbf{Given:} A 45 kVA, single-phase, 480 V primary / 120/240 V
secondary transformer is installed with primary-only overcurrent
protection.

\textbf{Find:} The primary rated current, the maximum primary
overcurrent device size per Table 450.3(B), and the secondary rated
current.

\textbf{Solution:} Primary rated current: I\textsubscript{primary} = S /
V = 45,000 / 480 = \textbf{93.75 A}.

Maximum primary OCPD per Table 450.3(B) (primary only): 125\% of rated
current = 1.25 × 93.75 = 117.2 A. Next standard size per 240.6(A):
\textbf{125 A}.

Secondary rated current: I\textsubscript{secondary} = 45,000 / 240 =
\textbf{187.5 A}.

Since there is no secondary overcurrent protection, the secondary
conductors must be rated for at least 187.5 A. From Table 310.16 at
75°C: 3/0 AWG copper = 200 A (sufficient for \textbf{187.5 A}).

\begin{center}\rule{0.5\linewidth}{0.5pt}\end{center}

\section{Problem 14.6.2}\label{problem-14.6.2}

\textbf{Given:} A 300 kVA, three-phase, 480 V delta primary / 208Y/120 V
secondary transformer has an impedance of 4.5\%. Overcurrent protection
is provided on both primary and secondary.

\textbf{Find:} The primary and secondary OCPD sizes, and the maximum
secondary fault current.

\textbf{Solution:} Primary rated current: I\textsubscript{primary} =
300,000 / (√3 × 480) = 300,000 / 831.4 = \textbf{360.8 A}.

Maximum primary OCPD (250\% with secondary protection): 2.50 × 360.8 =
902.1 A. Next standard size: \textbf{900 A} (or 800 A for more
conservative design).

Secondary rated current: I\textsubscript{secondary} = 300,000 / (√3 ×
208) = 300,000 / 360.3 = \textbf{832.6 A}.

Maximum secondary OCPD (125\%): 1.25 × 832.6 = 1,040.8 A. Next standard
size: \textbf{1,000 A}.

Check: The 1,000 A OCPD does not exceed the conductor ampacity.
Secondary conductors must be rated for at least 832.6 A.

Maximum secondary fault current (infinite primary bus):
I\textsubscript{fault} = I\textsubscript{secondary} /
Z\textsubscript{pu} = 832.6 / 0.045 = \textbf{18,502 A}.

The secondary OCPD must have an interrupting rating ≥ \textbf{22 kAIC}
(next standard above 18,502 A).

\begin{center}\rule{0.5\linewidth}{0.5pt}\end{center}

\section{Problem 14.6.3}\label{problem-14.6.3}

\textbf{Given:} A 75 kVA, three-phase, 480/208Y/120 V transformer has
its primary protected by a 125 A breaker. The secondary conductors run
22 feet to a panelboard with a 250 A main breaker.

\textbf{Find:} The minimum secondary conductor size per the 240.21(C)(6)
25-foot tap rule.

\textbf{Solution:} Secondary rated current: I\textsubscript{sec} =
75,000 / (√3 × 208) = 75,000 / 360.3 = \textbf{208.2 A}.

The 25-foot rule (240.21(C)(6)) applies: tap length = 22 feet ≤ 25 feet.

The secondary conductors must be sized for at least the transformer
secondary rated current: 208.2 A.

Primary OCPD equivalent on secondary: 125 × (480/208) = 125 × 2.308 =
288.5 A equivalent. One-third: 288.5/3 = 96.2 A.

The governing requirement is the larger of 208.2 A (transformer
capacity) and 96.2 A (one-third rule).

From Table 310.16 at 75°C: - 3/0 AWG copper = 200 A (insufficient, 200
\textless{} 208.2). - 4/0 AWG copper = 230 A (sufficient).

Minimum secondary conductor: \textbf{4/0 AWG copper} per phase, with
22-foot run to 250 A main breaker.

\begin{center}\rule{0.5\linewidth}{0.5pt}\end{center}

\section{Problem 14.6.4}\label{problem-14.6.4}

\textbf{Given:} A 500 kVA, 480 V delta / 208Y/120 V wye transformer is a
separately derived system. The secondary conductors are 2 sets of 350
kcmil copper per phase (700 kcmil equivalent).

\textbf{Find:} The system bonding jumper size and the grounding
electrode conductor size per Article 250.30.

\textbf{Solution:} System bonding jumper per 250.30(A)(1) and Table
250.102(C)(1): For the equivalent of 700 kcmil ungrounded conductors
(between 500 and 750 kcmil range): Minimum bonding jumper = \textbf{1/0
AWG copper}.

Grounding electrode conductor per 250.30(A)(5) and Table 250.66: For 700
kcmil equivalent: the GEC = \textbf{2/0 AWG copper}.

However, per 250.30(A)(5), if using a structural metal electrode or
concrete-encased electrode, the GEC need not be larger than \textbf{3/0
AWG copper} (the maximum per the exception).

The neutral-to-ground bond is made at the transformer secondary terminal
compartment only. Downstream, the neutral is kept isolated (insulated)
from enclosures and ground.

\begin{center}\rule{0.5\linewidth}{0.5pt}\end{center}

\section{Problem 14.6.5}\label{problem-14.6.5}

\textbf{Given:} A 112.5 kVA, 480 V delta / 208Y/120 V wye transformer
serves an office floor with predominantly computer loads (SMPS). The
balanced three-phase load draws 312 A per phase, with measured harmonic
content: 3rd harmonic = 28\%, 5th = 10\%, 7th = 4\%.

\textbf{Find:} The neutral current from triplen harmonics and the
minimum neutral conductor size.

\textbf{Solution:} Third harmonic current per phase: I₃ = 0.28 × 312 =
87.4 A.

Triplen harmonics add in the neutral: I\textsubscript{N(3rd)} = 3 × 87.4
= \textbf{262.1 A}.

The 5th and 7th harmonics are not triplen (not multiples of 3) and
cancel in a balanced three-phase system, so they do not contribute to
neutral current.

The 9th harmonic (if present) would also be triplen and add to the
neutral. Assuming 9th = 5\%: I₉ = 0.05 × 312 = 15.6 A per phase.
I\textsubscript{N(9th)} = 3 × 15.6 = 46.8 A.

Total neutral current (RSS of triplen components): I\textsubscript{N} =
√(262.1² + 46.8²) = √(68,696 + 2,190) = √70,886 = \textbf{266.2 A}.

The neutral current is 266.2/312 = \textbf{85.3\%} of the phase current.

From Table 310.16 at 75°C for 266.2 A: - 300 kcmil copper = 285 A
(sufficient).

Minimum neutral conductor: \textbf{300 kcmil copper}.

Per 310.15(E), this neutral counts as a current-carrying conductor for
conduit fill derating purposes.

\begin{center}\rule{0.5\linewidth}{0.5pt}\end{center}

\section{Problem 14.6.6}\label{problem-14.6.6}

\textbf{Given:} A standard 225 kVA, 480/208Y/120 V dry-type transformer
(K-1 rated, P\textsubscript{EC-R} = 10\% of rated load loss) serves a
building with the following measured harmonic spectrum: I₁ = 1.00, I₃ =
0.65, I₅ = 0.45, I₇ = 0.25, I₉ = 0.12, I₁₁ = 0.05.

\textbf{Find:} The K-factor, the required K-rated transformer, and the
derated capacity if the existing K-1 transformer is used.

\textbf{Solution:} K-factor = Σ(I\textsubscript{h}² × h²) /
Σ(I\textsubscript{h}²).

Numerator: 1.00² × 1² + 0.65² × 3² + 0.45² × 5² + 0.25² × 7² + 0.12² ×
9² + 0.05² × 11² = 1.000 + 0.4225 × 9 + 0.2025 × 25 + 0.0625 × 49 +
0.0144 × 81 + 0.0025 × 121 = 1.000 + 3.803 + 5.063 + 3.063 + 1.166 +
0.303 = \textbf{14.398}.

Denominator: 1.000 + 0.4225 + 0.2025 + 0.0625 + 0.0144 + 0.0025 =
\textbf{1.704}.

K = 14.398 / 1.704 = \textbf{8.45}.

Required K-rated transformer: \textbf{K-13} (next standard K-rating
above 8.45).

Derating the K-1 transformer per IEEE C57.110: I\textsubscript{max} =
√(1 / (1 + P\textsubscript{EC-R} × (K - 1))) = √(1 / (1 + 0.10 × (8.45 -
1))) = √(1 / (1 + 0.745)) = √(1 / 1.745) = √0.573 = \textbf{0.757 pu}.

Derated capacity: 0.757 × 225 = \textbf{170.3 kVA}.

The transformer can safely handle \textbf{170 kVA} of this harmonic-rich
load, a \textbf{24.3\% reduction} from its nameplate rating. If the
building load exceeds 170 kVA, either install a K-13 transformer or add
harmonic filtering.

\begin{center}\rule{0.5\linewidth}{0.5pt}\end{center}

\section{Problem 14.6.7}\label{problem-14.6.7}

\textbf{Given:} A 50 kVA, 240 V delta primary / 120/240 V secondary,
single-phase transformer has primary-and-secondary protection. The
transformer impedance is 3.0\%.

\textbf{Find:} The primary OCPD size, the secondary OCPD size, and the
maximum secondary fault current.

\textbf{Solution:} Primary rated current: I\textsubscript{primary} =
50,000 / 240 = \textbf{208.3 A}.

Maximum primary OCPD (250\% with secondary protection): 2.50 × 208.3 =
520.8 A. Next standard: \textbf{500 A}.

Secondary rated current: I\textsubscript{secondary} = 50,000 / 240 =
\textbf{208.3 A}.

Maximum secondary OCPD (125\%): 1.25 × 208.3 = 260.4 A. Next standard:
\textbf{250 A}.

Maximum secondary fault current: I\textsubscript{fault} =
I\textsubscript{secondary} / Z\textsubscript{pu} = 208.3 / 0.03 =
\textbf{6,943 A}.

The 250 A secondary OCPD must have an interrupting rating ≥ \textbf{10
kAIC} (standard rating above 6,943 A).

\chapter{Chapter 14 --- Section 14.7: Voltage Drop
Calculations}\label{chapter-14-section-14.7-voltage-drop-calculations}

Practice problems covering single-phase voltage drop, three-phase
voltage drop, parallel conductors, conductor sizing for voltage drop,
and service load calculations.

\begin{center}\rule{0.5\linewidth}{0.5pt}\end{center}

\section{Problem 14.7.1}\label{problem-14.7.1}

\textbf{Given:} A 120 V, single-phase branch circuit supplies a 12 A
continuous load at the end of a 200-foot run. The conductors are 10 AWG
copper in PVC conduit (R = 1.21 Ω/1000 ft, X = 0.050 Ω/1000 ft). The
load power factor is 0.95 lagging.

\textbf{Find:} The voltage drop, the percent voltage drop, and whether
it meets the NEC 3\% recommendation.

\textbf{Solution:} cos(θ) = 0.95, sin(θ) = 0.312.

V\textsubscript{drop} = 2 × I × L × (R cos(θ) + X sin(θ)) / 1000 = 2 ×
12 × 200 × (1.21 × 0.95 + 0.050 × 0.312) / 1000 = 4,800 × (1.1495 +
0.01560) / 1000 = 4,800 × 1.1651 / 1000 = \textbf{5.59 V}.

Percent voltage drop: 5.59 / 120 × 100 = \textbf{4.66\%}.

This exceeds the NEC 3\% recommendation for branch circuits.

To achieve 3\%: V\textsubscript{drop(max)} = 0.03 × 120 = 3.6 V.
Required R\textsubscript{max} = 3.6 × 1000 / (2 × 12 × 200 × 0.95) =
3,600 / 4,560 = 0.789 Ω/1000 ft.

From Chapter 9, Table 9: 8 AWG copper in PVC = 0.764 Ω/1000 ft
(sufficient). V\textsubscript{drop} = 2 × 12 × 200 × (0.764 × 0.95 +
0.050 × 0.312) / 1000 = 4,800 × 0.7414 / 1000 = \textbf{3.56 V} =
2.97\%.

Upgrade to \textbf{8 AWG copper} for NEC compliance.

\begin{center}\rule{0.5\linewidth}{0.5pt}\end{center}

\section{Problem 14.7.2}\label{problem-14.7.2}

\textbf{Given:} A 480 V, three-phase feeder supplies a 250 A motor load
at 0.88 power factor lagging. The run is 300 feet using 350 kcmil copper
in steel conduit (R = 0.0382 Ω/1000 ft, X = 0.0441 Ω/1000 ft).

\textbf{Find:} The running voltage drop, the percent drop, and the
voltage drop during motor starting at 6× FLC.

\textbf{Solution:} cos(θ) = 0.88, sin(θ) = 0.475.

Running voltage drop: V\textsubscript{drop} = √3 × I × L × (R cos(θ) + X
sin(θ)) / 1000 = 1.732 × 250 × 300 × (0.0382 × 0.88 + 0.0441 × 0.475) /
1000 = 129,900 × (0.03362 + 0.02095) / 1000 = 129,900 × 0.05457 / 1000 =
\textbf{7.09 V}.

Percent: 7.09 / 480 × 100 = \textbf{1.48\%} (within 3\% recommendation).

Voltage at motor terminals: 480 - 7.09 = \textbf{472.9 V}.

During motor starting (I\textsubscript{start} = 6 × 250 = 1,500 A):
Starting PF ≈ 0.30 lagging. cos(θ) = 0.30, sin(θ) = 0.954.
V\textsubscript{drop(start)} = 1.732 × 1,500 × 300 × (0.0382 × 0.30 +
0.0441 × 0.954) / 1000 = 779,400 × (0.01146 + 0.04209) / 1000 = 779,400
× 0.05355 / 1000 = \textbf{41.74 V}.

Starting voltage: 480 - 41.74 = \textbf{438.3 V} = 91.3\% of rated. This
is above the 80\% minimum typically needed for successful motor
starting.

\begin{center}\rule{0.5\linewidth}{0.5pt}\end{center}

\section{Problem 14.7.3}\label{problem-14.7.3}

\textbf{Given:} A 208 V, three-phase, 600 A feeder uses three 300 kcmil
copper conductors per phase in parallel, installed in three separate PVC
conduits. The run is 350 feet at 0.85 power factor lagging. From Table
9: R = 0.0437 Ω/1000 ft, X = 0.0527 Ω/1000 ft for 300 kcmil in PVC.

\textbf{Find:} The voltage drop with parallel conductors and the percent
drop.

\textbf{Solution:} Effective impedance per phase (three conductors in
parallel): R\textsubscript{eff} = 0.0437 / 3 = 0.01457 Ω/1000 ft.
X\textsubscript{eff} = 0.0527 / 3 = 0.01757 Ω/1000 ft.

cos(θ) = 0.85, sin(θ) = 0.527.

V\textsubscript{drop} = √3 × 600 × 350 × (0.01457 × 0.85 + 0.01757 ×
0.527) / 1000 = 1.732 × 600 × 350 × (0.01238 + 0.00926) / 1000 = 363,720
× 0.02164 / 1000 = \textbf{7.87 V}.

Percent: 7.87 / 208 × 100 = \textbf{3.78\%}.

This exceeds the 3\% recommendation. Options: 1. Increase to four
conductors per phase: R\textsubscript{eff} = 0.0437/4,
V\textsubscript{drop} ≈ 5.90 V = \textbf{2.84\%} (acceptable). 2. Upsize
to 350 kcmil per parallel set: lower R and similar result. 3. Accept
3.78\% if the branch circuit drop is minimal (total ≤ 5\%).

\begin{center}\rule{0.5\linewidth}{0.5pt}\end{center}

\section{Problem 14.7.4}\label{problem-14.7.4}

\textbf{Given:} A 240 V, single-phase branch circuit supplies a 30 A
continuous load (electric vehicle charger) at the end of a 100-foot run
in PVC conduit. Maximum allowable voltage drop is 3\%.

\textbf{Find:} The minimum conductor size to meet both ampacity and
voltage drop requirements.

\textbf{Solution:} \textbf{Ampacity requirement:} 1.25 × 30 = 37.5 A
continuous load. From Table 310.16 at 75°C: 8 AWG copper = 50 A
(sufficient).

\textbf{Voltage drop requirement:} V\textsubscript{drop(max)} = 0.03 ×
240 = 7.2 V.

At unity PF (resistive EV charger load): R\textsubscript{max} =
V\textsubscript{drop} × 1000 / (2 × I × L) = 7.2 × 1000 / (2 × 30 × 100)
= 7,200 / 6,000 = \textbf{1.20 Ω/1000 ft}.

From Chapter 9, Table 9 (PVC conduit): - 10 AWG: R = 1.21 Ω/1000 ft.
V\textsubscript{drop} = 2 × 30 × 100 × 1.21/1000 = 7.26 V = 3.03\%
(marginal). - 8 AWG: R = 0.764 Ω/1000 ft. V\textsubscript{drop} = 2 × 30
× 100 × 0.764/1000 = 4.58 V = \textbf{1.91\%} (meets 3\%).

The \textbf{8 AWG copper} satisfies both ampacity (50 A ≥ 37.5 A) and
voltage drop (1.91\% \textless{} 3\%).

However, if the run were 200 feet: V\textsubscript{drop} = 2 × 30 × 200
× 0.764/1000 = 9.17 V = 3.82\% (exceeds 3\%). Would need 6 AWG (R =
0.491): V\textsubscript{drop} = 5.89 V = \textbf{2.46\%}.

\begin{center}\rule{0.5\linewidth}{0.5pt}\end{center}

\section{Problem 14.7.5}\label{problem-14.7.5}

\textbf{Given:} A 1,800 ft² dwelling has: general lighting (3 VA/ft²),
two small-appliance circuits, one laundry circuit, one 8 kW electric
range, one 4.5 kW clothes dryer, one 4 kW water heater, one 3-ton (3.5
kW) heat pump, and one 15 kW electric furnace (backup heating).

\textbf{Find:} The service demand load using the standard method and the
minimum service size.

\textbf{Solution:} \textbf{General lighting:} 1,800 × 3 = 5,400 VA.
\textbf{Small-appliance circuits:} 2 × 1,500 = 3,000 VA.
\textbf{Laundry:} 1,500 VA. Total general: 5,400 + 3,000 + 1,500 =
\textbf{9,900 VA}.

Table 220.42 demand factors: First 3,000 VA at 100\% = 3,000 VA.
Remaining 6,900 VA at 35\% = 2,415 VA. Net general = 3,000 + 2,415 =
\textbf{5,415 VA}.

\textbf{Range:} Table 220.55, Column C for one range ≤ 12 kW:
\textbf{8,000 VA}. (The 8 kW nameplate is ≤ 12 kW, so the demand is 8 kW
per column C.)

\textbf{Dryer:} Table 220.54: 5,000 VA minimum or nameplate, whichever
is larger = \textbf{5,000 VA}.

\textbf{Water heater:} 4,000 VA at 100\% (fewer than 4 fixed appliances)
= \textbf{4,000 VA}.

\textbf{Heating/cooling (220.60):} Use the larger load (they don't
operate simultaneously). Electric furnace: 15,000 VA \textgreater{} heat
pump: 3,500 VA. Use heating: \textbf{15,000 VA}.

Total demand: 5,415 + 8,000 + 5,000 + 4,000 + 15,000 = \textbf{37,415
VA}.

Service current (240 V, single-phase): I = 37,415 / 240 = \textbf{155.9
A}.

Minimum service per 230.79: at least 100 A. For 155.9 A, select
\textbf{200 A} service.

Service entrance conductors: From Table 310.16 at 75°C: - 2/0 AWG copper
= 175 A (insufficient for 200 A OCPD). - 4/0 AWG copper = 230 A
(sufficient).

Use \textbf{4/0 AWG copper} service entrance conductors with \textbf{200
A} main breaker.

\begin{center}\rule{0.5\linewidth}{0.5pt}\end{center}

\section{Problem 14.7.6}\label{problem-14.7.6}

\textbf{Given:} A 277 V, single-phase lighting circuit supplies 20 A at
0.98 power factor (LED lighting with power factor correction). The run
is 250 feet in EMT conduit using 10 AWG copper (R = 1.18 Ω/1000 ft, X =
0.044 Ω/1000 ft).

\textbf{Find:} The voltage drop and whether it meets the 3\% branch
circuit recommendation.

\textbf{Solution:} cos(θ) = 0.98, sin(θ) = 0.199.

V\textsubscript{drop} = 2 × I × L × (R cos(θ) + X sin(θ)) / 1000 = 2 ×
20 × 250 × (1.18 × 0.98 + 0.044 × 0.199) / 1000 = 10,000 × (1.1564 +
0.00876) / 1000 = 10,000 × 1.165 / 1000 = \textbf{11.65 V}.

Percent: 11.65 / 277 × 100 = \textbf{4.21\%} (exceeds 3\%).

To meet 3\%: V\textsubscript{drop(max)} = 0.03 × 277 = 8.31 V.
R\textsubscript{max} = 8.31 × 1000 / (2 × 20 × 250 × 0.98) = 8,310 /
9,800 = 0.848 Ω/1000 ft.

From Table 9: 8 AWG = 0.743 Ω/1000 ft. V\textsubscript{drop} = 2 × 20 ×
250 × (0.743 × 0.98 + 0.044 × 0.199) / 1000 = 10,000 × 0.737 / 1000 =
\textbf{7.37 V} = 2.66\%.

Upgrade to \textbf{8 AWG copper} for compliance.

\begin{center}\rule{0.5\linewidth}{0.5pt}\end{center}

\section{Problem 14.7.7}\label{problem-14.7.7}

\textbf{Given:} Using the optional method (220.82), calculate the
service demand for a 2,200 ft² dwelling with: total connected load (all
circuits including lighting, receptacles, appliances) = 45 kVA, a 5-ton
(7 kW) central air conditioner, and a 12 kW electric furnace.

\textbf{Find:} The demand load and service size using the optional
method.

\textbf{Solution:} Per 220.82(B): \textbf{General loads} (everything
except heating and cooling): Total connected = 45 kVA. Subtract HVAC: 45
- 7 - 12 = \textbf{26 kVA} of general loads. First 10 kVA at 100\% =
10,000 VA. Remainder (16 kVA) at 40\% = 6,400 VA. General demand =
10,000 + 6,400 = \textbf{16,400 VA}.

\textbf{Heating/cooling per 220.60:} use the larger. Furnace 12,000 VA
\textgreater{} A/C 7,000 VA.

Per 220.82(C): heating and cooling loads at 100\% of the largest: HVAC
demand = \textbf{12,000 VA}.

Total demand: 16,400 + 12,000 = \textbf{28,400 VA}.

Service current: I = 28,400 / 240 = \textbf{118.3 A}.

Minimum service: \textbf{150 A} (next common size above 118.3 A, and
exceeds the 100 A minimum per 230.79).

Service entrance conductors: 1/0 AWG copper = 150 A at 75°C (matches 150
A service).

Note: The optional method yields a lower demand (28.4 kVA) than the
standard method would for this dwelling, because the 40\% factor on
general loads above 10 kVA is more aggressive than the standard method's
combination of demand factors.

\chapter{Chapter 14 --- Section 14.8: Emergency and Standby Power
Systems}\label{chapter-14-section-14.8-emergency-and-standby-power-systems}

Practice problems covering generator sizing, transfer switch
requirements, and emergency/standby system design.

\begin{center}\rule{0.5\linewidth}{0.5pt}\end{center}

\section{Problem 14.8.1}\label{problem-14.8.1}

\textbf{Given:} A commercial building requires an emergency generator
for the following Article 700 (emergency) loads: egress lighting (15
kW), exit signs (3 kW), fire alarm (5 kW), and a fire pump with a 30 HP,
460 V motor (FLC = 40 A, locked-rotor = 240 A). The generator is
480Y/277 V, three-phase. All loads except the fire pump are at unity
power factor. The fire pump motor operates at 0.85 PF.

\textbf{Find:} The total running load in kW, the generator kVA during
fire pump starting, and the recommended generator size.

\textbf{Solution:} \textbf{Running loads:} Lighting: 15 kW. Exit signs:
3 kW. Fire alarm: 5 kW. Fire pump running: P = √3 × 480 × 40 × 0.85 =
28.2 kW.

Total running load: 15 + 3 + 5 + 28.2 = \textbf{51.2 kW}.

\textbf{Fire pump starting analysis:} Fire pump starting kVA = √3 × 480
× 240 / 1000 = \textbf{199.7 kVA} (at \textasciitilde0.30 PF starting =
59.9 kW).

Running load without fire pump: 51.2 - 28.2 = 23.0 kW. Total kW during
fire pump start: 23.0 + 59.9 = 82.9 kW.

Total kVA during start: 23.0/1.0 + 199.7 = 23.0 + 199.7 = \textbf{222.7
kVA} (at mixed PF).

\textbf{Generator selection:} Continuous rating must exceed running
load: 51.2 kW / 0.8 PF = 64.0 kVA minimum continuous. Motor starting
capability must handle 222.7 kVA transient with voltage dip ≤ 10\%.

Select a \textbf{75 kW / 94 kVA} standby-rated generator with motor
starting capability of at least 225 kVA.

Per Article 700.12, the generator must have on-site fuel for at least
\textbf{2 hours} at full load. Fuel consumption ≈ 5.5 gal/hr for a 75 kW
diesel. Minimum tank: 5.5 × 2 = \textbf{11 gallons} (use 50-gallon tank
for practical margin).

\begin{center}\rule{0.5\linewidth}{0.5pt}\end{center}

\section{Problem 14.8.2}\label{problem-14.8.2}

\textbf{Given:} A data center uses an optional standby generator
(Article 702) to back up 400 kW of IT load at 0.90 PF and 80 kW of
cooling at 0.85 PF. The cooling includes two 30 HP chiller compressor
motors (FLC = 40 A each, locked-rotor = 240 A each). Load transfer is
sequenced: cooling starts first, then IT loads are transferred.

\textbf{Find:} The generator size accounting for the load sequencing and
motor starting.

\textbf{Solution:} \textbf{Step 1 -- First motor start (no other load):}
Motor 1 starting kVA = √3 × 480 × 240 / 1000 = 199.7 kVA. Motor 1
starting kW = 199.7 × 0.30 = 59.9 kW.

\textbf{Step 2 -- Second motor start (Motor 1 running):} Motor 1
running: P = √3 × 480 × 40 × 0.85 = 28.2 kW, S = 28.2/0.85 = 33.2 kVA.
Motor 2 starting: 199.7 kVA. Total Step 2: 33.2 + 199.7 = \textbf{232.9
kVA} (peak demand during motor 2 start). kW: 28.2 + 59.9 = 88.1 kW.

\textbf{Step 3 -- Both motors running + IT load transfer:} Both motors
running: 2 × 28.2 = 56.4 kW. Other cooling: 80 - 56.4 = 23.6 kW. IT
load: 400 kW. Total running: 56.4 + 23.6 + 400 = \textbf{480 kW}. Total
kVA: 56.4/0.85 + 23.6/0.85 + 400/0.90 = 66.4 + 27.8 + 444.4 =
\textbf{538.6 kVA}.

\textbf{Generator selection:} Continuous: 480 kW at mixed PF, requiring
538.6 kVA. Motor starting: 232.9 kVA transient (this is less than the
continuous kVA, so continuous governs).

At 0.8 PF rating: 538.6 kVA × 0.8 = 430.9 kW (insufficient if generator
is 0.8 PF rated). Need a generator rated at \textbf{500 kW / 625 kVA}
(at 0.8 PF) to cover the 480 kW / 538.6 kVA steady-state with 4\%
margin.

Alternatively, with load management that sheds some IT load during motor
starting: a \textbf{500 kW} generator is adequate.

\begin{center}\rule{0.5\linewidth}{0.5pt}\end{center}

\section{Problem 14.8.3}\label{problem-14.8.3}

\textbf{Given:} A hospital essential electrical system (Article 517.26)
must transfer emergency loads within 10 seconds per Article 700. The
automatic transfer switch (ATS) has the following time delays: - Time to
detect utility failure: 1.5 seconds - Engine start signal to generator
running at rated voltage: 5.0 seconds - Transfer delay after generator
ready: 0.5 seconds

\textbf{Find:} The total transfer time, whether it meets the 10-second
requirement, and the maximum engine start time that would still comply.

\textbf{Solution:} Total transfer time: t\textsubscript{total} =
t\textsubscript{detect} + t\textsubscript{start} +
t\textsubscript{transfer} = 1.5 + 5.0 + 0.5 = \textbf{7.0 seconds}.

Since 7.0 seconds \textless{} 10 seconds: the system \textbf{meets} the
Article 700 requirement with \textbf{3.0 seconds of margin}.

Maximum allowable engine start time: t\textsubscript{start(max)} = 10 -
t\textsubscript{detect} - t\textsubscript{transfer} = 10 - 1.5 - 0.5 =
\textbf{8.0 seconds}.

However, NFPA 110 for Type 10 emergency systems (healthcare) actually
requires the generator to accept load within 10 seconds, which is
tighter than just reaching rated voltage. The engine must reach rated
speed and voltage AND the ATS must complete the transfer within the
10-second window.

For legally required standby (Article 701): the restoration time is
\textbf{60 seconds}, providing much more margin.

Per 700.5(B), the ATS must be \textbf{automatic} and electrically
operated, mechanically held. Manual transfer is not permitted for
emergency systems.

\begin{center}\rule{0.5\linewidth}{0.5pt}\end{center}

\section{Problem 14.8.4}\label{problem-14.8.4}

\textbf{Given:} An emergency generator serves three load categories with
separate automatic transfer switches: - Emergency (Article 700): 50 kW
(must transfer in 10 seconds) - Legally required standby (Article 701):
80 kW (transfer in 60 seconds) - Optional standby (Article 702): 120 kW
(no time requirement)

The generator is rated 300 kW at 0.8 PF. Load sequencing adds loads in
priority order with 5-second intervals between transfers.

\textbf{Find:} The load on the generator at each step of the transfer
sequence and verify the generator is not overloaded at any step.

\textbf{Solution:} \textbf{t = 0 s:} Utility failure detected. Generator
start signal sent. Generator load = \textbf{0 kW}.

\textbf{t = 7 s:} Generator at rated voltage. Emergency ATS transfers.
Generator load = \textbf{50 kW} (16.7\% of rating).

\textbf{t = 12 s:} (5 seconds after emergency transfer) Legally required
standby ATS transfers. Generator load = 50 + 80 = \textbf{130 kW}
(43.3\% of rating).

\textbf{t = 17 s:} (5 seconds after LRSB transfer) Optional standby ATS
transfers. Generator load = 50 + 80 + 120 = \textbf{250 kW} (83.3\% of
rating).

At no point does the load exceed the 300 kW generator rating.

\textbf{kVA check} (assuming 0.85 average PF): 250 / 0.85 = 294 kVA.
Generator kVA = 300/0.8 = 375 kVA. 294 \textless{} 375:
\textbf{adequate}.

If any transfer included a large motor start, the kVA surge must be
checked against the generator's motor starting capability (typically 3×
continuous kVA = 1,125 kVA for a well-designed generator).

Per 700.32 (2023 NEC), the emergency overcurrent protection must be
\textbf{selectively coordinated} so that a fault on any emergency branch
trips only the nearest device, not the generator main breaker.

\begin{center}\rule{0.5\linewidth}{0.5pt}\end{center}

\section{Problem 14.8.5}\label{problem-14.8.5}

\textbf{Given:} A standby generator for a nursing home must run for 96
hours per NFPA 110 (Type 10, Class 96). The generator is rated 150 kW
diesel with a fuel consumption rate of 11.2 gallons per hour at full
load and 6.5 gallons per hour at 50\% load. The expected average load is
60\% of rating.

\textbf{Find:} The minimum fuel tank size at full load and at the
expected average load. Also estimate the fuel consumption at 60\% load
by linear interpolation.

\textbf{Solution:} \textbf{Fuel consumption at 60\% load} (linear
interpolation between 50\% and 100\%): At 50\%: 6.5 gal/hr. At 100\%:
11.2 gal/hr. Slope: (11.2 - 6.5) / (100 - 50) = 4.7 / 50 = 0.094 gal/hr
per \% load. At 60\%: 6.5 + 0.094 × (60 - 50) = 6.5 + 0.94 =
\textbf{7.44 gal/hr}.

\textbf{Minimum tank at full load (worst case):} V\textsubscript{tank} =
11.2 × 96 = \textbf{1,075 gallons}.

\textbf{Minimum tank at expected 60\% average load:}
V\textsubscript{tank} = 7.44 × 96 = \textbf{714 gallons}.

Per NFPA 110, the fuel supply must be sufficient for the class duration
at the expected load. Using the expected load calculation: minimum
\textbf{714 gallons}.

However, prudent engineering practice sizes for full load to account for
unexpected load additions during an extended outage. A
\textbf{1,100-gallon} fuel tank provides the required capacity at full
load with margin.

The tank must include fuel level monitoring per NFPA 110 Section 7.9,
and fuel quality maintenance (fuel polishing or stabilizer additives) is
critical for tanks that may sit unused for extended periods.

\chapter{Chapter 15 --- Section 15.1: The OSI
Model}\label{chapter-15-section-15.1-the-osi-model}

Practice problems covering the seven-layer OSI model, encapsulation,
physical layer signaling, data link framing, network layer routing, and
transport layer protocols.

\begin{center}\rule{0.5\linewidth}{0.5pt}\end{center}

\section{Problem 15.1.1}\label{problem-15.1.1}

\textbf{Given:} A networked sensor transmits a 200-byte application
payload using TCP over IPv4 on an Ethernet LAN. The TCP header is 20
bytes, the IPv4 header is 20 bytes, the Ethernet header is 14 bytes, and
the FCS is 4 bytes. The preamble/SFD is 8 bytes and the inter-frame gap
(IFG) is 12 bytes.

\textbf{Find:} (a) The total frame size on the wire (excluding
preamble/SFD and IFG), (b) the total on-wire overhead per frame, and (c)
the protocol efficiency at the frame level and at the wire level.

\textbf{Solution:}

\begin{enumerate}
\def\labelenumi{(\alph{enumi})}
\item
  Frame size = 200 + 20 + 20 + 14 + 4 = \textbf{258 bytes}
\item
  Frame-level overhead = 20 (TCP) + 20 (IP) + 14 (Eth header) + 4 (FCS)
  = 58 bytes Wire-level overhead = 58 + 8 (preamble/SFD) + 12 (IFG) =
  \textbf{78 bytes}
\item
  Frame efficiency: η\textsubscript{frame} = 200 / 258 = \textbf{0.7752
  = 77.5\%} Wire efficiency: η\textsubscript{wire} = 200 / (200 + 78) =
  200 / 278 = \textbf{0.7194 = 71.9\%}
\end{enumerate}

Small payloads like this sensor data suffer significantly more from
protocol overhead compared to full 1,500-byte payloads.

\begin{center}\rule{0.5\linewidth}{0.5pt}\end{center}

\section{Problem 15.1.2}\label{problem-15.1.2}

\textbf{Given:} A 25GBASE-T Ethernet link uses PAM-4 signaling (4
amplitude levels) across four twisted pairs, each operating at a symbol
rate of 3,200 Msymbols/s.

\textbf{Find:} (a) The bits per symbol for PAM-4, (b) the raw bit rate
per pair, and (c) the total aggregate raw bit rate across all four
pairs.

\textbf{Solution:}

\begin{enumerate}
\def\labelenumi{(\alph{enumi})}
\item
  Bits per symbol = log₂(4) = \textbf{2 bits/symbol}
\item
  Bit rate per pair = 3,200 × 10⁶ × 2 = \textbf{6,400 Mbps = 6.4 Gbps}
\item
  Total raw bit rate = 4 × 6.4 = \textbf{25.6 Gbps}
\end{enumerate}

The effective data rate of 25 Gbps is achieved after subtracting coding
overhead. The raw rate of 25.6 Gbps allows for approximately 2.4\% FEC
and framing overhead.

\begin{center}\rule{0.5\linewidth}{0.5pt}\end{center}

\section{Problem 15.1.3}\label{problem-15.1.3}

\textbf{Given:} A Layer 2 switch manages a MAC address table with 32,768
entries, each storing a 48-bit MAC address and a 10-bit port identifier.
An 802.1Q-tagged frame arrives with VLAN ID = 150, PCP = 5, and a
1,200-byte payload.

\textbf{Find:} (a) The memory required for the full MAC address table,
(b) the total tagged frame size (including Ethernet header, VLAN tag,
and FCS), and (c) the number of usable VLANs with a 12-bit VLAN ID
field.

\textbf{Solution:}

\begin{enumerate}
\def\labelenumi{(\alph{enumi})}
\item
  Bits per entry = 48 + 10 = 58 bits. Including VLAN association (12
  bits per entry for VLAN-aware table): 58 + 12 = 70 bits ≈ 9 bytes
  (rounded up to byte boundary). Total memory = 32,768 × 9 =
  \textbf{294,912 bytes = 288 KB}
\item
  Tagged frame size: Ethernet header: 14 bytes (dst MAC 6 + src MAC 6 +
  EtherType 2) 802.1Q tag: 4 bytes (TPID 2 + TCI 2) Payload: 1,200 bytes
  FCS: 4 bytes Total = 14 + 4 + 1,200 + 4 = \textbf{1,222 bytes}
\item
  12-bit VLAN ID: 2\textsuperscript{12} = 4,096 values. VID 0 and VID
  4095 are reserved. Usable VLANs = \textbf{4,094}
\end{enumerate}

\begin{center}\rule{0.5\linewidth}{0.5pt}\end{center}

\section{Problem 15.1.4}\label{problem-15.1.4}

\textbf{Given:} A router has an interface processing rate of 25 Mpps
(million packets per second). Traffic consists of 40\% packets at 64
bytes, 30\% at 512 bytes, and 30\% at 1,500 bytes. The initial TTL on
all packets is 128.

\textbf{Find:} (a) The weighted average packet size, (b) the throughput
in Gbps, and (c) the maximum number of router hops a packet can traverse
before being discarded.

\textbf{Solution:}

\begin{enumerate}
\def\labelenumi{(\alph{enumi})}
\item
  Average packet size = 0.40 × 64 + 0.30 × 512 + 0.30 × 1,500 = 25.6 +
  153.6 + 450.0 = \textbf{629.2 bytes}
\item
  Throughput = 25 × 10⁶ × 629.2 × 8 = 125,840,000,000 bps =
  \textbf{125.8 Gbps}
\item
  Maximum hops = \textbf{128} (TTL is decremented by 1 at each router;
  the packet is discarded when TTL reaches 0 at the 129th router).
\end{enumerate}

\begin{center}\rule{0.5\linewidth}{0.5pt}\end{center}

\section{Problem 15.1.5}\label{problem-15.1.5}

\textbf{Given:} A TCP connection operates over a satellite link with RTT
= 600 ms. The receiver advertises a window size of 131,072 bytes (128
KB) using the window scaling option (RFC 7323). The link capacity is 50
Mbps.

\textbf{Find:} (a) The bandwidth-delay product of the link, (b) the
maximum throughput limited by the receive window, (c) the link
utilization, and (d) the window size required to fully utilize the link.

\textbf{Solution:}

\begin{enumerate}
\def\labelenumi{(\alph{enumi})}
\item
  BDP = 50 × 10⁶ × 0.600 = 30,000,000 bits = \textbf{3,750,000 bytes =
  3.75 MB}
\item
  Throughput = Window / RTT = 131,072 / 0.600 = 218,453 bytes/s =
  218,453 × 8 = 1,747,627 bps = \textbf{1.75 Mbps}
\item
  Utilization = 1.75 / 50 = \textbf{3.5\%}
\end{enumerate}

The high-latency satellite link severely limits TCP throughput with a
small window.

\begin{enumerate}
\def\labelenumi{(\alph{enumi})}
\setcounter{enumi}{3}
\tightlist
\item
  Window needed = BDP = \textbf{3,750,000 bytes ≈ 3.58 MB}
\end{enumerate}

This requires window scaling with a scale factor of at least 6 (2⁶ = 64,
allowing a window up to 65,535 × 64 = 4,194,240 bytes). TCP performance
enhancing proxies (PEPs) or protocol optimizers are often deployed on
satellite links to address this issue.

\begin{center}\rule{0.5\linewidth}{0.5pt}\end{center}

\section{Problem 15.1.6}\label{problem-15.1.6}

\textbf{Given:} An Ethernet network carries frames with an average
payload of 800 bytes. The transport layer uses UDP with an 8-byte
header. IPv4 adds a 20-byte header. Ethernet adds 14 bytes header + 4
bytes FCS + 8 bytes preamble/SFD + 12 bytes IFG. The link operates at 10
Gbps.

\textbf{Find:} (a) The maximum number of frames per second the link can
carry, and (b) the application-layer goodput in Gbps.

\textbf{Solution:}

\begin{enumerate}
\def\labelenumi{(\alph{enumi})}
\tightlist
\item
  Total on-wire bytes per frame: Payload: 800 bytes (includes UDP header
  and application data) Actually, if 800 bytes is the application
  payload: UDP: 8 bytes, IP: 20 bytes, Ethernet header: 14 bytes, FCS: 4
  bytes, Preamble/SFD: 8 bytes, IFG: 12 bytes Total on-wire = 800 + 8 +
  20 + 14 + 4 + 8 + 12 = 866 bytes = 6,928 bits
\end{enumerate}

Frames per second = 10 × 10⁹ / 6,928 = \textbf{1,443,420 fps}

\begin{enumerate}
\def\labelenumi{(\alph{enumi})}
\setcounter{enumi}{1}
\tightlist
\item
  Goodput = 1,443,420 × 800 × 8 = 9,227,891,156 bps = \textbf{9.23 Gbps}
\end{enumerate}

\begin{center}\rule{0.5\linewidth}{0.5pt}\end{center}

\chapter{Chapter 15 --- Section 15.2: Physical Media:
Copper}\label{chapter-15-section-15.2-physical-media-copper}

Practice problems covering coaxial cable impedance and propagation,
twisted-pair attenuation and categories, signal bandwidth, crosstalk
(NEXT/FEXT/ACR), and cable testing/certification.

\begin{center}\rule{0.5\linewidth}{0.5pt}\end{center}

\section{Problem 15.2.1}\label{problem-15.2.1}

\textbf{Given:} An RG-58 coaxial cable has a center conductor diameter d
= 0.9 mm, a shield inner diameter D = 2.95 mm, and a solid polyethylene
dielectric with ε\textsubscript{r} = 2.25.

\textbf{Find:} (a) The characteristic impedance Z₀, (b) the velocity of
propagation as a percentage of the speed of light, and (c) the one-way
propagation delay per meter.

\textbf{Solution:}

\begin{enumerate}
\def\labelenumi{(\alph{enumi})}
\item
  Z₀ = (138 / √ε\textsubscript{r}) × log₁₀(D/d) √2.25 = 1.50 138 / 1.50
  = 92.0 log₁₀(2.95 / 0.9) = log₁₀(3.278) = 0.5156 Z₀ = 92.0 × 0.5156 =
  \textbf{47.4 Ω ≈ 50 Ω}
\item
  v\textsubscript{p} = c / √ε\textsubscript{r} = c / 1.50 = 0.6667c
  Velocity of propagation = \textbf{66.7\% of the speed of light}
\item
  Delay = 1 / v\textsubscript{p} = 1 / (0.6667 × 3 × 10⁸) = 1 / (2.0 ×
  10⁸) = 5.0 × 10⁻⁹ s/m = \textbf{5.0 ns/m}
\end{enumerate}

\begin{center}\rule{0.5\linewidth}{0.5pt}\end{center}

\section{Problem 15.2.2}\label{problem-15.2.2}

\textbf{Given:} A Cat 6 UTP cable has a maximum attenuation of 19.8 dB
at 100 MHz over a 100 m permanent link. A 10BASE-T signal at 10 MHz is
transmitted at +2 dBm over a 90 m cable run. The attenuation at 10 MHz
is 6.6 dB/100 m.

\textbf{Find:} (a) The total cable attenuation at 10 MHz over the 90 m
run, (b) the received signal power in dBm, and (c) the received power in
microwatts.

\textbf{Solution:}

\begin{enumerate}
\def\labelenumi{(\alph{enumi})}
\item
  Attenuation = 6.6 × (90/100) = \textbf{5.94 dB}
\item
  P\textsubscript{rx} = P\textsubscript{tx} − Attenuation = 2.0 − 5.94 =
  \textbf{−3.94 dBm}
\item
  P\textsubscript{rx} = 10\textsuperscript{(−3.94/10)} mW =
  10\textsuperscript{−0.394} = 0.4035 mW = \textbf{403.5 μW}
\end{enumerate}

\begin{center}\rule{0.5\linewidth}{0.5pt}\end{center}

\section{Problem 15.2.3}\label{problem-15.2.3}

\textbf{Given:} A 50 Ω coaxial cable has measured attenuation of 3.2
dB/100 m at 100 MHz and 10.5 dB/100 m at 1 GHz. A signal at 500 MHz must
be delivered over a 120 m cable run with no more than 12 dB total loss.

\textbf{Find:} (a) The estimated attenuation at 500 MHz using
square-root frequency scaling from the 100 MHz specification, (b) the
total attenuation for the 120 m run, and (c) whether the loss budget is
met.

\textbf{Solution:}

\begin{enumerate}
\def\labelenumi{(\alph{enumi})}
\item
  α(500) ≈ α(100) × √(500/100) = 3.2 × √5 = 3.2 × 2.236 = \textbf{7.16
  dB/100 m}
\item
  Total attenuation = 7.16 × (120/100) = \textbf{8.59 dB}
\item
  8.59 dB \textless{} 12.0 dB limit, so the loss budget \textbf{is met}
  with 3.41 dB of margin.
\end{enumerate}

Verification: the estimate of 7.16 dB/100 m at 500 MHz falls between 3.2
(at 100 MHz) and 10.5 (at 1 GHz), confirming reasonableness.

\begin{center}\rule{0.5\linewidth}{0.5pt}\end{center}

\section{Problem 15.2.4}\label{problem-15.2.4}

\textbf{Given:} A Cat 6A cable is tested at 500 MHz with the following
results: NEXT = 39.9 dB, FEXT = 29.5 dB, and insertion loss = 28.1 dB.
The minimum ACR requirement is 3 dB.

\textbf{Find:} (a) The near-end ACR, (b) the far-end ACR (ACRF),
accounting for the cable attenuation on the FEXT path, and (c) whether
the cable meets the ACR requirement at both ends.

\textbf{Solution:}

\begin{enumerate}
\def\labelenumi{(\alph{enumi})}
\item
  ACR (near-end) = NEXT − Insertion Loss = 39.9 − 28.1 = \textbf{11.8
  dB}
\item
  ACRF = FEXT − Insertion Loss = 29.5 − 28.1 = \textbf{1.4 dB}
\end{enumerate}

Note: FEXT is measured at the far end and already accounts for cable
attenuation on the interfering signal, so ACRF directly compares far-end
crosstalk to far-end signal level.

\begin{enumerate}
\def\labelenumi{(\alph{enumi})}
\setcounter{enumi}{2}
\tightlist
\item
  Near-end ACR = 11.8 dB \textgreater{} 3 dB → \textbf{PASS} Far-end
  ACRF = 1.4 dB \textless{} 3 dB → \textbf{FAIL}
\end{enumerate}

The cable passes near-end but fails far-end crosstalk requirements at
500 MHz. This may indicate manufacturing defects or damage affecting the
far-end pair isolation.

\begin{center}\rule{0.5\linewidth}{0.5pt}\end{center}

\section{Problem 15.2.5}\label{problem-15.2.5}

\textbf{Given:} A Cat 6A permanent link installation is certified with
the following test results at 250 MHz:

{\def\LTcaptype{none} % do not increment counter
\begin{longtable}[]{@{}lll@{}}
\toprule\noalign{}
Parameter & Measured & TIA-568 Limit \\
\midrule\noalign{}
\endhead
\bottomrule\noalign{}
\endlastfoot
Insertion loss & 15.4 dB & 19.3 dB \\
NEXT & 38.5 dB & 35.3 dB \\
PS-NEXT & 36.2 dB & 32.3 dB \\
Return loss & 15.8 dB & 14.0 dB \\
Propagation delay & 498 ns & 555 ns \\
\end{longtable}
}

\textbf{Find:} (a) The pass/fail status and headroom (margin) for each
parameter, (b) which parameter has the least headroom, and (c) the
velocity of propagation if the cable is 95 m long.

\textbf{Solution:}

\begin{enumerate}
\def\labelenumi{(\alph{enumi})}
\item
  For insertion loss, lower is better: 15.4 \textless{} 19.3 →
  \textbf{PASS}, headroom = 19.3 − 15.4 = \textbf{3.9 dB} For NEXT,
  higher is better: 38.5 \textgreater{} 35.3 → \textbf{PASS}, headroom =
  38.5 − 35.3 = \textbf{3.2 dB} For PS-NEXT, higher is better: 36.2
  \textgreater{} 32.3 → \textbf{PASS}, headroom = 36.2 − 32.3 =
  \textbf{3.9 dB} For return loss, higher is better: 15.8 \textgreater{}
  14.0 → \textbf{PASS}, headroom = 15.8 − 14.0 = \textbf{1.8 dB} For
  propagation delay, lower is better: 498 \textless{} 555 →
  \textbf{PASS}, headroom = 555 − 498 = \textbf{57 ns}
\item
  \textbf{Return loss} has the least headroom at 1.8 dB. This is the
  parameter most likely to fail first if connectors degrade or are
  improperly terminated.
\item
  Velocity of propagation = cable length / propagation delay = 95 / (498
  × 10⁻⁹) = 1.907 × 10⁸ m/s As a fraction of c: v\textsubscript{p}/c =
  1.907 × 10⁸ / 3.0 × 10⁸ = \textbf{63.6\% of the speed of light}
\end{enumerate}

\begin{center}\rule{0.5\linewidth}{0.5pt}\end{center}

\section{Problem 15.2.6}\label{problem-15.2.6}

\textbf{Given:} A network engineer must choose between Cat 5e (100 MHz
bandwidth, max attenuation 22 dB at 100 MHz per 100 m) and Cat 6A (500
MHz bandwidth, max attenuation 32.8 dB at 500 MHz per 100 m) for a 70 m
horizontal run that will carry 10GBASE-T traffic. 10GBASE-T requires Cat
6A or better and uses frequencies up to 500 MHz.

\textbf{Find:} (a) The Cat 5e attenuation at 100 MHz over the 70 m run,
(b) the Cat 6A attenuation at 500 MHz over the 70 m run, and (c) the
bandwidth-distance product for each cable category at the rated
frequency.

\textbf{Solution:}

\begin{enumerate}
\def\labelenumi{(\alph{enumi})}
\item
  Cat 5e at 100 MHz: 22.0 × (70/100) = \textbf{15.4 dB}
\item
  Cat 6A at 500 MHz: 32.8 × (70/100) = \textbf{22.96 dB}
\item
  Bandwidth-distance products: Cat 5e: 100 MHz × 100 m = \textbf{10,000
  MHz·m = 10 GHz·m} Cat 6A: 500 MHz × 100 m = \textbf{50,000 MHz·m = 50
  GHz·m}
\end{enumerate}

Cat 6A provides 5× the bandwidth-distance product of Cat 5e. Only Cat 6A
(or Cat 6 with alien crosstalk shielding) supports 10GBASE-T, which
requires the full 500 MHz bandwidth and the superior NEXT/PSANEXT
performance specified for Category 6A.

\begin{center}\rule{0.5\linewidth}{0.5pt}\end{center}

\section{Problem 15.2.7}\label{problem-15.2.7}

\textbf{Given:} A CATV distribution system uses RG-6 coaxial cable (75
Ω, foam PE dielectric ε\textsubscript{r} = 1.45) to deliver cable
television signals from a tap to a set-top box over a 30 m in-home run.
The signal at the tap is +10 dBmV at 600 MHz. The cable attenuation is
6.8 dB/100 m at 600 MHz. The set-top box requires a minimum of −5 dBmV
for reliable reception.

\textbf{Find:} (a) The cable attenuation over 30 m, (b) the signal level
at the set-top box, (c) the margin above the minimum receiver
sensitivity, and (d) whether a 2-way splitter (3.5 dB insertion loss)
can be added to serve a second TV.

\textbf{Solution:}

\begin{enumerate}
\def\labelenumi{(\alph{enumi})}
\item
  Cable attenuation = 6.8 × (30/100) = \textbf{2.04 dB}
\item
  Signal level = 10 − 2.04 = \textbf{+7.96 dBmV}
\item
  Margin = 7.96 − (−5) = \textbf{12.96 dB}
\item
  With a 2-way splitter: signal at each output = 7.96 − 3.5 = +4.46
  dBmV. Margin = 4.46 − (−5) = 9.46 dB → \textbf{Yes}, the splitter can
  be added with 9.46 dB of margin remaining. The system can support
  additional splitting if needed.
\end{enumerate}

\begin{center}\rule{0.5\linewidth}{0.5pt}\end{center}

\chapter{Chapter 15 --- Section 15.3: Physical Media: Fiber
Optics}\label{chapter-15-section-15.3-physical-media-fiber-optics}

Practice problems covering fiber structure, numerical aperture,
single-mode and multimode fiber, optical transmitters/receivers, link
budgets, and OTDR troubleshooting.

\begin{center}\rule{0.5\linewidth}{0.5pt}\end{center}

\section{Problem 15.3.1}\label{problem-15.3.1}

\textbf{Given:} A multimode step-index fiber has a core refractive index
n\textsubscript{core} = 1.492 and a cladding refractive index
n\textsubscript{clad} = 1.480.

\textbf{Find:} (a) The numerical aperture, (b) the critical angle for
total internal reflection, (c) the maximum acceptance half-angle in air,
and (d) the relative refractive index difference Δ =
(n\textsubscript{core} − n\textsubscript{clad}) / n\textsubscript{core}.

\textbf{Solution:}

\begin{enumerate}
\def\labelenumi{(\alph{enumi})}
\item
  NA = √(n\textsubscript{core}² − n\textsubscript{clad}²) = √(1.492² −
  1.480²) = √(2.2261 − 2.1904) = √0.03570 = \textbf{0.189}
\item
  sin θ\textsubscript{c} = n\textsubscript{clad} / n\textsubscript{core}
  = 1.480 / 1.492 = 0.99196 θ\textsubscript{c} = sin⁻¹(0.99196) =
  \textbf{82.73°}
\item
  sin θ\textsubscript{a} = NA = 0.189 θ\textsubscript{a} = sin⁻¹(0.189)
  = \textbf{10.90°}
\item
  Δ = (1.492 − 1.480) / 1.492 = 0.012 / 1.492 = \textbf{0.00804 =
  0.804\%}
\end{enumerate}

This is a relatively large NA (0.189) typical of standard multimode
fiber, making it easier to couple light from LEDs and large-core
sources.

\begin{center}\rule{0.5\linewidth}{0.5pt}\end{center}

\section{Problem 15.3.2}\label{problem-15.3.2}

\textbf{Given:} A single-mode fiber link at 1310 nm spans 50 km. The
fiber has attenuation of 0.35 dB/km and chromatic dispersion of 3.5
ps/(nm·km). The DFB laser source has a spectral width of 0.2 nm. The
system operates at 2.5 Gbps (OC-48/STM-16).

\textbf{Find:} (a) The total fiber attenuation, (b) the total chromatic
dispersion (pulse broadening), (c) whether the dispersion causes
intersymbol interference (compare to the bit period), and (d) the
maximum distance at 10 Gbps before dispersion exceeds the bit period.

\textbf{Solution:}

\begin{enumerate}
\def\labelenumi{(\alph{enumi})}
\item
  Total attenuation = 0.35 × 50 = \textbf{17.5 dB}
\item
  Δτ = D × L × Δλ = 3.5 × 50 × 0.2 = \textbf{35 ps}
\item
  Bit period at 2.5 Gbps = 1 / (2.5 × 10⁹) = 400 ps. 35 ps \textless{}
  400 ps → \textbf{No ISI}. The pulse broadening is only 8.75\% of the
  bit period.
\item
  At 10 Gbps, bit period = 100 ps. Maximum distance:
  L\textsubscript{max} = bit period / (D × Δλ) = 100 / (3.5 × 0.2) =
  \textbf{142.9 km}
\end{enumerate}

At 1310 nm with low dispersion (3.5 ps/nm·km), the link is
attenuation-limited rather than dispersion-limited at both 2.5 and 10
Gbps for distances under 50 km.

\begin{center}\rule{0.5\linewidth}{0.5pt}\end{center}

\section{Problem 15.3.3}\label{problem-15.3.3}

\textbf{Given:} An OM3 multimode fiber has a bandwidth-distance product
of 2,000 MHz·km at 850 nm. A data center link is 200 m long.

\textbf{Find:} (a) The modal bandwidth at 200 m, (b) the maximum NRZ bit
rate using BW ≈ 0.7 × bit rate, and (c) whether the fiber can support
25GBASE-SR (25 Gbps using NRZ) at this distance.

\textbf{Solution:}

\begin{enumerate}
\def\labelenumi{(\alph{enumi})}
\item
  BW = 2,000 / 0.200 = \textbf{10,000 MHz = 10.0 GHz}
\item
  Bit rate ≈ BW / 0.7 = 10,000 / 0.7 = \textbf{14,286 Mbps ≈ 14.3 Gbps}
\item
  25 Gbps \textgreater{} 14.3 Gbps → \textbf{No}, OM3 cannot support
  25GBASE-SR at 200 m with NRZ signaling. The IEEE 802.3 standard
  specifies 25GBASE-SR maximum reach on OM3 as 70 m. OM4 (4,700 MHz·km)
  extends the reach to approximately 100 m.
\end{enumerate}

\begin{center}\rule{0.5\linewidth}{0.5pt}\end{center}

\section{Problem 15.3.4}\label{problem-15.3.4}

\textbf{Given:} A QSFP28 transceiver for 100GBASE-LR4 uses four
wavelengths (1295.56 nm, 1300.05 nm, 1304.58 nm, 1309.14 nm), each
carrying 25 Gbps. The minimum per-lane transmit power is −4.3 dBm and
the receiver sensitivity is −10.6 dBm at BER = 10⁻¹² (with FEC).

\textbf{Find:} (a) The per-lane power budget, (b) the maximum fiber
distance at 0.35 dB/km (1310 nm window) with 2 connectors (0.5 dB each)
and 1 dB system margin, and (c) the total aggregate data rate.

\textbf{Solution:}

\begin{enumerate}
\def\labelenumi{(\alph{enumi})}
\item
  Power budget = P\textsubscript{tx,min} − P\textsubscript{rx,sens} =
  (−4.3) − (−10.6) = \textbf{6.3 dB}
\item
  Available for fiber = 6.3 − 2 × 0.5 − 1.0 = 6.3 − 2.0 = 4.3 dB Maximum
  distance = 4.3 / 0.35 = \textbf{12.3 km}
\end{enumerate}

The 100GBASE-LR4 standard specifies 10 km maximum reach, which is within
this calculated budget.

\begin{enumerate}
\def\labelenumi{(\alph{enumi})}
\setcounter{enumi}{2}
\tightlist
\item
  Total data rate = 4 × 25 = \textbf{100 Gbps}
\end{enumerate}

\begin{center}\rule{0.5\linewidth}{0.5pt}\end{center}

\section{Problem 15.3.5}\label{problem-15.3.5}

\textbf{Given:} A 60 km single-mode fiber link at 1550 nm connects two
buildings. Components: - Transmitter: +5 dBm output - Receiver
sensitivity: −22 dBm at BER = 10⁻¹² - Fiber attenuation: 0.22 dB/km at
1550 nm - Connectors: 4 pairs at 0.4 dB each - Fusion splices: 8 at 0.08
dB each - System margin: 3 dB

\textbf{Find:} (a) The total link loss, (b) the available power budget,
(c) the link margin after all losses, and (d) whether the link closes.

\textbf{Solution:}

\begin{enumerate}
\def\labelenumi{(\alph{enumi})}
\item
  Total link loss: Fiber: 0.22 × 60 = 13.20 dB Connectors: 4 × 0.4 =
  1.60 dB Splices: 8 × 0.08 = 0.64 dB Total = 13.20 + 1.60 + 0.64 =
  \textbf{15.44 dB}
\item
  Power budget = P\textsubscript{tx} − P\textsubscript{rx,sens} = (+5) −
  (−22) = \textbf{27.0 dB}
\item
  Margin = Budget − Total loss − System margin = 27.0 − 15.44 − 3.0 =
  \textbf{8.56 dB}
\item
  Since 8.56 dB \textgreater{} 0, the link \textbf{closes} with 8.56 dB
  of excess margin. This margin allows for future splice repairs (each
  adding approximately 0.1 dB) and fiber aging.
\end{enumerate}

\begin{center}\rule{0.5\linewidth}{0.5pt}\end{center}

\section{Problem 15.3.6}\label{problem-15.3.6}

\textbf{Given:} An OTDR tests a 15 km single-mode fiber at 1310 nm
(group refractive index n = 1.4677) using a 10 ns pulse width. The trace
shows: fiber launch at 0 m, fusion splice at 5.3 km (0.04 dB loss), a
connector at 10.1 km (0.35 dB loss with reflection), and fiber end at
15.0 km. Fiber attenuation slope is 0.34 dB/km.

\textbf{Find:} (a) The spatial resolution (one-way pulse length in the
fiber), (b) the total end-to-end link loss, and (c) the round-trip time
to the far end of the fiber.

\textbf{Solution:}

\begin{enumerate}
\def\labelenumi{(\alph{enumi})}
\tightlist
\item
  v\textsubscript{fiber} = c / n = 3.0 × 10⁸ / 1.4677 = 2.044 × 10⁸ m/s
  Pulse length = v\textsubscript{fiber} × pulse width = 2.044 × 10⁸ × 10
  × 10⁻⁹ = \textbf{2.044 m}
\end{enumerate}

Events closer than approximately 2 m cannot be individually resolved
with this pulse width.

\begin{enumerate}
\def\labelenumi{(\alph{enumi})}
\setcounter{enumi}{1}
\item
  Total link loss: Fiber: 0.34 × 15.0 = 5.10 dB Fusion splice: 0.04 dB
  Connector: 0.35 dB Total = 5.10 + 0.04 + 0.35 = \textbf{5.49 dB}
\item
  Round-trip time = 2 × distance / v\textsubscript{fiber} = 2 × 15,000 /
  (2.044 × 10⁸) = 30,000 / (2.044 × 10⁸) = \textbf{146.8 μs}
\end{enumerate}

\begin{center}\rule{0.5\linewidth}{0.5pt}\end{center}

\section{Problem 15.3.7}\label{problem-15.3.7}

\textbf{Given:} A fiber link upgrade replaces a PIN photodiode receiver
(sensitivity −18 dBm) with an APD receiver (sensitivity −28 dBm). The
transmitter output is +2 dBm. The total link loss is 24 dB including
fiber, connectors, splices, and system margin.

\textbf{Find:} (a) The link margin with the PIN receiver, (b) the link
margin with the APD receiver, (c) whether each receiver allows the link
to close, and (d) how many additional kilometers of fiber (at 0.22
dB/km) the APD receiver enables.

\textbf{Solution:}

\begin{enumerate}
\def\labelenumi{(\alph{enumi})}
\item
  PIN margin = (P\textsubscript{tx} − P\textsubscript{rx,PIN}) − link
  loss = (2 − (−18)) − 24 = 20 − 24 = \textbf{−4 dB}
\item
  APD margin = (P\textsubscript{tx} − P\textsubscript{rx,APD}) − link
  loss = (2 − (−28)) − 24 = 30 − 24 = \textbf{+6 dB}
\item
  PIN: margin is negative → \textbf{Link does not close} (insufficient
  budget). APD: margin is positive → \textbf{Link closes} with 6 dB
  margin.
\item
  Additional fiber distance = (sensitivity improvement) / attenuation =
  (28 − 18) / 0.22 = 10 / 0.22 = \textbf{45.5 km}
\end{enumerate}

The APD receiver provides 10 dB more sensitivity, enabling approximately
45 km more reach.

\begin{center}\rule{0.5\linewidth}{0.5pt}\end{center}

\chapter{Chapter 15 --- Section 15.4: Dense Wavelength Division
Multiplexing
(DWDM)}\label{chapter-15-section-15.4-dense-wavelength-division-multiplexing-dwdm}

Practice problems covering WDM channel spacing, EDFA amplifier design,
and DWDM system capacity planning.

\begin{center}\rule{0.5\linewidth}{0.5pt}\end{center}

\section{Problem 15.4.1}\label{problem-15.4.1}

\textbf{Given:} A CWDM system uses 18 channels with 20 nm spacing across
the 1270--1610 nm range. Each channel carries 10 Gbps. A DWDM system
uses the ITU-T 50 GHz grid in the C-band from 191.35 THz to 196.10 THz.

\textbf{Find:} (a) The total CWDM capacity, (b) the number of DWDM
channels on the 50 GHz grid, (c) the DWDM wavelength range, and (d) the
channel spacing in nm at the center of the C-band.

\textbf{Solution:}

\begin{enumerate}
\def\labelenumi{(\alph{enumi})}
\item
  CWDM capacity = 18 × 10 Gbps = \textbf{180 Gbps}
\item
  DWDM channels: Bandwidth = 196.10 − 191.35 = 4.75 THz = 4,750 GHz
  Channels = 4,750 / 50 + 1 = \textbf{96 channels}
\item
  λ\textsubscript{min} = c / f\textsubscript{max} = 3 × 10⁸ / (196.10 ×
  10¹²) = 1,529.8 nm λ\textsubscript{max} = c / f\textsubscript{min} = 3
  × 10⁸ / (191.35 × 10¹²) = 1,567.8 nm Wavelength range: \textbf{1,529.8
  nm to 1,567.8 nm} (38.0 nm span)
\item
  f\textsubscript{center} = (191.35 + 196.10) / 2 = 193.725 THz Δλ = c ×
  Δf / f² = (3 × 10⁸ × 50 × 10⁹) / (193.725 × 10¹²)² = 1.5 × 10¹⁹ /
  (3.753 × 10²⁸) = \textbf{0.400 nm ≈ 0.4 nm}
\end{enumerate}

\begin{center}\rule{0.5\linewidth}{0.5pt}\end{center}

\section{Problem 15.4.2}\label{problem-15.4.2}

\textbf{Given:} A long-haul DWDM link uses 8 inline EDFAs. Each EDFA has
a gain of 22 dB and a noise figure of 6.0 dB. The input signal power per
channel is −3 dBm. The span loss between amplifiers equals the EDFA
gain.

\textbf{Find:} (a) The OSNR at the receiver using OSNR ≈
P\textsubscript{in} − NF − 10 log₁₀(N) + 58 dBm, (b) whether the system
can support 200 Gbps DP-16QAM coherent detection (required OSNR ≈ 18
dB), and (c) the maximum number of amplifiers before OSNR drops below 18
dB.

\textbf{Solution:}

\begin{enumerate}
\def\labelenumi{(\alph{enumi})}
\item
  OSNR = P\textsubscript{in} − NF − 10 log₁₀(N) + 58 = −3 − 6.0 − 10
  log₁₀(8) + 58 = −3 − 6.0 − 9.03 + 58 = \textbf{39.97 dB ≈ 40.0 dB}
\item
  40.0 dB \textgreater\textgreater{} 18 dB → \textbf{Yes}, the system
  has 22 dB of OSNR margin above the 200G DP-16QAM requirement.
\item
  Required: P\textsubscript{in} − NF − 10 log₁₀(N) + 58 ≥ 18 −3 − 6.0 −
  10 log₁₀(N) + 58 ≥ 18 49 − 10 log₁₀(N) ≥ 18 10 log₁₀(N) ≤ 31 log₁₀(N)
  ≤ 3.1 N ≤ 10\textsuperscript{3.1} = 1,259
\end{enumerate}

Maximum amplifiers = \textbf{1,259}. At 22 dB gain per span (equivalent
to approximately 100 km span at 0.22 dB/km), this corresponds to
approximately 125,900 km -- clearly limited by other factors (fiber
nonlinearities) long before OSNR exhaustion. In practice, nonlinear
effects limit coherent systems to approximately 30-50 amplified spans.

\begin{center}\rule{0.5\linewidth}{0.5pt}\end{center}

\section{Problem 15.4.3}\label{problem-15.4.3}

\textbf{Given:} A metro DWDM ring carries 40 channels at 200 Gbps each
(DP-16QAM coherent) over a ring circumference of 400 km. The EDFA
amplifier spacing is 50 km. Fiber attenuation is 0.20 dB/km at 1550 nm.

\textbf{Find:} (a) The total system capacity, (b) the number of
amplifier sites on the ring, (c) the span loss each EDFA must
compensate, and (d) the total fiber-only attenuation around the full
ring.

\textbf{Solution:}

\begin{enumerate}
\def\labelenumi{(\alph{enumi})}
\item
  Total capacity = 40 × 200 Gbps = 8,000 Gbps = \textbf{8 Tbps}
\item
  Amplifier sites = ring circumference / span length = 400 / 50 =
  \textbf{8 sites}
\item
  Span loss = 0.20 × 50 = \textbf{10.0 dB} per span. Each EDFA must
  provide at least 10 dB gain.
\item
  Total ring attenuation = 0.20 × 400 = \textbf{80.0 dB}
\end{enumerate}

\begin{center}\rule{0.5\linewidth}{0.5pt}\end{center}

\section{Problem 15.4.4}\label{problem-15.4.4}

\textbf{Given:} A submarine DWDM cable system operates over 6,000 km
with 80 channels at 100 Gbps each (DP-QPSK). EDFA amplifier spacing is
60 km. Each EDFA provides 12 dB gain with a 4.5 dB noise figure. Launch
power per channel is 0 dBm.

\textbf{Find:} (a) The total system capacity, (b) the number of inline
amplifiers, (c) the span loss, and (d) the OSNR at the far end.

\textbf{Solution:}

\begin{enumerate}
\def\labelenumi{(\alph{enumi})}
\item
  Capacity = 80 × 100 = \textbf{8,000 Gbps = 8 Tbps}
\item
  Spans = 6,000 / 60 = 100. Inline amplifiers = 100 − 1 = \textbf{99}
  (plus booster and pre-amp).
\item
  Span loss = 0.20 × 60 = \textbf{12.0 dB} (matching the EDFA gain ---
  the system is in gain equilibrium).
\item
  OSNR = P\textsubscript{in} − NF − 10 log₁₀(N) + 58 N = 100 (total
  amplifiers including booster) = 0 − 4.5 − 10 log₁₀(100) + 58 = 0 − 4.5
  − 20.0 + 58 = \textbf{33.5 dB}
\end{enumerate}

For 100G DP-QPSK, required OSNR ≈ 12 dB. The system has 21.5 dB of OSNR
margin, which is consumed in practice by fiber nonlinearities, component
aging, and repair margin.

\begin{center}\rule{0.5\linewidth}{0.5pt}\end{center}

\section{Problem 15.4.5}\label{problem-15.4.5}

\textbf{Given:} A DWDM system is being upgraded from 100 GHz channel
spacing (45 channels at 100 Gbps each) to 50 GHz spacing with 400 Gbps
per channel using DP-16QAM coherent transceivers. The fiber
infrastructure and amplifiers remain the same.

\textbf{Find:} (a) The original system capacity, (b) the new number of
channels at 50 GHz spacing (same C-band from 191.7 THz to 196.1 THz),
(c) the new system capacity, and (d) the capacity increase factor.

\textbf{Solution:}

\begin{enumerate}
\def\labelenumi{(\alph{enumi})}
\item
  Original capacity = 45 × 100 = \textbf{4,500 Gbps = 4.5 Tbps}
\item
  Bandwidth = 196.1 − 191.7 = 4.4 THz = 4,400 GHz New channels = 4,400 /
  50 + 1 = \textbf{89 channels}
\item
  New capacity = 89 × 400 = \textbf{35,600 Gbps = 35.6 Tbps}
\item
  Capacity increase = 35,600 / 4,500 = \textbf{7.91×}
\end{enumerate}

The combination of halving the channel spacing (approximately doubling
channels from 45 to 89) and quadrupling the per-channel rate (100 to 400
Gbps) yields nearly 8× total capacity increase.

\begin{center}\rule{0.5\linewidth}{0.5pt}\end{center}

\chapter{Chapter 15 --- Section 15.5:
Ethernet}\label{chapter-15-section-15.5-ethernet}

Practice problems covering Ethernet frame structure, MAC addressing,
physical standards, Power over Ethernet, and Spanning Tree Protocol.

\begin{center}\rule{0.5\linewidth}{0.5pt}\end{center}

\section{Problem 15.5.1}\label{problem-15.5.1}

\textbf{Given:} A network monitoring tool captures 10 seconds of traffic
on a 10 Gbps Ethernet link. During that interval, 1,200,000 frames are
captured. The frame size distribution is: 40\% at 64 bytes, 25\% at 576
bytes, and 35\% at 1,518 bytes (all sizes exclude preamble/SFD and IFG).

\textbf{Find:} (a) The weighted average frame size, (b) the total data
volume captured, (c) the average link utilization, and (d) the average
frame rate in frames per second.

\textbf{Solution:}

\begin{enumerate}
\def\labelenumi{(\alph{enumi})}
\item
  Average frame size = 0.40 × 64 + 0.25 × 576 + 0.35 × 1,518 = 25.6 +
  144.0 + 531.3 = \textbf{700.9 bytes}
\item
  Total data = 1,200,000 × 700.9 = 841,080,000 bytes = \textbf{841.1 MB}
\item
  On-wire average size (add 8 preamble + 12 IFG) = 700.9 + 20 = 720.9
  bytes Total on-wire bits = 1,200,000 × 720.9 × 8 = 6,920,640,000 bits
  Utilization = 6,920,640,000 / (10 × 10⁹ × 10) = 6.92 × 10⁹ / 10¹¹ =
  \textbf{6.92\%}
\item
  Frame rate = 1,200,000 / 10 = \textbf{120,000 fps}
\end{enumerate}

\begin{center}\rule{0.5\linewidth}{0.5pt}\end{center}

\section{Problem 15.5.2}\label{problem-15.5.2}

\textbf{Given:} A data center interconnect uses 100GBASE-SR4 optics (4
lanes × 25 Gbps) over OM4 multimode fiber. Each lane uses an 850 nm
VCSEL. Per-lane minimum transmit power is −8.4 dBm and receiver
sensitivity is −10.3 dBm. Fiber attenuation at 850 nm is 3.5 dB/km.

\textbf{Find:} (a) The per-lane power budget, (b) the maximum fiber
distance with 2 MPO connectors (0.75 dB each), and (c) the total
aggregate throughput.

\textbf{Solution:}

\begin{enumerate}
\def\labelenumi{(\alph{enumi})}
\item
  Power budget = (−8.4) − (−10.3) = \textbf{1.9 dB}
\item
  Available for fiber = 1.9 − 2 × 0.75 = 1.9 − 1.5 = 0.4 dB Max distance
  = 0.4 / 3.5 = 0.114 km = \textbf{114 m}
\end{enumerate}

The 100GBASE-SR4 standard specifies 100 m maximum on OM4, consistent
with this tight power budget. Modal bandwidth, not just attenuation, is
the primary distance limiter.

\begin{enumerate}
\def\labelenumi{(\alph{enumi})}
\setcounter{enumi}{2}
\tightlist
\item
  Total throughput = 4 × 25 = \textbf{100 Gbps}
\end{enumerate}

\begin{center}\rule{0.5\linewidth}{0.5pt}\end{center}

\section{Problem 15.5.3}\label{problem-15.5.3}

\textbf{Given:} A PoE++ switch port (IEEE 802.3bt Type 4) delivers 90 W
to a high-power LED lighting fixture over a 50-meter Cat 6A cable run.
The DC resistance of Cat 6A is 7.0 Ω/100 m per conductor. All four pairs
carry power. The PSE output voltage is 52 V.

\textbf{Find:} (a) The round-trip resistance per pair, (b) the total
cable resistance with four pairs in parallel, (c) the current drawn, (d)
the cable power loss, and (e) the power and voltage delivered to the PD.

\textbf{Solution:}

\begin{enumerate}
\def\labelenumi{(\alph{enumi})}
\item
  R\textsubscript{pair} = 2 × 7.0 × (50/100) = 2 × 3.5 = \textbf{7.0 Ω
  per pair}
\item
  Four pairs in parallel: R\textsubscript{cable} = 7.0 / 4 =
  \textbf{1.75 Ω}
\item
  I = P / V = 90 / 52 = \textbf{1.731 A}
\item
  P\textsubscript{loss} = I² × R\textsubscript{cable} = 1.731² × 1.75 =
  2.996 × 1.75 = \textbf{5.24 W}
\item
  Power at PD: P\textsubscript{PD} = 90 − 5.24 = \textbf{84.76 W}
  Voltage at PD: V\textsubscript{PD} = 52 − (1.731 × 1.75) = 52 − 3.03 =
  \textbf{48.97 V} Efficiency: η = 84.76 / 90 = \textbf{94.2\%}
\end{enumerate}

The PD voltage of 48.97 V is well above the 42.5 V minimum required by
802.3bt.

\begin{center}\rule{0.5\linewidth}{0.5pt}\end{center}

\section{Problem 15.5.4}\label{problem-15.5.4}

\textbf{Given:} Five switches (S1-S5) are connected as follows: S1-S2
(10 Gbps, cost 2), S2-S3 (10 Gbps, cost 2), S3-S4 (1 Gbps, cost 4),
S4-S5 (10 Gbps, cost 2), S5-S1 (10 Gbps, cost 2). Bridge priorities: S1
= 4096, S2 = 8192, S3 = 16384, S4 = 32768, S5 = 32768.

\textbf{Find:} (a) The RSTP root bridge, (b) the root port on each
non-root switch, (c) the path cost from each switch to the root, and (d)
which port is placed in the alternate (blocking) state.

\textbf{Solution:}

\begin{enumerate}
\def\labelenumi{(\alph{enumi})}
\item
  S1 has the lowest bridge priority (4096), so \textbf{S1 is the root
  bridge}.
\item
  Root port selection (lowest cost path to root):
\end{enumerate}

\begin{itemize}
\tightlist
\item
  S2: port facing S1 (direct, cost 2) vs.~via S5-S4-S3 (cost
  2+2+4+2=10). Root port = \textbf{port facing S1}, cost 2.
\item
  S5: port facing S1 (direct, cost 2) vs.~via S2-S3-S4 (cost
  2+2+4+2=10). Root port = \textbf{port facing S1}, cost 2.
\item
  S3: via S2 (cost 2+2=4) vs.~via S4-S5 (cost 4+2+2=8). Root port =
  \textbf{port facing S2}, cost 4.
\item
  S4: via S3-S2 (cost 4+2+2=8) vs.~via S5 (cost 2+2=4). Root port =
  \textbf{port facing S5}, cost 4.
\end{itemize}

\begin{enumerate}
\def\labelenumi{(\alph{enumi})}
\setcounter{enumi}{2}
\item
  Path costs to root: S1 = 0, S2 = \textbf{2}, S5 = \textbf{2}, S3 =
  \textbf{4}, S4 = \textbf{4}.
\item
  The link S3-S4 is the redundant link. On that segment:
\end{enumerate}

\begin{itemize}
\tightlist
\item
  S3's root path cost = 4, S4's root path cost = 4 (tie).
\item
  Tiebreaker: lower bridge ID. S3 priority (16384) \textless{} S4
  priority (32768).
\item
  S3's port facing S4 is designated (forwarding), S4's port facing S3 is
  \textbf{alternate (blocking)}.
\end{itemize}

Active topology: S1-S2-S3 and S1-S5-S4, with S3-S4 blocked.

\begin{center}\rule{0.5\linewidth}{0.5pt}\end{center}

\section{Problem 15.5.5}\label{problem-15.5.5}

\textbf{Given:} A PoE system uses 802.3af (Type 1, 15.4 W at PSE) to
power a VoIP phone over a 90 m Cat 5e cable. Cat 5e DC resistance is
9.38 Ω/100 m per conductor. Power is delivered on two pairs (Mode A).
The PSE voltage is 48 V. The phone requires 8 W minimum to operate.

\textbf{Find:} (a) The total cable resistance for two-pair power
delivery, (b) the current drawn, (c) the cable power loss, (d) the power
at the PD, and (e) whether the phone can operate.

\textbf{Solution:}

\begin{enumerate}
\def\labelenumi{(\alph{enumi})}
\item
  R\textsubscript{pair} = 2 × 9.38 × (90/100) = 2 × 8.442 = 16.884 Ω Two
  pairs in parallel: R\textsubscript{cable} = 16.884 / 2 = \textbf{8.442
  Ω}
\item
  I = P / V = 15.4 / 48 = \textbf{0.321 A}
\item
  P\textsubscript{loss} = I² × R = 0.321² × 8.442 = 0.1030 × 8.442 =
  \textbf{0.870 W}
\item
  P\textsubscript{PD} = 15.4 − 0.870 = \textbf{14.53 W}
\item
  14.53 W \textgreater{} 8.0 W → \textbf{Yes}, the phone can operate
  with 6.53 W of margin. The 802.3af standard guarantees 12.95 W at the
  PD; this installation delivers 14.53 W.
\end{enumerate}

\begin{center}\rule{0.5\linewidth}{0.5pt}\end{center}

\section{Problem 15.5.6}\label{problem-15.5.6}

\textbf{Given:} An Ethernet frame with a 46-byte payload (minimum) is
transmitted on a 1 Gbps link. The total on-wire frame includes: 8 bytes
preamble/SFD, 14 bytes header, 46 bytes payload, 4 bytes FCS, and 12
bytes IFG.

\textbf{Find:} (a) The total on-wire bytes, (b) the transmission time
for one minimum frame, (c) the maximum frame rate at wire speed, and (d)
the payload throughput at minimum frame size as a percentage of line
rate.

\textbf{Solution:}

\begin{enumerate}
\def\labelenumi{(\alph{enumi})}
\item
  On-wire bytes = 8 + 14 + 46 + 4 + 12 = \textbf{84 bytes = 672 bits}
\item
  Transmission time = 672 / (1 × 10⁹) = \textbf{672 ns}
\item
  Max frame rate = 1 × 10⁹ / 672 = \textbf{1,488,095 fps}
\item
  Payload throughput = 1,488,095 × 46 × 8 = 547,738,095 bps = 547.7 Mbps
  Percentage of line rate = 547.7 / 1,000 = \textbf{54.8\%}
\end{enumerate}

At minimum frame size, Ethernet achieves less than 55\% efficiency. This
is why small-packet forwarding is the most demanding test for
switch/router performance.

\begin{center}\rule{0.5\linewidth}{0.5pt}\end{center}

\chapter{Chapter 15 --- Section 15.6: Internet Protocol
(IP)}\label{chapter-15-section-15.6-internet-protocol-ip}

Practice problems covering IPv4 subnetting, IPv6 addressing, OSPF/BGP
routing, NAT/PAT, and CIDR route aggregation.

\begin{center}\rule{0.5\linewidth}{0.5pt}\end{center}

\section{Problem 15.6.1}\label{problem-15.6.1}

\textbf{Given:} An ISP assigns a customer the network 192.168.100.0/24.
The customer has four departments: Engineering (50 hosts), Sales (25
hosts), Accounting (10 hosts), and Management (5 hosts). VLSM (Variable
Length Subnet Masks) will be used to allocate the smallest subnets that
meet each department's needs.

\textbf{Find:} (a) The prefix length and subnet for each department, (b)
the network address, broadcast address, and usable host range for each,
and (c) the total number of IP addresses used versus wasted.

\textbf{Solution:}

\begin{enumerate}
\def\labelenumi{(\alph{enumi})}
\item
  Allocate largest subnet first: Engineering (50 hosts): 2ⁿ − 2 ≥ 50 → n
  = 6 (62 hosts) → \textbf{/26} Sales (25 hosts): 2ⁿ − 2 ≥ 25 → n = 5
  (30 hosts) → \textbf{/27} Accounting (10 hosts): 2ⁿ − 2 ≥ 10 → n = 4
  (14 hosts) → \textbf{/28} Management (5 hosts): 2ⁿ − 2 ≥ 5 → n = 3 (6
  hosts) → \textbf{/29}
\item
  Subnet allocation (in order of decreasing size): \textbf{Engineering:
  192.168.100.0/26} Network: 192.168.100.0, Broadcast: 192.168.100.63,
  Hosts: 192.168.100.1 -- 192.168.100.62 (62 usable)
\end{enumerate}

\textbf{Sales: 192.168.100.64/27} Network: 192.168.100.64, Broadcast:
192.168.100.95, Hosts: 192.168.100.65 -- 192.168.100.94 (30 usable)

\textbf{Accounting: 192.168.100.96/28} Network: 192.168.100.96,
Broadcast: 192.168.100.111, Hosts: 192.168.100.97 -- 192.168.100.110 (14
usable)

\textbf{Management: 192.168.100.112/29} Network: 192.168.100.112,
Broadcast: 192.168.100.119, Hosts: 192.168.100.113 -- 192.168.100.118 (6
usable)

\begin{enumerate}
\def\labelenumi{(\alph{enumi})}
\setcounter{enumi}{2}
\tightlist
\item
  Total addresses used = 64 + 32 + 16 + 8 = 120. Usable hosts = 62 + 30
  + 14 + 6 = 112. Required hosts = 50 + 25 + 10 + 5 = 90. Wasted usable
  addresses = 112 − 90 = \textbf{22 addresses} Remaining unallocated
  from /24 = 256 − 120 = \textbf{136 addresses} (192.168.100.120 --
  192.168.100.255)
\end{enumerate}

\begin{center}\rule{0.5\linewidth}{0.5pt}\end{center}

\section{Problem 15.6.2}\label{problem-15.6.2}

\textbf{Given:} A server has the MAC address A4:83:E7:2F:00:1B and is on
the IPv6 subnet 2001:0db8:cafe:0001::/64.

\textbf{Find:} (a) The full 128-bit IPv6 address using Modified EUI-64,
(b) the abbreviated notation, and (c) the total number of addresses in
the /64 subnet.

\textbf{Solution:}

\begin{enumerate}
\def\labelenumi{(\alph{enumi})}
\tightlist
\item
  Step 1 --- Split MAC and insert FFFE: A4:83:E7 →
  A4:83:E7:FF:FE:2F:00:1B
\end{enumerate}

Step 2 --- Flip the 7th bit (U/L bit) of the first byte: A4 = 1010 0100
→ flip bit 6 → 1010 0110 = A6

Modified: A6:83:E7:FF:FE:2F:00:1B

Step 3 --- Form interface ID (16-bit groups): A683:E7FF:FE2F:001B

Step 4 --- Combine: Full: 2001:0db8:cafe:0001:A683:E7FF:FE2F:001B

\begin{enumerate}
\def\labelenumi{(\alph{enumi})}
\setcounter{enumi}{1}
\item
  Abbreviated: \textbf{2001:db8:cafe:1:a683:e7ff:fe2f:1b}
\item
  Addresses in /64: 2\textsuperscript{64} = \textbf{1.844 ×
  10\textsuperscript{19} addresses} (18.4 quintillion)
\end{enumerate}

\begin{center}\rule{0.5\linewidth}{0.5pt}\end{center}

\section{Problem 15.6.3}\label{problem-15.6.3}

\textbf{Given:} A network has two paths from router R1 to destination
prefix 10.20.0.0/16: - Path A: OSPF route, cost 150 (AD = 110) - Path B:
iBGP route, AS path length 2, local preference 200 (AD = 200) - Path C:
Static route via next-hop 10.0.0.1 (AD = 1)

All three paths are available simultaneously.

\textbf{Find:} (a) Which route is installed in the routing table and
why, (b) what happens if the static route is removed, and (c) what
happens if both the static and OSPF routes are removed.

\textbf{Solution:}

\begin{enumerate}
\def\labelenumi{(\alph{enumi})}
\item
  The route with the lowest administrative distance is preferred: Static
  (AD = 1), OSPF (AD = 110), iBGP (AD = 200). \textbf{The static route
  (Path C) is installed}, with AD = 1 being the lowest.
\item
  Without the static route, the remaining options are OSPF (110) and
  iBGP (200). \textbf{The OSPF route (Path A) is installed}, with AD =
  110 \textless{} 200.
\item
  With only iBGP remaining, \textbf{the iBGP route (Path B) is
  installed} as the sole option.
\end{enumerate}

This demonstrates the administrative distance hierarchy: static (1)
\textgreater{} OSPF (110) \textgreater{} iBGP (200). The BGP local
preference of 200 is only used for path selection within BGP itself, not
for comparison against other routing protocols.

\begin{center}\rule{0.5\linewidth}{0.5pt}\end{center}

\section{Problem 15.6.4}\label{problem-15.6.4}

\textbf{Given:} A medium-sized office has 150 workstations and 30
servers, all using private addresses in the 10.1.0.0/16 space. The ISP
provides a pool of 8 public IPv4 addresses (203.0.113.8/29) for dynamic
NAT of servers and a single additional public IP (203.0.113.16) for PAT
of all workstations. During peak hours, each workstation averages 60
simultaneous sessions and each server averages 200 sessions.

\textbf{Find:} (a) The total sessions for workstations on the PAT
address, (b) the PAT port utilization, (c) the total sessions for
servers on the dynamic NAT pool, and (d) whether the 8-address NAT pool
is sufficient for the 30 servers.

\textbf{Solution:}

\begin{enumerate}
\def\labelenumi{(\alph{enumi})}
\item
  Workstation sessions = 150 × 60 = \textbf{9,000 simultaneous NAT
  entries}
\item
  Ephemeral ports per protocol: 65,535 − 1,024 + 1 = 64,512. For TCP +
  UDP: 2 × 64,512 = 129,024 mappings. Utilization = 9,000 / 129,024 =
  \textbf{6.98\%} --- well within capacity.
\item
  Server sessions: Each server with dynamic NAT gets a dedicated public
  IP while active. Total server sessions = 30 × 200 = \textbf{6,000
  sessions} across the pool.
\item
  Usable addresses in /29: 2³ − 2 = 6 (network and broadcast excluded).
  With 30 servers needing public addresses but only 6 available, dynamic
  NAT can serve a maximum of \textbf{6 servers simultaneously}. The
  remaining 24 must wait or use PAT instead. To support all 30 servers
  with dedicated public IPs, a /27 (30 usable addresses) or larger block
  is needed.
\end{enumerate}

\begin{center}\rule{0.5\linewidth}{0.5pt}\end{center}

\section{Problem 15.6.5}\label{problem-15.6.5}

\textbf{Given:} An ISP owns eight contiguous /24 networks:
198.51.100.0/24 through 198.51.107.0/24.

\textbf{Find:} (a) The CIDR summary route covering all eight networks,
(b) verification that the block starts on the correct boundary, (c) the
network address, broadcast address, and total host count for the
aggregate, and (d) how many BGP routing table entries this aggregation
saves.

\textbf{Solution:}

\begin{enumerate}
\def\labelenumi{(\alph{enumi})}
\item
  Eight /24 networks: 2\textsuperscript{k} = 8, so k = 3. Summary prefix
  = /24 − 3 = \textbf{/21}. Summary route: \textbf{198.51.100.0/21}
\item
  Verify alignment: third octet 100 in binary = 01100100. A /21 mask
  means the first 21 bits are network bits. The first 16 bits cover
  octets 1-2 (198.51). Octet 3 contributes 5 bits to the network. 100 =
  01100\textbf{100} --- the last 3 bits (100 = 4) must be the start of
  the variable portion. 100 AND (256 − 8) = 100 AND 248 = 96 ≠ 100.
\end{enumerate}

Wait --- let me recheck: 198.51.100.0 through 198.51.107.255. The third
octet ranges from 100 to 107. 100 in binary: 01100\textbf{100}, 107:
01101\textbf{011}. These span from 100 to 107 = range of 8. For a /21
block: the third octet mask is 11111\textbf{000} = 248. 100 AND 248 =
96, not 100. The /21 boundary would start at 198.51.96.0, not
198.51.100.0.

\textbf{The eight /24s starting at 100 do NOT align to a /21 boundary.}
The correct aggregate requires checking alignment: 104 AND 248 = 104. So
198.51.104.0/21 covers 104-111. For 100-107: this needs to be split into
198.51.100.0/22 (100-103) and 198.51.104.0/22 (104-107) = \textbf{two
/22 routes}.

\begin{enumerate}
\def\labelenumi{(\alph{enumi})}
\setcounter{enumi}{2}
\item
  198.51.100.0/22: Network 198.51.100.0, Broadcast 198.51.103.255, Hosts
  = 1,022. 198.51.104.0/22: Network 198.51.104.0, Broadcast
  198.51.107.255, Hosts = 1,022. Total hosts = \textbf{2,044}
\item
  Aggregation reduces 8 routes to \textbf{2 routes}, saving 6 BGP table
  entries. A single /21 aggregate is not possible due to misalignment.
\end{enumerate}

\begin{center}\rule{0.5\linewidth}{0.5pt}\end{center}

\section{Problem 15.6.6}\label{problem-15.6.6}

\textbf{Given:} A host has IP address 172.20.45.130 with a subnet mask
of /20.

\textbf{Find:} (a) The subnet mask in dotted-decimal notation, (b) the
network address, (c) the broadcast address, (d) the usable host range,
and (e) the total number of usable hosts.

\textbf{Solution:}

\begin{enumerate}
\def\labelenumi{(\alph{enumi})}
\item
  /20 mask: 20 ones followed by 12 zeros.
  11111111.11111111.11110000.00000000 = \textbf{255.255.240.0}
\item
  Network address: 172.20.45.130 AND 255.255.240.0 Third octet: 45 AND
  240 = 00101101 AND 11110000 = 00100000 = 32 Network =
  \textbf{172.20.32.0}
\item
  Broadcast: set all 12 host bits to 1. Third octet: 32 OR 15 = 47.
  Fourth octet: 255. Broadcast = \textbf{172.20.47.255}
\item
  Usable range: \textbf{172.20.32.1 -- 172.20.47.254}
\item
  Usable hosts: 2¹² − 2 = 4,096 − 2 = \textbf{4,094 hosts}
\end{enumerate}

\begin{center}\rule{0.5\linewidth}{0.5pt}\end{center}

\chapter{Chapter 15 --- Section 15.7: Transport
Protocols}\label{chapter-15-section-15.7-transport-protocols}

Practice problems covering TCP throughput and window sizing, UDP
overhead, socket connections, WSGI server capacity, and QUIC protocol
comparisons.

\begin{center}\rule{0.5\linewidth}{0.5pt}\end{center}

\section{Problem 15.7.1}\label{problem-15.7.1}

\textbf{Given:} A data center replication job uses TCP to transfer a 2
GB file between sites connected by a 10 Gbps link with RTT = 5 ms. The
TCP receive window is 4 MB (with window scaling enabled). Assume no
packet loss or congestion.

\textbf{Find:} (a) The bandwidth-delay product of the link, (b) the
maximum TCP throughput, (c) whether the window is large enough to
saturate the link, and (d) the minimum transfer time.

\textbf{Solution:}

\begin{enumerate}
\def\labelenumi{(\alph{enumi})}
\item
  BDP = 10 × 10⁹ × 0.005 = 50,000,000 bits = \textbf{6,250,000 bytes =
  6.25 MB}
\item
  Throughput = Window / RTT = 4 × 10⁶ / 0.005 = 800,000,000 bytes/s =
  800 × 10⁶ × 8 = 6,400,000,000 bps = \textbf{6.4 Gbps}
\item
  BDP = 6.25 MB, Window = 4 MB. Since 4 MB \textless{} 6.25 MB, the
  window is \textbf{not large enough} to saturate the 10 Gbps link. A
  window of at least 6.25 MB is needed.
\item
  At 6.4 Gbps: Transfer time = 2 × 10⁹ × 8 / (6.4 × 10⁹) = 16 × 10⁹ /
  6.4 × 10⁹ = \textbf{2.5 seconds}
\end{enumerate}

With a 6.25 MB window (saturating the link): time = 16 × 10⁹ / (10 ×
10⁹) = \textbf{1.6 seconds}.

\begin{center}\rule{0.5\linewidth}{0.5pt}\end{center}

\section{Problem 15.7.2}\label{problem-15.7.2}

\textbf{Given:} A video surveillance system streams 16 IP cameras, each
using H.265 encoding at 4 Mbps. Each camera sends 1,258-byte RTP/UDP/IP
packets (1,200 bytes video payload + 12 bytes RTP + 8 bytes UDP + 20
bytes IPv4 + 14 bytes Ethernet header + 4 bytes FCS). Packets are sent
every 2.4 ms (matching the video frame slicing).

\textbf{Find:} (a) The total video bitrate for all 16 cameras, (b) the
packet rate per camera, (c) the total network bandwidth including all
headers, and (d) the header overhead percentage.

\textbf{Solution:}

\begin{enumerate}
\def\labelenumi{(\alph{enumi})}
\item
  Total video bitrate = 16 × 4 Mbps = \textbf{64 Mbps}
\item
  Packets per second per camera: Payload per packet = 1,200 bytes =
  9,600 bits Packets/s = 4 × 10⁶ / 9,600 = \textbf{416.7 pps}
\end{enumerate}

Alternatively: 1 / 0.0024 = 416.7 pps.

\begin{enumerate}
\def\labelenumi{(\alph{enumi})}
\setcounter{enumi}{2}
\tightlist
\item
  Total frame size = 1,258 bytes (without preamble/IFG) per the given
  breakdown: Actually: 1,200 + 12 + 8 + 20 + 14 + 4 = 1,258 bytes. With
  preamble (8) and IFG (12): 1,278 bytes on wire.
\end{enumerate}

Per camera bandwidth = 416.7 × 1,278 × 8 = 4,259,200 bps = 4.26 Mbps
Total 16 cameras = 16 × 4.26 = \textbf{68.1 Mbps}

\begin{enumerate}
\def\labelenumi{(\alph{enumi})}
\setcounter{enumi}{3}
\tightlist
\item
  Overhead per packet = 1,258 − 1,200 = 58 bytes. Percentage = (68.1 −
  64) / 68.1 = 4.1 / 68.1 = \textbf{6.0\%}
\end{enumerate}

The overhead is modest for these large video packets compared to the
22\%+ overhead seen with small VoIP packets.

\begin{center}\rule{0.5\linewidth}{0.5pt}\end{center}

\section{Problem 15.7.3}\label{problem-15.7.3}

\textbf{Given:} A web server at 10.0.1.100 listens on port 443 (HTTPS).
During a load test, 5,000 clients connect simultaneously. Each client
uses a unique ephemeral port from the range 49152--65535. Client IPs are
distributed across 50 unique source addresses.

\textbf{Find:} (a) The 4-tuple for a specific connection from client
192.168.5.20 port 51234, (b) the total number of sockets held by the
server process, (c) the average connections per client IP, and (d)
whether the ephemeral port range can support the load from a single
client IP.

\textbf{Solution:}

\begin{enumerate}
\def\labelenumi{(\alph{enumi})}
\item
  4-tuple: \textbf{(192.168.5.20, 51234, 10.0.1.100, 443)}
\item
  Server sockets = 1 listening socket + 5,000 connected sockets =
  \textbf{5,001 sockets}
\item
  Average connections per IP = 5,000 / 50 = \textbf{100 connections per
  client IP}
\item
  Ephemeral range = 65,535 − 49,152 + 1 = 16,384 ports. 100 connections
  requires 100 unique ports from the 16,384 available. 100 / 16,384 =
  0.6\% → \textbf{Yes}, easily supported. Even all 5,000 connections
  from a single IP (5,000/16,384 = 30.5\%) would be feasible.
\end{enumerate}

\begin{center}\rule{0.5\linewidth}{0.5pt}\end{center}

\section{Problem 15.7.4}\label{problem-15.7.4}

\textbf{Given:} A Gunicorn WSGI deployment uses async gevent workers.
Each worker can handle 100 concurrent connections (greenlets). The
server has 4 workers. Average request processing time is 200 ms, but
60\% of that time is I/O wait (database queries, API calls) during which
gevent yields to other greenlets.

\textbf{Find:} (a) The total concurrent connections the server can
handle, (b) the effective per-worker throughput accounting for I/O
concurrency, (c) the total server RPS, and (d) the number of workers
needed for 2,000 RPS.

\textbf{Solution:}

\begin{enumerate}
\def\labelenumi{(\alph{enumi})}
\item
  Total concurrent connections = 4 × 100 = \textbf{400 connections}
\item
  Each request takes 200 ms wall clock. During 120 ms (60\%) of I/O
  wait, the greenlet yields and other requests are processed. The
  effective CPU time per request is 80 ms. Per-worker throughput = 100
  greenlets × (1 / 0.200 s) = \textbf{500 RPS per worker} (bounded by
  concurrency limit)
\end{enumerate}

Alternatively, CPU-limited: with 80 ms CPU per request, one CPU can
process 1/0.080 = 12.5 sequential requests/s, but with 100 concurrent
greenlets overlapping I/O, throughput = min(100/0.200, 1/0.080) ×
concurrency factor.

The effective rate per worker: each greenlet completes in 200 ms, so 100
greenlets produce 100/0.200 = \textbf{500 RPS} (assuming the CPU can
keep up with 100 × 80 ms / 1000 ms = 8 concurrent CPU-active tasks).

\begin{enumerate}
\def\labelenumi{(\alph{enumi})}
\setcounter{enumi}{2}
\item
  Total RPS = 4 × 500 = \textbf{2,000 RPS}
\item
  Workers needed = 2,000 / 500 = \textbf{4 workers} --- the current
  deployment already meets the target.
\end{enumerate}

\begin{center}\rule{0.5\linewidth}{0.5pt}\end{center}

\section{Problem 15.7.5}\label{problem-15.7.5}

\textbf{Given:} A mobile application loads a web page requiring 80
resources from the same server. The network has RTT = 60 ms. Compare
connection setup and data transfer start for: (a) HTTP/1.1 with 6
parallel TCP connections, (b) HTTP/2 over TCP + TLS 1.3, and (c) HTTP/3
over QUIC with 0-RTT resumption.

\textbf{Find:} The time to first byte (TTFB) for the first resource, and
the connection overhead for each protocol.

\textbf{Solution:}

\begin{enumerate}
\def\labelenumi{(\alph{enumi})}
\item
  \textbf{HTTP/1.1 with 6 parallel TCP connections:} Each connection: 1
  RTT (TCP handshake) + 1 RTT (TLS 1.3) = 2 × 60 = 120 ms setup. TTFB =
  120 + 60 (request/response RTT) = \textbf{180 ms} All 6 connections
  are established in parallel, so 6 connections finish setup at 120 ms.
  80 resources / 6 connections = \textasciitilde14 sequential requests
  per connection, each taking 60 ms. Total page load ≈ 120 + 14 × 60 =
  \textbf{960 ms}
\item
  \textbf{HTTP/2 over TCP + TLS 1.3:} Single connection: 1 RTT (TCP) + 1
  RTT (TLS) = 2 × 60 = \textbf{120 ms setup} TTFB = 120 + 60 =
  \textbf{180 ms} All 80 resources multiplexed over one connection.
  Total page load ≈ 120 + 60 = \textbf{180 ms} (all requests sent
  immediately after setup, responses streamed back concurrently)
\item
  \textbf{HTTP/3 over QUIC (0-RTT resumption):} 0-RTT: data is sent with
  the first packet. TTFB = \textbf{60 ms} (one RTT for request/response,
  no handshake delay) Total page load ≈ \textbf{60 ms} (all requests
  sent immediately in the first flight)
\end{enumerate}

QUIC 0-RTT saves 120 ms compared to HTTP/2, a \textbf{67\% reduction} in
connection overhead.

\begin{center}\rule{0.5\linewidth}{0.5pt}\end{center}

\section{Problem 15.7.6}\label{problem-15.7.6}

\textbf{Given:} A TCP connection with an initial congestion window of 10
segments (MSS = 1,460 bytes) undergoes slow start on a 1 Gbps link with
10 ms RTT. No packets are lost.

\textbf{Find:} (a) The data sent in the first RTT, (b) the congestion
window after 4 RTTs, (c) the throughput achieved after 4 RTTs, and (d)
the number of RTTs needed for the congestion window to reach the BDP.

\textbf{Solution:}

\begin{enumerate}
\def\labelenumi{(\alph{enumi})}
\item
  First RTT: cwnd = 10 segments = 10 × 1,460 = \textbf{14,600 bytes =
  116,800 bits} Data sent = 14,600 bytes at rate = 14,600 × 8 / 0.010 =
  \textbf{11.68 Mbps}
\item
  Slow start doubles cwnd each RTT: RTT 1: cwnd = 10 → 20 (10 ACKs, each
  increases by 1) RTT 2: cwnd = 20 → 40 RTT 3: cwnd = 40 → 80 RTT 4:
  cwnd = 80 → \textbf{160 segments}
\item
  After 4 RTTs: cwnd = 160 × 1,460 = 233,600 bytes Throughput = 233,600
  × 8 / 0.010 = \textbf{186.9 Mbps}
\item
  BDP = 1 × 10⁹ × 0.010 / 8 = 1,250,000 bytes. In segments: 1,250,000 /
  1,460 = 856 segments. After n RTTs: cwnd = 10 × 2ⁿ. Need 10 × 2ⁿ ≥ 856
  → 2ⁿ ≥ 85.6 → n ≥ log₂(85.6) = 6.42. After \textbf{7 RTTs} (70 ms),
  cwnd = 10 × 128 = 1,280 segments = 1,868,800 bytes \textgreater{} BDP,
  and the link is saturated.
\end{enumerate}

\begin{center}\rule{0.5\linewidth}{0.5pt}\end{center}

\chapter{Chapter 15 --- Section 15.8: Wireless Networking
(Wi-Fi)}\label{chapter-15-section-15.8-wireless-networking-wi-fi}

Practice problems covering Wi-Fi data rate calculations, RF propagation
and link budgets, and wireless security.

\begin{center}\rule{0.5\linewidth}{0.5pt}\end{center}

\section{Problem 15.8.1}\label{problem-15.8.1}

\textbf{Given:} A Wi-Fi 7 (802.11be) access point operates on a 320 MHz
channel in the 6 GHz band using 4096-QAM with a 5/6 coding rate. The
OFDM parameters are: 3,984 data subcarriers, 12.8 μs symbol duration,
and 0.8 μs guard interval. The device uses 2 spatial streams.

\textbf{Find:} (a) The bits per subcarrier per symbol, (b) the symbol
rate, (c) the PHY data rate for a single spatial stream, and (d) the PHY
data rate with 2 spatial streams.

\textbf{Solution:}

\begin{enumerate}
\def\labelenumi{(\alph{enumi})}
\item
  Bits per subcarrier = log₂(4096) × (5/6) = 12 × 0.8333 = \textbf{10.0
  coded bits/subcarrier/symbol}
\item
  Symbol rate = 1 / (12.8 + 0.8) μs = 1 / 13.6 μs = \textbf{73,529
  symbols/s}
\item
  Single stream rate = 3,984 × 10.0 × 73,529 = 2,929,395,360 bps =
  \textbf{2,929.4 Mbps ≈ 2.93 Gbps}
\item
  With 2 spatial streams: 2 × 2,929.4 = \textbf{5,858.8 Mbps ≈ 5.86
  Gbps}
\end{enumerate}

The spectral efficiency = 5,858.8 / 320 = 18.3 bps/Hz, demonstrating the
efficiency gains of 4096-QAM and dense subcarrier spacing in Wi-Fi 7.

\begin{center}\rule{0.5\linewidth}{0.5pt}\end{center}

\section{Problem 15.8.2}\label{problem-15.8.2}

\textbf{Given:} A Wi-Fi 6 access point at 2.4 GHz (2,437 MHz) transmits
at 23 dBm (200 mW) with an antenna gain of 3 dBi. A laptop client has an
antenna gain of 0 dBi and receiver sensitivity of −70 dBm for MCS9
(256-QAM, 5/6). The path between the AP and client includes 2 interior
drywall walls (3.5 dB loss each).

\textbf{Find:} (a) The free-space path loss at 20 m, (b) the total path
loss including wall attenuation, (c) the received signal strength, and
(d) the link margin.

\textbf{Solution:}

\begin{enumerate}
\def\labelenumi{(\alph{enumi})}
\item
  FSPL = 20 log₁₀(0.020) + 20 log₁₀(2437) + 32.44 = 20 × (−1.699) + 20 ×
  3.387 + 32.44 = −33.98 + 67.73 + 32.44 = \textbf{66.19 dB}
\item
  Total path loss = FSPL + wall losses = 66.19 + 2 × 3.5 = 66.19 + 7.0 =
  \textbf{73.19 dB}
\item
  P\textsubscript{rx} = P\textsubscript{tx} + G\textsubscript{tx} +
  G\textsubscript{rx} − path loss = 23 + 3 + 0 − 73.19 = \textbf{−47.19
  dBm}
\item
  Link margin = P\textsubscript{rx} − sensitivity = −47.19 − (−70) =
  \textbf{22.81 dB}
\end{enumerate}

The link has excellent margin for MCS9 operation. Each additional
drywall wall would reduce the margin by 3.5 dB.

\begin{center}\rule{0.5\linewidth}{0.5pt}\end{center}

\section{Problem 15.8.3}\label{problem-15.8.3}

\textbf{Given:} A Wi-Fi 6E access point operates at 6 GHz (6,000 MHz)
with 30 dBm EIRP (regulatory limit in many regions). A client device at
15 m distance has 0 dBi antenna gain and requires −65 dBm for MCS11
(1024-QAM, 5/6). No walls are between the AP and client.

\textbf{Find:} (a) The free-space path loss at 15 m, (b) the received
signal strength, (c) the link margin, and (d) the maximum free-space
range for MCS11 operation.

\textbf{Solution:}

\begin{enumerate}
\def\labelenumi{(\alph{enumi})}
\item
  FSPL = 20 log₁₀(0.015) + 20 log₁₀(6000) + 32.44 = 20 × (−1.824) + 20 ×
  3.778 + 32.44 = −36.48 + 75.56 + 32.44 = \textbf{71.52 dB}
\item
  EIRP already includes transmit power + antenna gain:
  P\textsubscript{tx} + G\textsubscript{tx} = 30 dBm P\textsubscript{rx}
  = 30 + 0 − 71.52 = \textbf{−41.52 dBm}
\item
  Margin = −41.52 − (−65) = \textbf{23.48 dB}
\item
  Maximum allowable path loss = 30 + 0 − (−65) = 95 dB 95 = 20 log₁₀(d)
  + 20 log₁₀(6000) + 32.44 20 log₁₀(d) = 95 − 75.56 − 32.44 = −13.0
  log₁₀(d) = −0.65 d = 10\textsuperscript{−0.65} = 0.2239 km =
  \textbf{223.9 m} (free space)
\end{enumerate}

In practice, indoor range at 6 GHz is significantly shorter (10-20 m for
MCS11) due to wall losses and multipath.

\begin{center}\rule{0.5\linewidth}{0.5pt}\end{center}

\section{Problem 15.8.4}\label{problem-15.8.4}

\textbf{Given:} A corporate WPA2-Enterprise network uses EAP-TLS
authentication. An auditor wants to evaluate the risk of a brute-force
attack on the TLS session. The TLS 1.2 connection uses AES-128-GCM. A
password-based WPA2-Personal network on the guest SSID uses a
10-character passphrase consisting of mixed-case letters and digits (62
possible characters per position).

\textbf{Find:} (a) The AES-128 keyspace size, (b) the time to
brute-force AES-128 at 10\textsuperscript{12} keys/second, (c) the
WPA2-Personal passphrase keyspace, and (d) the time for an offline GPU
attack at 500,000 passphrases/second.

\textbf{Solution:}

\begin{enumerate}
\def\labelenumi{(\alph{enumi})}
\item
  AES-128 keyspace = 2\textsuperscript{128} = \textbf{3.403 ×
  10\textsuperscript{38} keys}
\item
  Time = 3.403 × 10\textsuperscript{38} / 10\textsuperscript{12} = 3.403
  × 10\textsuperscript{26} seconds = 3.403 × 10\textsuperscript{26} /
  (3.156 × 10⁷) = \textbf{1.078 × 10\textsuperscript{19} years}
\end{enumerate}

This is approximately 780 million times the age of the universe. AES-128
is computationally infeasible to brute-force.

\begin{enumerate}
\def\labelenumi{(\alph{enumi})}
\setcounter{enumi}{2}
\item
  Passphrase keyspace = 62\textsuperscript{10} = \textbf{8.393 ×
  10\textsuperscript{17} possible passphrases}
\item
  Offline attack time = 8.393 × 10\textsuperscript{17} / 500,000 = 1.679
  × 10\textsuperscript{12} seconds = 1.679 × 10\textsuperscript{12} /
  (3.156 × 10⁷) = \textbf{53,200 years}
\end{enumerate}

A 10-character mixed-case alphanumeric passphrase provides strong
protection against offline attacks. However, dictionary words or common
patterns would be found much faster using rule-based attacks.

\begin{center}\rule{0.5\linewidth}{0.5pt}\end{center}

\section{Problem 15.8.5}\label{problem-15.8.5}

\textbf{Given:} A warehouse Wi-Fi deployment uses 5 GHz (5,500 MHz)
access points with 20 dBm transmit power and 6 dBi gain directional
antennas aimed down aisles. The metal shelving creates significant
multipath but no direct wall penetration is needed. The path loss model
for the warehouse is: PL(dB) = 38 + 28 log₁₀(d), where d is distance in
meters. Client sensitivity is −75 dBm for 802.11ac MCS5 (64-QAM, 2/3).

\textbf{Find:} (a) The EIRP, (b) the path loss at 50 m, (c) the received
signal at 50 m (client antenna gain = 0 dBi), and (d) the maximum range
for MCS5 operation.

\textbf{Solution:}

\begin{enumerate}
\def\labelenumi{(\alph{enumi})}
\item
  EIRP = P\textsubscript{tx} + G\textsubscript{tx} = 20 + 6 = \textbf{26
  dBm}
\item
  PL(50 m) = 38 + 28 × log₁₀(50) = 38 + 28 × 1.699 = 38 + 47.57 =
  \textbf{85.57 dB}
\item
  P\textsubscript{rx} = EIRP + G\textsubscript{rx} − PL = 26 + 0 − 85.57
  = \textbf{−59.57 dBm}
\end{enumerate}

Margin = −59.57 − (−75) = 15.43 dB.

\begin{enumerate}
\def\labelenumi{(\alph{enumi})}
\setcounter{enumi}{3}
\tightlist
\item
  Maximum range: 26 + 0 − (38 + 28 log₁₀(d)) = −75 28 log₁₀(d) = 26 + 75
  − 38 = 63 log₁₀(d) = 63/28 = 2.25 d = 10\textsuperscript{2.25} =
  \textbf{177.8 m}
\end{enumerate}

With this indoor propagation model, the AP can serve MCS5 clients up to
approximately 178 m down a warehouse aisle. In practice, a 10 dB fade
margin is typically added, reducing the reliable range to approximately
100 m.

\begin{center}\rule{0.5\linewidth}{0.5pt}\end{center}

\chapter{Chapter 15 --- Section 15.9: Network
Infrastructure}\label{chapter-15-section-15.9-network-infrastructure}

Practice problems covering switches, routers, structured cabling, VLANs,
network security, and software-defined networking.

\begin{center}\rule{0.5\linewidth}{0.5pt}\end{center}

\section{Problem 15.9.1}\label{problem-15.9.1}

\textbf{Given:} A Layer 3 campus switch has 24 × 10 Gbps SFP+ ports and
2 × 40 Gbps QSFP+ uplinks. The switching fabric is rated at 560 Gbps and
the forwarding rate is 416.67 Mpps.

\textbf{Find:} (a) The total full-duplex bandwidth requirement, (b)
whether the switch fabric is non-blocking, (c) the wire-speed forwarding
rate at 64-byte minimum frames for all ports, and (d) whether the
forwarding engine can sustain wire speed.

\textbf{Solution:}

\begin{enumerate}
\def\labelenumi{(\alph{enumi})}
\item
  Full-duplex bandwidth: 10G ports: 24 × 10 × 2 = 480 Gbps 40G uplinks:
  2 × 40 × 2 = 160 Gbps Total = 480 + 160 = \textbf{640 Gbps}
\item
  Switching fabric = 560 Gbps. 560 \textless{} 640 → \textbf{Blocking}
  -- the fabric cannot handle all ports at full duplex simultaneously.
  In practice, oversubscription of access ports toward uplinks is common
  and acceptable.
\item
  Wire-speed frame rate: Per 10G port: 10 × 10⁹ / (84 × 8) = 14,880,952
  fps Per 40G port: 40 × 10⁹ / (84 × 8) = 59,523,810 fps Total (all
  ports, one direction): 24 × 14,880,952 + 2 × 59,523,810 = 357,142,857
  + 119,047,619 = \textbf{476.2 Mpps}
\item
  Forwarding engine = 416.67 Mpps \textless{} 476.2 Mpps →
  \textbf{Cannot sustain wire speed} at 64-byte frames across all ports
  simultaneously. At larger frame sizes (e.g., 128+ bytes), the packet
  rate drops and wire speed becomes achievable.
\end{enumerate}

\begin{center}\rule{0.5\linewidth}{0.5pt}\end{center}

\section{Problem 15.9.2}\label{problem-15.9.2}

\textbf{Given:} A core router has eight 100 Gbps interfaces and a 2.4
Tbps switching capacity. The router holds 1.2 million IPv4 routes in its
TCAM, with each entry consuming 80 bytes (prefix, mask, next-hop,
action, counters). The router processes 64-byte packets at line rate.

\textbf{Find:} (a) The total bidirectional bandwidth, (b) the wire-speed
packet rate for all interfaces, (c) the TCAM memory used by the routing
table, and (d) the minimum TCAM size needed.

\textbf{Solution:}

\begin{enumerate}
\def\labelenumi{(\alph{enumi})}
\tightlist
\item
  Total bidirectional bandwidth = 8 × 100 × 2 = \textbf{1,600 Gbps = 1.6
  Tbps}
\end{enumerate}

The switching capacity of 2.4 Tbps \textgreater{} 1.6 Tbps →
non-blocking.

\begin{enumerate}
\def\labelenumi{(\alph{enumi})}
\setcounter{enumi}{1}
\item
  Per 100G interface at 64-byte frames: 100 × 10⁹ / (84 × 8) =
  148,809,524 fps Total all 8 interfaces = 8 × 148,809,524 =
  \textbf{1,190.5 Mpps}
\item
  TCAM memory = 1,200,000 × 80 = 96,000,000 bytes = \textbf{96 MB}
\item
  Current Internet IPv4 table ≈ 950,000 prefixes. With 20\% growth
  margin: 1,140,000 entries × 80 = 91.2 MB. Minimum TCAM = \textbf{96
  MB} (to hold 1.2M routes at 80 bytes each).
\end{enumerate}

\begin{center}\rule{0.5\linewidth}{0.5pt}\end{center}

\section{Problem 15.9.3}\label{problem-15.9.3}

\textbf{Given:} A new five-story office building requires structured
cabling. Each floor has 120 data drops. The MDF is in the basement. Each
floor has one IDF. The average horizontal cable run is 55 m. Backbone
fiber between MDF and each IDF averages 40 m per floor. Patch cords are
3 m at the IDF and 5 m at the work area.

\textbf{Find:} (a) The total channel length per drop, (b) TIA-568
compliance, (c) the number of 24-port patch panels per IDF, (d) the
number of 48-port switches per IDF, and (e) the total backbone fiber
needed (assuming 12-strand OM4 per IDF).

\textbf{Solution:}

\begin{enumerate}
\def\labelenumi{(\alph{enumi})}
\item
  Channel = IDF patch cord + horizontal cable + work-area patch cord = 3
  + 55 + 5 = \textbf{63 m}
\item
  63 m \textless{} 100 m channel limit → \textbf{Compliant} 55 m
  permanent link \textless{} 90 m limit → \textbf{Compliant}
\item
  Panels per IDF = 120 / 24 = 5.0 → \textbf{5 patch panels} per IDF
  (exactly 120 ports) Total panels = 5 × 5 floors = \textbf{25 patch
  panels}
\item
  Switches per IDF = 120 / 48 = 2.5 → \textbf{3 switches} per IDF (144
  ports, 24 spare) Total switches = 3 × 5 = \textbf{15 access switches}
\item
  Backbone fiber = 5 IDFs × 1 cable × 12 strands × average 40 m = 5 × 40
  = 200 m of 12-strand fiber. Total = \textbf{5 runs of 12-strand OM4}
  (40 m average per run, actual lengths vary by floor).
\end{enumerate}

\begin{center}\rule{0.5\linewidth}{0.5pt}\end{center}

\section{Problem 15.9.4}\label{problem-15.9.4}

\textbf{Given:} A hospital network requires five VLANs: VLAN 10 (Medical
devices, 300 hosts), VLAN 20 (Staff workstations, 500 hosts), VLAN 30
(Patient Wi-Fi, 200 hosts), VLAN 40 (VoIP phones, 400 hosts), and VLAN
99 (Network management, 15 hosts). The IP address space is 10.10.0.0/16.

\textbf{Find:} (a) The smallest CIDR prefix for each VLAN using VLSM,
(b) the subnet assignments from the 10.10.0.0/16 space, (c) the 802.1Q
tag overhead on a trunk carrying 200,000 frames/second, and (d) the
total usable host capacity across all VLANs.

\textbf{Solution:}

\begin{enumerate}
\def\labelenumi{(\alph{enumi})}
\item
  Prefix sizing (allocate largest first): VLAN 20 (500 hosts): 2ⁿ − 2 ≥
  500 → n = 9 (510 hosts) → \textbf{/23} VLAN 40 (400 hosts): 2ⁿ − 2 ≥
  400 → n = 9 (510 hosts) → \textbf{/23} VLAN 10 (300 hosts): 2ⁿ − 2 ≥
  300 → n = 9 (510 hosts) → \textbf{/23} VLAN 30 (200 hosts): 2ⁿ − 2 ≥
  200 → n = 8 (254 hosts) → \textbf{/24} VLAN 99 (15 hosts): 2ⁿ − 2 ≥ 15
  → n = 5 (30 hosts) → \textbf{/27}
\item
  Subnet assignments: VLAN 20: \textbf{10.10.0.0/23} (10.10.0.1 --
  10.10.1.254, 510 hosts) VLAN 40: \textbf{10.10.2.0/23} (10.10.2.1 --
  10.10.3.254, 510 hosts) VLAN 10: \textbf{10.10.4.0/23} (10.10.4.1 --
  10.10.5.254, 510 hosts) VLAN 30: \textbf{10.10.6.0/24} (10.10.6.1 --
  10.10.6.254, 254 hosts) VLAN 99: \textbf{10.10.7.0/27} (10.10.7.1 --
  10.10.7.30, 30 hosts)
\item
  802.1Q overhead = 200,000 × 4 bytes × 8 bits = 6,400,000 bps =
  \textbf{6.4 Mbps}
\item
  Total usable hosts = 510 + 510 + 510 + 254 + 30 = \textbf{1,814 hosts}
  Total required = 300 + 500 + 200 + 400 + 15 = 1,415 hosts. Surplus =
  399 hosts for growth.
\end{enumerate}

\begin{center}\rule{0.5\linewidth}{0.5pt}\end{center}

\section{Problem 15.9.5}\label{problem-15.9.5}

\textbf{Given:} A firewall processes traffic at 10 Gbps throughput with
a session table capacity of 2,000,000 entries. Each stateful session
entry uses 128 bytes. During a DDoS attack, the attacker sends 500,000
SYN packets per second from spoofed source IPs. The firewall's SYN flood
protection limits half-open connections to 100,000.

\textbf{Find:} (a) The memory used by the session table at full
capacity, (b) the time to fill the SYN flood limit without protection,
(c) the percentage of session table consumed by the SYN flood limit, and
(d) the bandwidth consumed by the SYN flood (each SYN packet is 60 bytes
including IP and Ethernet headers).

\textbf{Solution:}

\begin{enumerate}
\def\labelenumi{(\alph{enumi})}
\item
  Session table memory = 2,000,000 × 128 = 256,000,000 bytes =
  \textbf{256 MB}
\item
  Time to fill SYN limit = 100,000 / 500,000 = \textbf{0.2 seconds}
\end{enumerate}

Without SYN flood protection, the attacker would consume 500,000
entries/second, filling the entire 2M session table in 2,000,000 /
500,000 = \textbf{4 seconds}.

\begin{enumerate}
\def\labelenumi{(\alph{enumi})}
\setcounter{enumi}{2}
\tightlist
\item
  SYN flood limit as \% of table = 100,000 / 2,000,000 = \textbf{5.0\%}
\end{enumerate}

This ensures that 95\% of the session table remains available for
legitimate connections.

\begin{enumerate}
\def\labelenumi{(\alph{enumi})}
\setcounter{enumi}{3}
\tightlist
\item
  SYN flood bandwidth = 500,000 × 60 × 8 = 240,000,000 bps = \textbf{240
  Mbps}
\end{enumerate}

This is only 2.4\% of the firewall's 10 Gbps capacity, illustrating that
DDoS attacks often aim to exhaust state tables rather than bandwidth.

\begin{center}\rule{0.5\linewidth}{0.5pt}\end{center}

\section{Problem 15.9.6}\label{problem-15.9.6}

\textbf{Given:} An SDN controller manages a leaf-spine data center
fabric with 64 leaf (ToR) switches and 8 spine switches. Each leaf has
48 × 25 Gbps server ports and 8 × 100 Gbps uplinks to the spines. Each
spine has 64 × 100 Gbps ports. Each leaf switch has 32 MB of TCAM with
256-byte flow entries. The controller installs flows reactively (on
first packet miss).

\textbf{Find:} (a) The total server-facing bandwidth, (b) the
leaf-to-spine oversubscription ratio, (c) the maximum flow entries per
leaf switch, and (d) the total flow table capacity across the fabric.

\textbf{Solution:}

\begin{enumerate}
\def\labelenumi{(\alph{enumi})}
\item
  Server bandwidth per leaf: 48 × 25 = 1,200 Gbps = 1.2 Tbps Total
  server bandwidth: 64 × 1,200 = \textbf{76,800 Gbps = 76.8 Tbps}
\item
  Server-side bandwidth per leaf: 1,200 Gbps Uplink bandwidth per leaf:
  8 × 100 = 800 Gbps Oversubscription = 1,200 / 800 = \textbf{1.5:1}
\end{enumerate}

This means 67\% of server traffic can be forwarded to the spine
simultaneously -- a moderate and common oversubscription ratio.

\begin{enumerate}
\def\labelenumi{(\alph{enumi})}
\setcounter{enumi}{2}
\item
  Flow entries per leaf = 32 × 10⁶ / 256 = \textbf{125,000 entries}
\item
  Total fabric capacity = 64 × 125,000 = \textbf{8,000,000 flow entries}
\end{enumerate}

At 256 bytes each, total TCAM across the fabric = 8,000,000 × 256 =
2,048,000,000 bytes = \textbf{2.048 GB}.

\begin{center}\rule{0.5\linewidth}{0.5pt}\end{center}

\section{Problem 15.9.7}\label{problem-15.9.7}

\textbf{Given:} An enterprise extends a VLAN across two buildings using
an 802.1Q trunk over a 1 Gbps single-mode fiber link. The trunk carries
VLANs 10, 20, 30, and 99. Average traffic per VLAN: VLAN 10 = 200 Mbps,
VLAN 20 = 300 Mbps, VLAN 30 = 150 Mbps, VLAN 99 = 10 Mbps. Each frame
averages 750 bytes.

\textbf{Find:} (a) The total trunk utilization, (b) the frame rate, (c)
the 802.1Q tagging overhead in Mbps, and (d) whether the trunk has
capacity for all VLANs.

\textbf{Solution:}

\begin{enumerate}
\def\labelenumi{(\alph{enumi})}
\item
  Total traffic = 200 + 300 + 150 + 10 = \textbf{660 Mbps} Utilization =
  660 / 1,000 = \textbf{66.0\%}
\item
  Total data rate in bytes/s = 660 × 10⁶ / 8 = 82,500,000 bytes/s Frame
  rate = 82,500,000 / 750 = \textbf{110,000 fps}
\item
  Tag overhead = 110,000 × 4 bytes × 8 bits = 3,520,000 bps =
  \textbf{3.52 Mbps}
\item
  Total with tagging = 660 + 3.52 = 663.52 Mbps. Since 663.52
  \textless{} 1,000 Mbps → \textbf{Yes}, the trunk has sufficient
  capacity with 33.6\% headroom.
\end{enumerate}

\begin{center}\rule{0.5\linewidth}{0.5pt}\end{center}

\chapter{Chapter 15 --- Section 15.10: Network
Performance}\label{chapter-15-section-15.10-network-performance}

Practice problems covering latency and propagation delay, throughput and
bandwidth utilization, bit error rate, network timing and
synchronization, and quality of service.

\begin{center}\rule{0.5\linewidth}{0.5pt}\end{center}

\section{Problem 15.10.1}\label{problem-15.10.1}

\textbf{Given:} A 1,500-byte packet traverses a path consisting of 2,000
km of single-mode fiber (refractive index n = 1.468), passes through 5
routers each adding 75 μs of processing delay, and the link rate is 100
Gbps.

\textbf{Find:} (a) The propagation delay, (b) the serialization delay,
(c) the total one-way latency, and (d) the round-trip time.

\textbf{Solution:}

\begin{enumerate}
\def\labelenumi{(\alph{enumi})}
\item
  Propagation delay: v\textsubscript{fiber} = c/n = 3.0 × 10⁸ / 1.468 =
  2.044 × 10⁸ m/s t\textsubscript{prop} = 2,000,000 / (2.044 × 10⁸) =
  \textbf{9,785 μs = 9.785 ms}
\item
  Serialization delay: t\textsubscript{serial} = (1,500 × 8) / (100 ×
  10⁹) = 12,000 / 10¹¹ = \textbf{0.12 μs}
\item
  Total one-way latency: Processing delay = 5 × 75 = 375 μs
  t\textsubscript{total} = 9,785 + 0.12 + 375 = \textbf{10,160 μs ≈
  10.16 ms}
\item
  Round-trip time: RTT = 2 × 10.16 = \textbf{20.32 ms}
\end{enumerate}

Propagation delay dominates at 96.3\% of the total.

\begin{center}\rule{0.5\linewidth}{0.5pt}\end{center}

\section{Problem 15.10.2}\label{problem-15.10.2}

\textbf{Given:} A 10 Gbps Ethernet link carries traffic at 7.5 Gbps
average utilization. The frames are 1,000 bytes average size with 20
bytes of Ethernet preamble and interframe gap overhead per frame.

\textbf{Find:} (a) The link utilization percentage, (b) the frame rate
in frames per second, (c) the goodput if each frame carries 954 bytes of
payload (after headers), and (d) the protocol efficiency.

\textbf{Solution:}

\begin{enumerate}
\def\labelenumi{(\alph{enumi})}
\item
  Link utilization: U = 7.5 / 10 = \textbf{75\%}
\item
  Each frame on the wire = 1,000 + 20 = 1,020 bytes = 8,160 bits Frame
  rate = 7.5 × 10⁹ / 8,160 = \textbf{919,118 frames/s ≈ 919 kfps}
\end{enumerate}

Wait --- the 7.5 Gbps includes all overhead. Recalculating: Frame rate =
7.5 × 10⁹ / (1,020 × 8) = 7.5 × 10⁹ / 8,160 = \textbf{919,118 frames/s}

\begin{enumerate}
\def\labelenumi{(\alph{enumi})}
\setcounter{enumi}{2}
\item
  Goodput (payload throughput): Goodput = 919,118 × 954 × 8 = 919,118 ×
  7,632 = 7.014 × 10⁹ bps = \textbf{7.01 Gbps}
\item
  Protocol efficiency: η = payload / total on-wire = 954 / 1,020 =
  \textbf{93.5\%}
\end{enumerate}

\begin{center}\rule{0.5\linewidth}{0.5pt}\end{center}

\section{Problem 15.10.3}\label{problem-15.10.3}

\textbf{Given:} A 40 Gbps fiber optic link operates continuously. The
pre-FEC BER is 10⁻⁴ and the FEC (with 7\% overhead) corrects to a
post-FEC BER of 10⁻¹⁵.

\textbf{Find:} (a) The number of pre-FEC errors per second, (b) the
effective line rate with FEC overhead, (c) the expected time between
post-FEC errors, and (d) the FEC coding gain in orders of magnitude.

\textbf{Solution:}

\begin{enumerate}
\def\labelenumi{(\alph{enumi})}
\item
  Pre-FEC errors per second: Line rate with FEC = 40 × 1.07 = 42.8 Gbps
  Errors/s = 42.8 × 10⁹ × 10⁻⁴ = \textbf{4.28 × 10⁶ = 4.28 million
  errors/s}
\item
  Effective line rate: \textbf{42.8 Gbps} on the wire, carrying
  \textbf{40 Gbps} of user data
\item
  Post-FEC error rate: Errors/s = 40 × 10⁹ × 10⁻¹⁵ = 4.0 × 10⁻⁵ errors/s
  Mean time between errors = 1 / (4.0 × 10⁻⁵) = \textbf{25,000 seconds ≈
  6.94 hours}
\item
  FEC coding gain: Pre-FEC BER = 10⁻⁴, Post-FEC BER = 10⁻¹⁵ Improvement
  = 10⁻⁴ / 10⁻¹⁵ = 10¹¹ Coding gain = \textbf{11 orders of magnitude}
\end{enumerate}

\begin{center}\rule{0.5\linewidth}{0.5pt}\end{center}

\section{Problem 15.10.4}\label{problem-15.10.4}

\textbf{Given:} A PTP grandmaster clock with a rubidium oscillator
(±0.001 ppm accuracy) synchronizes a boundary clock over a network with
mean path delay of 5.2 μs. The boundary clock has a TCXO oscillator with
±2 ppm free-running accuracy.

\textbf{Find:} (a) The clock drift of the boundary clock over a
10-minute holdover period, (b) the maximum PTP sync interval to maintain
±500 ns accuracy, (c) the grandmaster drift over 24 hours, and (d) the
delay asymmetry error if the forward path is 200 ns longer than the
return path.

\textbf{Solution:}

\begin{enumerate}
\def\labelenumi{(\alph{enumi})}
\item
  Boundary clock drift during holdover: Drift rate = ±2 ppm = ±2 μs/s
  Over 10 minutes: drift = 2 × 10⁻⁶ × 600 = \textbf{1,200 μs = 1.2 ms}
\item
  Maximum sync interval for ±500 ns: T\textsubscript{sync} \textless{}
  500 ns / (2 μs/s) = 500 × 10⁻⁹ / (2 × 10⁻⁶) = \textbf{0.25 s = 250 ms}
\end{enumerate}

PTP sync messages should be sent at least 4 per second. A typical
configuration of 8 messages/s provides adequate margin.

\begin{enumerate}
\def\labelenumi{(\alph{enumi})}
\setcounter{enumi}{2}
\item
  Grandmaster drift over 24 hours: Drift = 0.001 × 10⁻⁶ × 86,400 =
  \textbf{86.4 ns} over 24 hours
\item
  Delay asymmetry error: Asymmetry = 200 ns → time offset = 200/2 =
  \textbf{100 ns}
\end{enumerate}

This 100 ns error is constant and can be compensated if the asymmetry is
measured and configured.

\begin{center}\rule{0.5\linewidth}{0.5pt}\end{center}

\section{Problem 15.10.5}\label{problem-15.10.5}

\textbf{Given:} A copper Ethernet link has a usable bandwidth of 500 MHz
and achieves an SNR of 35 dB.

\textbf{Find:} (a) The Shannon channel capacity, (b) the spectral
efficiency of a 10GBASE-T system operating over this channel, and (c)
the margin to the Shannon limit.

\textbf{Solution:}

\begin{enumerate}
\def\labelenumi{(\alph{enumi})}
\item
  Shannon capacity: SNR (linear) = 10\textsuperscript{35/10} = 3,162 C =
  B × log₂(1 + SNR) = 500 × 10⁶ × log₂(3,163) = 500 × 10⁶ × 11.627 C =
  \textbf{5,813.5 Mbps ≈ 5.81 Gbps}
\item
  Spectral efficiency of 10GBASE-T: η = 10,000 / 500 = \textbf{20.0
  bps/Hz} (across all four pairs)
\end{enumerate}

Per pair: η = 2,500 / 125 = \textbf{20.0 bps/Hz} (using 125 MHz per pair
with PAM-16 encoding and DSP)

\begin{enumerate}
\def\labelenumi{(\alph{enumi})}
\setcounter{enumi}{2}
\tightlist
\item
  Shannon limit per pair at 125 MHz with same SNR: C\textsubscript{pair}
  = 125 × 10⁶ × 11.627 = 1,453 Mbps
\end{enumerate}

Actual per pair = 2,500 Mbps, which exceeds the single-pair Shannon
limit. This is because 10GBASE-T uses crosstalk cancellation (MIMO-like
processing across all four pairs), effectively exploiting the full 500
MHz × 4 channel. The total capacity of 10 Gbps vs.~Shannon limit of 5.81
Gbps suggests the effective SNR with DSP processing is higher, or the
system uses additional coding dimensions.

\begin{center}\rule{0.5\linewidth}{0.5pt}\end{center}

\section{Problem 15.10.6}\label{problem-15.10.6}

\textbf{Given:} An enterprise WAN link has 200 Mbps capacity. Traffic
classes: 30 concurrent G.711 VoIP calls (87.2 kbps each), 50 Mbps of
business application traffic (AF21), and the remainder is best-effort.
The QoS policy allocates 5\% LLQ for EF, 40\% WFQ for AF21, and 55\% for
BE.

\textbf{Find:} (a) The VoIP bandwidth requirement, (b) whether the LLQ
allocation is sufficient, (c) the bandwidth available for best-effort
traffic, and (d) the serialization delay for a voice packet (160 bytes +
headers = 200 bytes total).

\textbf{Solution:}

\begin{enumerate}
\def\labelenumi{(\alph{enumi})}
\item
  VoIP bandwidth: 30 × 87.2 = \textbf{2,616 kbps = 2.62 Mbps}
\item
  LLQ allocation: LLQ = 200 × 0.05 = 10 Mbps VoIP = 2.62 Mbps
  Utilization = 2.62 / 10 = 26.2\% → \textbf{Sufficient} with 73.8\%
  headroom
\item
  Business traffic gets: 200 × 0.40 = 80 Mbps (50 Mbps used, 30 Mbps
  unused) Best-effort allocation: 200 × 0.55 = 110 Mbps
\end{enumerate}

With unused LLQ and AF21 bandwidth available, best-effort can burst
higher: Available = 200 − 2.62 − 50 = \textbf{147.4 Mbps} maximum for
best-effort during normal operation

\begin{enumerate}
\def\labelenumi{(\alph{enumi})}
\setcounter{enumi}{3}
\tightlist
\item
  Serialization delay for 200-byte voice packet: t = (200 × 8) / (200 ×
  10⁶) = 1,600 / 2 × 10⁸ = \textbf{8.0 μs}
\end{enumerate}

\begin{center}\rule{0.5\linewidth}{0.5pt}\end{center}

\section{Problem 15.10.7}\label{problem-15.10.7}

\textbf{Given:} A SyncE-enabled network distributes frequency from a PRC
(Primary Reference Clock, ±1 × 10⁻¹¹) through 10 cascaded SyncE nodes.
Each node has a noise generation limit of 0.01 UI peak-to-peak jitter at
2.048 MHz.

\textbf{Find:} (a) The PRC frequency accuracy in Hz at 2.048 MHz, (b)
the maximum wander accumulation through 10 nodes, (c) the time interval
error (TIE) accumulated over 1 second per node, and (d) whether the
chain meets the ±4.6 ppm end-to-end requirement.

\textbf{Solution:}

\begin{enumerate}
\def\labelenumi{(\alph{enumi})}
\item
  PRC frequency accuracy: Δf = 2.048 × 10⁶ × 1 × 10⁻¹¹ = \textbf{2.048 ×
  10⁻⁵ Hz = 0.0205 mHz}
\item
  SyncE wander accumulates as the square root of the number of nodes for
  uncorrelated noise: Total jitter ≈ 0.01 × √10 = 0.01 × 3.162 =
  \textbf{0.032 UI} peak-to-peak
\item
  TIE per node: 1 UI at 2.048 MHz = 1 / 2.048 × 10⁶ = 488 ns 0.01 UI =
  4.88 ns per node per observation period
\item
  The end-to-end frequency accuracy from PRC through 10 SyncE nodes:
  Each node adds negligible frequency error when locked (specified
  \textless{} ±0.01 ppm). After 10 nodes: total ≈ 10 × 0.01 = 0.1 ppm
  worst case, which is \textbf{well within} the ±4.6 ppm requirement for
  EEC (Ethernet Equipment Clock, per G.8262).
\end{enumerate}

\begin{center}\rule{0.5\linewidth}{0.5pt}\end{center}

\section{Problem 15.10.8}\label{problem-15.10.8}

\textbf{Given:} A data center interconnect uses 400G ZR coherent optics
at 1550 nm over 80 km of single-mode fiber. The transmitter launches +1
dBm per channel. Fiber attenuation is 0.20 dB/km, and connector losses
total 2.0 dB.

\textbf{Find:} (a) The total link loss, (b) the received power, (c) the
required OSNR for DP-16QAM at BER = 4 × 10⁻² (approximately 16 dB), and
(d) the OSNR margin.

\textbf{Solution:}

\begin{enumerate}
\def\labelenumi{(\alph{enumi})}
\item
  Total link loss: Fiber: 0.20 × 80 = 16.0 dB Connectors: 2.0 dB Total =
  \textbf{18.0 dB}
\item
  Received power: P\textsubscript{rx} = +1 − 18.0 = \textbf{−17.0 dBm}
\item
  Required OSNR = \textbf{16 dB} (for DP-16QAM at soft-decision FEC
  threshold with implementation penalty)
\item
  OSNR at receiver (unamplified link, using receiver noise as limiting
  factor): For a coherent receiver with integrated LO, the OSNR is
  determined by the received signal power relative to the receiver
  noise. With P\textsubscript{rx} = −17 dBm and typical coherent
  receiver sensitivity of −22 dBm at required OSNR:
\end{enumerate}

Margin = P\textsubscript{rx} − P\textsubscript{sensitivity} = −17 −
(−22) = \textbf{5 dB margin}

This provides adequate margin for connector aging, splice repairs, and
temperature variations.

\begin{center}\rule{0.5\linewidth}{0.5pt}\end{center}

\section{Problem 15.10.9}\label{problem-15.10.9}

\textbf{Given:} A network monitoring system measures the following
latency statistics over a 24-hour period for a WAN link: mean latency =
45 ms, 95th percentile = 62 ms, 99th percentile = 85 ms, maximum = 210
ms. The VoIP quality requirement is one-way delay \textless{} 150 ms.

\textbf{Find:} (a) The percentage of time the link meets the VoIP
requirement, (b) the jitter (variation in delay), (c) the estimated
packet loss rate if packets arriving \textgreater{} 150 ms are discarded
by the jitter buffer, and (d) the MOS score impact.

\textbf{Solution:}

\begin{enumerate}
\def\labelenumi{(\alph{enumi})}
\item
  Since the 99th percentile is 85 ms (well below 150 ms), and the
  maximum is 210 ms, the link meets the VoIP requirement for
  \textbf{\textgreater99\% but \textless100\%} of the time. The
  exceedances are rare outliers.
\item
  Jitter (approximated as the difference between 95th percentile and
  mean): Jitter ≈ 62 − 45 = \textbf{17 ms} (P95 − mean) Peak jitter =
  210 − 45 = 165 ms
\item
  Packets exceeding 150 ms threshold: The 99th percentile is 85 ms. With
  a roughly exponential tail, packets \textgreater{} 150 ms would
  represent approximately \textbf{0.1--0.5\%} of traffic (estimated as
  fewer than 1 in 200 packets).
\item
  ITU G.107 E-model guidelines:
\end{enumerate}

\begin{itemize}
\tightlist
\item
  Mean delay 45 ms: minimal impact (\textless{} 100 ms is rated ``very
  good'')
\item
  Jitter 17 ms: manageable with a 60 ms jitter buffer
\item
  Packet loss 0.1--0.5\%: slight degradation
\item
  Expected MOS: approximately \textbf{4.0--4.1} (good quality, on a 1--5
  scale)
\end{itemize}

\begin{center}\rule{0.5\linewidth}{0.5pt}\end{center}

\section{Problem 15.10.10}\label{problem-15.10.10}

\textbf{Given:} A WRED (Weighted Random Early Detection) policy is
configured on a router interface with 100 Mbps capacity. The policy
specifies: AF21 traffic begins dropping at 40\% queue depth and reaches
100\% drop at 80\% queue depth. AF11 traffic begins dropping at 60\%
queue depth and reaches 100\% drop at 90\%. The queue buffer is 1 MB.

\textbf{Find:} (a) The queue depth thresholds in bytes for AF21, (b) the
queue depth thresholds for AF11, (c) the drop probability for AF21 at
50\% queue depth (linear interpolation), and (d) the maximum queuing
delay at full queue.

\textbf{Solution:}

\begin{enumerate}
\def\labelenumi{(\alph{enumi})}
\item
  AF21 thresholds: Min threshold: 0.40 × 1,048,576 = \textbf{419,430
  bytes} (≈ 410 KB) Max threshold: 0.80 × 1,048,576 = \textbf{838,861
  bytes} (≈ 819 KB)
\item
  AF11 thresholds: Min threshold: 0.60 × 1,048,576 = \textbf{629,146
  bytes} (≈ 614 KB) Max threshold: 0.90 × 1,048,576 = \textbf{943,718
  bytes} (≈ 922 KB)
\item
  AF21 drop probability at 50\% queue depth: Queue at 50\% = 524,288
  bytes Linear interpolation between 40\% (0\% drop) and 80\% (100\%
  drop): Position = (50\% − 40\%) / (80\% − 40\%) = 10/40 = 0.25 Drop
  probability = 0.25 × 100\% = \textbf{25\%}
\item
  Maximum queuing delay (full 1 MB buffer): t = buffer size / link rate
  = (1,048,576 × 8) / (100 × 10⁶) = 8,388,608 / 10⁸ = \textbf{83.9 ms}
\end{enumerate}

WRED prevents the queue from reaching this level for most traffic,
keeping typical queuing delays well below 83.9 ms.

\chapter{Chapter 16 --- Section 16.1: Antenna
Fundamentals}\label{chapter-16-section-16.1-antenna-fundamentals}

Practice problems covering radiation mechanisms and patterns, antenna
parameters (directivity, gain, effective aperture, EIRP, bandwidth,
polarization), and the Friis transmission equation and link budgets.

\begin{center}\rule{0.5\linewidth}{0.5pt}\end{center}

\section{Problem 16.1.1}\label{problem-16.1.1}

\textbf{Given:} A horn antenna has a measured half-power beamwidth of
18° in the E-plane and 22° in the H-plane. The peak sidelobe level is
−20 dB below the main lobe.

\textbf{Find:} (a) The estimated directivity using D ≈ 32,400 /
(θ\textsubscript{E} × θ\textsubscript{H}), (b) the directivity in dBi,
(c) the sidelobe power as a fraction of the main lobe, and (d) the
first-null beamwidth (FNBW) if FNBW ≈ 2.5 × HPBW.

\textbf{Solution:}

\begin{enumerate}
\def\labelenumi{(\alph{enumi})}
\item
  Directivity: D ≈ 32,400 / (18 × 22) = 32,400 / 396 = \textbf{81.8
  (linear)}
\item
  D (dBi) = 10 log₁₀(81.8) = \textbf{19.1 dBi}
\item
  Sidelobe level = −20 dB: P\textsubscript{SL}/P\textsubscript{main} =
  10⁻²·⁰ = 0.01 = \textbf{1.0\%} of the peak power density
\item
  FNBW in E-plane ≈ 2.5 × 18° = \textbf{45°} FNBW in H-plane ≈ 2.5 × 22°
  = \textbf{55°}
\end{enumerate}

\begin{center}\rule{0.5\linewidth}{0.5pt}\end{center}

\section{Problem 16.1.2}\label{problem-16.1.2}

\textbf{Given:} A parabolic dish antenna has a diameter of 3 m and
operates at 4 GHz (C-band). The aperture efficiency is 0.65.

\textbf{Find:} (a) The wavelength, (b) the gain in dBi, (c) the
effective aperture, (d) the HPBW, and (e) the EIRP if the transmitter
power is 20 W.

\textbf{Solution:}

\begin{enumerate}
\def\labelenumi{(\alph{enumi})}
\item
  λ = c/f = 3 × 10⁸ / 4 × 10⁹ = \textbf{0.075 m}
\item
  G = η(πD/λ)² = 0.65 × (π × 3 / 0.075)² = 0.65 × (125.66)² = 0.65 ×
  15,791 = 10,264 G (dBi) = 10 log₁₀(10,264) = \textbf{40.1 dBi}
\item
  A\textsubscript{e} = Gλ² / (4π) = 10,264 × 0.075² / (4π) = 10,264 ×
  5.625 × 10⁻³ / 12.566 = \textbf{4.59 m²} Check: η ×
  A\textsubscript{phys} = 0.65 × π × 1.5² = 0.65 × 7.069 = 4.59 m² ✓
\item
  HPBW ≈ 70λ/D = 70 × 0.075 / 3 = \textbf{1.75°}
\item
  EIRP = P\textsubscript{t} × G = 20 × 10,264 = 205,280 W EIRP (dBW) =
  10 log₁₀(20) + 40.1 = 13.0 + 40.1 = \textbf{53.1 dBW}
\end{enumerate}

\begin{center}\rule{0.5\linewidth}{0.5pt}\end{center}

\section{Problem 16.1.3}\label{problem-16.1.3}

\textbf{Given:} A point-to-point microwave link operates at 6 GHz over a
distance of 30 km. The transmit antenna gain is 35 dBi and the receive
antenna gain is 35 dBi. The transmitter output power is 1 W (30 dBm).

\textbf{Find:} (a) The free-space path loss (FSPL), (b) the received
power in dBm, (c) the received power in watts, and (d) the system margin
if the receiver sensitivity is −80 dBm.

\textbf{Solution:}

\begin{enumerate}
\def\labelenumi{(\alph{enumi})}
\item
  λ = c/f = 3 × 10⁸ / 6 × 10⁹ = 0.05 m FSPL = 20 log₁₀(4πd/λ) = 20
  log₁₀(4π × 30,000 / 0.05) FSPL = 20 log₁₀(7.54 × 10⁶) = 20 × 6.877 =
  \textbf{137.5 dB}
\item
  P\textsubscript{r} = P\textsubscript{t} + G\textsubscript{t} +
  G\textsubscript{r} − FSPL P\textsubscript{r} = 30 + 35 + 35 − 137.5 =
  \textbf{−37.5 dBm}
\item
  P\textsubscript{r} = 10⁻³·⁷⁵ mW = 1.78 × 10⁻⁴ mW = \textbf{178 nW}
\item
  System margin = P\textsubscript{r} − Sensitivity = −37.5 − (−80) =
  \textbf{42.5 dB}
\end{enumerate}

This generous margin accounts for rain fade, multipath, and equipment
aging over the life of the link.

\begin{center}\rule{0.5\linewidth}{0.5pt}\end{center}

\section{Problem 16.1.4}\label{problem-16.1.4}

\textbf{Given:} A radar system operates at 9.4 GHz (X-band) with a peak
transmit power of 25 kW, an antenna gain of 34 dBi, and a target radar
cross section (RCS) of σ = 5 m². The minimum detectable received power
is −110 dBm.

\textbf{Find:} (a) The wavelength, (b) the maximum detection range using
the radar range equation P\textsubscript{r} = P\textsubscript{t}G²λ²σ /
((4π)³R⁴), (c) the received power at a range of 50 km, and (d) whether
the target is detectable at 50 km.

\textbf{Solution:}

\begin{enumerate}
\def\labelenumi{(\alph{enumi})}
\item
  λ = c/f = 3 × 10⁸ / 9.4 × 10⁹ = \textbf{0.03191 m}
\item
  Convert gain: G = 10\textsuperscript{34/10} = 2,512 (linear) Rearrange
  for R: R⁴ = P\textsubscript{t}G²λ²σ / ((4π)³P\textsubscript{r,min})
  P\textsubscript{r,min} = 10⁻¹¹⁰/¹⁰ mW = 10⁻¹¹ mW = 10⁻¹⁴ W
\end{enumerate}

R⁴ = (25,000 × 2,512² × 0.03191² × 5) / ((4π)³ × 10⁻¹⁴) Numerator =
25,000 × 6.310 × 10⁶ × 1.018 × 10⁻³ × 5 = 25,000 × 32,120 = 8.030 × 10⁸
Denominator = 1,984 × 10⁻¹⁴ = 1.984 × 10⁻¹¹ R⁴ = 8.030 × 10⁸ / 1.984 ×
10⁻¹¹ = 4.047 × 10¹⁹ R = (4.047 × 10¹⁹)⁰·²⁵ = \textbf{79.8 km}

\begin{enumerate}
\def\labelenumi{(\alph{enumi})}
\setcounter{enumi}{2}
\item
  At R = 50 km: P\textsubscript{r} = (25,000 × 6.310 × 10⁶ × 1.018 ×
  10⁻³ × 5) / ((4π)³ × (50,000)⁴) P\textsubscript{r} = 8.030 × 10⁸ /
  (1,984 × 6.25 × 10¹⁸) = 8.030 × 10⁸ / 1.24 × 10²² = 6.476 × 10⁻¹⁴ W
  P\textsubscript{r} = 10 log₁₀(6.476 × 10⁻¹⁴) + 30 = (−131.9) + 30 =
  \textbf{−101.9 dBm}
\item
  Since −101.9 dBm \textgreater{} −110 dBm, the target \textbf{is
  detectable} at 50 km with a margin of 8.1 dB.
\end{enumerate}

\begin{center}\rule{0.5\linewidth}{0.5pt}\end{center}

\section{Problem 16.1.5}\label{problem-16.1.5}

\textbf{Given:} An antenna has a gain of 12 dBi and operates at 2.4 GHz.
The antenna's radiation efficiency is 85\%.

\textbf{Find:} (a) The directivity, (b) the effective aperture, (c) the
physical aperture if the antenna is a horn with aperture efficiency
η\textsubscript{ap} = 0.51, and (d) the beam solid angle
Ω\textsubscript{A} = 4π/D.

\textbf{Solution:}

\begin{enumerate}
\def\labelenumi{(\alph{enumi})}
\item
  G = ηD, so D = G / η G (linear) = 10\textsuperscript{12/10} = 15.85 D
  = 15.85 / 0.85 = \textbf{18.65 (12.7 dBi)}
\item
  λ = c/f = 3 × 10⁸ / 2.4 × 10⁹ = 0.125 m A\textsubscript{e} = Gλ² /
  (4π) = 15.85 × 0.125² / (4π) = 15.85 × 0.01563 / 12.566 =
  \textbf{0.01972 m²} (197.2 cm²)
\item
  For a horn: A\textsubscript{e} = η\textsubscript{ap} ×
  A\textsubscript{phys} A\textsubscript{phys} = A\textsubscript{e} /
  η\textsubscript{ap} = 0.01972 / 0.51 = \textbf{0.03867 m²} (386.7 cm²)
  For a square aperture: side = √0.03867 = \textbf{0.197 m} (19.7 cm)
\item
  Ω\textsubscript{A} = 4π / D = 4π / 18.65 = \textbf{0.674 steradians}
\end{enumerate}

\begin{center}\rule{0.5\linewidth}{0.5pt}\end{center}

\section{Problem 16.1.6}\label{problem-16.1.6}

\textbf{Given:} A vertically polarized transmit antenna communicates
with a receive antenna tilted 30° from vertical (polarization mismatch).
The receive antenna gain is 8 dBi. The transmitter sends 100 mW at 5.8
GHz over a 200 m line-of-sight path.

\textbf{Find:} (a) The polarization loss factor, (b) the FSPL, (c) the
received power without polarization mismatch, and (d) the received power
with polarization mismatch.

\textbf{Solution:}

\begin{enumerate}
\def\labelenumi{(\alph{enumi})}
\item
  Polarization loss factor (PLF) = cos²(Δφ) = cos²(30°) = (0.866)² =
  0.75 PLF (dB) = 10 log₁₀(0.75) = \textbf{−1.25 dB}
\item
  λ = c/f = 3 × 10⁸ / 5.8 × 10⁹ = 0.05172 m FSPL = 20 log₁₀(4πd/λ) = 20
  log₁₀(4π × 200 / 0.05172) FSPL = 20 log₁₀(4.862 × 10⁴) = 20 × 4.687 =
  \textbf{93.7 dB}
\item
  Assume G\textsubscript{t} = 0 dBi (isotropic transmit antenna for
  simplicity): P\textsubscript{r} = P\textsubscript{t} +
  G\textsubscript{t} + G\textsubscript{r} − FSPL = 20 + 0 + 8 − 93.7 =
  \textbf{−65.7 dBm}
\item
  With polarization mismatch: P\textsubscript{r} = −65.7 − 1.25 =
  \textbf{−67.0 dBm}
\end{enumerate}

The 1.25 dB polarization loss reduces the received power from 269 pW to
200 pW.

\begin{center}\rule{0.5\linewidth}{0.5pt}\end{center}

\section{Problem 16.1.7}\label{problem-16.1.7}

\textbf{Given:} A Ku-band satellite link operates at 14 GHz (uplink).
The earth station transmits 50 W through a 3.8 m dish with 62\% aperture
efficiency. The satellite is at geostationary orbit (35,786 km
altitude).

\textbf{Find:} (a) The transmit antenna gain, (b) the EIRP, (c) the
free-space path loss, and (d) the power flux density at the satellite in
dBW/m².

\textbf{Solution:}

\begin{enumerate}
\def\labelenumi{(\alph{enumi})}
\item
  λ = c/f = 3 × 10⁸ / 14 × 10⁹ = 0.02143 m G\textsubscript{t} = η(πD/λ)²
  = 0.62 × (π × 3.8 / 0.02143)² = 0.62 × (556.9)² = 0.62 × 310,137 =
  192,285 G\textsubscript{t} (dBi) = 10 log₁₀(192,285) = \textbf{52.8
  dBi}
\item
  EIRP = P\textsubscript{t} × G\textsubscript{t} = 50 × 192,285 = 9.614
  × 10⁶ W EIRP (dBW) = 10 log₁₀(50) + 52.8 = 17.0 + 52.8 = \textbf{69.8
  dBW}
\item
  FSPL = 20 log₁₀(4π × 35,786,000 / 0.02143) FSPL = 20 log₁₀(2.096 ×
  10¹⁰) = 20 × 10.321 = \textbf{206.4 dB}
\item
  Power flux density at satellite: Φ = EIRP / (4πd²) = 9.614 × 10⁶ / (4π
  × (3.5786 × 10⁷)²) Φ = 9.614 × 10⁶ / (1.609 × 10¹⁶) = 5.975 × 10⁻¹⁰
  W/m² Φ (dBW/m²) = 10 log₁₀(5.975 × 10⁻¹⁰) = \textbf{−92.2 dBW/m²}
\end{enumerate}

\begin{center}\rule{0.5\linewidth}{0.5pt}\end{center}

\section{Problem 16.1.8}\label{problem-16.1.8}

\textbf{Given:} An antenna has a radiation resistance
R\textsubscript{rad} = 73 Ω, a loss resistance R\textsubscript{loss} = 5
Ω, and an antenna reactance X\textsubscript{ant} = +j25 Ω. It is
connected to a 50 Ω transmission line.

\textbf{Find:} (a) The radiation efficiency, (b) the input impedance,
(c) the reflection coefficient, (d) the VSWR, and (e) the overall
efficiency including mismatch loss.

\textbf{Solution:}

\begin{enumerate}
\def\labelenumi{(\alph{enumi})}
\item
  Radiation efficiency: η\textsubscript{rad} = R\textsubscript{rad} /
  (R\textsubscript{rad} + R\textsubscript{loss}) = 73 / (73 + 5) = 73 /
  78 = \textbf{0.936 (93.6\%)}
\item
  Input impedance: Z\textsubscript{in} = R\textsubscript{rad} +
  R\textsubscript{loss} + jX\textsubscript{ant} = 73 + 5 + j25 =
  \textbf{78 + j25 Ω}
\item
  Reflection coefficient: Γ = (Z\textsubscript{in} − Z₀) /
  (Z\textsubscript{in} + Z₀) = (78 + j25 − 50) / (78 + j25 + 50) = (28 +
  j25) / (128 + j25) \textbar Γ\textbar{} = √(28² + 25²) / √(128² + 25²)
  = √(784 + 625) / √(16,384 + 625) = √1,409 / √17,009 = 37.54 / 130.4 =
  \textbf{0.288}
\item
  VSWR = (1 + 0.288) / (1 − 0.288) = 1.288 / 0.712 = \textbf{1.81:1}
\item
  Mismatch efficiency = 1 − \textbar Γ\textbar² = 1 − 0.0830 = 0.917
  Overall efficiency = η\textsubscript{rad} × η\textsubscript{mismatch}
  = 0.936 × 0.917 = \textbf{0.858 (85.8\%)}
\end{enumerate}

\begin{center}\rule{0.5\linewidth}{0.5pt}\end{center}

\section{Problem 16.1.9}\label{problem-16.1.9}

\textbf{Given:} A circularly polarized (RHCP) transmit antenna sends a
signal to a linearly polarized receive antenna. The receive antenna gain
is 6 dBi at 1.575 GHz (GPS L1), and the satellite transmits with an EIRP
of 26 dBW at a range of 20,200 km.

\textbf{Find:} (a) The polarization loss for circular-to-linear
reception, (b) the FSPL, (c) the received power, and (d) the received
power if the receive antenna were also RHCP with the same gain.

\textbf{Solution:}

\begin{enumerate}
\def\labelenumi{(\alph{enumi})}
\item
  A linearly polarized antenna receives only one component of the
  circular polarization. Polarization loss = \textbf{3 dB} (half the
  power is in each orthogonal linear component)
\item
  λ = c/f = 3 × 10⁸ / 1.575 × 10⁹ = 0.1905 m FSPL = 20 log₁₀(4π ×
  20,200,000 / 0.1905) FSPL = 20 log₁₀(1.332 × 10⁹) = 20 × 9.125 =
  \textbf{182.5 dB}
\item
  P\textsubscript{r} = EIRP + G\textsubscript{r} − FSPL − PLF = 26 + 6 −
  182.5 − 3 = \textbf{−153.5 dBW} (or −123.5 dBm)
\item
  With RHCP receive antenna (matched polarization), PLF = 0 dB:
  P\textsubscript{r} = 26 + 6 − 182.5 − 0 = \textbf{−150.5 dBW} (or
  −120.5 dBm)
\end{enumerate}

The 3 dB improvement from matched polarization is significant for GPS
receivers operating near the noise floor.

\begin{center}\rule{0.5\linewidth}{0.5pt}\end{center}

\section{Problem 16.1.10}\label{problem-16.1.10}

\textbf{Given:} An antenna range requires far-field testing of a
parabolic dish with diameter D = 1.5 m at 12 GHz. The reference antenna
has a known gain of 22.0 dBi, and during comparison testing the antenna
under test (AUT) receives 16.5 dB more power than the reference antenna.

\textbf{Find:} (a) The far-field distance, (b) the gain of the AUT, (c)
the effective aperture of the AUT, and (d) the HPBW of the AUT.

\textbf{Solution:}

\begin{enumerate}
\def\labelenumi{(\alph{enumi})}
\item
  λ = c/f = 3 × 10⁸ / 12 × 10⁹ = 0.025 m d\textsubscript{ff} = 2D² / λ =
  2 × 1.5² / 0.025 = 2 × 2.25 / 0.025 = \textbf{180 m}
\item
  G\textsubscript{AUT} = G\textsubscript{ref} + ΔP = 22.0 + 16.5 =
  \textbf{38.5 dBi}
\item
  G\textsubscript{AUT} (linear) = 10\textsuperscript{38.5/10} = 7,079
  A\textsubscript{e} = Gλ² / (4π) = 7,079 × 0.025² / (4π) = 7,079 × 6.25
  × 10⁻⁴ / 12.566 = \textbf{0.352 m²}
\item
  HPBW ≈ 70λ / D = 70 × 0.025 / 1.5 = \textbf{1.17°}
\end{enumerate}

The aperture efficiency can be determined: η = A\textsubscript{e} /
A\textsubscript{phys} = 0.352 / (π × 0.75²) = 0.352 / 1.767 = 0.199.
This low efficiency (20\%) suggests significant feed spillover,
blockage, or surface errors that warrant further investigation.

\chapter{Chapter 16 --- Section 16.2: Wire
Antennas}\label{chapter-16-section-16.2-wire-antennas}

Practice problems covering dipole antennas (short dipole, half-wave
dipole, folded dipole), monopole and ground plane antennas, loop
antennas, and Yagi-Uda antennas.

\begin{center}\rule{0.5\linewidth}{0.5pt}\end{center}

\section{Problem 16.2.1}\label{problem-16.2.1}

\textbf{Given:} A short dipole antenna has a total length of 15 cm and
operates at 300 MHz.

\textbf{Find:} (a) The wavelength and L/λ ratio, (b) the radiation
resistance using R\textsubscript{rad} = 80π²(L/λ)², (c) the radiation
resistance of a half-wave dipole at the same frequency, and (d) the
efficiency of the short dipole if the ohmic loss resistance is 1.5 Ω.

\textbf{Solution:}

\begin{enumerate}
\def\labelenumi{(\alph{enumi})}
\item
  λ = c/f = 3 × 10⁸ / 300 × 10⁶ = 1.0 m L/λ = 0.15 / 1.0 = \textbf{0.15}
\item
  R\textsubscript{rad} = 80π²(L/λ)² = 80 × 9.8696 × (0.15)² = 789.57 ×
  0.0225 = \textbf{17.8 Ω}
\item
  Half-wave dipole: R\textsubscript{rad} = \textbf{73 Ω} (standard value
  at resonance)
\item
  Efficiency of the short dipole: η = R\textsubscript{rad} /
  (R\textsubscript{rad} + R\textsubscript{loss}) = 17.8 / (17.8 + 1.5) =
  17.8 / 19.3 = \textbf{0.922 (92.2\%)}
\end{enumerate}

The short dipole at L = 0.15λ has reasonable efficiency because its
radiation resistance (17.8 Ω) still dominates the loss resistance. For
much shorter dipoles (L \textless{} λ/20), the radiation resistance
drops below 2 Ω and efficiency degrades severely.

\begin{center}\rule{0.5\linewidth}{0.5pt}\end{center}

\section{Problem 16.2.2}\label{problem-16.2.2}

\textbf{Given:} A half-wave dipole operates at 440 MHz (UHF amateur
band). The antenna is made of aluminum tubing with an ohmic loss
resistance of 0.3 Ω.

\textbf{Find:} (a) The physical half-wavelength, (b) the practical
element length using the 0.95 shortening factor, (c) the radiation
efficiency, (d) the gain in dBi accounting for losses, and (e) the input
impedance.

\textbf{Solution:}

\begin{enumerate}
\def\labelenumi{(\alph{enumi})}
\item
  λ = c/f = 3 × 10⁸ / 440 × 10⁶ = 0.6818 m Half-wavelength = λ/2 =
  \textbf{0.341 m (34.1 cm)}
\item
  Practical length = 0.95 × 0.341 = \textbf{0.324 m (32.4 cm)} Each arm
  = 16.2 cm from the feed point.
\item
  η = R\textsubscript{rad} / (R\textsubscript{rad} +
  R\textsubscript{loss}) = 73 / (73 + 0.3) = 73 / 73.3 = \textbf{0.996
  (99.6\%)}
\item
  Directivity of a half-wave dipole = 2.15 dBi (1.64 linear). G = ηD =
  0.996 × 1.64 = 1.633 G (dBi) = 10 log₁₀(1.633) = \textbf{2.13 dBi}
\end{enumerate}

The 0.02 dB loss from ohmic resistance is negligible for this full-size
element.

\begin{enumerate}
\def\labelenumi{(\alph{enumi})}
\setcounter{enumi}{4}
\tightlist
\item
  At the shortened resonant length: Z\textsubscript{in} ≈ \textbf{73 +
  j0 Ω} (the shortening factor eliminates the reactive component).
\end{enumerate}

\begin{center}\rule{0.5\linewidth}{0.5pt}\end{center}

\section{Problem 16.2.3}\label{problem-16.2.3}

\textbf{Given:} A folded dipole antenna is designed for the FM broadcast
band at 98 MHz. Both conductors have the same diameter.

\textbf{Find:} (a) The wavelength and element length, (b) the input
impedance, (c) the VSWR when connected to 300 Ω twin-lead, (d) the VSWR
when connected to 50 Ω coax without a matching network, and (e) the
bandwidth advantage over a simple dipole.

\textbf{Solution:}

\begin{enumerate}
\def\labelenumi{(\alph{enumi})}
\item
  λ = c/f = 3 × 10⁸ / 98 × 10⁶ = 3.061 m Element length = 0.95 × λ/2 =
  0.95 × 1.531 = \textbf{1.454 m}
\item
  Folded dipole impedance = 4 × 73 = \textbf{292 Ω}
\item
  VSWR on 300 Ω twin-lead: Γ = (300 − 292) / (300 + 292) = 8 / 592 =
  0.0135 VSWR = (1 + 0.0135) / (1 − 0.0135) = \textbf{1.03:1} --- an
  excellent match
\item
  VSWR on 50 Ω coax: Γ = (292 − 50) / (292 + 50) = 242 / 342 = 0.708
  VSWR = (1 + 0.708) / (1 − 0.708) = 1.708 / 0.292 = \textbf{5.85:1} ---
  a severe mismatch requiring a 4:1 balun
\item
  A simple dipole has a bandwidth of approximately \textbf{5--8\%} of
  center frequency. The folded dipole achieves \textbf{10--20\%}
  bandwidth due to its broader impedance characteristic. At 98 MHz, this
  corresponds to approximately 10--20 MHz, covering a significant
  portion of the FM band (88--108 MHz).
\end{enumerate}

\begin{center}\rule{0.5\linewidth}{0.5pt}\end{center}

\section{Problem 16.2.4}\label{problem-16.2.4}

\textbf{Given:} A quarter-wave monopole is designed for a marine VHF
radio at 156.8 MHz (Channel 16, international distress). The antenna is
mounted on a metal mast with a ground plane radius of 0.3λ.

\textbf{Find:} (a) The element length, (b) the input impedance, (c) the
gain, (d) the VSWR on 50 Ω coax, and (e) the minimum ground plane radius
in centimeters.

\textbf{Solution:}

\begin{enumerate}
\def\labelenumi{(\alph{enumi})}
\item
  λ = c/f = 3 × 10⁸ / 156.8 × 10⁶ = 1.913 m Quarter-wave length = λ/4 =
  1.913 / 4 = \textbf{0.478 m (47.8 cm)}
\item
  Z\textsubscript{in} ≈ \textbf{36.5 Ω} (half the dipole impedance due
  to ground plane image)
\item
  Gain = \textbf{5.15 dBi} (3.0 dBd) --- the ground plane doubles the
  directivity compared to a dipole
\item
  VSWR on 50 Ω coax: Γ = (50 − 36.5) / (50 + 36.5) = 13.5 / 86.5 = 0.156
  VSWR = (1 + 0.156) / (1 − 0.156) = 1.156 / 0.844 = \textbf{1.37:1}
\end{enumerate}

This is an acceptable match (return loss = 16.1 dB).

\begin{enumerate}
\def\labelenumi{(\alph{enumi})}
\setcounter{enumi}{4}
\tightlist
\item
  Minimum ground plane radius ≈ λ/4 = 1.913 / 4 = 0.478 m = \textbf{47.8
  cm} The specified 0.3λ = 0.574 m exceeds this minimum, providing
  adequate ground plane performance.
\end{enumerate}

\begin{center}\rule{0.5\linewidth}{0.5pt}\end{center}

\section{Problem 16.2.5}\label{problem-16.2.5}

\textbf{Given:} An AM broadcast tower operates at 1,000 kHz as a
quarter-wave monopole. The tower has 120 buried radial wires, each λ/4
long. The total ground system loss resistance is 2 Ω.

\textbf{Find:} (a) The wavelength and tower height, (b) the radiation
resistance, (c) the radiation efficiency, (d) the radial wire length,
and (e) the total wire required for the ground system.

\textbf{Solution:}

\begin{enumerate}
\def\labelenumi{(\alph{enumi})}
\item
  λ = c/f = 3 × 10⁸ / 1 × 10⁶ = 300 m Tower height = λ/4 = \textbf{75 m
  (246 ft)}
\item
  R\textsubscript{rad} = \textbf{36.5 Ω} (quarter-wave monopole over
  ground)
\item
  η = R\textsubscript{rad} / (R\textsubscript{rad} +
  R\textsubscript{loss}) = 36.5 / (36.5 + 2) = 36.5 / 38.5 =
  \textbf{0.948 (94.8\%)}
\item
  Each radial wire = λ/4 = \textbf{75 m}
\item
  Total wire = 120 × 75 = \textbf{9,000 m (9.0 km)}
\end{enumerate}

This extensive ground system is standard for AM broadcast. Reducing the
number of radials to 60 would approximately double the ground loss
resistance to 4 Ω, lowering efficiency to 90.1\%, which represents a
0.22 dB reduction in radiated power compared to the 120-radial
configuration.

\begin{center}\rule{0.5\linewidth}{0.5pt}\end{center}

\section{Problem 16.2.6}\label{problem-16.2.6}

\textbf{Given:} A circular loop antenna has a radius of 10 cm and 1
turn, operating at 150 MHz.

\textbf{Find:} (a) The circumference in wavelengths, (b) whether the
loop is electrically small or resonant, (c) the radiation resistance,
(d) the gain, and (e) the direction of maximum radiation.

\textbf{Solution:}

\begin{enumerate}
\def\labelenumi{(\alph{enumi})}
\item
  λ = c/f = 3 × 10⁸ / 150 × 10⁶ = 2.0 m Circumference C = 2π × 0.10 =
  0.6283 m C/λ = 0.6283 / 2.0 = \textbf{0.314}
\item
  Since C/λ = 0.314 is between the small-loop regime (C \textless{}
  0.1λ) and the full-wave loop (C = λ), this is a \textbf{partial-wave
  loop} that does not fit neatly into either category. It is closer to a
  small loop but its radiation resistance will be significantly higher
  than the small-loop approximation suggests. For an accurate design,
  numerical methods (such as NEC) would be needed.
\item
  Using the small-loop formula as an approximation: A = π × 0.10² =
  0.03142 m² R\textsubscript{rad} = 320π⁴(A/λ²)² = 320 × 97.41 ×
  (0.03142 / 4.0)² = 31,171 × (7.854 × 10⁻³)² = 31,171 × 6.168 × 10⁻⁵ =
  \textbf{1.92 Ω}
\end{enumerate}

Note: This formula underestimates the actual radiation resistance for
C/λ = 0.314. The true value would be higher.

\begin{enumerate}
\def\labelenumi{(\alph{enumi})}
\setcounter{enumi}{3}
\item
  A small loop has gain ≈ \textbf{1.76 dBi} (same pattern as a short
  dipole but rotated 90°).
\item
  Maximum radiation is \textbf{in the plane of the loop} (perpendicular
  to the loop axis), with nulls along the loop axis.
\end{enumerate}

\begin{center}\rule{0.5\linewidth}{0.5pt}\end{center}

\section{Problem 16.2.7}\label{problem-16.2.7}

\textbf{Given:} A 3-element Yagi-Uda antenna for the 70 cm amateur band
operates at 432 MHz. The design uses a reflector, a driven element
(folded dipole), and one director with 0.20λ reflector spacing and 0.25λ
director spacing.

\textbf{Find:} (a) The wavelength, (b) the element lengths, (c) the
total boom length, (d) the approximate gain, and (e) the input
impedance.

\textbf{Solution:}

\begin{enumerate}
\def\labelenumi{(\alph{enumi})}
\item
  λ = c/f = 3 × 10⁸ / 432 × 10⁶ = \textbf{0.694 m (69.4 cm)}
\item
  Element lengths: Reflector = 1.05 × λ/2 = 1.05 × 0.347 = \textbf{0.365
  m (36.5 cm)} Driven element (folded dipole) = 0.95 × λ/2 = 0.95 ×
  0.347 = \textbf{0.330 m (33.0 cm)} Director = 0.91 × λ/2 = 0.91 ×
  0.347 = \textbf{0.316 m (31.6 cm)}
\item
  Total boom length: Boom = reflector-to-driven + driven-to-director =
  0.20λ + 0.25λ = 0.45λ Boom = 0.45 × 0.694 = \textbf{0.312 m (31.2 cm)}
\item
  A 3-element Yagi achieves approximately \textbf{7.5 dBi} (5.3 dBd)
  with a front-to-back ratio of 15--20 dB.
\item
  Without the folded dipole, Z\textsubscript{in} ≈ 20--25 Ω due to
  mutual coupling. With the folded dipole (4:1 transformation):
  Z\textsubscript{in} ≈ 4 × 22 = \textbf{88 Ω}. A gamma match or beta
  match can bring this to 50 Ω for direct coax connection.
\end{enumerate}

\begin{center}\rule{0.5\linewidth}{0.5pt}\end{center}

\section{Problem 16.2.8}\label{problem-16.2.8}

\textbf{Given:} A 10-element Yagi-Uda antenna for a 2.4 GHz Wi-Fi link
uses a half-wave dipole driven element, one reflector, and eight
directors. Element spacing is 0.20λ for the reflector and 0.30λ for the
directors.

\textbf{Find:} (a) The wavelength, (b) the total boom length, (c) the
expected gain, (d) the HPBW estimate using θ ≈
65°/√G\textsubscript{linear}, and (e) the input impedance.

\textbf{Solution:}

\begin{enumerate}
\def\labelenumi{(\alph{enumi})}
\item
  λ = c/f = 3 × 10⁸ / 2.4 × 10⁹ = \textbf{0.125 m (12.5 cm)}
\item
  Boom = reflector spacing + 8 × director spacing = 0.20λ + 8 × 0.30λ =
  0.20 + 2.40 = 2.60λ Boom length = 2.60 × 0.125 = \textbf{0.325 m (32.5
  cm)}
\item
  A 10-element Yagi achieves approximately \textbf{13 dBi} (10.8 dBd)
  with a front-to-back ratio exceeding 25 dB.
\item
  G\textsubscript{linear} = 10\textsuperscript{13/10} = 20.0 θ ≈ 65° /
  √20.0 = 65° / 4.47 = \textbf{14.5°} (approximate HPBW in both planes)
\item
  With a simple dipole driven element and strong mutual coupling from 8
  directors: Z\textsubscript{in} ≈ \textbf{18--22 Ω} --- requiring a
  matching network (gamma match, hairpin match, or quarter-wave
  transformer) for 50 Ω coax.
\end{enumerate}

\begin{center}\rule{0.5\linewidth}{0.5pt}\end{center}

\section{Problem 16.2.9}\label{problem-16.2.9}

\textbf{Given:} A ground-plane antenna for 462 MHz (GMRS radio) uses a
vertical quarter-wave element with four radials angled downward at 45°
from horizontal.

\textbf{Find:} (a) The quarter-wave element length, (b) the radial
length, (c) the impedance increase due to drooping radials (impedance
increases from 36.5 Ω toward 50 Ω as radials droop), (d) the VSWR on 50
Ω coax, and (e) the gain.

\textbf{Solution:}

\begin{enumerate}
\def\labelenumi{(\alph{enumi})}
\item
  λ = c/f = 3 × 10⁸ / 462 × 10⁶ = 0.6494 m Element length = λ/4 =
  \textbf{0.1623 m (16.2 cm)}
\item
  Radial length = λ/4 = \textbf{0.1623 m (16.2 cm)} each
\item
  With radials drooped 45° from horizontal, the input impedance
  increases from 36.5 Ω to approximately \textbf{50 Ω} --- the drooping
  radials modify the image current distribution, effectively raising the
  impedance toward a direct match to 50 Ω coax. Empirically, 45° droop
  raises the impedance by a factor of approximately 1.37: 36.5 × 1.37 =
  \textbf{50.0 Ω}.
\item
  VSWR = (50 / 50) = \textbf{1.0:1} --- a nearly perfect match, which is
  why drooping-radial ground-plane antennas are widely used for direct
  50 Ω feed.
\item
  The drooping radials tilt the radiation pattern slightly upward from
  the horizon. Gain is approximately \textbf{4.5 dBi} (2.3 dBd) ---
  slightly less than a flat-radial ground plane due to the altered
  pattern shape.
\end{enumerate}

\begin{center}\rule{0.5\linewidth}{0.5pt}\end{center}

\section{Problem 16.2.10}\label{problem-16.2.10}

\textbf{Given:} An RFID reader uses a rectangular loop antenna with
dimensions 8 cm × 12 cm and 5 turns, operating at 13.56 MHz (HF RFID).

\textbf{Find:} (a) The circumference in wavelengths, (b) the loop area,
(c) the single-turn radiation resistance, (d) the total radiation
resistance for 5 turns, and (e) the inductance of the loop if L ≈ μ₀N²(a
+ b)/π × {[}ln(2(a + b)/w) − 0.774{]}, where a = 0.08 m, b = 0.12 m, and
wire diameter w = 1 mm.

\textbf{Solution:}

\begin{enumerate}
\def\labelenumi{(\alph{enumi})}
\item
  λ = c/f = 3 × 10⁸ / 13.56 × 10⁶ = 22.12 m Circumference = 2(0.08 +
  0.12) = 0.40 m C/λ = 0.40 / 22.12 = \textbf{0.0181} --- well within
  the small-loop regime (C \textless{} 0.1λ)
\item
  A = 0.08 × 0.12 = \textbf{9.6 × 10⁻³ m²} (96 cm²)
\item
  Single-turn radiation resistance: R\textsubscript{rad} = 320π⁴(A/λ²)²
  = 320 × 97.41 × (9.6 × 10⁻³ / 489.3)² R\textsubscript{rad} = 31,171 ×
  (1.962 × 10⁻⁵)² = 31,171 × 3.850 × 10⁻¹⁰ = \textbf{1.20 × 10⁻⁵ Ω}
  (12.0 μΩ)
\item
  For N = 5 turns, R\textsubscript{rad} scales as N²:
  R\textsubscript{total} = 25 × 1.20 × 10⁻⁵ = \textbf{3.00 × 10⁻⁴ Ω}
  (300 μΩ)
\item
  Inductance: L = μ₀N²(a + b)/π × {[}ln(2(a + b)/w) − 0.774{]} L = 4π ×
  10⁻⁷ × 25 × 0.20 / π × {[}ln(2 × 0.20 / 0.001) − 0.774{]} L = 4 × 10⁻⁷
  × 25 × 0.20 × {[}ln(400) − 0.774{]} L = 2.0 × 10⁻⁶ × {[}5.991 −
  0.774{]} L = 2.0 × 10⁻⁶ × 5.217 = \textbf{10.4 μH}
\end{enumerate}

The extremely low radiation resistance confirms that HF RFID operates
via near-field inductive coupling, not radiation. The loop functions as
one half of a loosely coupled transformer with the tag coil.

\chapter{Chapter 16 --- Section 16.3: Aperture
Antennas}\label{chapter-16-section-16.3-aperture-antennas}

Practice problems covering horn antennas (pyramidal, sectoral),
parabolic reflector antennas (prime focus, Cassegrain, offset), and slot
and cavity-backed antennas.

\begin{center}\rule{0.5\linewidth}{0.5pt}\end{center}

\section{Problem 16.3.1}\label{problem-16.3.1}

\textbf{Given:} A pyramidal horn antenna operating at 8 GHz (X-band) has
aperture dimensions of a = 15 cm and b = 12 cm. The aperture efficiency
is η\textsubscript{ap} = 0.51.

\textbf{Find:} (a) The wavelength, (b) the physical aperture area, (c)
the effective aperture, (d) the gain in dBi, and (e) the HPBW in both
planes.

\textbf{Solution:}

\begin{enumerate}
\def\labelenumi{(\alph{enumi})}
\item
  λ = c/f = 3 × 10⁸ / 8 × 10⁹ = \textbf{0.0375 m}
\item
  A\textsubscript{phys} = a × b = 0.15 × 0.12 = \textbf{0.018 m²} (180
  cm²)
\item
  A\textsubscript{e} = η\textsubscript{ap} × A\textsubscript{phys} =
  0.51 × 0.018 = \textbf{9.18 × 10⁻³ m²}
\item
  G = 4πA\textsubscript{e} / λ² = 4π × 9.18 × 10⁻³ / (0.0375)² = 0.1151
  / 1.406 × 10⁻³ = 81.9 G (dBi) = 10 log₁₀(81.9) = \textbf{19.1 dBi}
\item
  HPBW in H-plane ≈ 70λ/a = 70 × 0.0375 / 0.15 = \textbf{17.5°} HPBW in
  E-plane ≈ 70λ/b = 70 × 0.0375 / 0.12 = \textbf{21.9°}
\end{enumerate}

\begin{center}\rule{0.5\linewidth}{0.5pt}\end{center}

\section{Problem 16.3.2}\label{problem-16.3.2}

\textbf{Given:} An optimum-gain pyramidal horn antenna is needed with a
target gain of 23 dBi at 12 GHz. The horn is fed by WR-90 waveguide
(22.86 mm × 10.16 mm internal dimensions). Use η\textsubscript{ap} =
0.51.

\textbf{Find:} (a) The required effective aperture, (b) the physical
aperture dimensions (assume a square aperture), (c) the horn axial
length assuming a flare semi-angle of 12° in each plane, and (d) the
HPBW.

\textbf{Solution:}

\begin{enumerate}
\def\labelenumi{(\alph{enumi})}
\item
  λ = c/f = 3 × 10⁸ / 12 × 10⁹ = 0.025 m G = 10\textsuperscript{23/10} =
  199.5 (linear) A\textsubscript{e} = Gλ² / (4π) = 199.5 × 0.025² / (4π)
  = 199.5 × 6.25 × 10⁻⁴ / 12.566 = \textbf{9.92 × 10⁻³ m²}
\item
  A\textsubscript{phys} = A\textsubscript{e} / η\textsubscript{ap} =
  9.92 × 10⁻³ / 0.51 = 1.945 × 10⁻² m² For a square aperture: a = b =
  √(1.945 × 10⁻²) = \textbf{0.1395 m (13.95 cm)}
\item
  The horn flares from the waveguide dimensions to the aperture. Taking
  the H-plane: Half-aperture = a/2 = 0.0698 m; half-waveguide = 0.01143
  m Flare extent = 0.0698 − 0.01143 = 0.0584 m Horn length = flare
  extent / tan(12°) = 0.0584 / 0.2126 = \textbf{0.275 m (27.5 cm)}
\item
  HPBW ≈ 70λ / a = 70 × 0.025 / 0.1395 = \textbf{12.5°} in both planes
\end{enumerate}

\begin{center}\rule{0.5\linewidth}{0.5pt}\end{center}

\section{Problem 16.3.3}\label{problem-16.3.3}

\textbf{Given:} A prime-focus parabolic reflector antenna has a diameter
of 4.5 m and operates at 7.5 GHz for a satellite earth station. The
aperture efficiency is 0.58 and the f/D ratio is 0.38.

\textbf{Find:} (a) The gain, (b) the HPBW, (c) the focal length, (d) the
feed illumination half-angle, and (e) the far-field distance.

\textbf{Solution:}

\begin{enumerate}
\def\labelenumi{(\alph{enumi})}
\item
  λ = c/f = 3 × 10⁸ / 7.5 × 10⁹ = 0.04 m G = η(πD/λ)² = 0.58 × (π × 4.5
  / 0.04)² = 0.58 × (353.4)² = 0.58 × 124,892 = 72,437 G (dBi) = 10
  log₁₀(72,437) = \textbf{48.6 dBi}
\item
  HPBW ≈ 70λ / D = 70 × 0.04 / 4.5 = \textbf{0.622°}
\item
  Focal length: f = (f/D) × D = 0.38 × 4.5 = \textbf{1.71 m}
\item
  Feed illumination half-angle: θ\textsubscript{f} = 2 arctan(1 / (4 ×
  f/D)) = 2 arctan(1 / 1.52) = 2 arctan(0.658) θ\textsubscript{f} = 2 ×
  33.3° = \textbf{66.7°}
\end{enumerate}

The feed antenna must illuminate the dish out to ±33.3° from the axis.

\begin{enumerate}
\def\labelenumi{(\alph{enumi})}
\setcounter{enumi}{4}
\tightlist
\item
  Far-field distance: d\textsubscript{ff} = 2D² / λ = 2 × 4.5² / 0.04 =
  2 × 20.25 / 0.04 = \textbf{1,012.5 m} (approximately 1 km)
\end{enumerate}

\begin{center}\rule{0.5\linewidth}{0.5pt}\end{center}

\section{Problem 16.3.4}\label{problem-16.3.4}

\textbf{Given:} A Cassegrain antenna has a main reflector diameter of
2.4 m, a subreflector diameter of 0.3 m, and operates at 18 GHz. The
aperture efficiency is 0.62.

\textbf{Find:} (a) The wavelength, (b) the gain, (c) the HPBW, (d) the
blockage ratio (subreflector area to main reflector area), and (e) the
gain reduction due to subreflector blockage (approximately ΔG ≈ −20
log₁₀(1 − blockage ratio)).

\textbf{Solution:}

\begin{enumerate}
\def\labelenumi{(\alph{enumi})}
\item
  λ = c/f = 3 × 10⁸ / 18 × 10⁹ = \textbf{0.01667 m}
\item
  G = η(πD/λ)² = 0.62 × (π × 2.4 / 0.01667)² = 0.62 × (452.2)² = 0.62 ×
  204,485 = 126,781 G (dBi) = 10 log₁₀(126,781) = \textbf{51.0 dBi}
\item
  HPBW ≈ 70λ / D = 70 × 0.01667 / 2.4 = \textbf{0.486°}
\item
  Blockage ratio (area): BR = (d\textsubscript{sub}/D)² = (0.3/2.4)² =
  0.125² = \textbf{0.01563 (1.56\%)}
\item
  Gain reduction: ΔG ≈ −20 log₁₀(1 − 0.01563) = −20 log₁₀(0.9844) = −20
  × (−0.00683) = \textbf{0.14 dB}
\end{enumerate}

The subreflector blockage is minor. In practice, the Cassegrain's
advantage of mounting the feed behind the dish (shorter feed line,
easier maintenance) far outweighs the small blockage penalty.

\begin{center}\rule{0.5\linewidth}{0.5pt}\end{center}

\section{Problem 16.3.5}\label{problem-16.3.5}

\textbf{Given:} An offset parabolic reflector for Ku-band satellite
reception has a projected aperture of 0.75 m × 0.50 m (elliptical due to
offset geometry) and operates at 12.5 GHz. The aperture efficiency is
0.68 (higher than prime-focus due to no blockage).

\textbf{Find:} (a) The wavelength, (b) the projected aperture area, (c)
the gain, (d) the effective aperture, and (e) the HPBW in azimuth and
elevation.

\textbf{Solution:}

\begin{enumerate}
\def\labelenumi{(\alph{enumi})}
\item
  λ = c/f = 3 × 10⁸ / 12.5 × 10⁹ = \textbf{0.024 m}
\item
  A\textsubscript{phys} = π × (0.75/2) × (0.50/2) = π × 0.375 × 0.25 =
  \textbf{0.2945 m²}
\item
  G = η × 4πA\textsubscript{phys} / λ² = 0.68 × 4π × 0.2945 / 0.024² G =
  0.68 × 3.696 / 5.76 × 10⁻⁴ = 2.513 / 5.76 × 10⁻⁴ = 4,363 G (dBi) = 10
  log₁₀(4,363) = \textbf{36.4 dBi}
\item
  A\textsubscript{e} = η × A\textsubscript{phys} = 0.68 × 0.2945 =
  \textbf{0.200 m²}
\item
  HPBW in azimuth (wider dimension): θ\textsubscript{az} ≈ 70λ / 0.75 =
  70 × 0.024 / 0.75 = \textbf{2.24°} HPBW in elevation (narrower
  dimension): θ\textsubscript{el} ≈ 70λ / 0.50 = 70 × 0.024 / 0.50 =
  \textbf{3.36°}
\end{enumerate}

The elliptical beam shape reflects the asymmetric aperture of the offset
geometry.

\begin{center}\rule{0.5\linewidth}{0.5pt}\end{center}

\section{Problem 16.3.6}\label{problem-16.3.6}

\textbf{Given:} A half-wave slot antenna is cut in a ground plane for an
S-band radar at 3.0 GHz. A complementary dipole at the same frequency
has an impedance of Z\textsubscript{dipole} = 73 Ω.

\textbf{Find:} (a) The slot length, (b) the slot impedance using
Babinet's principle, (c) the impedance with a λ/4 cavity backing, (d)
the quarter-wave transformer impedance to match the cavity-backed slot
to 50 Ω, and (e) the gain of the cavity-backed slot.

\textbf{Solution:}

\begin{enumerate}
\def\labelenumi{(\alph{enumi})}
\item
  λ = c/f = 3 × 10⁸ / 3.0 × 10⁹ = 0.10 m Slot length = λ/2 =
  \textbf{50.0 mm}
\item
  Z\textsubscript{slot} = η² / (4 × Z\textsubscript{dipole}) = 377² / (4
  × 73) = 142,129 / 292 = \textbf{486.7 Ω}
\item
  Cavity backing eliminates rear radiation, reducing the impedance by
  approximately half: Z\textsubscript{cavity} ≈ 486.7 / 2 = \textbf{243
  Ω}
\item
  Quarter-wave transformer: Z\textsubscript{match} = √(Z₀ ×
  Z\textsubscript{cavity}) = √(50 × 243) = √12,150 = \textbf{110.2 Ω}
\item
  The unbacked slot has gain ≈ 2.15 dBi (same as a dipole). The cavity
  backing adds 2--3 dB by eliminating rear radiation:
  G\textsubscript{cavity} ≈ 2.15 + 3 = \textbf{5.15 dBi}
\end{enumerate}

\begin{center}\rule{0.5\linewidth}{0.5pt}\end{center}

\section{Problem 16.3.7}\label{problem-16.3.7}

\textbf{Given:} A standard-gain horn antenna is used as a reference for
calibration at 15 GHz. The horn aperture is 10 cm × 8 cm with
η\textsubscript{ap} = 0.51. During a comparison test, the antenna under
test (AUT) receives 8.5 dB more power than the reference horn.

\textbf{Find:} (a) The horn gain, (b) the gain of the AUT, (c) the
effective aperture of the AUT, and (d) the AUT's HPBW if it is a
circular aperture antenna.

\textbf{Solution:}

\begin{enumerate}
\def\labelenumi{(\alph{enumi})}
\item
  λ = c/f = 3 × 10⁸ / 15 × 10⁹ = 0.02 m G\textsubscript{horn} = 4π ×
  η\textsubscript{ap} × a × b / λ² = 4π × 0.51 × 0.10 × 0.08 / 0.02²
  G\textsubscript{horn} = 4π × 4.08 × 10⁻³ / 4 × 10⁻⁴ = 4π × 10.2 =
  128.2 G\textsubscript{horn} (dBi) = 10 log₁₀(128.2) = \textbf{21.1
  dBi}
\item
  G\textsubscript{AUT} = 21.1 + 8.5 = \textbf{29.6 dBi}
\item
  G\textsubscript{AUT} (linear) = 10\textsuperscript{29.6/10} = 912.0
  A\textsubscript{e} = Gλ² / (4π) = 912.0 × 0.02² / (4π) = 912.0 × 4 ×
  10⁻⁴ / 12.566 = \textbf{0.0290 m²} (290 cm²)
\item
  If circular: A\textsubscript{e} = η × π(D/2)². Assuming η = 0.60:
  π(D/2)² = 0.0290 / 0.60 = 0.0484 m² D/2 = √(0.0484/π) = √(0.01540) =
  0.1241 m → D = \textbf{0.248 m} HPBW ≈ 70λ / D = 70 × 0.02 / 0.248 =
  \textbf{5.65°}
\end{enumerate}

\begin{center}\rule{0.5\linewidth}{0.5pt}\end{center}

\section{Problem 16.3.8}\label{problem-16.3.8}

\textbf{Given:} A weather radar uses a parabolic dish with D = 8.5 m at
5.6 GHz (C-band). The aperture efficiency is 0.55 and the peak transmit
power is 250 kW.

\textbf{Find:} (a) The gain, (b) the HPBW, (c) the EIRP in dBW, (d) the
effective aperture, and (e) the far-field distance.

\textbf{Solution:}

\begin{enumerate}
\def\labelenumi{(\alph{enumi})}
\item
  λ = c/f = 3 × 10⁸ / 5.6 × 10⁹ = 0.05357 m G = η(πD/λ)² = 0.55 × (π ×
  8.5 / 0.05357)² = 0.55 × (498.4)² = 0.55 × 248,402 = 136,621 G (dBi) =
  10 log₁₀(136,621) = \textbf{51.4 dBi}
\item
  HPBW ≈ 70λ / D = 70 × 0.05357 / 8.5 = \textbf{0.441°}
\item
  EIRP = P\textsubscript{t} × G = 250,000 × 136,621 = 3.416 × 10¹⁰ W
  EIRP (dBW) = 10 log₁₀(250,000) + 51.4 = 54.0 + 51.4 = \textbf{105.4
  dBW}
\item
  A\textsubscript{e} = η × π(D/2)² = 0.55 × π × 4.25² = 0.55 × 56.75 =
  \textbf{31.2 m²}
\item
  d\textsubscript{ff} = 2D² / λ = 2 × 8.5² / 0.05357 = 144.5 / 0.05357 =
  \textbf{2,697 m (2.7 km)}
\end{enumerate}

\begin{center}\rule{0.5\linewidth}{0.5pt}\end{center}

\section{Problem 16.3.9}\label{problem-16.3.9}

\textbf{Given:} A cavity-backed slot antenna array consists of 4 × 4 =
16 cavity-backed slot elements at 10 GHz. Each element has a gain of 6
dBi, and the elements are spaced at 0.7λ in both dimensions. The array
efficiency (including feed network losses) is 80\%.

\textbf{Find:} (a) The element spacing in mm, (b) the total array
aperture, (c) the array gain, (d) the HPBW, and (e) the total gain in
dBi.

\textbf{Solution:}

\begin{enumerate}
\def\labelenumi{(\alph{enumi})}
\item
  λ = c/f = 3 × 10⁸ / 10 × 10⁹ = 0.03 m Element spacing = 0.7 × 0.03 =
  0.021 m = \textbf{21.0 mm}
\item
  Array aperture = (4 × 21.0) × (4 × 21.0) = 84.0 mm × 84.0 mm =
  \textbf{70.56 cm²}
\item
  Ideal array gain = N × G\textsubscript{element} ×
  η\textsubscript{array} G\textsubscript{element} (linear) =
  10\textsuperscript{6/10} = 3.981 G\textsubscript{total} = 16 × 3.981 ×
  0.80 = 50.96 G\textsubscript{total} (dBi) = 10 log₁₀(50.96) =
  \textbf{17.1 dBi}
\item
  For a 4-element linear array at 0.7λ spacing: HPBW ≈ 0.886λ / (N × d)
  = 0.886 × 0.03 / (4 × 0.021) = 0.02658 / 0.084 = 0.3164 rad =
  \textbf{18.1°} in both planes
\item
  The total gain is \textbf{17.1 dBi} as calculated in part (c). This
  represents an 11.1 dB improvement over a single element (12 dB from
  the 16-element array factor minus 0.9 dB for feed losses).
\end{enumerate}

\begin{center}\rule{0.5\linewidth}{0.5pt}\end{center}

\section{Problem 16.3.10}\label{problem-16.3.10}

\textbf{Given:} A Ku-band satellite earth station uses a Cassegrain
antenna with D = 6.1 m at 14 GHz (uplink) and 11.7 GHz (downlink). The
aperture efficiency is 0.60 at both frequencies. The subreflector
diameter is 0.6 m.

\textbf{Find:} (a) The gain at 14 GHz and 11.7 GHz, (b) the HPBW at each
frequency, (c) the effective aperture at each frequency, and (d) the
blockage loss due to the subreflector.

\textbf{Solution:}

\begin{enumerate}
\def\labelenumi{(\alph{enumi})}
\tightlist
\item
  At 14 GHz: λ = 3 × 10⁸ / 14 × 10⁹ = 0.02143 m G\textsubscript{14} =
  0.60 × (π × 6.1 / 0.02143)² = 0.60 × (894.2)² = 0.60 × 799,594 =
  479,756 G\textsubscript{14} (dBi) = 10 log₁₀(479,756) = \textbf{56.8
  dBi}
\end{enumerate}

At 11.7 GHz: λ = 3 × 10⁸ / 11.7 × 10⁹ = 0.02564 m G\textsubscript{11.7}
= 0.60 × (π × 6.1 / 0.02564)² = 0.60 × (747.6)² = 0.60 × 558,905 =
335,343 G\textsubscript{11.7} (dBi) = 10 log₁₀(335,343) = \textbf{55.3
dBi}

\begin{enumerate}
\def\labelenumi{(\alph{enumi})}
\setcounter{enumi}{1}
\item
  At 14 GHz: HPBW ≈ 70 × 0.02143 / 6.1 = \textbf{0.246°} At 11.7 GHz:
  HPBW ≈ 70 × 0.02564 / 6.1 = \textbf{0.294°}
\item
  At 14 GHz: A\textsubscript{e} = 0.60 × π × 3.05² = 0.60 × 29.22 =
  \textbf{17.53 m²} At 11.7 GHz: A\textsubscript{e} = same physical
  antenna, so A\textsubscript{e} = \textbf{17.53 m²}
\end{enumerate}

Note: The effective aperture depends on the physical aperture and
efficiency, not frequency. The gain is higher at 14 GHz because
A\textsubscript{e}/λ² is larger at higher frequency.

\begin{enumerate}
\def\labelenumi{(\alph{enumi})}
\setcounter{enumi}{3}
\tightlist
\item
  Blockage ratio = (d\textsubscript{sub}/D)² = (0.6/6.1)² = 0.0983² =
  \textbf{0.00967 (0.97\%)} Blockage loss ≈ −20 log₁₀(1 − 0.00967) = −20
  × (−0.00421) = \textbf{0.08 dB} --- negligible
\end{enumerate}

\chapter{Chapter 16 --- Section 16.4: Printed and Microstrip
Antennas}\label{chapter-16-section-16.4-printed-and-microstrip-antennas}

Practice problems covering rectangular patch antenna design (dimensions,
effective dielectric constant, fringing extensions, impedance), patch
antenna variations (circular patches, circular polarization, stacked
patches), and feed techniques.

\begin{center}\rule{0.5\linewidth}{0.5pt}\end{center}

\section{Problem 16.4.1}\label{problem-16.4.1}

\textbf{Given:} A rectangular patch antenna is designed on Rogers
RO4003C substrate (ε\textsubscript{r} = 3.55, h = 0.813 mm) for a 5.8
GHz ISM-band application.

\textbf{Find:} (a) The patch width W, (b) the effective dielectric
constant ε\textsubscript{r,eff}, (c) the fringing length extension ΔL,
(d) the patch length L, and (e) the expected bandwidth if Q ≈
c√ε\textsubscript{r}/(4hf\textsubscript{r}).

\textbf{Solution:}

\begin{enumerate}
\def\labelenumi{(\alph{enumi})}
\item
  Patch width: W = c/(2f\textsubscript{r}) × √(2/(ε\textsubscript{r} +
  1)) = (3 × 10⁸)/(2 × 5.8 × 10⁹) × √(2/4.55) W = 0.02586 × 0.6630 =
  \textbf{17.1 mm}
\item
  Effective dielectric constant: ε\textsubscript{r,eff} =
  (ε\textsubscript{r} + 1)/2 + (ε\textsubscript{r} − 1)/2 × (1 +
  12h/W)⁻⁰·⁵ ε\textsubscript{r,eff} = 4.55/2 + 2.55/2 × (1 + 12 ×
  0.813/17.1)⁻⁰·⁵ ε\textsubscript{r,eff} = 2.275 + 1.275 × (1 +
  0.5705)⁻⁰·⁵ = 2.275 + 1.275 × (1.5705)⁻⁰·⁵ ε\textsubscript{r,eff} =
  2.275 + 1.275 × 0.7981 = 2.275 + 1.018 = \textbf{3.293}
\item
  Fringing extension: ΔL = 0.412h × (ε\textsubscript{r,eff} + 0.3)(W/h +
  0.264) / ((ε\textsubscript{r,eff} − 0.258)(W/h + 0.8)) W/h =
  17.1/0.813 = 21.03 ΔL = 0.412 × 0.813 × (3.593)(21.29) /
  ((3.035)(21.83)) ΔL = 0.3350 × 76.52 / 66.25 = \textbf{0.387 mm}
\item
  Patch length: L = c/(2f\textsubscript{r}√ε\textsubscript{r,eff}) − 2ΔL
  = (3 × 10⁸)/(2 × 5.8 × 10⁹ × √3.293) − 2 × 0.387 L = 0.02586 / 1.8147
  − 0.774 = 14.25 − 0.77 = \textbf{13.5 mm}
\item
  Q ≈ c√ε\textsubscript{r}/(4hf\textsubscript{r}) = 3 × 10⁸ × 1.884 / (4
  × 0.813 × 10⁻³ × 5.8 × 10⁹) Q = 5.652 × 10⁸ / 1.886 × 10⁷ = 30.0
  Bandwidth ≈ 1/Q = 1/30.0 = 0.033 = \textbf{3.3\%} (approximately 192
  MHz centered on 5.8 GHz)
\end{enumerate}

\begin{center}\rule{0.5\linewidth}{0.5pt}\end{center}

\section{Problem 16.4.2}\label{problem-16.4.2}

\textbf{Given:} A rectangular patch antenna on Taconic TLY-5 substrate
(ε\textsubscript{r} = 2.2, h = 1.575 mm) operates at 10 GHz. The patch
dimensions are W = 12.2 mm and L = 9.8 mm.

\textbf{Find:} (a) The effective dielectric constant, (b) the resonant
frequency, (c) the edge impedance R\textsubscript{edge} using
R\textsubscript{edge} ≈ 90 × ε\textsubscript{r}²/(ε\textsubscript{r} −
1) × (L/W)², (d) the inset feed distance for 50 Ω match using
R\textsubscript{in}(y₀) = R\textsubscript{edge} × cos²(πy₀/L), and (e)
the gain.

\textbf{Solution:}

\begin{enumerate}
\def\labelenumi{(\alph{enumi})}
\item
  ε\textsubscript{r,eff} = (2.2 + 1)/2 + (2.2 − 1)/2 × (1 + 12 ×
  1.575/12.2)⁻⁰·⁵ ε\textsubscript{r,eff} = 1.6 + 0.6 × (1 + 1.549)⁻⁰·⁵ =
  1.6 + 0.6 × (2.549)⁻⁰·⁵ ε\textsubscript{r,eff} = 1.6 + 0.6 × 0.6262 =
  1.6 + 0.376 = \textbf{1.976}
\item
  f\textsubscript{r} = c / (2(L + 2ΔL)√ε\textsubscript{r,eff}) ΔL =
  0.412 × 1.575 × (1.976 + 0.3)(12.2/1.575 + 0.264) / ((1.976 −
  0.258)(12.2/1.575 + 0.8)) ΔL = 0.649 × (2.276)(8.010) /
  ((1.718)(8.546)) = 0.649 × 18.23 / 14.69 = 0.805 mm f\textsubscript{r}
  = 3 × 10⁸ / (2 × (9.8 + 1.61) × 10⁻³ × 1.406) = 3 × 10⁸ / (2 × 11.41 ×
  10⁻³ × 1.406) f\textsubscript{r} = 3 × 10⁸ / 0.03209 = \textbf{9.35
  GHz}
\end{enumerate}

The resonant frequency is lower than 10 GHz because the given dimensions
produce a slightly larger patch than needed.

\begin{enumerate}
\def\labelenumi{(\alph{enumi})}
\setcounter{enumi}{2}
\item
  R\textsubscript{edge} ≈ 90 × (2.2)² / (2.2 − 1) × (9.8/12.2)²
  R\textsubscript{edge} = 90 × 4.84 / 1.2 × 0.6448 = 363.0 × 0.6448 =
  \textbf{234 Ω}
\item
  For 50 Ω match: cos²(πy₀/L) =
  R\textsubscript{in}/R\textsubscript{edge} = 50/234 = 0.2137 cos(πy₀/L)
  = 0.4623 πy₀/L = arccos(0.4623) = 1.089 rad y₀ = 1.089 × 9.8 / π =
  \textbf{3.40 mm} from the radiating edge
\item
  A single rectangular patch on a low-ε\textsubscript{r} substrate
  typically achieves \textbf{7--8 dBi}. With W/L = 1.24, the gain is
  approximately \textbf{7.5 dBi}.
\end{enumerate}

\begin{center}\rule{0.5\linewidth}{0.5pt}\end{center}

\section{Problem 16.4.3}\label{problem-16.4.3}

\textbf{Given:} A circular patch antenna is designed for GPS L1
reception at 1.575 GHz on a substrate with ε\textsubscript{r} = 4.4 and
h = 1.6 mm (FR-4).

\textbf{Find:} (a) The effective radius using a\textsubscript{eff} =
1.8412c/(2πf\textsubscript{r}√ε\textsubscript{r}), (b) the physical
radius after fringing correction a = a\textsubscript{eff} / √(1 +
2h/(πε\textsubscript{r}a\textsubscript{eff}) ×
(ln(πa\textsubscript{eff}/(2h)) + 1.7726)), (c) the gain, and (d) the
bandwidth.

\textbf{Solution:}

\begin{enumerate}
\def\labelenumi{(\alph{enumi})}
\item
  Effective radius: a\textsubscript{eff} = 1.8412 × c /
  (2πf\textsubscript{r}√ε\textsubscript{r}) = 1.8412 × 3 × 10⁸ / (2π ×
  1.575 × 10⁹ × √4.4) a\textsubscript{eff} = 5.524 × 10⁸ / (2π × 1.575 ×
  10⁹ × 2.098) a\textsubscript{eff} = 5.524 × 10⁸ / (2.076 × 10¹⁰) =
  \textbf{0.02662 m (26.62 mm)}
\item
  Physical radius: Correction factor = 1 +
  2h/(πε\textsubscript{r}a\textsubscript{eff}) ×
  (ln(πa\textsubscript{eff}/(2h)) + 1.7726) = 1 + 2 × 1.6/(π × 4.4 ×
  26.62) × (ln(π × 26.62/(2 × 1.6)) + 1.7726) = 1 + 3.2/(367.6) ×
  (ln(26.17) + 1.7726) = 1 + 0.008706 × (3.264 + 1.7726) = 1 + 0.008706
  × 5.037 = 1 + 0.04386 = 1.04386 a = a\textsubscript{eff} / √(1.04386)
  = 26.62 / 1.0217 = \textbf{26.05 mm}
\end{enumerate}

The physical radius is \textbf{26.1 mm}, giving a patch diameter of 52.2
mm.

\begin{enumerate}
\def\labelenumi{(\alph{enumi})}
\setcounter{enumi}{2}
\item
  A circular patch on FR-4 achieves approximately \textbf{5--6 dBi}. The
  higher ε\textsubscript{r} reduces radiation efficiency. Estimated gain
  ≈ \textbf{5.5 dBi}
\item
  Q ≈ c√ε\textsubscript{r}/(4hf\textsubscript{r}) = 3 × 10⁸ × 2.098 / (4
  × 1.6 × 10⁻³ × 1.575 × 10⁹) Q = 6.293 × 10⁸ / 1.008 × 10⁷ = 62.4
  Bandwidth ≈ 1/Q = 1/62.4 = 0.016 = \textbf{1.6\%} (approximately 25
  MHz)
\end{enumerate}

This narrow bandwidth is marginal for GPS L1, which requires about 2
MHz. The low substrate thickness is the limiting factor.

\begin{center}\rule{0.5\linewidth}{0.5pt}\end{center}

\section{Problem 16.4.4}\label{problem-16.4.4}

\textbf{Given:} A circularly polarized square patch antenna uses the
truncated-corner technique at 2.45 GHz on Rogers RT/duroid 5880
(ε\textsubscript{r} = 2.2, h = 1.575 mm). The square patch side length
is 41.2 mm.

\textbf{Find:} (a) The patch quality factor Q, (b) the truncation
dimension ΔC ≈ L/(2Q), (c) the area of each truncated corner triangle,
(d) the axial ratio bandwidth (typically 1--2\% for single-feed CP), and
(e) the gain.

\textbf{Solution:}

\begin{enumerate}
\def\labelenumi{(\alph{enumi})}
\item
  Q ≈ c√ε\textsubscript{r}/(4hf\textsubscript{r}) = 3 × 10⁸ × √2.2 / (4
  × 1.575 × 10⁻³ × 2.45 × 10⁹) Q = 3 × 10⁸ × 1.483 / (1.544 × 10⁷) =
  4.449 × 10⁸ / 1.544 × 10⁷ = \textbf{28.8}
\item
  ΔC ≈ L/(2Q) = 41.2 / (2 × 28.8) = 41.2 / 57.6 = \textbf{0.715 mm}
\item
  Each truncated corner removes a right isosceles triangle with legs =
  ΔC: A\textsubscript{triangle} = 0.5 × ΔC² = 0.5 × 0.715² = 0.5 × 0.511
  = \textbf{0.256 mm²}
\end{enumerate}

Two opposite corners are truncated, removing a total of 0.512 mm² from
the original 1,697 mm² patch area --- less than 0.03\% of the total
area.

\begin{enumerate}
\def\labelenumi{(\alph{enumi})}
\setcounter{enumi}{3}
\tightlist
\item
  The axial ratio bandwidth (AR \textless{} 3 dB) for a single-feed
  truncated-corner CP patch is approximately: BW\textsubscript{AR} ≈ 1/Q
  = 1/28.8 = \textbf{3.5\%} --- about 86 MHz at 2.45 GHz.
\end{enumerate}

In practice, the 3 dB AR bandwidth is typically 1--2\%, as the axial
ratio degrades faster than the impedance match away from center
frequency.

\begin{enumerate}
\def\labelenumi{(\alph{enumi})}
\setcounter{enumi}{4}
\tightlist
\item
  A CP patch has gain approximately 1 dB less than a linearly polarized
  patch of the same size due to cross-polarization losses near the band
  edges. Estimated gain ≈ \textbf{6.5 dBi} (RHCP or LHCP depending on
  which corners are truncated).
\end{enumerate}

\begin{center}\rule{0.5\linewidth}{0.5pt}\end{center}

\section{Problem 16.4.5}\label{problem-16.4.5}

\textbf{Given:} A microstrip patch antenna for 3.5 GHz (5G sub-6 GHz
band) is designed on a substrate with ε\textsubscript{r} = 3.0 and h =
1.524 mm. The antenna uses probe feeding at a distance y₀ from the patch
center.

\textbf{Find:} (a) The patch width W, (b) the effective dielectric
constant, (c) the patch length L, (d) the edge impedance, and (e) the
probe position y₀ for a 50 Ω match.

\textbf{Solution:}

\begin{enumerate}
\def\labelenumi{(\alph{enumi})}
\item
  W = c/(2f\textsubscript{r}) × √(2/(ε\textsubscript{r} + 1)) = (3 ×
  10⁸)/(2 × 3.5 × 10⁹) × √(2/4.0) W = 0.04286 × 0.7071 = \textbf{30.3
  mm}
\item
  ε\textsubscript{r,eff} = (3.0 + 1)/2 + (3.0 − 1)/2 × (1 + 12 ×
  1.524/30.3)⁻⁰·⁵ ε\textsubscript{r,eff} = 2.0 + 1.0 × (1 + 0.6038)⁻⁰·⁵
  = 2.0 + 1.0 × (1.6038)⁻⁰·⁵ ε\textsubscript{r,eff} = 2.0 + 1.0 × 0.7895
  = \textbf{2.790}
\item
  ΔL = 0.412 × 1.524 × (2.790 + 0.3)(30.3/1.524 + 0.264) / ((2.790 −
  0.258)(30.3/1.524 + 0.8)) ΔL = 0.628 × (3.090)(20.15) /
  ((2.532)(20.68)) ΔL = 0.628 × 62.26 / 52.36 = \textbf{0.747 mm}
\end{enumerate}

L = c/(2f\textsubscript{r}√ε\textsubscript{r,eff}) − 2ΔL = (3 × 10⁸)/(2
× 3.5 × 10⁹ × 1.670) − 2 × 0.747 L = 0.04286 / 1.670 − 1.494 = 25.67 −
1.49 = \textbf{24.2 mm}

\begin{enumerate}
\def\labelenumi{(\alph{enumi})}
\setcounter{enumi}{3}
\item
  R\textsubscript{edge} ≈ 90 × ε\textsubscript{r}²/(ε\textsubscript{r} −
  1) × (L/W)² = 90 × 9.0/2.0 × (24.2/30.3)² R\textsubscript{edge} = 405
  × 0.638 = \textbf{258 Ω}
\item
  For probe feeding, the impedance varies as R\textsubscript{in}(y₀) =
  R\textsubscript{edge} × cos²(πy₀/L). For 50 Ω: cos²(πy₀/L) = 50/258 =
  0.1938 cos(πy₀/L) = 0.4403 πy₀/L = arccos(0.4403) = 1.115 rad y₀ =
  1.115 × 24.2 / π = \textbf{8.59 mm} from the patch center (or 12.1 −
  8.59 = 3.51 mm from the radiating edge)
\end{enumerate}

\begin{center}\rule{0.5\linewidth}{0.5pt}\end{center}

\section{Problem 16.4.6}\label{problem-16.4.6}

\textbf{Given:} A 4 × 4 rectangular patch array is designed for 28 GHz
(5G mmWave). Each element has a gain of 7 dBi, and elements are spaced
at λ/2 in both dimensions. The corporate feed network has 1.5 dB of
loss.

\textbf{Find:} (a) The element spacing in mm, (b) the array dimensions,
(c) the array directivity, (d) the realized gain (including feed loss),
and (e) the HPBW in both planes.

\textbf{Solution:}

\begin{enumerate}
\def\labelenumi{(\alph{enumi})}
\item
  λ = c/f = 3 × 10⁸ / 28 × 10⁹ = 0.01071 m Element spacing = λ/2 =
  \textbf{5.36 mm}
\item
  Array dimensions = 4 × 5.36 = 21.4 mm per side = \textbf{21.4 mm ×
  21.4 mm}
\item
  G\textsubscript{element} (linear) = 10\textsuperscript{7/10} = 5.012
  D\textsubscript{array} = N × G\textsubscript{element} = 16 × 5.012 =
  80.2 D\textsubscript{array} (dBi) = 10 log₁₀(80.2) = \textbf{19.0 dBi}
\item
  G\textsubscript{realized} = D\textsubscript{array} − feed loss = 19.0
  − 1.5 = \textbf{17.5 dBi}
\item
  HPBW ≈ 0.886λ / (4 × λ/2) = 0.886/2 = 0.443 rad = \textbf{25.4°} in
  both planes
\end{enumerate}

\begin{center}\rule{0.5\linewidth}{0.5pt}\end{center}

\section{Problem 16.4.7}\label{problem-16.4.7}

\textbf{Given:} A stacked patch antenna uses two rectangular patches
separated by a foam spacer (ε\textsubscript{r} ≈ 1.05). The driven patch
is on a substrate with ε\textsubscript{r} = 2.2 and h₁ = 1.575 mm. The
parasitic patch is separated by h₂ = 3.0 mm of foam. The center
frequency is 5.0 GHz.

\textbf{Find:} (a) The driven patch dimensions (W × L), (b) the
parasitic patch dimensions (typically 5--10\% larger to create a second
resonance), (c) the total antenna height, (d) the expected bandwidth
improvement over a single patch, and (e) the gain.

\textbf{Solution:}

\begin{enumerate}
\def\labelenumi{(\alph{enumi})}
\tightlist
\item
  Driven patch: W = c/(2f\textsubscript{r}) × √(2/(ε\textsubscript{r} +
  1)) = (3 × 10⁸)/(2 × 5 × 10⁹) × √(2/3.2) = 0.03 × 0.7906 =
  \textbf{23.7 mm}
\end{enumerate}

ε\textsubscript{r,eff} ≈ (2.2 + 1)/2 + (2.2 − 1)/2 × (1 + 12 ×
1.575/23.7)⁻⁰·⁵ ε\textsubscript{r,eff} = 1.6 + 0.6 × (1.797)⁻⁰·⁵ = 1.6 +
0.6 × 0.7458 = 2.047

L = c/(2f\textsubscript{r}√ε\textsubscript{r,eff}) − 2ΔL ≈ 0.03/1.431 −
2 × 0.68 mm ≈ 20.96 − 1.36 = \textbf{19.6 mm}

Driven patch: \textbf{23.7 mm × 19.6 mm}

\begin{enumerate}
\def\labelenumi{(\alph{enumi})}
\setcounter{enumi}{1}
\tightlist
\item
  Parasitic patch (7\% larger): W\textsubscript{p} = 1.07 × 23.7 =
  \textbf{25.4 mm} L\textsubscript{p} = 1.07 × 19.6 = \textbf{21.0 mm}
\end{enumerate}

The larger parasitic patch resonates at a slightly lower frequency, and
the coupling between patches creates a dual-resonance response.

\begin{enumerate}
\def\labelenumi{(\alph{enumi})}
\setcounter{enumi}{2}
\item
  Total height = h₁ + h₂ = 1.575 + 3.0 = \textbf{4.575 mm}
  (approximately 0.076λ at 5 GHz)
\item
  A single patch on the 1.575 mm substrate would have BW ≈ 3--4\%. The
  stacked configuration expands bandwidth to approximately
  \textbf{15--20\%} (750 MHz to 1 GHz centered on 5 GHz) by creating two
  coupled resonances.
\item
  The stacked patch achieves approximately \textbf{8--9 dBi} ---
  slightly higher than a single patch due to the increased effective
  aperture from the parasitic element.
\end{enumerate}

\begin{center}\rule{0.5\linewidth}{0.5pt}\end{center}

\section{Problem 16.4.8}\label{problem-16.4.8}

\textbf{Given:} A microstrip patch antenna at 900 MHz is designed on a
thick substrate (ε\textsubscript{r} = 2.55, h = 6.35 mm) to maximize
bandwidth for IoT applications.

\textbf{Find:} (a) The patch width, (b) the patch length, (c) the Q
factor, (d) the impedance bandwidth, and (e) the tradeoff compared to a
thinner substrate.

\textbf{Solution:}

\begin{enumerate}
\def\labelenumi{(\alph{enumi})}
\item
  W = c/(2f\textsubscript{r}) × √(2/(ε\textsubscript{r} + 1)) = (3 ×
  10⁸)/(2 × 900 × 10⁶) × √(2/3.55) W = 0.1667 × 0.7505 = \textbf{125.1
  mm}
\item
  ε\textsubscript{r,eff} = (2.55 + 1)/2 + (2.55 − 1)/2 × (1 + 12 ×
  6.35/125.1)⁻⁰·⁵ ε\textsubscript{r,eff} = 1.775 + 0.775 × (1 +
  0.609)⁻⁰·⁵ = 1.775 + 0.775 × 0.7886 = 1.775 + 0.611 = 2.386
\end{enumerate}

ΔL = 0.412 × 6.35 × (2.686)(20.0) / ((2.128)(20.5)) = 2.616 × 53.72 /
43.62 = \textbf{3.22 mm}

L = c/(2f\textsubscript{r}√ε\textsubscript{r,eff}) − 2ΔL = 0.1667 /
1.545 − 6.44 = 107.9 − 6.4 = \textbf{101.5 mm}

\begin{enumerate}
\def\labelenumi{(\alph{enumi})}
\setcounter{enumi}{2}
\item
  Q ≈ c√ε\textsubscript{r}/(4hf\textsubscript{r}) = 3 × 10⁸ × 1.597 / (4
  × 6.35 × 10⁻³ × 900 × 10⁶) Q = 4.79 × 10⁸ / 2.286 × 10⁷ =
  \textbf{20.9}
\item
  BW ≈ 1/Q = 1/20.9 = 0.048 = \textbf{4.8\%} (approximately 43 MHz at
  900 MHz)
\item
  Compared to h = 1.6 mm (Q ≈ 83, BW ≈ 1.2\%): the thick substrate
  provides \textbf{4× the bandwidth} but increases surface wave
  excitation and antenna size. The 125 mm × 102 mm patch is large for a
  900 MHz IoT device. A thinner substrate with external matching or a
  stacked configuration may be more practical.
\end{enumerate}

\begin{center}\rule{0.5\linewidth}{0.5pt}\end{center}

\section{Problem 16.4.9}\label{problem-16.4.9}

\textbf{Given:} A series-fed 8-element linear patch array at 24 GHz uses
microstrip patches on Rogers RO3003 (ε\textsubscript{r} = 3.0, h = 0.508
mm). Elements are spaced at one guided wavelength λ\textsubscript{g} for
in-phase excitation.

\textbf{Find:} (a) The guided wavelength λ\textsubscript{g} =
λ₀/√ε\textsubscript{r,eff}, (b) the element spacing, (c) the array
length, (d) the estimated array gain, and (e) the HPBW along the array
axis.

\textbf{Solution:}

\begin{enumerate}
\def\labelenumi{(\alph{enumi})}
\item
  λ₀ = c/f = 3 × 10⁸ / 24 × 10⁹ = 0.0125 m Estimate
  ε\textsubscript{r,eff} ≈ (3.0 + 1)/2 = 2.0 (for a wide patch, h/W
  \textless\textless{} 1) λ\textsubscript{g} = λ₀ /
  √ε\textsubscript{r,eff} = 0.0125 / √2.0 = 0.0125 / 1.414 =
  \textbf{8.84 mm}
\item
  Element spacing = λ\textsubscript{g} = \textbf{8.84 mm}
\item
  Array length = (N − 1) × d = 7 × 8.84 = \textbf{61.9 mm}
\item
  Single patch gain ≈ 7 dBi. Array factor directivity for N = 8 at
  λ\textsubscript{g} spacing: D\textsubscript{AF} ≈ N = 8 (since spacing
  ≈ λ₀/√ε\textsubscript{r,eff} \textgreater{} λ₀/2, some grating lobe
  energy is lost) In practice: D\textsubscript{AF} ≈ 7 (accounting for
  imperfect contributions) G\textsubscript{array} ≈ 7 + 10 log₁₀(7) = 7
  + 8.5 = \textbf{15.5 dBi} (before feed losses)
\item
  HPBW along array axis ≈ 0.886λ₀ / (N × d × cos θ₀) Since d =
  λ\textsubscript{g} ≈ 0.707λ₀ and broadside operation: HPBW ≈ 0.886 ×
  0.0125 / (8 × 8.84 × 10⁻³) = 0.01108 / 0.0707 = 0.157 rad =
  \textbf{8.97°}
\end{enumerate}

\begin{center}\rule{0.5\linewidth}{0.5pt}\end{center}

\section{Problem 16.4.10}\label{problem-16.4.10}

\textbf{Given:} A dual-feed circularly polarized patch antenna at 5.8
GHz uses a square patch with two orthogonal feeds connected through a
90° hybrid coupler. The substrate is ε\textsubscript{r} = 2.2, h = 1.575
mm. The square patch side length is 18.5 mm.

\textbf{Find:} (a) The theoretical resonant frequency of the patch, (b)
the return loss at each port if S₁₁ = −18 dB, (c) the isolation between
ports (typically 20--25 dB for a 90° hybrid), (d) the axial ratio
bandwidth compared to a single-feed truncated-corner design, and (e) the
gain.

\textbf{Solution:}

\begin{enumerate}
\def\labelenumi{(\alph{enumi})}
\tightlist
\item
  ε\textsubscript{r,eff} ≈ (2.2 + 1)/2 + (2.2 − 1)/2 × (1 + 12 ×
  1.575/18.5)⁻⁰·⁵ ε\textsubscript{r,eff} = 1.6 + 0.6 × (2.021)⁻⁰·⁵ = 1.6
  + 0.6 × 0.7032 = 1.6 + 0.422 = 2.022
\end{enumerate}

ΔL ≈ 0.7 mm (estimated for this geometry) f\textsubscript{r} = c / (2(L
+ 2ΔL)√ε\textsubscript{r,eff}) = 3 × 10⁸ / (2 × (18.5 + 1.4) × 10⁻³ ×
1.422) f\textsubscript{r} = 3 × 10⁸ / (2 × 19.9 × 10⁻³ × 1.422) = 3 ×
10⁸ / 0.05660 = \textbf{5.30 GHz}

The patch is slightly too large for 5.8 GHz; trimming to approximately
17.0 mm would center the resonance at 5.8 GHz.

\begin{enumerate}
\def\labelenumi{(\alph{enumi})}
\setcounter{enumi}{1}
\item
  Return loss at each port = \textbf{18 dB} (\textbar Γ\textbar{} =
  10⁻⁰·⁹ = 0.126) Reflected power = 0.126² = 1.6\% at each port.
\item
  A well-designed 90° hybrid coupler provides \textbf{20--25 dB} of
  isolation between the two input ports. Reflected power from the
  antenna at each feed is routed to the isolation port (terminated in 50
  Ω), improving the system-level return loss beyond the individual port
  values.
\item
  The dual-feed technique with a 90° hybrid provides an axial ratio
  bandwidth of \textbf{10--15\%}, compared to 1--2\% for a single-feed
  truncated-corner design. This is because the hybrid maintains the 90°
  phase relationship over a broader bandwidth than the
  perturbation-based single-feed approach.
\item
  Gain ≈ \textbf{7.0 dBi} (RHCP or LHCP, depending on which port is
  excited). The dual-feed approach provides slightly better CP purity
  than the truncated-corner method, especially at band edges.
\end{enumerate}

\chapter{Chapter 16 --- Section 16.5: Antenna
Arrays}\label{chapter-16-section-16.5-antenna-arrays}

Practice problems covering array theory and pattern multiplication
(uniform linear arrays, broadside and end-fire arrays, grating lobes),
beamforming and smart antennas (analog, digital, hybrid, massive MIMO),
and phased array scanning and grating lobes.

\begin{center}\rule{0.5\linewidth}{0.5pt}\end{center}

\section{Problem 16.5.1}\label{problem-16.5.1}

\textbf{Given:} A uniform linear array (ULA) of 8 isotropic elements
operates at 2.4 GHz with half-wavelength spacing. The array is
configured for broadside radiation (β = 0).

\textbf{Find:} (a) The element spacing in mm, (b) the array factor
directivity, (c) the HPBW, (d) the first-null beamwidth (FNBW), and (e)
the first sidelobe level below the main beam.

\textbf{Solution:}

\begin{enumerate}
\def\labelenumi{(\alph{enumi})}
\item
  λ = c/f = 3 × 10⁸ / 2.4 × 10⁹ = 0.125 m d = λ/2 = 0.125/2 =
  \textbf{62.5 mm}
\item
  D\textsubscript{AF} = 2Nd/λ = 2 × 8 × 0.0625 / 0.125 = 1.0 / 0.125 =
  \textbf{8.0 (9.0 dB)}
\item
  HPBW ≈ 0.886λ/(Nd) = 0.886 × 0.125 / (8 × 0.0625) = 0.11075 / 0.5 =
  0.2215 rad = \textbf{12.7°}
\item
  FNBW ≈ 2λ/(Nd) = 2 × 0.125 / 0.5 = 0.5 rad = \textbf{28.6°}
\item
  For a uniform linear array, the first sidelobe level is approximately
  \textbf{−13.3 dB} below the main beam peak. This is a fundamental
  property of the sinc-like array factor pattern with uniform amplitude
  weighting.
\end{enumerate}

\begin{center}\rule{0.5\linewidth}{0.5pt}\end{center}

\section{Problem 16.5.2}\label{problem-16.5.2}

\textbf{Given:} A 16-element ULA at 5.8 GHz with d = λ/2 spacing uses
half-wave dipole elements oriented perpendicular to the array axis. The
array radiates broadside.

\textbf{Find:} (a) The array factor directivity, (b) the element gain,
(c) the total array directivity using pattern multiplication, (d) the
total array length, and (e) the HPBW of the total pattern.

\textbf{Solution:}

\begin{enumerate}
\def\labelenumi{(\alph{enumi})}
\item
  D\textsubscript{AF} = 2Nd/λ = 2 × 16 × 0.5 = \textbf{16.0 (12.0 dB)}
\item
  Half-wave dipole gain = \textbf{2.15 dBi} (1.64 linear)
\item
  D\textsubscript{total} = D\textsubscript{AF} ×
  D\textsubscript{element} = 16.0 × 1.64 = 26.24 D\textsubscript{total}
  (dBi) = 12.0 + 2.15 = \textbf{14.2 dBi}
\item
  λ = c/f = 3 × 10⁸ / 5.8 × 10⁹ = 0.05172 m Array length = (N − 1) × d =
  15 × λ/2 = 15 × 0.02586 = \textbf{0.388 m (38.8 cm)}
\item
  HPBW ≈ 0.886λ/(Nd) = 0.886 × 0.05172 / (16 × 0.02586) HPBW = 0.04582 /
  0.4138 = 0.1107 rad = \textbf{6.34°}
\end{enumerate}

The HPBW in the plane perpendicular to the array axis is determined by
the element pattern (dipole): approximately \textbf{78°} in that plane.

\begin{center}\rule{0.5\linewidth}{0.5pt}\end{center}

\section{Problem 16.5.3}\label{problem-16.5.3}

\textbf{Given:} An end-fire array of 6 elements at 1 GHz uses d = λ/4
spacing and a progressive phase shift β = −kd = −π/2 radians.

\textbf{Find:} (a) The element spacing, (b) the direction of the main
beam, (c) the end-fire directivity, (d) the HPBW, and (e) whether
grating lobes are present.

\textbf{Solution:}

\begin{enumerate}
\def\labelenumi{(\alph{enumi})}
\item
  λ = c/f = 3 × 10⁸ / 1 × 10⁹ = 0.3 m d = λ/4 = \textbf{0.075 m (75 mm)}
\item
  Main beam direction: ψ = kd cos θ + β = 0 at the main beam. kd = 2π/λ
  × λ/4 = π/2 cos θ₀ = −β/(kd) = −(−π/2)/(π/2) = 1 → θ₀ = \textbf{0°}
  (along the array axis, end-fire direction)
\item
  End-fire directivity for ordinary end-fire: D = 4Nd/λ = 4 × 6 × 0.075
  / 0.3 = 1.8 / 0.3 = \textbf{6.0 (7.8 dB)}
\end{enumerate}

For the Hansen-Woodyard condition (β = −kd − π/N), the directivity
increases to approximately 1.789 × 4Nd/λ = 1.789 × 6.0 = 10.7 (10.3 dB),
but this requires β = −π/2 − π/6 = −2.094 rad.

\begin{enumerate}
\def\labelenumi{(\alph{enumi})}
\setcounter{enumi}{3}
\tightlist
\item
  HPBW for ordinary end-fire ≈ 2 × √(0.886λ/(Nd)) = 2 × √(0.886 × 0.3 /
  (6 × 0.075)) HPBW = 2 × √(0.2658 / 0.45) = 2 × √0.5907 = 2 × 0.7686
  rad = 1.537 rad = \textbf{88.1°}
\end{enumerate}

End-fire arrays have much wider beamwidths than broadside arrays of the
same length.

\begin{enumerate}
\def\labelenumi{(\alph{enumi})}
\setcounter{enumi}{4}
\tightlist
\item
  For grating lobes: sin θ\textsubscript{GL} = sin θ₀ + nλ/d.~With θ₀ =
  0° (cos θ₀ = 1): For n = −1: the grating lobe appears when cos
  θ\textsubscript{GL} = cos 0° − λ/d = 1 − 4 = −3. Since \textbar cos
  θ\textsubscript{GL}\textbar{} \textgreater{} 1, \textbf{no grating
  lobes} exist. At d = λ/4, the spacing is far too small for grating
  lobes.
\end{enumerate}

\begin{center}\rule{0.5\linewidth}{0.5pt}\end{center}

\section{Problem 16.5.4}\label{problem-16.5.4}

\textbf{Given:} A planar array of 8 × 8 = 64 patch antenna elements
operates at 28 GHz with d = 0.5λ spacing in both dimensions. Each
element has a gain of 6 dBi. The array efficiency is 85\%.

\textbf{Find:} (a) The physical array dimensions, (b) the broadside
gain, (c) the HPBW in both planes, (d) the total number of elements, and
(e) the gain if the array is expanded to 16 × 16.

\textbf{Solution:}

\begin{enumerate}
\def\labelenumi{(\alph{enumi})}
\item
  λ = c/f = 3 × 10⁸ / 28 × 10⁹ = 0.01071 m d = 0.5 × 0.01071 = 0.005357
  m Array size = 8 × 5.357 mm = 42.86 mm per side ≈ \textbf{42.9 mm ×
  42.9 mm}
\item
  G\textsubscript{element} = 10\textsuperscript{6/10} = 3.981 G = η × N
  × G\textsubscript{element} = 0.85 × 64 × 3.981 = 216.6 G (dBi) = 10
  log₁₀(216.6) = \textbf{23.4 dBi}
\item
  HPBW ≈ 0.886λ / (8 × d) = 0.886 × 0.01071 / (8 × 0.005357) = 9.489 ×
  10⁻³ / 0.04286 HPBW = 0.2214 rad = \textbf{12.7°} in both planes
  (symmetric square array)
\item
  Total elements = 8 × 8 = \textbf{64 elements}
\item
  For 16 × 16 = 256 elements: G = 0.85 × 256 × 3.981 = 866.3 G (dBi) =
  10 log₁₀(866.3) = \textbf{29.4 dBi}
\end{enumerate}

The gain increases by 10 log₁₀(256/64) = 10 log₁₀(4) = 6.0 dB, as
expected from quadrupling the number of elements.

\begin{center}\rule{0.5\linewidth}{0.5pt}\end{center}

\section{Problem 16.5.5}\label{problem-16.5.5}

\textbf{Given:} A phased array with 20 elements at 6 GHz is steered to
θ₀ = 30° from broadside. Element spacing is d = 0.6λ.

\textbf{Find:} (a) The progressive phase shift β, (b) the broadside
HPBW, (c) the scanned HPBW at 30°, (d) the gain reduction due to
scanning, and (e) the maximum scan angle before a grating lobe appears.

\textbf{Solution:}

\begin{enumerate}
\def\labelenumi{(\alph{enumi})}
\item
  λ = c/f = 3 × 10⁸ / 6 × 10⁹ = 0.05 m; d = 0.6 × 0.05 = 0.03 m β = −kd
  sin θ₀ = −(2π/0.05) × 0.03 × sin 30° = −125.66 × 0.03 × 0.5 β =
  \textbf{−1.885 rad (−108.0°)}
\item
  Broadside HPBW ≈ 0.886λ/(Nd) = 0.886 × 0.05 / (20 × 0.03) = 0.0443 /
  0.6 = 0.0738 rad = \textbf{4.23°}
\item
  Scanned HPBW = broadside HPBW / cos θ₀ = 4.23° / cos 30° = 4.23° /
  0.866 = \textbf{4.88°}
\item
  Gain reduction = 10 log₁₀(cos 30°) = 10 log₁₀(0.866) = \textbf{−0.625
  dB}
\item
  Grating lobe condition: d \textless{} λ/(1 + \textbar sin
  θ\textsubscript{max}\textbar) 0.6λ \textless{} λ/(1 + sin
  θ\textsubscript{max}) 1 + sin θ\textsubscript{max} \textless{} 1/0.6 =
  1.667 sin θ\textsubscript{max} \textless{} 0.667 θ\textsubscript{max}
  = arcsin(0.667) = \textbf{41.8°}
\end{enumerate}

Beyond 41.8°, a grating lobe enters visible space. The 0.6λ spacing
limits the useful scan volume compared to the 0.5λ standard.

\begin{center}\rule{0.5\linewidth}{0.5pt}\end{center}

\section{Problem 16.5.6}\label{problem-16.5.6}

\textbf{Given:} A 5G massive MIMO base station uses a 128-element array
(16 × 8) at 3.5 GHz with λ/2 spacing. Each element gain is 5 dBi. Array
efficiency is 88\%.

\textbf{Find:} (a) The panel dimensions, (b) the broadside gain, (c) the
HPBW in azimuth (8-element dimension) and elevation (16-element
dimension), (d) the array gain in dB relative to a single element, and
(e) the number of simultaneous user beams with full digital beamforming.

\textbf{Solution:}

\begin{enumerate}
\def\labelenumi{(\alph{enumi})}
\item
  λ = c/f = 3 × 10⁸ / 3.5 × 10⁹ = 0.08571 m; d = λ/2 = 0.04286 m Azimuth
  (8 elements): 8 × 42.86 mm = \textbf{342.9 mm} Elevation (16
  elements): 16 × 42.86 mm = \textbf{685.7 mm} Panel: \textbf{34.3 cm ×
  68.6 cm}
\item
  G\textsubscript{element} = 10\textsuperscript{5/10} = 3.162 G = η × N
  × G\textsubscript{element} = 0.88 × 128 × 3.162 = 356.1 G (dBi) = 10
  log₁₀(356.1) = \textbf{25.5 dBi}
\item
  Azimuth HPBW ≈ 0.886λ/(8 × λ/2) = 0.886/4 = 0.2215 rad =
  \textbf{12.7°} Elevation HPBW ≈ 0.886λ/(16 × λ/2) = 0.886/8 = 0.1108
  rad = \textbf{6.35°}
\item
  Array gain over single element = 10 log₁₀(η × N) = 10 log₁₀(0.88 ×
  128) = 10 log₁₀(112.6) = \textbf{20.5 dB}
\item
  With full digital beamforming (one RF chain per element): up to
  \textbf{128 simultaneous beams} theoretically. In practice, 16--32
  simultaneous users are typical, limited by inter-user interference and
  spatial resolution.
\end{enumerate}

\begin{center}\rule{0.5\linewidth}{0.5pt}\end{center}

\section{Problem 16.5.7}\label{problem-16.5.7}

\textbf{Given:} A phased array radar at 9.5 GHz has 32 elements with d =
0.55λ spacing. The phase shifters have 4-bit resolution (16 phase
states).

\textbf{Find:} (a) The phase quantization step, (b) the maximum phase
error, (c) the average sidelobe level due to phase quantization, (d) the
RMS phase error, and (e) the gain reduction due to quantization.

\textbf{Solution:}

\begin{enumerate}
\def\labelenumi{(\alph{enumi})}
\item
  Phase step = 360° / 2⁴ = 360° / 16 = \textbf{22.5°}
\item
  Maximum phase error = ±(phase step)/2 = ±22.5°/2 = \textbf{±11.25°}
\item
  Average sidelobe level due to quantization ≈ −6B dB = −6 × 4 =
  \textbf{−24 dB} below the main beam.
\end{enumerate}

This is the average level; individual quantization lobes may be higher
depending on the scan angle.

\begin{enumerate}
\def\labelenumi{(\alph{enumi})}
\setcounter{enumi}{3}
\item
  RMS phase error for uniform quantization: σ\textsubscript{φ} = (phase
  step) / √12 = 22.5° / √12 = 22.5° / 3.464 = \textbf{6.50°} (0.1134
  rad)
\item
  Gain reduction = −10 log₁₀(1 − σ\textsubscript{φ}²) where
  σ\textsubscript{φ} is in radians: ΔG = −10 log₁₀(1 − 0.1134²) = −10
  log₁₀(1 − 0.01286) = −10 log₁₀(0.9871) ΔG = −10 × (−0.00564) =
  \textbf{0.056 dB} --- negligible.
\end{enumerate}

The 4-bit phase shifter provides adequate performance for most radar and
communication applications.

\begin{center}\rule{0.5\linewidth}{0.5pt}\end{center}

\section{Problem 16.5.8}\label{problem-16.5.8}

\textbf{Given:} An 8-element ULA at 10 GHz with d = 0.5λ uses
Dolph-Chebyshev amplitude weighting to achieve −25 dB sidelobes
(compared to −13.3 dB for uniform weighting).

\textbf{Find:} (a) The sidelobe improvement over uniform weighting, (b)
the approximate HPBW broadening factor (typically 1.1--1.3× for −25 dB
SLL), (c) the HPBW with uniform weighting, (d) the HPBW with Chebyshev
weighting, and (e) the directivity reduction compared to uniform
weighting.

\textbf{Solution:}

\begin{enumerate}
\def\labelenumi{(\alph{enumi})}
\item
  Sidelobe improvement = −25 − (−13.3) = \textbf{11.7 dB} improvement in
  peak sidelobe level
\item
  For −25 dB Chebyshev sidelobes, the beamwidth broadening factor is
  approximately \textbf{1.15} relative to uniform weighting.
\item
  λ = 0.03 m; d = 0.015 m Uniform HPBW ≈ 0.886λ/(Nd) = 0.886 × 0.03 / (8
  × 0.015) = 0.02658 / 0.12 = 0.2215 rad = \textbf{12.7°}
\item
  Chebyshev HPBW ≈ 1.15 × 12.7° = \textbf{14.6°}
\item
  Directivity with uniform weighting: D\textsubscript{uniform} = 2Nd/λ =
  2 × 8 × 0.5 = 8.0 (9.0 dB). Chebyshev directivity is reduced by the
  taper efficiency, approximately: η\textsubscript{taper} ≈ 1/1.15² ≈
  0.756 (approximate, based on beamwidth broadening squared)
  D\textsubscript{Cheb} ≈ 0.756 × 8.0 = 6.05 Directivity reduction ≈ 9.0
  − 10 log₁₀(6.05) = 9.0 − 7.8 = \textbf{1.2 dB}
\end{enumerate}

This 1.2 dB sacrifice in directivity provides an 11.7 dB improvement in
sidelobe suppression --- a worthwhile tradeoff for radar and
communication systems where interference rejection is critical.

\begin{center}\rule{0.5\linewidth}{0.5pt}\end{center}

\section{Problem 16.5.9}\label{problem-16.5.9}

\textbf{Given:} A phased array steers its beam to θ₀ = 50° from
broadside at 3 GHz. The array has 24 elements with d = 0.5λ. The
elements are patch antennas with a cos θ element pattern.

\textbf{Find:} (a) The progressive phase shift, (b) the element pattern
gain at 50° relative to broadside, (c) the array factor directivity, (d)
the scanned beamwidth, and (e) the total gain at the scanned angle.

\textbf{Solution:}

\begin{enumerate}
\def\labelenumi{(\alph{enumi})}
\item
  λ = c/f = 3 × 10⁸ / 3 × 10⁹ = 0.1 m; d = 0.05 m β = −kd sin θ₀ =
  −(2π/0.1) × 0.05 × sin 50° = −62.83 × 0.05 × 0.766 β = \textbf{−2.407
  rad (−137.9°)}
\item
  Element pattern at 50°: E(θ) = cos θ Power pattern:
  \textbar E(50°)\textbar² = cos²(50°) = 0.413 Element gain reduction =
  10 log₁₀(0.413) = \textbf{−3.84 dB} relative to broadside
\item
  Array factor directivity = 2Nd/λ = 2 × 24 × 0.5 = \textbf{24.0 (13.8
  dB)}
\item
  Broadside HPBW = 0.886λ/(Nd) = 0.886 × 0.1 / (24 × 0.05) = 0.0738 rad
  = 4.23° Scanned HPBW = 4.23° / cos 50° = 4.23° / 0.643 =
  \textbf{6.58°}
\item
  Broadside array gain = D\textsubscript{AF} (dB) +
  G\textsubscript{element,broadside} Assuming G\textsubscript{element} =
  6 dBi at broadside: G\textsubscript{scanned} = 13.8 + 6 − 3.84 − 10
  log₁₀(1/cos 50°) = 13.8 + 6 − 3.84 − 1.92 = \textbf{14.0 dBi}
\end{enumerate}

The combined effect of element pattern roll-off (−3.84 dB) and projected
aperture reduction (−1.92 dB) yields a total scan loss of 5.76 dB at
50°.

\begin{center}\rule{0.5\linewidth}{0.5pt}\end{center}

\section{Problem 16.5.10}\label{problem-16.5.10}

\textbf{Given:} A 5G mmWave system at 39 GHz uses hybrid beamforming
with 4 RF chains and a 64-element array (8 × 8) with d = 0.5λ. Each RF
chain drives a subarray of 16 elements (4 × 4). Each element has a gain
of 5 dBi.

\textbf{Find:} (a) The element spacing, (b) the subarray gain (analog
beamforming within each subarray), (c) the total array gain with
coherent combining of all subarrays, (d) the number of simultaneous
beams, and (e) the subarray HPBW.

\textbf{Solution:}

\begin{enumerate}
\def\labelenumi{(\alph{enumi})}
\item
  λ = c/f = 3 × 10⁸ / 39 × 10⁹ = 7.692 mm d = λ/2 = \textbf{3.85 mm}
\item
  Subarray (4 × 4 = 16 elements): G\textsubscript{element} =
  10\textsuperscript{5/10} = 3.162 G\textsubscript{subarray} = 16 ×
  3.162 = 50.6 (assuming 100\% subarray efficiency)
  G\textsubscript{subarray} (dBi) = 10 log₁₀(50.6) = \textbf{17.0 dBi}
\item
  Total array gain = 4 subarrays × G\textsubscript{subarray} = 4 × 50.6
  = 202.4 (with coherent combining) G\textsubscript{total} (dBi) = 10
  log₁₀(202.4) = \textbf{23.1 dBi}
\end{enumerate}

Equivalently: G\textsubscript{total} = 64 × 3.162 = 202.4, confirming
the result.

\begin{enumerate}
\def\labelenumi{(\alph{enumi})}
\setcounter{enumi}{3}
\item
  With 4 RF chains and digital beamforming across subarrays: \textbf{4
  simultaneous beams} can be formed, each pointing in a different
  direction. Each beam uses the full array gain when all subarrays steer
  to the same direction, or subarray-level gain (17.0 dBi) when pointing
  independently.
\item
  Subarray HPBW: Each subarray has 4 elements per dimension at λ/2
  spacing: HPBW ≈ 0.886λ/(4 × λ/2) = 0.886/2 = 0.443 rad =
  \textbf{25.4°} in both planes
\end{enumerate}

The full 8 × 8 array has HPBW = 0.886/4 = 12.7° per plane.

\chapter{Chapter 16 --- Section 16.6: Impedance Matching and Practical
Design}\label{chapter-16-section-16.6-impedance-matching-and-practical-design}

Practice problems covering antenna impedance matching (VSWR, return
loss, mismatch loss, L-networks, quarter-wave transformers, baluns) and
antenna measurement and testing (VNA measurements, far-field distance,
gain measurement methods).

\begin{center}\rule{0.5\linewidth}{0.5pt}\end{center}

\section{Problem 16.6.1}\label{problem-16.6.1}

\textbf{Given:} A Yagi-Uda antenna has an input impedance of
Z\textsubscript{ant} = 28 + j12 Ω at 146 MHz and is connected directly
to 50 Ω coaxial cable.

\textbf{Find:} (a) The reflection coefficient, (b) the VSWR, (c) the
return loss, (d) the mismatch loss, and (e) the percentage of power
reflected.

\textbf{Solution:}

\begin{enumerate}
\def\labelenumi{(\alph{enumi})}
\item
  Γ = (Z\textsubscript{ant} − Z₀) / (Z\textsubscript{ant} + Z₀) = (28 +
  j12 − 50) / (28 + j12 + 50) = (−22 + j12) / (78 + j12)
  \textbar Γ\textbar{} = √(22² + 12²) / √(78² + 12²) = √(484 + 144) /
  √(6,084 + 144) = √628 / √6,228 = 25.06 / 78.92 \textbar Γ\textbar{} =
  \textbf{0.318}
\item
  VSWR = (1 + 0.318) / (1 − 0.318) = 1.318 / 0.682 = \textbf{1.93:1}
\item
  RL = −20 log₁₀(0.318) = −20 × (−0.498) = \textbf{9.95 dB}
\item
  ML = −10 log₁₀(1 − \textbar Γ\textbar²) = −10 log₁₀(1 − 0.101) = −10
  log₁₀(0.899) = \textbf{0.46 dB}
\item
  Reflected power = \textbar Γ\textbar² = 0.318² = 0.101 =
  \textbf{10.1\%}
\end{enumerate}

With VSWR just under 2:1, this match is borderline acceptable. A gamma
match or hairpin match would improve performance.

\begin{center}\rule{0.5\linewidth}{0.5pt}\end{center}

\section{Problem 16.6.2}\label{problem-16.6.2}

\textbf{Given:} A 50 Ω patch antenna has S₁₁ measurements at three
frequencies: −8 dB at 2.35 GHz, −22 dB at 2.45 GHz (center), and −10 dB
at 2.55 GHz.

\textbf{Find:} (a) The reflection coefficient at each frequency, (b) the
VSWR at each frequency, (c) the mismatch loss at each frequency, (d) the
impedance bandwidth (frequencies where VSWR \textless{} 2:1), and (e)
the fractional bandwidth.

\textbf{Solution:}

\begin{enumerate}
\def\labelenumi{(\alph{enumi})}
\item
  \textbar Γ\textbar{} = 10\textsuperscript{S₁₁/20}: At 2.35 GHz:
  \textbar Γ\textbar{} = 10⁻⁰·⁴ = \textbf{0.398} At 2.45 GHz:
  \textbar Γ\textbar{} = 10⁻¹·¹ = \textbf{0.0794} At 2.55 GHz:
  \textbar Γ\textbar{} = 10⁻⁰·⁵ = \textbf{0.316}
\item
  VSWR: At 2.35 GHz: VSWR = (1 + 0.398)/(1 − 0.398) = \textbf{2.32:1} At
  2.45 GHz: VSWR = (1 + 0.0794)/(1 − 0.0794) = \textbf{1.17:1} At 2.55
  GHz: VSWR = (1 + 0.316)/(1 − 0.316) = \textbf{1.92:1}
\item
  Mismatch loss ML = −10 log₁₀(1 − \textbar Γ\textbar²): At 2.35 GHz: ML
  = −10 log₁₀(1 − 0.158) = −10 log₁₀(0.842) = \textbf{0.75 dB} At 2.45
  GHz: ML = −10 log₁₀(1 − 0.0063) = −10 log₁₀(0.9937) = \textbf{0.027
  dB} At 2.55 GHz: ML = −10 log₁₀(1 − 0.100) = −10 log₁₀(0.900) =
  \textbf{0.46 dB}
\item
  VSWR \textless{} 2:1 corresponds to S₁₁ \textless{} −9.54 dB. The 2.35
  GHz point (−8 dB) is outside the band, and 2.55 GHz (−10 dB) is
  inside. By interpolation, the −10 dB bandwidth spans approximately
  \textbf{2.37 GHz to 2.55 GHz = 180 MHz}. The VSWR 2:1 bandwidth is
  slightly wider at approximately \textbf{2.36 GHz to 2.56 GHz = 200
  MHz}.
\item
  Fractional bandwidth = 200/2,450 = 0.082 = \textbf{8.2\%}
\end{enumerate}

\begin{center}\rule{0.5\linewidth}{0.5pt}\end{center}

\section{Problem 16.6.3}\label{problem-16.6.3}

\textbf{Given:} A loop antenna with Z\textsubscript{ant} = 120 Ω needs
to be matched to a 50 Ω coaxial cable at 435 MHz using an L-network
(series inductor, shunt capacitor).

\textbf{Find:} (a) The Q factor of the L-network, (b) the series
inductance, (c) the shunt capacitance, (d) the 3 dB bandwidth of the
matching network, and (e) a quarter-wave transformer alternative
impedance.

\textbf{Solution:}

\begin{enumerate}
\def\labelenumi{(\alph{enumi})}
\item
  Q = √(R\textsubscript{high}/R\textsubscript{low} − 1) = √(120/50 − 1)
  = √(2.4 − 1) = √1.4 = \textbf{1.183}
\item
  Series element (inductor in series with 50 Ω side): X\textsubscript{L}
  = Q × R\textsubscript{low} = 1.183 × 50 = 59.15 Ω L =
  X\textsubscript{L} / (2πf) = 59.15 / (2π × 435 × 10⁶) = 59.15 / (2.733
  × 10⁹) = \textbf{21.6 nH}
\item
  Shunt element (capacitor across 120 Ω side): X\textsubscript{C} =
  R\textsubscript{high} / Q = 120 / 1.183 = 101.4 Ω C = 1 /
  (2πfX\textsubscript{C}) = 1 / (2π × 435 × 10⁶ × 101.4) = 1 / (2.772 ×
  10¹¹) = \textbf{3.61 pF}
\item
  The 3 dB bandwidth of an L-network with Q = 1.183: BW = f / Q = 435 /
  1.183 = \textbf{368 MHz}
\end{enumerate}

The low Q means the matching network has very wide bandwidth --- wider
than the antenna bandwidth itself, so the match is not the
bandwidth-limiting factor.

\begin{enumerate}
\def\labelenumi{(\alph{enumi})}
\setcounter{enumi}{4}
\tightlist
\item
  Quarter-wave transformer: Z\textsubscript{match} = √(Z₀ ×
  Z\textsubscript{ant}) = √(50 × 120) = √6,000 = \textbf{77.5 Ω} λ/4 at
  435 MHz = 0.6897/4 = 0.172 m = 172 mm. A 77.5 Ω section could be
  fabricated as a coaxial line or microstrip.
\end{enumerate}

\begin{center}\rule{0.5\linewidth}{0.5pt}\end{center}

\section{Problem 16.6.4}\label{problem-16.6.4}

\textbf{Given:} A dipole antenna (balanced, Z\textsubscript{ant} = 73 Ω)
is connected to 50 Ω coaxial cable (unbalanced) at 150 MHz. A 1:1
current balun is used, and the dipole is shortened to eliminate
reactance.

\textbf{Find:} (a) The VSWR without any matching (dipole to 50 Ω), (b)
the return loss, (c) why a balun is needed even though the impedance is
close to 50 Ω, (d) a 4:1 balun alternative using a folded dipole, and
(e) the resulting VSWR with a folded dipole and 4:1 balun.

\textbf{Solution:}

\begin{enumerate}
\def\labelenumi{(\alph{enumi})}
\item
  Γ = (73 − 50) / (73 + 50) = 23 / 123 = 0.187 VSWR = (1 + 0.187) / (1 −
  0.187) = 1.187 / 0.813 = \textbf{1.46:1}
\item
  RL = −20 log₁₀(0.187) = \textbf{14.6 dB} --- acceptable without
  further matching
\item
  A balun is required because the dipole is a \textbf{balanced}
  structure (equal and opposite currents on each arm), while coax is
  \textbf{unbalanced} (current on center conductor, return on shield).
  Without a balun, common-mode current flows on the outside of the coax
  shield, causing:
\end{enumerate}

\begin{itemize}
\tightlist
\item
  Distortion of the radiation pattern
\item
  Radiation from the feed line
\item
  Increased susceptibility to interference
\item
  Unpredictable input impedance
\end{itemize}

\begin{enumerate}
\def\labelenumi{(\alph{enumi})}
\setcounter{enumi}{3}
\item
  A folded dipole has Z\textsubscript{ant} = 292 Ω. A 4:1 balun
  transforms this to: Z\textsubscript{transformed} = 292 / 4 =
  \textbf{73 Ω} --- same as a simple dipole but with wider bandwidth.
  Alternatively, a 6:1 balun gives 292/6 = 48.7 Ω ≈ 50 Ω.
\item
  With a folded dipole (292 Ω) and a 4:1 balun: Z\textsubscript{in} = 73
  Ω. VSWR on 50 Ω = \textbf{1.46:1} (same as without balun, but now with
  proper balance and wider bandwidth). With a 6:1 balun:
  Z\textsubscript{in} = 48.7 Ω → Γ = (50 − 48.7)/(50 + 48.7) = 1.3/98.7
  = 0.0132 → VSWR = \textbf{1.03:1}.
\end{enumerate}

\begin{center}\rule{0.5\linewidth}{0.5pt}\end{center}

\section{Problem 16.6.5}\label{problem-16.6.5}

\textbf{Given:} An antenna under test (AUT) is a 0.6 m dish antenna
operating at 18 GHz. Testing will be performed on an outdoor antenna
range.

\textbf{Find:} (a) The wavelength, (b) the far-field distance, (c) the
AUT gain estimate (assuming η = 0.60), (d) the HPBW, and (e) the
required angular positioning accuracy to stay within 0.1 dB of peak gain
(approximately θ\textsubscript{0.1dB} ≈ 0.3 × HPBW).

\textbf{Solution:}

\begin{enumerate}
\def\labelenumi{(\alph{enumi})}
\item
  λ = c/f = 3 × 10⁸ / 18 × 10⁹ = \textbf{0.01667 m (16.67 mm)}
\item
  d\textsubscript{ff} = 2D² / λ = 2 × 0.6² / 0.01667 = 2 × 0.36 /
  0.01667 = \textbf{43.2 m}
\item
  G = η(πD/λ)² = 0.60 × (π × 0.6 / 0.01667)² = 0.60 × (113.1)² = 0.60 ×
  12,792 = 7,675 G (dBi) = 10 log₁₀(7,675) = \textbf{38.9 dBi}
\item
  HPBW ≈ 70λ/D = 70 × 0.01667 / 0.6 = \textbf{1.94°}
\item
  θ\textsubscript{0.1dB} ≈ 0.3 × HPBW = 0.3 × 1.94° = \textbf{0.58°}
\end{enumerate}

The positioner must maintain pointing accuracy within ±0.58° to keep
measurements within 0.1 dB of the true peak. For a 43 m range, this is a
moderate requirement achievable with standard antenna positioners.

\begin{center}\rule{0.5\linewidth}{0.5pt}\end{center}

\section{Problem 16.6.6}\label{problem-16.6.6}

\textbf{Given:} A VNA measures the following S₁₁ data for a monopole
antenna at 915 MHz: S₁₁ = 0.35 ∠ −65° (magnitude and phase of the
reflection coefficient). The system impedance is Z₀ = 50 Ω.

\textbf{Find:} (a) The return loss, (b) the VSWR, (c) the antenna
impedance Z\textsubscript{ant} = Z₀ × (1 + Γ)/(1 − Γ), (d) the
resistance and reactance components, and (e) the mismatch loss.

\textbf{Solution:}

\begin{enumerate}
\def\labelenumi{(\alph{enumi})}
\item
  RL = −20 log₁₀(0.35) = −20 × (−0.456) = \textbf{9.12 dB}
\item
  VSWR = (1 + 0.35) / (1 − 0.35) = 1.35 / 0.65 = \textbf{2.08:1}
\item
  Γ = 0.35 ∠ −65° = 0.35(cos(−65°) + j sin(−65°)) = 0.35 × (0.4226 −
  j0.9063) Γ = 0.1479 − j0.3172
\end{enumerate}

Z\textsubscript{ant} = Z₀ × (1 + Γ) / (1 − Γ) = 50 × (1.1479 − j0.3172)
/ (0.8521 + j0.3172)

Multiply numerator and denominator by the conjugate of the denominator:
Denominator magnitude² = 0.8521² + 0.3172² = 0.7261 + 0.1006 = 0.8267
Numerator = (1.1479 − j0.3172)(0.8521 − j0.3172) = 1.1479 × 0.8521 −
1.1479 × j0.3172 − j0.3172 × 0.8521 + j² × 0.3172² = 0.9782 − j0.3641 −
j0.2703 − 0.1006 = 0.8776 − j0.6344

Z\textsubscript{ant} = 50 × (0.8776 − j0.6344) / 0.8267 = 50 × (1.0616 −
j0.7674) Z\textsubscript{ant} = \textbf{53.1 − j38.4 Ω}

\begin{enumerate}
\def\labelenumi{(\alph{enumi})}
\setcounter{enumi}{3}
\tightlist
\item
  R = \textbf{53.1 Ω}, X = \textbf{−38.4 Ω} (capacitive reactance)
\end{enumerate}

The monopole is slightly shorter than resonant --- the negative
reactance indicates a capacitive input, which could be corrected by
lengthening the element or adding a series inductor.

\begin{enumerate}
\def\labelenumi{(\alph{enumi})}
\setcounter{enumi}{4}
\tightlist
\item
  ML = −10 log₁₀(1 − 0.35²) = −10 log₁₀(1 − 0.1225) = −10 log₁₀(0.8775)
  ML = \textbf{0.57 dB}
\end{enumerate}

\begin{center}\rule{0.5\linewidth}{0.5pt}\end{center}

\section{Problem 16.6.7}\label{problem-16.6.7}

\textbf{Given:} Three unknown antennas (A, B, C) are used in a
three-antenna gain measurement at 10 GHz. The following power ratios are
measured at a fixed range of 10 m: P\textsubscript{AB} = −42.3 dB (A
transmit, B receive), P\textsubscript{AC} = −39.8 dB (A transmit, C
receive), P\textsubscript{BC} = −44.1 dB (B transmit, C receive). All
measurements use 0 dBm transmit power.

\textbf{Find:} The individual gains G\textsubscript{A},
G\textsubscript{B}, and G\textsubscript{C} in dBi.

\textbf{Solution:}

The Friis equation gives: P\textsubscript{r}/P\textsubscript{t} (dB) =
G\textsubscript{t} + G\textsubscript{r} − FSPL

First, calculate FSPL at 10 GHz, 10 m range: λ = 0.03 m FSPL = 20
log₁₀(4π × 10 / 0.03) = 20 log₁₀(4,189) = 20 × 3.622 = \textbf{72.4 dB}

Now set up equations (P\textsubscript{r} in dBm with P\textsubscript{t}
= 0 dBm, so P\textsubscript{r} = G\textsubscript{t} + G\textsubscript{r}
− FSPL): G\textsubscript{A} + G\textsubscript{B} = P\textsubscript{AB} +
FSPL = −42.3 + 72.4 = 30.1 dB \ldots{} (1) G\textsubscript{A} +
G\textsubscript{C} = P\textsubscript{AC} + FSPL = −39.8 + 72.4 = 32.6 dB
\ldots{} (2) G\textsubscript{B} + G\textsubscript{C} =
P\textsubscript{BC} + FSPL = −44.1 + 72.4 = 28.3 dB \ldots{} (3)

From (1) and (2): G\textsubscript{C} − G\textsubscript{B} = 32.6 − 30.1
= 2.5 dB \ldots{} (4) From (3): G\textsubscript{B} + G\textsubscript{C}
= 28.3 \ldots{} (3)

Adding (3) and (4): 2G\textsubscript{C} = 30.8 → G\textsubscript{C} =
\textbf{15.4 dBi} From (3): G\textsubscript{B} = 28.3 − 15.4 =
\textbf{12.9 dBi} From (1): G\textsubscript{A} = 30.1 − 12.9 =
\textbf{17.2 dBi}

Verification: G\textsubscript{A} + G\textsubscript{C} = 17.2 + 15.4 =
32.6 ✓

\begin{center}\rule{0.5\linewidth}{0.5pt}\end{center}

\section{Problem 16.6.8}\label{problem-16.6.8}

\textbf{Given:} A quarter-wave transformer matches a patch antenna
(Z\textsubscript{ant} = 200 Ω) to a 50 Ω microstrip line at 5 GHz on a
substrate with ε\textsubscript{r} = 2.2.

\textbf{Find:} (a) The transformer impedance, (b) the transformer
physical length, (c) the VSWR at center frequency, (d) the bandwidth
over which VSWR \textless{} 1.5:1, and (e) the microstrip line width for
the transformer section (using approximate formula: w/h ≈
8e\textsuperscript{A}/(e\textsuperscript{2A} − 2) where A =
Z/(60√((ε\textsubscript{r}+1)/2)) +
(ε\textsubscript{r}−1)/(ε\textsubscript{r}+1) × (0.23 +
0.11/ε\textsubscript{r}), for w/h \textless{} 2).

\textbf{Solution:}

\begin{enumerate}
\def\labelenumi{(\alph{enumi})}
\item
  Z\textsubscript{match} = √(Z₀ × Z\textsubscript{ant}) = √(50 × 200) =
  √10,000 = \textbf{100 Ω}
\item
  λ₀ = c/f = 3 × 10⁸ / 5 × 10⁹ = 0.06 m ε\textsubscript{r,eff} ≈ (2.2 +
  1)/2 = 1.6 (approximate for narrow line) λ\textsubscript{g} = λ₀ /
  √ε\textsubscript{r,eff} = 0.06 / 1.265 = 0.04743 m Physical length =
  λ\textsubscript{g}/4 = \textbf{11.86 mm}
\item
  At center frequency, the transformer provides a perfect match: VSWR =
  \textbf{1.0:1}
\item
  The quarter-wave transformer bandwidth for VSWR \textless{} 1.5 is
  approximately: BW ≈ 2f₀ × (2/π) × arccos(Γ\textsubscript{max} ×
  2√(Z\textsubscript{ant}Z₀) / \textbar Z\textsubscript{ant} −
  Z₀\textbar)
\end{enumerate}

For VSWR = 1.5: Γ\textsubscript{max} = (1.5 − 1)/(1.5 + 1) = 0.2
Fractional BW ≈ 2 − (4/π) × arccos(0.2 × 2 × 100 / 150) = 2 − (4/π) ×
arccos(0.267) = 2 − 1.273 × 1.300 = 2 − 1.655 = 0.345 BW ≈ 0.345 × 5 GHz
= \textbf{1.73 GHz} (34.5\% fractional bandwidth)

\begin{enumerate}
\def\labelenumi{(\alph{enumi})}
\setcounter{enumi}{4}
\tightlist
\item
  For Z = 100 Ω on ε\textsubscript{r} = 2.2: A = 100/60 × √((2.2+1)/2) +
  (2.2−1)/(2.2+1) × (0.23 + 0.11/2.2) A = 1.667 × √1.6 + 0.375 × (0.23 +
  0.05) = 1.667 × 1.265 + 0.375 × 0.28 A = 2.108 + 0.105 = 2.213 w/h = 8
  × e²·²¹³ / (e⁴·⁴²⁶ − 2) = 8 × 9.14 / (83.6 − 2) = 73.1 / 81.6 = 0.896
\end{enumerate}

For h = 1.575 mm: w = 0.896 × 1.575 = \textbf{1.41 mm}

\begin{center}\rule{0.5\linewidth}{0.5pt}\end{center}

\section{Problem 16.6.9}\label{problem-16.6.9}

\textbf{Given:} A near-field spherical scanning measurement is performed
on a 1.0 m × 0.5 m phased array panel at 30 GHz. The scan radius is 1.5
m from the array center.

\textbf{Find:} (a) The far-field distance, (b) whether the 1.5 m scan
radius is in the near field, (c) the advantages of near-field scanning
for this antenna, (d) the angular sampling interval required (Δθ ≈
λ/(2D) radians), and (e) the approximate number of measurement points
for a full spherical scan.

\textbf{Solution:}

\begin{enumerate}
\def\labelenumi{(\alph{enumi})}
\item
  D = 1.0 m (largest dimension); λ = c/f = 3 × 10⁸ / 30 × 10⁹ = 0.01 m
  d\textsubscript{ff} = 2D² / λ = 2 × 1.0 / 0.01 = \textbf{200 m}
\item
  The scan radius of 1.5 m is \textbf{well within the near field} (1.5 m
  \textless\textless{} 200 m). This is only 0.75\% of the far-field
  distance.
\item
  Advantages of near-field scanning for this antenna:
\end{enumerate}

\begin{itemize}
\tightlist
\item
  A 200 m outdoor range is impractical and expensive
\item
  Near-field scanning can be performed in a compact anechoic chamber
\item
  Both amplitude and phase are measured, enabling full far-field pattern
  computation via Fourier transform
\item
  Multiple far-field cuts can be computed from a single near-field data
  set
\end{itemize}

\begin{enumerate}
\def\labelenumi{(\alph{enumi})}
\setcounter{enumi}{3}
\item
  Angular sampling: Δθ ≈ λ/(2D) = 0.01 / (2 × 1.0) = 0.005 rad =
  \textbf{0.286°}
\item
  For a full spherical scan: N\textsubscript{θ} = 180° / 0.286° ≈ 629
  points in elevation N\textsubscript{φ} = 360° / 0.286° ≈ 1,259 points
  in azimuth Total ≈ 629 × 1,259 = \textbf{791,711 measurement points}
\end{enumerate}

In practice, the back hemisphere may be sparsely sampled if the antenna
is known to have low back radiation, reducing the total to approximately
400,000--500,000 points. At typical near-field scanning rates of 10--50
points per second, this requires 3--14 hours of measurement time.

\begin{center}\rule{0.5\linewidth}{0.5pt}\end{center}

\section{Problem 16.6.10}\label{problem-16.6.10}

\textbf{Given:} A sleeve balun is designed for a dipole antenna at 400
MHz. The balun uses a quarter-wave metallic sleeve around the outside of
a 50 Ω coaxial cable (outer diameter 10.3 mm). The sleeve inner diameter
is 20 mm.

\textbf{Find:} (a) The quarter-wave sleeve length, (b) the
characteristic impedance of the coaxial region between the cable outer
conductor and the sleeve using Z = (138/√ε\textsubscript{r}) ×
log₁₀(D/d), where D = 20 mm, d = 10.3 mm, and ε\textsubscript{r} = 1
(air), (c) the impedance presented at the feed point by the sleeve, (d)
the common-mode suppression mechanism, and (e) the bandwidth of the
balun.

\textbf{Solution:}

\begin{enumerate}
\def\labelenumi{(\alph{enumi})}
\item
  λ = c/f = 3 × 10⁸ / 400 × 10⁶ = 0.75 m Sleeve length = λ/4 =
  \textbf{187.5 mm}
\item
  Z\textsubscript{sleeve} = (138/√1) × log₁₀(20/10.3) = 138 ×
  log₁₀(1.942) Z\textsubscript{sleeve} = 138 × 0.2882 = \textbf{39.8 Ω}
\item
  At the feed point, the quarter-wave sleeve acts as a short-circuited
  quarter-wave transmission line. A quarter-wave shorted stub presents
  an \textbf{open circuit} (very high impedance) at its input. This high
  impedance at the feed point chokes off common-mode current on the
  outside of the coax shield.
\end{enumerate}

Z\textsubscript{input} = jZ\textsubscript{sleeve} × tan(βl) → ∞ as βl →
π/2

The impedance is theoretically infinite at the center frequency:
\textbf{open circuit}

\begin{enumerate}
\def\labelenumi{(\alph{enumi})}
\setcounter{enumi}{3}
\item
  The sleeve balun works by presenting a high impedance to common-mode
  currents at the feed point. Without the balun, currents flowing on the
  outside of the coax shield would see a low impedance path to ground.
  The quarter-wave sleeve transforms the short circuit at its far end to
  an open circuit at the feed point, effectively blocking common-mode
  current flow on the outer conductor.
\item
  The balun provides good performance (common-mode impedance
  \textgreater{} 500 Ω) over approximately a \textbf{20--25\% bandwidth}
  centered on 400 MHz (approximately 320--480 MHz). At frequencies away
  from resonance, the quarter-wave condition is no longer satisfied and
  the choking impedance decreases. For wider bandwidth applications, a
  ferrite-core balun or multiple cascaded balun sections would be
  preferred.
\end{enumerate}

\chapter{Chapter 17 --- Section 17.1: Radar
Fundamentals}\label{chapter-17-section-17.1-radar-fundamentals}

Practice problems covering the radar range equation, radar cross
section, Doppler effect and velocity measurement, and radar clutter and
noise.

\begin{center}\rule{0.5\linewidth}{0.5pt}\end{center}

\section{Problem 17.1.1}\label{problem-17.1.1}

\textbf{Given:} A ground-based air defense radar operates at 5.6 GHz
(C-band) with a peak transmit power of 500 kW, antenna gain of 38 dBi,
and system noise temperature of 500 K. The noise bandwidth is 2 MHz and
the required SNR for detection is 15 dB.

\textbf{Find:} (a) The wavelength, (b) the minimum detectable signal
S\textsubscript{min}, (c) the maximum detection range for a fighter
aircraft with RCS = 5 m², and (d) the maximum detection range for a
stealth aircraft with RCS = 0.005 m².

\textbf{Solution:}

\begin{enumerate}
\def\labelenumi{(\alph{enumi})}
\item
  λ = c / f = 3 × 10⁸ / 5.6 × 10⁹ = \textbf{0.0536 m} (53.6 mm)
\item
  S\textsubscript{min} = kT\textsubscript{s}B\textsubscript{n} ×
  SNR\textsubscript{min}
\end{enumerate}

S\textsubscript{min} = 1.381 × 10⁻²³ × 500 × 2 × 10⁶ ×
10\textsuperscript{15/10}

S\textsubscript{min} = 1.381 × 10⁻¹⁴ × 31.62 = \textbf{4.37 × 10⁻¹³ W}
(−93.6 dBm)

\begin{enumerate}
\def\labelenumi{(\alph{enumi})}
\setcounter{enumi}{2}
\tightlist
\item
  G = 10\textsuperscript{38/10} = 6,310 (linear).
\end{enumerate}

R\textsubscript{max} = (P\textsubscript{t}G²λ²σ / ((4π)³ ×
S\textsubscript{min}))\textsuperscript{1/4}

Numerator = 500 × 10³ × 6,310² × 0.0536² × 5 = 500 × 10³ × 3.982 × 10⁷ ×
2.873 × 10⁻³ × 5

Numerator = 500 × 10³ × 3.982 × 10⁷ × 1.436 × 10⁻² = 2.859 × 10¹¹

Denominator = (4π)³ × 4.37 × 10⁻¹³ = 1,984 × 4.37 × 10⁻¹³ = 8.67 × 10⁻¹⁰

R\textsubscript{max} = (2.859 × 10¹¹ / 8.67 ×
10⁻¹⁰)\textsuperscript{1/4} = (3.297 × 10²⁰)\textsuperscript{1/4} =
\textbf{134.8 km}

\begin{enumerate}
\def\labelenumi{(\alph{enumi})}
\setcounter{enumi}{3}
\tightlist
\item
  For σ = 0.005 m², the range scales as σ\textsuperscript{1/4}:
\end{enumerate}

R\textsubscript{stealth} = 134.8 × (0.005 / 5)\textsuperscript{1/4} =
134.8 × (0.001)\textsuperscript{1/4} = 134.8 / 5.623 = \textbf{24.0 km}

The stealth aircraft's 1000× smaller RCS reduces detection range by a
factor of 5.6.

\begin{center}\rule{0.5\linewidth}{0.5pt}\end{center}

\section{Problem 17.1.2}\label{problem-17.1.2}

\textbf{Given:} A weather balloon carries a metallic corner reflector
with face dimension L = 0.3 m as a radar calibration target. The radar
operates at 5.6 GHz.

\textbf{Find:} (a) The wavelength, (b) the RCS of the corner reflector,
(c) the RCS in dBsm, and (d) the equivalent sphere radius that would
produce the same RCS.

\textbf{Solution:}

\begin{enumerate}
\def\labelenumi{(\alph{enumi})}
\item
  λ = c / f = 3 × 10⁸ / 5.6 × 10⁹ = \textbf{0.0536 m}
\item
  Corner reflector RCS: σ = 12πL⁴ / λ² = 12π × 0.3⁴ / 0.0536²
\end{enumerate}

σ = 12π × 0.0081 / 2.873 × 10⁻³ = 0.3054 / 2.873 × 10⁻³ = \textbf{106.3
m²}

\begin{enumerate}
\def\labelenumi{(\alph{enumi})}
\setcounter{enumi}{2}
\item
  σ (dBsm) = 10 log₁₀(106.3) = \textbf{20.3 dBsm}
\item
  For a sphere: σ\textsubscript{sphere} = πa², so a = √(σ/π) =
  √(106.3/π) = √(33.84) = \textbf{5.82 m}
\end{enumerate}

A sphere would need a radius of 5.82 m to match the RCS of a 0.3 m
corner reflector --- demonstrating the extreme retroreflective
efficiency of the corner reflector geometry.

\begin{center}\rule{0.5\linewidth}{0.5pt}\end{center}

\section{Problem 17.1.3}\label{problem-17.1.3}

\textbf{Given:} A flat metallic plate of area A = 1.0 m² is oriented
perpendicular to a 10 GHz radar beam.

\textbf{Find:} (a) The plate's RCS, (b) the RCS in dBsm, (c) the RCS
when the plate is tilted 5° off-normal (assume RCS drops as sinc²(2πA
sin(θ)/λ²) --- approximate 20 dB reduction at 5° for this size), and (d)
how this illustrates the principle of stealth shaping.

\textbf{Solution:}

\begin{enumerate}
\def\labelenumi{(\alph{enumi})}
\tightlist
\item
  λ = c / f = 3 × 10⁸ / 10 × 10⁹ = 0.03 m.
\end{enumerate}

Flat plate RCS: σ = 4πA² / λ² = 4π × 1.0² / 0.03² = 4π / 9 × 10⁻⁴ =
\textbf{13,963 m²} (41.4 dBsm)

\begin{enumerate}
\def\labelenumi{(\alph{enumi})}
\setcounter{enumi}{1}
\item
  σ (dBsm) = 10 log₁₀(13,963) = \textbf{41.4 dBsm}
\item
  At 5° tilt, the specular RCS drops dramatically. With the approximate
  20 dB reduction:
\end{enumerate}

σ\textsubscript{tilted} = 13,963 / 100 = \textbf{139.6 m²} (21.4 dBsm)

\begin{enumerate}
\def\labelenumi{(\alph{enumi})}
\setcounter{enumi}{3}
\tightlist
\item
  \textbf{Stealth aircraft use angled surfaces} to deflect the specular
  reflection away from the radar receiver. A flat plate perpendicular to
  the beam has enormous RCS (13,963 m²), but even a small tilt
  dramatically reduces the backscattered energy. By ensuring no surface
  is perpendicular to likely threat radar directions, stealth designs
  reduce RCS by 30--40 dB compared to conventional aircraft of similar
  size.
\end{enumerate}

\begin{center}\rule{0.5\linewidth}{0.5pt}\end{center}

\section{Problem 17.1.4}\label{problem-17.1.4}

\textbf{Given:} An airborne fire-control radar operates at 10 GHz with a
PRF of 20 kHz. A target aircraft is approaching at 600 m/s.

\textbf{Find:} (a) The wavelength, (b) the Doppler frequency shift, (c)
the maximum unambiguous velocity, (d) whether the velocity measurement
is ambiguous, and (e) the maximum unambiguous range.

\textbf{Solution:}

\begin{enumerate}
\def\labelenumi{(\alph{enumi})}
\item
  λ = c / f = 3 × 10⁸ / 10 × 10⁹ = \textbf{0.03 m} (30 mm)
\item
  f\textsubscript{d} = 2v\textsubscript{r} / λ = 2 × 600 / 0.03 =
  \textbf{40,000 Hz} (40 kHz)
\item
  v\textsubscript{max} = λ × PRF / 4 = 0.03 × 20,000 / 4 = \textbf{150
  m/s} (540 km/h)
\item
  The target velocity (600 m/s) exceeds v\textsubscript{max} (150 m/s),
  so \textbf{yes, the measurement is ambiguous}. The Doppler shift of 40
  kHz equals 2 × PRF, so the target appears at the same Doppler as
  stationary clutter (a blind speed). The radar would need staggered
  PRFs or a higher PRF to resolve this ambiguity.
\item
  R\textsubscript{ua} = c / (2 × PRF) = 3 × 10⁸ / (2 × 20,000) =
  \textbf{7,500 m} (7.5 km)
\end{enumerate}

This illustrates the range-Doppler ambiguity: the high PRF needed for
unambiguous velocity measurement severely limits the unambiguous range.

\begin{center}\rule{0.5\linewidth}{0.5pt}\end{center}

\section{Problem 17.1.5}\label{problem-17.1.5}

\textbf{Given:} A traffic monitoring radar at 24.125 GHz (K-band)
measures vehicles on a highway. The radar is mounted on an overpass with
a 30° depression angle relative to the road.

\textbf{Find:} (a) The wavelength, (b) the Doppler shift for a vehicle
traveling at 100 km/h along the road, (c) the radial velocity component,
and (d) the velocity resolution if the coherent integration time is 20
ms.

\textbf{Solution:}

\begin{enumerate}
\def\labelenumi{(\alph{enumi})}
\item
  λ = c / f = 3 × 10⁸ / 24.125 × 10⁹ = \textbf{0.01243 m} (12.43 mm)
\item
  The radial velocity is the projection of the vehicle's horizontal
  velocity along the radar beam. With a 30° depression angle, the beam
  makes 30° below horizontal, so the angle between beam and vehicle
  velocity vector is 30°:
\end{enumerate}

v\textsubscript{r} = v × cos(30°) = 100 × 0.866 = 86.6 km/h = 24.06 m/s

Doppler shift: f\textsubscript{d} = 2v\textsubscript{r} / λ = 2 × 24.06
/ 0.01243 = \textbf{3,871 Hz} (3.87 kHz)

\begin{enumerate}
\def\labelenumi{(\alph{enumi})}
\setcounter{enumi}{2}
\item
  \textbf{v\textsubscript{r} = 24.06 m/s} --- the radar measures only
  the component of velocity along the beam direction. The actual vehicle
  speed is recovered by dividing by cos(30°).
\item
  Velocity resolution: Δv = λ / (2T\textsubscript{dwell}) = 0.01243 / (2
  × 0.020) = \textbf{0.311 m/s} (1.12 km/h)
\end{enumerate}

This fine velocity resolution allows the radar to distinguish vehicles
traveling at slightly different speeds.

\begin{center}\rule{0.5\linewidth}{0.5pt}\end{center}

\section{Problem 17.1.6}\label{problem-17.1.6}

\textbf{Given:} A ship-based X-band radar (9.4 GHz) has a pulse width of
50 ns, azimuth beamwidth of 0.9°, and transmit power of 50 kW. A fishing
vessel with RCS σ = 10 m² is at 15 km range. The sea state produces a
normalized clutter coefficient σ⁰ = −25 dB (3.162 × 10⁻³ m²/m²) at the
3° grazing angle.

\textbf{Find:} (a) The clutter cell area, (b) the total clutter RCS, (c)
the signal-to-clutter ratio (SCR), and (d) the SCR improvement needed
for 15 dB output SCR, and the number of MTI canceller stages required.

\textbf{Solution:}

\begin{enumerate}
\def\labelenumi{(\alph{enumi})}
\tightlist
\item
  A\textsubscript{c} = R × θ\textsubscript{az} × cτ / (2 cos ψ)
\end{enumerate}

A\textsubscript{c} = 15,000 × (0.9 × π/180) × (3 × 10⁸ × 50 × 10⁻⁹) / (2
× cos 3°)

A\textsubscript{c} = 15,000 × 0.01571 × 15 / 1.9986

A\textsubscript{c} = 15,000 × 0.01571 × 7.504 = \textbf{1,768 m²}

\begin{enumerate}
\def\labelenumi{(\alph{enumi})}
\setcounter{enumi}{1}
\item
  σ\textsubscript{c} = σ⁰ × A\textsubscript{c} = 3.162 × 10⁻³ × 1,768 =
  \textbf{5.59 m²} (7.5 dBsm)
\item
  SCR = σ\textsubscript{target} / σ\textsubscript{c} = 10 / 5.59 =
  \textbf{1.79} (2.5 dB)
\item
  Required improvement: 15 − 2.5 = \textbf{12.5 dB}
\end{enumerate}

A single-delay MTI canceller provides 20--25 dB improvement, which is
sufficient. \textbf{One MTI canceller stage} would provide adequate
clutter suppression for detection with a comfortable margin.

\begin{center}\rule{0.5\linewidth}{0.5pt}\end{center}

\section{Problem 17.1.7}\label{problem-17.1.7}

\textbf{Given:} A radar receiver front-end consists of: LNA with noise
figure F₁ = 1.2 dB and gain G₁ = 30 dB, followed by a bandpass filter
with insertion loss of 2 dB (noise figure = 2 dB, gain = −2 dB),
followed by a mixer with noise figure F₃ = 8 dB and conversion gain G₃ =
0 dB.

\textbf{Find:} (a) The system noise figure using the Friis formula, (b)
the system noise temperature (assuming T₀ = 290 K), and (c) how the
system noise figure changes if the filter and LNA are swapped (filter
first).

\textbf{Solution:}

\begin{enumerate}
\def\labelenumi{(\alph{enumi})}
\tightlist
\item
  Convert to linear values:
\end{enumerate}

F₁ = 10\textsuperscript{1.2/10} = 1.318, G₁ = 10\textsuperscript{30/10}
= 1,000

F₂ = 10\textsuperscript{2/10} = 1.585, G₂ = 10\textsuperscript{-2/10} =
0.631

F₃ = 10\textsuperscript{8/10} = 6.310

F\textsubscript{sys} = F₁ + (F₂ − 1)/G₁ + (F₃ − 1)/(G₁ × G₂)

F\textsubscript{sys} = 1.318 + 0.585/1,000 + 5.310/(1,000 × 0.631)

F\textsubscript{sys} = 1.318 + 0.000585 + 0.008415 = 1.327

F\textsubscript{sys} (dB) = 10 log₁₀(1.327) = \textbf{1.23 dB}

The LNA dominates the noise performance; downstream contributions are
negligible.

\begin{enumerate}
\def\labelenumi{(\alph{enumi})}
\setcounter{enumi}{1}
\item
  T\textsubscript{sys} = T₀ × (F\textsubscript{sys} − 1) = 290 × (1.327
  − 1) = 290 × 0.327 = \textbf{94.8 K}
\item
  With filter first (F₂ then F₁):
\end{enumerate}

F\textsubscript{sys,swapped} = F₂ + (F₁ − 1)/G₂ + (F₃ − 1)/(G₂ × G₁)

F\textsubscript{sys,swapped} = 1.585 + 0.318/0.631 + 5.310/(0.631 ×
1,000)

F\textsubscript{sys,swapped} = 1.585 + 0.504 + 0.00841 = 2.097

F\textsubscript{sys,swapped} (dB) = 10 log₁₀(2.097) = \textbf{3.22 dB}

The system noise figure \textbf{degrades by 2.0 dB} when the filter
precedes the LNA. This demonstrates why the LNA must always be the first
active component after the antenna.

\begin{center}\rule{0.5\linewidth}{0.5pt}\end{center}

\section{Problem 17.1.8}\label{problem-17.1.8}

\textbf{Given:} A pulse-Doppler radar at 3 GHz uses a coherent
processing interval of 10 ms to measure target velocity. Two targets are
at the same range but have different radial velocities.

\textbf{Find:} (a) The velocity resolution, (b) the minimum velocity
difference the radar can distinguish, (c) the number of pulses processed
if PRF = 5 kHz, and (d) the maximum unambiguous velocity.

\textbf{Solution:}

\begin{enumerate}
\def\labelenumi{(\alph{enumi})}
\tightlist
\item
  Velocity resolution: Δv = λ / (2T\textsubscript{dwell})
\end{enumerate}

λ = c / f = 3 × 10⁸ / 3 × 10⁹ = 0.10 m

Δv = 0.10 / (2 × 0.010) = \textbf{5.0 m/s} (18 km/h)

\begin{enumerate}
\def\labelenumi{(\alph{enumi})}
\setcounter{enumi}{1}
\item
  Two targets must differ by at least Δv to be resolved in the Doppler
  domain, so the minimum distinguishable velocity difference is
  \textbf{5.0 m/s}.
\item
  Number of pulses: N = PRF × T\textsubscript{dwell} = 5,000 × 0.010 =
  \textbf{50 pulses}
\end{enumerate}

These 50 pulses are processed with a 50-point FFT (or DFT) to form the
Doppler filter bank.

\begin{enumerate}
\def\labelenumi{(\alph{enumi})}
\setcounter{enumi}{3}
\tightlist
\item
  v\textsubscript{max} = λ × PRF / 4 = 0.10 × 5,000 / 4 = \textbf{125
  m/s} (450 km/h)
\end{enumerate}

\begin{center}\rule{0.5\linewidth}{0.5pt}\end{center}

\section{Problem 17.1.9}\label{problem-17.1.9}

\textbf{Given:} A radar system designer must achieve a maximum detection
range of 300 km for targets with σ = 1 m². The system parameters are:
frequency = 3 GHz, antenna gain = 40 dBi, system noise temperature = 450
K, noise bandwidth = 500 kHz, and required SNR = 14 dB.

\textbf{Find:} The minimum required peak transmit power.

\textbf{Solution:}

λ = c / f = 3 × 10⁸ / 3 × 10⁹ = 0.10 m

G = 10\textsuperscript{40/10} = 10,000

S\textsubscript{min} = kT\textsubscript{s}B\textsubscript{n} × SNR =
1.381 × 10⁻²³ × 450 × 500 × 10³ × 10\textsuperscript{14/10}

S\textsubscript{min} = 1.381 × 10⁻²³ × 450 × 5 × 10⁵ × 25.12

S\textsubscript{min} = 1.381 × 10⁻²³ × 5.652 × 10⁹ = 7.806 × 10⁻¹⁴ W

From the radar range equation at R\textsubscript{max}:

P\textsubscript{t} = R\textsubscript{max}⁴ × (4π)³ ×
S\textsubscript{min} / (G²λ²σ)

R\textsubscript{max}⁴ = (300 × 10³)⁴ = (3 × 10⁵)⁴ = 8.1 × 10²¹

Numerator = 8.1 × 10²¹ × 1,984 × 7.806 × 10⁻¹⁴ = 8.1 × 10²¹ × 1.549 ×
10⁻¹⁰ = 1.254 × 10¹²

Denominator = 10,000² × 0.10² × 1 = 10⁸ × 0.01 = 10⁶

P\textsubscript{t} = 1.254 × 10¹² / 10⁶ = \textbf{1.254 × 10⁶ W = 1.254
MW}

The required peak transmit power is approximately \textbf{1.25 MW},
which is typical for a high-power S-band surveillance radar.

\begin{center}\rule{0.5\linewidth}{0.5pt}\end{center}

\section{Problem 17.1.10}\label{problem-17.1.10}

\textbf{Given:} A radar system achieves coherent integration of 256
pulses and non-coherent integration of 256 pulses in two different
operating modes. The single-pulse SNR is −5 dB.

\textbf{Find:} (a) The SNR after coherent integration, (b) the SNR after
non-coherent integration, (c) the ratio of detection ranges for the two
modes, and (d) explain which mode is preferred and why.

\textbf{Solution:}

\begin{enumerate}
\def\labelenumi{(\alph{enumi})}
\tightlist
\item
  Coherent integration gain = 10 log₁₀(N) = 10 log₁₀(256) = 24.1 dB
\end{enumerate}

SNR\textsubscript{coherent} = −5 + 24.1 = \textbf{19.1 dB}

\begin{enumerate}
\def\labelenumi{(\alph{enumi})}
\setcounter{enumi}{1}
\tightlist
\item
  Non-coherent integration gain ≈ 10 log₁₀(√N) = 10 log₁₀(16) = 12.0 dB
\end{enumerate}

SNR\textsubscript{non-coherent} = −5 + 12.0 = \textbf{7.0 dB}

\begin{enumerate}
\def\labelenumi{(\alph{enumi})}
\setcounter{enumi}{2}
\tightlist
\item
  Since range ∝ SNR\textsuperscript{1/4}, the range ratio is:
\end{enumerate}

R\textsubscript{coherent} / R\textsubscript{non-coherent} =
(10\textsuperscript{19.1/10} /
10\textsuperscript{7.0/10})\textsuperscript{1/4} = (81.28 /
5.012)\textsuperscript{1/4} = (16.22)\textsuperscript{1/4} =
\textbf{2.01}

Coherent integration provides approximately \textbf{2× greater detection
range}.

\begin{enumerate}
\def\labelenumi{(\alph{enumi})}
\setcounter{enumi}{3}
\tightlist
\item
  \textbf{Coherent integration is preferred} when the radar can maintain
  phase coherence over 256 pulses (requiring stable transmitter and
  known target motion). The 12.1 dB advantage over non-coherent
  integration arises because coherent processing preserves the signal
  phase, allowing voltage addition rather than power addition. However,
  coherent integration requires accurate compensation for target motion
  (Doppler) during the integration time and is more computationally
  intensive. Non-coherent integration is simpler and more robust for
  fluctuating targets.
\end{enumerate}

\chapter{Chapter 17 --- Section 17.2: Radar Waveforms and
Modulation}\label{chapter-17-section-17.2-radar-waveforms-and-modulation}

Practice problems covering pulsed radar, continuous wave (CW) and FMCW
radar, and pulse compression.

\begin{center}\rule{0.5\linewidth}{0.5pt}\end{center}

\section{Problem 17.2.1}\label{problem-17.2.1}

\textbf{Given:} A shipboard surveillance radar operates at 9.4 GHz
(X-band) with a peak power of 50 kW, pulse width of 1.0 μs, and PRF of
1,500 Hz.

\textbf{Find:} (a) The range resolution, (b) the maximum unambiguous
range, (c) the duty cycle, (d) the average transmit power, and (e) the
pulse energy.

\textbf{Solution:}

\begin{enumerate}
\def\labelenumi{(\alph{enumi})}
\item
  Range resolution: ΔR = cτ\textsubscript{p} / 2 = 3 × 10⁸ × 1.0 × 10⁻⁶
  / 2 = \textbf{150 m}
\item
  Maximum unambiguous range: R\textsubscript{ua} = c / (2 × PRF) = 3 ×
  10⁸ / (2 × 1,500) = \textbf{100 km}
\item
  Duty cycle = τ\textsubscript{p} × PRF = 1.0 × 10⁻⁶ × 1,500 =
  \textbf{1.5 × 10⁻³} (0.15\%)
\item
  Average power: P\textsubscript{avg} = P\textsubscript{peak} × duty
  cycle = 50,000 × 1.5 × 10⁻³ = \textbf{75 W}
\item
  Pulse energy: E = P\textsubscript{peak} × τ\textsubscript{p} = 50,000
  × 1.0 × 10⁻⁶ = \textbf{50 mJ} per pulse
\end{enumerate}

\begin{center}\rule{0.5\linewidth}{0.5pt}\end{center}

\section{Problem 17.2.2}\label{problem-17.2.2}

\textbf{Given:} A radar designer needs to detect targets at 200 km range
with 50 m range resolution. The transmitter peak power is 200 kW.

\textbf{Find:} (a) The required pulse width for 50 m resolution
(uncompressed), (b) the corresponding pulse energy, (c) the required PRF
for 200 km unambiguous range, (d) the average power, and (e) whether
this is practical.

\textbf{Solution:}

\begin{enumerate}
\def\labelenumi{(\alph{enumi})}
\item
  ΔR = cτ\textsubscript{p} / 2, so τ\textsubscript{p} = 2ΔR / c = 2 × 50
  / 3 × 10⁸ = \textbf{3.33 × 10⁻⁷ s} (333 ns)
\item
  Pulse energy: E = P\textsubscript{peak} × τ\textsubscript{p} = 200 ×
  10³ × 3.33 × 10⁻⁷ = \textbf{66.7 mJ}
\item
  R\textsubscript{ua} = c / (2 × PRF), so PRF = c / (2 ×
  R\textsubscript{ua}) = 3 × 10⁸ / (2 × 200 × 10³) = \textbf{750 Hz}
\item
  P\textsubscript{avg} = P\textsubscript{peak} × τ\textsubscript{p} ×
  PRF = 200 × 10³ × 3.33 × 10⁻⁷ × 750 = \textbf{50 W}
\item
  The pulse energy (66.7 mJ) may be insufficient for detection at 200 km
  depending on target RCS. Using pulse compression (Section 17.2.3), a
  20 μs pulse with 10 MHz bandwidth would provide the same 50 m
  resolution with pulse energy of 200 × 10³ × 20 × 10⁻⁶ = 4 J ---
  \textbf{60× more energy} while maintaining resolution, making
  long-range detection practical.
\end{enumerate}

\begin{center}\rule{0.5\linewidth}{0.5pt}\end{center}

\section{Problem 17.2.3}\label{problem-17.2.3}

\textbf{Given:} An automotive FMCW radar at 77 GHz has a sweep bandwidth
of 2 GHz, sweep time of 30 μs, and transmit power of 12 dBm.

\textbf{Find:} (a) The range resolution, (b) the beat frequency for a
vehicle at 80 m, (c) the beat frequency for a pedestrian at 15 m, (d)
the maximum range for a 25 MHz ADC sampling rate, and (e) the
wavelength.

\textbf{Solution:}

\begin{enumerate}
\def\labelenumi{(\alph{enumi})}
\item
  ΔR = c / (2B) = 3 × 10⁸ / (2 × 2 × 10⁹) = \textbf{7.5 cm}
\item
  f\textsubscript{b} = 2RB / (cT\textsubscript{sweep}) = 2 × 80 × 2 ×
  10⁹ / (3 × 10⁸ × 30 × 10⁻⁶)
\end{enumerate}

f\textsubscript{b} = 3.2 × 10¹¹ / 9 × 10³ = \textbf{35.6 MHz}

\begin{enumerate}
\def\labelenumi{(\alph{enumi})}
\setcounter{enumi}{2}
\tightlist
\item
  f\textsubscript{b} = 2 × 15 × 2 × 10⁹ / (3 × 10⁸ × 30 × 10⁻⁶)
\end{enumerate}

f\textsubscript{b} = 6 × 10¹⁰ / 9 × 10³ = \textbf{6.67 MHz}

\begin{enumerate}
\def\labelenumi{(\alph{enumi})}
\setcounter{enumi}{3}
\tightlist
\item
  Maximum beat frequency = f\textsubscript{s} / 2 = 12.5 MHz (Nyquist).
\end{enumerate}

R\textsubscript{max} = f\textsubscript{b,max} × cT\textsubscript{sweep}
/ (2B) = 12.5 × 10⁶ × 3 × 10⁸ × 30 × 10⁻⁶ / (2 × 2 × 10⁹)

R\textsubscript{max} = 12.5 × 10⁶ × 9 × 10³ / 4 × 10⁹ = 1.125 × 10¹¹ / 4
× 10⁹ = \textbf{28.1 m}

The 80 m vehicle target (35.6 MHz beat) would alias at this ADC rate.
The ADC rate must be increased or the sweep time lengthened.

\begin{enumerate}
\def\labelenumi{(\alph{enumi})}
\setcounter{enumi}{4}
\tightlist
\item
  λ = c / f = 3 × 10⁸ / 77 × 10⁹ = \textbf{3.90 mm}
\end{enumerate}

\begin{center}\rule{0.5\linewidth}{0.5pt}\end{center}

\section{Problem 17.2.4}\label{problem-17.2.4}

\textbf{Given:} An FMCW radar uses triangular modulation (up-sweep then
down-sweep) at 24 GHz with bandwidth B = 250 MHz and sweep time
T\textsubscript{sweep} = 1 ms per ramp. A target produces beat
frequencies of 8.33 kHz on the up-sweep and 12.33 kHz on the down-sweep.

\textbf{Find:} (a) The target range, (b) the target radial velocity, (c)
the range resolution, and (d) the Doppler frequency.

\textbf{Solution:}

For triangular FMCW, the sum of beat frequencies gives range and the
difference gives velocity:

f\textsubscript{range} = (f\textsubscript{up} + f\textsubscript{down}) /
2 = (8,330 + 12,330) / 2 = 10,330 Hz

f\textsubscript{Doppler} = (f\textsubscript{down} − f\textsubscript{up})
/ 2 = (12,330 − 8,330) / 2 = 2,000 Hz

\begin{enumerate}
\def\labelenumi{(\alph{enumi})}
\tightlist
\item
  R = f\textsubscript{range} × cT\textsubscript{sweep} / (2B) = 10,330 ×
  3 × 10⁸ × 1 × 10⁻³ / (2 × 250 × 10⁶)
\end{enumerate}

R = 10,330 × 3 × 10⁵ / 5 × 10⁸ = 3.099 × 10⁹ / 5 × 10⁸ = \textbf{6.20 m}

\begin{enumerate}
\def\labelenumi{(\alph{enumi})}
\setcounter{enumi}{1}
\tightlist
\item
  λ = c / f = 3 × 10⁸ / 24 × 10⁹ = 0.0125 m
\end{enumerate}

v\textsubscript{r} = f\textsubscript{Doppler} × λ / 2 = 2,000 × 0.0125 /
2 = \textbf{12.5 m/s} (45 km/h, approaching)

\begin{enumerate}
\def\labelenumi{(\alph{enumi})}
\setcounter{enumi}{2}
\item
  ΔR = c / (2B) = 3 × 10⁸ / (2 × 250 × 10⁶) = \textbf{0.60 m}
\item
  \textbf{f\textsubscript{Doppler} = 2,000 Hz} (computed above)
\end{enumerate}

\begin{center}\rule{0.5\linewidth}{0.5pt}\end{center}

\section{Problem 17.2.5}\label{problem-17.2.5}

\textbf{Given:} A military airborne radar transmits a 20 μs LFM chirp
with 20 MHz bandwidth at a peak power of 10 kW. A non-compressed backup
mode uses a 0.5 μs simple pulse at 10 kW.

\textbf{Find:} (a) The compression ratio, (b) the range resolution with
and without compression, (c) the effective peak power after
matched-filter compression, (d) the compressed pulse width, and (e) the
ratio of detection ranges for the two modes.

\textbf{Solution:}

\begin{enumerate}
\def\labelenumi{(\alph{enumi})}
\item
  CR = Bτ\textsubscript{p} = 20 × 10⁶ × 20 × 10⁻⁶ = \textbf{400} (26 dB)
\item
  With compression: ΔR = c / (2B) = 3 × 10⁸ / (2 × 20 × 10⁶) =
  \textbf{7.5 m}
\end{enumerate}

Without compression (0.5 μs pulse): ΔR = cτ\textsubscript{p} / 2 = 3 ×
10⁸ × 0.5 × 10⁻⁶ / 2 = \textbf{75 m}

\begin{enumerate}
\def\labelenumi{(\alph{enumi})}
\setcounter{enumi}{2}
\item
  P\textsubscript{eff} = P\textsubscript{peak} × CR = 10 × 10³ × 400 =
  \textbf{4 MW} equivalent peak power
\item
  τ\textsubscript{compressed} = 1 / B = 1 / (20 × 10⁶) = \textbf{50 ns}
\item
  The chirp pulse energy is E\textsubscript{chirp} = 10,000 × 20 × 10⁻⁶
  = 0.2 J.
\end{enumerate}

The simple pulse energy is E\textsubscript{simple} = 10,000 × 0.5 × 10⁻⁶
= 5 × 10⁻³ J.

The energy ratio is 0.2 / 0.005 = 40. Since range ∝
E\textsuperscript{1/4}:

R\textsubscript{chirp} / R\textsubscript{simple} =
40\textsuperscript{1/4} = \textbf{2.51}

The pulse compression mode provides \textbf{2.5× greater detection
range} with \textbf{10× better range resolution}.

\begin{center}\rule{0.5\linewidth}{0.5pt}\end{center}

\section{Problem 17.2.6}\label{problem-17.2.6}

\textbf{Given:} A ground surveillance radar uses Barker-coded phase
modulation with a 13-bit Barker code. The chip duration is 100 ns and
the peak power is 500 W.

\textbf{Find:} (a) The total pulse duration, (b) the compression ratio,
(c) the range resolution, (d) the sidelobe level of the compressed
pulse, and (e) the pulse energy.

\textbf{Solution:}

\begin{enumerate}
\def\labelenumi{(\alph{enumi})}
\item
  Total pulse duration: τ\textsubscript{p} = N × τ\textsubscript{chip} =
  13 × 100 × 10⁻⁹ = \textbf{1.3 μs}
\item
  Compression ratio for a Barker code: CR = N = \textbf{13} (11.1 dB)
\item
  The compressed pulse width equals the chip duration. Range resolution:
\end{enumerate}

ΔR = c × τ\textsubscript{chip} / 2 = 3 × 10⁸ × 100 × 10⁻⁹ / 2 =
\textbf{15 m}

Without compression, ΔR would be 3 × 10⁸ × 1.3 × 10⁻⁶ / 2 = 195 m.

\begin{enumerate}
\def\labelenumi{(\alph{enumi})}
\setcounter{enumi}{3}
\tightlist
\item
  Barker code sidelobe level = 1/N = 1/13 in amplitude, so the peak
  sidelobe level is:
\end{enumerate}

PSL = 20 log₁₀(1/13) = −20 log₁₀(13) = \textbf{−22.3 dB}

This is the best achievable sidelobe level for any binary phase code of
length 13, and N = 13 is the longest known Barker code.

\begin{enumerate}
\def\labelenumi{(\alph{enumi})}
\setcounter{enumi}{4}
\tightlist
\item
  Pulse energy: E = P\textsubscript{peak} × τ\textsubscript{p} = 500 ×
  1.3 × 10⁻⁶ = \textbf{0.65 mJ}
\end{enumerate}

\begin{center}\rule{0.5\linewidth}{0.5pt}\end{center}

\section{Problem 17.2.7}\label{problem-17.2.7}

\textbf{Given:} A CW police radar operates at 34.7 GHz (Ka-band). It
measures a Doppler shift of 6,250 Hz from a vehicle.

\textbf{Find:} (a) The wavelength, (b) the vehicle speed, (c) the
vehicle speed in miles per hour, and (d) the minimum integration time to
resolve two vehicles with speeds differing by 2 km/h.

\textbf{Solution:}

\begin{enumerate}
\def\labelenumi{(\alph{enumi})}
\item
  λ = c / f = 3 × 10⁸ / 34.7 × 10⁹ = \textbf{8.65 × 10⁻³ m} (8.65 mm)
\item
  f\textsubscript{d} = 2v\textsubscript{r} / λ, so v\textsubscript{r} =
  f\textsubscript{d} × λ / 2 = 6,250 × 8.65 × 10⁻³ / 2 = \textbf{27.03
  m/s} (97.3 km/h)
\item
  v = 97.3 km/h × 0.6214 = \textbf{60.5 mph}
\item
  Δv = 2 km/h = 0.556 m/s. Velocity resolution Δv = λ /
  (2T\textsubscript{dwell}):
\end{enumerate}

T\textsubscript{dwell} = λ / (2Δv) = 8.65 × 10⁻³ / (2 × 0.556) =
\textbf{7.78 ms}

This short integration time makes Ka-band CW radar excellent for
resolving individual vehicles in traffic flow.

\chapter{Chapter 17 --- Section 17.3: Radar Signal
Processing}\label{chapter-17-section-17.3-radar-signal-processing}

Practice problems covering MTI and clutter rejection, CFAR detection,
SAR imaging, and target tracking.

\begin{center}\rule{0.5\linewidth}{0.5pt}\end{center}

\section{Problem 17.3.1}\label{problem-17.3.1}

\textbf{Given:} An L-band surveillance radar operates at 1.3 GHz with a
PRF of 500 Hz and uses a 2-pulse MTI canceller with 25 dB clutter
improvement factor.

\textbf{Find:} (a) The wavelength, (b) the first blind speed, (c) the
second and third blind speeds, (d) a second PRF that eliminates the
first blind speed when used in a staggered-PRF scheme, and (e) the first
common blind speed for the two PRFs.

\textbf{Solution:}

\begin{enumerate}
\def\labelenumi{(\alph{enumi})}
\item
  λ = c / f = 3 × 10⁸ / 1.3 × 10⁹ = \textbf{0.2308 m}
\item
  First blind speed: v\textsubscript{blind1} = λ × PRF / 2 = 0.2308 ×
  500 / 2 = \textbf{57.7 m/s} (208 km/h)
\item
  Second blind speed: v\textsubscript{blind2} = 2 × 57.7 = \textbf{115.4
  m/s} (415 km/h)
\end{enumerate}

Third blind speed: v\textsubscript{blind3} = 3 × 57.7 = \textbf{173.1
m/s} (623 km/h)

\begin{enumerate}
\def\labelenumi{(\alph{enumi})}
\setcounter{enumi}{3}
\tightlist
\item
  Choose PRF₂ = 600 Hz (ratio 5:6 with PRF₁).
\end{enumerate}

Blind speeds of PRF₁ (500 Hz): 57.7, 115.4, 173.1, 230.8, 288.5,
346.2\ldots{} m/s

Blind speeds of PRF₂ (600 Hz): v\textsubscript{blind} = 0.2308 × 600 / 2
= 69.2, 138.5, 207.7, 276.9, 346.2\ldots{} m/s

\begin{enumerate}
\def\labelenumi{(\alph{enumi})}
\setcounter{enumi}{4}
\tightlist
\item
  The first common blind speed occurs at 346.2 m/s = \textbf{1,246
  km/h}. This is well above the speed of any conventional aircraft,
  effectively eliminating the blind speed problem.
\end{enumerate}

\textbf{PRF₂ = 600 Hz} with the first common blind speed at
\textbf{346.2 m/s}.

\begin{center}\rule{0.5\linewidth}{0.5pt}\end{center}

\section{Problem 17.3.2}\label{problem-17.3.2}

\textbf{Given:} An air traffic control radar has a 2-pulse MTI canceller
operating at 2.8 GHz with PRF = 1,200 Hz. An aircraft is flying at 200
m/s.

\textbf{Find:} (a) The Doppler frequency of the aircraft, (b) the first
blind speed, (c) the MTI filter response at the aircraft's Doppler
frequency (the response of a single-canceller is H(f) = 2 sin(πf/PRF)),
and (d) the canceller output relative to the peak response.

\textbf{Solution:}

\begin{enumerate}
\def\labelenumi{(\alph{enumi})}
\tightlist
\item
  λ = c / f = 3 × 10⁸ / 2.8 × 10⁹ = 0.1071 m
\end{enumerate}

f\textsubscript{d} = 2v\textsubscript{r} / λ = 2 × 200 / 0.1071 =
\textbf{3,733 Hz}

\begin{enumerate}
\def\labelenumi{(\alph{enumi})}
\setcounter{enumi}{1}
\item
  v\textsubscript{blind} = λ × PRF / 2 = 0.1071 × 1,200 / 2 =
  \textbf{64.3 m/s} (231 km/h)
\item
  The normalized Doppler frequency: f\textsubscript{d} / PRF = 3,733 /
  1,200 = 3.111
\end{enumerate}

The fractional part is 0.111 (since the response is periodic with period
PRF).

H(f) = 2\textbar sin(π × f\textsubscript{d} / PRF)\textbar{} =
2\textbar sin(π × 3.111)\textbar{} = 2\textbar sin(0.111π)\textbar{} = 2
× sin(0.3491)

H = 2 × 0.342 = \textbf{0.684}

\begin{enumerate}
\def\labelenumi{(\alph{enumi})}
\setcounter{enumi}{3}
\tightlist
\item
  The peak response is 2.0 (at f\textsubscript{d} = PRF/2). The response
  at the aircraft Doppler is:
\end{enumerate}

Relative response = 0.684 / 2.0 = 0.342 = \textbf{−9.3 dB}

The aircraft is near (but not at) a blind speed, suffering 9.3 dB of
signal attenuation by the MTI filter. A double canceller would suffer
18.6 dB attenuation at this velocity, so staggered PRFs should be used.

\begin{center}\rule{0.5\linewidth}{0.5pt}\end{center}

\section{Problem 17.3.3}\label{problem-17.3.3}

\textbf{Given:} A CA-CFAR processor uses N = 24 reference cells (12
leading, 12 trailing) with 3 guard cells on each side. The desired false
alarm probability is P\textsubscript{fa} = 10⁻⁴.

\textbf{Find:} (a) The threshold multiplier α, (b) the threshold in dB
above the noise estimate, (c) the CFAR loss, and (d) the total number of
cells in the sliding window.

\textbf{Solution:}

\begin{enumerate}
\def\labelenumi{(\alph{enumi})}
\tightlist
\item
  α = N × (P\textsubscript{fa}\textsuperscript{−1/N} − 1) = 24 ×
  ((10⁻⁴)\textsuperscript{−1/24} − 1) = 24 × (10\textsuperscript{4/24} −
  1)
\end{enumerate}

10\textsuperscript{4/24} = 10\textsuperscript{0.1667} = 1.468

α = 24 × (1.468 − 1) = 24 × 0.468 = \textbf{11.23}

\begin{enumerate}
\def\labelenumi{(\alph{enumi})}
\setcounter{enumi}{1}
\item
  Threshold = 10 log₁₀(11.23) = \textbf{10.5 dB} above the estimated
  noise floor
\item
  CFAR loss ≈ 10 log₁₀(1 + 2/N) = 10 log₁₀(1 + 2/24) = 10 log₁₀(1.0833)
  = \textbf{0.35 dB}
\item
  Total cells = 1 (CUT) + 6 (guard cells) + 24 (reference cells) =
  \textbf{31 range cells}
\end{enumerate}

\begin{center}\rule{0.5\linewidth}{0.5pt}\end{center}

\section{Problem 17.3.4}\label{problem-17.3.4}

\textbf{Given:} A CA-CFAR with N = 32 reference cells is compared
against an OS-CFAR using the k = 24th ranked cell (out of 32) for the
threshold estimate. Both have P\textsubscript{fa} = 10⁻⁶.

\textbf{Find:} (a) The CA-CFAR threshold multiplier, (b) the CA-CFAR
threshold in dB, (c) the CFAR loss for the CA-CFAR, and (d) explain
qualitatively why OS-CFAR is preferred in multi-target environments.

\textbf{Solution:}

\begin{enumerate}
\def\labelenumi{(\alph{enumi})}
\tightlist
\item
  α\textsubscript{CA} = N × (P\textsubscript{fa}\textsuperscript{−1/N} −
  1) = 32 × ((10⁻⁶)\textsuperscript{−1/32} − 1) = 32 ×
  (10\textsuperscript{6/32} − 1)
\end{enumerate}

10\textsuperscript{6/32} = 10\textsuperscript{0.1875} = 1.539

α\textsubscript{CA} = 32 × (1.539 − 1) = 32 × 0.539 = \textbf{17.25}

\begin{enumerate}
\def\labelenumi{(\alph{enumi})}
\setcounter{enumi}{1}
\item
  Threshold = 10 log₁₀(17.25) = \textbf{12.4 dB}
\item
  CFAR loss = 10 log₁₀(1 + 2/32) = 10 log₁₀(1.0625) = \textbf{0.26 dB}
\item
  \textbf{OS-CFAR ranks the reference cells by power and uses the k-th
  ordered value}, making it robust against interfering targets in the
  reference window. If 2--3 strong targets fall within the CA-CFAR
  reference window, they raise the average noise estimate and increase
  the threshold, masking weaker targets (the ``target masking''
  problem). OS-CFAR with k = 24 (the 75th percentile) tolerates up to N
  − k = 8 interfering targets in the window without significantly
  raising the threshold, at the cost of slightly higher CFAR loss
  (typically 1--2 dB more than CA-CFAR in homogeneous noise).
\end{enumerate}

\begin{center}\rule{0.5\linewidth}{0.5pt}\end{center}

\section{Problem 17.3.5}\label{problem-17.3.5}

\textbf{Given:} An airborne SAR operates at 9.6 GHz (X-band) with a
platform velocity of 250 m/s and antenna length D = 1.5 m. The
transmitted bandwidth is 300 MHz, and the target is at a slant range of
30 km.

\textbf{Find:} (a) The wavelength, (b) the cross-range (azimuth)
resolution, (c) the range resolution, (d) the synthetic aperture length,
(e) the coherent integration time, and (f) the required motion
compensation accuracy.

\textbf{Solution:}

\begin{enumerate}
\def\labelenumi{(\alph{enumi})}
\item
  λ = c / f = 3 × 10⁸ / 9.6 × 10⁹ = \textbf{0.03125 m}
\item
  Cross-range resolution: δ\textsubscript{cr} = D / 2 = 1.5 / 2 =
  \textbf{0.75 m}
\item
  Range resolution: δ\textsubscript{r} = c / (2B) = 3 × 10⁸ / (2 × 300 ×
  10⁶) = \textbf{0.50 m}
\item
  Synthetic aperture length: L\textsubscript{SA} = λR / D = 0.03125 ×
  30,000 / 1.5 = \textbf{625 m}
\item
  Integration time: T\textsubscript{int} = L\textsubscript{SA} / v = 625
  / 250 = \textbf{2.50 seconds}
\item
  Motion compensation accuracy must be a fraction of the wavelength:
\end{enumerate}

Required accuracy \textless{} λ / 4 = 0.03125 / 4 = \textbf{7.81 mm}

The platform's position must be known to better than 7.81 mm during the
2.5-second integration time, requiring high-precision INS/GPS navigation
or autofocus algorithms.

\begin{center}\rule{0.5\linewidth}{0.5pt}\end{center}

\section{Problem 17.3.6}\label{problem-17.3.6}

\textbf{Given:} A spotlight SAR steers its beam to dwell on a specific
area, achieving an integration angle of θ\textsubscript{int} = 5° (total
angular extent of data collection). The radar operates at 5.3 GHz
(C-band).

\textbf{Find:} (a) The wavelength, (b) the cross-range resolution
(δ\textsubscript{cr} = λ / (2θ\textsubscript{int}) for spotlight mode),
and (c) how this compares to the stripmap resolution with a 3 m antenna.

\textbf{Solution:}

\begin{enumerate}
\def\labelenumi{(\alph{enumi})}
\item
  λ = c / f = 3 × 10⁸ / 5.3 × 10⁹ = \textbf{0.0566 m}
\item
  θ\textsubscript{int} = 5° = 0.0873 rad
\end{enumerate}

δ\textsubscript{cr} = λ / (2θ\textsubscript{int}) = 0.0566 / (2 ×
0.0873) = 0.0566 / 0.1745 = \textbf{0.324 m} (32.4 cm)

\begin{enumerate}
\def\labelenumi{(\alph{enumi})}
\setcounter{enumi}{2}
\tightlist
\item
  Stripmap resolution: δ\textsubscript{cr,strip} = D / 2 = 3 / 2 = 1.50
  m
\end{enumerate}

The spotlight mode achieves \textbf{4.6× finer cross-range resolution}
(0.324 m vs.~1.50 m) by dwelling on the target area and collecting data
over a wider angular range, at the expense of area coverage rate.

\begin{center}\rule{0.5\linewidth}{0.5pt}\end{center}

\section{Problem 17.3.7}\label{problem-17.3.7}

\textbf{Given:} A surveillance radar tracks an aircraft at 80 km range
with range accuracy σ\textsubscript{R} = 75 m and azimuth accuracy
σ\textsubscript{θ} = 0.5°. The scan interval is 4 seconds and the
aircraft velocity is 300 m/s. A Kalman filter is used with a
constant-velocity model.

\textbf{Find:} (a) The cross-range position accuracy, (b) the prediction
uncertainty after one scan interval, (c) the track gate size for 95\%
association probability, and (d) the distance the aircraft travels
between scans.

\textbf{Solution:}

\begin{enumerate}
\def\labelenumi{(\alph{enumi})}
\item
  Cross-range accuracy: σ\textsubscript{cr} = R × σ\textsubscript{θ} =
  80,000 × (0.5 × π/180) = 80,000 × 0.008727 = \textbf{698 m}
\item
  The velocity estimate accuracy: σ\textsubscript{v} ≈
  σ\textsubscript{cr} / T = 698 / 4 = 174.5 m/s
\end{enumerate}

Predicted position uncertainty after one scan:

σ\textsubscript{pred} = √(σ\textsubscript{cr}² + (σ\textsubscript{v} ×
T)²) = √(698² + (174.5 × 4)²) = √(698² + 698²) = 698 × √2 = \textbf{987
m} (cross-range)

In range: σ\textsubscript{pred,R} = √(75² + (σ\textsubscript{vR} × 4)²)
≈ \textbf{106 m} (assuming similar ratio)

\begin{enumerate}
\def\labelenumi{(\alph{enumi})}
\setcounter{enumi}{2}
\tightlist
\item
  For 2D Gaussian, 95\% probability corresponds to a gate radius of
  \textasciitilde2.45σ (chi-squared with 2 DOF, p = 0.95):
\end{enumerate}

Range gate: 2.45 × 75 = \textbf{184 m}

Cross-range gate: 2.45 × 987 = \textbf{2,418 m}

\begin{enumerate}
\def\labelenumi{(\alph{enumi})}
\setcounter{enumi}{3}
\tightlist
\item
  Distance traveled: d = v × T = 300 × 4 = \textbf{1,200 m}
\end{enumerate}

The aircraft moves 1,200 m between scans, which is within the
cross-range gate (2,418 m), confirming that track association should
succeed for this scenario.

\begin{center}\rule{0.5\linewidth}{0.5pt}\end{center}

\section{Problem 17.3.8}\label{problem-17.3.8}

\textbf{Given:} A radar tracking system uses M-of-N logic for track
initiation. Two configurations are compared: 3-of-5 and 4-of-7. The
single-scan probability of detection is P\textsubscript{d} = 0.8 and the
false alarm probability per resolution cell is P\textsubscript{fa} =
10⁻⁶.

\textbf{Find:} (a) The probability of initiating a true track with
3-of-5 logic, (b) the probability of initiating a true track with 4-of-7
logic, and (c) explain the trade-off between the two configurations.

\textbf{Solution:}

\begin{enumerate}
\def\labelenumi{(\alph{enumi})}
\tightlist
\item
  P(3-of-5) uses the binomial distribution with p = P\textsubscript{d} =
  0.8:
\end{enumerate}

P(≥ 3 of 5) = P(3) + P(4) + P(5)

P(3) = C(5,3) × 0.8³ × 0.2² = 10 × 0.512 × 0.04 = 0.2048

P(4) = C(5,4) × 0.8⁴ × 0.2¹ = 5 × 0.4096 × 0.2 = 0.4096

P(5) = C(5,5) × 0.8⁵ × 0.2⁰ = 1 × 0.3277 = 0.3277

P(≥ 3 of 5) = 0.2048 + 0.4096 + 0.3277 = \textbf{0.942} (94.2\%)

\begin{enumerate}
\def\labelenumi{(\alph{enumi})}
\setcounter{enumi}{1}
\tightlist
\item
  P(≥ 4 of 7) = P(4) + P(5) + P(6) + P(7)
\end{enumerate}

P(4) = C(7,4) × 0.8⁴ × 0.2³ = 35 × 0.4096 × 0.008 = 0.1147

P(5) = C(7,5) × 0.8⁵ × 0.2² = 21 × 0.3277 × 0.04 = 0.2753

P(6) = C(7,6) × 0.8⁶ × 0.2¹ = 7 × 0.2621 × 0.2 = 0.3670

P(7) = C(7,7) × 0.8⁷ = 0.2097

P(≥ 4 of 7) = 0.1147 + 0.2753 + 0.3670 + 0.2097 = \textbf{0.967}
(96.7\%)

\begin{enumerate}
\def\labelenumi{(\alph{enumi})}
\setcounter{enumi}{2}
\tightlist
\item
  The 4-of-7 logic provides \textbf{higher track initiation probability}
  (96.7\% vs.~94.2\%) because it allows more missed detections (3 vs.~2)
  over a longer observation window. However, it requires 7 scans to
  confirm a track versus 5 scans, increasing the track initiation
  latency. For a radar with a 5-second scan time, 3-of-5 takes 25
  seconds while 4-of-7 takes 35 seconds. The choice depends on whether
  detection reliability or response time is more critical.
\end{enumerate}

\chapter{Chapter 17 --- Section 17.4: Radar
Applications}\label{chapter-17-section-17.4-radar-applications}

Practice problems covering air traffic control, weather radar,
automotive radar, ground-penetrating radar, missile defense radar, and
LiDAR.

\begin{center}\rule{0.5\linewidth}{0.5pt}\end{center}

\section{Problem 17.4.1}\label{problem-17.4.1}

\textbf{Given:} An ASR-9 airport surveillance radar rotates at 12.5 RPM
with an azimuth beamwidth of 1.4°, PRF of 1,200 Hz, frequency of 2.8
GHz, and peak power of 1.3 MW. The pulse width is 1.0 μs.

\textbf{Find:} (a) The scan time, (b) the number of pulses on target per
scan, (c) the non-coherent integration gain, (d) the maximum unambiguous
range, and (e) the range resolution.

\textbf{Solution:}

\begin{enumerate}
\def\labelenumi{(\alph{enumi})}
\item
  Scan time = 60 / RPM = 60 / 12.5 = \textbf{4.8 seconds}
\item
  Time on target: T\textsubscript{OT} = θ\textsubscript{az} / (360° ×
  RPM/60) = 1.4 / (360 × 12.5/60) = 1.4 / 75 = 0.01867 s
\end{enumerate}

Hits per scan = PRF × T\textsubscript{OT} = 1,200 × 0.01867 =
\textbf{22.4 ≈ 22 pulses}

\begin{enumerate}
\def\labelenumi{(\alph{enumi})}
\setcounter{enumi}{2}
\item
  Non-coherent integration gain ≈ √N = √22 = \textbf{4.69} (6.7 dB)
\item
  R\textsubscript{ua} = c / (2 × PRF) = 3 × 10⁸ / (2 × 1,200) =
  \textbf{125 km} (67.5 nmi)
\item
  ΔR = cτ\textsubscript{p} / 2 = 3 × 10⁸ × 1.0 × 10⁻⁶ / 2 = \textbf{150
  m}
\end{enumerate}

\begin{center}\rule{0.5\linewidth}{0.5pt}\end{center}

\section{Problem 17.4.2}\label{problem-17.4.2}

\textbf{Given:} A NEXRAD weather radar measures a storm cell at 80 km
range with a reflectivity of 45 dBZ. The radar operates at 2.8 GHz with
a dwell time of 40 ms.

\textbf{Find:} (a) The reflectivity in linear units (mm⁶/m³), (b) the
estimated rainfall rate using the Marshall-Palmer relation (Z =
200R\textsubscript{r}\textsuperscript{1.6}), (c) the storm
classification, and (d) the Doppler velocity resolution.

\textbf{Solution:}

\begin{enumerate}
\def\labelenumi{(\alph{enumi})}
\item
  Z = 10\textsuperscript{45/10} = 10\textsuperscript{4.5} =
  \textbf{31,623 mm⁶/m³}
\item
  R\textsubscript{r} = (Z / 200)\textsuperscript{1/1.6} = (31,623 /
  200)\textsuperscript{0.625} = (158.1)\textsuperscript{0.625}
\end{enumerate}

ln(158.1) = 5.063, × 0.625 = 3.164

R\textsubscript{r} = e\textsuperscript{3.164} = \textbf{23.7 mm/h}
(about 1 inch/hour)

\begin{enumerate}
\def\labelenumi{(\alph{enumi})}
\setcounter{enumi}{2}
\item
  At 45 dBZ, this is classified as \textbf{heavy rain} (40--50 dBZ
  range). This would appear as red on a standard weather radar display.
\item
  λ = c / f = 3 × 10⁸ / 2.8 × 10⁹ = 0.1071 m
\end{enumerate}

Δv = λ / (2T\textsubscript{dwell}) = 0.1071 / (2 × 0.040) = \textbf{1.34
m/s} (4.8 km/h)

\begin{center}\rule{0.5\linewidth}{0.5pt}\end{center}

\section{Problem 17.4.3}\label{problem-17.4.3}

\textbf{Given:} A NEXRAD radar observes a mesocyclone in a supercell
thunderstorm. The Doppler velocity measurements show +30 m/s on one side
and −25 m/s on the opposite side of the rotation, separated by 5 km at a
range of 60 km.

\textbf{Find:} (a) The total rotational velocity differential, (b) the
angular velocity of rotation, (c) whether this meets the criteria for a
tornado vortex signature (TVS requires velocity differential
\textgreater{} 30 m/s across a distance \textless{} 4 km), and (d) the
reflectivity if the rainfall rate is 75 mm/h.

\textbf{Solution:}

\begin{enumerate}
\def\labelenumi{(\alph{enumi})}
\item
  Velocity differential: ΔV = \textbar+30 − (−25)\textbar{} = \textbf{55
  m/s} (198 km/h)
\item
  Angular velocity: ω = ΔV / separation = 55 / 5,000 = \textbf{0.011
  rad/s} (0.63°/s)
\item
  The velocity differential of 55 m/s exceeds the 30 m/s threshold, but
  the separation of 5 km exceeds the 4 km distance threshold. This is a
  \textbf{strong mesocyclone} but does not meet the formal TVS criteria
  at this range. At closer range, the smaller beam width might resolve a
  tighter rotation. \textbf{A tornado warning would likely be issued}
  based on this strong mesocyclone signature combined with other
  factors.
\item
  Z = 200 × R\textsubscript{r}\textsuperscript{1.6} = 200 ×
  75\textsuperscript{1.6}
\end{enumerate}

75\textsuperscript{1.6} = e\textsuperscript{1.6 × ln(75)} =
e\textsuperscript{1.6 × 4.317} = e\textsuperscript{6.908} = 999.5

Z = 200 × 999.5 = 199,900 mm⁶/m³

dBZ = 10 log₁₀(199,900) = \textbf{53.0 dBZ} (classified as very heavy
rain with possible hail)

\begin{center}\rule{0.5\linewidth}{0.5pt}\end{center}

\section{Problem 17.4.4}\label{problem-17.4.4}

\textbf{Given:} A 77 GHz FMCW automotive radar (long-range mode) has
bandwidth B = 500 MHz, sweep time T\textsubscript{sweep} = 40 μs, and
uses 256 chirps in a frame. The transmit antenna gain is 25 dBi and the
transmit power is 12 dBm (15.8 mW).

\textbf{Find:} (a) The range resolution, (b) the velocity resolution,
(c) the maximum unambiguous velocity, (d) the maximum unambiguous range
for a 50 MHz maximum beat frequency, and (e) the wavelength.

\textbf{Solution:}

\begin{enumerate}
\def\labelenumi{(\alph{enumi})}
\item
  ΔR = c / (2B) = 3 × 10⁸ / (2 × 500 × 10⁶) = \textbf{30 cm}
\item
  λ = c / f = 3 × 10⁸ / 77 × 10⁹ = 3.896 × 10⁻³ m
\end{enumerate}

Frame time: T\textsubscript{frame} = 256 × 40 × 10⁻⁶ = 10.24 ms

Δv = λ / (2T\textsubscript{frame}) = 3.896 × 10⁻³ / (2 × 10.24 × 10⁻³) =
\textbf{0.190 m/s} (0.685 km/h)

\begin{enumerate}
\def\labelenumi{(\alph{enumi})}
\setcounter{enumi}{2}
\item
  v\textsubscript{max} = λ / (4 × T\textsubscript{sweep}) = 3.896 × 10⁻³
  / (4 × 40 × 10⁻⁶) = \textbf{24.4 m/s} (87.7 km/h)
\item
  R\textsubscript{max} = f\textsubscript{b,max} ×
  cT\textsubscript{sweep} / (2B) = 50 × 10⁶ × 3 × 10⁸ × 40 × 10⁻⁶ / (2 ×
  500 × 10⁶)
\end{enumerate}

R\textsubscript{max} = 50 × 10⁶ × 1.2 × 10⁴ / 10⁹ = 6 × 10¹¹ / 10⁹ =
\textbf{600 m}

\begin{enumerate}
\def\labelenumi{(\alph{enumi})}
\setcounter{enumi}{4}
\tightlist
\item
  \textbf{λ = 3.90 mm} (computed in step (b))
\end{enumerate}

\begin{center}\rule{0.5\linewidth}{0.5pt}\end{center}

\section{Problem 17.4.5}\label{problem-17.4.5}

\textbf{Given:} A 400 MHz GPR is used to survey a concrete bridge deck
for rebar corrosion. The concrete has ε\textsubscript{r} = 8 and
attenuation α = 2 dB/m. The rebar is at a depth of 7.5 cm. The system
bandwidth is 800 MHz and dynamic range is 60 dB.

\textbf{Find:} (a) The propagation velocity in concrete, (b) the
round-trip time to the rebar, (c) the two-way attenuation, (d) the
vertical resolution, and (e) the maximum penetration depth.

\textbf{Solution:}

\begin{enumerate}
\def\labelenumi{(\alph{enumi})}
\item
  v = c / √ε\textsubscript{r} = 3 × 10⁸ / √8 = 3 × 10⁸ / 2.828 =
  \textbf{1.061 × 10⁸ m/s} (35.4\% of c)
\item
  t = 2d / v = 2 × 0.075 / 1.061 × 10⁸ = 0.150 / 1.061 × 10⁸ =
  \textbf{1.41 ns}
\item
  Two-way attenuation: L = 2 × α × d = 2 × 2 × 0.075 = \textbf{0.30 dB}
  (negligible)
\item
  Vertical resolution: δ\textsubscript{v} = v / (2B) = 1.061 × 10⁸ / (2
  × 800 × 10⁶) = \textbf{6.63 cm}
\end{enumerate}

This is sufficient to detect rebar (typically 10--16 mm diameter) and
distinguish the rebar layer from the concrete surface.

\begin{enumerate}
\def\labelenumi{(\alph{enumi})}
\setcounter{enumi}{4}
\tightlist
\item
  Maximum penetration depth: d\textsubscript{max} = dynamic range / (2α)
  = 60 / (2 × 2) = \textbf{15 m}
\end{enumerate}

In practice, scattering from aggregates in concrete limits the useful
depth to approximately 0.3--0.5 m at 400 MHz.

\begin{center}\rule{0.5\linewidth}{0.5pt}\end{center}

\section{Problem 17.4.6}\label{problem-17.4.6}

\textbf{Given:} A 1.5 GHz GPR system is used to locate buried plastic
water pipes in sandy soil (ε\textsubscript{r} = 4, α = 0.5 dB/m). The
pipe is at a depth of 1.0 m. The reflection coefficient between dry sand
and water-filled pipe (ε\textsubscript{r} ≈ 81) is significant.

\textbf{Find:} (a) The propagation velocity, (b) the round-trip time,
(c) the two-way attenuation, (d) the reflection coefficient at the
sand-water interface, and (e) the vertical resolution for a bandwidth of
1.5 GHz.

\textbf{Solution:}

\begin{enumerate}
\def\labelenumi{(\alph{enumi})}
\item
  v = c / √ε\textsubscript{r} = 3 × 10⁸ / √4 = 3 × 10⁸ / 2 = \textbf{1.5
  × 10⁸ m/s}
\item
  t = 2d / v = 2 × 1.0 / 1.5 × 10⁸ = \textbf{13.3 ns}
\item
  L = 2 × α × d = 2 × 0.5 × 1.0 = \textbf{1.0 dB}
\item
  Γ = (√ε\textsubscript{r2} − √ε\textsubscript{r1}) /
  (√ε\textsubscript{r2} + √ε\textsubscript{r1}) = (√81 − √4) / (√81 +
  √4) = (9 − 2) / (9 + 2) = 7/11 = \textbf{0.636}
\end{enumerate}

The reflected power fraction: \textbar Γ\textbar² = 0.404 = \textbf{−3.9
dB}

The strong dielectric contrast makes the water-filled pipe easily
detectable.

\begin{enumerate}
\def\labelenumi{(\alph{enumi})}
\setcounter{enumi}{4}
\tightlist
\item
  δ\textsubscript{v} = v / (2B) = 1.5 × 10⁸ / (2 × 1.5 × 10⁹) =
  \textbf{5.0 cm}
\end{enumerate}

\begin{center}\rule{0.5\linewidth}{0.5pt}\end{center}

\section{Problem 17.4.7}\label{problem-17.4.7}

\textbf{Given:} A ground-based X-band missile defense radar (9.5 GHz)
has an antenna diameter of 10 m, aperture efficiency of 0.60, peak power
of 1 MW, and system noise temperature of 350 K. The noise bandwidth is 2
MHz.

\textbf{Find:} (a) The antenna gain, (b) the maximum detection range for
a ballistic warhead with RCS σ = 0.1 m² requiring SNR = 20 dB, and (c)
the number of coherent integration pulses needed if the single-pulse
range is only 500 km.

\textbf{Solution:}

\begin{enumerate}
\def\labelenumi{(\alph{enumi})}
\tightlist
\item
  λ = 3 × 10⁸ / 9.5 × 10⁹ = 0.03158 m
\end{enumerate}

G = η × (πD/λ)² = 0.60 × (π × 10 / 0.03158)² = 0.60 × (994.7)² = 0.60 ×
989,428 = 593,657

G (dBi) = 10 log₁₀(593,657) = \textbf{57.7 dBi}

\begin{enumerate}
\def\labelenumi{(\alph{enumi})}
\setcounter{enumi}{1}
\tightlist
\item
  S\textsubscript{min} = kT\textsubscript{s}B\textsubscript{n} × SNR =
  1.381 × 10⁻²³ × 350 × 2 × 10⁶ × 100
\end{enumerate}

S\textsubscript{min} = 1.381 × 10⁻²³ × 7 × 10¹⁰ = 9.667 × 10⁻¹³ W

R\textsubscript{max} = (P\textsubscript{t}G²λ²σ /
((4π)³S\textsubscript{min}))\textsuperscript{1/4}

Num = 10⁶ × (593,657)² × 0.03158² × 0.1 = 10⁶ × 3.524 × 10¹¹ × 9.97 ×
10⁻⁴ × 0.1

Num = 10⁶ × 3.514 × 10⁷ = 3.514 × 10¹³

Den = 1,984 × 9.667 × 10⁻¹³ = 1.918 × 10⁻⁹

R\textsubscript{max} = (3.514 × 10¹³ / 1.918 ×
10⁻⁹)\textsuperscript{1/4} = (1.832 × 10²²)\textsuperscript{1/4} =
\textbf{367 km}

\begin{enumerate}
\def\labelenumi{(\alph{enumi})}
\setcounter{enumi}{2}
\tightlist
\item
  To extend range from 500 km (hypothetical) to the needed range,
  consider:
\end{enumerate}

Required range is 367 km for single pulse. If we need 2,000 km:

Range ratio = 2,000 / 367 = 5.45

SNR must increase by 5.45⁴ = 882 (29.5 dB)

For coherent integration: N = 882 → \textbf{N = 882 pulses}

At PRF = 10 kHz, this requires a dwell time of 88.2 ms.

\begin{center}\rule{0.5\linewidth}{0.5pt}\end{center}

\section{Problem 17.4.8}\label{problem-17.4.8}

\textbf{Given:} A 905 nm pulsed LiDAR has a transmit pulse energy of 8
μJ, pulse width of 4 ns, receiver aperture diameter of 30 mm, and
detector NEP of 0.15 nW/√Hz. The detection bandwidth is 250 MHz.

\textbf{Find:} (a) The peak transmit power, (b) the received power from
a road sign (ρ = 0.8, Lambertian reflector) at 200 m, (c) the SNR, and
(d) the maximum detection range for a dark vehicle (ρ = 0.1) at SNR = 5.

\textbf{Solution:}

\begin{enumerate}
\def\labelenumi{(\alph{enumi})}
\item
  P\textsubscript{t} = E / τ = 8 × 10⁻⁶ / 4 × 10⁻⁹ = \textbf{2,000 W} (2
  kW peak)
\item
  A\textsubscript{r} = π(0.015)² = 7.069 × 10⁻⁴ m²
\end{enumerate}

P\textsubscript{r} = P\textsubscript{t} × ρ × A\textsubscript{r} / (πR²)
= 2,000 × 0.8 × 7.069 × 10⁻⁴ / (π × 200²)

P\textsubscript{r} = 1.131 / 125,664 = \textbf{9.00 × 10⁻⁶ W} (9.0 μW)

\begin{enumerate}
\def\labelenumi{(\alph{enumi})}
\setcounter{enumi}{2}
\tightlist
\item
  Noise: N = NEP × √B = 0.15 × 10⁻⁹ × √(250 × 10⁶) = 0.15 × 10⁻⁹ ×
  15,811 = 2.372 × 10⁻⁶ W
\end{enumerate}

SNR = 9.00 × 10⁻⁶ / 2.372 × 10⁻⁶ = \textbf{3.79} (5.8 dB)

\begin{enumerate}
\def\labelenumi{(\alph{enumi})}
\setcounter{enumi}{3}
\tightlist
\item
  For ρ = 0.1: P\textsubscript{r} = P\textsubscript{t} × ρ ×
  A\textsubscript{r} / (πR²) = SNR × N
\end{enumerate}

R\textsubscript{max} = √(P\textsubscript{t} × ρ × A\textsubscript{r} /
(π × SNR × N))

R\textsubscript{max} = √(2,000 × 0.1 × 7.069 × 10⁻⁴ / (π × 5 × 2.372 ×
10⁻⁶))

R\textsubscript{max} = √(0.14138 / 3.727 × 10⁻⁵) = √(3,793) =
\textbf{61.6 m}

Dark vehicles at 200 m would require multi-pulse averaging
(approximately (200/61.6)² ≈ 10.5, so \textasciitilde11 pulses
averaged).

\begin{center}\rule{0.5\linewidth}{0.5pt}\end{center}

\section{Problem 17.4.9}\label{problem-17.4.9}

\textbf{Given:} A 1550 nm FMCW LiDAR system achieves simultaneous range
and velocity measurement. It has a sweep bandwidth of 10 GHz, sweep time
of 100 μs, and transmit power of 20 mW.

\textbf{Find:} (a) The range resolution, (b) the maximum unambiguous
velocity (using the relation v\textsubscript{max} = λ /
(4T\textsubscript{sweep})), (c) the wavelength, and (d) explain two
advantages of 1550 nm over 905 nm for automotive LiDAR.

\textbf{Solution:}

\begin{enumerate}
\def\labelenumi{(\alph{enumi})}
\item
  ΔR = c / (2B) = 3 × 10⁸ / (2 × 10 × 10⁹) = \textbf{1.5 cm}
\item
  λ = c / f.~First find frequency: f = c / λ = 3 × 10⁸ / 1550 × 10⁻⁹ =
  1.935 × 10¹⁴ Hz
\item
  \textbf{λ = 1550 nm = 1.55 × 10⁻⁶ m}
\end{enumerate}

(b continued) For FMCW LiDAR, the Doppler shift is extremely large due
to the short wavelength:

v\textsubscript{max} = λ / (4T\textsubscript{sweep}) = 1.55 × 10⁻⁶ / (4
× 100 × 10⁻⁶) = \textbf{3.875 × 10⁻³ m/s} (extremely small)

This shows that FMCW LiDAR at optical wavelengths requires different
velocity disambiguation approaches than RF FMCW radar. In practice,
velocity is extracted from the Doppler-induced frequency shift
superimposed on the range beat frequency.

\begin{enumerate}
\def\labelenumi{(\alph{enumi})}
\setcounter{enumi}{3}
\tightlist
\item
  \textbf{Advantages of 1550 nm over 905 nm:}
\end{enumerate}

\begin{enumerate}
\def\labelenumi{\arabic{enumi}.}
\item
  \textbf{Eye safety:} The cornea absorbs 1550 nm radiation before it
  reaches the retina, allowing \textasciitilde100× higher transmit power
  within Class 1 eye-safe limits compared to 905 nm, which focuses on
  the retina.
\item
  \textbf{Reduced solar background:} The solar spectral irradiance at
  1550 nm is lower than at 905 nm, improving the signal-to-background
  ratio in daylight operation.
\end{enumerate}

\begin{center}\rule{0.5\linewidth}{0.5pt}\end{center}

\section{Problem 17.4.10}\label{problem-17.4.10}

\textbf{Given:} An airport precision approach radar (PAR) operates at 15
GHz (Ku-band) with a 10 m × 3 m antenna providing azimuth beamwidth of
0.4° and elevation beamwidth of 1.2°.

\textbf{Find:} (a) The wavelength, (b) the azimuth position accuracy at
20 km range (assuming accuracy ≈ beamwidth/10 for good SNR), (c) the
elevation position accuracy, and (d) the glideslope deviation that can
be detected if the standard 3° glideslope requires ±0.3° accuracy.

\textbf{Solution:}

\begin{enumerate}
\def\labelenumi{(\alph{enumi})}
\item
  λ = c / f = 3 × 10⁸ / 15 × 10⁹ = \textbf{0.020 m} (20 mm)
\item
  Azimuth accuracy ≈ θ\textsubscript{az} / 10 = 0.4° / 10 = 0.04°
\end{enumerate}

Position accuracy at 20 km: σ\textsubscript{az} = 20,000 × (0.04 ×
π/180) = 20,000 × 6.98 × 10⁻⁴ = \textbf{14.0 m}

\begin{enumerate}
\def\labelenumi{(\alph{enumi})}
\setcounter{enumi}{2}
\tightlist
\item
  Elevation accuracy ≈ θ\textsubscript{el} / 10 = 1.2° / 10 = 0.12°
\end{enumerate}

Position accuracy at 20 km: σ\textsubscript{el} = 20,000 × (0.12 ×
π/180) = 20,000 × 2.094 × 10⁻³ = \textbf{41.9 m}

\begin{enumerate}
\def\labelenumi{(\alph{enumi})}
\setcounter{enumi}{3}
\tightlist
\item
  The PAR can measure elevation to 0.12° accuracy. Since the required
  accuracy is ±0.3°, the PAR can detect glideslope deviations as small
  as 0.12°, which is \textbf{well within the ±0.3° requirement}. At 20
  km range, 0.3° corresponds to an altitude deviation of 20,000 ×
  tan(0.3°) = 20,000 × 0.00524 = \textbf{105 m} --- the PAR can guide
  aircraft to within approximately 42 m of the correct glideslope
  altitude.
\end{enumerate}

\chapter{Chapter 17 --- Section 17.5: Radar System
Comparison}\label{chapter-17-section-17.5-radar-system-comparison}

Practice problems covering frequency band selection and radar system
trade-offs.

\begin{center}\rule{0.5\linewidth}{0.5pt}\end{center}

\section{Problem 17.5.1}\label{problem-17.5.1}

\textbf{Given:} A coastal surveillance radar must detect small boats
(RCS = 3 m²) at 25 km range with range resolution better than 15 m. The
antenna aperture is limited to 2 m. Three candidate frequencies are
being evaluated: S-band (3 GHz), X-band (9.4 GHz), and Ka-band (35 GHz).

\textbf{Find:} (a) The minimum bandwidth required, (b) the antenna
beamwidth at each band, (c) the two-way rain attenuation at each band
for 6 mm/h rainfall over 25 km (approximate coefficients: S = 0.007
dB/km, X = 0.08 dB/km, Ka = 0.8 dB/km), and (d) the recommended band.

\textbf{Solution:}

\begin{enumerate}
\def\labelenumi{(\alph{enumi})}
\item
  B = c / (2ΔR) = 3 × 10⁸ / (2 × 15) = \textbf{10 MHz}
\item
  Beamwidth θ ≈ 70λ/D:
\end{enumerate}

S-band: λ = 0.10 m, θ = 70 × 0.10 / 2 = \textbf{3.50°}

X-band: λ = 0.0319 m, θ = 70 × 0.0319 / 2 = \textbf{1.12°}

Ka-band: λ = 0.00857 m, θ = 70 × 0.00857 / 2 = \textbf{0.30°}

\begin{enumerate}
\def\labelenumi{(\alph{enumi})}
\setcounter{enumi}{2}
\tightlist
\item
  Two-way rain attenuation (path = 2 × 25 km = 50 km):
\end{enumerate}

S-band: 0.007 × 50 = \textbf{0.35 dB} (negligible)

X-band: 0.08 × 50 = \textbf{4.0 dB} (moderate --- reduces range by
\textasciitilde20\%)

Ka-band: 0.8 × 50 = \textbf{40 dB} (devastating --- effectively blinds
the radar)

\begin{enumerate}
\def\labelenumi{(\alph{enumi})}
\setcounter{enumi}{3}
\tightlist
\item
  \textbf{Recommendation: X-band (9.4 GHz)}. It provides the best
  compromise: the 1.12° beamwidth gives good angular resolution for
  tracking small boats, the 4 dB rain attenuation is acceptable (can be
  compensated with a few dB of additional power margin), and the 10 MHz
  bandwidth is easily achievable with standard magnetron or solid-state
  transmitters. S-band has too-wide a beam (3.5°), and Ka-band is
  impractical in rain.
\end{enumerate}

\begin{center}\rule{0.5\linewidth}{0.5pt}\end{center}

\section{Problem 17.5.2}\label{problem-17.5.2}

\textbf{Given:} A drone detection radar must detect small drones (RCS =
0.01 m²) at 5 km range. Three candidate architectures are: (a) S-band
pulsed, (b) X-band pulsed, and (c) Ku-band FMCW. All have a 1 m antenna
aperture.

\textbf{Find:} For each option: the beamwidth, the required transmit
power (assuming G = 30 dBi, T\textsubscript{sys} = 500 K,
B\textsubscript{n} = 1 MHz, SNR\textsubscript{req} = 15 dB), and the
feasibility assessment.

\textbf{Solution:}

Common parameters: σ = 0.01 m², R = 5 km = 5,000 m, G = 10³ (30 dBi),
SNR = 31.62

S\textsubscript{min} = kT\textsubscript{s}B\textsubscript{n} × SNR =
1.381 × 10⁻²³ × 500 × 10⁶ × 31.62 = 2.184 × 10⁻¹³ W

From the radar equation: P\textsubscript{t} = R⁴ × (4π)³ ×
S\textsubscript{min} / (G²λ²σ)

Denominator (common): G²σ = (10³)² × 0.01 = 10⁴

R⁴ = (5 × 10³)⁴ = 6.25 × 10¹⁴

Numerator = 6.25 × 10¹⁴ × 1,984 × 2.184 × 10⁻¹³ = 6.25 × 10¹⁴ × 4.333 ×
10⁻¹⁰ = 2.708 × 10⁵

\textbf{(a) S-band (3 GHz):}

λ = 0.10 m, θ = 70 × 0.10 / 1 = \textbf{7.0°} (poor angular resolution)

P\textsubscript{t} = 2.708 × 10⁵ / (10⁴ × 0.01) = 2.708 × 10⁵ / 100 =
\textbf{2,708 W} (2.7 kW peak)

Assessment: The 7° beam provides poor angular accuracy and wide clutter
cell. The power is feasible. \textbf{Marginal} --- angular resolution is
inadequate for tracking small drones.

\textbf{(b) X-band (10 GHz):}

λ = 0.03 m, θ = 70 × 0.03 / 1 = \textbf{2.1°} (good angular resolution)

P\textsubscript{t} = 2.708 × 10⁵ / (10⁴ × 9 × 10⁻⁴) = 2.708 × 10⁵ / 9 =
\textbf{30,089 W} (30 kW peak)

Assessment: Good angular resolution, but 30 kW peak power is high for a
compact system. Pulse compression can reduce the required peak power.
\textbf{Good choice with pulse compression.}

\textbf{(c) Ku-band (15 GHz):}

λ = 0.02 m, θ = 70 × 0.02 / 1 = \textbf{1.4°} (excellent angular
resolution)

P\textsubscript{t} = 2.708 × 10⁵ / (10⁴ × 4 × 10⁻⁴) = 2.708 × 10⁵ / 4 =
\textbf{67,700 W} (67.7 kW peak)

Assessment: The λ² term in the radar equation means higher frequencies
require more power for the same range. However, FMCW with coherent
integration can achieve equivalent performance with much lower average
power. \textbf{Feasible with FMCW architecture}, but rain attenuation at
Ku-band (0.15 dB/km × 10 km two-way ≈ 1.5 dB) adds a modest penalty.

\textbf{Best choice: X-band pulsed with pulse compression} --- balances
angular resolution, transmit power, and all-weather capability.

\begin{center}\rule{0.5\linewidth}{0.5pt}\end{center}

\section{Problem 17.5.3}\label{problem-17.5.3}

\textbf{Given:} Compare the peak power required to detect a target with
σ = 1 m² at 100 km range at four different frequency bands, assuming the
same antenna area (A = 3 m²), T\textsubscript{sys} = 400 K,
B\textsubscript{n} = 1 MHz, and required SNR = 13 dB.

\textbf{Find:} The peak transmit power required at (a) L-band (1.3 GHz),
(b) S-band (3 GHz), (c) X-band (10 GHz), and (d) Ka-band (35 GHz).
Explain the trend.

\textbf{Solution:}

For a constant antenna area A, the gain is G = 4πAη/λ² (where η = 0.6):

G = 4π × 3 × 0.6 / λ² = 22.62 / λ²

S\textsubscript{min} = kT\textsubscript{s}B\textsubscript{n} × SNR =
1.381 × 10⁻²³ × 400 × 10⁶ × 10\textsuperscript{1.3} = 1.381 × 10⁻²³ ×
400 × 10⁶ × 19.95

S\textsubscript{min} = 1.102 × 10⁻¹³ W

P\textsubscript{t} = R⁴(4π)³S\textsubscript{min} / (G²λ²σ)

Since G² = (22.62/λ²)² = 511.7/λ⁴:

P\textsubscript{t} = R⁴(4π)³S\textsubscript{min} / (511.7 × λ²/λ⁴ × σ) =
R⁴(4π)³S\textsubscript{min}λ² / (511.7σ)

So P\textsubscript{t} ∝ λ² for constant antenna area --- higher
frequencies (smaller λ) require less power.

Common factor: R⁴(4π)³S\textsubscript{min} / (511.7 × 1) = (10⁵)⁴ ×
1,984 × 1.102 × 10⁻¹³ / 511.7

= 10²⁰ × 2.186 × 10⁻¹⁰ / 511.7 = 2.186 × 10¹⁰ / 511.7 = 4.273 × 10⁷

\textbf{(a) L-band:} λ = 0.2308 m. P\textsubscript{t} = 4.273 × 10⁷ ×
0.2308² = 4.273 × 10⁷ × 0.05327 = \textbf{2.28 MW}

G = 22.62 / 0.2308² = 424.6 (26.3 dBi)

\textbf{(b) S-band:} λ = 0.10 m. P\textsubscript{t} = 4.273 × 10⁷ × 0.01
= \textbf{427 kW}

G = 22.62 / 0.01 = 2,262 (33.5 dBi)

\textbf{(c) X-band:} λ = 0.03 m. P\textsubscript{t} = 4.273 × 10⁷ × 9 ×
10⁻⁴ = \textbf{38.5 kW}

G = 22.62 / 9 × 10⁻⁴ = 25,133 (44.0 dBi)

\textbf{(d) Ka-band:} λ = 0.00857 m. P\textsubscript{t} = 4.273 × 10⁷ ×
7.34 × 10⁻⁵ = \textbf{3.14 kW}

G = 22.62 / 7.34 × 10⁻⁵ = 308,174 (54.9 dBi)

\textbf{Trend:} For constant antenna area, higher frequencies require
dramatically less transmit power because the antenna gain increases as
(1/λ)². The Ka-band radar needs only 3.14 kW versus 2.28 MW at L-band
--- a 730× reduction. However, this advantage is offset by (1) rain
attenuation at higher frequencies, (2) more expensive/complex
high-frequency transmitters, and (3) narrower beams that may require
electronic scanning.

\chapter{Chapter 18 --- Section 18.1: Geometric
Optics}\label{chapter-18-section-18.1-geometric-optics}

Practice problems covering reflection and refraction (Snell's law),
lenses and imaging, mirrors and curved surfaces, and prisms and
dispersive elements.

\begin{center}\rule{0.5\linewidth}{0.5pt}\end{center}

\section{Problem 18.1.1}\label{problem-18.1.1}

\textbf{Given:} A light ray travels from water (n₁ = 1.333) into a
diamond (n₂ = 2.417) at an angle of incidence of 25° from the normal.

\textbf{Find:} (a) The refraction angle in diamond, (b) the critical
angle for total internal reflection at the diamond-water interface
(light going from diamond to water), and (c) whether total internal
reflection is possible when light passes from water into diamond.

\textbf{Solution:}

\begin{enumerate}
\def\labelenumi{(\alph{enumi})}
\tightlist
\item
  Snell's law: n₁ sin(θ₁) = n₂ sin(θ₂)
\end{enumerate}

1.333 × sin(25°) = 2.417 × sin(θ₂)

sin(θ₂) = 1.333 × 0.4226 / 2.417 = 0.5633 / 2.417 = 0.2331

θ₂ = arcsin(0.2331) = \textbf{13.5°}

The ray bends toward the normal when entering the denser medium.

\begin{enumerate}
\def\labelenumi{(\alph{enumi})}
\setcounter{enumi}{1}
\tightlist
\item
  Critical angle (diamond to water): θ\textsubscript{c} =
  arcsin(n\textsubscript{water} / n\textsubscript{diamond}) =
  arcsin(1.333 / 2.417)
\end{enumerate}

θ\textsubscript{c} = arcsin(0.5517) = \textbf{33.5°}

\begin{enumerate}
\def\labelenumi{(\alph{enumi})}
\setcounter{enumi}{2}
\tightlist
\item
  \textbf{No}, total internal reflection cannot occur when light passes
  from a less dense medium (water) into a denser medium (diamond). TIR
  only occurs when light travels from a higher-index medium to a
  lower-index medium and the angle of incidence exceeds the critical
  angle.
\end{enumerate}

\begin{center}\rule{0.5\linewidth}{0.5pt}\end{center}

\section{Problem 18.1.2}\label{problem-18.1.2}

\textbf{Given:} An optical fiber has a glass core with n₁ = 1.48 and a
cladding with n₂ = 1.46.

\textbf{Find:} (a) The critical angle for total internal reflection at
the core-cladding boundary, (b) the numerical aperture, (c) the maximum
acceptance half-angle for light entering the fiber from air, and (d) the
critical angle if the cladding is replaced with n₂ = 1.44.

\textbf{Solution:}

\begin{enumerate}
\def\labelenumi{(\alph{enumi})}
\tightlist
\item
  θ\textsubscript{c} = arcsin(n₂ / n₁) = arcsin(1.46 / 1.48) =
  arcsin(0.9865) = \textbf{80.6°}
\end{enumerate}

Light must strike the core-cladding interface at angles greater than
80.6° (measured from normal) for total internal reflection --- meaning
the ray must be nearly grazing along the fiber axis.

\begin{enumerate}
\def\labelenumi{(\alph{enumi})}
\setcounter{enumi}{1}
\item
  NA = √(n₁² − n₂²) = √(1.48² − 1.46²) = √(2.1904 − 2.1316) = √(0.0588)
  = \textbf{0.2425}
\item
  θ\textsubscript{max} = arcsin(NA) = arcsin(0.2425) = \textbf{14.0°}
\item
  With n₂ = 1.44: θ\textsubscript{c} = arcsin(1.44 / 1.48) =
  arcsin(0.9730) = \textbf{76.7°}
\end{enumerate}

NA = √(1.48² − 1.44²) = √(2.1904 − 2.0736) = √(0.1168) = \textbf{0.342}

The larger index difference provides a wider acceptance angle (20.0°)
but supports more modes in multimode fiber.

\begin{center}\rule{0.5\linewidth}{0.5pt}\end{center}

\section{Problem 18.1.3}\label{problem-18.1.3}

\textbf{Given:} A converging lens with focal length f = 75 mm is used to
image an object at three different distances: (a) d\textsubscript{o} =
300 mm, (b) d\textsubscript{o} = 150 mm, and (c) d\textsubscript{o} = 50
mm.

\textbf{Find:} For each case, the image distance, magnification, and
whether the image is real/virtual and upright/inverted.

\textbf{Solution:}

\textbf{(a) d\textsubscript{o} = 300 mm} (object beyond 2f):

1/d\textsubscript{i} = 1/f − 1/d\textsubscript{o} = 1/75 − 1/300 = (4 −
1)/300 = 3/300

d\textsubscript{i} = \textbf{100 mm} (positive → real image)

m = −d\textsubscript{i}/d\textsubscript{o} = −100/300 = \textbf{−0.333}
(inverted, reduced to 1/3 size)

\textbf{(b) d\textsubscript{o} = 150 mm} (object at 2f):

1/d\textsubscript{i} = 1/75 − 1/150 = (2 − 1)/150 = 1/150

d\textsubscript{i} = \textbf{150 mm} (positive → real image)

m = −150/150 = \textbf{−1.0} (inverted, same size --- the symmetric 2f
configuration)

\textbf{(c) d\textsubscript{o} = 50 mm} (object inside f):

1/d\textsubscript{i} = 1/75 − 1/50 = (2 − 3)/150 = −1/150

d\textsubscript{i} = \textbf{−150 mm} (negative → virtual image, on same
side as object)

m = −(−150)/50 = \textbf{+3.0} (upright, magnified 3×)

This is the magnifying glass configuration --- the object inside the
focal length produces a virtual, upright, enlarged image.

\begin{center}\rule{0.5\linewidth}{0.5pt}\end{center}

\section{Problem 18.1.4}\label{problem-18.1.4}

\textbf{Given:} A two-lens optical system consists of a converging lens
(f₁ = 100 mm) and a diverging lens (f₂ = −50 mm) separated by 80 mm.

\textbf{Find:} (a) The combined focal length of the system, (b) the
image distance for an object at infinity, and (c) whether the system is
a telephoto or reverse-telephoto configuration.

\textbf{Solution:}

\begin{enumerate}
\def\labelenumi{(\alph{enumi})}
\tightlist
\item
  Combined focal length: 1/f\textsubscript{total} = 1/f₁ + 1/f₂ − d/(f₁
  × f₂)
\end{enumerate}

1/f\textsubscript{total} = 1/100 + 1/(−50) − 80/(100 × (−50))

1/f\textsubscript{total} = 0.010 − 0.020 + 80/5,000

1/f\textsubscript{total} = 0.010 − 0.020 + 0.016 = 0.006

f\textsubscript{total} = 1/0.006 = \textbf{166.7 mm}

\begin{enumerate}
\def\labelenumi{(\alph{enumi})}
\setcounter{enumi}{1}
\tightlist
\item
  For an object at infinity, the first lens forms an intermediate image
  at f₁ = 100 mm from the first lens. The second lens is 80 mm from the
  first lens, so the intermediate image falls 100 − 80 = 20 mm past the
  second lens. This is a virtual object for the second lens (light is
  converging toward a point on the output side of L₂), so
  d\textsubscript{o2} = −20 mm:
\end{enumerate}

1/d\textsubscript{i2} = 1/f₂ − 1/d\textsubscript{o2} = 1/(−50) − 1/(−20)
= −0.020 + 0.050 = +0.030

d\textsubscript{i2} = +33.3 mm (real image, 33.3 mm past the second
lens)

The image is \textbf{33.3 mm past the second lens} (or 80 + 33.3 = 113.3
mm past the first lens). This can also be confirmed by ray tracing: a
ray at height h = 1 mm parallel to the axis arrives at L₂ with slope u =
−h/f₁ = −0.010 rad; after L₂ the slope is u′ = u − h′/f₂ = −0.010 −
0.2/(−50) = −0.006 rad where h′ = 1 − 80 × 0.01 = 0.2 mm; the ray
crosses the axis at 0.2/0.006 = 33.3 mm past L₂.

\begin{enumerate}
\def\labelenumi{(\alph{enumi})}
\setcounter{enumi}{2}
\tightlist
\item
  The combined focal length (166.7 mm) is longer than the physical
  system length (80 mm + back focal distance). This is a
  \textbf{telephoto configuration} --- the positive lens followed by a
  negative lens produces a long effective focal length in a compact
  package, as used in camera telephoto lenses.
\end{enumerate}

\begin{center}\rule{0.5\linewidth}{0.5pt}\end{center}

\section{Problem 18.1.5}\label{problem-18.1.5}

\textbf{Given:} A concave mirror with radius of curvature R = 60 cm is
used in a reflecting telescope. A star (effectively at infinity) is
observed, and a nearby planet is at an object distance of 10 m.

\textbf{Find:} (a) The focal length, (b) the image distance for the
star, (c) the image distance for the planet, and (d) the magnification
for the planet.

\textbf{Solution:}

\begin{enumerate}
\def\labelenumi{(\alph{enumi})}
\item
  f = R/2 = 60/2 = \textbf{30 cm}
\item
  For an object at infinity: d\textsubscript{i} = f = \textbf{30 cm}
  (image forms at the focal point)
\item
  1/d\textsubscript{i} = 1/f − 1/d\textsubscript{o} = 1/30 − 1/1,000 =
  (33.33 − 1)/1,000 = 32.33/1,000
\end{enumerate}

d\textsubscript{i} = 1,000/32.33 = \textbf{30.93 cm}

\begin{enumerate}
\def\labelenumi{(\alph{enumi})}
\setcounter{enumi}{3}
\tightlist
\item
  m = −d\textsubscript{i}/d\textsubscript{o} = −30.93/1,000 =
  \textbf{−0.0309} (inverted, greatly reduced)
\end{enumerate}

The planet image is only 0.93 cm farther from the mirror than the star
image --- illustrating why astronomical telescopes have very small
depth-of-focus issues for celestial objects.

\begin{center}\rule{0.5\linewidth}{0.5pt}\end{center}

\section{Problem 18.1.6}\label{problem-18.1.6}

\textbf{Given:} A spectrometer uses a diffraction grating with 1200
grooves/mm. Light from a hydrogen lamp containing the Balmer series is
analyzed in first order.

\textbf{Find:} (a) The grating spacing, (b) the diffraction angle for
H-α (656.3 nm), H-β (486.1 nm), and H-γ (434.0 nm), and (c) the
resolving power needed to separate the H-α line from a nearby
atmospheric oxygen line at 656.7 nm.

\textbf{Solution:}

\begin{enumerate}
\def\labelenumi{(\alph{enumi})}
\item
  d = 1/1200 mm = \textbf{833.3 nm} (8.333 × 10⁻⁷ m)
\item
  sin(θ) = mλ/d = λ/d (first order, m = 1):
\end{enumerate}

H-α: sin(θ) = 656.3/833.3 = 0.7876, θ = arcsin(0.7876) = \textbf{51.97°}

H-β: sin(θ) = 486.1/833.3 = 0.5833, θ = arcsin(0.5833) = \textbf{35.69°}

H-γ: sin(θ) = 434.0/833.3 = 0.5208, θ = arcsin(0.5208) = \textbf{31.38°}

\begin{enumerate}
\def\labelenumi{(\alph{enumi})}
\setcounter{enumi}{2}
\tightlist
\item
  Δλ = 656.7 − 656.3 = 0.4 nm
\end{enumerate}

R = λ/Δλ = 656.3/0.4 = \textbf{1,641}

Required number of grooves: N = R/m = 1,641/1 = 1,641 grooves

At 1200 grooves/mm, the minimum illuminated width is 1,641/1,200 =
\textbf{1.37 mm} --- a very small grating section suffices.

\begin{center}\rule{0.5\linewidth}{0.5pt}\end{center}

\section{Problem 18.1.7}\label{problem-18.1.7}

\textbf{Given:} A prism spectrometer uses an equilateral prism (apex
angle α = 60°) made of SF11 flint glass. At minimum deviation, the
refractive indices are: n(486 nm) = 1.806, n(546 nm) = 1.785, n(656 nm)
= 1.769.

\textbf{Find:} (a) The minimum deviation angle at 546 nm, (b) the
angular dispersion between 486 nm and 656 nm, and (c) the minimum
deviation at 486 nm and 656 nm.

\textbf{Solution:}

\begin{enumerate}
\def\labelenumi{(\alph{enumi})}
\tightlist
\item
  δ\textsubscript{min} = 2 arcsin(n sin(α/2)) − α = 2 arcsin(1.785 ×
  sin(30°)) − 60°
\end{enumerate}

δ\textsubscript{min} = 2 arcsin(1.785 × 0.500) − 60° = 2 arcsin(0.8925)
− 60°

δ\textsubscript{min} = 2 × 63.18° − 60° = 126.36° − 60° =
\textbf{66.36°}

\begin{enumerate}
\def\labelenumi{(\alph{enumi})}
\setcounter{enumi}{1}
\tightlist
\item
  δ\textsubscript{min}(486 nm) = 2 arcsin(1.806 × 0.500) − 60° = 2
  arcsin(0.903) − 60° = 2 × 64.55° − 60° = \textbf{69.10°}
\end{enumerate}

δ\textsubscript{min}(656 nm) = 2 arcsin(1.769 × 0.500) − 60° = 2
arcsin(0.8845) − 60° = 2 × 62.19° − 60° = \textbf{64.38°}

Angular dispersion = 69.10° − 64.38° = \textbf{4.72°}

\begin{enumerate}
\def\labelenumi{(\alph{enumi})}
\setcounter{enumi}{2}
\tightlist
\item
  Already computed above:
\end{enumerate}

At 486 nm (F line): \textbf{δ\textsubscript{min} = 69.10°}

At 656 nm (C line): \textbf{δ\textsubscript{min} = 64.38°}

The high dispersion of SF11 flint glass (Δn = 0.037 over the visible
range) produces nearly 5° of angular spread, making it effective for
spectroscopy applications.

\begin{center}\rule{0.5\linewidth}{0.5pt}\end{center}

\section{Problem 18.1.8}\label{problem-18.1.8}

\textbf{Given:} A camera lens has an f-number of f/2.8 with a focal
length of 50 mm.

\textbf{Find:} (a) The aperture diameter, (b) the diffraction-limited
Airy disk diameter at λ = 550 nm, (c) the f-number needed to achieve a
10 μm Airy disk (matching a typical pixel size), and (d) the depth of
field for an object at 2 m (approximate formula: DOF ≈ 2f²N × c / (f²)
where c is the circle of confusion ≈ 20 μm and N is the f-number;
simplified DOF ≈ 2Nc × d\textsubscript{o}² / f²).

\textbf{Solution:}

\begin{enumerate}
\def\labelenumi{(\alph{enumi})}
\item
  D = f / (f/\#) = 50 / 2.8 = \textbf{17.9 mm}
\item
  d\textsubscript{Airy} = 2.44 × λ × (f/\#) = 2.44 × 550 × 10⁻⁶ × 2.8 =
  2.44 × 0.00055 × 2.8 = \textbf{3.76 μm}
\item
  d\textsubscript{Airy} = 2.44λ(f/\#), so f/\# = d\textsubscript{Airy} /
  (2.44λ) = 10 × 10⁻⁶ / (2.44 × 550 × 10⁻⁹) = 10 / 1.342 =
  \textbf{f/7.5}
\item
  DOF ≈ 2Nc × d\textsubscript{o}² / f² = 2 × 2.8 × 0.020 × 2000² / 50²
\end{enumerate}

DOF = 2 × 2.8 × 0.020 × 4 × 10⁶ / 2500 = 0.112 × 1,600 = \textbf{179 mm}
(about 18 cm)

At f/2.8 and 2 m focus distance, the depth of field is approximately
\textbf{18 cm} --- objects from about 1.91 m to 2.09 m will appear
sharp.

\chapter{Chapter 18 --- Section 18.2: Wave
Optics}\label{chapter-18-section-18.2-wave-optics}

Practice problems covering interference, diffraction, polarization, and
thin-film optical coatings.

\begin{center}\rule{0.5\linewidth}{0.5pt}\end{center}

\section{Problem 18.2.1}\label{problem-18.2.1}

\textbf{Given:} In a Young's double-slit experiment, two slits separated
by d = 0.25 mm are illuminated by a monochromatic laser at λ = 632.8 nm
(HeNe). The interference pattern is observed on a screen L = 1.5 m from
the slits.

\textbf{Find:} (a) The fringe spacing on the screen, (b) the angle to
the third-order bright fringe, and (c) the total number of bright
fringes that fit within a ±30° angular range.

\textbf{Solution:}

\begin{enumerate}
\def\labelenumi{(\alph{enumi})}
\item
  Fringe spacing: Δy = λL/d = 632.8 × 10⁻⁹ × 1.5 / (0.25 × 10⁻³) = 9.492
  × 10⁻⁷ / 2.5 × 10⁻⁴ = \textbf{3.797 mm}
\item
  Angle to third-order bright fringe: d sin(θ) = mλ, so sin(θ₃) = 3 ×
  632.8 × 10⁻⁹ / (0.25 × 10⁻³) = 1.8984 × 10⁻⁶ / 2.5 × 10⁻⁴ = 7.594 ×
  10⁻³ θ₃ = arcsin(0.007594) = \textbf{0.435°}
\item
  Maximum order at θ = 30°: m\textsubscript{max} = d sin(30°) / λ = 0.25
  × 10⁻³ × 0.5 / 632.8 × 10⁻⁹ = 1.25 × 10⁻⁴ / 6.328 × 10⁻⁷ = 197.5
  Maximum integer order m = 197. Total bright fringes from m = −197 to m
  = +197 plus the central maximum: N = 2 × 197 + 1 = \textbf{395 bright
  fringes}
\end{enumerate}

\begin{center}\rule{0.5\linewidth}{0.5pt}\end{center}

\section{Problem 18.2.2}\label{problem-18.2.2}

\textbf{Given:} A single slit of width a = 50 μm is illuminated by light
at λ = 500 nm. The diffraction pattern is observed on a screen at
distance L = 2 m.

\textbf{Find:} (a) The angular width of the central maximum, (b) the
linear width of the central maximum on the screen, and (c) the angle to
the second-order minimum.

\textbf{Solution:}

\begin{enumerate}
\def\labelenumi{(\alph{enumi})}
\item
  The first minima occur at a sin(θ) = ±λ, so sin(θ₁) = λ/a = 500 × 10⁻⁹
  / 50 × 10⁻⁶ = 0.01. θ₁ = 0.573°. Angular width of central maximum =
  2θ₁ = \textbf{1.146°}
\item
  Linear half-width: y₁ = L tan(θ₁) ≈ L × sin(θ₁) = 2 × 0.01 = 0.02 m =
  20 mm. Full width of central maximum = 2 × 20 = \textbf{40 mm}
\item
  Second-order minimum: a sin(θ₂) = 2λ. sin(θ₂) = 2 × 500 × 10⁻⁹ / 50 ×
  10⁻⁶ = 0.02. θ₂ = arcsin(0.02) = \textbf{1.146°}
\end{enumerate}

\begin{center}\rule{0.5\linewidth}{0.5pt}\end{center}

\section{Problem 18.2.3}\label{problem-18.2.3}

\textbf{Given:} A telescope has an objective aperture diameter D = 100
mm and operates at λ = 600 nm. It is used to observe two point sources.

\textbf{Find:} (a) The diffraction-limited angular resolution (Rayleigh
criterion), (b) the minimum separation of two objects that can be
resolved at a distance of 10 km, and (c) the angular resolution if the
aperture is reduced to D = 25 mm by a stop.

\textbf{Solution:}

\begin{enumerate}
\def\labelenumi{(\alph{enumi})}
\tightlist
\item
  Rayleigh criterion: θ\textsubscript{min} = 1.22λ/D = 1.22 × 600 × 10⁻⁹
  / 0.100 = 7.32 × 10⁻⁶ rad = \textbf{7.32 μrad}
\end{enumerate}

Converting: θ\textsubscript{min} = 7.32 × 10⁻⁶ × 206,265 = \textbf{1.51
arc-seconds}

\begin{enumerate}
\def\labelenumi{(\alph{enumi})}
\setcounter{enumi}{1}
\item
  Minimum resolvable separation at 10 km: s = θ\textsubscript{min} × d =
  7.32 × 10⁻⁶ × 10,000 = \textbf{0.0732 m = 73.2 mm}
\item
  With D = 25 mm: θ\textsubscript{min} = 1.22 × 600 × 10⁻⁹ / 0.025 =
  2.928 × 10⁻⁵ rad = \textbf{29.3 μrad}
\end{enumerate}

The resolution degrades by a factor of 4 (proportional to the aperture
reduction), demonstrating the direct relationship between aperture size
and resolving power.

\begin{center}\rule{0.5\linewidth}{0.5pt}\end{center}

\section{Problem 18.2.4}\label{problem-18.2.4}

\textbf{Given:} Unpolarized light at intensity I₀ = 200 mW/cm² passes
through three linear polarizers in series. The first has its
transmission axis at 0° (vertical), the second at 30° from vertical, and
the third at 75° from vertical.

\textbf{Find:} (a) The intensity after each polarizer and (b) the
overall transmission ratio.

\textbf{Solution:}

\begin{enumerate}
\def\labelenumi{(\alph{enumi})}
\tightlist
\item
  After polarizer 1 (unpolarized → linear): I₁ = I₀/2 = 200/2 =
  \textbf{100 mW/cm²}
\end{enumerate}

After polarizer 2 (Malus's law, angle between axes = 30°): I₂ = I₁
cos²(30°) = 100 × (√3/2)² = 100 × 0.75 = \textbf{75 mW/cm²}

After polarizer 3 (angle between polarizer 2 and 3 = 75° − 30° = 45°):
I₃ = I₂ cos²(45°) = 75 × (√2/2)² = 75 × 0.5 = \textbf{37.5 mW/cm²}

\begin{enumerate}
\def\labelenumi{(\alph{enumi})}
\setcounter{enumi}{1}
\tightlist
\item
  Overall transmission: T = I₃/I₀ = 37.5/200 = \textbf{18.75\%}
\end{enumerate}

Without the middle polarizer, the angle between polarizers 1 and 3 would
be 75°, giving I = 100 × cos²(75°) = 100 × 0.0670 = 6.70 mW/cm² --- only
3.35\%. The intermediate polarizer increases transmission by rotating
the polarization in stages.

\begin{center}\rule{0.5\linewidth}{0.5pt}\end{center}

\section{Problem 18.2.5}\label{problem-18.2.5}

\textbf{Given:} A thin anti-reflection coating of SiO₂
(n\textsubscript{film} = 1.46) is applied to a silicon solar cell
(n\textsubscript{substrate} = 3.5) to minimize reflection at λ = 600 nm.

\textbf{Find:} (a) The quarter-wave coating thickness, (b) the residual
reflectance with the coating, and (c) the uncoated reflectance for
comparison.

\textbf{Solution:}

\begin{enumerate}
\def\labelenumi{(\alph{enumi})}
\item
  Quarter-wave thickness: t = λ / (4n\textsubscript{film}) = 600 / (4 ×
  1.46) = 600 / 5.84 = \textbf{102.7 nm}
\item
  Residual reflectance with coating: R\textsubscript{coated} =
  {[}(n\textsubscript{film}² − n₁ × n₂) / (n\textsubscript{film}² + n₁ ×
  n₂){]}² = {[}(1.46² − 1.00 × 3.5) / (1.46² + 1.00 × 3.5){]}² =
  {[}(2.1316 − 3.5) / (2.1316 + 3.5){]}² = {[}(−1.3684) / (5.6316){]}² =
  (0.2430)² = \textbf{5.90\%}
\item
  Uncoated reflectance (Fresnel, normal incidence):
  R\textsubscript{uncoated} = {[}(n₂ − n₁) / (n₂ + n₁){]}² = {[}(3.5 −
  1.0) / (3.5 + 1.0){]}² = (2.5/4.5)² = (0.5556)² = \textbf{30.86\%}
\end{enumerate}

The SiO₂ coating reduces reflectance from 30.86\% to 5.90\%. The ideal
coating index would be n\textsubscript{ideal} = √(1.0 × 3.5) = 1.871, so
Si₃N₄ (n ≈ 2.0) would be a better match for silicon solar cells.

\begin{center}\rule{0.5\linewidth}{0.5pt}\end{center}

\section{Problem 18.2.6}\label{problem-18.2.6}

\textbf{Given:} A Fabry-Perot interferometer has mirror reflectivity R =
0.95, mirror separation d = 5 mm, and operates at λ = 633 nm in air (n =
1).

\textbf{Find:} (a) The free spectral range (FSR) in frequency, (b) the
finesse, and (c) the minimum resolvable frequency difference.

\textbf{Solution:}

\begin{enumerate}
\def\labelenumi{(\alph{enumi})}
\item
  Free spectral range: FSR = c / (2nd) = 3 × 10⁸ / (2 × 1 × 5 × 10⁻³) =
  3 × 10⁸ / 0.01 = \textbf{30 GHz}
\item
  Finesse: F = π√R / (1 − R) = π × √0.95 / (1 − 0.95) = π × 0.9747 /
  0.05 = 3.062 / 0.05 = \textbf{61.2}
\item
  Minimum resolvable frequency difference: δf = FSR / F = 30 × 10⁹ /
  61.2 = \textbf{490 MHz}
\end{enumerate}

In wavelength terms: δλ = λ² × δf / c = (633 × 10⁻⁹)² × 4.90 × 10⁸ / (3
× 10⁸) = 6.54 × 10⁻⁴ nm = 0.654 pm. This is sufficient to resolve the
hyperfine structure of many atomic spectral lines.

\begin{center}\rule{0.5\linewidth}{0.5pt}\end{center}

\section{Problem 18.2.7}\label{problem-18.2.7}

\textbf{Given:} A diffraction grating with 1200 grooves/mm and 30 mm
illuminated width is used to analyze the hydrogen-alpha spectral line at
λ = 656.28 nm in first order.

\textbf{Find:} (a) The grating spacing, (b) the diffraction angle, (c)
the resolving power, and (d) the minimum wavelength difference that can
be resolved.

\textbf{Solution:}

\begin{enumerate}
\def\labelenumi{(\alph{enumi})}
\item
  Grating spacing: d = 1/1200 mm = 0.8333 μm = \textbf{833.3 nm}
\item
  Diffraction angle for first order: sin(θ₁) = mλ/d = 1 × 656.28 × 10⁻⁹
  / 833.3 × 10⁻⁹ = 0.7875 θ₁ = arcsin(0.7875) = \textbf{51.95°}
\item
  Total grooves: N = 1200 × 30 = 36,000. Resolving power: R = mN = 1 ×
  36,000 = \textbf{36,000}
\item
  Minimum resolvable wavelength difference: Δλ\textsubscript{min} = λ/R
  = 656.28 / 36,000 = \textbf{0.01823 nm}
\end{enumerate}

This resolving power easily separates the deuterium-alpha line (λ =
656.10 nm) from the hydrogen-alpha line, since Δλ = 0.18 nm
\textgreater\textgreater{} 0.018 nm.

\begin{center}\rule{0.5\linewidth}{0.5pt}\end{center}

\section{Problem 18.2.8}\label{problem-18.2.8}

\textbf{Given:} Light reflects from a glass surface (n = 1.52) at an
unknown angle. The reflected beam is found to be completely linearly
polarized.

\textbf{Find:} (a) Brewster's angle, (b) the refraction angle of the
transmitted ray, and (c) the reflectance for s-polarized light at this
angle.

\textbf{Solution:}

\begin{enumerate}
\def\labelenumi{(\alph{enumi})}
\item
  Brewster's angle: θ\textsubscript{B} = arctan(n₂/n₁) =
  arctan(1.52/1.00) = arctan(1.52) = \textbf{56.66°}
\item
  At Brewster's angle, the reflected and refracted rays are
  perpendicular: θ\textsubscript{r} + θ\textsubscript{t} = 90°, so
  θ\textsubscript{t} = 90° − 56.66° = \textbf{33.34°}
\end{enumerate}

Verification with Snell's law: n₁ sin(θ\textsubscript{B}) = n₂
sin(θ\textsubscript{t}) 1.00 × sin(56.66°) = 1.52 × sin(33.34°) 0.8355 =
1.52 × 0.5497 = 0.8355 ✓

\begin{enumerate}
\def\labelenumi{(\alph{enumi})}
\setcounter{enumi}{2}
\tightlist
\item
  Reflectance for s-polarized light (Fresnel equation):
  R\textsubscript{s} = {[}(n₁ cos θ\textsubscript{i} − n₂ cos
  θ\textsubscript{t}) / (n₁ cos θ\textsubscript{i} + n₂ cos
  θ\textsubscript{t}){]}² = {[}(1.00 × cos 56.66° − 1.52 × cos 33.34°) /
  (1.00 × cos 56.66° + 1.52 × cos 33.34°){]}² = {[}(0.5497 − 1.52 ×
  0.8355) / (0.5497 + 1.52 × 0.8355){]}² = {[}(0.5497 − 1.2700) /
  (0.5497 + 1.2700){]}² = {[}(−0.7203) / (1.8197){]}² = (0.3958)² =
  \textbf{15.67\%}
\end{enumerate}

At Brewster's angle, p-polarized reflectance is exactly 0\% while
s-polarized reflectance is about 15.67\%.

\begin{center}\rule{0.5\linewidth}{0.5pt}\end{center}

\section{Problem 18.2.9}\label{problem-18.2.9}

\textbf{Given:} A dielectric HR mirror for a CO₂ laser (λ = 10.6 μm)
uses alternating quarter-wave layers of ZnSe (n\textsubscript{H} = 2.40)
and ThF₄ (n\textsubscript{L} = 1.35). The mirror has N = 10 layer pairs.

\textbf{Find:} (a) The physical thickness of each quarter-wave layer,
(b) the total stack thickness, and (c) the reflectivity.

\textbf{Solution:}

\begin{enumerate}
\def\labelenumi{(\alph{enumi})}
\item
  Quarter-wave thicknesses: t\textsubscript{H} = λ /
  (4n\textsubscript{H}) = 10,600 / (4 × 2.40) = 10,600 / 9.60 =
  \textbf{1,104 nm} t\textsubscript{L} = λ / (4n\textsubscript{L}) =
  10,600 / (4 × 1.35) = 10,600 / 5.40 = \textbf{1,963 nm}
\item
  Total stack thickness for 10 pairs (20 layers + 1 terminating H
  layer): t\textsubscript{total} = 10 × (1,104 + 1,963) + 1,104 = 10 ×
  3,067 + 1,104 = 30,670 + 1,104 = \textbf{31,774 nm ≈ 31.8 μm}
\item
  Reflectivity: ratio = n\textsubscript{H}/n\textsubscript{L} =
  2.40/1.35 = 1.778 R = {[}(1.778²⁰ − 1) / (1.778²⁰ + 1){]}² 1.778⁴ =
  (1.778²)² = (3.161)² = 9.992 1.778⁸ = (9.992)² = 99.84 1.778²⁰ =
  1.778⁸ × 1.778⁸ × 1.778⁴ = 99.84 × 99.84 × 9.992 = 99,601 R =
  {[}(99,601 − 1) / (99,601 + 1){]}² = {[}99,600 / 99,602{]}² =
  (0.99998)² = \textbf{99.996\%}
\end{enumerate}

This extremely high reflectivity is needed for the back mirror of CO₂
laser cavities.

\begin{center}\rule{0.5\linewidth}{0.5pt}\end{center}

\section{Problem 18.2.10}\label{problem-18.2.10}

\textbf{Given:} A circular aperture of diameter D = 2 mm is illuminated
by a HeNe laser at λ = 632.8 nm. The diffraction pattern is observed on
a screen at L = 3 m.

\textbf{Find:} (a) The angular radius of the Airy disk (first dark
ring), (b) the linear radius of the Airy disk on the screen, and (c) the
percentage of total power contained within the Airy disk.

\textbf{Solution:}

\begin{enumerate}
\def\labelenumi{(\alph{enumi})}
\tightlist
\item
  Angular radius of first dark ring: θ = 1.22λ/D = 1.22 × 632.8 × 10⁻⁹ /
  2 × 10⁻³ = 7.72 × 10⁻⁷ / 2 × 10⁻³ = 3.86 × 10⁻⁴ rad = \textbf{0.386
  mrad}
\end{enumerate}

Converting: θ = 3.86 × 10⁻⁴ × (180/π) × 3600 = \textbf{79.6 arc-seconds}

\begin{enumerate}
\def\labelenumi{(\alph{enumi})}
\setcounter{enumi}{1}
\tightlist
\item
  Linear radius on the screen: r = L × θ = 3 × 3.86 × 10⁻⁴ =
  \textbf{1.158 mm}
\end{enumerate}

The Airy disk diameter is 2r = 2.316 mm.

\begin{enumerate}
\def\labelenumi{(\alph{enumi})}
\setcounter{enumi}{2}
\tightlist
\item
  For a circular aperture diffraction pattern, the fraction of total
  power within the first dark ring (Airy disk) is a well-known result:
  P\textsubscript{Airy}/P\textsubscript{total} = \textbf{83.8\%}
\end{enumerate}

The remaining 16.2\% of the power is distributed among the surrounding
concentric rings, which decrease rapidly in intensity. The first bright
ring contains about 7.2\% of the total power, and the second bright ring
about 2.8\%.

\chapter{Chapter 18 --- Section 18.3:
Lasers}\label{chapter-18-section-18.3-lasers}

Practice problems covering stimulated emission and laser operation,
laser types and characteristics, laser safety classifications, and fiber
lasers and ultrafast lasers.

\begin{center}\rule{0.5\linewidth}{0.5pt}\end{center}

\section{Problem 18.3.1}\label{problem-18.3.1}

\textbf{Given:} A HeNe laser (λ = 632.8 nm) has a cavity length of L =
30 cm with mirror reflectivities R₁ = 99.9\% and R₂ = 98\%, and internal
round-trip loss of 1\%.

\textbf{Find:} (a) The cavity mode spacing (free spectral range), (b)
the number of longitudinal modes within the 1.5 GHz Doppler-broadened
gain bandwidth, and (c) the minimum single-pass gain required to reach
threshold.

\textbf{Solution:}

\begin{enumerate}
\def\labelenumi{(\alph{enumi})}
\item
  Mode spacing: Δf = c / (2L) = 3 × 10⁸ / (2 × 0.30) = 3 × 10⁸ / 0.60 =
  \textbf{500 MHz}
\item
  Number of modes within gain bandwidth: N = gain bandwidth / mode
  spacing = 1.5 × 10⁹ / 500 × 10⁶ = \textbf{3 modes}
\end{enumerate}

A HeNe laser typically operates on 2--3 longitudinal modes
simultaneously.

\begin{enumerate}
\def\labelenumi{(\alph{enumi})}
\setcounter{enumi}{2}
\tightlist
\item
  Round-trip loss factor: δ = R₁ × R₂ × (1 − internal loss) = 0.999 ×
  0.98 × 0.99 = 0.969 Threshold single-pass gain:
  G\textsubscript{threshold} = 1/√δ = 1/√0.969 = 1/0.9844 =
  \textbf{1.016}
\end{enumerate}

The required single-pass gain of 1.6\% is very modest, which is why HeNe
lasers can operate with a low-gain gas discharge.

\begin{center}\rule{0.5\linewidth}{0.5pt}\end{center}

\section{Problem 18.3.2}\label{problem-18.3.2}

\textbf{Given:} A CO₂ laser operating at λ = 10.6 μm delivers 2 kW of CW
optical power. The laser is electrically pumped at 15 kW. The beam exits
through a cavity mirror with R₂ = 90\% (output coupler) and the back
mirror has R₁ = 99.5\%.

\textbf{Find:} (a) The wall-plug efficiency, (b) the photon energy, (c)
the photon emission rate, and (d) the intracavity power.

\textbf{Solution:}

\begin{enumerate}
\def\labelenumi{(\alph{enumi})}
\item
  Wall-plug efficiency: η = P\textsubscript{opt} / P\textsubscript{elec}
  = 2,000 / 15,000 = 0.1333 = \textbf{13.3\%}
\item
  Photon energy: E = hc/λ = (6.626 × 10⁻³⁴ × 3 × 10⁸) / (10.6 × 10⁻⁶) =
  1.988 × 10⁻²⁵ / 1.06 × 10⁻⁵ = \textbf{1.875 × 10⁻²⁰ J = 0.117 eV}
\item
  Photon emission rate: N = P\textsubscript{opt} / E = 2,000 / 1.875 ×
  10⁻²⁰ = \textbf{1.067 × 10²³ photons/s}
\end{enumerate}

The low photon energy at 10.6 μm means very high photon counts for a
given power level.

\begin{enumerate}
\def\labelenumi{(\alph{enumi})}
\setcounter{enumi}{3}
\tightlist
\item
  The output coupler transmits T₂ = 1 − R₂ = 10\% of the intracavity
  power: P\textsubscript{intracavity} = P\textsubscript{out} / T₂ =
  2,000 / 0.10 = \textbf{20 kW}
\end{enumerate}

\begin{center}\rule{0.5\linewidth}{0.5pt}\end{center}

\section{Problem 18.3.3}\label{problem-18.3.3}

\textbf{Given:} A pulsed Nd:YAG laser (λ = 1064 nm) operates in
Q-switched mode, producing 10 ns pulses at a repetition rate of 20 Hz.
The average output power is 50 W.

\textbf{Find:} (a) The pulse energy, (b) the peak power, (c) the photon
energy, and (d) the number of photons per pulse.

\textbf{Solution:}

\begin{enumerate}
\def\labelenumi{(\alph{enumi})}
\item
  Pulse energy: E\textsubscript{pulse} = P\textsubscript{avg} /
  f\textsubscript{rep} = 50 / 20 = \textbf{2.5 J}
\item
  Peak power: P\textsubscript{peak} = E\textsubscript{pulse} /
  τ\textsubscript{p} = 2.5 / (10 × 10⁻⁹) = \textbf{250 MW}
\item
  Photon energy: E\textsubscript{photon} = hc/λ = (6.626 × 10⁻³⁴ × 3 ×
  10⁸) / (1064 × 10⁻⁹) = 1.988 × 10⁻²⁵ / 1.064 × 10⁻⁶ = \textbf{1.868 ×
  10⁻¹⁹ J = 1.165 eV}
\item
  Photons per pulse: N = E\textsubscript{pulse} /
  E\textsubscript{photon} = 2.5 / 1.868 × 10⁻¹⁹ = \textbf{1.338 × 10¹⁹
  photons}
\end{enumerate}

\begin{center}\rule{0.5\linewidth}{0.5pt}\end{center}

\section{Problem 18.3.4}\label{problem-18.3.4}

\textbf{Given:} A semiconductor laser diode operates at λ = 850 nm with
a threshold current of I\textsubscript{th} = 15 mA, slope efficiency of
η\textsubscript{s} = 0.5 W/A, and is driven at I = 45 mA with a forward
voltage of 1.8 V.

\textbf{Find:} (a) The optical output power, (b) the wall-plug
efficiency, (c) the external quantum efficiency, and (d) the number of
photons emitted per second.

\textbf{Solution:}

\begin{enumerate}
\def\labelenumi{(\alph{enumi})}
\item
  Output power (above threshold): P\textsubscript{opt} =
  η\textsubscript{s} × (I − I\textsubscript{th}) = 0.5 × (45 − 15) ×
  10⁻³ = 0.5 × 0.030 = \textbf{15 mW}
\item
  Wall-plug efficiency: P\textsubscript{elec} = I × V = 0.045 × 1.8 = 81
  mW η\textsubscript{WP} = P\textsubscript{opt} / P\textsubscript{elec}
  = 15 / 81 = \textbf{18.5\%}
\item
  External quantum efficiency (photons out per electron in):
  E\textsubscript{photon} = hc/λ = 1.988 × 10⁻²⁵ / 850 × 10⁻⁹ = 2.339 ×
  10⁻¹⁹ J Photon rate = P\textsubscript{opt} / E\textsubscript{photon} =
  15 × 10⁻³ / 2.339 × 10⁻¹⁹ = 6.413 × 10¹⁶ photons/s Electron rate = I/q
  = 0.045 / 1.602 × 10⁻¹⁹ = 2.809 × 10¹⁷ electrons/s
  η\textsubscript{ext} = photon rate / electron rate = 6.413 × 10¹⁶ /
  2.809 × 10¹⁷ = \textbf{22.8\%}
\item
  Photon emission rate = \textbf{6.41 × 10¹⁶ photons/s} (computed
  above).
\end{enumerate}

\begin{center}\rule{0.5\linewidth}{0.5pt}\end{center}

\section{Problem 18.3.5}\label{problem-18.3.5}

\textbf{Given:} A Class 4 Nd:YAG laser emits 5 W at λ = 1064 nm with a
beam divergence of 2 mrad (full angle) and an initial beam diameter of 3
mm. The MPE for a 10 s exposure at 1064 nm is 5.0 mW/cm².

\textbf{Find:} (a) The beam diameter at 100 m, (b) the irradiance at 100
m, (c) the NOHD, and (d) the required optical density (OD) for
protective eyewear.

\textbf{Solution:}

\begin{enumerate}
\def\labelenumi{(\alph{enumi})}
\item
  Beam diameter at 100 m: d(r) = d₀ + θ × r = 3 × 10⁻³ + 2 × 10⁻³ × 100
  = 0.003 + 0.200 = \textbf{0.203 m = 203 mm}
\item
  Irradiance at 100 m: A = π(d/2)² = π(0.1015)² = 0.03237 m² = 323.7 cm²
  E = P/A = 5 / 323.7 = \textbf{0.01545 W/cm² = 15.45 mW/cm²}
\end{enumerate}

This exceeds the MPE of 5.0 mW/cm², so the beam is hazardous at 100 m.

\begin{enumerate}
\def\labelenumi{(\alph{enumi})}
\setcounter{enumi}{2}
\item
  NOHD (where irradiance = MPE): At the NOHD, d ≈ θ × r (for large r, d₀
  is negligible): P / {[}π(θr/2)²{]} = MPE r = √(4P / (π × θ² × MPE)) =
  √(4 × 5 / (π × (2 × 10⁻³)² × 50)) = √(20 / (π × 4 × 10⁻⁶ × 50)) = √(20
  / 6.283 × 10⁻⁴) = √(31,831) = \textbf{178 m}
\item
  Required OD: OD = log₁₀(H\textsubscript{beam} / MPE) at the eye
  position. At close range (beam diameter \textasciitilde{} 3 mm,
  smaller than pupil): Irradiance at laser output = P / {[}π(d₀/2)²{]} =
  5 / {[}π(1.5 × 10⁻³)²{]} = 5 / 7.069 × 10⁻⁶ = 707,355 W/m² = 70.7
  W/cm² OD = log₁₀(70.7 / 0.005) = log₁₀(14,140) = \textbf{4.15}
\end{enumerate}

Eyewear rated OD ≥ 5 at 1064 nm would be selected with adequate safety
margin.

\begin{center}\rule{0.5\linewidth}{0.5pt}\end{center}

\section{Problem 18.3.6}\label{problem-18.3.6}

\textbf{Given:} A mode-locked Ti:sapphire laser produces 50 fs pulses at
λ = 800 nm, with a repetition rate of 76 MHz and average power of 500
mW. The cavity length is 1.974 m.

\textbf{Find:} (a) The pulse energy, (b) the peak power, (c) the
transform-limited spectral bandwidth, and (d) the number of locked
longitudinal modes.

\textbf{Solution:}

\begin{enumerate}
\def\labelenumi{(\alph{enumi})}
\item
  Pulse energy: E = P\textsubscript{avg} / f\textsubscript{rep} = 0.500
  / (76 × 10⁶) = \textbf{6.58 nJ}
\item
  Peak power: P\textsubscript{peak} = E / τ\textsubscript{p} = 6.58 ×
  10⁻⁹ / (50 × 10⁻¹⁵) = \textbf{131.6 kW}
\item
  Transform-limited bandwidth (Gaussian pulse): Δf = 0.44 /
  τ\textsubscript{p} = 0.44 / (50 × 10⁻¹⁵) = \textbf{8.8 × 10¹² Hz = 8.8
  THz}
\end{enumerate}

In wavelength: Δλ = λ² × Δf / c = (800 × 10⁻⁹)² × 8.8 × 10¹² / (3 × 10⁸)
= 6.4 × 10⁻¹³ × 8.8 × 10¹² / (3 × 10⁸) = 5.632 × 10⁰ / 3 × 10⁸ =
\textbf{18.8 nm}

\begin{enumerate}
\def\labelenumi{(\alph{enumi})}
\setcounter{enumi}{3}
\tightlist
\item
  Mode spacing: Δf\textsubscript{mode} = c / (2L) = 3 × 10⁸ / (2 ×
  1.974) = 76 × 10⁶ Hz = 76 MHz ✓ Number of locked modes: N = Δf /
  Δf\textsubscript{mode} = 8.8 × 10¹² / 76 × 10⁶ = \textbf{115,789
  modes}
\end{enumerate}

The enormous number of coherent modes explains the ability to produce
such ultrashort pulses.

\begin{center}\rule{0.5\linewidth}{0.5pt}\end{center}

\section{Problem 18.3.7}\label{problem-18.3.7}

\textbf{Given:} A chirped pulse amplification (CPA) system stretches 100
fs input pulses to 200 ps before amplification, amplifies to 5 mJ pulse
energy, then recompresses to 120 fs (due to imperfect compression).

\textbf{Find:} (a) The stretch ratio, (b) the peak power during
amplification (stretched pulse), (c) the peak power after compression,
and (d) the peak intensity if focused to a 10 μm diameter spot.

\textbf{Solution:}

\begin{enumerate}
\def\labelenumi{(\alph{enumi})}
\item
  Stretch ratio: R = τ\textsubscript{stretched} / τ\textsubscript{input}
  = 200 × 10⁻¹² / 100 × 10⁻¹⁵ = \textbf{2,000×}
\item
  Peak power during amplification: P\textsubscript{amp} = E /
  τ\textsubscript{stretched} = 5 × 10⁻³ / 200 × 10⁻¹² = \textbf{25 MW}
\end{enumerate}

This is manageable and below the damage threshold of the amplifier
optics.

\begin{enumerate}
\def\labelenumi{(\alph{enumi})}
\setcounter{enumi}{2}
\item
  Peak power after compression: P\textsubscript{compressed} = E /
  τ\textsubscript{compressed} = 5 × 10⁻³ / 120 × 10⁻¹⁵ = \textbf{41.7
  GW}
\item
  Peak intensity at focus: A = π(d/2)² = π(5 × 10⁻⁶)² = 7.854 × 10⁻¹¹ m²
  I = P / A = 41.7 × 10⁹ / 7.854 × 10⁻¹¹ = \textbf{5.31 × 10²⁰ W/m²}
\end{enumerate}

This intensity exceeds the threshold for ionizing air
(\textasciitilde10¹⁸ W/m²), so focusing must be done in vacuum for
ultrafast laser experiments.

\begin{center}\rule{0.5\linewidth}{0.5pt}\end{center}

\section{Problem 18.3.8}\label{problem-18.3.8}

\textbf{Given:} An erbium-doped fiber laser operates at λ = 1550 nm with
976 nm pump diodes. The pump power is 500 mW, the signal output power is
200 mW, and the quantum defect is the only fundamental efficiency limit.

\textbf{Find:} (a) The pump photon energy, (b) the signal photon energy,
(c) the quantum defect efficiency limit, (d) the actual slope
efficiency, and (e) the wasted power as heat.

\textbf{Solution:}

\begin{enumerate}
\def\labelenumi{(\alph{enumi})}
\item
  Pump photon energy: E\textsubscript{pump} = hc/λ\textsubscript{pump} =
  1.988 × 10⁻²⁵ / 976 × 10⁻⁹ = \textbf{2.037 × 10⁻¹⁹ J = 1.271 eV}
\item
  Signal photon energy: E\textsubscript{signal} =
  hc/λ\textsubscript{signal} = 1.988 × 10⁻²⁵ / 1550 × 10⁻⁹ =
  \textbf{1.282 × 10⁻¹⁹ J = 0.800 eV}
\item
  Quantum defect efficiency limit: η\textsubscript{QD} =
  E\textsubscript{signal} / E\textsubscript{pump} =
  λ\textsubscript{pump} / λ\textsubscript{signal} = 976 / 1550 =
  \textbf{63.0\%}
\end{enumerate}

Each pump photon can at most produce one signal photon, and the energy
difference (quantum defect) becomes heat.

\begin{enumerate}
\def\labelenumi{(\alph{enumi})}
\setcounter{enumi}{3}
\tightlist
\item
  Actual slope efficiency: η = P\textsubscript{out} /
  P\textsubscript{pump} = 200 / 500 = \textbf{40.0\%}
\end{enumerate}

This is 40/63 = 63.5\% of the quantum defect limit.

\begin{enumerate}
\def\labelenumi{(\alph{enumi})}
\setcounter{enumi}{4}
\tightlist
\item
  Wasted power: P\textsubscript{heat} = P\textsubscript{pump} −
  P\textsubscript{out} = 500 − 200 = \textbf{300 mW}
\end{enumerate}

\begin{center}\rule{0.5\linewidth}{0.5pt}\end{center}

\section{Problem 18.3.9}\label{problem-18.3.9}

\textbf{Given:} A VCSEL (Vertical-Cavity Surface-Emitting Laser) array
for a 3D sensing module contains 200 emitters, each producing 2 mW at λ
= 940 nm. The array is pulsed at 100 kHz with 10 ns pulse width for
time-of-flight ranging. Each emitter is driven at 5 mA and 2.0 V during
the pulse.

\textbf{Find:} (a) The total peak optical power, (b) the average optical
power, (c) the total average electrical power, and (d) the wall-plug
efficiency during the pulse.

\textbf{Solution:}

\begin{enumerate}
\def\labelenumi{(\alph{enumi})}
\item
  Total peak optical power: P\textsubscript{peak,total} = 200 × 2 mW =
  \textbf{400 mW}
\item
  Average optical power: Duty cycle = f × τ = 100 × 10³ × 10 × 10⁻⁹ =
  10⁻³ = 0.1\% P\textsubscript{avg,opt} = P\textsubscript{peak,total} ×
  duty cycle = 400 × 10⁻³ = \textbf{0.4 mW}
\item
  Peak electrical power per emitter: P\textsubscript{elec,peak} = I × V
  = 5 × 10⁻³ × 2.0 = 10 mW Total average electrical power:
  P\textsubscript{avg,elec} = 200 × 10 × 10⁻³ × 10⁻³ = \textbf{2.0 mW}
\item
  Wall-plug efficiency during pulse: η = P\textsubscript{opt,peak} /
  P\textsubscript{elec,peak} per emitter = 2 / 10 = \textbf{20\%}
\end{enumerate}

The low duty cycle keeps average power dissipation minimal, enabling
reliable operation without active cooling.

\begin{center}\rule{0.5\linewidth}{0.5pt}\end{center}

\section{Problem 18.3.10}\label{problem-18.3.10}

\textbf{Given:} A frequency-doubled Nd:YAG laser produces green light at
λ = 532 nm from the fundamental at λ = 1064 nm. The fundamental beam
power entering the KTP doubling crystal is 10 W, and the conversion
efficiency of the crystal is 45\%.

\textbf{Find:} (a) The green (532 nm) output power, (b) the residual
infrared (1064 nm) power, (c) the photon energy at each wavelength, and
(d) the photon rate ratio (green photons out vs.~IR photons consumed).

\textbf{Solution:}

\begin{enumerate}
\def\labelenumi{(\alph{enumi})}
\item
  Green output power: P\textsubscript{532} = η × P\textsubscript{1064} =
  0.45 × 10 = \textbf{4.5 W}
\item
  Residual IR power: P\textsubscript{1064,residual} =
  P\textsubscript{1064} − P\textsubscript{1064,consumed} Since 4.5 W of
  green is produced and each green photon requires two IR photons
  (energy conservation: E\textsubscript{532} = 2 ×
  E\textsubscript{1064}), the IR power consumed equals the green power
  produced. P\textsubscript{1064,residual} = 10 − 4.5 = \textbf{5.5 W}
\item
  Photon energies: E\textsubscript{1064} = hc/λ = 1.988 × 10⁻²⁵ / 1064 ×
  10⁻⁹ = \textbf{1.868 × 10⁻¹⁹ J = 1.165 eV} E\textsubscript{532} = hc/λ
  = 1.988 × 10⁻²⁵ / 532 × 10⁻⁹ = \textbf{3.737 × 10⁻¹⁹ J = 2.331 eV}
\end{enumerate}

E\textsubscript{532} = 2 × E\textsubscript{1064} confirms energy
conservation in the frequency doubling process.

\begin{enumerate}
\def\labelenumi{(\alph{enumi})}
\setcounter{enumi}{3}
\tightlist
\item
  Photon rate ratio: IR photons consumed per second: N\textsubscript{IR}
  = P\textsubscript{consumed} / E\textsubscript{1064} = 4.5 / 1.868 ×
  10⁻¹⁹ = 2.409 × 10¹⁹ /s Green photons produced per second:
  N\textsubscript{532} = P\textsubscript{532} / E\textsubscript{532} =
  4.5 / 3.737 × 10⁻¹⁹ = 1.204 × 10¹⁹ /s Ratio: N\textsubscript{532} /
  N\textsubscript{IR} = 1.204 / 2.409 = \textbf{0.50 (exactly 1:2)}
\end{enumerate}

Two infrared photons combine to produce one green photon, confirming the
second-harmonic generation process.

\chapter{Chapter 18 --- Section 18.4:
Photodetectors}\label{chapter-18-section-18.4-photodetectors}

Practice problems covering PIN and avalanche photodiodes,
phototransistors and photomultipliers, image sensors, and noise in
optical detection.

\begin{center}\rule{0.5\linewidth}{0.5pt}\end{center}

\section{Problem 18.4.1}\label{problem-18.4.1}

\textbf{Given:} A silicon PIN photodiode has a quantum efficiency of
70\% at λ = 850 nm and is reverse-biased. The incident optical power is
5 μW.

\textbf{Find:} (a) The responsivity, (b) the photocurrent, (c) the
photon rate arriving at the detector, and (d) the electron-hole pair
generation rate.

\textbf{Solution:}

\begin{enumerate}
\def\labelenumi{(\alph{enumi})}
\item
  Responsivity: R = ηqλ / (hc) = 0.70 × 1.602 × 10⁻¹⁹ × 850 × 10⁻⁹ /
  (6.626 × 10⁻³⁴ × 3 × 10⁸) = 0.70 × 1.602 × 10⁻¹⁹ × 8.50 × 10⁻⁷ /
  (1.988 × 10⁻²⁵) = 0.70 × 6.849 × 10⁻¹ = \textbf{0.479 A/W}
\item
  Photocurrent: I\textsubscript{ph} = R × P = 0.479 × 5 × 10⁻⁶ =
  \textbf{2.40 μA}
\item
  Photon rate: E\textsubscript{photon} = hc/λ = 1.988 × 10⁻²⁵ / 850 ×
  10⁻⁹ = 2.339 × 10⁻¹⁹ J Photon rate = P / E\textsubscript{photon} = 5 ×
  10⁻⁶ / 2.339 × 10⁻¹⁹ = \textbf{2.138 × 10¹³ photons/s}
\item
  Electron-hole pair generation rate: N\textsubscript{e-h} = η × photon
  rate = 0.70 × 2.138 × 10¹³ = \textbf{1.497 × 10¹³ pairs/s}
\end{enumerate}

Verification: I\textsubscript{ph} = N\textsubscript{e-h} × q = 1.497 ×
10¹³ × 1.602 × 10⁻¹⁹ = 2.40 × 10⁻⁶ A ✓

\begin{center}\rule{0.5\linewidth}{0.5pt}\end{center}

\section{Problem 18.4.2}\label{problem-18.4.2}

\textbf{Given:} An InGaAs APD has a primary responsivity (without gain)
of R₀ = 1.0 A/W at λ = 1550 nm, avalanche gain M = 10, and excess noise
factor F(M) = M⁰·⁷. The incident optical power is 1 μW.

\textbf{Find:} (a) The APD responsivity, (b) the multiplied
photocurrent, (c) the excess noise factor, and (d) the shot noise
current for bandwidth B = 1 GHz.

\textbf{Solution:}

\begin{enumerate}
\def\labelenumi{(\alph{enumi})}
\item
  APD responsivity: R\textsubscript{APD} = M × R₀ = 10 × 1.0 =
  \textbf{10 A/W}
\item
  Multiplied photocurrent: I\textsubscript{ph} = R\textsubscript{APD} ×
  P = 10 × 1 × 10⁻⁶ = \textbf{10 μA}
\item
  Excess noise factor: F(M) = M⁰·⁷ = 10⁰·⁷ = \textbf{5.012}
\item
  APD shot noise: i\textsubscript{shot} = M ×
  √(2qI\textsubscript{primary}F(M)B) I\textsubscript{primary} = R₀ × P =
  1.0 × 1 × 10⁻⁶ = 1 μA i\textsubscript{shot} = 10 × √(2 × 1.602 × 10⁻¹⁹
  × 1 × 10⁻⁶ × 5.012 × 1 × 10⁹) = 10 × √(1.606 × 10⁻¹⁵) = 10 × 4.008 ×
  10⁻⁸ = \textbf{0.401 μA RMS}
\end{enumerate}

The signal-to-noise ratio from shot noise alone: SNR =
I\textsubscript{ph} / i\textsubscript{shot} = 10 / 0.401 = 24.9, or 27.9
dB.

\begin{center}\rule{0.5\linewidth}{0.5pt}\end{center}

\section{Problem 18.4.3}\label{problem-18.4.3}

\textbf{Given:} A photomultiplier tube has 12 dynodes with secondary
emission ratio δ = 5 per dynode. The photocathode quantum efficiency is
20\% at λ = 400 nm. A light source produces 10⁴ photons/s incident on
the photocathode.

\textbf{Find:} (a) The total PMT gain, (b) the photocathode current, (c)
the anode current, and (d) the single-photoelectron pulse charge.

\textbf{Solution:}

\begin{enumerate}
\def\labelenumi{(\alph{enumi})}
\item
  Total gain: M = δ¹² = 5¹² = 5⁶ × 5⁶ = 15,625 × 15,625 = \textbf{2.441
  × 10⁸}
\item
  Photocathode current: Photoelectron rate = η × photon rate = 0.20 ×
  10⁴ = 2,000 photoelectrons/s I\textsubscript{cathode} =
  N\textsubscript{pe} × q = 2,000 × 1.602 × 10⁻¹⁹ = \textbf{3.204 ×
  10⁻¹⁶ A = 0.320 fA}
\item
  Anode current: I\textsubscript{anode} = M × I\textsubscript{cathode} =
  2.441 × 10⁸ × 3.204 × 10⁻¹⁶ = \textbf{78.2 nA}
\end{enumerate}

This is easily measurable with standard electronics, even though only
2,000 photoelectrons/s leave the cathode.

\begin{enumerate}
\def\labelenumi{(\alph{enumi})}
\setcounter{enumi}{3}
\tightlist
\item
  Single-photoelectron pulse charge: Q = M × q = 2.441 × 10⁸ × 1.602 ×
  10⁻¹⁹ = \textbf{3.91 × 10⁻¹¹ C = 39.1 pC}
\end{enumerate}

With a 5 ns pulse width, the single-photoelectron pulse current is I =
Q/t = 39.1 × 10⁻¹² / 5 × 10⁻⁹ = 7.82 mA --- large enough for direct
detection without additional amplification.

\begin{center}\rule{0.5\linewidth}{0.5pt}\end{center}

\section{Problem 18.4.4}\label{problem-18.4.4}

\textbf{Given:} A CMOS image sensor has the following specifications: 24
megapixels, 3.75 μm pixel pitch, 6:4 aspect ratio (3:2), full-well
capacity of 30,000 electrons, and read noise of 5 electrons RMS.

\textbf{Find:} (a) The pixel array dimensions, (b) the sensor physical
dimensions, (c) the dynamic range in dB, and (d) the dynamic range in
stops.

\textbf{Solution:}

\begin{enumerate}
\def\labelenumi{(\alph{enumi})}
\item
  Total pixels: 24 × 10⁶ with aspect ratio 3:2. N\textsubscript{x} ×
  N\textsubscript{y} = 24 × 10⁶, N\textsubscript{x}/N\textsubscript{y} =
  3/2. N\textsubscript{y} = √(24 × 10⁶ × 2/3) = √(16 × 10⁶) = 4,000
  pixels. N\textsubscript{x} = 3/2 × 4,000 = \textbf{6,000 × 4,000
  pixels}
\item
  Sensor dimensions: W = 6,000 × 3.75 μm = 22,500 μm = \textbf{22.5 mm}
  H = 4,000 × 3.75 μm = 15,000 μm = \textbf{15.0 mm}
\end{enumerate}

This is approximately an APS-C format sensor (22.5 × 15.0 mm).

\begin{enumerate}
\def\labelenumi{(\alph{enumi})}
\setcounter{enumi}{2}
\item
  Dynamic range: DR = 20 × log₁₀(full-well / read noise) = 20 ×
  log₁₀(30,000 / 5) = 20 × log₁₀(6,000) = 20 × 3.778 = \textbf{75.6 dB}
\item
  In stops (each stop = 6.02 dB): DR\textsubscript{stops} = 75.6 / 6.02
  = \textbf{12.6 stops}
\end{enumerate}

This is typical of a mid-range DSLR sensor.

\begin{center}\rule{0.5\linewidth}{0.5pt}\end{center}

\section{Problem 18.4.5}\label{problem-18.4.5}

\textbf{Given:} An InGaAs PIN photodiode with R = 0.95 A/W and dark
current I\textsubscript{d} = 2 nA is used in a 2.5 Gbps receiver with
bandwidth B = 1.875 GHz. The transimpedance amplifier has
R\textsubscript{f} = 2 kΩ (effective load resistance) and T = 300 K.

\textbf{Find:} (a) The thermal noise current, (b) the dark current
noise, (c) the total noise current (neglecting shot noise for small
signals), and (d) the receiver sensitivity for BER = 10⁻⁹ (Q = 6).

\textbf{Solution:}

\begin{enumerate}
\def\labelenumi{(\alph{enumi})}
\item
  Thermal noise: i\textsubscript{thermal} = √(4k\textsubscript{B}TB /
  R\textsubscript{f}) = √(4 × 1.381 × 10⁻²³ × 300 × 1.875 × 10⁹ / 2,000)
  = √(4 × 1.381 × 10⁻²³ × 300 × 9.375 × 10⁵) = √(1.553 × 10⁻¹⁴) =
  \textbf{124.6 nA RMS}
\item
  Dark current noise: i\textsubscript{dark} = √(2qI\textsubscript{d}B) =
  √(2 × 1.602 × 10⁻¹⁹ × 2 × 10⁻⁹ × 1.875 × 10⁹) = √(1.202 × 10⁻¹⁸) =
  \textbf{1.096 nA RMS}
\item
  Total noise current: i\textsubscript{n} = √(i\textsubscript{thermal}²
  + i\textsubscript{dark}²) = √(124.6² + 1.096²) ≈ \textbf{124.6 nA RMS}
\end{enumerate}

The thermal noise dominates by a factor of over 100.

\begin{enumerate}
\def\labelenumi{(\alph{enumi})}
\setcounter{enumi}{3}
\tightlist
\item
  Receiver sensitivity: P\textsubscript{sens} = Q × i\textsubscript{n} /
  R = 6 × 124.6 × 10⁻⁹ / 0.95 = 787 × 10⁻⁹ = 787 nW = \textbf{-31.0 dBm}
\end{enumerate}

The high transimpedance (2 kΩ) significantly improves sensitivity
compared to a 50 Ω load, which would give i\textsubscript{thermal} = 790
nA and sensitivity of −22.0 dBm.

\begin{center}\rule{0.5\linewidth}{0.5pt}\end{center}

\section{Problem 18.4.6}\label{problem-18.4.6}

\textbf{Given:} A CCD sensor for astronomical imaging has 2048 × 2048
pixels with 15 μm pixel pitch, quantum efficiency η = 90\% at 600 nm,
read noise of 3 electrons RMS, dark current of 0.01 electrons/pixel/s at
−40°C, and full-well capacity of 100,000 electrons.

\textbf{Find:} (a) The sensor dimensions, (b) the dynamic range, (c) the
minimum detectable signal for SNR = 5 in a 300 s exposure, and (d) the
dark current noise for the 300 s exposure.

\textbf{Solution:}

\begin{enumerate}
\def\labelenumi{(\alph{enumi})}
\tightlist
\item
  Sensor dimensions: W = H = 2048 × 15 μm = 30,720 μm = \textbf{30.72
  mm} (square sensor)
\end{enumerate}

Total area = 30.72² = 943.7 mm² ≈ 9.44 cm²

\begin{enumerate}
\def\labelenumi{(\alph{enumi})}
\setcounter{enumi}{1}
\tightlist
\item
  Dynamic range: DR = 20 × log₁₀(100,000 / 3) = 20 × log₁₀(33,333) = 20
  × 4.523 = \textbf{90.5 dB}
\end{enumerate}

This is approximately 15 stops --- excellent for astrophotography.

\begin{enumerate}
\def\labelenumi{(\alph{enumi})}
\setcounter{enumi}{2}
\tightlist
\item
  Dark current accumulation in 300 s: N\textsubscript{dark} = 0.01 × 300
  = 3 electrons Dark noise = √N\textsubscript{dark} = √3 = 1.73
  electrons RMS
\end{enumerate}

Total noise = √(read noise² + dark noise²) = √(9 + 3) = √12 = 3.46
electrons RMS

Minimum detectable signal at SNR = 5: For low signal, SNR ≈
N\textsubscript{signal} / total noise, so N\textsubscript{signal} = 5 ×
3.46 = \textbf{17.3 electrons}

\begin{enumerate}
\def\labelenumi{(\alph{enumi})}
\setcounter{enumi}{3}
\tightlist
\item
  Minimum detectable optical power: Photon energy at 600 nm: E = hc/λ =
  1.988 × 10⁻²⁵ / 600 × 10⁻⁹ = 3.313 × 10⁻¹⁹ J Photons needed:
  N\textsubscript{photon} = N\textsubscript{signal} / η = 17.3 / 0.90 =
  19.2 photons over 300 s Power = N\textsubscript{photon} × E / t = 19.2
  × 3.313 × 10⁻¹⁹ / 300 = \textbf{2.12 × 10⁻²⁰ W per pixel}
\end{enumerate}

\begin{center}\rule{0.5\linewidth}{0.5pt}\end{center}

\section{Problem 18.4.7}\label{problem-18.4.7}

\textbf{Given:} An optical receiver uses an InGaAs PIN photodiode (R =
1.05 A/W, I\textsubscript{d} = 10 nA) followed by a TIA with noise
figure equivalent to R\textsubscript{L} = 500 Ω at T = 300 K. The signal
power is P = 100 μW and the bandwidth is B = 10 GHz.

\textbf{Find:} (a) The photocurrent, (b) the shot noise, (c) the thermal
noise, (d) the total noise, and (e) the electrical SNR.

\textbf{Solution:}

\begin{enumerate}
\def\labelenumi{(\alph{enumi})}
\item
  Photocurrent: I\textsubscript{ph} = R × P = 1.05 × 100 × 10⁻⁶ =
  \textbf{105 μA}
\item
  Shot noise: i\textsubscript{shot} = √(2q(I\textsubscript{ph} +
  I\textsubscript{d})B) = √(2 × 1.602 × 10⁻¹⁹ × (105 + 0.01) × 10⁻⁶ × 10
  × 10⁹) = √(2 × 1.602 × 10⁻¹⁹ × 1.0501 × 10⁻⁴ × 10¹⁰) = √(3.365 ×
  10⁻¹³) = \textbf{0.580 μA RMS}
\item
  Thermal noise: i\textsubscript{thermal} = √(4k\textsubscript{B}TB /
  R\textsubscript{L}) = √(4 × 1.381 × 10⁻²³ × 300 × 10 × 10⁹ / 500) =
  √(3.314 × 10⁻¹³) = \textbf{0.576 μA RMS}
\item
  Total noise: i\textsubscript{n} = √(i\textsubscript{shot}² +
  i\textsubscript{thermal}²) = √(0.580² + 0.576²) = √(0.336 + 0.332) =
  √0.668 = \textbf{0.817 μA RMS}
\end{enumerate}

The receiver is shot-noise and thermal-noise co-limited (both
approximately equal at these conditions).

\begin{enumerate}
\def\labelenumi{(\alph{enumi})}
\setcounter{enumi}{4}
\tightlist
\item
  Electrical SNR: SNR = I\textsubscript{ph} / i\textsubscript{n} = 105 /
  0.817 = 128.5 SNR (dB) = 20 × log₁₀(128.5) = \textbf{42.2 dB}
\end{enumerate}

\begin{center}\rule{0.5\linewidth}{0.5pt}\end{center}

\section{Problem 18.4.8}\label{problem-18.4.8}

\textbf{Given:} An InGaAs APD receiver is compared to a PIN receiver for
a 10 Gbps link. Both have bandwidth B = 7.5 GHz and T = 300 K. The PIN
has R = 1.0 A/W with R\textsubscript{L} = 50 Ω. The APD has R₀ = 0.9
A/W, gain M = 10, excess noise factor exponent x = 0.7, and same
R\textsubscript{L} = 50 Ω.

\textbf{Find:} (a) The PIN sensitivity for BER = 10⁻⁹ (Q = 6), (b) the
APD sensitivity, and (c) the improvement in dB.

\textbf{Solution:}

\begin{enumerate}
\def\labelenumi{(\alph{enumi})}
\item
  PIN thermal noise: i\textsubscript{th} = √(4k\textsubscript{B}TB /
  R\textsubscript{L}) = √(4 × 1.381 × 10⁻²³ × 300 × 7.5 × 10⁹ / 50) =
  √(2.486 × 10⁻¹²) = 1.577 μA PIN sensitivity: P\textsubscript{PIN} = Q
  × i\textsubscript{th} / R = 6 × 1.577 × 10⁻⁶ / 1.0 = \textbf{9.46 μW =
  −20.2 dBm}
\item
  APD: the total noise includes both thermal and amplified shot noise.
  For sensitivity calculation, the signal-dependent shot noise makes
  this iterative. Using the approximation: P\textsubscript{APD} ≈ Q ×
  i\textsubscript{th} / (M × R₀) for the thermal-noise-limited case:
  P\textsubscript{APD} = 6 × 1.577 × 10⁻⁶ / (10 × 0.9) = 6 × 1.577 ×
  10⁻⁶ / 9.0 = \textbf{1.051 μW = −29.8 dBm}
\end{enumerate}

Checking shot noise at this power: I\textsubscript{primary} = R₀ × P =
0.9 × 1.051 × 10⁻⁶ = 0.946 μA F(M) = 10⁰·⁷ = 5.012 i\textsubscript{shot}
= M × √(2qI\textsubscript{primary}F(M)B) = 10 × √(2 × 1.602 × 10⁻¹⁹ ×
9.46 × 10⁻⁷ × 5.012 × 7.5 × 10⁹) = 10 × √(1.139 × 10⁻¹⁴) = 10 × 1.067 ×
10⁻⁷ = 1.067 μA

Total noise = √(1.577² + 1.067²) = √(2.487 + 1.139) = √3.626 = 1.904 μA
Refined: P\textsubscript{APD} = Q × i\textsubscript{total} / (M × R₀) =
6 × 1.904 × 10⁻⁶ / 9.0 = \textbf{1.269 μW = −29.0 dBm}

\begin{enumerate}
\def\labelenumi{(\alph{enumi})}
\setcounter{enumi}{2}
\tightlist
\item
  Improvement: ΔP = −20.2 − (−29.0) = \textbf{8.8 dB improvement} with
  the APD receiver.
\end{enumerate}

\begin{center}\rule{0.5\linewidth}{0.5pt}\end{center}

\section{Problem 18.4.9}\label{problem-18.4.9}

\textbf{Given:} A time-of-flight lidar system uses a pulsed laser at λ =
905 nm with pulse energy 200 nJ and 5 ns pulse width. The detector is a
silicon APD with R₀ = 0.5 A/W, M = 50, dark current I\textsubscript{d} =
50 nA, and bandwidth B = 200 MHz. The target is at 100 m with
reflectivity ρ = 0.3, and the receiver aperture diameter is 50 mm.

\textbf{Find:} (a) The received optical energy per pulse, (b) the peak
received power, (c) the signal photocurrent, and (d) the SNR.

\textbf{Solution:}

\begin{enumerate}
\def\labelenumi{(\alph{enumi})}
\item
  For a Lambertian target, received power fraction: The laser
  illuminates a spot; the return energy is: E\textsubscript{rx} =
  E\textsubscript{tx} × ρ × A\textsubscript{rx} / (π × R²)
  A\textsubscript{rx} = π(D/2)² = π(0.025)² = 1.964 × 10⁻³ m²
  E\textsubscript{rx} = 200 × 10⁻⁹ × 0.3 × 1.964 × 10⁻³ / (π × 100²) =
  200 × 10⁻⁹ × 0.3 × 1.964 × 10⁻³ / 31,416 = \textbf{3.75 × 10⁻¹⁵ J =
  3.75 fJ}
\item
  Peak received power: P\textsubscript{rx} = E\textsubscript{rx} / τ =
  3.75 × 10⁻¹⁵ / 5 × 10⁻⁹ = \textbf{0.75 μW}
\item
  Signal photocurrent (with APD gain): I\textsubscript{signal} = M × R₀
  × P\textsubscript{rx} = 50 × 0.5 × 0.75 × 10⁻⁶ = \textbf{18.75 μA}
\item
  Noise analysis (R\textsubscript{L} = 50 Ω, T = 300 K):
  i\textsubscript{thermal} = √(4k\textsubscript{B}TB/R\textsubscript{L})
  = √(4 × 1.381 × 10⁻²³ × 300 × 200 × 10⁶ / 50) = √(6.629 × 10⁻¹³) =
  \textbf{257 nA} F(M) = 50⁰·⁷ = 15.46 i\textsubscript{shot} = M ×
  √(2q(I\textsubscript{d}/M + R₀ × P\textsubscript{rx})F(M)B) = 50 × √(2
  × 1.602 × 10⁻¹⁹ × (1 × 10⁻⁹ + 375 × 10⁻⁹) × 15.46 × 200 × 10⁶) = 50 ×
  √(2 × 1.602 × 10⁻¹⁹ × 376 × 10⁻⁹ × 3.092 × 10⁹) = 50 × √(3.733 ×
  10⁻¹⁶) = 50 × 1.932 × 10⁻⁸ = \textbf{965 nA} i\textsubscript{total} =
  √(257² + 965²) = √(66,049 + 931,225) = √997,274 = \textbf{999 nA ≈ 1.0
  μA} SNR = 18,750 / 999 = 18.8 = \textbf{+25.5 dB} (single pulse)
\end{enumerate}

This strong single-pulse SNR confirms the system can reliably detect the
target at 100 m.

\begin{center}\rule{0.5\linewidth}{0.5pt}\end{center}

\section{Problem 18.4.10}\label{problem-18.4.10}

\textbf{Given:} A photoconductive detector made of HgCdTe operates at λ
= 10 μm with active area A = 1 mm², specific detectivity D* = 2 × 10¹⁰
cm·√Hz/W, and bandwidth B = 100 kHz.

\textbf{Find:} (a) The NEP, (b) the minimum detectable power for SNR =
10, (c) the noise current if the responsivity is R = 500 V/W with
R\textsubscript{L} = 10 kΩ, and (d) the photon energy at 10 μm.

\textbf{Solution:}

\begin{enumerate}
\def\labelenumi{(\alph{enumi})}
\tightlist
\item
  NEP (noise-equivalent power): NEP = √A / D* = √(1 × 10⁻² cm²) / (2 ×
  10¹⁰) = 0.1 / (2 × 10¹⁰) = \textbf{5 × 10⁻¹² W/√Hz}
\end{enumerate}

For the given bandwidth: NEP\textsubscript{total} = NEP × √B = 5 × 10⁻¹²
× √(100 × 10³) = 5 × 10⁻¹² × 316.2 = \textbf{1.581 × 10⁻⁹ W = 1.58 nW}

\begin{enumerate}
\def\labelenumi{(\alph{enumi})}
\setcounter{enumi}{1}
\item
  Minimum detectable power for SNR = 10: P\textsubscript{min} = SNR ×
  NEP\textsubscript{total} = 10 × 1.581 × 10⁻⁹ = \textbf{15.81 nW}
\item
  Noise voltage: V\textsubscript{noise} = NEP\textsubscript{total} × R =
  1.581 × 10⁻⁹ × 500 = 7.905 × 10⁻⁷ V = 0.791 μV Noise current:
  i\textsubscript{noise} = V\textsubscript{noise} / R\textsubscript{L} =
  7.905 × 10⁻⁷ / 10,000 = \textbf{79.1 pA RMS}
\item
  Photon energy at 10 μm: E = hc/λ = 1.988 × 10⁻²⁵ / 10 × 10⁻⁶ =
  \textbf{1.988 × 10⁻²⁰ J = 0.124 eV}
\end{enumerate}

This very low photon energy (thermal IR) is why HgCdTe detectors must be
cooled to liquid nitrogen temperatures (77 K) --- the thermal energy
k\textsubscript{B}T at room temperature (26 meV) is a significant
fraction of the photon energy, producing excessive dark current.

\chapter{Chapter 18 --- Section 18.5: Optical Fiber and
Communication}\label{chapter-18-section-18.5-optical-fiber-and-communication}

Practice problems covering fiber modes and propagation, attenuation and
dispersion, optical amplifiers and WDM, and coherent optical
communication.

\begin{center}\rule{0.5\linewidth}{0.5pt}\end{center}

\section{Problem 18.5.1}\label{problem-18.5.1}

\textbf{Given:} A multimode step-index fiber has core index n₁ = 1.480,
cladding index n₂ = 1.460, and core diameter d = 62.5 μm. The operating
wavelength is λ = 850 nm.

\textbf{Find:} (a) The numerical aperture, (b) the maximum acceptance
half-angle, (c) the V number, and (d) the approximate number of guided
modes.

\textbf{Solution:}

\begin{enumerate}
\def\labelenumi{(\alph{enumi})}
\item
  Numerical aperture: NA = √(n₁² − n₂²) = √(1.480² − 1.460²) = √(2.1904
  − 2.1316) = √(0.0588) = \textbf{0.2425}
\item
  Maximum acceptance half-angle: θ\textsubscript{max} = arcsin(NA) =
  arcsin(0.2425) = \textbf{14.03°}
\item
  V number: V = πd × NA / λ = π × 62.5 × 10⁻⁶ × 0.2425 / (850 × 10⁻⁹) =
  4.762 × 10⁻⁵ / 8.50 × 10⁻⁷ = \textbf{56.0}
\item
  Approximate number of guided modes (step-index): M ≈ V²/2 = 56.0² / 2
  = 3,136 / 2 = \textbf{1,568 modes}
\end{enumerate}

This large number of modes explains why multimode fiber has high modal
dispersion and is limited to shorter distances.

\begin{center}\rule{0.5\linewidth}{0.5pt}\end{center}

\section{Problem 18.5.2}\label{problem-18.5.2}

\textbf{Given:} A single-mode fiber link at 1550 nm has the following
characteristics: fiber attenuation = 0.20 dB/km, chromatic dispersion D
= 17 ps/(nm·km), link length = 120 km, and laser spectral width Δλ =
0.05 nm (DFB laser).

\textbf{Find:} (a) The total fiber attenuation, (b) the chromatic
dispersion broadening, (c) the maximum bit rate limited by dispersion
(NRZ, rule: Δτ \textless{} 0.7 × bit period), and (d) the
dispersion-limited distance for a 40 Gbps signal.

\textbf{Solution:}

\begin{enumerate}
\def\labelenumi{(\alph{enumi})}
\item
  Total attenuation: α\textsubscript{total} = 0.20 × 120 = \textbf{24.0
  dB}
\item
  Chromatic dispersion broadening: Δτ = D × L × Δλ = 17 × 120 × 0.05 =
  \textbf{102 ps}
\item
  Maximum bit rate: Bit period \textgreater{} Δτ / 0.7 = 102 / 0.7 =
  145.7 ps B\textsubscript{max} = 1 / 145.7 × 10⁻¹² = \textbf{6.86 Gbps}
\end{enumerate}

A 10 Gbps NRZ signal (bit period = 100 ps) would suffer significant ISI
on this link.

\begin{enumerate}
\def\labelenumi{(\alph{enumi})}
\setcounter{enumi}{3}
\tightlist
\item
  Dispersion-limited distance at 40 Gbps: Bit period = 25 ps. Maximum Δτ
  = 0.7 × 25 = 17.5 ps. L\textsubscript{max} = Δτ / (D × Δλ) = 17.5 /
  (17 × 0.05) = 17.5 / 0.85 = \textbf{20.6 km}
\end{enumerate}

Dispersion compensation is essential for high-speed, long-haul links.

\begin{center}\rule{0.5\linewidth}{0.5pt}\end{center}

\section{Problem 18.5.3}\label{problem-18.5.3}

\textbf{Given:} An EDFA has the following specifications: gain G = 25
dB, noise figure NF = 5.5 dB, input signal power = −20 dBm per channel,
and operates at 1550 nm with reference bandwidth B\textsubscript{ref} =
12.5 GHz (0.1 nm).

\textbf{Find:} (a) The output signal power, (b) the ASE noise power per
channel, (c) the output OSNR, and (d) the output OSNR if the input
signal is −30 dBm.

\textbf{Solution:}

\begin{enumerate}
\def\labelenumi{(\alph{enumi})}
\item
  Output signal power: P\textsubscript{out} = P\textsubscript{in} + G =
  −20 + 25 = \textbf{+5 dBm}
\item
  ASE noise power (per polarization, in reference bandwidth):
  P\textsubscript{ASE} = NF × hf × G × B\textsubscript{ref} (in linear
  units) In dBm: P\textsubscript{ASE} = NF + G +
  10log₁₀(hfB\textsubscript{ref}) 10log₁₀(hfB\textsubscript{ref}) =
  10log₁₀(6.626 × 10⁻³⁴ × 193.1 × 10¹² × 12.5 × 10⁹) = 10log₁₀(1.599 ×
  10⁻⁹ W) = 10log₁₀(1.599 × 10⁻⁶ mW) = −58.0 dBm P\textsubscript{ASE} =
  5.5 + 25 + (−58.0) = \textbf{−27.5 dBm} per polarization
\end{enumerate}

For both polarizations: P\textsubscript{ASE,total} = −27.5 + 3 =
\textbf{−24.5 dBm}

\begin{enumerate}
\def\labelenumi{(\alph{enumi})}
\setcounter{enumi}{2}
\item
  Output OSNR: OSNR = P\textsubscript{out} − P\textsubscript{ASE,total}
  = 5 − (−24.5) = \textbf{29.5 dB}
\item
  With −30 dBm input: P\textsubscript{out} = −30 + 25 = −5 dBm OSNR = −5
  − (−24.5) = \textbf{19.5 dB}
\end{enumerate}

A single EDFA provides moderate OSNR; the challenge arises from
cascading many amplifiers where ASE noise accumulates.

\begin{center}\rule{0.5\linewidth}{0.5pt}\end{center}

\section{Problem 18.5.4}\label{problem-18.5.4}

\textbf{Given:} A DWDM long-haul system has 40 channels at 100 GHz
spacing, each at 0 dBm launch power. The link consists of 20 spans of
100 km each. Fiber attenuation is 0.2 dB/km, and each EDFA has gain
matched to span loss and noise figure NF = 5 dB.

\textbf{Find:} (a) The span loss, (b) the EDFA gain required, (c) the
OSNR after 20 spans using the standard formula, and (d) whether the OSNR
is sufficient for 100 Gbps DP-QPSK (required OSNR ≈ 12 dB).

\textbf{Solution:}

\begin{enumerate}
\def\labelenumi{(\alph{enumi})}
\item
  Span loss: L\textsubscript{span} = 0.2 × 100 = \textbf{20 dB}
\item
  EDFA gain = span loss = \textbf{20 dB}
\item
  OSNR after N spans (standard formula): OSNR = 58 +
  P\textsubscript{launch} − L\textsubscript{span} − NF − 10log₁₀(N) = 58
  + 0 − 20 − 5 − 10log₁₀(20) = 58 − 20 − 5 − 13.01 = \textbf{20.0 dB}
\item
  OSNR margin: Margin = 20.0 − 12 = \textbf{8.0 dB}
\end{enumerate}

This is sufficient for DP-QPSK operation. However, for DP-16QAM
(required OSNR ≈ 18 dB), the margin would be only 2 dB, which is
marginal.

\begin{center}\rule{0.5\linewidth}{0.5pt}\end{center}

\section{Problem 18.5.5}\label{problem-18.5.5}

\textbf{Given:} A fiber link must carry 10 Gbps NRZ data at 1310 nm over
standard single-mode fiber. The transmitter power is +2 dBm, the
receiver sensitivity is −23 dBm (BER = 10⁻¹²). The link has 6 connectors
(0.5 dB each), 4 fusion splices (0.1 dB each), and fiber attenuation is
0.35 dB/km.

\textbf{Find:} (a) The total connector and splice loss, (b) the
available power budget after accounting for a 3 dB system margin, (c)
the maximum fiber length, and (d) the total link loss at that length.

\textbf{Solution:}

\begin{enumerate}
\def\labelenumi{(\alph{enumi})}
\item
  Connector and splice losses: L\textsubscript{connectors} = 6 × 0.5 =
  3.0 dB L\textsubscript{splices} = 4 × 0.1 = 0.4 dB Total fixed losses
  = 3.0 + 0.4 = \textbf{3.4 dB}
\item
  Available power budget: Total budget = P\textsubscript{tx} −
  P\textsubscript{sens} = 2 − (−23) = 25 dB After system margin: 25 − 3
  = 22 dB After fixed losses: 22 − 3.4 = \textbf{18.6 dB available for
  fiber}
\item
  Maximum fiber length: L\textsubscript{max} = 18.6 / 0.35 =
  \textbf{53.1 km}
\item
  Total link loss at 53.1 km: L\textsubscript{total} = 0.35 × 53.1 + 3.4
  = 18.6 + 3.4 = \textbf{22.0 dB} Received power = 2 − 22.0 = −20.0 dBm,
  with 3.0 dB margin above the −23 dBm sensitivity.
\end{enumerate}

\begin{center}\rule{0.5\linewidth}{0.5pt}\end{center}

\section{Problem 18.5.6}\label{problem-18.5.6}

\textbf{Given:} A coherent transceiver operates at 64 GBaud with DP-QPSK
modulation at 1550 nm. FEC overhead is 15\%.

\textbf{Find:} (a) The bits per symbol per polarization, (b) the raw
data rate, (c) the net data rate after FEC, and (d) the spectral
efficiency for a 50 GHz channel spacing.

\textbf{Solution:}

\begin{enumerate}
\def\labelenumi{(\alph{enumi})}
\item
  QPSK encodes 2 bits per symbol. With dual polarization: Bits per
  symbol = 2 × 2 = \textbf{4 bits/symbol total}
\item
  Raw data rate: R\textsubscript{raw} = 64 × 10⁹ × 4 = \textbf{256 Gbps}
\item
  Net data rate: R\textsubscript{net} = R\textsubscript{raw} / (1 + FEC
  overhead) = 256 / 1.15 = \textbf{222.6 Gbps} (approximately 200G
  class)
\item
  Spectral efficiency: SE = R\textsubscript{net} / channel spacing =
  222.6 / 50 = \textbf{4.45 b/s/Hz}
\end{enumerate}

This is close to the practical limit for QPSK with dual polarization
(theoretical max = 4 b/s/Hz before FEC overhead).

\begin{center}\rule{0.5\linewidth}{0.5pt}\end{center}

\section{Problem 18.5.7}\label{problem-18.5.7}

\textbf{Given:} A submarine optical cable system uses the C-band
(1530--1565 nm) with 80 channels at 50 GHz spacing, 400 Gbps per channel
using DP-16QAM at 60 GBaud. The system spans 6,000 km with 75 km
repeater spans. Each EDFA has NF = 4.5 dB and fiber loss is 0.20 dB/km.

\textbf{Find:} (a) The number of repeater spans, (b) the total system
capacity, (c) the OSNR after all spans, and (d) the OSNR margin for
DP-16QAM (required ≈ 16 dB with soft-decision FEC).

\textbf{Solution:}

\begin{enumerate}
\def\labelenumi{(\alph{enumi})}
\item
  Number of spans: N = 6,000 / 75 = \textbf{80 spans}
\item
  Total capacity: C = 80 channels × 400 Gbps = \textbf{32 Tbps} per
  fiber pair
\item
  OSNR: Span loss = 0.20 × 75 = 15 dB OSNR = 58 +
  P\textsubscript{launch} − L\textsubscript{span} − NF − 10log₁₀(N)
  Assuming P\textsubscript{launch} = −1 dBm per channel (typical
  submarine): = 58 + (−1) − 15 − 4.5 − 10log₁₀(80) = 58 − 1 − 15 − 4.5 −
  19.03 = \textbf{18.5 dB}
\item
  Margin: Margin = 18.5 − 16 = \textbf{2.5 dB}
\end{enumerate}

This margin is tight but typical for submarine systems, which use
probabilistic constellation shaping and advanced FEC to operate with
thin margins.

\begin{center}\rule{0.5\linewidth}{0.5pt}\end{center}

\section{Problem 18.5.8}\label{problem-18.5.8}

\textbf{Given:} A fiber Bragg grating (FBG) is written in single-mode
fiber with effective refractive index n\textsubscript{eff} = 1.447 and
grating period Λ = 535.25 nm. The FBG length is 10 mm with index
modulation Δn = 5 × 10⁻⁴.

\textbf{Find:} (a) The Bragg wavelength, (b) the reflection bandwidth,
and (c) the peak reflectivity.

\textbf{Solution:}

\begin{enumerate}
\def\labelenumi{(\alph{enumi})}
\tightlist
\item
  Bragg wavelength: λ\textsubscript{B} = 2 × n\textsubscript{eff} × Λ =
  2 × 1.447 × 535.25 = \textbf{1,549.0 nm}
\end{enumerate}

This is in the C-band, making it suitable for DWDM channel filtering.

\begin{enumerate}
\def\labelenumi{(\alph{enumi})}
\setcounter{enumi}{1}
\item
  Reflection bandwidth (for strong grating, Δn
  \textgreater\textgreater{} λ/(πL)): Δλ = λ\textsubscript{B} ×
  √((Δn/n\textsubscript{eff})² +
  (λ\textsubscript{B}/(n\textsubscript{eff} × L))²)
  Δn/n\textsubscript{eff} = 5 × 10⁻⁴ / 1.447 = 3.455 × 10⁻⁴
  λ\textsubscript{B}/(n\textsubscript{eff} × L) = 1549 × 10⁻⁹ / (1.447 ×
  0.01) = 1.070 × 10⁻⁴ Δλ = 1549 × √((3.455 × 10⁻⁴)² + (1.070 × 10⁻⁴)²)
  = 1549 × √(1.194 × 10⁻⁷ + 1.145 × 10⁻⁸) = 1549 × √(1.308 × 10⁻⁷) =
  1549 × 3.617 × 10⁻⁴ = \textbf{0.560 nm}
\item
  Peak reflectivity: κ = πΔn / λ\textsubscript{B} = π × 5 × 10⁻⁴ / 1549
  × 10⁻⁹ = 1013.6 m⁻¹ κL = 1013.6 × 0.01 = 10.136 R = tanh²(κL) =
  tanh²(10.136) ≈ \textbf{99.99\%} (tanh is essentially 1 for such large
  arguments)
\end{enumerate}

\begin{center}\rule{0.5\linewidth}{0.5pt}\end{center}

\section{Problem 18.5.9}\label{problem-18.5.9}

\textbf{Given:} An optical time-domain reflectometer (OTDR) operates at
1550 nm with a pulse width of 100 ns and peak power of 100 mW in a fiber
with n = 1.468 and attenuation of 0.22 dB/km.

\textbf{Find:} (a) The spatial resolution (two-point), (b) the maximum
range for a 30 dB dynamic range (one-way loss limit), (c) the round-trip
time for a reflection at 50 km, and (d) the distance to an event that
appears at t = 600 μs on the OTDR trace.

\textbf{Solution:}

\begin{enumerate}
\def\labelenumi{(\alph{enumi})}
\tightlist
\item
  Spatial resolution: Δz = c × τ / (2n) = 3 × 10⁸ × 100 × 10⁻⁹ / (2 ×
  1.468) = 30 / 2.936 = \textbf{10.22 m}
\end{enumerate}

Two events closer than 10.2 m cannot be resolved as separate features.

\begin{enumerate}
\def\labelenumi{(\alph{enumi})}
\setcounter{enumi}{1}
\tightlist
\item
  Maximum range: L\textsubscript{max} = dynamic range / (2 × α) --- but
  wait, OTDR measures round-trip, so the one-way range based on
  backscatter: L\textsubscript{max} = dynamic range / α = 30 / 0.22 =
  \textbf{136.4 km} (one-way fiber loss of 30 dB)
\end{enumerate}

In practice, the round-trip attenuation and backscatter coefficient
reduce this, but the dynamic range specification accounts for the round
trip.

\begin{enumerate}
\def\labelenumi{(\alph{enumi})}
\setcounter{enumi}{2}
\item
  Round-trip time at 50 km: t = 2nL / c = 2 × 1.468 × 50 × 10³ / (3 ×
  10⁸) = 146,800 / 3 × 10⁸ = \textbf{489.3 μs}
\item
  Distance for t = 600 μs: L = c × t / (2n) = 3 × 10⁸ × 600 × 10⁻⁶ / (2
  × 1.468) = 180,000 / 2.936 = \textbf{61.3 km}
\end{enumerate}

\begin{center}\rule{0.5\linewidth}{0.5pt}\end{center}

\section{Problem 18.5.10}\label{problem-18.5.10}

\textbf{Given:} A data center interconnect (DCI) uses 400G-ZR coherent
pluggable optics. The transceiver operates at 60 GBaud, DP-16QAM, with
20\% FEC overhead. The link is 80 km of standard SMF (0.2 dB/km at 1550
nm) with 4 connectors at 0.3 dB each. Transmit power is 0 dBm and
receiver sensitivity is −18 dBm.

\textbf{Find:} (a) The raw and net data rates, (b) the total link loss,
(c) the received power and system margin, and (d) the dispersion
accumulation and whether it is compensated.

\textbf{Solution:}

\begin{enumerate}
\def\labelenumi{(\alph{enumi})}
\item
  Data rates: Raw: 60 GBaud × 4 bits/symbol × 2 polarizations =
  \textbf{480 Gbps} Net: 480 / 1.20 = \textbf{400 Gbps} (400G ZR)
\item
  Total link loss: Fiber: 0.2 × 80 = 16.0 dB Connectors: 4 × 0.3 = 1.2
  dB Total = 16.0 + 1.2 = \textbf{17.2 dB}
\item
  Received power: P\textsubscript{rx} = 0 − 17.2 = \textbf{−17.2 dBm}
  Margin = P\textsubscript{rx} − P\textsubscript{sens} = −17.2 − (−18) =
  \textbf{0.8 dB}
\end{enumerate}

This is tight. Aging and temperature variations may push the link below
threshold. A mid-span EDFA or higher launch power would be advisable.

\begin{enumerate}
\def\labelenumi{(\alph{enumi})}
\setcounter{enumi}{3}
\tightlist
\item
  Dispersion: Δτ\textsubscript{CD} = D × L = 17 × 80 = 1,360 ps/nm of
  accumulated chromatic dispersion. At 60 GBaud, the symbol period is
  16.67 ps. The accumulated dispersion is very large, but coherent
  receivers use DSP-based chromatic dispersion compensation (CDC) that
  can compensate tens of thousands of ps/nm digitally. The 1,360 ps/nm
  is \textbf{well within the ±50,000 ps/nm CDC range} of modern coherent
  DSP ASICs, so no optical dispersion compensation is needed.
\end{enumerate}

\chapter{Chapter 18 --- Section 18.6: Optical System
Design}\label{chapter-18-section-18.6-optical-system-design}

Practice problems covering optical power and radiometry, optical link
budgets, and lens design and aberrations.

\begin{center}\rule{0.5\linewidth}{0.5pt}\end{center}

\section{Problem 18.6.1}\label{problem-18.6.1}

\textbf{Given:} An LED emits 80 lumens of white light with a luminous
efficacy of 120 lm/W. The LED has a Lambertian emission pattern (uniform
radiant intensity over a hemisphere).

\textbf{Find:} (a) The radiant flux (optical power), (b) the radiant
intensity, (c) the irradiance at a distance of 1 m on axis, and (d) the
illuminance at 1 m.

\textbf{Solution:}

\begin{enumerate}
\def\labelenumi{(\alph{enumi})}
\item
  Radiant flux: Φ = luminous flux / luminous efficacy = 80 / 120 =
  \textbf{0.667 W}
\item
  For a Lambertian source, the intensity varies as I(θ) = I₀ cos(θ).
  Total flux: Φ = π × I₀ (integrating over hemisphere). I₀ = Φ/π =
  0.667/π = \textbf{0.2122 W/sr} (on-axis intensity)
\item
  Irradiance at 1 m on axis: E = I₀/r² = 0.2122/1² = \textbf{0.2122
  W/m²}
\item
  Illuminance at 1 m: For a Lambertian source, the luminous intensity on
  axis is: I\textsubscript{v} = luminous flux / π = 80/π = 25.46 cd
  Illuminance: E\textsubscript{v} = I\textsubscript{v}/r² = 25.46/1 =
  \textbf{25.46 lux}
\end{enumerate}

\begin{center}\rule{0.5\linewidth}{0.5pt}\end{center}

\section{Problem 18.6.2}\label{problem-18.6.2}

\textbf{Given:} A security camera lens has focal length f = 8 mm and
aperture diameter D = 4 mm. The sensor has 2.0 μm pixel pitch and
operates at λ = 550 nm.

\textbf{Find:} (a) The f-number, (b) the diffraction-limited Airy disk
diameter, (c) whether the system is diffraction-limited or
pixel-limited, and (d) the angular field of view if the sensor is 6.4 mm
wide.

\textbf{Solution:}

\begin{enumerate}
\def\labelenumi{(\alph{enumi})}
\item
  f-number: f/\# = f/D = 8/4 = \textbf{f/2.0}
\item
  Airy disk diameter: d\textsubscript{Airy} = 2.44 × λ × f/\# = 2.44 ×
  550 × 10⁻⁹ × 2.0 = 2.44 × 1.1 × 10⁻⁶ = \textbf{2.68 μm}
\item
  Comparison with pixel size: Airy disk diameter (2.68 μm) is larger
  than the pixel pitch (2.0 μm). The system is
  \textbf{diffraction-limited} --- the optics, not the sensor, limit
  resolution. The Airy disk spans approximately 1.34 pixels, so the
  Nyquist condition (at least 2 pixels per resolution element) is not
  met. The effective resolution is limited by aliasing.
\item
  Angular field of view: θ = 2 × arctan(w/(2f)) = 2 × arctan(6.4/(2 ×
  8)) = 2 × arctan(0.4) = 2 × 21.8° = \textbf{43.6°}
\end{enumerate}

\begin{center}\rule{0.5\linewidth}{0.5pt}\end{center}

\section{Problem 18.6.3}\label{problem-18.6.3}

\textbf{Given:} A free-space optical (FSO) communication link operates
at λ = 1550 nm between two buildings 500 m apart. The transmitter has a
collimated beam with initial diameter D\textsubscript{tx} = 25 mm and
divergence θ = 0.5 mrad. The receiver has an aperture diameter
D\textsubscript{rx} = 100 mm. Transmit power is +20 dBm and receiver
sensitivity is −35 dBm.

\textbf{Find:} (a) The beam diameter at the receiver, (b) the geometric
loss, (c) the received power, and (d) the system margin.

\textbf{Solution:}

\begin{enumerate}
\def\labelenumi{(\alph{enumi})}
\item
  Beam diameter at receiver: D\textsubscript{beam} = D\textsubscript{tx}
  + θ × L = 0.025 + 0.5 × 10⁻³ × 500 = 0.025 + 0.250 = \textbf{0.275 m =
  275 mm}
\item
  Geometric (spreading) loss --- ratio of receiver aperture area to beam
  area: L\textsubscript{geo} =
  (D\textsubscript{rx}/D\textsubscript{beam})² = (100/275)² = (0.3636)²
  = 0.1322 In dB: L\textsubscript{geo} = 10log₁₀(0.1322) = \textbf{−8.79
  dB}
\item
  Received power (ignoring atmospheric absorption): P\textsubscript{rx}
  = P\textsubscript{tx} + L\textsubscript{geo} = 20 + (−8.79) =
  \textbf{+11.2 dBm}
\item
  System margin: M = P\textsubscript{rx} − P\textsubscript{sens} = 11.2
  − (−35) = \textbf{46.2 dB}
\end{enumerate}

This large margin accommodates atmospheric attenuation (fog, rain),
scintillation, and beam pointing errors. Dense fog can cause 100+ dB/km
attenuation, which would exceed the margin at 500 m (50 dB loss).
Clear-air attenuation is typically 0.5--5 dB/km.

\begin{center}\rule{0.5\linewidth}{0.5pt}\end{center}

\section{Problem 18.6.4}\label{problem-18.6.4}

\textbf{Given:} A camera lens system has the following specifications:
focal length f = 50 mm, f-number f/1.8, and the image of a point source
has a measured spot diameter due to aberrations of
d\textsubscript{aberr} = 15 μm. Operating wavelength is λ = 550 nm.

\textbf{Find:} (a) The aperture diameter, (b) the diffraction-limited
spot diameter (Airy disk), (c) the ratio of aberration blur to
diffraction limit, and (d) the effective combined spot diameter
(root-sum-square approximation).

\textbf{Solution:}

\begin{enumerate}
\def\labelenumi{(\alph{enumi})}
\item
  Aperture diameter: D = f/(f/\#) = 50/1.8 = \textbf{27.8 mm}
\item
  Diffraction-limited Airy disk diameter: d\textsubscript{Airy} = 2.44 ×
  λ × f/\# = 2.44 × 550 × 10⁻⁹ × 1.8 = 2.44 × 9.9 × 10⁻⁷ = \textbf{2.42
  μm}
\item
  Ratio: d\textsubscript{aberr}/d\textsubscript{Airy} = 15/2.42 =
  \textbf{6.2×}
\end{enumerate}

The system is aberration-limited, with the actual spot 6.2 times the
diffraction limit.

\begin{enumerate}
\def\labelenumi{(\alph{enumi})}
\setcounter{enumi}{3}
\tightlist
\item
  Combined spot diameter (RSS): d\textsubscript{total} =
  √(d\textsubscript{aberr}² + d\textsubscript{Airy}²) = √(15² + 2.42²) =
  √(225 + 5.86) = √230.86 = \textbf{15.2 μm}
\end{enumerate}

The aberration dominates; the diffraction contribution is negligible at
f/1.8. Stopping down to f/5.6 would increase d\textsubscript{Airy} to
7.5 μm while reducing spherical aberration by roughly (1.8/5.6)³ ≈ 3.3×,
potentially reducing d\textsubscript{aberr} to \textasciitilde4.5 μm and
achieving a better-balanced design.

\begin{center}\rule{0.5\linewidth}{0.5pt}\end{center}

\section{Problem 18.6.5}\label{problem-18.6.5}

\textbf{Given:} An optical fiber link at 1310 nm uses the following
components: transmitter power = +1 dBm, single-mode fiber with α = 0.35
dB/km over 25 km, 2 connector pairs at 0.5 dB each, 1 fusion splice at
0.1 dB, and a 3 dB coupler (for monitoring tap). The receiver
sensitivity is −30 dBm.

\textbf{Find:} (a) The total link loss, (b) the received power, (c) the
system margin, and (d) the maximum additional fiber length that could be
added while maintaining 3 dB margin.

\textbf{Solution:}

\begin{enumerate}
\def\labelenumi{(\alph{enumi})}
\item
  Total link loss: Fiber: 0.35 × 25 = 8.75 dB Connectors: 2 × 0.5 = 1.0
  dB Splice: 1 × 0.1 = 0.1 dB Coupler: 3.0 dB Total = 8.75 + 1.0 + 0.1 +
  3.0 = \textbf{12.85 dB}
\item
  Received power: P\textsubscript{rx} = +1 − 12.85 = \textbf{−11.85 dBm}
\item
  System margin: M = P\textsubscript{rx} − P\textsubscript{sens} =
  −11.85 − (−30) = \textbf{18.15 dB}
\item
  Additional fiber with 3 dB margin: Available for additional fiber =
  18.15 − 3.0 = 15.15 dB L\textsubscript{additional} = 15.15 / 0.35 =
  \textbf{43.3 km} (total link would be 68.3 km)
\end{enumerate}

\begin{center}\rule{0.5\linewidth}{0.5pt}\end{center}

\section{Problem 18.6.6}\label{problem-18.6.6}

\textbf{Given:} A projector uses a 5000-lumen lamp. The projection lens
has f/2.5, and the screen is 3 m wide by 2 m tall at a distance of 5 m.

\textbf{Find:} (a) The screen area, (b) the screen illuminance (assuming
60\% optical efficiency and uniform distribution), (c) the screen
luminance for a matte white screen (gain = 1.0), and (d) the required
lamp power if the luminous efficacy is 90 lm/W.

\textbf{Solution:}

\begin{enumerate}
\def\labelenumi{(\alph{enumi})}
\item
  Screen area: A = 3 × 2 = \textbf{6 m²}
\item
  Screen illuminance: Effective lumens on screen:
  Φ\textsubscript{screen} = 5,000 × 0.60 = 3,000 lm E\textsubscript{v} =
  Φ\textsubscript{screen} / A = 3,000 / 6 = \textbf{500 lux}
\item
  Screen luminance: For a Lambertian (matte) screen with gain 1.0: L =
  E\textsubscript{v} / π = 500 / π = \textbf{159.2 cd/m²}
\end{enumerate}

This is adequate for a moderately lit room (recommended \textgreater{}
100 cd/m² for projection displays).

\begin{enumerate}
\def\labelenumi{(\alph{enumi})}
\setcounter{enumi}{3}
\tightlist
\item
  Required lamp power: P = total lumens / efficacy = 5,000 / 90 =
  \textbf{55.6 W}
\end{enumerate}

\begin{center}\rule{0.5\linewidth}{0.5pt}\end{center}

\section{Problem 18.6.7}\label{problem-18.6.7}

\textbf{Given:} An achromatic doublet lens is designed for f = 200 mm by
combining a BK7 crown glass element (n\textsubscript{d} = 1.5168, Abbe
number V₁ = 64.17) and an SF2 flint glass element (n\textsubscript{d} =
1.6477, Abbe number V₂ = 33.85). The doublet must have zero chromatic
aberration.

\textbf{Find:} (a) The individual focal lengths f₁ and f₂ of the crown
and flint elements (using the achromatic condition: φ₁/V₁ + φ₂/V₂ = 0,
where φ = 1/f), and (b) the optical powers.

\textbf{Solution:}

\begin{enumerate}
\def\labelenumi{(\alph{enumi})}
\tightlist
\item
  For an achromatic doublet, two conditions must be met:
\end{enumerate}

\begin{itemize}
\tightlist
\item
  Total power: φ = φ₁ + φ₂ = 1/200 = 0.005 mm⁻¹
\item
  Achromatic condition: φ₁/V₁ + φ₂/V₂ = 0
\end{itemize}

From the achromatic condition: φ₁/64.17 + φ₂/33.85 = 0 φ₁ = −φ₂ ×
(64.17/33.85) = −1.896 × φ₂

Substituting into total power: −1.896φ₂ + φ₂ = 0.005 −0.896φ₂ = 0.005 φ₂
= −0.005578 mm⁻¹ f₂ = 1/φ₂ = \textbf{−179.3 mm} (diverging flint
element)

φ₁ = 0.005 − (−0.005578) = 0.010578 mm⁻¹ f₁ = 1/φ₁ = \textbf{94.5 mm}
(converging crown element)

\begin{enumerate}
\def\labelenumi{(\alph{enumi})}
\setcounter{enumi}{1}
\tightlist
\item
  Optical powers: Crown: φ₁ = \textbf{+10.578 diopters} (converging)
  Flint: φ₂ = \textbf{−5.578 diopters} (diverging)
\end{enumerate}

The converging crown element is much stronger than the combined doublet,
and the flint element partially cancels the power while correcting the
chromatic aberration.

\begin{center}\rule{0.5\linewidth}{0.5pt}\end{center}

\section{Problem 18.6.8}\label{problem-18.6.8}

\textbf{Given:} A solar concentrator uses a parabolic mirror with
diameter D = 2 m and focal length f = 1 m. The solar irradiance is 1000
W/m² and the mirror reflectivity is 92\%. The concentrated sunlight is
directed onto a circular receiver of diameter d = 20 mm.

\textbf{Find:} (a) The collection area, (b) the power collected by the
mirror, (c) the concentration ratio, and (d) the irradiance at the
receiver.

\textbf{Solution:}

\begin{enumerate}
\def\labelenumi{(\alph{enumi})}
\item
  Collection area: A\textsubscript{mirror} = π(D/2)² = π(1)² =
  \textbf{3.142 m²}
\item
  Collected power: P = E\textsubscript{solar} × A\textsubscript{mirror}
  × ρ = 1000 × 3.142 × 0.92 = \textbf{2,890 W}
\item
  Geometric concentration ratio: C = A\textsubscript{mirror} /
  A\textsubscript{receiver} = π(1)² / π(0.01)² = 1/0.0001 =
  \textbf{10,000×}
\item
  Irradiance at receiver: E\textsubscript{receiver} = P /
  A\textsubscript{receiver} = 2,890 / (π × (0.01)²) = 2,890 / 3.142 ×
  10⁻⁴ = \textbf{9.2 × 10⁶ W/m² = 9.2 MW/m²}
\end{enumerate}

This is 9,200 times the solar irradiance --- sufficient to melt most
metals. Such concentrators are used in solar furnaces and concentrated
solar power (CSP) systems.

\begin{center}\rule{0.5\linewidth}{0.5pt}\end{center}

\section{Problem 18.6.9}\label{problem-18.6.9}

\textbf{Given:} A machine vision system uses a telecentric lens with
magnification m = −0.5, working distance = 200 mm, and sensor pixel size
= 5 μm. The lens has a modulation transfer function (MTF) of 50\% at 100
lp/mm at the sensor plane.

\textbf{Find:} (a) The object-space pixel resolution, (b) the field of
view if the sensor is 12.8 mm × 9.6 mm, (c) the Nyquist frequency of the
sensor, and (d) whether the MTF supports the Nyquist resolution.

\textbf{Solution:}

\begin{enumerate}
\def\labelenumi{(\alph{enumi})}
\item
  Object-space pixel resolution: Δx\textsubscript{object} = pixel size /
  \textbar m\textbar{} = 5 μm / 0.5 = \textbf{10 μm} per pixel
\item
  Field of view: FOV\textsubscript{x} = sensor width /
  \textbar m\textbar{} = 12.8 / 0.5 = \textbf{25.6 mm}
  FOV\textsubscript{y} = 9.6 / 0.5 = \textbf{19.2 mm}
\item
  Nyquist frequency at the sensor: f\textsubscript{Nyquist} = 1 / (2 ×
  pixel size) = 1 / (2 × 5 × 10⁻³ mm) = \textbf{100 lp/mm}
\item
  The MTF at the Nyquist frequency is 50\%. This means the lens can
  resolve features at the Nyquist limit but with 50\% contrast
  reduction. A common guideline requires MTF \textgreater{} 30\% at
  Nyquist for acceptable imaging, so this lens \textbf{meets the
  requirement} with margin. Features at half the Nyquist frequency (50
  lp/mm) would have even higher contrast.
\end{enumerate}

\begin{center}\rule{0.5\linewidth}{0.5pt}\end{center}

\section{Problem 18.6.10}\label{problem-18.6.10}

\textbf{Given:} A fiber-optic endoscope uses a coherent fiber bundle
with 30,000 individual fibers, each with core diameter of 3 μm and
center-to-center spacing of 4 μm, arranged in a hexagonal pattern. The
distal lens has f = 2 mm and f/3.

\textbf{Find:} (a) The bundle diameter (assuming circular packing), (b)
the effective pixel count, (c) the angular resolution limit from the
fiber spacing, and (d) the diffraction-limited resolution of the distal
lens at λ = 550 nm.

\textbf{Solution:}

\begin{enumerate}
\def\labelenumi{(\alph{enumi})}
\item
  Bundle diameter: For hexagonal packing, the area per fiber is
  A\textsubscript{hex} = (√3/2) × s² where s = 4 μm spacing.
  A\textsubscript{hex} = (√3/2) × (4 × 10⁻³)² = 0.866 × 16 × 10⁻⁶ =
  13.86 × 10⁻⁶ mm² Total bundle area = 30,000 × 13.86 × 10⁻⁶ = 0.4157
  mm² Bundle diameter = √(4A/π) = √(4 × 0.4157 / π) = √(0.5295) =
  \textbf{0.728 mm}
\item
  Each fiber acts as one pixel: Effective resolution = \textbf{30,000
  pixels}
\end{enumerate}

Equivalent image format: √30,000 ≈ 173, so roughly a 200 × 150 pixel
image.

\begin{enumerate}
\def\labelenumi{(\alph{enumi})}
\setcounter{enumi}{2}
\item
  Angular resolution from fiber spacing: θ\textsubscript{fiber} = s / f
  = 4 × 10⁻³ / 2 = 2 × 10⁻³ rad = \textbf{2 mrad}
\item
  Diffraction limit of distal lens: D = f/(f/\#) = 2/3 = 0.667 mm
  θ\textsubscript{diff} = 1.22λ/D = 1.22 × 550 × 10⁻⁹ / 0.667 × 10⁻³ =
  6.71 × 10⁻⁷ / 6.67 × 10⁻⁴ = \textbf{1.006 × 10⁻³ rad = 1.0 mrad}
\end{enumerate}

The fiber spacing (2 mrad) is the limiting factor, not diffraction (1
mrad). The system is \textbf{sampling-limited} by the fiber bundle,
which is typical for fiber endoscopes.

\chapter{Chapter 19 --- Section 19.1: Time Value of
Money}\label{chapter-19-section-19.1-time-value-of-money}

Practice problems covering simple interest, compound interest, nominal
and effective interest rates, and continuous compounding. Problems range
from straightforward applications to multi-step PE-exam-style questions.

\begin{center}\rule{0.5\linewidth}{0.5pt}\end{center}

\section{Problem 19.1.1}\label{problem-19.1.1}

\textbf{Given:} A municipality borrows \$750,000 for a streetlight LED
conversion project at 5\% simple annual interest for 4 years.

\textbf{Find:} (a) The total interest paid, and (b) the total amount
owed at maturity.

\textbf{Solution:}

\begin{enumerate}
\def\labelenumi{(\alph{enumi})}
\item
  Total interest: I = P × i × n = 750,000 × 0.05 × 4 =
  \textbf{\$150,000}
\item
  Total amount owed: F = P + I = 750,000 + 150,000 = \textbf{\$900,000}
\end{enumerate}

\begin{center}\rule{0.5\linewidth}{0.5pt}\end{center}

\section{Problem 19.1.2}\label{problem-19.1.2}

\textbf{Given:} An engineering firm deposits \$250,000 into a project
escrow account earning 4\% annual interest compounded annually for 6
years.

\textbf{Find:} (a) The account balance after 6 years, and (b) the total
interest earned.

\textbf{Solution:}

\begin{enumerate}
\def\labelenumi{(\alph{enumi})}
\item
  F = P(1 + i)ⁿ = 250,000 × (1.04)⁶ = 250,000 × 1.2653 =
  \textbf{\$316,325}
\item
  Total interest: I = F − P = 316,325 − 250,000 = \textbf{\$66,325}
\end{enumerate}

\begin{center}\rule{0.5\linewidth}{0.5pt}\end{center}

\section{Problem 19.1.3}\label{problem-19.1.3}

\textbf{Given:} A contractor invests \$180,000 at 6\% annual interest
compounded monthly for 5 years.

\textbf{Find:} (a) The future value, and (b) the difference between
compound and simple interest over the same period.

\textbf{Solution:}

\begin{enumerate}
\def\labelenumi{(\alph{enumi})}
\item
  Monthly rate: i = 0.06/12 = 0.005. Periods: n = 5 × 12 = 60.
  F\textsubscript{compound} = 180,000 × (1.005)⁶⁰ = 180,000 × 1.3489 =
  \textbf{\$242,802}
\item
  Simple interest future value: F\textsubscript{simple} = P(1 + i × n) =
  180,000 × (1 + 0.06 × 5) = 180,000 × 1.30 = \$234,000
\end{enumerate}

Difference = 242,802 − 234,000 = \textbf{\$8,802}

Compound interest earns \$8,802 more than simple interest over 5 years
due to interest-on-interest.

\begin{center}\rule{0.5\linewidth}{0.5pt}\end{center}

\section{Problem 19.1.4}\label{problem-19.1.4}

\textbf{Given:} A power plant financing option quotes 9\% annual
interest compounded quarterly.

\textbf{Find:} (a) The effective annual interest rate, (b) the effective
quarterly rate, and (c) the effective monthly rate.

\textbf{Solution:}

\begin{enumerate}
\def\labelenumi{(\alph{enumi})}
\item
  i\textsubscript{eff,annual} = (1 + r/m)ᵐ − 1 = (1 + 0.09/4)⁴ − 1 =
  (1.0225)⁴ − 1 = 1.09308 − 1 = \textbf{9.31\%}
\item
  The effective quarterly rate is the periodic rate itself:
  i\textsubscript{quarterly} = 0.09/4 = \textbf{2.25\%}
\item
  Effective monthly rate from the quarterly rate:
  i\textsubscript{monthly} = (1 + 0.0225)\textsuperscript{1/3} − 1 =
  (1.0225)\textsuperscript{0.3333} − 1 = 1.00744 − 1 = \textbf{0.744\%}
\end{enumerate}

\begin{center}\rule{0.5\linewidth}{0.5pt}\end{center}

\section{Problem 19.1.5}\label{problem-19.1.5}

\textbf{Given:} Two equipment lease options are offered. Option A quotes
8.5\% compounded monthly. Option B quotes 8.7\% compounded semiannually.

\textbf{Find:} Which option has the lower effective annual rate?

\textbf{Solution:}

Option A: i\textsubscript{eff} = (1 + 0.085/12)¹² − 1 = (1.007083)¹² − 1
= 1.08839 − 1 = \textbf{8.84\%}

Option B: i\textsubscript{eff} = (1 + 0.087/2)² − 1 = (1.0435)² − 1 =
1.08889 − 1 = \textbf{8.89\%}

\textbf{Option A has the lower effective annual rate} (8.84\%
vs.~8.89\%), so it is the less expensive financing option despite having
a lower nominal rate.

\begin{center}\rule{0.5\linewidth}{0.5pt}\end{center}

\section{Problem 19.1.6}\label{problem-19.1.6}

\textbf{Given:} A utility invests \$5,000,000 in a decommissioning fund
at 6\% continuously compounded interest for 20 years.

\textbf{Find:} (a) The fund value after 20 years, (b) the effective
annual interest rate, and (c) the total interest earned.

\textbf{Solution:}

\begin{enumerate}
\def\labelenumi{(\alph{enumi})}
\item
  F = Pe\textsuperscript{rn} = 5,000,000 × e\textsuperscript{0.06 × 20}
  = 5,000,000 × e\textsuperscript{1.20} = 5,000,000 × 3.3201 =
  \textbf{\$16,600,584}
\item
  i\textsubscript{eff} = e\textsuperscript{0.06} − 1 = 1.06184 − 1 =
  \textbf{6.18\%}
\item
  Total interest: I = 16,600,584 − 5,000,000 = \textbf{\$11,600,584}
\end{enumerate}

\begin{center}\rule{0.5\linewidth}{0.5pt}\end{center}

\section{Problem 19.1.7}\label{problem-19.1.7}

\textbf{Given:} A wind farm developer needs to compare three investment
options for a \$1,000,000 reserve fund over 10 years: - Option X: 7\%
compounded annually - Option Y: 6.8\% compounded daily (365 days) -
Option Z: 6.7\% compounded continuously

\textbf{Find:} The future value under each option and which yields the
highest return.

\textbf{Solution:}

Option X: F = 1,000,000 × (1.07)¹⁰ = 1,000,000 × 1.9672 =
\textbf{\$1,967,151}

Option Y: i\textsubscript{period} = 0.068/365 = 0.000186301; n = 365 ×
10 = 3,650 F = 1,000,000 × (1.000186301)³⁶⁵⁰ = 1,000,000 ×
e\textsuperscript{0.068 × 10} ≈ 1,000,000 × (1 + 0.068/365)³⁶⁵⁰

Effective annual rate for Y: i\textsubscript{eff} = (1 + 0.068/365)³⁶⁵ −
1 = 1.07036 − 1 = 7.036\% F = 1,000,000 × (1.07036)¹⁰ = 1,000,000 ×
1.9738 = \textbf{\$1,973,812}

Option Z: F = 1,000,000 × e\textsuperscript{0.067 × 10} = 1,000,000 ×
e\textsuperscript{0.67} = 1,000,000 × 1.9542 = \textbf{\$1,954,237}

\textbf{Option Y yields the highest return} (\$1,973,812) because its
effective annual rate of 7.04\% exceeds the other options.

\begin{center}\rule{0.5\linewidth}{0.5pt}\end{center}

\section{Problem 19.1.8}\label{problem-19.1.8}

\textbf{Given:} A \$400,000 substation monitoring system is financed at
10\% compounded continuously. The loan must be repaid as a single lump
sum after 3 years.

\textbf{Find:} (a) The amount owed at repayment, (b) the total interest,
and (c) how much less would be owed if the same nominal rate were
compounded annually instead.

\textbf{Solution:}

\begin{enumerate}
\def\labelenumi{(\alph{enumi})}
\item
  Continuous compounding: F\textsubscript{cont} = 400,000 ×
  e\textsuperscript{0.10 × 3} = 400,000 × e\textsuperscript{0.30} =
  400,000 × 1.3499 = \textbf{\$539,944}
\item
  Total interest: I = 539,944 − 400,000 = \textbf{\$139,944}
\item
  Annual compounding: F\textsubscript{annual} = 400,000 × (1.10)³ =
  400,000 × 1.3310 = \$532,400 Difference = 539,944 − 532,400 =
  \textbf{\$7,544}
\end{enumerate}

The continuous compounding costs \$7,544 more in interest than annual
compounding at the same nominal rate.

\chapter{Chapter 19 --- Section 19.2: Cash Flow Diagrams and Economic
Equivalence}\label{chapter-19-section-19.2-cash-flow-diagrams-and-economic-equivalence}

Practice problems covering cash flow diagram conventions and economic
equivalence calculations. Problems range from straightforward diagram
interpretation to multi-step equivalence comparisons.

\begin{center}\rule{0.5\linewidth}{0.5pt}\end{center}

\section{Problem 19.2.1}\label{problem-19.2.1}

\textbf{Given:} A solar inverter installation costs \$220,000 at time 0.
It produces energy savings of \$42,000/year for years 1 through 8,
requires a \$15,000 capacitor replacement at year 4, and has a salvage
value of \$18,000 at the end of year 8.

\textbf{Find:} (a) The net cash flow in years 1--3, year 4, years 5--7,
and year 8, and (b) the total undiscounted net cash flow over the
project.

\textbf{Solution:}

\begin{enumerate}
\def\labelenumi{(\alph{enumi})}
\tightlist
\item
  Cash flow diagram:
\end{enumerate}

\begin{itemize}
\tightlist
\item
  Time 0: ↓ \$220,000 (initial cost)
\item
  Years 1--3: ↑ \$42,000/year (annual savings)
\item
  Year 4: ↑ \$42,000 − ↓ \$15,000 = ↑ \textbf{\$27,000 net}
\item
  Years 5--7: ↑ \$42,000/year (annual savings)
\item
  Year 8: ↑ \$42,000 + ↑ \$18,000 = ↑ \textbf{\$60,000 net}
\end{itemize}

\begin{enumerate}
\def\labelenumi{(\alph{enumi})}
\setcounter{enumi}{1}
\tightlist
\item
  Total undiscounted net cash flow: = −220,000 + 7 × 42,000 + 27,000 +
  18,000 = −220,000 + 294,000 + 27,000 + 18,000 = −220,000 + 339,000 =
  \textbf{\$119,000}
\end{enumerate}

Note: This positive total does not guarantee profitability --- a
time-value-of-money analysis is required.

\begin{center}\rule{0.5\linewidth}{0.5pt}\end{center}

\section{Problem 19.2.2}\label{problem-19.2.2}

\textbf{Given:} A utility can pay \$1,200,000 now or make 5 equal annual
payments for a substation transformer bank. The interest rate is 7\%.

\textbf{Find:} (a) The equivalent annual payment that makes the
installment plan equal to the lump sum, and (b) the total paid under the
installment plan.

\textbf{Solution:}

\begin{enumerate}
\def\labelenumi{(\alph{enumi})}
\item
  The annual payment A that is equivalent to the present lump sum: A = P
  × (A/P, 7\%, 5) = 1,200,000 × {[}0.07 × (1.07)⁵{]} / {[}(1.07)⁵ − 1{]}
  = 1,200,000 × {[}0.07 × 1.4026{]} / {[}1.4026 − 1{]} = 1,200,000 ×
  0.09818 / 0.4026 = 1,200,000 × 0.24389 = \textbf{\$292,668/year}
\item
  Total paid: 292,668 × 5 = \textbf{\$1,463,340}
\end{enumerate}

The installment plan costs \$263,340 more in total, but the payments are
spread over 5 years.

\begin{center}\rule{0.5\linewidth}{0.5pt}\end{center}

\section{Problem 19.2.3}\label{problem-19.2.3}

\textbf{Given:} An industrial facility can lease a power quality
analyzer for \$8,500/year for 6 years or purchase it outright for
\$38,000 with a \$5,000 resale value at the end of 6 years. The
facility's MARR is 10\%.

\textbf{Find:} (a) The present worth of the lease option, (b) the
present worth of the purchase option, and (c) which option is more
economical.

\textbf{Solution:}

\begin{enumerate}
\def\labelenumi{(\alph{enumi})}
\item
  PW\textsubscript{lease} = 8,500 × (P/A, 10\%, 6) (P/A, 10\%, 6) =
  {[}(1.10)⁶ − 1{]} / {[}0.10 × (1.10)⁶{]} = {[}1.7716 − 1{]} / {[}0.10
  × 1.7716{]} = 0.7716 / 0.17716 = 4.3553 PW\textsubscript{lease} =
  8,500 × 4.3553 = \textbf{\$37,020}
\item
  PW\textsubscript{purchase} = 38,000 − 5,000 × (P/F, 10\%, 6) (P/F,
  10\%, 6) = 1/(1.10)⁶ = 1/1.7716 = 0.5645 PW\textsubscript{purchase} =
  38,000 − 5,000 × 0.5645 = 38,000 − 2,823 = \textbf{\$35,177}
\item
  The \textbf{purchase option is more economical} by \$37,020 − \$35,177
  = \$1,843 in present worth, considering the resale value offsets the
  higher upfront cost.
\end{enumerate}

\begin{center}\rule{0.5\linewidth}{0.5pt}\end{center}

\section{Problem 19.2.4}\label{problem-19.2.4}

\textbf{Given:} A data center operator receives two proposals for an
emergency generator system. Proposal 1 requires \$500,000 now and
\$50,000 at year 3 for a major service. Proposal 2 requires \$200,000
now, \$200,000 at year 2, and \$200,000 at year 4. Both proposals
deliver identical service over the same period. The interest rate is
9\%.

\textbf{Find:} (a) The present worth of each proposal, and (b) the more
economical choice.

\textbf{Solution:}

\begin{enumerate}
\def\labelenumi{(\alph{enumi})}
\tightlist
\item
  PW of Proposal 1: PW₁ = 500,000 + 50,000 × (P/F, 9\%, 3) (P/F, 9\%, 3)
  = 1/(1.09)³ = 1/1.2950 = 0.7722 PW₁ = 500,000 + 50,000 × 0.7722 =
  500,000 + 38,610 = \textbf{\$538,610}
\end{enumerate}

PW of Proposal 2: PW₂ = 200,000 + 200,000 × (P/F, 9\%, 2) + 200,000 ×
(P/F, 9\%, 4) (P/F, 9\%, 2) = 1/(1.09)² = 1/1.1881 = 0.8417 (P/F, 9\%,
4) = 1/(1.09)⁴ = 1/1.4116 = 0.7084 PW₂ = 200,000 + 200,000 × 0.8417 +
200,000 × 0.7084 = 200,000 + 168,340 + 141,680 = \textbf{\$510,020}

\begin{enumerate}
\def\labelenumi{(\alph{enumi})}
\setcounter{enumi}{1}
\tightlist
\item
  \textbf{Proposal 2 is more economical} by \$538,610 − \$510,020 =
  \$28,590 in present worth. Despite requiring the same total
  undiscounted amount (\$600,000 each), the deferred payments in
  Proposal 2 are worth less in present-value terms.
\end{enumerate}

\chapter{Chapter 19 --- Section 19.3: Single Payment and Uniform Series
Factors}\label{chapter-19-section-19.3-single-payment-and-uniform-series-factors}

Practice problems covering F/P, P/F, P/A, A/P, A/F, F/A factors,
arithmetic gradient series, and geometric gradient series. Problems
range from direct factor application to multi-step PE-exam-style
questions.

\begin{center}\rule{0.5\linewidth}{0.5pt}\end{center}

\section{Problem 19.3.1}\label{problem-19.3.1}

\textbf{Given:} A utility sets aside \$2,000,000 today for future
transmission line right-of-way acquisition in 10 years. The fund earns
5\% annually.

\textbf{Find:} (a) The future value of the investment, and (b) the total
interest earned.

\textbf{Solution:}

\begin{enumerate}
\def\labelenumi{(\alph{enumi})}
\item
  F = P × (F/P, 5\%, 10) = 2,000,000 × (1.05)¹⁰ = 2,000,000 × 1.6289 =
  \textbf{\$3,257,789}
\item
  Interest earned: I = F − P = 3,257,789 − 2,000,000 =
  \textbf{\$1,257,789}
\end{enumerate}

\begin{center}\rule{0.5\linewidth}{0.5pt}\end{center}

\section{Problem 19.3.2}\label{problem-19.3.2}

\textbf{Given:} A cable replacement project will cost \$4,200,000 in 12
years. The utility's investment fund earns 7\% annually.

\textbf{Find:} The amount that must be invested today to cover the
future cost.

\textbf{Solution:}

P = F × (P/F, 7\%, 12) = 4,200,000 × 1/(1.07)¹² = 4,200,000 × 1/2.2522 =
4,200,000 × 0.4440 = \textbf{\$1,864,791}

The utility must invest approximately \$1.86 million today to have \$4.2
million available in 12 years.

\begin{center}\rule{0.5\linewidth}{0.5pt}\end{center}

\section{Problem 19.3.3}\label{problem-19.3.3}

\textbf{Given:} A manufacturing company purchases a \$650,000 automated
motor winding machine financed over 8 years at 6.5\% annual interest.

\textbf{Find:} (a) The uniform annual payment, (b) the total amount
paid, and (c) the total interest paid.

\textbf{Solution:}

\begin{enumerate}
\def\labelenumi{(\alph{enumi})}
\item
  A = P × (A/P, 6.5\%, 8) = 650,000 × {[}0.065 × (1.065)⁸{]} /
  {[}(1.065)⁸ − 1{]} (1.065)⁸ = 1.6550 = 650,000 × {[}0.065 × 1.6550{]}
  / {[}1.6550 − 1{]} = 650,000 × 0.10758 / 0.6550 = 650,000 × 0.16425 =
  \textbf{\$106,763/year}
\item
  Total paid: 106,763 × 8 = \textbf{\$854,104}
\item
  Total interest: 854,104 − 650,000 = \textbf{\$204,104}
\end{enumerate}

\begin{center}\rule{0.5\linewidth}{0.5pt}\end{center}

\section{Problem 19.3.4}\label{problem-19.3.4}

\textbf{Given:} A power company offers industrial customers a rebate
program that pays \$85,000/year for 12 years for installing
high-efficiency equipment. The company's discount rate is 9\%.

\textbf{Find:} The present worth of the entire rebate stream.

\textbf{Solution:}

P = A × (P/A, 9\%, 12) = 85,000 × {[}(1.09)¹² − 1{]} / {[}0.09 ×
(1.09)¹²{]} (1.09)¹² = 2.8127 = 85,000 × {[}2.8127 − 1{]} / {[}0.09 ×
2.8127{]} = 85,000 × 1.8127 / 0.25314 = 85,000 × 7.1607 =
\textbf{\$608,660}

\begin{center}\rule{0.5\linewidth}{0.5pt}\end{center}

\section{Problem 19.3.5}\label{problem-19.3.5}

\textbf{Given:} A utility must accumulate \$8,000,000 in 15 years for a
substation rebuild. The sinking fund earns 4.5\% annually.

\textbf{Find:} (a) The required uniform annual deposit, and (b) the
total amount deposited versus the total interest earned.

\textbf{Solution:}

\begin{enumerate}
\def\labelenumi{(\alph{enumi})}
\item
  A = F × (A/F, 4.5\%, 15) = 8,000,000 × 0.045 / {[}(1.045)¹⁵ − 1{]}
  (1.045)¹⁵ = 1.9353 = 8,000,000 × 0.045 / {[}1.9353 − 1{]} = 8,000,000
  × 0.045 / 0.9353 = 8,000,000 × 0.04811 = \textbf{\$384,904/year}
\item
  Total deposited: 384,904 × 15 = \$5,773,560 Interest earned: 8,000,000
  − 5,773,560 = \textbf{\$2,226,440}
\end{enumerate}

\begin{center}\rule{0.5\linewidth}{0.5pt}\end{center}

\section{Problem 19.3.6}\label{problem-19.3.6}

\textbf{Given:} An electrical contractor deposits \$25,000/year into a
business expansion fund earning 8\% annually for 10 years.

\textbf{Find:} The future value of the fund at the end of 10 years.

\textbf{Solution:}

F = A × (F/A, 8\%, 10) = 25,000 × {[}(1.08)¹⁰ − 1{]} / 0.08 (1.08)¹⁰ =
2.1589 = 25,000 × {[}2.1589 − 1{]} / 0.08 = 25,000 × 1.1589 / 0.08 =
25,000 × 14.487 = \textbf{\$362,168}

\begin{center}\rule{0.5\linewidth}{0.5pt}\end{center}

\section{Problem 19.3.7}\label{problem-19.3.7}

\textbf{Given:} Annual maintenance costs for a 138 kV circuit breaker
start at \$4,000 in year 1 and increase by \$600/year (arithmetic
gradient) for 15 years. The interest rate is 7\%.

\textbf{Find:} (a) The maintenance cost in year 15, (b) the present
worth of all maintenance costs, and (c) the equivalent uniform annual
maintenance cost.

\textbf{Solution:}

\begin{enumerate}
\def\labelenumi{(\alph{enumi})}
\item
  Cost in year 15: A₁₅ = 4,000 + (15 − 1) × 600 = 4,000 + 8,400 =
  \textbf{\$12,400}
\item
  P = A₁ × (P/A, 7\%, 15) + G × (P/G, 7\%, 15)
\end{enumerate}

(P/A, 7\%, 15) = {[}(1.07)¹⁵ − 1{]} / {[}0.07 × (1.07)¹⁵{]} (1.07)¹⁵ =
2.7590 = {[}2.7590 − 1{]} / {[}0.07 × 2.7590{]} = 1.7590 / 0.19313 =
9.1079

(P/G, 7\%, 15) = {[}(1.07)¹⁵ − 0.07 × 15 − 1{]} / {[}0.07² × (1.07)¹⁵{]}
= {[}2.7590 − 1.05 − 1{]} / {[}0.0049 × 2.7590{]} = 0.7090 / 0.013519 =
52.446

P = 4,000 × 9.1079 + 600 × 52.446 = 36,432 + 31,468 = \textbf{\$67,900}

\begin{enumerate}
\def\labelenumi{(\alph{enumi})}
\setcounter{enumi}{2}
\tightlist
\item
  Equivalent uniform annual cost: A\textsubscript{eq} = P × (A/P, 7\%,
  15) = 67,900 × {[}0.07 × 2.7590{]} / {[}2.7590 − 1{]} = 67,900 ×
  0.19313 / 1.7590 = 67,900 × 0.10979 = \textbf{\$7,455/year}
\end{enumerate}

\begin{center}\rule{0.5\linewidth}{0.5pt}\end{center}

\section{Problem 19.3.8}\label{problem-19.3.8}

\textbf{Given:} Operating costs for a SCADA system start at \$30,000 in
year 1 and increase by \$2,500/year for 8 years. At i = 10\%, a vendor
offers to replace the system with one that has a flat annual operating
cost.

\textbf{Find:} The maximum flat annual cost the utility should accept
for the replacement to be equivalent.

\textbf{Solution:}

The equivalent uniform annual cost of the existing system:
A\textsubscript{eq} = A₁ + G × (A/G, 10\%, 8)

(A/G, 10\%, 8) = {[}(1.10)⁸ − 0.10 × 8 − 1{]} / {[}0.10 × ((1.10)⁸ −
1){]} (1.10)⁸ = 2.1436 = {[}2.1436 − 0.80 − 1{]} / {[}0.10 × 1.1436{]} =
0.3436 / 0.11436 = 3.0045

A\textsubscript{eq} = 30,000 + 2,500 × 3.0045 = 30,000 + 7,511 =
\textbf{\$37,511/year}

The utility should accept any flat annual cost up to
\textbf{\$37,511/year} for the replacement system.

\begin{center}\rule{0.5\linewidth}{0.5pt}\end{center}

\section{Problem 19.3.9}\label{problem-19.3.9}

\textbf{Given:} Energy costs for a pumping station are \$120,000 in year
1 and escalate at 5\%/year (geometric gradient). The discount rate is
10\% over a 20-year planning horizon.

\textbf{Find:} The present worth of all energy costs.

\textbf{Solution:}

Since g ≠ i, use the geometric gradient formula: P = A₁ × {[}1 − (1 +
g)ⁿ(1 + i)⁻ⁿ{]} / (i − g) = 120,000 × {[}1 − (1.05)²⁰(1.10)⁻²⁰{]} /
(0.10 − 0.05)

(1.05)²⁰ = 2.6533; (1.10)²⁰ = 6.7275

= 120,000 × {[}1 − 2.6533/6.7275{]} / 0.05 = 120,000 × {[}1 − 0.3943{]}
/ 0.05 = 120,000 × 0.6057 / 0.05 = 120,000 × 12.114 =
\textbf{\$1,453,680}

\begin{center}\rule{0.5\linewidth}{0.5pt}\end{center}

\section{Problem 19.3.10}\label{problem-19.3.10}

\textbf{Given:} A consulting firm's revenue is \$500,000 in year 1 and
grows at 8\%/year. The firm's cost of capital is also 8\%. The planning
period is 10 years.

\textbf{Find:} The present worth of all revenue over 10 years.

\textbf{Solution:}

Since g = i = 8\%, use the special case formula: P = A₁ × n / (1 + i) =
500,000 × 10 / 1.08 = 5,000,000 / 1.08 = \textbf{\$4,629,630}

When the escalation rate equals the discount rate, the present worth
simplifies to n payments discounted by one period. This is significantly
higher than the non-escalating case, which would be P = 500,000 × (P/A,
8\%, 10) = 500,000 × 6.7101 = \$3,355,050.

\chapter{Chapter 19 --- Section 19.4: Present Worth
Analysis}\label{chapter-19-section-19.4-present-worth-analysis}

Practice problems covering net present value, comparing alternatives
with equal lives, and comparing alternatives with unequal lives using
the LCM method. Problems range from single-project NPV to
multi-alternative comparisons.

\begin{center}\rule{0.5\linewidth}{0.5pt}\end{center}

\section{Problem 19.4.1}\label{problem-19.4.1}

\textbf{Given:} A manufacturing plant evaluates a \$1,800,000 power
factor correction system. It reduces demand charges by \$320,000/year
over 8 years. The equipment has a salvage value of \$150,000 at the end
of year 8. The MARR is 12\%.

\textbf{Find:} (a) The NPV, and (b) whether the project is justified.

\textbf{Solution:}

\begin{enumerate}
\def\labelenumi{(\alph{enumi})}
\tightlist
\item
  NPV = −1,800,000 + 320,000 × (P/A, 12\%, 8) + 150,000 × (P/F, 12\%, 8)
\end{enumerate}

(P/A, 12\%, 8) = {[}(1.12)⁸ − 1{]} / {[}0.12 × (1.12)⁸{]} (1.12)⁸ =
2.4760 = {[}2.4760 − 1{]} / {[}0.12 × 2.4760{]} = 1.4760 / 0.29712 =
4.9676

(P/F, 12\%, 8) = 1/2.4760 = 0.4039

NPV = −1,800,000 + 320,000 × 4.9676 + 150,000 × 0.4039 = −1,800,000 +
1,589,632 + 60,585 = \textbf{−\$149,783}

\begin{enumerate}
\def\labelenumi{(\alph{enumi})}
\setcounter{enumi}{1}
\tightlist
\item
  Since NPV = −\$149,783 \textless{} 0, the project \textbf{is not
  justified} at a 12\% MARR.
\end{enumerate}

\begin{center}\rule{0.5\linewidth}{0.5pt}\end{center}

\section{Problem 19.4.2}\label{problem-19.4.2}

\textbf{Given:} A hospital evaluates a \$500,000 backup generator that
will save \$95,000/year in avoided outage costs over 10 years with no
salvage value. The MARR is 8\%.

\textbf{Find:} (a) The NPV, (b) whether the project is justified, and
(c) the minimum annual savings needed to justify the project.

\textbf{Solution:}

\begin{enumerate}
\def\labelenumi{(\alph{enumi})}
\tightlist
\item
  (P/A, 8\%, 10) = {[}(1.08)¹⁰ − 1{]} / {[}0.08 × (1.08)¹⁰{]} (1.08)¹⁰ =
  2.1589 = {[}2.1589 − 1{]} / {[}0.08 × 2.1589{]} = 1.1589 / 0.17271 =
  6.7101
\end{enumerate}

NPV = −500,000 + 95,000 × 6.7101 = −500,000 + 637,460 =
\textbf{\$137,460}

\begin{enumerate}
\def\labelenumi{(\alph{enumi})}
\setcounter{enumi}{1}
\item
  Since NPV = \$137,460 \textgreater{} 0, the project \textbf{is
  justified}.
\item
  Minimum annual savings: A\textsubscript{min} = P / (P/A, 8\%, 10) =
  500,000 / 6.7101 = \textbf{\$74,516/year}
\end{enumerate}

Any annual savings above \$74,516 makes the project economically
justified at the 8\% MARR.

\begin{center}\rule{0.5\linewidth}{0.5pt}\end{center}

\section{Problem 19.4.3}\label{problem-19.4.3}

\textbf{Given:} Compare two transformer cooling options over 15 years at
i = 9\%. Option A (forced-air): \$180,000 first cost, \$22,000/year
operating cost, \$15,000 salvage. Option B (forced-oil): \$280,000 first
cost, \$12,000/year operating cost, \$30,000 salvage.

\textbf{Find:} Which option is more economical.

\textbf{Solution:}

(P/A, 9\%, 15) = {[}(1.09)¹⁵ − 1{]} / {[}0.09 × (1.09)¹⁵{]} (1.09)¹⁵ =
3.6425 = {[}3.6425 − 1{]} / {[}0.09 × 3.6425{]} = 2.6425 / 0.32783 =
8.0607

(P/F, 9\%, 15) = 1/3.6425 = 0.2745

PW\textsubscript{A} = −180,000 − 22,000 × 8.0607 + 15,000 × 0.2745 =
−180,000 − 177,335 + 4,118 = \textbf{−\$353,217}

PW\textsubscript{B} = −280,000 − 12,000 × 8.0607 + 30,000 × 0.2745 =
−280,000 − 96,728 + 8,235 = \textbf{−\$368,493}

\textbf{Option A (forced-air) is more economical} by \$368,493 −
\$353,217 = \$15,276 in present worth. The lower first cost and smaller
operating cost penalty outweigh the smaller salvage value.

\begin{center}\rule{0.5\linewidth}{0.5pt}\end{center}

\section{Problem 19.4.4}\label{problem-19.4.4}

\textbf{Given:} A utility compares three protective relay schemes for a
substation over 20 years at MARR = 7\%. All have equal service lives.

{\def\LTcaptype{none} % do not increment counter
\begin{longtable}[]{@{}llll@{}}
\toprule\noalign{}
Parameter & Electromechanical & Solid-State & Digital \\
\midrule\noalign{}
\endhead
\bottomrule\noalign{}
\endlastfoot
First cost & \$45,000 & \$65,000 & \$90,000 \\
Annual maintenance & \$6,000 & \$3,500 & \$1,500 \\
Annual savings (faster clearing) & \$0 (base) & \$8,000 & \$14,000 \\
Salvage value & \$2,000 & \$5,000 & \$10,000 \\
\end{longtable}
}

\textbf{Find:} The most economical relay scheme using NPV analysis.

\textbf{Solution:}

(P/A, 7\%, 20) = {[}(1.07)²⁰ − 1{]} / {[}0.07 × (1.07)²⁰{]} (1.07)²⁰ =
3.8697 = {[}3.8697 − 1{]} / {[}0.07 × 3.8697{]} = 2.8697 / 0.27088 =
10.594

(P/F, 7\%, 20) = 1/3.8697 = 0.2584

NPV\textsubscript{EM} = −45,000 − 6,000 × 10.594 + 2,000 × 0.2584 =
−45,000 − 63,564 + 517 = \textbf{−\$108,047}

NPV\textsubscript{SS} = −65,000 + (8,000 − 3,500) × 10.594 + 5,000 ×
0.2584 = −65,000 + 4,500 × 10.594 + 1,292 = −65,000 + 47,673 + 1,292 =
\textbf{−\$16,035}

NPV\textsubscript{Digital} = −90,000 + (14,000 − 1,500) × 10.594 +
10,000 × 0.2584 = −90,000 + 12,500 × 10.594 + 2,584 = −90,000 + 132,425
+ 2,584 = \textbf{\$45,009}

\textbf{The digital relay scheme is most economical} with the only
positive NPV of \$45,009. The solid-state option has a negative NPV and
the electromechanical option is the worst performer.

\begin{center}\rule{0.5\linewidth}{0.5pt}\end{center}

\section{Problem 19.4.5}\label{problem-19.4.5}

\textbf{Given:} Compare a 10-year lighting system with a 15-year
lighting system for a warehouse at i = 8\%. - System A (LED): \$60,000
first cost, 10-year life, \$5,000/year energy cost, \$4,000 salvage. -
System B (advanced LED): \$85,000 first cost, 15-year life, \$3,500/year
energy cost, \$6,000 salvage.

\textbf{Find:} The more economical option using the LCM method.

\textbf{Solution:}

LCM of 10 and 15 = 30 years. System A repeats 3 times (years 0, 10, 20);
System B repeats 2 times (years 0, 15).

(P/A, 8\%, 30) = {[}(1.08)³⁰ − 1{]} / {[}0.08 × (1.08)³⁰{]} (1.08)³⁰ =
10.063 = {[}10.063 − 1{]} / {[}0.08 × 10.063{]} = 9.063 / 0.8050 =
11.258

(P/F, 8\%, 10) = 1/(1.08)¹⁰ = 1/2.1589 = 0.4632 (P/F, 8\%, 15) =
1/(1.08)¹⁵ = 1/3.1722 = 0.3152 (P/F, 8\%, 20) = 1/(1.08)²⁰ = 1/4.6610 =
0.2145 (P/F, 8\%, 30) = 1/10.063 = 0.09938

PW\textsubscript{A} = −60,000 − (60,000 − 4,000) × 0.4632 − (60,000 −
4,000) × 0.2145 + 4,000 × 0.09938 − 5,000 × 11.258 = −60,000 − 56,000 ×
0.4632 − 56,000 × 0.2145 + 398 − 56,290 = −60,000 − 25,939 − 12,012 +
398 − 56,290 = \textbf{−\$153,843}

PW\textsubscript{B} = −85,000 − (85,000 − 6,000) × 0.3152 + 6,000 ×
0.09938 − 3,500 × 11.258 = −85,000 − 79,000 × 0.3152 + 596 − 39,403 =
−85,000 − 24,901 + 596 − 39,403 = \textbf{−\$148,708}

\textbf{System B (advanced LED) is more economical} by \$153,843 −
\$148,708 = \$5,135 over the 30-year study period.

\begin{center}\rule{0.5\linewidth}{0.5pt}\end{center}

\section{Problem 19.4.6}\label{problem-19.4.6}

\textbf{Given:} A chemical plant compares two pump options at MARR =
10\%. - Pump X: \$35,000 first cost, 6-year life, \$8,000/year O\&M,
\$3,000 salvage. - Pump Y: \$50,000 first cost, 9-year life,
\$5,500/year O\&M, \$5,000 salvage.

\textbf{Find:} The more economical option using the LCM method.

\textbf{Solution:}

LCM of 6 and 9 = 18 years. Pump X repeats 3 times (years 0, 6, 12); Pump
Y repeats 2 times (years 0, 9).

(P/A, 10\%, 18) = {[}(1.10)¹⁸ − 1{]} / {[}0.10 × (1.10)¹⁸{]} (1.10)¹⁸ =
5.5599 = {[}5.5599 − 1{]} / {[}0.10 × 5.5599{]} = 4.5599 / 0.55599 =
8.2014

(P/F, 10\%, 6) = 1/(1.10)⁶ = 1/1.7716 = 0.5645 (P/F, 10\%, 9) =
1/(1.10)⁹ = 1/2.3579 = 0.4241 (P/F, 10\%, 12) = 1/(1.10)¹² = 1/3.1384 =
0.3186 (P/F, 10\%, 18) = 1/5.5599 = 0.1799

PW\textsubscript{X} = −35,000 − (35,000 − 3,000) × 0.5645 − (35,000 −
3,000) × 0.3186 + 3,000 × 0.1799 − 8,000 × 8.2014 = −35,000 − 32,000 ×
0.5645 − 32,000 × 0.3186 + 540 − 65,611 = −35,000 − 18,064 − 10,195 +
540 − 65,611 = \textbf{−\$128,330}

PW\textsubscript{Y} = −50,000 − (50,000 − 5,000) × 0.4241 + 5,000 ×
0.1799 − 5,500 × 8.2014 = −50,000 − 45,000 × 0.4241 + 900 − 45,108 =
−50,000 − 19,085 + 900 − 45,108 = \textbf{−\$113,293}

\textbf{Pump Y is more economical} by \$128,330 − \$113,293 = \$15,037
over the 18-year period. Its longer life and lower O\&M costs justify
the higher first cost.

\chapter{Chapter 19 --- Section 19.5: Annual Worth
Analysis}\label{chapter-19-section-19.5-annual-worth-analysis}

Practice problems covering capital recovery cost, total annual worth,
and comparing alternatives by annual worth. Problems range from
single-asset annual cost to multi-alternative comparisons with different
service lives.

\begin{center}\rule{0.5\linewidth}{0.5pt}\end{center}

\section{Problem 19.5.1}\label{problem-19.5.1}

\textbf{Given:} A 1,000 kVA pad-mounted transformer costs \$125,000, has
a 25-year life, annual maintenance of \$3,500, and a salvage value of
\$8,000. The interest rate is 7\%.

\textbf{Find:} (a) The capital recovery cost, and (b) the total annual
cost of the transformer.

\textbf{Solution:}

\begin{enumerate}
\def\labelenumi{(\alph{enumi})}
\tightlist
\item
  CR = (P − S) × (A/P, 7\%, 25) + S × i
\end{enumerate}

(A/P, 7\%, 25) = {[}0.07 × (1.07)²⁵{]} / {[}(1.07)²⁵ − 1{]} (1.07)²⁵ =
5.4274 = {[}0.07 × 5.4274{]} / {[}5.4274 − 1{]} = 0.37992 / 4.4274 =
0.08581

CR = (125,000 − 8,000) × 0.08581 + 8,000 × 0.07 = 117,000 × 0.08581 +
560 = 10,040 + 560 = \textbf{\$10,600/year}

\begin{enumerate}
\def\labelenumi{(\alph{enumi})}
\setcounter{enumi}{1}
\tightlist
\item
  Total annual cost: AW = −CR − Annual maintenance = −10,600 − 3,500 =
  \textbf{−\$14,100/year}
\end{enumerate}

\begin{center}\rule{0.5\linewidth}{0.5pt}\end{center}

\section{Problem 19.5.2}\label{problem-19.5.2}

\textbf{Given:} A solar installation generates \$75,000/year in energy
savings. The system costs \$420,000, has a 20-year life, annual O\&M of
\$8,000, and a salvage value of \$25,000. The MARR is 6\%.

\textbf{Find:} (a) The capital recovery cost, (b) the annual worth, and
(c) whether the project is justified.

\textbf{Solution:}

\begin{enumerate}
\def\labelenumi{(\alph{enumi})}
\tightlist
\item
  (A/P, 6\%, 20) = {[}0.06 × (1.06)²⁰{]} / {[}(1.06)²⁰ − 1{]} (1.06)²⁰ =
  3.2071 = {[}0.06 × 3.2071{]} / {[}3.2071 − 1{]} = 0.19243 / 2.2071 =
  0.08718
\end{enumerate}

CR = (420,000 − 25,000) × 0.08718 + 25,000 × 0.06 = 395,000 × 0.08718 +
1,500 = 34,436 + 1,500 = \textbf{\$35,936/year}

\begin{enumerate}
\def\labelenumi{(\alph{enumi})}
\setcounter{enumi}{1}
\item
  AW = Revenue − CR − O\&M = 75,000 − 35,936 − 8,000 =
  \textbf{\$31,064/year}
\item
  Since AW = \$31,064 \textgreater{} 0, the project \textbf{is
  justified} and generates a surplus of \$31,064/year above the 6\% MARR
  requirement.
\end{enumerate}

\begin{center}\rule{0.5\linewidth}{0.5pt}\end{center}

\section{Problem 19.5.3}\label{problem-19.5.3}

\textbf{Given:} Compare three cable types for a 2-mile underground
distribution run at i = 8\%.

{\def\LTcaptype{none} % do not increment counter
\begin{longtable}[]{@{}llll@{}}
\toprule\noalign{}
Parameter & XLPE & EPR & PILC \\
\midrule\noalign{}
\endhead
\bottomrule\noalign{}
\endlastfoot
First cost & \$350,000 & \$420,000 & \$290,000 \\
Life & 30 years & 35 years & 25 years \\
Annual O\&M & \$5,000 & \$3,000 & \$9,000 \\
Salvage & \$10,000 & \$15,000 & \$5,000 \\
\end{longtable}
}

\textbf{Find:} The most economical cable type using annual worth
analysis.

\textbf{Solution:}

XLPE --- (A/P, 8\%, 30): (1.08)³⁰ = 10.063 (A/P, 8\%, 30) = {[}0.08 ×
10.063{]} / {[}10.063 − 1{]} = 0.80504 / 9.063 = 0.08883

CR\textsubscript{XLPE} = (350,000 − 10,000) × 0.08883 + 10,000 × 0.08 =
340,000 × 0.08883 + 800 = 30,202 + 800 = \$31,002 AW\textsubscript{XLPE}
= −31,002 − 5,000 = \textbf{−\$36,002/year}

EPR --- (A/P, 8\%, 35): (1.08)³⁵ = 14.785 (A/P, 8\%, 35) = {[}0.08 ×
14.785{]} / {[}14.785 − 1{]} = 1.1828 / 13.785 = 0.08580

CR\textsubscript{EPR} = (420,000 − 15,000) × 0.08580 + 15,000 × 0.08 =
405,000 × 0.08580 + 1,200 = 34,749 + 1,200 = \$35,949
AW\textsubscript{EPR} = −35,949 − 3,000 = \textbf{−\$38,949/year}

PILC --- (A/P, 8\%, 25): (1.08)²⁵ = 6.8485 (A/P, 8\%, 25) = {[}0.08 ×
6.8485{]} / {[}6.8485 − 1{]} = 0.54788 / 5.8485 = 0.09368

CR\textsubscript{PILC} = (290,000 − 5,000) × 0.09368 + 5,000 × 0.08 =
285,000 × 0.09368 + 400 = 26,699 + 400 = \$27,099 AW\textsubscript{PILC}
= −27,099 − 9,000 = \textbf{−\$36,099/year}

\textbf{XLPE cable is the most economical} at −\$36,002/year, barely
edging out PILC (−\$36,099/year). Note that the AW method handles the
different service lives (25, 30, 35 years) without requiring LCM
calculations.

\begin{center}\rule{0.5\linewidth}{0.5pt}\end{center}

\section{Problem 19.5.4}\label{problem-19.5.4}

\textbf{Given:} A plant manager compares two VFD options for a cooling
tower fan at i = 10\%. - Standard VFD: \$18,000, 8-year life,
\$4,800/year energy, no salvage. - Premium VFD: \$26,000, 12-year life,
\$3,600/year energy, \$2,000 salvage.

\textbf{Find:} The more economical choice using annual worth comparison.

\textbf{Solution:}

Standard VFD: (A/P, 10\%, 8) = {[}0.10 × (1.10)⁸{]} / {[}(1.10)⁸ − 1{]}
= {[}0.10 × 2.1436{]} / {[}2.1436 − 1{]} = 0.21436 / 1.1436 = 0.18744

AW\textsubscript{std} = −18,000 × 0.18744 − 4,800 = −3,374 − 4,800 =
\textbf{−\$8,174/year}

Premium VFD: (A/P, 10\%, 12) = {[}0.10 × (1.10)¹²{]} / {[}(1.10)¹² −
1{]} = {[}0.10 × 3.1384{]} / {[}3.1384 − 1{]} = 0.31384 / 2.1384 =
0.14676

CR = (26,000 − 2,000) × 0.14676 + 2,000 × 0.10 = 24,000 × 0.14676 + 200
= 3,522 + 200 = \$3,722

AW\textsubscript{prem} = −3,722 − 3,600 = \textbf{−\$7,322/year}

\textbf{The premium VFD is more economical} by \$8,174 − \$7,322 =
\$852/year. Its longer life and lower energy costs more than justify the
higher purchase price.

\chapter{Chapter 19 --- Section 19.6: Rate of Return
Analysis}\label{chapter-19-section-19.6-rate-of-return-analysis}

Practice problems covering internal rate of return for single projects,
incremental rate of return for mutually exclusive alternatives, and
modified IRR for non-conventional cash flows. Problems range from direct
IRR calculation to multi-step incremental and MIRR analyses.

\begin{center}\rule{0.5\linewidth}{0.5pt}\end{center}

\section{Problem 19.6.1}\label{problem-19.6.1}

\textbf{Given:} A \$250,000 energy management system reduces electricity
costs by \$48,000/year for 10 years with no salvage value.

\textbf{Find:} (a) The IRR, and (b) whether the project is acceptable at
MARR = 10\%.

\textbf{Solution:}

\begin{enumerate}
\def\labelenumi{(\alph{enumi})}
\tightlist
\item
  Set NPV = 0: 0 = −250,000 + 48,000 × (P/A, i\emph{, 10) (P/A, i}, 10)
  = 250,000 / 48,000 = 5.2083
\end{enumerate}

Try i = 14\%: (P/A, 14\%, 10) = {[}(1.14)¹⁰ − 1{]} / {[}0.14 ×
(1.14)¹⁰{]} (1.14)¹⁰ = 3.7072 = {[}3.7072 − 1{]} / {[}0.14 × 3.7072{]} =
2.7072 / 0.51901 = 5.2163 (NPV slightly \textgreater{} 0)

Try i = 15\%: (P/A, 15\%, 10) = {[}(1.15)¹⁰ − 1{]} / {[}0.15 ×
(1.15)¹⁰{]} (1.15)¹⁰ = 4.0456 = {[}4.0456 − 1{]} / {[}0.15 × 4.0456{]} =
3.0456 / 0.60684 = 5.0188 (NPV \textless{} 0)

Interpolate: i* = 14\% + 1\% × (5.2163 − 5.2083) / (5.2163 − 5.0188) =
14\% + 1\% × 0.0080 / 0.1975 = 14\% + 0.04\% = \textbf{14.04\%}

\begin{enumerate}
\def\labelenumi{(\alph{enumi})}
\setcounter{enumi}{1}
\tightlist
\item
  Since IRR = 14.04\% \textgreater{} MARR = 10\%, the project \textbf{is
  acceptable}.
\end{enumerate}

\begin{center}\rule{0.5\linewidth}{0.5pt}\end{center}

\section{Problem 19.6.2}\label{problem-19.6.2}

\textbf{Given:} A \$400,000 solar thermal system generates \$72,000/year
in savings for 12 years and has a salvage value of \$30,000 at year 12.

\textbf{Find:} The IRR of the investment.

\textbf{Solution:}

Set NPV = 0: 0 = −400,000 + 72,000 × (P/A, i\emph{, 12) + 30,000 × (P/F,
i}, 12)

Try i = 14\%: (1.14)¹² = 4.8179 (P/A, 14\%, 12) = {[}4.8179 − 1{]} /
{[}0.14 × 4.8179{]} = 3.8179 / 0.67451 = 5.6603 (P/F, 14\%, 12) =
1/4.8179 = 0.2076 NPV = −400,000 + 72,000 × 5.6603 + 30,000 × 0.2076 =
−400,000 + 407,542 + 6,228 = +\$13,770

Try i = 15\%: (1.15)¹² = 5.3503 (P/A, 15\%, 12) = {[}5.3503 − 1{]} /
{[}0.15 × 5.3503{]} = 4.3503 / 0.80255 = 5.4205 (P/F, 15\%, 12) =
1/5.3503 = 0.1869 NPV = −400,000 + 72,000 × 5.4205 + 30,000 × 0.1869 =
−400,000 + 390,276 + 5,607 = −\$4,117

Interpolate: i* = 14\% + 1\% × 13,770 / (13,770 + 4,117) = 14\% + 1\% ×
0.770 = 14\% + 0.77\% = \textbf{14.77\%}

\begin{center}\rule{0.5\linewidth}{0.5pt}\end{center}

\section{Problem 19.6.3}\label{problem-19.6.3}

\textbf{Given:} Two switchgear options are available for a 15-year
service life. Option A: \$200,000 cost, \$42,000/year savings. Option B:
\$340,000 cost, \$65,000/year savings. Neither has salvage value. MARR =
9\%.

\textbf{Find:} Using incremental IRR analysis, which option should be
selected.

\textbf{Solution:}

First verify both meet MARR individually: Option A: (P/A, i\emph{, 15) =
200,000/42,000 = 4.7619 → IRR ≈ 18.9\% \textgreater{} 9\% ✓ Option B:
(P/A, i}, 15) = 340,000/65,000 = 5.2308 → IRR ≈ 16.6\% \textgreater{}
9\% ✓

Incremental analysis (B − A): ΔCost = \$140,000, ΔSavings =
\$23,000/year 0 = −140,000 + 23,000 × (P/A, Δi\emph{, 15) (P/A, Δi}, 15)
= 140,000/23,000 = 6.0870

Try i = 13\%: (1.13)¹⁵ = 6.2543 (P/A, 13\%, 15) = {[}6.2543 − 1{]} /
{[}0.13 × 6.2543{]} = 5.2543 / 0.81306 = 6.4626 (NPV \textgreater{} 0)

Try i = 15\%: (1.15)¹⁵ = 8.1371 (P/A, 15\%, 15) = {[}8.1371 − 1{]} /
{[}0.15 × 8.1371{]} = 7.1371 / 1.2206 = 5.8470 (NPV \textless{} 0)

Interpolate: Δi* = 13\% + 2\% × (6.4626 − 6.0870) / (6.4626 − 5.8470) =
13\% + 2\% × 0.3756/0.6156 = 13\% + 1.22\% = \textbf{14.22\%}

Since ΔIRR = 14.22\% \textgreater{} MARR = 9\%, the extra investment in
Option B is justified. \textbf{Select Option B.}

\begin{center}\rule{0.5\linewidth}{0.5pt}\end{center}

\section{Problem 19.6.4}\label{problem-19.6.4}

\textbf{Given:} Three motor control center (MCC) options over 10 years
at MARR = 11\%. No salvage values.

{\def\LTcaptype{none} % do not increment counter
\begin{longtable}[]{@{}lll@{}}
\toprule\noalign{}
Option & First Cost & Annual Savings \\
\midrule\noalign{}
\endhead
\bottomrule\noalign{}
\endlastfoot
A & \$80,000 & \$18,000 \\
B & \$130,000 & \$27,000 \\
C & \$175,000 & \$34,000 \\
\end{longtable}
}

\textbf{Find:} The best option using incremental IRR analysis.

\textbf{Solution:}

Rank by first cost: A, B, C.

Check A vs.~do-nothing: (P/A, i*, 10) = 80,000/18,000 = 4.4444 Try i =
18\%: (P/A, 18\%, 10) = {[}(1.18)¹⁰ − 1{]} / {[}0.18 × (1.18)¹⁰{]}
(1.18)¹⁰ = 5.2338; = 4.2338 / 0.94208 = 4.4941 Try i = 19\%: (1.19)¹⁰ =
5.6947; = 4.6947 / 1.0820 = 4.3389 Interpolate: IRR\textsubscript{A} ≈
18\% + 1\% × (4.4941 − 4.4444)/(4.4941 − 4.3389) = 18.32\%
IRR\textsubscript{A} = 18.32\% \textgreater{} 11\% ✓ → A is the current
best.

B vs.~A: ΔCost = \$50,000, ΔSavings = \$9,000/year (P/A, Δi\emph{, 10) =
50,000/9,000 = 5.5556 Try i = 12\%: (1.12)¹⁰ = 3.1058; (P/A) =
2.1058/0.37270 = 5.6502 (NPV \textgreater{} 0) Try i = 13\%: (1.13)¹⁰ =
3.3946; (P/A) = 2.3946/0.44130 = 5.4262 (NPV \textless{} 0) Δi} ≈ 12\% +
1\% × (5.6502 − 5.5556)/(5.6502 − 5.4262) = 12\% + 0.42\% = 12.42\% ΔIRR
= 12.42\% \textgreater{} 11\% → B is justified over A. B is the new
best.

C vs.~B: ΔCost = \$45,000, ΔSavings = \$7,000/year (P/A, Δi\emph{, 10) =
45,000/7,000 = 6.4286 Try i = 9\%: (1.09)¹⁰ = 2.3674; (P/A) =
1.3674/0.21307 = 6.4178 (NPV slightly \textless{} 0) Try i = 8\%:
(1.08)¹⁰ = 2.1589; (P/A) = 1.1589/0.17271 = 6.7101 (NPV \textgreater{}
0) Δi} ≈ 8\% + 1\% × (6.7101 − 6.4286)/(6.7101 − 6.4178) = 8\% + 1\% ×
0.2815/0.2923 = 8.96\% ΔIRR = 8.96\% \textless{} 11\% → C is not
justified over B.

\textbf{Select Option B} --- the increment from A to B is justified, but
the increment from B to C is not.

\begin{center}\rule{0.5\linewidth}{0.5pt}\end{center}

\section{Problem 19.6.5}\label{problem-19.6.5}

\textbf{Given:} A nuclear plant decommissioning project has the
following cash flows (in \$millions): Year 0: −\$80; Years 1--6:
+\$18/year; Year 7: −\$25 (decommissioning cost). The reinvestment rate
is 8\% and the finance rate is 6\%.

\textbf{Find:} (a) The number of sign changes in the cash flow, and (b)
the MIRR.

\textbf{Solution:}

\begin{enumerate}
\def\labelenumi{(\alph{enumi})}
\item
  Cash flow signs: −, +, +, +, +, +, +, − Sign changes: −→+ at year 1
  and +→− at year 7 = \textbf{2 sign changes} Up to 2 IRR values may
  exist, so MIRR is needed.
\item
  Future value of positive cash flows at reinvestment rate (8\%), all
  moved to year 7: FV\textsubscript{positive} = 18 × (F/A, 8\%, 6) ×
  (F/P, 8\%, 1) (F/A, 8\%, 6) = {[}(1.08)⁶ − 1{]} / 0.08 = {[}1.5869 −
  1{]} / 0.08 = 0.5869 / 0.08 = 7.3359 FV\textsubscript{positive} = 18 ×
  7.3359 × 1.08 = 18 × 7.923 = \$142.6M
\end{enumerate}

Present value of negative cash flows at finance rate (6\%):
PV\textsubscript{negative} = 80 + 25 × (P/F, 6\%, 7) = 80 + 25/(1.06)⁷ =
80 + 25/1.5036 = 80 + 16.627 = \$96.63M

MIRR =
(FV\textsubscript{positive}/PV\textsubscript{negative})\textsuperscript{1/n}
− 1 = (142.6/96.63)\textsuperscript{1/7} − 1 =
(1.4758)\textsuperscript{0.14286} − 1 = 1.0573 − 1 = \textbf{5.73\%}

\begin{center}\rule{0.5\linewidth}{0.5pt}\end{center}

\section{Problem 19.6.6}\label{problem-19.6.6}

\textbf{Given:} A cogeneration facility has the following cash flows:
Year 0: −\$10M initial investment; Years 1--5: +\$3M/year revenue; Year
6: −\$2M (major overhaul); Years 7--10: +\$3.5M/year revenue; Year 11:
−\$4M decommissioning. The reinvestment rate is 12\% and the finance
rate is 10\%.

\textbf{Find:} The MIRR of the project.

\textbf{Solution:}

Sign changes: −, +, +, +, +, +, −, +, +, +, +, − → \textbf{3 sign
changes} (need MIRR)

Future value of positive cash flows at 12\%, moved to year 11: FV₁₋₅ = 3
× (F/A, 12\%, 5) × (F/P, 12\%, 6) (F/A, 12\%, 5) = {[}(1.12)⁵ − 1{]} /
0.12 = {[}1.7623 − 1{]} / 0.12 = 6.3528 (F/P, 12\%, 6) = (1.12)⁶ =
1.9738 FV₁₋₅ = 3 × 6.3528 × 1.9738 = 3 × 12.538 = \$37.61M

FV₇₋₁₀ = 3.5 × (F/A, 12\%, 4) × (F/P, 12\%, 1) (F/A, 12\%, 4) =
{[}(1.12)⁴ − 1{]} / 0.12 = {[}1.5735 − 1{]} / 0.12 = 4.7793 (F/P, 12\%,
1) = 1.12 FV₇₋₁₀ = 3.5 × 4.7793 × 1.12 = 3.5 × 5.3528 = \$18.73M

FV\textsubscript{positive} = 37.61 + 18.73 = \$56.34M

Present value of negative cash flows at 10\%: PV\textsubscript{negative}
= 10 + 2 × (P/F, 10\%, 6) + 4 × (P/F, 10\%, 11) (P/F, 10\%, 6) =
1/(1.10)⁶ = 1/1.7716 = 0.5645 (P/F, 10\%, 11) = 1/(1.10)¹¹ = 1/2.8531 =
0.3505 PV\textsubscript{negative} = 10 + 2 × 0.5645 + 4 × 0.3505 = 10 +
1.129 + 1.402 = \$12.53M

MIRR = (56.34/12.53)\textsuperscript{1/11} − 1 =
(4.496)\textsuperscript{0.09091} − 1 = 1.1464 − 1 = \textbf{14.64\%}

\chapter{Chapter 19 --- Section 19.7: Benefit-Cost
Analysis}\label{chapter-19-section-19.7-benefit-cost-analysis}

Practice problems covering conventional benefit-cost ratios and
incremental B/C analysis for multiple public-sector alternatives.
Problems use realistic utility and infrastructure scenarios.

\begin{center}\rule{0.5\linewidth}{0.5pt}\end{center}

\section{Problem 19.7.1}\label{problem-19.7.1}

\textbf{Given:} A regional utility invests \$6,000,000 in an automated
meter infrastructure (AMI) system. Benefits include \$900,000/year in
reduced meter reading costs, \$200,000/year in outage detection
improvements, and \$50,000/year in theft detection. The system has a
20-year life and i = 5\%.

\textbf{Find:} (a) The conventional B/C ratio, and (b) whether the
project is justified.

\textbf{Solution:}

\begin{enumerate}
\def\labelenumi{(\alph{enumi})}
\tightlist
\item
  Total annual benefits: B = 900,000 + 200,000 + 50,000 =
  \$1,150,000/year
\end{enumerate}

AW of cost: C = 6,000,000 × (A/P, 5\%, 20) (A/P, 5\%, 20) = {[}0.05 ×
(1.05)²⁰{]} / {[}(1.05)²⁰ − 1{]} (1.05)²⁰ = 2.6533 = {[}0.05 × 2.6533{]}
/ {[}2.6533 − 1{]} = 0.13267 / 1.6533 = 0.08024

C = 6,000,000 × 0.08024 = \$481,440/year

B/C = 1,150,000 / 481,440 = \textbf{2.39}

\begin{enumerate}
\def\labelenumi{(\alph{enumi})}
\setcounter{enumi}{1}
\tightlist
\item
  Since B/C = 2.39 \textgreater{} 1.0, the project \textbf{is
  justified}.
\end{enumerate}

\begin{center}\rule{0.5\linewidth}{0.5pt}\end{center}

\section{Problem 19.7.2}\label{problem-19.7.2}

\textbf{Given:} A city evaluates a \$2,500,000 underground cable
conversion project. Benefits are \$180,000/year in reduced storm outage
costs and \$60,000/year in reduced tree trimming. Disbenefits include
\$25,000/year in longer repair times for underground faults. Annual O\&M
is \$40,000. Project life is 30 years at i = 4\%.

\textbf{Find:} (a) The conventional B/C ratio, and (b) the modified B/C
ratio.

\textbf{Solution:}

(A/P, 4\%, 30) = {[}0.04 × (1.04)³⁰{]} / {[}(1.04)³⁰ − 1{]} (1.04)³⁰ =
3.2434 = {[}0.04 × 3.2434{]} / {[}3.2434 − 1{]} = 0.12974 / 2.2434 =
0.05783

AW of initial cost: 2,500,000 × 0.05783 = \$144,575/year Total AW of
costs (conventional): 144,575 + 40,000 = \$184,575/year

Net annual benefits: B − D = (180,000 + 60,000) − 25,000 =
\$215,000/year

\begin{enumerate}
\def\labelenumi{(\alph{enumi})}
\item
  Conventional B/C = (B − D) / Costs = 215,000 / 184,575 = \textbf{1.16}
\item
  Modified B/C = (B − D − O\&M) / Initial Investment AW = (215,000 −
  40,000) / 144,575 = 175,000 / 144,575 = \textbf{1.21}
\end{enumerate}

Both ratios exceed 1.0 --- the project \textbf{is justified} under
either formulation.

\begin{center}\rule{0.5\linewidth}{0.5pt}\end{center}

\section{Problem 19.7.3}\label{problem-19.7.3}

\textbf{Given:} A state DOT evaluates three highway lighting upgrade
levels for a 10-mile corridor over 25 years at i = 6\%.

{\def\LTcaptype{none} % do not increment counter
\begin{longtable}[]{@{}llll@{}}
\toprule\noalign{}
Level & First Cost & Annual Benefits & Annual O\&M \\
\midrule\noalign{}
\endhead
\bottomrule\noalign{}
\endlastfoot
1 --- Basic LED & \$800,000 & \$120,000 & \$15,000 \\
2 --- Smart LED & \$1,400,000 & \$195,000 & \$25,000 \\
3 --- Smart LED + adaptive & \$2,200,000 & \$290,000 & \$40,000 \\
\end{longtable}
}

\textbf{Find:} The best alternative using incremental B/C analysis.

\textbf{Solution:}

(A/P, 6\%, 25) = {[}0.06 × (1.06)²⁵{]} / {[}(1.06)²⁵ − 1{]} (1.06)²⁵ =
4.2919 = {[}0.06 × 4.2919{]} / {[}4.2919 − 1{]} = 0.25751 / 3.2919 =
0.07823

AW of costs (capital + O\&M): C₁ = 800,000 × 0.07823 + 15,000 = 62,584 +
15,000 = \$77,584 C₂ = 1,400,000 × 0.07823 + 25,000 = 109,522 + 25,000 =
\$134,522 C₃ = 2,200,000 × 0.07823 + 40,000 = 172,106 + 40,000 =
\$212,106

Level 1 vs.~do-nothing: B/C = 120,000 / 77,584 = 1.55 ≥ 1.0 →
\textbf{Level 1 justified} (new base)

Level 2 vs.~Level 1: ΔB/ΔC = (195,000 − 120,000) / (134,522 − 77,584) =
75,000 / 56,938 = 1.32 ≥ 1.0 → \textbf{Level 2 justified} (new base)

Level 3 vs.~Level 2: ΔB/ΔC = (290,000 − 195,000) / (212,106 − 134,522) =
95,000 / 77,584 = 1.22 ≥ 1.0 → \textbf{Level 3 justified}

\textbf{Select Level 3} --- each incremental investment is justified
with ΔB/ΔC ≥ 1.0.

\begin{center}\rule{0.5\linewidth}{0.5pt}\end{center}

\section{Problem 19.7.4}\label{problem-19.7.4}

\textbf{Given:} A flood control district evaluates four levee
improvement options over 50 years at i = 3\%.

{\def\LTcaptype{none} % do not increment counter
\begin{longtable}[]{@{}llll@{}}
\toprule\noalign{}
Option & First Cost & Annual Flood Damage Reduction & Annual O\&M \\
\midrule\noalign{}
\endhead
\bottomrule\noalign{}
\endlastfoot
A & \$5,000,000 & \$300,000 & \$20,000 \\
B & \$8,000,000 & \$480,000 & \$30,000 \\
C & \$12,000,000 & \$650,000 & \$45,000 \\
D & \$18,000,000 & \$850,000 & \$60,000 \\
\end{longtable}
}

\textbf{Find:} The best option using incremental B/C analysis
(conventional formulation with O\&M in denominator).

\textbf{Solution:}

(A/P, 3\%, 50) = {[}0.03 × (1.03)⁵⁰{]} / {[}(1.03)⁵⁰ − 1{]} (1.03)⁵⁰ =
4.3839 = {[}0.03 × 4.3839{]} / {[}4.3839 − 1{]} = 0.13152 / 3.3839 =
0.03887

Total AW of costs (capital recovery + O\&M): C\textsubscript{A} =
5,000,000 × 0.03887 + 20,000 = 194,350 + 20,000 = \$214,350
C\textsubscript{B} = 8,000,000 × 0.03887 + 30,000 = 310,960 + 30,000 =
\$340,960 C\textsubscript{C} = 12,000,000 × 0.03887 + 45,000 = 466,440 +
45,000 = \$511,440 C\textsubscript{D} = 18,000,000 × 0.03887 + 60,000 =
699,660 + 60,000 = \$759,660

A vs.~do-nothing: B/C = 300,000 / 214,350 = 1.40 ≥ 1.0 → A justified
(base)

B vs.~A: ΔB/ΔC = (480,000 − 300,000) / (340,960 − 214,350) = 180,000 /
126,610 = 1.42 ≥ 1.0 → B justified (new base)

C vs.~B: ΔB/ΔC = (650,000 − 480,000) / (511,440 − 340,960) = 170,000 /
170,480 = 0.997 \textless{} 1.0 → C not justified

\textbf{Select Option B} --- the increment from B to C yields ΔB/ΔC =
0.997 \textless{} 1.0, so the additional investment in Option C is not
justified. Option D is also not justified since ΔB/ΔC = 200,000 /
248,220 = 0.81 \textless{} 1.0.

\chapter{Chapter 19 --- Section 19.8:
Depreciation}\label{chapter-19-section-19.8-depreciation}

Practice problems covering straight-line, declining balance, MACRS, and
units-of-production depreciation methods. Problems range from direct
calculation to comparison across methods.

\begin{center}\rule{0.5\linewidth}{0.5pt}\end{center}

\section{Problem 19.8.1}\label{problem-19.8.1}

\textbf{Given:} A utility installs a \$240,000 recloser system with a
salvage value of \$15,000 and a useful life of 15 years.

\textbf{Find:} Using straight-line depreciation: (a) the annual
depreciation charge, (b) the book value after 6 years, and (c) the book
value after 12 years.

\textbf{Solution:}

\begin{enumerate}
\def\labelenumi{(\alph{enumi})}
\item
  D = (P − S) / n = (240,000 − 15,000) / 15 = 225,000 / 15 =
  \textbf{\$15,000/year}
\item
  BV₆ = P − 6 × D = 240,000 − 6 × 15,000 = 240,000 − 90,000 =
  \textbf{\$150,000}
\item
  BV₁₂ = P − 12 × D = 240,000 − 12 × 15,000 = 240,000 − 180,000 =
  \textbf{\$60,000}
\end{enumerate}

\begin{center}\rule{0.5\linewidth}{0.5pt}\end{center}

\section{Problem 19.8.2}\label{problem-19.8.2}

\textbf{Given:} A \$500,000 industrial motor test bench has a 10-year
useful life and a \$50,000 salvage value. The company uses straight-line
depreciation.

\textbf{Find:} (a) The annual depreciation, (b) the depreciation rate as
a percentage, (c) the book value after 4 years, and (d) the year in
which the book value first drops below \$200,000.

\textbf{Solution:}

\begin{enumerate}
\def\labelenumi{(\alph{enumi})}
\item
  D = (500,000 − 50,000) / 10 = 450,000 / 10 = \textbf{\$45,000/year}
\item
  Depreciation rate: d = D / P = 45,000 / 500,000 = \textbf{9.0\%} of
  original cost per year
\item
  BV₄ = 500,000 − 4 × 45,000 = 500,000 − 180,000 = \textbf{\$320,000}
\item
  BV\textsubscript{t} \textless{} 200,000 → 500,000 − 45,000t
  \textless{} 200,000 → t \textgreater{} 300,000/45,000 = 6.67 The book
  value first drops below \$200,000 at the end of \textbf{year 7} (BV₇ =
  \$185,000).
\end{enumerate}

\begin{center}\rule{0.5\linewidth}{0.5pt}\end{center}

\section{Problem 19.8.3}\label{problem-19.8.3}

\textbf{Given:} A \$180,000 power quality monitoring system uses double
declining balance (DDB) depreciation with a 6-year life and a \$20,000
salvage value.

\textbf{Find:} The depreciation charge and book value for each year,
switching to straight-line when it produces a larger deduction.

\textbf{Solution:}

DDB rate: d = 2/n = 2/6 = 0.3333

SL remaining for year t = (BV\textsubscript{t−1} − S) / (remaining
years). Switch to SL when SL deduction exceeds DDB.

{\def\LTcaptype{none} % do not increment counter
\begin{longtable}[]{@{}
  >{\raggedright\arraybackslash}p{(\linewidth - 12\tabcolsep) * \real{0.0923}}
  >{\raggedright\arraybackslash}p{(\linewidth - 12\tabcolsep) * \real{0.1692}}
  >{\raggedright\arraybackslash}p{(\linewidth - 12\tabcolsep) * \real{0.0769}}
  >{\raggedright\arraybackslash}p{(\linewidth - 12\tabcolsep) * \real{0.2000}}
  >{\raggedright\arraybackslash}p{(\linewidth - 12\tabcolsep) * \real{0.1231}}
  >{\raggedright\arraybackslash}p{(\linewidth - 12\tabcolsep) * \real{0.2000}}
  >{\raggedright\arraybackslash}p{(\linewidth - 12\tabcolsep) * \real{0.1385}}@{}}
\toprule\noalign{}
\begin{minipage}[b]{\linewidth}\raggedright
Year
\end{minipage} & \begin{minipage}[b]{\linewidth}\raggedright
BV (start)
\end{minipage} & \begin{minipage}[b]{\linewidth}\raggedright
DDB
\end{minipage} & \begin{minipage}[b]{\linewidth}\raggedright
SL Remaining
\end{minipage} & \begin{minipage}[b]{\linewidth}\raggedright
Method
\end{minipage} & \begin{minipage}[b]{\linewidth}\raggedright
Depreciation
\end{minipage} & \begin{minipage}[b]{\linewidth}\raggedright
BV (end)
\end{minipage} \\
\midrule\noalign{}
\endhead
\bottomrule\noalign{}
\endlastfoot
1 & \$180,000 & \$60,000 & \$26,667 & DDB & \$60,000 & \$120,000 \\
2 & \$120,000 & \$40,000 & \$20,000 & DDB & \$40,000 & \$80,000 \\
3 & \$80,000 & \$26,667 & \$15,000 & DDB & \$26,667 & \$53,333 \\
4 & \$53,333 & \$17,778 & \$11,111 & DDB & \$17,778 & \$35,555 \\
5 & \$35,555 & \$11,852 & \$7,778 & DDB & \$11,852 & \$23,703 \\
6 & \$23,703 & \$7,901* & \$3,703 & SL & \$3,703 & \textbf{\$20,000} \\
\end{longtable}
}

*Year 6: DDB would give \$7,901, but BV would drop to \$15,802
\textless{} \$20,000 salvage --- not allowed. Switch to SL: D = \$23,703
− \$20,000 = \$3,703.

Year 5: DDB (\$11,852) \textgreater{} SL (\$7,778) and BV after DDB =
\$23,703 \textgreater{} \$20,000 salvage, so DDB is used. Switch occurs
in year 6 only when DDB would violate the salvage value floor.

Total depreciation = 60,000 + 40,000 + 26,667 + 17,778 + 11,852 + 3,703
= \textbf{\$160,000} = P − S ✓

\begin{center}\rule{0.5\linewidth}{0.5pt}\end{center}

\section{Problem 19.8.4}\label{problem-19.8.4}

\textbf{Given:} A manufacturing company purchases a \$350,000 robotic
welding cell classified as 7-year MACRS property.

\textbf{Find:} The depreciation charges for years 1 through 4 and the
book value after year 4.

\textbf{Solution:}

MACRS 7-year percentages: Year 1 = 14.29\%, Year 2 = 24.49\%, Year 3 =
17.49\%, Year 4 = 12.49\%

{\def\LTcaptype{none} % do not increment counter
\begin{longtable}[]{@{}llll@{}}
\toprule\noalign{}
Year & MACRS \% & Depreciation & Cumulative \\
\midrule\noalign{}
\endhead
\bottomrule\noalign{}
\endlastfoot
1 & 14.29\% & \$50,015 & \$50,015 \\
2 & 24.49\% & \$85,715 & \$135,730 \\
3 & 17.49\% & \$61,215 & \$196,945 \\
4 & 12.49\% & \$43,715 & \$240,660 \\
\end{longtable}
}

Book value after year 4: BV₄ = 350,000 − 240,660 = \textbf{\$109,340}

After 4 years, 68.76\% of the cost basis has been recovered through
depreciation.

\begin{center}\rule{0.5\linewidth}{0.5pt}\end{center}

\section{Problem 19.8.5}\label{problem-19.8.5}

\textbf{Given:} A \$2,400,000 natural gas peaking turbine is classified
as 15-year MACRS property.

\textbf{Find:} (a) The depreciation for years 1 through 3, and (b) the
book value after year 3.

\textbf{Solution:}

MACRS 15-year percentages: Year 1 = 5.00\%, Year 2 = 9.50\%, Year 3 =
8.55\%

{\def\LTcaptype{none} % do not increment counter
\begin{longtable}[]{@{}llll@{}}
\toprule\noalign{}
Year & MACRS \% & Depreciation & Cumulative \\
\midrule\noalign{}
\endhead
\bottomrule\noalign{}
\endlastfoot
1 & 5.00\% & \$120,000 & \$120,000 \\
2 & 9.50\% & \$228,000 & \$348,000 \\
3 & 8.55\% & \$205,200 & \$553,200 \\
\end{longtable}
}

\begin{enumerate}
\def\labelenumi{(\alph{enumi})}
\item
  Depreciation charges: Year 1 = \textbf{\$120,000}, Year 2 =
  \textbf{\$228,000}, Year 3 = \textbf{\$205,200}
\item
  BV₃ = 2,400,000 − 553,200 = \textbf{\$1,846,800}
\end{enumerate}

\begin{center}\rule{0.5\linewidth}{0.5pt}\end{center}

\section{Problem 19.8.6}\label{problem-19.8.6}

\textbf{Given:} A \$90,000 oscilloscope is rated for 30,000 hours of use
with a salvage value of \$6,000. Usage records show: Year 1 = 4,500
hours, Year 2 = 5,800 hours, Year 3 = 6,200 hours, Year 4 = 3,500 hours.

\textbf{Find:} Using units-of-production depreciation: (a) the
depreciation rate per hour, (b) the depreciation for each year, and (c)
the book value after year 4.

\textbf{Solution:}

\begin{enumerate}
\def\labelenumi{(\alph{enumi})}
\item
  d\textsubscript{unit} = (P − S) / Total hours = (90,000 − 6,000) /
  30,000 = 84,000 / 30,000 = \textbf{\$2.80/hour}
\item
  Depreciation by year:
\end{enumerate}

\begin{itemize}
\tightlist
\item
  Year 1: D₁ = 2.80 × 4,500 = \textbf{\$12,600}
\item
  Year 2: D₂ = 2.80 × 5,800 = \textbf{\$16,240}
\item
  Year 3: D₃ = 2.80 × 6,200 = \textbf{\$17,360}
\item
  Year 4: D₄ = 2.80 × 3,500 = \textbf{\$9,800}
\end{itemize}

\begin{enumerate}
\def\labelenumi{(\alph{enumi})}
\setcounter{enumi}{2}
\tightlist
\item
  Total depreciation through year 4: 12,600 + 16,240 + 17,360 + 9,800 =
  \$56,000 BV₄ = 90,000 − 56,000 = \textbf{\$34,000}
\end{enumerate}

Remaining depreciable hours: 30,000 − 20,000 = 10,000 hours; remaining
depreciable amount: 34,000 − 6,000 = \$28,000.

\begin{center}\rule{0.5\linewidth}{0.5pt}\end{center}

\section{Problem 19.8.7}\label{problem-19.8.7}

\textbf{Given:} A \$150,000 generator is rated for 100,000 kWh of total
production with a \$10,000 salvage value. In its first year, it produces
22,000 kWh.

\textbf{Find:} (a) The units-of-production depreciation for year 1, and
(b) the book value after year 1.

\textbf{Solution:}

\begin{enumerate}
\def\labelenumi{(\alph{enumi})}
\item
  d\textsubscript{unit} = (150,000 − 10,000) / 100,000 = 140,000 /
  100,000 = \$1.40/kWh D₁ = 1.40 × 22,000 = \textbf{\$30,800}
\item
  BV₁ = 150,000 − 30,800 = \textbf{\$119,200}
\end{enumerate}

\begin{center}\rule{0.5\linewidth}{0.5pt}\end{center}

\section{Problem 19.8.8}\label{problem-19.8.8}

\textbf{Given:} A \$600,000 CNC milling machine has a 10-year life and
\$40,000 salvage value. Compare the depreciation in year 1 and book
value after year 3 using three methods: (a) straight-line, (b) double
declining balance, and (c) 5-year MACRS (the machine qualifies as 5-year
MACRS property).

\textbf{Find:} Year 1 depreciation and book value after year 3 for each
method.

\textbf{Solution:}

\textbf{(a) Straight-line:} D = (600,000 − 40,000) / 10 = \$56,000/year
Year 1 depreciation: \textbf{\$56,000} BV₃ = 600,000 − 3 × 56,000 =
600,000 − 168,000 = \textbf{\$432,000}

\textbf{(b) Double declining balance (DDB):} Rate: d = 2/10 = 0.20 Year
1: D₁ = 0.20 × 600,000 = \$120,000; BV₁ = \$480,000 Year 2: D₂ = 0.20 ×
480,000 = \$96,000; BV₂ = \$384,000 Year 3: D₃ = 0.20 × 384,000 =
\$76,800; BV₃ = \$307,200

Year 1 depreciation: \textbf{\$120,000} BV₃ = \textbf{\$307,200}

\textbf{(c) 5-year MACRS:} MACRS 5-year percentages: Year 1 = 20.00\%,
Year 2 = 32.00\%, Year 3 = 19.20\% Year 1: D₁ = 0.20 × 600,000 =
\$120,000 Year 2: D₂ = 0.32 × 600,000 = \$192,000 Year 3: D₃ = 0.192 ×
600,000 = \$115,200 Cumulative: 120,000 + 192,000 + 115,200 = \$427,200

Year 1 depreciation: \textbf{\$120,000} BV₃ = 600,000 − 427,200 =
\textbf{\$172,800}

Summary: MACRS provides the fastest depreciation (BV₃ = \$172,800),
followed by DDB (\$307,200), then straight-line (\$432,000). Faster
depreciation methods provide greater early-year tax shields.

\chapter{Chapter 19 --- Section 19.9: Taxes and After-Tax
Analysis}\label{chapter-19-section-19.9-taxes-and-after-tax-analysis}

Practice problems covering after-tax cash flow analysis, tax credits and
incentives, and gain/loss on disposal. Problems range from
straightforward ATCF calculations to multi-step after-tax NPV analyses
with tax credits.

\begin{center}\rule{0.5\linewidth}{0.5pt}\end{center}

\section{Problem 19.9.1}\label{problem-19.9.1}

\textbf{Given:} A \$300,000 building automation system generates
\$65,000/year in energy savings over 10 years. The system is depreciated
using straight-line depreciation with no salvage value. The marginal tax
rate is 28\% and the after-tax MARR is 9\%.

\textbf{Find:} (a) The annual depreciation, (b) the after-tax cash flow,
and (c) the after-tax NPV.

\textbf{Solution:}

\begin{enumerate}
\def\labelenumi{(\alph{enumi})}
\item
  D = 300,000 / 10 = \textbf{\$30,000/year}
\item
  ATCF = BTCF × (1 − T) + D × T = 65,000 × (1 − 0.28) + 30,000 × 0.28 =
  65,000 × 0.72 + 30,000 × 0.28 = 46,800 + 8,400 =
  \textbf{\$55,200/year}
\item
  (P/A, 9\%, 10) = {[}(1.09)¹⁰ − 1{]} / {[}0.09 × (1.09)¹⁰{]} (1.09)¹⁰ =
  2.3674 = {[}2.3674 − 1{]} / {[}0.09 × 2.3674{]} = 1.3674 / 0.21307 =
  6.4177
\end{enumerate}

After-tax NPV = −300,000 + 55,200 × 6.4177 = −300,000 + 354,257 =
\textbf{\$54,257}

The project is justified on an after-tax basis (NPV \textgreater{} 0).

\begin{center}\rule{0.5\linewidth}{0.5pt}\end{center}

\section{Problem 19.9.2}\label{problem-19.9.2}

\textbf{Given:} A manufacturing facility invests \$800,000 in a
high-efficiency transformer system. Annual savings are \$150,000 over 8
years. The equipment uses 7-year MACRS depreciation. The tax rate is
25\% and the after-tax MARR is 10\%.

\textbf{Find:} The after-tax cash flows for years 1 through 4 and year 8
(year with no depreciation).

\textbf{Solution:}

MACRS 7-year percentages: Year 1 = 14.29\%, Year 2 = 24.49\%, Year 3 =
17.49\%, Year 4 = 12.49\%, Year 8 = 4.46\%

{\def\LTcaptype{none} % do not increment counter
\begin{longtable}[]{@{}
  >{\raggedright\arraybackslash}p{(\linewidth - 12\tabcolsep) * \real{0.0923}}
  >{\raggedright\arraybackslash}p{(\linewidth - 12\tabcolsep) * \real{0.0923}}
  >{\raggedright\arraybackslash}p{(\linewidth - 12\tabcolsep) * \real{0.1385}}
  >{\raggedright\arraybackslash}p{(\linewidth - 12\tabcolsep) * \real{0.2000}}
  >{\raggedright\arraybackslash}p{(\linewidth - 12\tabcolsep) * \real{0.2308}}
  >{\raggedright\arraybackslash}p{(\linewidth - 12\tabcolsep) * \real{0.1538}}
  >{\raggedright\arraybackslash}p{(\linewidth - 12\tabcolsep) * \real{0.0923}}@{}}
\toprule\noalign{}
\begin{minipage}[b]{\linewidth}\raggedright
Year
\end{minipage} & \begin{minipage}[b]{\linewidth}\raggedright
BTCF
\end{minipage} & \begin{minipage}[b]{\linewidth}\raggedright
MACRS \%
\end{minipage} & \begin{minipage}[b]{\linewidth}\raggedright
Depreciation
\end{minipage} & \begin{minipage}[b]{\linewidth}\raggedright
Taxable Income
\end{minipage} & \begin{minipage}[b]{\linewidth}\raggedright
Tax (25\%)
\end{minipage} & \begin{minipage}[b]{\linewidth}\raggedright
ATCF
\end{minipage} \\
\midrule\noalign{}
\endhead
\bottomrule\noalign{}
\endlastfoot
1 & \$150,000 & 14.29\% & \$114,320 & \$35,680 & \$8,920 & \$141,080 \\
2 & \$150,000 & 24.49\% & \$195,920 & −\$45,920 & −\$11,480 &
\$161,480 \\
3 & \$150,000 & 17.49\% & \$139,920 & \$10,080 & \$2,520 & \$147,480 \\
4 & \$150,000 & 12.49\% & \$99,920 & \$50,080 & \$12,520 & \$137,480 \\
\end{longtable}
}

Year 8 (last MACRS year): Depreciation = 0.0446 × 800,000 = \$35,680
ATCF₈ = 150,000 × (1 − 0.25) + 35,680 × 0.25 = 112,500 + 8,920 =
\textbf{\$121,420}

Note: In year 2, the large MACRS depreciation creates a tax loss that
provides a \$11,480 tax benefit, boosting ATCF above the BTCF.

\begin{center}\rule{0.5\linewidth}{0.5pt}\end{center}

\section{Problem 19.9.3}\label{problem-19.9.3}

\textbf{Given:} A commercial building installs a \$450,000 geothermal
heat pump system qualifying for a 30\% investment tax credit. Annual
energy savings are \$60,000 over 20 years. The depreciable basis is
reduced to 85\% of cost (ITC adjustment). Using 5-year straight-line
depreciation on the adjusted basis, a 24\% tax rate, and after-tax MARR
of 7\%.

\textbf{Find:} (a) The ITC amount, (b) the ATCF for years 1--5 (with
depreciation), and (c) the ATCF for years 6--20 (without depreciation).

\textbf{Solution:}

\begin{enumerate}
\def\labelenumi{(\alph{enumi})}
\item
  ITC = 450,000 × 0.30 = \textbf{\$135,000} (received in year 1)
\item
  Depreciable basis = 450,000 × 0.85 = \$382,500 Annual depreciation
  (years 1--5): D = 382,500 / 5 = \$76,500
\end{enumerate}

TI = 60,000 − 76,500 = −\$16,500 (tax loss) Tax = −16,500 × 0.24 =
−\$3,960 (tax benefit) ATCF (years 2--5) = 60,000 + 3,960 =
\textbf{\$63,960/year} ATCF (year 1) = 63,960 + 135,000 =
\textbf{\$198,960}

\begin{enumerate}
\def\labelenumi{(\alph{enumi})}
\setcounter{enumi}{2}
\tightlist
\item
  ATCF (years 6--20): No depreciation ATCF = 60,000 × (1 − 0.24) =
  60,000 × 0.76 = \textbf{\$45,600/year}
\end{enumerate}

\begin{center}\rule{0.5\linewidth}{0.5pt}\end{center}

\section{Problem 19.9.4}\label{problem-19.9.4}

\textbf{Given:} A \$350,000 test system is depreciated using 5-year
MACRS. After 4 years, it is sold for \$120,000. The tax rate is 25\%.

\textbf{Find:} (a) The book value at disposal, (b) the gain or loss, and
(c) the after-tax salvage value.

\textbf{Solution:}

\begin{enumerate}
\def\labelenumi{(\alph{enumi})}
\item
  MACRS 5-year cumulative through year 4: Year 1: 20.00\%, Year 2:
  32.00\%, Year 3: 19.20\%, Year 4: 11.52\% Total: 82.72\% Cumulative
  depreciation = 0.8272 × 350,000 = \$289,520 BV₄ = 350,000 − 289,520 =
  \textbf{\$60,480}
\item
  Selling price (\$120,000) \textgreater{} Book value (\$60,480) Gain =
  120,000 − 60,480 = \textbf{\$59,520 (taxable gain)}
\item
  Tax on gain = 59,520 × 0.25 = \$14,880 After-tax salvage:
  S\textsubscript{AT} = 120,000 − 14,880 = \textbf{\$105,120}
\end{enumerate}

\begin{center}\rule{0.5\linewidth}{0.5pt}\end{center}

\section{Problem 19.9.5}\label{problem-19.9.5}

\textbf{Given:} A \$160,000 CNC lathe is depreciated using straight-line
over 8 years with a \$16,000 salvage value. After 5 years, the machine
is sold for \$50,000 because it is being replaced. The tax rate is 30\%.

\textbf{Find:} (a) The book value at year 5, (b) the gain or loss on
disposal, and (c) the after-tax salvage value.

\textbf{Solution:}

\begin{enumerate}
\def\labelenumi{(\alph{enumi})}
\item
  D = (160,000 − 16,000) / 8 = 144,000 / 8 = \$18,000/year BV₅ = 160,000
  − 5 × 18,000 = 160,000 − 90,000 = \textbf{\$70,000}
\item
  Selling price (\$50,000) \textless{} Book value (\$70,000) Loss =
  70,000 − 50,000 = \textbf{\$20,000 (deductible loss)}
\item
  Tax benefit from loss = 20,000 × 0.30 = \$6,000 After-tax salvage:
  S\textsubscript{AT} = 50,000 + 6,000 = \textbf{\$56,000}
\end{enumerate}

The loss on disposal provides a tax benefit that increases the effective
after-tax proceeds.

\begin{center}\rule{0.5\linewidth}{0.5pt}\end{center}

\section{Problem 19.9.6}\label{problem-19.9.6}

\textbf{Given:} A utility evaluates a \$1,200,000 capacitor bank
installation. Annual demand charge savings are \$220,000 for 10 years.
The equipment uses 7-year MACRS depreciation and has a \$50,000 salvage
value at year 10. The tax rate is 24\% and the after-tax MARR is 8\%.

\textbf{Find:} (a) The after-tax NPV considering MACRS depreciation and
disposal, and (b) whether the project is justified.

\textbf{Solution:}

MACRS 7-year percentages: 14.29\%, 24.49\%, 17.49\%, 12.49\%, 8.93\%,
8.92\%, 8.93\%, 4.46\% (Depreciation ends after year 8; book value = 0
after MACRS is complete.)

ATCF for years 1--8 (with depreciation): ATCF\textsubscript{t} = 220,000
× (1 − 0.24) + D\textsubscript{t} × 0.24 = 167,200 + 0.24 ×
D\textsubscript{t}

{\def\LTcaptype{none} % do not increment counter
\begin{longtable}[]{@{}llll@{}}
\toprule\noalign{}
Year & Depreciation & Tax Shield (0.24 × D) & ATCF \\
\midrule\noalign{}
\endhead
\bottomrule\noalign{}
\endlastfoot
1 & \$171,480 & \$41,155 & \$208,355 \\
2 & \$293,880 & \$70,531 & \$237,731 \\
3 & \$209,880 & \$50,371 & \$217,571 \\
4 & \$149,880 & \$35,971 & \$203,171 \\
5 & \$107,160 & \$25,718 & \$192,918 \\
6 & \$107,040 & \$25,690 & \$192,890 \\
7 & \$107,160 & \$25,718 & \$192,918 \\
8 & \$53,520 & \$12,845 & \$180,045 \\
\end{longtable}
}

Years 9--10: No depreciation. ATCF = 220,000 × (1 − 0.24) =
\$167,200/year

Year 10 disposal: BV = 0 (fully depreciated). Selling for \$50,000
creates a gain. Tax on gain = 50,000 × 0.24 = \$12,000 After-tax salvage
= 50,000 − 12,000 = \$38,000

After-tax NPV = −1,200,000 + Σ ATCF\textsubscript{t} /
(1.08)\textsuperscript{t} + 38,000 / (1.08)¹⁰

Computing the PV of each year's ATCF: Year 1: 208,355 / 1.08 = 192,921
Year 2: 237,731 / 1.1664 = 203,816 Year 3: 217,571 / 1.2597 = 172,720
Year 4: 203,171 / 1.3605 = 149,337 Year 5: 192,918 / 1.4693 = 131,296
Year 6: 192,890 / 1.5869 = 121,558 Year 7: 192,918 / 1.7138 = 112,563
Year 8: 180,045 / 1.8509 = 97,274 Year 9: 167,200 / 1.9990 = 83,642 Year
10: 167,200 / 2.1589 = 77,446 Salvage PV: 38,000 / 2.1589 = 17,601

Total PV of inflows = 192,921 + 203,816 + 172,720 + 149,337 + 131,296 +
121,558 + 112,563 + 97,274 + 83,642 + 77,446 + 17,601 = \$1,360,174

After-tax NPV = −1,200,000 + 1,360,174 = \textbf{\$160,174}

\begin{enumerate}
\def\labelenumi{(\alph{enumi})}
\setcounter{enumi}{1}
\tightlist
\item
  Since after-tax NPV = \$160,174 \textgreater{} 0, the project
  \textbf{is justified} on an after-tax basis.
\end{enumerate}

\chapter{Chapter 19 --- Section 19.10: Replacement
Analysis}\label{chapter-19-section-19.10-replacement-analysis}

Practice problems covering economic service life, defender-challenger
comparison, and replacement with a fixed study period. Problems range
from ESL tabulation to multi-option study-period analysis.

\begin{center}\rule{0.5\linewidth}{0.5pt}\end{center}

\section{Problem 19.10.1}\label{problem-19.10.1}

\textbf{Given:} A 15-year-old voltage regulator has a current market
value of \$12,000. Annual operating costs are \$5,000 in year 1,
increasing by \$1,500/year. Assume zero salvage value in all future
years. The interest rate is 8\%.

\textbf{Find:} The economic service life by computing the EUAC for 1
through 5 additional years.

\textbf{Solution:}

(A/P, 8\%, n): n=1: 1.0800; n=2: 0.56077; n=3: 0.38803; n=4: 0.30192;
n=5: 0.25046

(A/G, 8\%, n): n=1: 0; n=2: 0.4808; n=3: 0.9487; n=4: 1.4040; n=5:
1.8465

AW of O\&M = A₁ + G × (A/G, 8\%, n) = 5,000 + 1,500 × (A/G)

{\def\LTcaptype{none} % do not increment counter
\begin{longtable}[]{@{}llll@{}}
\toprule\noalign{}
Years & CR = 12,000 × (A/P) & AW of O\&M & EUAC \\
\midrule\noalign{}
\endhead
\bottomrule\noalign{}
\endlastfoot
1 & \$12,960 & \$5,000 & \$17,960 \\
2 & \$6,729 & \$5,721 & \$12,450 \\
3 & \$4,656 & \$6,423 & \$11,079 \\
4 & \$3,623 & \$7,106 & \$10,729 \\
5 & \$3,006 & \$7,770 & \$10,776 \\
\end{longtable}
}

The EUAC reaches a minimum at \textbf{year 4} (\$10,729/year), then
begins to increase. The economic service life is \textbf{4 years}.

\begin{center}\rule{0.5\linewidth}{0.5pt}\end{center}

\section{Problem 19.10.2}\label{problem-19.10.2}

\textbf{Given:} An aging distribution transformer (defender) has a
market value of \$15,000, EUAC of \$14,200/year over its remaining ESL
of 3 years. A new high-efficiency transformer (challenger) costs
\$95,000 with EUAC of \$11,800/year over a 20-year ESL. The interest
rate is 9\%.

\textbf{Find:} (a) Whether to replace now, and (b) the annual savings
from replacement.

\textbf{Solution:}

\begin{enumerate}
\def\labelenumi{(\alph{enumi})}
\tightlist
\item
  Defender EUAC: \$14,200/year Challenger EUAC: \$11,800/year
\end{enumerate}

Since challenger EUAC (\$11,800) \textless{} defender EUAC (\$14,200),
\textbf{replace now}.

\begin{enumerate}
\def\labelenumi{(\alph{enumi})}
\setcounter{enumi}{1}
\tightlist
\item
  Annual savings = 14,200 − 11,800 = \textbf{\$2,400/year}
\end{enumerate}

The new transformer saves \$2,400/year on an equivalent annual cost
basis.

\begin{center}\rule{0.5\linewidth}{0.5pt}\end{center}

\section{Problem 19.10.3}\label{problem-19.10.3}

\textbf{Given:} A plant's existing motor (defender) has a market value
of \$5,000, annual O\&M of \$4,500 increasing by \$500/year, and zero
salvage value. A replacement motor (challenger) costs \$22,000, has
annual O\&M of \$1,800, a 12-year life, and \$2,000 salvage value. The
interest rate is 10\%.

\textbf{Find:} (a) The defender's EUAC for 1--4 years, (b) the
challenger's EUAC, and (c) the replacement decision.

\textbf{Solution:}

\begin{enumerate}
\def\labelenumi{(\alph{enumi})}
\tightlist
\item
  Defender ESL analysis: (A/P, 10\%, n): n=1: 1.1000; n=2: 0.57619; n=3:
  0.40211; n=4: 0.31547 (A/G, 10\%, n): n=1: 0; n=2: 0.4762; n=3:
  0.9366; n=4: 1.3812
\end{enumerate}

{\def\LTcaptype{none} % do not increment counter
\begin{longtable}[]{@{}llll@{}}
\toprule\noalign{}
Years & CR = 5,000 × (A/P) & AW of O\&M & EUAC \\
\midrule\noalign{}
\endhead
\bottomrule\noalign{}
\endlastfoot
1 & \$5,500 & \$4,500 & \$10,000 \\
2 & \$2,881 & \$4,738 & \$7,619 \\
3 & \$2,011 & \$4,968 & \$6,979 \\
4 & \$1,577 & \$5,191 & \$6,768 \\
\end{longtable}
}

The defender's minimum EUAC is \textbf{\$6,768/year at 4 years} (ESL = 4
years).

\begin{enumerate}
\def\labelenumi{(\alph{enumi})}
\setcounter{enumi}{1}
\tightlist
\item
  Challenger EUAC: (A/P, 10\%, 12) = {[}0.10 × (1.10)¹²{]} / {[}(1.10)¹²
  − 1{]} = {[}0.10 × 3.1384{]} / {[}3.1384 − 1{]} = 0.31384 / 2.1384 =
  0.14676
\end{enumerate}

CR = (22,000 − 2,000) × 0.14676 + 2,000 × 0.10 = 20,000 × 0.14676 + 200
= 2,935 + 200 = \$3,135 EUAC\textsubscript{challenger} = 3,135 + 1,800 =
\textbf{\$4,935/year}

\begin{enumerate}
\def\labelenumi{(\alph{enumi})}
\setcounter{enumi}{2}
\tightlist
\item
  Since challenger EUAC (\$4,935) \textless{} defender EUAC (\$6,768),
  \textbf{replace now}. The savings are \$1,833/year.
\end{enumerate}

\begin{center}\rule{0.5\linewidth}{0.5pt}\end{center}

\section{Problem 19.10.4}\label{problem-19.10.4}

\textbf{Given:} Management imposes a 12-year study period. The defender
can serve 5 more years with EUAC = \$16,000/year. The challenger has a
12-year life with EUAC = \$13,500/year. After the defender is retired,
the challenger serves the remaining years. The interest rate is 7\%.

\textbf{Find:} (a) PW of replacing now, and (b) PW of replacing at year
5. Select the better option.

\textbf{Solution:}

\begin{enumerate}
\def\labelenumi{(\alph{enumi})}
\tightlist
\item
  Replace now --- challenger serves all 12 years: (P/A, 7\%, 12) =
  {[}(1.07)¹² − 1{]} / {[}0.07 × (1.07)¹²{]} (1.07)¹² = 2.2522 =
  {[}2.2522 − 1{]} / {[}0.07 × 2.2522{]} = 1.2522 / 0.15765 = 7.9427
\end{enumerate}

PW = 13,500 × 7.9427 = \textbf{\$107,227}

\begin{enumerate}
\def\labelenumi{(\alph{enumi})}
\setcounter{enumi}{1}
\tightlist
\item
  Replace at year 5 --- defender serves years 1--5, challenger serves
  years 6--12: (P/A, 7\%, 5) = {[}(1.07)⁵ − 1{]} / {[}0.07 × (1.07)⁵{]}
  (1.07)⁵ = 1.4026 = {[}1.4026 − 1{]} / {[}0.07 × 1.4026{]} = 0.4026 /
  0.09818 = 4.1002
\end{enumerate}

(P/A, 7\%, 7) = {[}(1.07)⁷ − 1{]} / {[}0.07 × (1.07)⁷{]} (1.07)⁷ =
1.6058 = {[}1.6058 − 1{]} / {[}0.07 × 1.6058{]} = 0.6058 / 0.11241 =
5.3893

(P/F, 7\%, 5) = 1/1.4026 = 0.7130

PW = 16,000 × 4.1002 + 13,500 × 5.3893 × 0.7130 = 65,603 + 13,500 ×
3.8425 = 65,603 + 51,874 = \textbf{\$117,477}

\textbf{Replace now} --- it saves \$117,477 − \$107,227 = \$10,250 in
present worth.

\begin{center}\rule{0.5\linewidth}{0.5pt}\end{center}

\section{Problem 19.10.5}\label{problem-19.10.5}

\textbf{Given:} A utility's existing SCADA server (defender) has a
market value of \$8,000 and can serve 2 more years at \$12,000/year O\&M
or 3 more years at \$12,000, \$14,000, \$17,000 in years 1--3
respectively. A new server (challenger) costs \$45,000, has O\&M of
\$4,000/year, and a 6-year life with \$5,000 salvage. The interest rate
is 10\%. Management sets a 6-year study period.

\textbf{Find:} The best replacement timing --- now, at year 2, or at
year 3.

\textbf{Solution:}

Challenger EUAC: (A/P, 10\%, 6) = {[}0.10 × (1.10)⁶{]} / {[}(1.10)⁶ −
1{]} = {[}0.10 × 1.7716{]} / {[}0.7716{]} = 0.22961 CR = (45,000 −
5,000) × 0.22961 + 5,000 × 0.10 = 40,000 × 0.22961 + 500 = 9,184 + 500 =
\$9,684 EUAC\textsubscript{chall} = 9,684 + 4,000 = \$13,684/year

\textbf{Option 1 --- Replace now} (challenger serves 6 years): PW =
13,684 × (P/A, 10\%, 6) = 13,684 × 4.3553 = \textbf{\$59,587}

\textbf{Option 2 --- Replace at year 2} (defender 2 years, challenger 4
years): (P/A, 10\%, 2) = 1.7355; (P/F, 10\%, 2) = 0.8264

Defender cost: 8,000 + 12,000 × 1.7355 = 8,000 + 20,826 = \$28,826
Challenger cost for 4 years at year 2: (A/P, 10\%, 4) = {[}0.10 ×
1.4641{]} / {[}0.4641{]} = 0.31547 Partial CR = (45,000 − 5,000) ×
0.31547 + 5,000 × 0.10 = 12,619 + 500 = \$13,119 Annual cost = 13,119 +
4,000 = \$17,119 PW of challenger = 17,119 × (P/A, 10\%, 4) × (P/F,
10\%, 2) = 17,119 × 3.1699 × 0.8264 = \$44,840

PW₂ = 28,826 + 44,840 = \textbf{\$73,666}

\textbf{Option 3 --- Replace at year 3} (defender 3 years, challenger 3
years): PW of defender O\&M: 12,000/(1.10) + 14,000/(1.10)² +
17,000/(1.10)³ = 10,909 + 11,570 + 12,772 = \$35,251 Defender cost =
8,000 + 35,251 = \$43,251

Challenger for 3 years: (A/P, 10\%, 3) = {[}0.10 × 1.3310{]} /
{[}0.3310{]} = 0.40211 Partial CR = (45,000 − 5,000) × 0.40211 + 500 =
16,084 + 500 = \$16,584 Annual cost = 16,584 + 4,000 = \$20,584 (P/A,
10\%, 3) = 2.4869; (P/F, 10\%, 3) = 0.7513 PW of challenger = 20,584 ×
2.4869 × 0.7513 = \$38,468

PW₃ = 43,251 + 38,468 = \textbf{\$81,719}

\textbf{Replace now (Option 1)} with PW = \$59,587 is the most
economical choice.

\begin{center}\rule{0.5\linewidth}{0.5pt}\end{center}

\section{Problem 19.10.6}\label{problem-19.10.6}

\textbf{Given:} A fleet of 10 delivery trucks (defender) has a combined
market value of \$80,000 and EUAC of \$95,000/year for 2 more years of
service. A fleet of new electric trucks (challenger) costs \$450,000
with EUAC of \$72,000/year over 8 years. The interest rate is 8\%.

\textbf{Find:} (a) Whether to replace now, and (b) the present worth of
savings over the challenger's 8-year life.

\textbf{Solution:}

\begin{enumerate}
\def\labelenumi{(\alph{enumi})}
\tightlist
\item
  Defender EUAC: \$95,000/year Challenger EUAC: \$72,000/year
\end{enumerate}

Since challenger EUAC (\$72,000) \textless{} defender EUAC (\$95,000),
\textbf{replace now}.

\begin{enumerate}
\def\labelenumi{(\alph{enumi})}
\setcounter{enumi}{1}
\tightlist
\item
  Annual savings = 95,000 − 72,000 = \$23,000/year PW of savings over 8
  years = 23,000 × (P/A, 8\%, 8) (P/A, 8\%, 8) = {[}(1.08)⁸ − 1{]} /
  {[}0.08 × (1.08)⁸{]} = {[}1.8509 − 1{]} / {[}0.08 × 1.8509{]} = 0.8509
  / 0.14807 = 5.7466
\end{enumerate}

PW of savings = 23,000 × 5.7466 = \textbf{\$132,172}

Note: This is a simplified comparison. A more rigorous analysis would
account for the defender's 2-year remaining life and what replaces it
after year 2.

\chapter{Chapter 19 --- Section 19.11: Inflation and Price
Changes}\label{chapter-19-section-19.11-inflation-and-price-changes}

Practice problems covering inflation-adjusted analysis using
constant-dollar and actual-dollar methods, and differential escalation
of specific commodities. Problems range from direct rate conversion to
multi-step present worth calculations.

\begin{center}\rule{0.5\linewidth}{0.5pt}\end{center}

\section{Problem 19.11.1}\label{problem-19.11.1}

\textbf{Given:} A utility budgets \$80,000/year (in today's dollars) for
transformer oil replacement. General inflation is 3.5\% and the market
MARR is 11\%. The planning horizon is 10 years.

\textbf{Find:} (a) The real interest rate, and (b) the present worth of
maintenance using the constant-dollar approach.

\textbf{Solution:}

\begin{enumerate}
\def\labelenumi{(\alph{enumi})}
\item
  Real rate: i\textsubscript{r} = (i\textsubscript{f} − f) / (1 + f) =
  (0.11 − 0.035) / (1.035) = 0.075 / 1.035 = \textbf{7.246\%}
\item
  Constant-dollar approach: (P/A, 7.246\%, 10) = {[}(1.07246)¹⁰ − 1{]} /
  {[}0.07246 × (1.07246)¹⁰{]} (1.07246)¹⁰ = 2.0108 = {[}2.0108 − 1{]} /
  {[}0.07246 × 2.0108{]} = 1.0108 / 0.14572 = 6.936
\end{enumerate}

PW = 80,000 × 6.936 = \textbf{\$554,880}

\begin{center}\rule{0.5\linewidth}{0.5pt}\end{center}

\section{Problem 19.11.2}\label{problem-19.11.2}

\textbf{Given:} An industrial facility has annual energy costs of
\$200,000 in today's dollars. Inflation is 2.5\%, and the market rate is
9\%. The planning period is 15 years.

\textbf{Find:} The present worth using (a) the constant-dollar approach
and (b) the actual-dollar approach. Verify both methods agree.

\textbf{Solution:}

\begin{enumerate}
\def\labelenumi{(\alph{enumi})}
\tightlist
\item
  Constant-dollar approach: i\textsubscript{r} = (0.09 − 0.025) /
  (1.025) = 0.065 / 1.025 = 6.341\%
\end{enumerate}

(P/A, 6.341\%, 15) = {[}(1.06341)¹⁵ − 1{]} / {[}0.06341 × (1.06341)¹⁵{]}
(1.06341)¹⁵ = 2.5050 = {[}2.5050 − 1{]} / {[}0.06341 × 2.5050{]} =
1.5050 / 0.15884 = 9.475

PW\textsubscript{constant} = 200,000 × 9.475 = \textbf{\$1,895,000}

\begin{enumerate}
\def\labelenumi{(\alph{enumi})}
\setcounter{enumi}{1}
\tightlist
\item
  Actual-dollar approach: Each year's cost in actual dollars:
  A\textsubscript{t} = 200,000 × (1.025)\textsuperscript{t} PW = Σ
  {[}200,000 × (1.025)\textsuperscript{t} / (1.09)\textsuperscript{t}{]}
  for t = 1 to 15 = 200,000 × Σ {[}(1.025/1.09)\textsuperscript{t}{]} =
  200,000 × Σ {[}(0.94037)\textsuperscript{t}{]}
\end{enumerate}

Geometric series: S = 0.94037 × {[}1 − (0.94037)¹⁵{]} / {[}1 −
0.94037{]} (0.94037)¹⁵ = 0.3993 = 0.94037 × {[}1 − 0.3993{]} / 0.05963 =
0.94037 × 0.6007 / 0.05963 = 0.94037 × 10.074 = 9.475

PW\textsubscript{actual} = 200,000 × 9.475 = \textbf{\$1,895,000} ✓

Both methods yield the same present worth, confirming consistency.

\begin{center}\rule{0.5\linewidth}{0.5pt}\end{center}

\section{Problem 19.11.3}\label{problem-19.11.3}

\textbf{Given:} Aluminum conductor prices are escalating at 6\%/year
while general inflation is 2.5\%/year. A transmission project requires
\$350,000 of aluminum in year 1. The market interest rate is 10\% and
the planning horizon is 8 years.

\textbf{Find:} (a) The differential escalation rate, and (b) the present
worth of aluminum purchases over 8 years.

\textbf{Solution:}

\begin{enumerate}
\def\labelenumi{(\alph{enumi})}
\item
  Differential escalation rate: e\textsubscript{d} = 6\% − 2.5\% =
  \textbf{3.5\%/year} above general inflation
\item
  Geometric gradient with g = 6\% (aluminum escalation) and i = 10\%: P
  = A₁ × {[}1 − (1 + g)ⁿ(1 + i)⁻ⁿ{]} / (i − g) = 350,000 × {[}1 −
  (1.06)⁸(1.10)⁻⁸{]} / (0.10 − 0.06)
\end{enumerate}

(1.06)⁸ = 1.5938; (1.10)⁸ = 2.1436

= 350,000 × {[}1 − 1.5938/2.1436{]} / 0.04 = 350,000 × {[}1 − 0.7435{]}
/ 0.04 = 350,000 × 0.2565 / 0.04 = 350,000 × 6.4125 =
\textbf{\$2,244,375}

\begin{center}\rule{0.5\linewidth}{0.5pt}\end{center}

\section{Problem 19.11.4}\label{problem-19.11.4}

\textbf{Given:} A 20-year power purchase agreement specifies
\$500,000/year in today's dollars for electricity, but the contract
escalates at the general inflation rate of 3\%/year. Simultaneously, the
utility's internal labor costs of \$100,000/year escalate at 5\%/year
(differential escalation). The market discount rate is 11\%.

\textbf{Find:} The combined present worth of electricity and labor costs
over 20 years.

\textbf{Solution:}

PW of electricity (geometric gradient, g = 3\%, i = 11\%):
P\textsubscript{elec} = 500,000 × {[}1 − (1.03)²⁰(1.11)⁻²⁰{]} / (0.11 −
0.03) (1.03)²⁰ = 1.8061; (1.11)²⁰ = 8.0623 = 500,000 × {[}1 −
1.8061/8.0623{]} / 0.08 = 500,000 × {[}1 − 0.2240{]} / 0.08 = 500,000 ×
0.7760 / 0.08 = 500,000 × 9.700 = \$4,850,000

PW of labor (geometric gradient, g = 5\%, i = 11\%):
P\textsubscript{labor} = 100,000 × {[}1 − (1.05)²⁰(1.11)⁻²⁰{]} / (0.11 −
0.05) (1.05)²⁰ = 2.6533 = 100,000 × {[}1 − 2.6533/8.0623{]} / 0.06 =
100,000 × {[}1 − 0.3292{]} / 0.06 = 100,000 × 0.6708 / 0.06 = 100,000 ×
11.180 = \$1,118,000

Combined PW = 4,850,000 + 1,118,000 = \textbf{\$5,968,000}

The differential escalation of labor (5\% vs.~3\% inflation) adds
significant cost over 20 years --- the labor PW is about 23\% of the
electricity PW despite being only 20\% of the base-year cost.

\chapter{Chapter 19 --- Section 19.12: Capital Budgeting and Project
Selection}\label{chapter-19-section-19.12-capital-budgeting-and-project-selection}

Practice problems covering independent project selection under budget
constraints and MARR determination. Problems range from portfolio
optimization to MARR-based accept/reject decisions.

\begin{center}\rule{0.5\linewidth}{0.5pt}\end{center}

\section{Problem 19.12.1}\label{problem-19.12.1}

\textbf{Given:} A utility has \$5,000,000 in capital for the coming year
and six independent project proposals. Each project is indivisible (all
or nothing). The MARR is 10\%.

{\def\LTcaptype{none} % do not increment counter
\begin{longtable}[]{@{}lll@{}}
\toprule\noalign{}
Project & Investment & NPV at 10\% \\
\midrule\noalign{}
\endhead
\bottomrule\noalign{}
\endlastfoot
A --- Feeder upgrade & \$1,500,000 & \$280,000 \\
B --- Substation rebuild & \$2,200,000 & \$350,000 \\
C --- SCADA modernization & \$800,000 & \$140,000 \\
D --- Capacitor installation & \$600,000 & \$105,000 \\
E --- Underground conversion & \$3,000,000 & \$420,000 \\
F --- Transformer replacement & \$1,000,000 & \$160,000 \\
\end{longtable}
}

\textbf{Find:} The portfolio that maximizes total NPV within the budget.

\textbf{Solution:}

Profitability index (PI = NPV/Investment): A = 280,000/1,500,000 = 0.187
B = 350,000/2,200,000 = 0.159 C = 140,000/800,000 = 0.175 D =
105,000/600,000 = 0.175 F = 160,000/1,000,000 = 0.160 E =
420,000/3,000,000 = 0.140

Rank by PI: A (0.187), C (0.175), D (0.175), F (0.160), B (0.159), E
(0.140)

Select in order: A (\$1,500,000) + C (\$800,000) + D (\$600,000) + F
(\$1,000,000) = \$3,900,000, NPV = \$685,000 Remaining budget:
\$1,100,000 --- B needs \$2,200,000 (no), E needs \$3,000,000 (no). No
more projects fit.

Check alternative combinations: A + B + C + D = \$5,100,000 (exceeds
budget) A + B + D = \$4,300,000, NPV = \$735,000; remaining \$700,000
--- C needs \$800,000 (no), F needs \$1,000,000 (no) A + B + F =
\$4,700,000, NPV = \$790,000; remaining \$300,000 --- nothing fits B + C
+ D + F = \$4,600,000, NPV = \$755,000; remaining \$400,000 --- nothing
fits A + C + D + F = \$3,900,000, NPV = \$685,000 --- already evaluated
E + A = \$4,500,000, NPV = \$700,000; remaining \$500,000 --- D fits! →
E + A + D = \$5,100,000 (exceeds) E + C + D = \$4,400,000, NPV =
\$665,000; remaining \$600,000 --- D already included, F needs
\$1,000,000 E + F = \$4,000,000, NPV = \$580,000; can add C + D =
\$1,400,000? Total \$5,400,000 (exceeds) E + C = \$3,800,000, NPV =
\$560,000 + D = \$4,400,000, NPV = \$665,000 + F = \$5,400,000 (exceeds)

Best found: \textbf{A + B + F} at \$4,700,000 with total NPV =
\textbf{\$790,000}

Verify: A (\$280,000) + B (\$350,000) + F (\$160,000) = \$790,000.
Investment = \$4,700,000 ≤ \$5,000,000 ✓

\textbf{Select projects A, B, and F} for a total NPV of
\textbf{\$790,000} within the \$5,000,000 budget.

\begin{center}\rule{0.5\linewidth}{0.5pt}\end{center}

\section{Problem 19.12.2}\label{problem-19.12.2}

\textbf{Given:} A company evaluates four independent projects. Equity
capital costs 14\%, debt capital costs 8\%, the debt-to-equity ratio is
60/40, and a 4\% risk premium applies for each project.

{\def\LTcaptype{none} % do not increment counter
\begin{longtable}[]{@{}llll@{}}
\toprule\noalign{}
Project & Investment & Annual Savings & Life (years) \\
\midrule\noalign{}
\endhead
\bottomrule\noalign{}
\endlastfoot
W & \$200,000 & \$42,000 & 10 \\
X & \$350,000 & \$68,000 & 10 \\
Y & \$150,000 & \$35,000 & 10 \\
Z & \$500,000 & \$90,000 & 10 \\
\end{longtable}
}

\textbf{Find:} (a) The MARR, and (b) which projects are acceptable.

\textbf{Solution:}

\begin{enumerate}
\def\labelenumi{(\alph{enumi})}
\item
  WACC = (weight\textsubscript{debt} × cost\textsubscript{debt}) +
  (weight\textsubscript{equity} × cost\textsubscript{equity}) = (0.60 ×
  0.08) + (0.40 × 0.14) = 0.048 + 0.056 = 0.104 = 10.4\% MARR = WACC +
  risk premium = 10.4\% + 4\% = \textbf{14.4\%}
\item
  Compute NPV at MARR = 14.4\% (use 14\% for factor approximation):
  (P/A, 14\%, 10) = {[}(1.14)¹⁰ − 1{]} / {[}0.14 × (1.14)¹⁰{]} (1.14)¹⁰
  = 3.7072 = {[}3.7072 − 1{]} / {[}0.14 × 3.7072{]} = 2.7072 / 0.51901 =
  5.2163
\end{enumerate}

More precisely at 14.4\%: (1.144)¹⁰ ≈ 3.8416 (P/A, 14.4\%, 10) =
{[}3.8416 − 1{]} / {[}0.144 × 3.8416{]} = 2.8416 / 0.55319 = 5.1366

NPV\textsubscript{W} = −200,000 + 42,000 × 5.1366 = −200,000 + 215,737 =
\textbf{+\$15,737} ✓ NPV\textsubscript{X} = −350,000 + 68,000 × 5.1366 =
−350,000 + 349,289 = \textbf{−\$711} ✗ NPV\textsubscript{Y} = −150,000 +
35,000 × 5.1366 = −150,000 + 179,781 = \textbf{+\$29,781} ✓
NPV\textsubscript{Z} = −500,000 + 90,000 × 5.1366 = −500,000 + 462,294 =
\textbf{−\$37,706} ✗

\textbf{Projects W and Y are acceptable} (positive NPV at the 14.4\%
MARR). Projects X and Z do not meet the hurdle rate.

\begin{center}\rule{0.5\linewidth}{0.5pt}\end{center}

\section{Problem 19.12.3}\label{problem-19.12.3}

\textbf{Given:} A utility has \$2,000,000 in capital and four
independent projects:

{\def\LTcaptype{none} % do not increment counter
\begin{longtable}[]{@{}lll@{}}
\toprule\noalign{}
Project & Investment & NPV \\
\midrule\noalign{}
\endhead
\bottomrule\noalign{}
\endlastfoot
A & \$700,000 & \$95,000 \\
B & \$500,000 & \$80,000 \\
C & \$900,000 & \$130,000 \\
D & \$1,100,000 & \$145,000 \\
\end{longtable}
}

\textbf{Find:} The optimal portfolio using enumeration of all feasible
combinations.

\textbf{Solution:}

Feasible combinations within \$2,000,000 budget:

{\def\LTcaptype{none} % do not increment counter
\begin{longtable}[]{@{}lll@{}}
\toprule\noalign{}
Combination & Investment & Total NPV \\
\midrule\noalign{}
\endhead
\bottomrule\noalign{}
\endlastfoot
A only & \$700,000 & \$95,000 \\
B only & \$500,000 & \$80,000 \\
C only & \$900,000 & \$130,000 \\
D only & \$1,100,000 & \$145,000 \\
A + B & \$1,200,000 & \$175,000 \\
A + C & \$1,600,000 & \$225,000 \\
A + D & \$1,800,000 & \$240,000 \\
B + C & \$1,400,000 & \$210,000 \\
B + D & \$1,600,000 & \$225,000 \\
C + D & \$2,000,000 & \$275,000 \\
A + B + C & \$2,100,000 & Exceeds budget \\
A + B + D & \$2,300,000 & Exceeds budget \\
B + C + D & \$2,500,000 & Exceeds budget \\
\end{longtable}
}

\textbf{Select projects C and D} for a total investment of \$2,000,000
and maximum NPV of \textbf{\$275,000}.

Note: The PI ranking (C = 0.144, A = 0.136, D = 0.132, B = 0.160) would
suggest starting with B, but enumeration reveals that C + D is optimal
--- demonstrating that PI ranking is a heuristic that does not always
find the global optimum.

\begin{center}\rule{0.5\linewidth}{0.5pt}\end{center}

\section{Problem 19.12.4}\label{problem-19.12.4}

\textbf{Given:} A solar developer has \$8,000,000 and evaluates five
sites. Each site is independent (can build on any subset).

{\def\LTcaptype{none} % do not increment counter
\begin{longtable}[]{@{}lllll@{}}
\toprule\noalign{}
Site & Investment & Annual Revenue & Life & NPV at 9\% \\
\midrule\noalign{}
\endhead
\bottomrule\noalign{}
\endlastfoot
1 & \$2,000,000 & \$310,000 & 25 & \$1,043,000 \\
2 & \$3,500,000 & \$520,000 & 25 & \$1,608,000 \\
3 & \$1,500,000 & \$200,000 & 25 & \$466,000 \\
4 & \$2,800,000 & \$400,000 & 25 & \$1,130,000 \\
5 & \$4,000,000 & \$580,000 & 25 & \$1,696,000 \\
\end{longtable}
}

\textbf{Find:} The portfolio that maximizes NPV within the \$8,000,000
budget.

\textbf{Solution:}

PI ranking: 5 (0.424), 1 (0.522), 2 (0.459), 4 (0.404), 3 (0.311)
Reranked: 1 (0.522), 2 (0.459), 5 (0.424), 4 (0.404), 3 (0.311)

By PI: 1 (\$2M) + 2 (\$3.5M) = \$5.5M (NPV \$2,651,000); remaining
\$2.5M → 3 (\$1.5M) fits → total \$7M, NPV \$3,117,000; remaining \$1M
--- nothing fits.

Check: 1 + 2 + 3 = \$7.0M, NPV = \$3,117,000 1 + 2 + 4 = \$8.3M
(exceeds) 1 + 4 + 3 = \$6.3M, NPV = \$2,639,000 2 + 5 = \$7.5M, NPV =
\$3,304,000; remaining \$500K --- nothing fits 1 + 5 = \$6.0M, NPV =
\$2,739,000; remaining \$2M → 3 fits: 1 + 5 + 3 = \$7.5M, NPV =
\$3,205,000 1 + 4 = \$4.8M; + 3 = \$6.3M; + 2 would be \$8.3M (no).
Remaining from 1+4+3 = \$1.7M --- nothing 2 + 4 = \$6.3M, NPV =
\$2,738,000; + 3 = \$7.8M, NPV = \$3,204,000 4 + 5 = \$6.8M, NPV =
\$2,826,000; + 3 = \$8.3M (exceeds) 1 + 2 + 3 = \$7.0M, NPV =
\$3,117,000 2 + 5 = \$7.5M, NPV = \$3,304,000

\textbf{Select Sites 2 and 5} for a total investment of \$7,500,000 and
maximum NPV of \textbf{\$3,304,000} within the \$8,000,000 budget.

\chapter{Chapter 19 --- Section 19.13: Breakeven and Sensitivity
Analysis}\label{chapter-19-section-19.13-breakeven-and-sensitivity-analysis}

Practice problems covering breakeven analysis, single-parameter
sensitivity, and scenario analysis. Problems range from direct breakeven
calculations to multi-scenario NPV evaluations.

\begin{center}\rule{0.5\linewidth}{0.5pt}\end{center}

\section{Problem 19.13.1}\label{problem-19.13.1}

\textbf{Given:} A plant considers replacing a standard 50 HP pump motor
(efficiency 89\%, cost \$4,500) with a premium efficiency motor
(efficiency 94\%, cost \$7,200). The motor operates at 75\% average
load. Electricity costs \$0.12/kWh, i = 9\%, and both motors have a
12-year life with no salvage value.

\textbf{Find:} The breakeven operating hours per year.

\textbf{Solution:}

Power input (standard): P₁ = 50 × 0.746 × 0.75 / 0.89 = 27.975 / 0.89 =
31.43 kW Power input (premium): P₂ = 50 × 0.746 × 0.75 / 0.94 = 27.975 /
0.94 = 29.76 kW Power savings: ΔP = 31.43 − 29.76 = 1.67 kW

Annual energy cost savings for H hours: ΔE = 1.67 × H × \$0.12 =
\$0.2004 × H

Capital recovery of extra cost: (A/P, 9\%, 12) = {[}0.09 × (1.09)¹²{]} /
{[}(1.09)¹² − 1{]} = {[}0.09 × 2.8127{]} / {[}1.8127{]} = 0.25314 /
1.8127 = 0.13965 ΔCR = (7,200 − 4,500) × 0.13965 = 2,700 × 0.13965 =
\$377.06/year

Breakeven: 0.2004 × H = 377.06 H = 377.06 / 0.2004 = \textbf{1,881
hours/year}

If the motor runs more than 1,881 hours/year (about 7.2 hours/day on
weekdays), the premium motor is more economical.

\begin{center}\rule{0.5\linewidth}{0.5pt}\end{center}

\section{Problem 19.13.2}\label{problem-19.13.2}

\textbf{Given:} A utility considers installing automated switching on a
feeder. The system costs \$250,000 and saves \$C per year in reduced
outage costs over 15 years with no salvage value. The MARR is 8\%.

\textbf{Find:} The breakeven annual savings C that yields NPV = 0.

\textbf{Solution:}

NPV = 0: 0 = −250,000 + C × (P/A, 8\%, 15)

(P/A, 8\%, 15) = {[}(1.08)¹⁵ − 1{]} / {[}0.08 × (1.08)¹⁵{]} (1.08)¹⁵ =
3.1722 = {[}3.1722 − 1{]} / {[}0.08 × 3.1722{]} = 2.1722 / 0.25378 =
8.5595

C = 250,000 / 8.5595 = \textbf{\$29,207/year}

If the automated switching reduces outage costs by more than
\$29,207/year, the investment is justified at the 8\% MARR.

\begin{center}\rule{0.5\linewidth}{0.5pt}\end{center}

\section{Problem 19.13.3}\label{problem-19.13.3}

\textbf{Given:} A \$2,000,000 battery storage project has expected
annual revenue of \$340,000 over 12 years with no salvage value. At MARR
= 10\%.

\textbf{Find:} (a) The base-case NPV, and (b) the NPV at ±15\% and ±30\%
variation in annual revenue.

\textbf{Solution:}

(P/A, 10\%, 12) = {[}(1.10)¹² − 1{]} / {[}0.10 × (1.10)¹²{]} (1.10)¹² =
3.1384 = {[}3.1384 − 1{]} / {[}0.10 × 3.1384{]} = 2.1384 / 0.31384 =
6.8137

\begin{enumerate}
\def\labelenumi{(\alph{enumi})}
\item
  Base NPV = −2,000,000 + 340,000 × 6.8137 = −2,000,000 + 2,316,658 =
  \textbf{\$316,658}
\item
  {\def\LTcaptype{none} % do not increment counter
  \begin{longtable}[]{@{}lll@{}}
  \toprule\noalign{}
  Revenue Variation & Annual Revenue & NPV \\
  \midrule\noalign{}
  \endhead
  \bottomrule\noalign{}
  \endlastfoot
  −30\% & \$238,000 & −2,000,000 + 238,000 × 6.8137 = −\$378,338 \\
  −15\% & \$289,000 & −2,000,000 + 289,000 × 6.8137 = −\$30,840 \\
  Base & \$340,000 & \textbf{\$316,658} \\
  +15\% & \$391,000 & −2,000,000 + 391,000 × 6.8137 = \$664,156 \\
  +30\% & \$442,000 & −2,000,000 + 442,000 × 6.8137 = \$1,011,654 \\
  \end{longtable}
  }
\end{enumerate}

The project becomes unviable if revenue drops by more than about 14\%
below the expected value. At −15\%, the NPV is barely negative
(−\$30,840), indicating high sensitivity to revenue estimates.

\begin{center}\rule{0.5\linewidth}{0.5pt}\end{center}

\section{Problem 19.13.4}\label{problem-19.13.4}

\textbf{Given:} A \$1,500,000 solar carport installation at MARR = 7\%
over 20 years, no salvage. Base case: annual savings = \$180,000.

\textbf{Find:} (a) The base NPV, (b) the breakeven discount rate (IRR),
and (c) the breakeven first cost (at base savings and MARR).

\textbf{Solution:}

\begin{enumerate}
\def\labelenumi{(\alph{enumi})}
\tightlist
\item
  (P/A, 7\%, 20) = {[}(1.07)²⁰ − 1{]} / {[}0.07 × (1.07)²⁰{]} (1.07)²⁰ =
  3.8697 = {[}3.8697 − 1{]} / {[}0.07 × 3.8697{]} = 2.8697 / 0.27088 =
  10.594
\end{enumerate}

Base NPV = −1,500,000 + 180,000 × 10.594 = −1,500,000 + 1,906,920 =
\textbf{\$406,920}

\begin{enumerate}
\def\labelenumi{(\alph{enumi})}
\setcounter{enumi}{1}
\tightlist
\item
  Breakeven discount rate (IRR): (P/A, i*, 20) = 1,500,000 / 180,000 =
  8.3333
\end{enumerate}

Try i = 10\%: (P/A, 10\%, 20) = {[}(1.10)²⁰ − 1{]} / {[}0.10 ×
(1.10)²⁰{]} = {[}6.7275 − 1{]} / {[}0.67275{]} = 8.5136 Try i = 11\%:
(1.11)²⁰ = 8.0623; (P/A) = 7.0623 / 0.88684 = 7.9633

Interpolate: i* = 10\% + 1\% × (8.5136 − 8.3333) / (8.5136 − 7.9633) =
10\% + 1\% × 0.1803/0.5503 = 10\% + 0.33\% = \textbf{10.33\%}

The project remains viable for any MARR below 10.33\%.

\begin{enumerate}
\def\labelenumi{(\alph{enumi})}
\setcounter{enumi}{2}
\tightlist
\item
  Breakeven first cost: P\textsubscript{max} = 180,000 × (P/A, 7\%, 20)
  = 180,000 × 10.594 = \textbf{\$1,906,920}
\end{enumerate}

The first cost can increase to \$1,906,920 (a 27\% overrun) before the
project becomes unviable.

\begin{center}\rule{0.5\linewidth}{0.5pt}\end{center}

\section{Problem 19.13.5}\label{problem-19.13.5}

\textbf{Given:} A wind farm (\$5,000,000, 25-year life, no salvage) has
three scenarios:

{\def\LTcaptype{none} % do not increment counter
\begin{longtable}[]{@{}llll@{}}
\toprule\noalign{}
Parameter & Pessimistic & Most Likely & Optimistic \\
\midrule\noalign{}
\endhead
\bottomrule\noalign{}
\endlastfoot
Capacity factor & 22\% & 30\% & 36\% \\
Annual revenue & \$380,000 & \$520,000 & \$630,000 \\
Annual O\&M & \$95,000 & \$75,000 & \$65,000 \\
\end{longtable}
}

The MARR is 8\%.

\textbf{Find:} The NPV for each scenario.

\textbf{Solution:}

(P/A, 8\%, 25) = {[}(1.08)²⁵ − 1{]} / {[}0.08 × (1.08)²⁵{]} (1.08)²⁵ =
6.8485 = {[}6.8485 − 1{]} / {[}0.08 × 6.8485{]} = 5.8485 / 0.54788 =
10.675

Net annual cash flow for each scenario:

Pessimistic: A = 380,000 − 95,000 = \$285,000 NPV = −5,000,000 + 285,000
× 10.675 = −5,000,000 + 3,042,375 = \textbf{−\$1,957,625}

Most likely: A = 520,000 − 75,000 = \$445,000 NPV = −5,000,000 + 445,000
× 10.675 = −5,000,000 + 4,750,375 = \textbf{−\$249,625}

Optimistic: A = 630,000 − 65,000 = \$565,000 NPV = −5,000,000 + 565,000
× 10.675 = −5,000,000 + 6,031,375 = \textbf{\$1,031,375}

The project is only viable in the optimistic scenario. Even the
most-likely case produces a negative NPV, suggesting the project carries
significant risk at the 8\% MARR. A lower MARR or additional revenue
streams (such as renewable energy credits) would be needed to justify
the investment.

\begin{center}\rule{0.5\linewidth}{0.5pt}\end{center}

\section{Problem 19.13.6}\label{problem-19.13.6}

\textbf{Given:} A factory evaluates a \$400,000 power quality
improvement system. Base estimates: annual savings = \$78,000, life = 10
years, MARR = 9\%, no salvage. The three most uncertain parameters are
savings, life, and MARR.

\textbf{Find:} The NPV sensitivity to ±20\% changes in each parameter
(one at a time).

\textbf{Solution:}

Base case: (P/A, 9\%, 10) = 6.4177 NPV\textsubscript{base} = −400,000 +
78,000 × 6.4177 = −400,000 + 500,581 = \textbf{\$100,581}

\textbf{Savings sensitivity (±20\%):} At \$62,400 (−20\%): NPV =
−400,000 + 62,400 × 6.4177 = −400,000 + 400,464 = \textbf{\$464} At
\$93,600 (+20\%): NPV = −400,000 + 93,600 × 6.4177 = −400,000 + 600,698
= \textbf{\$200,698} Range: \$200,234

\textbf{Life sensitivity (±20\%):} At 8 years: (P/A, 9\%, 8) = 5.5348
NPV = −400,000 + 78,000 × 5.5348 = −400,000 + 431,714 =
\textbf{\$31,714} At 12 years: (P/A, 9\%, 12) = 7.1607 NPV = −400,000 +
78,000 × 7.1607 = −400,000 + 558,535 = \textbf{\$158,535} Range:
\$126,821

\textbf{MARR sensitivity (±20\%):} At 7.2\%: (P/A, 7.2\%, 10) =
{[}(1.072)¹⁰ − 1{]} / {[}0.072 × (1.072)¹⁰{]} (1.072)¹⁰ = 2.0042; =
1.0042 / 0.14430 = 6.960 NPV = −400,000 + 78,000 × 6.960 = −400,000 +
542,880 = \textbf{\$142,880}

At 10.8\%: (P/A, 10.8\%, 10) = {[}(1.108)¹⁰ − 1{]} / {[}0.108 ×
(1.108)¹⁰{]} (1.108)¹⁰ = 2.7889; = 1.7889 / 0.30120 = 5.939 NPV =
−400,000 + 78,000 × 5.939 = −400,000 + 463,242 = \textbf{\$63,242}
Range: \$79,638

\textbf{Sensitivity ranking:} 1. Annual savings --- most sensitive
(range \$200,234) 2. Project life --- moderate sensitivity (range
\$126,821) 3. MARR --- least sensitive (range \$79,638)

The annual savings estimate deserves the most careful validation, as it
has the greatest impact on the project's economic outcome.

\chapter{Chapter 19 --- Section 19.14: Life Cycle Cost
Analysis}\label{chapter-19-section-19.14-life-cycle-cost-analysis}

Practice problems covering life cycle cost component analysis and
levelized cost of energy (LCOE) calculations. Problems use realistic
infrastructure and generation scenarios.

\begin{center}\rule{0.5\linewidth}{0.5pt}\end{center}

\section{Problem 19.14.1}\label{problem-19.14.1}

\textbf{Given:} A wastewater treatment plant compares two blower systems
over 20 years at i = 7\%.

{\def\LTcaptype{none} % do not increment counter
\begin{longtable}[]{@{}lll@{}}
\toprule\noalign{}
Cost Category & System A (Centrifugal) & System B (Rotary Screw) \\
\midrule\noalign{}
\endhead
\bottomrule\noalign{}
\endlastfoot
Purchase & \$250,000 & \$180,000 \\
Installation & \$35,000 & \$25,000 \\
Annual energy & \$48,000/year & \$58,000/year \\
Annual maintenance & \$6,000/year & \$9,000/year \\
Overhaul & \$30,000 at year 10 & \$20,000 at years 7, 14 \\
Disposal & \$5,000 at year 20 & \$4,000 at year 20 \\
\end{longtable}
}

\textbf{Find:} The life cycle cost of each system and which is more
economical.

\textbf{Solution:}

(P/A, 7\%, 20) = {[}(1.07)²⁰ − 1{]} / {[}0.07 × (1.07)²⁰{]} (1.07)²⁰ =
3.8697 = {[}3.8697 − 1{]} / {[}0.07 × 3.8697{]} = 2.8697 / 0.27088 =
10.594

(P/F, 7\%, 7) = 1/(1.07)⁷ = 1/1.6058 = 0.6228 (P/F, 7\%, 10) =
1/(1.07)¹⁰ = 1/1.9672 = 0.5083 (P/F, 7\%, 14) = 1/(1.07)¹⁴ = 1/2.5785 =
0.3878 (P/F, 7\%, 20) = 1/3.8697 = 0.2584

\textbf{System A:} LCC\textsubscript{A} = 250,000 + 35,000 + (48,000 +
6,000) × 10.594 + 30,000 × 0.5083 + 5,000 × 0.2584 = 285,000 + 54,000 ×
10.594 + 15,249 + 1,292 = 285,000 + 572,076 + 15,249 + 1,292 =
\textbf{\$873,617}

\textbf{System B:} LCC\textsubscript{B} = 180,000 + 25,000 + (58,000 +
9,000) × 10.594 + 20,000 × (0.6228 + 0.3878) + 4,000 × 0.2584 = 205,000
+ 67,000 × 10.594 + 20,000 × 1.0106 + 1,034 = 205,000 + 709,798 + 20,212
+ 1,034 = \textbf{\$936,044}

\textbf{System A has a lower LCC} by \$936,044 − \$873,617 = \$62,427.
Despite the higher purchase price, System A's lower energy costs
(\$10,000/year savings) accumulate to a significant advantage over 20
years.

\begin{center}\rule{0.5\linewidth}{0.5pt}\end{center}

\section{Problem 19.14.2}\label{problem-19.14.2}

\textbf{Given:} A 5 MW wind turbine costs \$6,500,000 to install. Annual
O\&M is \$100,000 escalating at 2\%/year. Annual energy production is
15,000 MWh with 1\%/year degradation. The project life is 20 years and
the discount rate is 8\%.

\textbf{Find:} The LCOE in \$/MWh.

\textbf{Solution:}

\textbf{PW of costs:} PW of capital: \$6,500,000

PW of O\&M (geometric gradient, g = 2\%, i = 8\%):
PW\textsubscript{O\&M} = 100,000 × {[}1 − (1.02)²⁰(1.08)⁻²⁰{]} / (0.08 −
0.02) (1.02)²⁰ = 1.4859; (1.08)²⁰ = 4.6610 = 100,000 × {[}1 −
1.4859/4.6610{]} / 0.06 = 100,000 × {[}1 − 0.3188{]} / 0.06 = 100,000 ×
0.6812 / 0.06 = 100,000 × 11.353 = \$1,135,300

PW of total costs = 6,500,000 + 1,135,300 = \$7,635,300

\textbf{PW of energy production} (geometric gradient, g = −1\%, i =
8\%): PW\textsubscript{energy} = 15,000 × {[}1 − (0.99)²⁰(1.08)⁻²⁰{]} /
(0.08 − (−0.01)) (0.99)²⁰ = 0.8179 = 15,000 × {[}1 − 0.8179/4.6610{]} /
0.09 = 15,000 × {[}1 − 0.1755{]} / 0.09 = 15,000 × 0.8245 / 0.09 =
15,000 × 9.161 = 137,415 MWh (present-worth equivalent)

\textbf{LCOE = 7,635,300 / 137,415 = \$55.56/MWh}

\begin{center}\rule{0.5\linewidth}{0.5pt}\end{center}

\section{Problem 19.14.3}\label{problem-19.14.3}

\textbf{Given:} A hospital compares two emergency generator systems over
25 years at i = 5\%.

{\def\LTcaptype{none} % do not increment counter
\begin{longtable}[]{@{}lll@{}}
\toprule\noalign{}
Category & Diesel Generator & Natural Gas Generator \\
\midrule\noalign{}
\endhead
\bottomrule\noalign{}
\endlastfoot
Purchase + install & \$400,000 & \$520,000 \\
Annual fuel & \$18,000 & \$12,000 \\
Annual maintenance & \$8,000 & \$5,000 \\
Major overhaul & \$45,000 at years 8, 16 & \$35,000 at years 10, 20 \\
Emissions compliance & \$5,000/year & \$0/year \\
Disposal & \$10,000 at year 25 & \$8,000 at year 25 \\
\end{longtable}
}

\textbf{Find:} The LCC of each system and the preferred choice.

\textbf{Solution:}

(P/A, 5\%, 25) = {[}(1.05)²⁵ − 1{]} / {[}0.05 × (1.05)²⁵{]} (1.05)²⁵ =
3.3864 = {[}3.3864 − 1{]} / {[}0.05 × 3.3864{]} = 2.3864 / 0.16932 =
14.094

(P/F, 5\%, 8) = 1/(1.05)⁸ = 1/1.4775 = 0.6768 (P/F, 5\%, 10) =
1/(1.05)¹⁰ = 1/1.6289 = 0.6139 (P/F, 5\%, 16) = 1/(1.05)¹⁶ = 1/2.1829 =
0.4581 (P/F, 5\%, 20) = 1/(1.05)²⁰ = 1/2.6533 = 0.3769 (P/F, 5\%, 25) =
1/3.3864 = 0.2953

\textbf{Diesel:} LCC = 400,000 + (18,000 + 8,000 + 5,000) × 14.094 +
45,000 × (0.6768 + 0.4581) + 10,000 × 0.2953 = 400,000 + 31,000 × 14.094
+ 45,000 × 1.1349 + 2,953 = 400,000 + 436,914 + 51,071 + 2,953 =
\textbf{\$890,938}

\textbf{Natural Gas:} LCC = 520,000 + (12,000 + 5,000) × 14.094 + 35,000
× (0.6139 + 0.3769) + 8,000 × 0.2953 = 520,000 + 17,000 × 14.094 +
35,000 × 0.9908 + 2,362 = 520,000 + 239,598 + 34,678 + 2,362 =
\textbf{\$796,638}

\textbf{The natural gas generator has a lower LCC} by \$890,938 −
\$796,638 = \$94,300. Lower fuel costs, lower maintenance, and no
emissions compliance costs more than offset the higher purchase price
over 25 years.

\begin{center}\rule{0.5\linewidth}{0.5pt}\end{center}

\section{Problem 19.14.4}\label{problem-19.14.4}

\textbf{Given:} A 20 MW solar PV plant costs \$18,000,000. Annual O\&M
is \$200,000 escalating at 2.5\%/year. Annual production is 35,000 MWh
with 0.6\%/year degradation. An inverter replacement of \$1,500,000
occurs at year 12. The project life is 30 years and the discount rate is
6\%.

\textbf{Find:} The LCOE in \$/MWh.

\textbf{Solution:}

\textbf{PW of costs:} Capital: \$18,000,000

PW of O\&M (geometric gradient, g = 2.5\%, i = 6\%):
PW\textsubscript{O\&M} = 200,000 × {[}1 − (1.025)³⁰(1.06)⁻³⁰{]} / (0.06
− 0.025) (1.025)³⁰ = 2.0976; (1.06)³⁰ = 5.7435 = 200,000 × {[}1 −
2.0976/5.7435{]} / 0.035 = 200,000 × {[}1 − 0.3652{]} / 0.035 = 200,000
× 0.6348 / 0.035 = 200,000 × 18.137 = \$3,627,400

Inverter replacement: 1,500,000 × (P/F, 6\%, 12) = 1,500,000 ×
1/(1.06)¹² = 1,500,000 × 1/2.0122 = 1,500,000 × 0.4970 = \$745,500

PW of total costs = 18,000,000 + 3,627,400 + 745,500 = \$22,372,900

\textbf{PW of energy production} (geometric gradient, g = −0.6\%, i =
6\%): PW\textsubscript{energy} = 35,000 × {[}1 − (0.994)³⁰(1.06)⁻³⁰{]} /
(0.06 − (−0.006)) (0.994)³⁰ = 0.8348 = 35,000 × {[}1 − 0.8348/5.7435{]}
/ 0.066 = 35,000 × {[}1 − 0.1453{]} / 0.066 = 35,000 × 0.8547 / 0.066 =
35,000 × 12.950 = 453,250 MWh (present-worth equivalent)

\textbf{LCOE = 22,372,900 / 453,250 = \$49.36/MWh}

This LCOE of approximately \$49/MWh reflects a competitive utility-scale
solar installation, with the inverter replacement at year 12 adding
about \$1.64/MWh to the levelized cost.

\chapter{Appendix A -- Section A.1: Imaginary
Numbers}\label{appendix-a-section-a.1-imaginary-numbers}

Practice problems covering the imaginary unit j, simplification of
square roots of negative numbers, and the cyclic powers of j.

\begin{center}\rule{0.5\linewidth}{0.5pt}\end{center}

\section{Problem A.1.1}\label{problem-a.1.1}

\textbf{Given:} The expression √(-144).

\textbf{Find:} Simplify and express as an imaginary number.

\textbf{Solution:} √(-144) = √(144 x (-1)) = √144 x √(-1) = 12 x j =
j12.

\textbf{The result is j12.}

\begin{center}\rule{0.5\linewidth}{0.5pt}\end{center}

\section{Problem A.1.2}\label{problem-a.1.2}

\textbf{Given:} The expression √(-0.25).

\textbf{Find:} Simplify and express as an imaginary number.

\textbf{Solution:} √(-0.25) = √(0.25 x (-1)) = √0.25 x √(-1) = 0.5 x j =
j0.5.

\textbf{The result is j0.5.}

\begin{center}\rule{0.5\linewidth}{0.5pt}\end{center}

\section{Problem A.1.3}\label{problem-a.1.3}

\textbf{Given:} The expression j⁶ + j¹³ - j²⁰.

\textbf{Find:} Simplify to a single value.

\textbf{Solution:} j⁶: 6 mod 4 = 2, so j⁶ = j² = -1. j¹³: 13 mod 4 = 1,
so j¹³ = j¹ = j. j²⁰: 20 mod 4 = 0, so j²⁰ = j⁰ = 1.

j⁶ + j¹³ - j²⁰ = -1 + j - 1 = -2 + j.

\textbf{The result is -2 + j.}

\begin{center}\rule{0.5\linewidth}{0.5pt}\end{center}

\section{Problem A.1.4}\label{problem-a.1.4}

\textbf{Given:} The expression j⁻² + j⁻⁵ + j⁻⁸.

\textbf{Find:} Simplify to a single value.

\textbf{Solution:} j⁻²: -2 mod 4 = 2, so j⁻² = j² = -1. j⁻⁵: -5 mod 4 =
3, so j⁻⁵ = j³ = -j. j⁻⁸: -8 mod 4 = 0, so j⁻⁸ = j⁰ = 1.

j⁻² + j⁻⁵ + j⁻⁸ = -1 + (-j) + 1 = -j.

\textbf{The result is -j.}

\begin{center}\rule{0.5\linewidth}{0.5pt}\end{center}

\section{Problem A.1.5}\label{problem-a.1.5}

\textbf{Given:} The expression (j³)⁴ x (j²)⁵.

\textbf{Find:} Simplify to a single value.

\textbf{Solution:} (j³)⁴ = j¹²: 12 mod 4 = 0, so j¹² = 1. (j²)⁵ = j¹⁰:
10 mod 4 = 2, so j¹⁰ = j² = -1.

(j³)⁴ x (j²)⁵ = 1 x (-1) = -1.

\textbf{The result is -1.}

\begin{center}\rule{0.5\linewidth}{0.5pt}\end{center}

\section{Problem A.1.6}\label{problem-a.1.6}

\textbf{Given:} The expression √(-18) x √(-2).

\textbf{Find:} Simplify to a real number.

\textbf{Solution:} √(-18) = j√18 = j x 3√2. √(-2) = j√2.

√(-18) x √(-2) = (j x 3√2)(j x √2) = j² x 3 x 2 = (-1) x 6 = -6.

\textbf{The result is -6.}

\chapter{Appendix A -- Section A.2: Complex
Numbers}\label{appendix-a-section-a.2-complex-numbers}

Practice problems covering rectangular form, complex arithmetic, and the
complex conjugate.

\begin{center}\rule{0.5\linewidth}{0.5pt}\end{center}

\section{Problem A.2.1}\label{problem-a.2.1}

\textbf{Given:} An impedance Z = 33 + j56 Ω.

\textbf{Find:} The resistance, reactance, magnitude, and angle.

\textbf{Solution:} Resistance: R = Re\{Z\} = 33 Ω. Reactance: X =
Im\{Z\} = 56 Ω (inductive, since positive).

Magnitude: \textbar Z\textbar{} = √(33² + 56²) = √(1089 + 3136) = √4225
= 65 Ω. Angle: θ = arctan(56/33) = arctan(1.6970) = 59.49°.

\textbf{Z = 65∠59.49° Ω. The resistance is 33 Ω and the reactance is 56
Ω.}

\begin{center}\rule{0.5\linewidth}{0.5pt}\end{center}

\section{Problem A.2.2}\label{problem-a.2.2}

\textbf{Given:} Z₁ = 8 - j6 and Z₂ = -3 + j4.

\textbf{Find:} Z₁ + Z₂, Z₁ - Z₂, and Z₁ x Z₂.

\textbf{Solution:} Addition: Z₁ + Z₂ = (8 + (-3)) + j(-6 + 4) = 5 - j2.

Subtraction: Z₁ - Z₂ = (8 - (-3)) + j(-6 - 4) = 11 - j10.

Multiplication: Z₁ x Z₂ = (8)(-3) - (-6)(4) + j((8)(4) + (-6)(-3)) = -24
+ 24 + j(32 + 18) = 0 + j50 = j50.

\textbf{Z₁ + Z₂ = 5 - j2, Z₁ - Z₂ = 11 - j10, Z₁ x Z₂ = j50.}

\begin{center}\rule{0.5\linewidth}{0.5pt}\end{center}

\section{Problem A.2.3}\label{problem-a.2.3}

\textbf{Given:} Z₁ = 10 + j5 Ω and Z₂ = 4 - j3 Ω.

\textbf{Find:} Z₁ / Z₂ in rectangular form.

\textbf{Solution:} Z₁ / Z₂ = (10 + j5) / (4 - j3). Multiply numerator
and denominator by the conjugate of Z₂:

Numerator: (10 + j5)(4 + j3) = 40 + j30 + j20 + j²15 = 40 + j30 + j20 -
15 = 25 + j50. Denominator: 4² + 3² = 16 + 9 = 25.

Z₁ / Z₂ = (25 + j50) / 25 = 1 + j2.

\textbf{Z₁ / Z₂ = 1 + j2.}

\begin{center}\rule{0.5\linewidth}{0.5pt}\end{center}

\section{Problem A.2.4}\label{problem-a.2.4}

\textbf{Given:} Z = 7 - j24.

\textbf{Find:} The complex conjugate Z\emph{, the product Z x Z}, and
verify that Z x Z* = \textbar Z\textbar².

\textbf{Solution:} Z* = 7 + j24.

Z x Z* = (7 - j24)(7 + j24) = 49 + j168 - j168 - j²576 = 49 + 576 = 625.

\textbar Z\textbar{} = √(7² + 24²) = √(49 + 576) = √625 = 25.
\textbar Z\textbar² = 625.

\textbf{Z* = 7 + j24, Z x Z* = 625, and \textbar Z\textbar² = 625,
confirming Z x Z* = \textbar Z\textbar².}

\begin{center}\rule{0.5\linewidth}{0.5pt}\end{center}

\section{Problem A.2.5}\label{problem-a.2.5}

\textbf{Given:} Three impedances in series: Z₁ = 10 + j0 Ω (resistor),
Z₂ = 0 + j15 Ω (inductor), Z₃ = 0 - j8 Ω (capacitor).

\textbf{Find:} The total impedance Z\textsubscript{total} in rectangular
and polar form.

\textbf{Solution:} Z\textsubscript{total} = Z₁ + Z₂ + Z₃ = (10 + 0 + 0)
+ j(0 + 15 - 8) = 10 + j7 Ω.

Magnitude: \textbar Z\textsubscript{total}\textbar{} = √(10² + 7²) =
√(100 + 49) = √149 = 12.21 Ω. Angle: θ = arctan(7/10) = arctan(0.7) =
34.99°.

\textbf{Z\textsubscript{total} = 10 + j7 Ω = 12.21∠34.99° Ω.}

\begin{center}\rule{0.5\linewidth}{0.5pt}\end{center}

\section{Problem A.2.6}\label{problem-a.2.6}

\textbf{Given:} Two impedances in parallel: Z₁ = 20 + j0 Ω and Z₂ = 0 +
j20 Ω.

\textbf{Find:} The parallel combination Z\textsubscript{p} = (Z₁ x Z₂) /
(Z₁ + Z₂).

\textbf{Solution:} Z₁ + Z₂ = 20 + j20 Ω. Z₁ x Z₂ = (20)(j20) = j400.

Z\textsubscript{p} = j400 / (20 + j20). Multiply by conjugate:
Z\textsubscript{p} = j400(20 - j20) / (20² + 20²) = (j8000 - j²8000) /
800 = (8000 + j8000) / 800 = 10 + j10 Ω.

Verification in polar form: \textbar Z\textsubscript{p}\textbar{} =
√(100 + 100) = √200 = 14.14 Ω, θ = 45°.

\textbf{Z\textsubscript{p} = 10 + j10 Ω = 14.14∠45° Ω.}

\begin{center}\rule{0.5\linewidth}{0.5pt}\end{center}

\section{Problem A.2.7}\label{problem-a.2.7}

\textbf{Given:} The equation Z² = -9 + j40, where Z = a + jb is a
complex impedance.

\textbf{Find:} The value of Z.

\textbf{Solution:} Let Z = a + jb. Then Z² = a² - b² + j2ab. Setting
real and imaginary parts equal: a² - b² = -9 \ldots{} (1) 2ab = 40, so b
= 20/a \ldots{} (2)

Substitute (2) into (1): a² - (20/a)² = -9 a² - 400/a² = -9 a⁴ + 9a² -
400 = 0

Let u = a²: u² + 9u - 400 = 0. u = (-9 + √(81 + 1600)) / 2 = (-9 +
√1681) / 2 = (-9 + 41) / 2 = 16. So a² = 16, a = 4 (taking positive
root). b = 20/4 = 5.

Verification: Z = 4 + j5, Z² = 16 - 25 + j40 = -9 + j40.

\textbf{Z = 4 + j5 (or Z = -4 - j5 for the negative root).}

\chapter{Appendix A -- Section A.3: Polar and Exponential
Forms}\label{appendix-a-section-a.3-polar-and-exponential-forms}

Practice problems covering polar form conversion, Euler's formula, and
rectangular-polar interconversion.

\begin{center}\rule{0.5\linewidth}{0.5pt}\end{center}

\section{Problem A.3.1}\label{problem-a.3.1}

\textbf{Given:} Z = -5 - j12.

\textbf{Find:} Express Z in polar form (degrees).

\textbf{Solution:} Magnitude: \textbar Z\textbar{} = √((-5)² + (-12)²) =
√(25 + 144) = √169 = 13.

Reference angle: arctan(12/5) = arctan(2.4) = 67.38°. Since both the
real and imaginary parts are negative, Z lies in the third quadrant: θ =
-(180° - 67.38°) = -112.62° (or equivalently 247.38°).

\textbf{Z = 13∠-112.62°.}

\begin{center}\rule{0.5\linewidth}{0.5pt}\end{center}

\section{Problem A.3.2}\label{problem-a.3.2}

\textbf{Given:} Z₁ = 8∠30° and Z₂ = 4∠-60°.

\textbf{Find:} Z₁ x Z₂ and Z₁ / Z₂ in polar form.

\textbf{Solution:} Multiplication: Z₁ x Z₂ = (8 x 4)∠(30° + (-60°)) =
32∠-30°.

Division: Z₁ / Z₂ = (8/4)∠(30° - (-60°)) = 2∠90°.

\textbf{Z₁ x Z₂ = 32∠-30° and Z₁ / Z₂ = 2∠90°.}

\begin{center}\rule{0.5\linewidth}{0.5pt}\end{center}

\section{Problem A.3.3}\label{problem-a.3.3}

\textbf{Given:} Z = 50∠-135°.

\textbf{Find:} Express in exponential form and convert to rectangular
form.

\textbf{Solution:} Exponential form: Convert angle to radians: -135° x
(π/180°) = -3π/4 rad. Z = 50 x e\textsuperscript{-j3π/4}.

Rectangular form: a = 50 x cos(-135°) = 50 x (-0.7071) = -35.36. b = 50
x sin(-135°) = 50 x (-0.7071) = -35.36.

Verification: \textbar Z\textbar{} = √(35.36² + 35.36²) = √(1250 + 1250)
= √2500 = 50.

\textbf{Z = 50 x e\textsuperscript{-j3π/4} = -35.36 - j35.36.}

\begin{center}\rule{0.5\linewidth}{0.5pt}\end{center}

\section{Problem A.3.4}\label{problem-a.3.4}

\textbf{Given:} V₁ = 30∠20° V and V₂ = 40∠-70° V in series.

\textbf{Find:} The total voltage V\textsubscript{total} in both
rectangular and polar form.

\textbf{Solution:} Convert to rectangular: V₁ = 30 cos(20°) + j30
sin(20°) = 30(0.9397) + j30(0.3420) = 28.19 + j10.26 V. V₂ = 40
cos(-70°) + j40 sin(-70°) = 40(0.3420) + j40(-0.9397) = 13.68 - j37.59
V.

V\textsubscript{total} = (28.19 + 13.68) + j(10.26 - 37.59) = 41.87 -
j27.33 V.

Convert back to polar: \textbar V\textsubscript{total}\textbar{} =
√(41.87² + 27.33²) = √(1753.1 + 746.9) = √2500 = 50.0 V. θ =
arctan(-27.33/41.87) = arctan(-0.6528) = -33.13°.

\textbf{V\textsubscript{total} = 41.87 - j27.33 V = 50.0∠-33.13° V.}

\begin{center}\rule{0.5\linewidth}{0.5pt}\end{center}

\section{Problem A.3.5}\label{problem-a.3.5}

\textbf{Given:} The complex number Z = 6 x e\textsuperscript{jπ/6}.

\textbf{Find:} Express in polar form (degrees) and rectangular form.

\textbf{Solution:} The exponential form is Z = 6 x
e\textsuperscript{jπ/6}. Angle in degrees: π/6 rad = 30°. Polar form: Z
= 6∠30°.

Rectangular form: a = 6 cos(30°) = 6 x 0.8660 = 5.196. b = 6 sin(30°) =
6 x 0.5 = 3.0.

\textbf{Z = 6∠30° = 5.196 + j3.0.}

\begin{center}\rule{0.5\linewidth}{0.5pt}\end{center}

\section{Problem A.3.6}\label{problem-a.3.6}

\textbf{Given:} An AC current has the phasor I = 2.5∠-45° A, and it
flows through an impedance Z = 100∠60° Ω.

\textbf{Find:} The voltage phasor V = I x Z in polar form, then convert
to rectangular form.

\textbf{Solution:} V = I x Z = (2.5 x 100)∠(-45° + 60°) = 250∠15° V.

Rectangular form: a = 250 cos(15°) = 250 x 0.9659 = 241.5 V. b = 250
sin(15°) = 250 x 0.2588 = 64.70 V.

\textbf{V = 250∠15° V = 241.5 + j64.70 V.}

\begin{center}\rule{0.5\linewidth}{0.5pt}\end{center}

\section{Problem A.3.7}\label{problem-a.3.7}

\textbf{Given:} Three impedances: Z₁ = 10∠0° Ω, Z₂ = 20∠90° Ω, Z₃ =
15∠-90° Ω, all in series.

\textbf{Find:} The total impedance in rectangular and polar form.

\textbf{Solution:} Convert to rectangular: Z₁ = 10 + j0 Ω. Z₂ = 0 + j20
Ω. Z₃ = 0 - j15 Ω.

Z\textsubscript{total} = 10 + j(20 - 15) = 10 + j5 Ω.

Convert to polar: \textbar Z\textsubscript{total}\textbar{} = √(10² +
5²) = √(100 + 25) = √125 = 11.18 Ω. θ = arctan(5/10) = arctan(0.5) =
26.57°.

\textbf{Z\textsubscript{total} = 10 + j5 Ω = 11.18∠26.57° Ω.}

\begin{center}\rule{0.5\linewidth}{0.5pt}\end{center}

\section{Problem A.3.8}\label{problem-a.3.8}

\textbf{Given:} Euler's identity states e\textsuperscript{jπ} + 1 = 0.

\textbf{Find:} Use Euler's formula to verify this, and compute
e\textsuperscript{j2π}.

\textbf{Solution:} Euler's formula: e\textsuperscript{jθ} = cos(θ) + j
sin(θ).

For θ = π: e\textsuperscript{jπ} = cos(π) + j sin(π) = -1 + j(0) = -1.
Therefore e\textsuperscript{jπ} + 1 = -1 + 1 = 0. Verified.

For θ = 2π: e\textsuperscript{j2π} = cos(2π) + j sin(2π) = 1 + j(0) = 1.

\textbf{e\textsuperscript{jπ} = -1, confirming Euler's identity.
e\textsuperscript{j2π} = 1, confirming that a full rotation of 360°
returns to the starting point.}

\chapter{Appendix A -- Section A.4:
Phasors}\label{appendix-a-section-a.4-phasors}

Practice problems covering sinusoidal representation, phasor notation,
phasor arithmetic, and phasor diagrams.

\begin{center}\rule{0.5\linewidth}{0.5pt}\end{center}

\section{Problem A.4.1}\label{problem-a.4.1}

\textbf{Given:} A current waveform i(t) = 14.14 cos(2π x 60t + 45°) A.

\textbf{Find:} The peak amplitude, RMS value, frequency, period, angular
frequency, and phase angle.

\textbf{Solution:} Peak amplitude: I\textsubscript{m} = 14.14 A. RMS
value: I\textsubscript{rms} = I\textsubscript{m} / √2 = 14.14 / 1.414 =
10.0 A. Angular frequency: ω = 2π x 60 = 376.99 rad/s. Frequency: f = 60
Hz. Period: T = 1/f = 1/60 = 16.67 ms. Phase angle: φ = 45° (leading the
reference cosine by 45°).

\textbf{I\textsubscript{m} = 14.14 A, I\textsubscript{rms} = 10.0 A, f =
60 Hz, T = 16.67 ms, ω = 377.0 rad/s, φ = 45°.}

\begin{center}\rule{0.5\linewidth}{0.5pt}\end{center}

\section{Problem A.4.2}\label{problem-a.4.2}

\textbf{Given:} A voltage v(t) = 339.4 cos(314.16t - 90°) V.

\textbf{Find:} Express as an RMS phasor.

\textbf{Solution:} Peak voltage: V\textsubscript{m} = 339.4 V. RMS
voltage: V\textsubscript{rms} = 339.4 / √2 = 339.4 / 1.414 = 240.0 V.
Phase angle: φ = -90°. Frequency: ω = 314.16 rad/s, so f = 314.16 / (2π)
= 50 Hz.

\textbf{The phasor is V = 240.0∠-90° V\textsubscript{rms} at 50 Hz.}

\begin{center}\rule{0.5\linewidth}{0.5pt}\end{center}

\section{Problem A.4.3}\label{problem-a.4.3}

\textbf{Given:} A phasor V = 120∠30° V\textsubscript{rms} at a frequency
of 60 Hz.

\textbf{Find:} The time-domain expression v(t).

\textbf{Solution:} Peak voltage: V\textsubscript{m} = 120 x √2 = 120 x
1.414 = 169.7 V. Angular frequency: ω = 2π x 60 = 376.99 rad/s. Phase
angle: φ = 30°.

\textbf{v(t) = 169.7 cos(377.0t + 30°) V.}

\begin{center}\rule{0.5\linewidth}{0.5pt}\end{center}

\section{Problem A.4.4}\label{problem-a.4.4}

\textbf{Given:} Three voltage phasors in series: V₁ = 50∠0° V, V₂ =
30∠120° V, V₃ = 40∠-90° V.

\textbf{Find:} The total voltage V\textsubscript{total} in polar form.

\textbf{Solution:} Convert to rectangular: V₁ = 50 cos(0°) + j50 sin(0°)
= 50 + j0 V. V₂ = 30 cos(120°) + j30 sin(120°) = 30(-0.5) + j30(0.8660)
= -15 + j25.98 V. V₃ = 40 cos(-90°) + j40 sin(-90°) = 0 - j40 V.

V\textsubscript{total} = (50 - 15 + 0) + j(0 + 25.98 - 40) = 35 - j14.02
V.

\textbar V\textsubscript{total}\textbar{} = √(35² + 14.02²) = √(1225 +
196.6) = √1421.6 = 37.71 V. θ = arctan(-14.02/35) = arctan(-0.4006) =
-21.83°.

\textbf{V\textsubscript{total} = 37.71∠-21.83° V.}

\begin{center}\rule{0.5\linewidth}{0.5pt}\end{center}

\section{Problem A.4.5}\label{problem-a.4.5}

\textbf{Given:} A series RL circuit with R = 47 Ω and X\textsubscript{L}
= 100 Ω, driven by V\textsubscript{s} = 120∠0° V\textsubscript{rms} at
60 Hz.

\textbf{Find:} The current phasor, the voltage across R, and the voltage
across L.

\textbf{Solution:} Z\textsubscript{total} = R + jX\textsubscript{L} = 47
+ j100 Ω. \textbar Z\textsubscript{total}\textbar{} = √(47² + 100²) =
√(2209 + 10000) = √12209 = 110.5 Ω. θ\textsubscript{Z} = arctan(100/47)
= arctan(2.1277) = 64.82°.

Current: I = V\textsubscript{s} / Z\textsubscript{total} = 120∠0° /
110.5∠64.82° = 1.086∠-64.82° A\textsubscript{rms}.

V\textsubscript{R} = I x R = 1.086∠-64.82° x 47 = 51.04∠-64.82°
V\textsubscript{rms}. V\textsubscript{L} = I x jX\textsubscript{L} =
1.086∠-64.82° x 100∠90° = 108.6∠25.18° V\textsubscript{rms}.

Verification: V\textsubscript{R} + V\textsubscript{L} in rectangular:
V\textsubscript{R} = 51.04 cos(-64.82°) + j51.04 sin(-64.82°) = 21.74 -
j46.19 V. V\textsubscript{L} = 108.6 cos(25.18°) + j108.6 sin(25.18°) =
98.26 + j46.23 V. V\textsubscript{total} = (21.74 + 98.26) + j(-46.19 +
46.23) = 120.0 + j0.04 ≈ 120∠0° V. Confirmed.

\textbf{I = 1.086∠-64.82° A\textsubscript{rms}, V\textsubscript{R} =
51.04∠-64.82° V, V\textsubscript{L} = 108.6∠25.18° V.}

\begin{center}\rule{0.5\linewidth}{0.5pt}\end{center}

\section{Problem A.4.6}\label{problem-a.4.6}

\textbf{Given:} Two current sources feeding a node: I₁ = 5∠0° A and I₂ =
5∠180° A.

\textbf{Find:} The total current entering the node and describe what the
phasor diagram shows.

\textbf{Solution:} Convert to rectangular: I₁ = 5 + j0 A. I₂ = 5
cos(180°) + j5 sin(180°) = -5 + j0 A.

I\textsubscript{total} = (5 - 5) + j(0 + 0) = 0 + j0 = 0 A.

The two currents are exactly 180° out of phase (anti-phase) with equal
magnitudes, so they completely cancel. On the phasor diagram, I₁ points
to the right along the positive real axis and I₂ points to the left
along the negative real axis, forming a straight line through the origin
with zero resultant.

\textbf{I\textsubscript{total} = 0 A. The two equal and opposite phasors
cancel completely.}

\begin{center}\rule{0.5\linewidth}{0.5pt}\end{center}

\section{Problem A.4.7}\label{problem-a.4.7}

\textbf{Given:} A phasor diagram shows a current I = 10∠0° A through a
series RLC circuit where V\textsubscript{R} = 50∠0° V,
V\textsubscript{L} = 80∠90° V, and V\textsubscript{C} = 30∠-90° V.

\textbf{Find:} The total voltage, impedance, and describe whether the
circuit is inductive or capacitive.

\textbf{Solution:} V\textsubscript{total} = V\textsubscript{R} +
V\textsubscript{L} + V\textsubscript{C}. In rectangular form:
V\textsubscript{R} = 50 + j0 V. V\textsubscript{L} = 0 + j80 V.
V\textsubscript{C} = 0 - j30 V.

V\textsubscript{total} = 50 + j(80 - 30) = 50 + j50 V.
\textbar V\textsubscript{total}\textbar{} = √(50² + 50²) = √5000 = 70.71
V. θ = arctan(50/50) = 45°.

Z = V\textsubscript{total} / I = 70.71∠45° / 10∠0° = 7.071∠45° Ω = 5 +
j5 Ω. The impedance angle is positive (45°), meaning voltage leads
current. The circuit is net inductive since X\textsubscript{L}
\textgreater{} X\textsubscript{C}.

\textbf{V\textsubscript{total} = 70.71∠45° V. Z = 7.071∠45° Ω. The
circuit is inductive (voltage leads current by 45°).}

\begin{center}\rule{0.5\linewidth}{0.5pt}\end{center}

\section{Problem A.4.8}\label{problem-a.4.8}

\textbf{Given:} An AC voltage source v(t) = 100 cos(1000t) V drives a 50
μF capacitor.

\textbf{Find:} The current phasor and the time-domain expression for
i(t).

\textbf{Solution:} Phasor: V = 100∠0° V (peak). Angular frequency: ω =
1000 rad/s.

Z\textsubscript{C} = 1/(jωC) = 1/(j x 1000 x 50 x 10⁻⁶) = 1/(j0.05) =
-j20 Ω = 20∠-90° Ω.

I = V / Z\textsubscript{C} = 100∠0° / 20∠-90° = 5∠90° A.

Time-domain: i(t) = 5 cos(1000t + 90°) A.

This confirms that in a capacitor, current leads voltage by 90°.

\textbf{I = 5∠90° A (peak), i(t) = 5 cos(1000t + 90°) A.}

\chapter{Appendix A -- Section A.5: Applications in Circuit
Analysis}\label{appendix-a-section-a.5-applications-in-circuit-analysis}

Practice problems covering impedance and admittance, voltage and current
phasors, and power in phasor form.

\begin{center}\rule{0.5\linewidth}{0.5pt}\end{center}

\section{Problem A.5.1}\label{problem-a.5.1}

\textbf{Given:} A 220 Ω resistor, a 33 mH inductor, and a 22 nF
capacitor are connected in series at a frequency of 10 kHz.

\textbf{Find:} The impedance of each element, the total impedance, and
the total admittance.

\textbf{Solution:} ω = 2π x 10,000 = 62,831.9 rad/s.

Z\textsubscript{R} = 220 Ω. Z\textsubscript{L} = jωL = j x 62,831.9 x
0.033 = j2073.5 Ω. Z\textsubscript{C} = -j/(ωC) = -j/(62,831.9 x 22 x
10⁻⁹) = -j/(1.3823 x 10⁻³) = -j723.4 Ω.

Z\textsubscript{total} = 220 + j2073.5 - j723.4 = 220 + j1350.1 Ω.
\textbar Z\textsubscript{total}\textbar{} = √(220² + 1350.1²) = √(48,400
+ 1,822,770) = √1,871,170 = 1368.3 Ω. θ = arctan(1350.1/220) =
arctan(6.137) = 80.74°.

Y = 1/Z\textsubscript{total} = 1/1368.3∠80.74° = 7.308 x 10⁻⁴∠-80.74° S.
Y = 7.308 x 10⁻⁴(cos(-80.74°) + j sin(-80.74°)) = (1.175 - j7.213) x
10⁻⁴ S.

\textbf{Z\textsubscript{total} = 220 + j1350.1 Ω = 1368.3∠80.74° Ω. Y =
7.308 x 10⁻⁴∠-80.74° S.}

\begin{center}\rule{0.5\linewidth}{0.5pt}\end{center}

\section{Problem A.5.2}\label{problem-a.5.2}

\textbf{Given:} A parallel RC circuit with R = 1 kΩ and C = 10 μF at f =
60 Hz.

\textbf{Find:} The total admittance, total impedance, and the phase
angle.

\textbf{Solution:} ω = 2π x 60 = 376.99 rad/s.

Y\textsubscript{R} = 1/R = 1/1000 = 1.0 x 10⁻³ S. Y\textsubscript{C} =
jωC = j x 376.99 x 10 x 10⁻⁶ = j3.770 x 10⁻³ S.

Y\textsubscript{total} = Y\textsubscript{R} + Y\textsubscript{C} = (1.0
+ j3.770) x 10⁻³ S. \textbar Y\textsubscript{total}\textbar{} = √(1.0² +
3.770²) x 10⁻³ = √(1 + 14.21) x 10⁻³ = 3.900 x 10⁻³ S.
θ\textsubscript{Y} = arctan(3.770/1.0) = 75.14°.

Z\textsubscript{total} = 1/Y\textsubscript{total} = 1/(3.900 x
10⁻³∠75.14°) = 256.4∠-75.14° Ω. In rectangular: Z\textsubscript{total} =
256.4 cos(-75.14°) + j256.4 sin(-75.14°) = 65.72 - j247.9 Ω.

\textbf{Y\textsubscript{total} = 3.900 x 10⁻³∠75.14° S.
Z\textsubscript{total} = 256.4∠-75.14° Ω = 65.72 - j247.9 Ω.}

\begin{center}\rule{0.5\linewidth}{0.5pt}\end{center}

\section{Problem A.5.3}\label{problem-a.5.3}

\textbf{Given:} V\textsubscript{s} = 240∠0° V\textsubscript{rms} at 50
Hz drives a series RL circuit with R = 30 Ω and L = 80 mH.

\textbf{Find:} The current phasor, the voltage across the inductor, and
the power factor.

\textbf{Solution:} ω = 2π x 50 = 314.16 rad/s. Z\textsubscript{L} = jωL
= j x 314.16 x 0.08 = j25.13 Ω. Z\textsubscript{total} = 30 + j25.13 Ω.
\textbar Z\textsubscript{total}\textbar{} = √(30² + 25.13²) = √(900 +
631.5) = √1531.5 = 39.13 Ω. θ = arctan(25.13/30) = 39.96°.

I = V\textsubscript{s}/Z\textsubscript{total} = 240∠0° / 39.13∠39.96° =
6.133∠-39.96° A\textsubscript{rms}.

V\textsubscript{L} = I x Z\textsubscript{L} = 6.133∠-39.96° x 25.13∠90°
= 154.1∠50.04° V\textsubscript{rms}.

Power factor: pf = cos(39.96°) = 0.766 lagging (inductive circuit).

\textbf{I = 6.133∠-39.96° A\textsubscript{rms}. V\textsubscript{L} =
154.1∠50.04° V\textsubscript{rms}. Power factor = 0.766 lagging.}

\begin{center}\rule{0.5\linewidth}{0.5pt}\end{center}

\section{Problem A.5.4}\label{problem-a.5.4}

\textbf{Given:} A load draws I = 15∠-36.87° A\textsubscript{rms} from a
source V = 480∠0° V\textsubscript{rms}.

\textbf{Find:} The complex power, real power, reactive power, apparent
power, and power factor.

\textbf{Solution:} Complex power: S = V x I* = 480∠0° x 15∠36.87° =
7200∠36.87° VA. (Note: I* = 15∠+36.87° since conjugation negates the
angle.)

Real power: P = \textbar S\textbar{} cos(36.87°) = 7200 x 0.8 = 5760 W.
Reactive power: Q = \textbar S\textbar{} sin(36.87°) = 7200 x 0.6 = 4320
VAR. Apparent power: \textbar S\textbar{} = 7200 VA. Power factor: pf =
cos(36.87°) = 0.8 lagging (Q is positive, so inductive).

\textbf{S = 7200∠36.87° VA. P = 5760 W. Q = 4320 VAR.
\textbar S\textbar{} = 7200 VA. Power factor = 0.8 lagging.}

\begin{center}\rule{0.5\linewidth}{0.5pt}\end{center}

\section{Problem A.5.5}\label{problem-a.5.5}

\textbf{Given:} A load has S = 1500 + j800 VA (complex power).

\textbf{Find:} The apparent power, power factor, and the load impedance
if V = 120∠0° V\textsubscript{rms}.

\textbf{Solution:} Apparent power: \textbar S\textbar{} = √(1500² +
800²) = √(2,250,000 + 640,000) = √2,890,000 = 1700 VA. Power factor: pf
= P/\textbar S\textbar{} = 1500/1700 = 0.882 lagging (Q \textgreater{}
0, inductive). Power factor angle: θ = arctan(800/1500) = 28.07°.

Current magnitude: I\textsubscript{rms} =
\textbar S\textbar/V\textsubscript{rms} = 1700/120 = 14.17 A. Current
phasor: I = 14.17∠-28.07° A\textsubscript{rms} (lagging the voltage).

Load impedance: Z = V/I = 120∠0° / 14.17∠-28.07° = 8.468∠28.07° Ω. In
rectangular: Z = 8.468 cos(28.07°) + j8.468 sin(28.07°) = 7.474 + j3.983
Ω.

Verification: S = V²/Z* = 120²/(8.468∠-28.07°) = 14400/8.468 ∠28.07° =
1700.8∠28.07° VA. Confirmed.

\textbf{\textbar S\textbar{} = 1700 VA. Power factor = 0.882 lagging. Z
= 8.468∠28.07° Ω = 7.474 + j3.983 Ω.}

\begin{center}\rule{0.5\linewidth}{0.5pt}\end{center}

\section{Problem A.5.6}\label{problem-a.5.6}

\textbf{Given:} A 50 Ω resistor and a 100 Ω inductor reactance are in
parallel at 60 Hz, driven by V = 100∠0° V\textsubscript{rms}.

\textbf{Find:} The total current, the power drawn from the source, and
the power factor.

\textbf{Solution:} I\textsubscript{R} = V/R = 100∠0° / 50 = 2∠0°
A\textsubscript{rms}. I\textsubscript{L} = V/(jX\textsubscript{L}) =
100∠0° / 100∠90° = 1∠-90° A\textsubscript{rms}.

I\textsubscript{total} = I\textsubscript{R} + I\textsubscript{L} = (2 +
j0) + (0 - j1) = 2 - j1 A. \textbar I\textsubscript{total}\textbar{} =
√(4 + 1) = √5 = 2.236 A\textsubscript{rms}. θ\textsubscript{I} =
arctan(-1/2) = -26.57°.

Complex power: S = V x I\textsubscript{total}* = 100∠0° x 2.236∠26.57° =
223.6∠26.57° VA. P = 223.6 cos(26.57°) = 200 W. Q = 223.6 sin(26.57°) =
100 VAR. Power factor: pf = cos(26.57°) = 0.894 lagging.

\textbf{I\textsubscript{total} = 2.236∠-26.57° A\textsubscript{rms}. P =
200 W. Power factor = 0.894 lagging.}

\begin{center}\rule{0.5\linewidth}{0.5pt}\end{center}

\section{Problem A.5.7}\label{problem-a.5.7}

\textbf{Given:} A motor draws S = 10,000∠25° VA from a 480∠0°
V\textsubscript{rms} supply. A capacitor bank is to be added in parallel
to correct the power factor to 0.95 lagging.

\textbf{Find:} The required capacitive reactive power and the capacitor
value at 60 Hz.

\textbf{Solution:} Current power: P = 10,000 cos(25°) = 9063 W.
Q\textsubscript{old} = 10,000 sin(25°) = 4226 VAR.

New power factor angle: θ\textsubscript{new} = arccos(0.95) = 18.19°.
New reactive power: Q\textsubscript{new} = P x tan(18.19°) = 9063 x
0.3287 = 2979 VAR. Required capacitive VAR: Q\textsubscript{C} =
Q\textsubscript{old} - Q\textsubscript{new} = 4226 - 2979 = 1247 VAR.

Capacitor current: I\textsubscript{C} = Q\textsubscript{C}/V = 1247/480
= 2.598 A. X\textsubscript{C} = V/I\textsubscript{C} = 480/2.598 = 184.8
Ω. C = 1/(ωX\textsubscript{C}) = 1/(2π x 60 x 184.8) = 1/(69,655) =
14.36 μF.

\textbf{Q\textsubscript{C} = 1247 VAR. C = 14.36 μF is required to
correct the power factor to 0.95 lagging.}

\chapter{Appendix B -- Section B.1: The Arctangent
Function}\label{appendix-b-section-b.1-the-arctangent-function}

Practice problems covering the definition, range, and quadrant ambiguity
of the arctan function.

\begin{center}\rule{0.5\linewidth}{0.5pt}\end{center}

\section{Problem B.1.1}\label{problem-b.1.1}

\textbf{Given:} The ratios: arctan(√3), arctan(0), and arctan(-√3).

\textbf{Find:} The angle for each and identify the quadrant.

\textbf{Solution:} arctan(√3) = arctan(1.732) = 60° (quadrant I).
arctan(0) = 0° (on the positive real axis). arctan(-√3) = arctan(-1.732)
= -60° (quadrant IV).

All results fall within the range -90° to +90°, confirming that arctan
outputs only quadrant I or quadrant IV angles.

\textbf{arctan(√3) = 60°, arctan(0) = 0°, arctan(-√3) = -60°.}

\begin{center}\rule{0.5\linewidth}{0.5pt}\end{center}

\section{Problem B.1.2}\label{problem-b.1.2}

\textbf{Given:} Two impedances Z₁ = 4 + j3 Ω and Z₂ = -4 - j3 Ω.

\textbf{Find:} Compute arctan(b/a) for each and show the quadrant
ambiguity. Determine the correct angles.

\textbf{Solution:} For Z₁: arctan(3/4) = arctan(0.75) = 36.87°. Z₁ is in
quadrant I (a \textgreater{} 0, b \textgreater{} 0). The angle 36.87° is
correct.

For Z₂: arctan((-3)/(-4)) = arctan(0.75) = 36.87°. Same result, but Z₂
is in quadrant III (a \textless{} 0, b \textless{} 0). The correct angle
is 36.87° - 180° = -143.13°.

arctan returns 36.87° for both, a 180° error for Z₂.

\textbf{Correct angles: Z₁ = 5∠36.87° Ω, Z₂ = 5∠-143.13° Ω. arctan
cannot distinguish them.}

\begin{center}\rule{0.5\linewidth}{0.5pt}\end{center}

\section{Problem B.1.3}\label{problem-b.1.3}

\textbf{Given:} An impedance Z = -8 + j6 Ω (quadrant II).

\textbf{Find:} Compute arctan(b/a), identify the error, and determine
the correct angle manually.

\textbf{Solution:} arctan(6/(-8)) = arctan(-0.75) = -36.87°. This places
Z in quadrant IV, which is incorrect.

Manual correction: since a \textless{} 0 (real part is negative), add
180°: θ = -36.87° + 180° = 143.13° (quadrant II).

Magnitude: \textbar Z\textbar{} = √(64 + 36) = √100 = 10 Ω.

\textbf{Z = 10∠143.13° Ω. arctan gives -36.87° (wrong by 180°). The
correct angle is 143.13°.}

\begin{center}\rule{0.5\linewidth}{0.5pt}\end{center}

\section{Problem B.1.4}\label{problem-b.1.4}

\textbf{Given:} A purely imaginary impedance Z = j50 Ω (a pure
inductor).

\textbf{Find:} Attempt to compute the angle using arctan(b/a) and
identify the problem.

\textbf{Solution:} Z = 0 + j50, so a = 0 and b = 50. arctan(b/a) =
arctan(50/0) = arctan(infinity) = 90°.

In this case, arctan happens to give the correct answer (90°) because
the limit of arctan as the argument approaches +infinity is +90°.
However, for Z = -j50 (a pure capacitor), arctan(-50/0) = -90°, which is
also correct.

The real problem occurs at Z = -R + j0 (purely negative real):
arctan(0/(-R)) = arctan(0) = 0° instead of the correct 180°.

\textbf{arctan gives 90° for j50 (correct), but this is a coincidence.
arctan fails at axis points like Z = -R where it returns 0° instead of
180°.}

\chapter{Appendix B -- Section B.2: The Two-Argument
Arctangent}\label{appendix-b-section-b.2-the-two-argument-arctangent}

Practice problems covering atan2 definition, quadrant resolution, and
special cases on the axes.

\begin{center}\rule{0.5\linewidth}{0.5pt}\end{center}

\section{Problem B.2.1}\label{problem-b.2.1}

\textbf{Given:} Four complex numbers: Z₁ = 5 + j5, Z₂ = -5 + j5, Z₃ = -5
- j5, Z₄ = 5 - j5.

\textbf{Find:} The angle of each using atan2(b, a) and verify the
quadrant.

\textbf{Solution:} atan2(5, 5) = 45° (quadrant I: a \textgreater{} 0, b
\textgreater{} 0). Correct. atan2(5, -5) = 135° (quadrant II: a
\textless{} 0, b \textgreater{} 0). Correct. atan2(-5, -5) = -135°
(quadrant III: a \textless{} 0, b \textless{} 0). Correct. atan2(-5, 5)
= -45° (quadrant IV: a \textgreater{} 0, b \textless{} 0). Correct.

All four magnitudes are \textbar Z\textbar{} = √(25 + 25) = √50 = 7.071.

\textbf{Z₁ = 7.071∠45°, Z₂ = 7.071∠135°, Z₃ = 7.071∠-135°, Z₄ =
7.071∠-45°. atan2 correctly resolves all four quadrants.}

\begin{center}\rule{0.5\linewidth}{0.5pt}\end{center}

\section{Problem B.2.2}\label{problem-b.2.2}

\textbf{Given:} An impedance Z = -12 + j16 Ω.

\textbf{Find:} The polar form using atan2, and compare with the arctan
result.

\textbf{Solution:} Magnitude: \textbar Z\textbar{} = √((-12)² + 16²) =
√(144 + 256) = √400 = 20 Ω.

Using atan2: θ = atan2(16, -12). Since a \textless{} 0 and b
\textgreater{} 0, this is quadrant II. θ = 180° - arctan(16/12) = 180° -
arctan(1.333) = 180° - 53.13° = 126.87°.

Using arctan: arctan(16/(-12)) = arctan(-1.333) = -53.13° (quadrant IV
-- wrong).

\textbf{Z = 20∠126.87° Ω using atan2. arctan gives -53.13°, an error of
180°.}

\begin{center}\rule{0.5\linewidth}{0.5pt}\end{center}

\section{Problem B.2.3}\label{problem-b.2.3}

\textbf{Given:} The following axis points: Z₁ = 75 + j0 Ω, Z₂ = -75 + j0
Ω, Z₃ = 0 + j75 Ω, Z₄ = 0 - j75 Ω.

\textbf{Find:} The angle of each using atan2.

\textbf{Solution:} Z₁ = 75 + j0: atan2(0, 75) = 0°. This is a pure
resistance (positive real axis). Z₂ = -75 + j0: atan2(0, -75) = 180°.
This is a negative resistance (negative real axis). Z₃ = 0 + j75:
atan2(75, 0) = 90°. This is a pure inductance (+90° impedance angle). Z₄
= 0 - j75: atan2(-75, 0) = -90°. This is a pure capacitance (-90°
impedance angle).

All four have magnitude 75 Ω.

\textbf{Z₁ = 75∠0° Ω, Z₂ = 75∠180° Ω, Z₃ = 75∠90° Ω, Z₄ = 75∠-90° Ω.
atan2 handles all axis points correctly.}

\begin{center}\rule{0.5\linewidth}{0.5pt}\end{center}

\section{Problem B.2.4}\label{problem-b.2.4}

\textbf{Given:} A circuit has R = 0 Ω and X\textsubscript{L} = 2π x 400
x 0.025 = 62.83 Ω at 400 Hz (a pure 25 mH inductor).

\textbf{Find:} The impedance angle using atan2 and explain why
arctan(b/a) fails here.

\textbf{Solution:} Z = 0 + j62.83 Ω, so a = 0 and b = 62.83.

Using atan2: θ = atan2(62.83, 0) = 90°. Correct.

Using arctan: arctan(62.83/0) -- this requires dividing by zero. While
the limit arctan(+infinity) = 90°, the division b/a is undefined, and
software may produce an error or NaN.

atan2 avoids the division entirely by examining the signs of a and b
independently: since a = 0 and b \textgreater{} 0, the angle is defined
as exactly +90°.

\textbf{θ = 90°. atan2 handles the a = 0 case directly, while arctan
requires division by zero.}

\begin{center}\rule{0.5\linewidth}{0.5pt}\end{center}

\section{Problem B.2.5}\label{problem-b.2.5}

\textbf{Given:} A signal with in-phase component I = -3.5 V and
quadrature component Q = -6.062 V (from an I/Q demodulator).

\textbf{Find:} The magnitude and phase angle using atan2.

\textbf{Solution:} Magnitude: \textbar V\textbar{} = √((-3.5)² +
(-6.062)²) = √(12.25 + 36.75) = √49 = 7.0 V.

Phase angle: θ = atan2(-6.062, -3.5). Since a \textless{} 0 and b
\textless{} 0, this is quadrant III. θ = -(180° - arctan(6.062/3.5)) =
-(180° - arctan(1.732)) = -(180° - 60°) = -120°.

Using arctan instead: arctan((-6.062)/(-3.5)) = arctan(1.732) = 60°
(quadrant I -- wrong by 180°).

\textbf{V = 7.0∠-120°. atan2 correctly places the phasor in quadrant
III.}

\chapter{Appendix B -- Section B.3: Applications in Electrical
Engineering}\label{appendix-b-section-b.3-applications-in-electrical-engineering}

Practice problems covering impedance angle calculation, phasor angle
from rectangular components, and power factor angle.

\begin{center}\rule{0.5\linewidth}{0.5pt}\end{center}

\section{Problem B.3.1}\label{problem-b.3.1}

\textbf{Given:} A negative impedance converter (NIC) produces Z = -25 +
j25 Ω.

\textbf{Find:} The polar form using atan2, and show the error that
arctan would produce.

\textbf{Solution:} Magnitude: \textbar Z\textbar{} = √((-25)² + 25²) =
√(625 + 625) = √1250 = 35.36 Ω.

Using atan2: θ = atan2(25, -25) = 180° - arctan(25/25) = 180° - 45° =
135° (quadrant II).

Using arctan: arctan(25/(-25)) = arctan(-1) = -45° (quadrant IV --
wrong).

\textbf{Z = 35.36∠135° Ω. arctan gives -45°, a 180° error.}

\begin{center}\rule{0.5\linewidth}{0.5pt}\end{center}

\section{Problem B.3.2}\label{problem-b.3.2}

\textbf{Given:} A lock-in amplifier measures in-phase I = 4.0 mV and
quadrature Q = -6.928 mV from a sensor signal.

\textbf{Find:} The signal magnitude and phase angle using atan2.

\textbf{Solution:} Magnitude: \textbar V\textbar{} = √(4.0² + (-6.928)²)
= √(16 + 48) = √64 = 8.0 mV.

Phase angle: θ = atan2(-6.928, 4.0). Since a \textgreater{} 0 and b
\textless{} 0, this is quadrant IV. θ = arctan(-6.928/4.0) =
arctan(-1.732) = -60°.

In this case, a \textgreater{} 0, so arctan gives the same result as
atan2: -60°. Both are correct for quadrant IV.

\textbf{V = 8.0∠-60° mV.}

\begin{center}\rule{0.5\linewidth}{0.5pt}\end{center}

\section{Problem B.3.3}\label{problem-b.3.3}

\textbf{Given:} A motor operating in regenerative braking mode has
complex power S = -1200 + j900 VA.

\textbf{Find:} The power factor angle using both atan2 and arctan, the
apparent power, and the power factor.

\textbf{Solution:} Apparent power: \textbar S\textbar{} = √((-1200)² +
900²) = √(1,440,000 + 810,000) = √2,250,000 = 1500 VA.

Using atan2: φ = atan2(900, -1200). Since P \textless{} 0 and Q
\textgreater{} 0 (quadrant II): φ = 180° - arctan(900/1200) = 180° -
36.87° = 143.13°.

Using arctan: φ = arctan(900/(-1200)) = arctan(-0.75) = -36.87°
(quadrant IV -- wrong).

Power factor: pf = cos(143.13°) = -0.8. The negative power factor
indicates reverse power flow (regeneration).

\textbf{φ = 143.13° (atan2). Apparent power = 1500 VA. Power factor =
-0.8 (regenerative).}

\begin{center}\rule{0.5\linewidth}{0.5pt}\end{center}

\section{Problem B.3.4}\label{problem-b.3.4}

\textbf{Given:} A three-phase system measurement at a bus yields P =
2500 kW and Q = -1500 kVAR (capacitive reactive power).

\textbf{Find:} The power factor angle using atan2, the power factor, and
whether the load is leading or lagging.

\textbf{Solution:} Using atan2: φ = atan2(-1500, 2500). Since P
\textgreater{} 0 and Q \textless{} 0 (quadrant IV): φ =
arctan(-1500/2500) = arctan(-0.6) = -30.96°. (Since P \textgreater{} 0,
atan2 and arctan agree.)

Apparent power: \textbar S\textbar{} = √(2500² + 1500²) = √(6,250,000 +
2,250,000) = √8,500,000 = 2915.5 kVA. Power factor: pf = cos(-30.96°) =
0.857. Since Q \textless{} 0, the load is capacitive (leading).

\textbf{φ = -30.96°. Power factor = 0.857 leading.}

\begin{center}\rule{0.5\linewidth}{0.5pt}\end{center}

\section{Problem B.3.5}\label{problem-b.3.5}

\textbf{Given:} An I/Q demodulator outputs I = -0.707 mV and Q = 0.707
mV for a QAM constellation point.

\textbf{Find:} The magnitude and phase using atan2. Compare with arctan.

\textbf{Solution:} Magnitude: \textbar V\textbar{} = √((-0.707)² +
0.707²) = √(0.5 + 0.5) = √1.0 = 1.0 mV.

Using atan2: θ = atan2(0.707, -0.707). Since a \textless{} 0 and b
\textgreater{} 0 (quadrant II): θ = 180° - arctan(0.707/0.707) = 180° -
45° = 135°.

Using arctan: arctan(0.707/(-0.707)) = arctan(-1) = -45° (quadrant IV --
wrong by 180°).

In a QPSK constellation, 135° corresponds to the symbol in the second
quadrant, carrying bit pattern ``01'' (Gray-coded). The arctan error
would map it to the fourth quadrant symbol, causing a bit error.

\textbf{V = 1.0∠135° mV. atan2 is essential for correct symbol detection
in digital communications.}

\chapter{Appendix C -- Section C.1: Definition and
Fundamentals}\label{appendix-c-section-c.1-definition-and-fundamentals}

Practice problems covering power ratios in dB, voltage/current ratios in
dB, and common decibel values.

\begin{center}\rule{0.5\linewidth}{0.5pt}\end{center}

\section{Problem C.1.1}\label{problem-c.1.1}

\textbf{Given:} A power amplifier receives 2 mW of input and delivers 5
W of output.

\textbf{Find:} The power gain in decibels.

\textbf{Solution:} G = 10 x
log₁₀(P\textsubscript{out}/P\textsubscript{in}) = 10 x log₁₀(5000/2) =
10 x log₁₀(2500) = 10 x 3.3979 = 33.98 dB.

\textbf{The power gain is 33.98 dB.}

\begin{center}\rule{0.5\linewidth}{0.5pt}\end{center}

\section{Problem C.1.2}\label{problem-c.1.2}

\textbf{Given:} A cable introduces 2.5 dB of power loss per 100 m. The
cable is 350 m long.

\textbf{Find:} The total loss in dB and the fraction of input power
delivered to the output.

\textbf{Solution:} Total loss: L = 2.5 x (350/100) = 2.5 x 3.5 = 8.75
dB.

Fraction of power delivered: P\textsubscript{out}/P\textsubscript{in} =
10\textsuperscript{-8.75/10} = 10\textsuperscript{-0.875} = 0.1334.

\textbf{The total loss is 8.75 dB. Only 13.34\% of the input power
reaches the output.}

\begin{center}\rule{0.5\linewidth}{0.5pt}\end{center}

\section{Problem C.1.3}\label{problem-c.1.3}

\textbf{Given:} A voltage amplifier has V\textsubscript{in} = 50
mV\textsubscript{rms} and V\textsubscript{out} = 3.5
V\textsubscript{rms}.

\textbf{Find:} The voltage gain in dB.

\textbf{Solution:} A\textsubscript{v} = 20 x
log₁₀(V\textsubscript{out}/V\textsubscript{in}) = 20 x log₁₀(3.5/0.05) =
20 x log₁₀(70) = 20 x 1.8451 = 36.90 dB.

\textbf{The voltage gain is 36.90 dB.}

\begin{center}\rule{0.5\linewidth}{0.5pt}\end{center}

\section{Problem C.1.4}\label{problem-c.1.4}

\textbf{Given:} A passive filter reduces a signal from 1.2
V\textsubscript{rms} to 0.6 V\textsubscript{rms} (voltage halved).

\textbf{Find:} The attenuation in dB, and verify using the power ratio
(assuming equal impedances).

\textbf{Solution:} Voltage gain: A\textsubscript{v} = 20 x
log₁₀(0.6/1.2) = 20 x log₁₀(0.5) = 20 x (-0.3010) = -6.02 dB.

Power ratio (equal impedances): P\textsubscript{out}/P\textsubscript{in}
= (V\textsubscript{out}/V\textsubscript{in})² = 0.5² = 0.25. Power gain:
G = 10 x log₁₀(0.25) = 10 x (-0.6021) = -6.02 dB. Matches.

\textbf{The attenuation is 6.02 dB (halving voltage = halving power x 2
= -6 dB).}

\begin{center}\rule{0.5\linewidth}{0.5pt}\end{center}

\section{Problem C.1.5}\label{problem-c.1.5}

\textbf{Given:} An amplifier has a power gain of 37 dB.

\textbf{Find:} Estimate the power ratio using rules of thumb, then
compute the exact ratio.

\textbf{Solution:} Using rules of thumb: 37 dB = 30 dB + 7 dB = 30 dB +
10 dB - 3 dB. 30 dB = power ratio of 1,000. 10 dB = power ratio of 10.
-3 dB = power ratio of 0.5. Estimate: 1,000 x 10 x 0.5 = 5,000.

Exact: 10\textsuperscript{37/10} = 10\textsuperscript{3.7} = 5,012.

\textbf{The estimated power ratio is 5,000. The exact ratio is 5,012.
The estimate is within 0.24\%.}

\begin{center}\rule{0.5\linewidth}{0.5pt}\end{center}

\section{Problem C.1.6}\label{problem-c.1.6}

\textbf{Given:} Two stages in series: stage 1 has a voltage gain of 14
dB, and stage 2 has a voltage gain of -8 dB.

\textbf{Find:} The total voltage gain in dB and the overall linear
voltage ratio.

\textbf{Solution:} Total gain: A\textsubscript{total} = 14 + (-8) = 6
dB.

Linear voltage ratio: 10\textsuperscript{6/20} = 10\textsuperscript{0.3}
= 1.995 (approximately 2:1).

This confirms the rule of thumb: 6 dB corresponds to a voltage ratio of
2.

\textbf{The total voltage gain is 6 dB, corresponding to a voltage ratio
of approximately 2:1.}

\begin{center}\rule{0.5\linewidth}{0.5pt}\end{center}

\section{Problem C.1.7}\label{problem-c.1.7}

\textbf{Given:} A current amplifier has a current gain of 26 dB.

\textbf{Find:} The linear current ratio.

\textbf{Solution:} Current uses the voltage/field formula (factor of
20): I\textsubscript{out}/I\textsubscript{in} =
10\textsuperscript{26/20} = 10\textsuperscript{1.3} = 19.95.

Rule of thumb check: 26 dB = 20 dB + 6 dB. 20 dB current ratio = 10, 6
dB current ratio = 2. Estimate: 10 x 2 = 20. Close to the exact value.

\textbf{The current ratio is 19.95:1 (approximately 20:1).}

\chapter{Appendix C -- Section C.2: Absolute Reference
Levels}\label{appendix-c-section-c.2-absolute-reference-levels}

Practice problems covering dBm, dBW, dBV, and dBuV conversions.

\begin{center}\rule{0.5\linewidth}{0.5pt}\end{center}

\section{Problem C.2.1}\label{problem-c.2.1}

\textbf{Given:} A Wi-Fi access point transmits at +17 dBm into a 50 Ω
antenna connector.

\textbf{Find:} The output power in milliwatts, watts, and the RMS
voltage at the connector.

\textbf{Solution:} Power: P = 1 mW x 10\textsuperscript{17/10} = 1 mW x
10\textsuperscript{1.7} = 1 mW x 50.12 = 50.12 mW = 0.05012 W.

RMS voltage: V\textsubscript{rms} = √(P x R) = √(0.05012 x 50) = √2.506
= 1.583 V.

\textbf{P = 50.12 mW = 0.0501 W. V\textsubscript{rms} = 1.583 V.}

\begin{center}\rule{0.5\linewidth}{0.5pt}\end{center}

\section{Problem C.2.2}\label{problem-c.2.2}

\textbf{Given:} A signal generator outputs -10 dBm.

\textbf{Find:} The power in milliwatts, microwatts, and dBW.

\textbf{Solution:} Power: P = 1 mW x 10\textsuperscript{-10/10} = 1 mW x
10⁻¹ = 0.1 mW = 100 μW.

dBW: P(dBW) = P(dBm) - 30 = -10 - 30 = -40 dBW.

Verification: P = 1 W x 10\textsuperscript{-40/10} = 10⁻⁴ W = 0.1 mW.
Confirmed.

\textbf{P = 0.1 mW = 100 μW = -10 dBm = -40 dBW.}

\begin{center}\rule{0.5\linewidth}{0.5pt}\end{center}

\section{Problem C.2.3}\label{problem-c.2.3}

\textbf{Given:} A broadcast FM transmitter has a power output of 25 kW.

\textbf{Find:} Express in dBW and dBm.

\textbf{Solution:} dBW: P = 10 x log₁₀(25,000/1) = 10 x log₁₀(25,000) =
10 x 4.3979 = 43.98 dBW.

dBm: P = 43.98 + 30 = 73.98 dBm.

Verification: 10\textsuperscript{73.98/10} mW =
10\textsuperscript{7.398} mW = 2.5 x 10⁷ mW = 25 kW. Confirmed.

\textbf{P = 43.98 dBW = 73.98 dBm.}

\begin{center}\rule{0.5\linewidth}{0.5pt}\end{center}

\section{Problem C.2.4}\label{problem-c.2.4}

\textbf{Given:} An EMC test measures 54 dBμV/m at 200 MHz.

\textbf{Find:} The field strength in dBV/m, in V/m, and in mV/m.

\textbf{Solution:} dBV: V(dBV) = V(dBμV) - 120 = 54 - 120 = -66 dBV/m.

Linear voltage: E = 1 μV/m x 10\textsuperscript{54/20} = 10⁻⁶ x
10\textsuperscript{2.7} = 10⁻⁶ x 501.2 = 501.2 μV/m = 0.5012 mV/m.

Verification: 20 x log₁₀(501.2 x 10⁻⁶ / 10⁻⁶) = 20 x log₁₀(501.2) = 20 x
2.7 = 54 dBμV/m. Confirmed.

\textbf{E = 54 dBμV/m = -66 dBV/m = 501.2 μV/m = 0.501 mV/m.}

\begin{center}\rule{0.5\linewidth}{0.5pt}\end{center}

\section{Problem C.2.5}\label{problem-c.2.5}

\textbf{Given:} An audio interface specifies a maximum output level of
+4 dBV (professional line level).

\textbf{Find:} The output voltage in V\textsubscript{rms}, and convert
to dBm assuming a 600 Ω load (standard audio impedance).

\textbf{Solution:} Voltage: V = 1 V x 10\textsuperscript{4/20} =
10\textsuperscript{0.2} = 1.585 V\textsubscript{rms}.

Power into 600 Ω: P = V²/R = (1.585)²/600 = 2.512/600 = 4.187 mW. dBm: P
= 10 x log₁₀(4.187/1) = 10 x 0.6218 = 6.22 dBm.

\textbf{V = 1.585 V\textsubscript{rms}. Into 600 Ω, this is +6.22 dBm.}

\begin{center}\rule{0.5\linewidth}{0.5pt}\end{center}

\section{Problem C.2.6}\label{problem-c.2.6}

\textbf{Given:} A cable TV system specifies signal levels in dBmV. The
minimum acceptable level at a TV set is 0 dBmV and the maximum is +15
dBmV.

\textbf{Find:} The voltage range in mV and μV, and convert the maximum
to dBμV.

\textbf{Solution:} 0 dBmV corresponds to 1 mV\textsubscript{rms} = 1000
μV\textsubscript{rms}.

+15 dBmV: V = 1 mV x 10\textsuperscript{15/20} = 1 mV x
10\textsuperscript{0.75} = 1 mV x 5.623 = 5.623 mV\textsubscript{rms} =
5623 μV\textsubscript{rms}.

Convert to dBμV: dBμV = dBmV + 60 (since 1 mV = 1000 μV = 60 dBμV).
Maximum: 15 + 60 = 75 dBμV. Minimum: 0 + 60 = 60 dBμV.

\textbf{Signal range: 1.0 mV to 5.623 mV (60 dBμV to 75 dBμV).}

\chapter{Appendix C -- Section C.3: Decibel
Arithmetic}\label{appendix-c-section-c.3-decibel-arithmetic}

Practice problems covering cascaded gains/losses, linear-to-dB
conversion, and adding powers in decibels.

\begin{center}\rule{0.5\linewidth}{0.5pt}\end{center}

\section{Problem C.3.1}\label{problem-c.3.1}

\textbf{Given:} A satellite downlink chain consists of: satellite
transmitter at +10 dBW, transmit antenna gain +35 dBi, free-space path
loss -200 dB, receive antenna gain +45 dBi, cable loss -2 dB, and LNA
gain +25 dB.

\textbf{Find:} The signal level at the LNA output in dBW and dBm.

\textbf{Solution:} P\textsubscript{out} = P\textsubscript{tx} +
G\textsubscript{tx} + L\textsubscript{path} + G\textsubscript{rx} +
L\textsubscript{cable} + G\textsubscript{LNA} P\textsubscript{out} = 10
+ 35 + (-200) + 45 + (-2) + 25 = -87 dBW.

Convert to dBm: -87 + 30 = -57 dBm.

Linear power: P = 10\textsuperscript{-87/10} W =
10\textsuperscript{-8.7} W = 2.0 nW.

\textbf{P\textsubscript{out} = -87 dBW = -57 dBm = 2.0 nW.}

\begin{center}\rule{0.5\linewidth}{0.5pt}\end{center}

\section{Problem C.3.2}\label{problem-c.3.2}

\textbf{Given:} An RF signal chain has three stages: a preamplifier with
+15 dB gain, a bandpass filter with -4 dB insertion loss, and a power
amplifier with +30 dB gain. The input power is -20 dBm.

\textbf{Find:} The output power in dBm and watts.

\textbf{Solution:} Total gain: G\textsubscript{total} = 15 + (-4) + 30 =
41 dB. Output power: P\textsubscript{out} = -20 + 41 = +21 dBm.

Linear: P = 10\textsuperscript{21/10} mW = 10\textsuperscript{2.1} mW =
125.9 mW = 0.126 W.

\textbf{P\textsubscript{out} = +21 dBm = 125.9 mW.}

\begin{center}\rule{0.5\linewidth}{0.5pt}\end{center}

\section{Problem C.3.3}\label{problem-c.3.3}

\textbf{Given:} A noise figure specification of 6 dB for a receiver.

\textbf{Find:} The linear noise factor F and the equivalent noise
temperature (T₀ = 290 K).

\textbf{Solution:} F = 10\textsuperscript{6/10} =
10\textsuperscript{0.6} = 3.981.

T\textsubscript{e} = T₀ x (F - 1) = 290 x (3.981 - 1) = 290 x 2.981 =
864.5 K.

\textbf{F = 3.981 (linear). T\textsubscript{e} = 864.5 K.}

\begin{center}\rule{0.5\linewidth}{0.5pt}\end{center}

\section{Problem C.3.4}\label{problem-c.3.4}

\textbf{Given:} A power gain of 250 (linear ratio).

\textbf{Find:} The gain in dB using the power formula.

\textbf{Solution:} G = 10 x log₁₀(250) = 10 x 2.3979 = 23.98 dB.

Rule of thumb check: 250 = 100 x 2.5 = 100 x 2 x 1.25. 20 dB (100) + 3
dB (2) + 1 dB (1.259) = 24 dB. Close to exact.

\textbf{G = 23.98 dB.}

\begin{center}\rule{0.5\linewidth}{0.5pt}\end{center}

\section{Problem C.3.5}\label{problem-c.3.5}

\textbf{Given:} A voltage gain of -34 dB.

\textbf{Find:} The linear voltage ratio.

\textbf{Solution:} V\textsubscript{out}/V\textsubscript{in} =
10\textsuperscript{-34/20} = 10\textsuperscript{-1.7} = 0.01995.

This means the output is about 2\% of the input voltage.

Rule of thumb: -34 dB = -40 dB + 6 dB. Voltage ratio for -40 dB = 0.01,
for 6 dB = 2. Total = 0.01 x 2 = 0.02. Close to exact.

\textbf{The linear voltage ratio is 0.01995 (approximately 1/50).}

\begin{center}\rule{0.5\linewidth}{0.5pt}\end{center}

\section{Problem C.3.6}\label{problem-c.3.6}

\textbf{Given:} Three uncorrelated noise sources at a combiner input:
-80 dBm, -80 dBm, and -80 dBm.

\textbf{Find:} The total noise power at the combiner output.

\textbf{Solution:} Convert each to linear: P =
10\textsuperscript{-80/10} mW = 10⁻⁸ mW = 10 pW each.

Total: P\textsubscript{total} = 10 + 10 + 10 = 30 pW = 3 x 10⁻⁸ mW.

Convert back: P\textsubscript{total} = 10 x log₁₀(3 x 10⁻⁸) = 10 x
(-7.523) = -75.23 dBm.

The three equal sources combine to a level 4.77 dB higher than any
single source: -80 + 10 x log₁₀(3) = -80 + 4.77 = -75.23 dBm.

\textbf{P\textsubscript{total} = -75.23 dBm. Three equal powers combine
to give +4.77 dB above each individual level.}

\begin{center}\rule{0.5\linewidth}{0.5pt}\end{center}

\section{Problem C.3.7}\label{problem-c.3.7}

\textbf{Given:} Two signals at a combiner: +10 dBm and +20 dBm.

\textbf{Find:} The total power in dBm.

\textbf{Solution:} Convert to linear: P₁ = 10\textsuperscript{10/10} mW
= 10 mW. P₂ = 10\textsuperscript{20/10} mW = 100 mW.

Total: P\textsubscript{total} = 10 + 100 = 110 mW.

Convert back: P\textsubscript{total} = 10 x log₁₀(110) = 10 x 2.0414 =
20.41 dBm.

The result is only 0.41 dB above the stronger signal because the weaker
signal (10 mW) is only 10\% of the stronger signal (100 mW). The
stronger signal dominates.

\textbf{P\textsubscript{total} = 20.41 dBm.}

\begin{center}\rule{0.5\linewidth}{0.5pt}\end{center}

\section{Problem C.3.8}\label{problem-c.3.8}

\textbf{Given:} A fiber optic link has a +5 dBm laser, two connectors at
-0.3 dB each, one splice at -0.1 dB, 40 km of fiber at 0.25 dB/km loss,
and the receiver sensitivity is -32 dBm.

\textbf{Find:} The received power and the link margin.

\textbf{Solution:} Fiber loss: 40 x 0.25 = 10 dB. Connector losses: 2 x
0.3 = 0.6 dB. Splice loss: 0.1 dB. Total loss: 10 + 0.6 + 0.1 = 10.7 dB.

Received power: P\textsubscript{rx} = +5 - 10.7 = -5.7 dBm. Link margin:
-5.7 - (-32) = 26.3 dB.

\textbf{P\textsubscript{rx} = -5.7 dBm. Link margin = 26.3 dB above
receiver sensitivity.}

\chapter{Appendix C -- Section C.4: Applications in Electrical
Engineering}\label{appendix-c-section-c.4-applications-in-electrical-engineering}

Practice problems covering amplifier gain/bandwidth, signal-to-noise
ratio, and link budgets.

\begin{center}\rule{0.5\linewidth}{0.5pt}\end{center}

\section{Problem C.4.1}\label{problem-c.4.1}

\textbf{Given:} An op-amp has a gain-bandwidth product of 5 MHz and is
configured for a closed-loop gain of 26 dB.

\textbf{Find:} The linear closed-loop gain, the closed-loop bandwidth,
and the gain at 500 kHz.

\textbf{Solution:} Linear gain: A\textsubscript{CL} =
10\textsuperscript{26/20} = 10\textsuperscript{1.3} = 19.95
(approximately 20).

Bandwidth: BW = GBW / A\textsubscript{CL} = 5 x 10⁶ / 19.95 = 250.6 kHz.

At 500 kHz, the gain rolls off. The gain at frequency f for a
single-pole op-amp is: A(f) = GBW / f = 5 x 10⁶ / 500 x 10³ = 10. In dB:
20 x log₁₀(10) = 20 dB.

Since 500 kHz is about 2x the -3 dB bandwidth, the gain has dropped from
26 dB to 20 dB.

\textbf{A\textsubscript{CL} = 20 (linear). BW = 250.6 kHz. Gain at 500
kHz = 20 dB.}

\begin{center}\rule{0.5\linewidth}{0.5pt}\end{center}

\section{Problem C.4.2}\label{problem-c.4.2}

\textbf{Given:} A 16-bit ADC has a full-scale range of 5 V and
quantization noise only (ideal ADC).

\textbf{Find:} The theoretical SNR and ENOB.

\textbf{Solution:} For an ideal N-bit ADC: SNR = 6.02N + 1.76 dB. SNR =
6.02 x 16 + 1.76 = 96.32 + 1.76 = 98.08 dB.

ENOB = (SNR - 1.76) / 6.02 = (98.08 - 1.76) / 6.02 = 96.32 / 6.02 = 16.0
bits.

For a real ADC with additional noise, the actual SNR would be lower,
reducing the ENOB below 16.

\textbf{Theoretical SNR = 98.08 dB. ENOB = 16.0 bits (ideal).}

\begin{center}\rule{0.5\linewidth}{0.5pt}\end{center}

\section{Problem C.4.3}\label{problem-c.4.3}

\textbf{Given:} A measurement system has a noise floor of -110 dBm and a
maximum input of +10 dBm before clipping.

\textbf{Find:} The dynamic range in dB.

\textbf{Solution:} Dynamic range = maximum signal - noise floor = +10 -
(-110) = 120 dB.

In terms of power ratio: 10\textsuperscript{120/10} = 10¹² =
1,000,000,000,000 (one trillion to one).

\textbf{The dynamic range is 120 dB, corresponding to a power ratio of
10¹².}

\begin{center}\rule{0.5\linewidth}{0.5pt}\end{center}

\section{Problem C.4.4}\label{problem-c.4.4}

\textbf{Given:} An audio amplifier has a signal output of 2
V\textsubscript{rms} and a noise output of 20 μV\textsubscript{rms}.

\textbf{Find:} The SNR in dB.

\textbf{Solution:} SNR = 20 x
log₁₀(V\textsubscript{signal}/V\textsubscript{noise}) = 20 x log₁₀(2/20
x 10⁻⁶) = 20 x log₁₀(100,000) = 20 x 5 = 100 dB.

\textbf{The SNR is 100 dB.}

\begin{center}\rule{0.5\linewidth}{0.5pt}\end{center}

\section{Problem C.4.5}\label{problem-c.4.5}

\textbf{Given:} A 2.4 GHz Wi-Fi link spans 100 m in free space. The
transmitter outputs +20 dBm with a +3 dBi antenna. The receiver has a +3
dBi antenna and -85 dBm sensitivity.

\textbf{Find:} The free-space path loss, received power, and link
margin.

\textbf{Solution:} Wavelength: λ = c/f = 3 x 10⁸ / 2.4 x 10⁹ = 0.125 m.

Free-space path loss: L\textsubscript{path} = 20 x log₁₀(4π x 100 /
0.125) = 20 x log₁₀(10,053) = 20 x 4.0023 = 80.05 dB.

Received power: P\textsubscript{rx} = P\textsubscript{tx} +
G\textsubscript{tx} - L\textsubscript{path} + G\textsubscript{rx} = 20 +
3 - 80.05 + 3 = -54.05 dBm.

Link margin: -54.05 - (-85) = 30.95 dB.

\textbf{L\textsubscript{path} = 80.05 dB. P\textsubscript{rx} = -54.05
dBm. Link margin = 30.95 dB.}

\begin{center}\rule{0.5\linewidth}{0.5pt}\end{center}

\section{Problem C.4.6}\label{problem-c.4.6}

\textbf{Given:} An amplifier chain has the following stages: first stage
gain = 20 dB with noise figure 3 dB, second stage gain = 10 dB with
noise figure 10 dB.

\textbf{Find:} The overall noise figure using Friis' noise formula
(expressed in dB and linear).

\textbf{Solution:} Convert to linear noise factors: F₁ =
10\textsuperscript{3/10} = 2.0. F₂ = 10\textsuperscript{10/10} = 10.0.
G₁ = 10\textsuperscript{20/10} = 100 (linear power gain).

Friis' formula: F\textsubscript{total} = F₁ + (F₂ - 1)/G₁ = 2.0 + (10 -
1)/100 = 2.0 + 0.09 = 2.09.

NF\textsubscript{total} = 10 x log₁₀(2.09) = 10 x 0.3201 = 3.20 dB.

The overall noise figure (3.20 dB) is dominated by the first stage (3
dB) because the high gain of the first stage (20 dB) suppresses the
noise contribution of the second stage.

\textbf{F\textsubscript{total} = 2.09. NF\textsubscript{total} = 3.20
dB.}

\begin{center}\rule{0.5\linewidth}{0.5pt}\end{center}

\section{Problem C.4.7}\label{problem-c.4.7}

\textbf{Given:} A microwave point-to-point link at 18 GHz spans 5 km.
Transmit power is +30 dBm, each antenna has +38 dBi gain, and total
cable/connector losses are 3 dB at each end.

\textbf{Find:} The free-space path loss and the received power.

\textbf{Solution:} Wavelength: λ = 3 x 10⁸ / 18 x 10⁹ = 0.01667 m.

Free-space path loss: L\textsubscript{path} = 20 x log₁₀(4π x 5000 /
0.01667) = 20 x log₁₀(3.770 x 10⁶) = 20 x 6.5763 = 131.53 dB.

Received power: P\textsubscript{rx} = +30 + 38 - 3 - 131.53 + 38 - 3 =
-31.53 dBm.

Linear: P = 10\textsuperscript{-31.53/10} mW =
10\textsuperscript{-3.153} mW = 7.03 x 10⁻⁴ mW = 0.703 μW.

\textbf{L\textsubscript{path} = 131.53 dB. P\textsubscript{rx} = -31.53
dBm = 0.703 μW.}

\chapter{Appendix D -- Section D.1: SI Base
Units}\label{appendix-d-section-d.1-si-base-units}

Practice problems covering the seven base units and their relationship
to electrical quantities.

\begin{center}\rule{0.5\linewidth}{0.5pt}\end{center}

\section{Problem D.1.1}\label{problem-d.1.1}

\textbf{Given:} A copper wire has a length of 500 m, a cross-sectional
area of 4 mm², and copper resistivity ρ = 1.68 x 10⁻⁸ Ω·m.

\textbf{Find:} The wire resistance, and express the cross-sectional area
in base SI units (m²).

\textbf{Solution:} Area: 4 mm² = 4 x (10⁻³)² m² = 4 x 10⁻⁶ m².

R = ρL/A = (1.68 x 10⁻⁸ x 500) / (4 x 10⁻⁶) = 8.4 x 10⁻⁶ / 4 x 10⁻⁶ =
2.1 Ω.

\textbf{Area = 4 x 10⁻⁶ m². R = 2.1 Ω.}

\begin{center}\rule{0.5\linewidth}{0.5pt}\end{center}

\section{Problem D.1.2}\label{problem-d.1.2}

\textbf{Given:} The equation for energy stored in an inductor: E =
(1/2)LI².

\textbf{Find:} Verify that the units are consistent by expressing both
sides in SI base units.

\textbf{Solution:} Left side: Energy E is in joules. J = kg·m²·s⁻².

Right side: L is in henries (H = kg·m²·s⁻²·A⁻²). I² is in A². LI² =
(kg·m²·s⁻²·A⁻²) x A² = kg·m²·s⁻² = J.

The factor of 1/2 is dimensionless.

\textbf{Both sides have units of kg·m²·s⁻² = J. The equation is
dimensionally consistent.}

\begin{center}\rule{0.5\linewidth}{0.5pt}\end{center}

\section{Problem D.1.3}\label{problem-d.1.3}

\textbf{Given:} A 20 A circuit breaker on a 240 V supply.

\textbf{Find:} The maximum power in watts, and express the watt in base
SI units to verify P = V x I.

\textbf{Solution:} P = V x I = 240 x 20 = 4800 W.

Dimensional check: V has units kg·m²·s⁻³·A⁻¹. A has units A. V x A =
kg·m²·s⁻³·A⁻¹ x A = kg·m²·s⁻³ = W. Consistent.

\textbf{P = 4800 W. The watt equals kg·m²·s⁻³, confirmed by V x A.}

\begin{center}\rule{0.5\linewidth}{0.5pt}\end{center}

\section{Problem D.1.4}\label{problem-d.1.4}

\textbf{Given:} A charge of 1 coulomb flows through a circuit.

\textbf{Find:} How many electrons make up 1 C, given the elementary
charge e = 1.602 x 10⁻¹⁹ C.

\textbf{Solution:} Number of electrons: n = Q/e = 1 / (1.602 x 10⁻¹⁹) =
6.242 x 10¹⁸ electrons.

This enormous number illustrates why the coulomb (ampere-second) is a
practical unit even though individual electron charges are unimaginably
small.

\textbf{1 C = 6.242 x 10¹⁸ electrons.}

\chapter{Appendix D -- Section D.2: SI Derived Units for Electrical
Engineering}\label{appendix-d-section-d.2-si-derived-units-for-electrical-engineering}

Practice problems covering voltage/resistance/power units,
capacitance/inductance/charge units, and frequency/time units.

\begin{center}\rule{0.5\linewidth}{0.5pt}\end{center}

\section{Problem D.2.1}\label{problem-d.2.1}

\textbf{Given:} A 24 V battery powers a circuit drawing 6 A.

\textbf{Find:} The power dissipated, the load resistance, and the
conductance, with units verified at each step.

\textbf{Solution:} Power: P = V x I = 24 V x 6 A = 144 W. Resistance: R
= V/I = 24 V / 6 A = 4 Ω. Conductance: G = 1/R = 1/4 Ω = 0.25 S.

Check via conductance: I = G x V = 0.25 S x 24 V = 6 A. Confirmed. Check
via power: P = V²/R = 576/4 = 144 W = I²R = 36 x 4 = 144 W. Confirmed.

\textbf{P = 144 W. R = 4 Ω. G = 0.25 S.}

\begin{center}\rule{0.5\linewidth}{0.5pt}\end{center}

\section{Problem D.2.2}\label{problem-d.2.2}

\textbf{Given:} A 100 μF capacitor is charged to 50 V.

\textbf{Find:} The stored charge and energy, with appropriate SI
prefixes.

\textbf{Solution:} Charge: Q = CV = 100 x 10⁻⁶ F x 50 V = 5 x 10⁻³ C = 5
mC.

Energy: E = (1/2)CV² = 0.5 x 100 x 10⁻⁶ x 50² = 0.5 x 100 x 10⁻⁶ x 2500
= 125 x 10⁻³ J = 125 mJ.

\textbf{Q = 5 mC. E = 125 mJ.}

\begin{center}\rule{0.5\linewidth}{0.5pt}\end{center}

\section{Problem D.2.3}\label{problem-d.2.3}

\textbf{Given:} An inductor of 2.2 mH carries a current of 3 A.

\textbf{Find:} The stored energy and the voltage across the inductor if
the current changes at 1000 A/s.

\textbf{Solution:} Energy: E = (1/2)LI² = 0.5 x 2.2 x 10⁻³ x 3² = 0.5 x
2.2 x 10⁻³ x 9 = 9.9 x 10⁻³ J = 9.9 mJ.

Voltage: V = L x (dI/dt) = 2.2 x 10⁻³ H x 1000 A/s = 2.2 V.

Units check: H x A/s = (V·s/A) x (A/s) = V. Consistent.

\textbf{E = 9.9 mJ. V = 2.2 V across the inductor.}

\begin{center}\rule{0.5\linewidth}{0.5pt}\end{center}

\section{Problem D.2.4}\label{problem-d.2.4}

\textbf{Given:} A microcontroller clock frequency of 72 MHz.

\textbf{Find:} The clock period in nanoseconds and the angular frequency
in rad/s.

\textbf{Solution:} Period: T = 1/f = 1/(72 x 10⁶) = 13.89 x 10⁻⁹ s =
13.89 ns.

Angular frequency: ω = 2πf = 2π x 72 x 10⁶ = 452.4 x 10⁶ rad/s = 452.4
Mrad/s.

\textbf{T = 13.89 ns. ω = 452.4 Mrad/s.}

\begin{center}\rule{0.5\linewidth}{0.5pt}\end{center}

\section{Problem D.2.5}\label{problem-d.2.5}

\textbf{Given:} A magnetic core has a flux of 0.5 mWb through a
cross-sectional area of 2 cm².

\textbf{Find:} The magnetic flux density in tesla.

\textbf{Solution:} Convert area: 2 cm² = 2 x (10⁻²)² m² = 2 x 10⁻⁴ m².
Flux: Φ = 0.5 x 10⁻³ Wb.

B = Φ/A = (0.5 x 10⁻³) / (2 x 10⁻⁴) = 2.5 T.

Units check: Wb/m² = T. Confirmed.

\textbf{B = 2.5 T.}

\chapter{Appendix D -- Section D.3: SI
Prefixes}\label{appendix-d-section-d.3-si-prefixes}

Practice problems covering prefix usage, prefix arithmetic, and
engineering notation.

\begin{center}\rule{0.5\linewidth}{0.5pt}\end{center}

\section{Problem D.3.1}\label{problem-d.3.1}

\textbf{Given:} The following raw values: 0.000022 F, 4,700,000 Ω,
0.00000000015 H, and 56,000 Hz.

\textbf{Find:} Express each using appropriate SI prefixes.

\textbf{Solution:} 0.000022 F = 22 x 10⁻⁶ F = 22 μF. 4,700,000 Ω = 4.7 x
10⁶ Ω = 4.7 MΩ. 0.00000000015 H = 150 x 10⁻¹² H = 150 pH. 56,000 Hz = 56
x 10³ Hz = 56 kHz.

\textbf{22 μF, 4.7 MΩ, 150 pH, 56 kHz.}

\begin{center}\rule{0.5\linewidth}{0.5pt}\end{center}

\section{Problem D.3.2}\label{problem-d.3.2}

\textbf{Given:} A 10 kΩ resistor with 5 V across it.

\textbf{Find:} The current (with prefix), the power (with prefix), and
the time constant if paired with a 47 nF capacitor.

\textbf{Solution:} Current: I = V/R = 5 / (10 x 10³) = 5 x 10⁻⁴ A = 500
μA.

Power: P = V²/R = 25 / (10 x 10³) = 2.5 x 10⁻³ W = 2.5 mW.

Time constant: τ = RC = 10 kΩ x 47 nF. Using the shortcut: kΩ x nF = 10³
x 10⁻⁹ = 10⁻⁶ = μs. τ = 10 x 47 μs = 470 μs.

\textbf{I = 500 μA. P = 2.5 mW. τ = 470 μs.}

\begin{center}\rule{0.5\linewidth}{0.5pt}\end{center}

\section{Problem D.3.3}\label{problem-d.3.3}

\textbf{Given:} The values 3.75 x 10⁻⁴ A, 8.2 x 10⁷ Ω, and 6.5 x 10⁻¹⁰
F.

\textbf{Find:} Convert each to engineering notation with SI prefixes.

\textbf{Solution:} 3.75 x 10⁻⁴ A: shift exponent to -3: 0.375 x 10⁻³ A =
375 x 10⁻⁶ A = 375 μA. (Preferred: 375 μA keeps the mantissa between 1
and 999.)

8.2 x 10⁷ Ω: shift exponent to 6: 82 x 10⁶ Ω = 82 MΩ.

6.5 x 10⁻¹⁰ F: shift exponent to -9: 0.65 x 10⁻⁹ F = 650 x 10⁻¹² F = 650
pF.

\textbf{375 μA, 82 MΩ, 650 pF.}

\begin{center}\rule{0.5\linewidth}{0.5pt}\end{center}

\section{Problem D.3.4}\label{problem-d.3.4}

\textbf{Given:} A circuit operates at 2.4 GHz. A transmission line has a
propagation velocity of 2 x 10⁸ m/s.

\textbf{Find:} The wavelength in appropriate engineering notation and SI
prefix.

\textbf{Solution:} λ = v/f = (2 x 10⁸) / (2.4 x 10⁹) = 0.08333 m.

Convert to engineering notation: 0.08333 m = 83.33 x 10⁻³ m = 83.33 mm.

Alternatively: a quarter wavelength (common for antenna design) =
83.33/4 = 20.83 mm.

\textbf{λ = 83.33 mm.}

\begin{center}\rule{0.5\linewidth}{0.5pt}\end{center}

\section{Problem D.3.5}\label{problem-d.3.5}

\textbf{Given:} Prefix arithmetic shortcut: ``mA x kΩ = ?''

\textbf{Find:} Determine the result unit and verify with a numerical
example.

\textbf{Solution:} mA x kΩ = (10⁻³ A) x (10³ Ω) = 10⁰ V = V.

The milli and kilo prefixes cancel (10⁻³ x 10³ = 10⁰).

Numerical example: 2.5 mA x 4.7 kΩ = 2.5 x 4.7 V = 11.75 V.
Verification: 2.5 x 10⁻³ A x 4.7 x 10³ Ω = 11.75 V. Confirmed.

\textbf{mA x kΩ = V (the prefixes cancel).}

\chapter{Appendix D -- Section D.4: Common Unit
Conversions}\label{appendix-d-section-d.4-common-unit-conversions}

Practice problems covering energy/charge units and temperature scale
conversions.

\begin{center}\rule{0.5\linewidth}{0.5pt}\end{center}

\section{Problem D.4.1}\label{problem-d.4.1}

\textbf{Given:} A lithium-ion cell rated at 3400 mAh and 3.7 V nominal.

\textbf{Find:} The stored energy in watt-hours, joules, and the total
charge in coulombs.

\textbf{Solution:} Energy: E = 3400 mAh x 3.7 V = 12,580 mWh = 12.58 Wh.

In joules: E = 12.58 x 3600 = 45,288 J = 45.29 kJ.

Charge: Q = 3400 mAh = 3.4 Ah = 3.4 x 3600 C = 12,240 C.

\textbf{E = 12.58 Wh = 45.29 kJ. Q = 12,240 C.}

\begin{center}\rule{0.5\linewidth}{0.5pt}\end{center}

\section{Problem D.4.2}\label{problem-d.4.2}

\textbf{Given:} The bandgap energy of silicon is 1.12 eV.

\textbf{Find:} The bandgap energy in joules, and the corresponding
wavelength of a photon with this energy (c = 3 x 10⁸ m/s, h = 6.626 x
10⁻³⁴ J·s).

\textbf{Solution:} Energy: E = 1.12 eV x 1.602 x 10⁻¹⁹ J/eV = 1.794 x
10⁻¹⁹ J.

Wavelength: λ = hc/E = (6.626 x 10⁻³⁴ x 3 x 10⁸) / (1.794 x 10⁻¹⁹) =
1.988 x 10⁻²⁵ / 1.794 x 10⁻¹⁹ = 1.108 x 10⁻⁶ m = 1108 nm.

This is in the infrared range, which is why silicon photodetectors are
sensitive to near-IR light.

\textbf{E = 1.794 x 10⁻¹⁹ J. λ = 1108 nm (infrared).}

\begin{center}\rule{0.5\linewidth}{0.5pt}\end{center}

\section{Problem D.4.3}\label{problem-d.4.3}

\textbf{Given:} A power MOSFET has T\textsubscript{j,max} = 150°C,
ambient temperature is 40°C, junction-to-case thermal resistance
R\textsubscript{θJC} = 1.5°C/W, and case-to-ambient thermal resistance
R\textsubscript{θCA} = 25°C/W.

\textbf{Find:} The maximum power dissipation, and express both
temperatures in Kelvin and Fahrenheit.

\textbf{Solution:} Total thermal resistance: R\textsubscript{θJA} =
R\textsubscript{θJC} + R\textsubscript{θCA} = 1.5 + 25 = 26.5°C/W.

Maximum power: P\textsubscript{max} = (T\textsubscript{j,max} -
T\textsubscript{ambient}) / R\textsubscript{θJA} = (150 - 40) / 26.5 =
110 / 26.5 = 4.15 W.

Temperature conversions: T\textsubscript{j,max}: 150°C = 150 + 273.15 =
423.15 K = 150 x 9/5 + 32 = 302°F. T\textsubscript{ambient}: 40°C = 40 +
273.15 = 313.15 K = 40 x 9/5 + 32 = 104°F.

\textbf{P\textsubscript{max} = 4.15 W. T\textsubscript{j,max} = 150°C =
423.15 K = 302°F. T\textsubscript{ambient} = 40°C = 313.15 K = 104°F.}

\begin{center}\rule{0.5\linewidth}{0.5pt}\end{center}

\section{Problem D.4.4}\label{problem-d.4.4}

\textbf{Given:} A household uses an average of 30 kWh of electricity per
day. The electricity rate is \$0.12 per kWh.

\textbf{Find:} The average power consumption in watts, the daily energy
in joules and megajoules, and the daily cost.

\textbf{Solution:} Average power: P = E/t = 30 kWh / 24 h = 1.25 kW =
1250 W.

In joules: E = 30 x 3.6 x 10⁶ J = 108 x 10⁶ J = 108 MJ.

Daily cost: 30 x \$0.12 = \$3.60.

\textbf{P\textsubscript{avg} = 1250 W. E = 108 MJ/day. Cost =
\$3.60/day.}

\begin{center}\rule{0.5\linewidth}{0.5pt}\end{center}

\section{Problem D.4.5}\label{problem-d.4.5}

\textbf{Given:} A noise temperature specification of 75 K for a
satellite receiver LNA.

\textbf{Find:} The noise temperature in °C and °F, and the noise figure
in dB (T₀ = 290 K).

\textbf{Solution:} Temperature: T = 75 K = 75 - 273.15 = -198.15°C =
-198.15 x 9/5 + 32 = -324.67°F.

Noise factor: F = 1 + T\textsubscript{e}/T₀ = 1 + 75/290 = 1 + 0.2586 =
1.2586.

Noise figure: NF = 10 x log₁₀(1.2586) = 10 x 0.1000 = 1.00 dB.

\textbf{T = -198.15°C = -324.67°F. NF = 1.00 dB.}

\chapter{Appendix E --- Section E.1: Installing
Python}\label{appendix-e-section-e.1-installing-python}

Practice problems covering downloading and installing Python, verifying
the installation, and setting up virtual environments.

\begin{center}\rule{0.5\linewidth}{0.5pt}\end{center}

\section{Problem E.1.1}\label{problem-e.1.1}

\textbf{Given:} A user installs Python 3.12.4 on a Windows machine and
checks the installation from the Command Prompt.

\textbf{Find:} The command to verify the Python version and the expected
output.

\textbf{Solution:}

Run the following in the Command Prompt:

\begin{verbatim}
python --version
\end{verbatim}

Expected output: \textbf{Python 3.12.4}

On Windows, the command is typically \texttt{python} (not
\texttt{python3}). If the command is not found, Python was not added to
the system PATH during installation. Reinstall and check ``Add Python to
PATH.''

\begin{center}\rule{0.5\linewidth}{0.5pt}\end{center}

\section{Problem E.1.2}\label{problem-e.1.2}

\textbf{Given:} A macOS user has both the system Python 2.7 (legacy) and
a Homebrew-installed Python 3.12 on their machine.

\textbf{Find:} (a) The command to check which Python 3 is being used,
(b) the command to verify pip is associated with Python 3, and (c) how
to determine the installation path.

\textbf{Solution:}

\begin{enumerate}
\def\labelenumi{(\alph{enumi})}
\tightlist
\item
  Check Python 3 version:
\end{enumerate}

\begin{verbatim}
python3 --version
\end{verbatim}

Expected output: \textbf{Python 3.12.x}

\begin{enumerate}
\def\labelenumi{(\alph{enumi})}
\setcounter{enumi}{1}
\tightlist
\item
  Verify pip:
\end{enumerate}

\begin{verbatim}
pip3 --version
\end{verbatim}

Expected output:
\texttt{pip\ 24.x\ from\ /opt/homebrew/lib/python3.12/site-packages/pip\ (python\ 3.12)}

\begin{enumerate}
\def\labelenumi{(\alph{enumi})}
\setcounter{enumi}{2}
\tightlist
\item
  Installation path:
\end{enumerate}

\begin{verbatim}
which python3
\end{verbatim}

Expected output: \textbf{/opt/homebrew/bin/python3} (Homebrew) or
\texttt{/usr/local/bin/python3} (python.org installer).

Never use \texttt{python} without the \texttt{3} on macOS, as it may
invoke the legacy Python 2.7.

\begin{center}\rule{0.5\linewidth}{0.5pt}\end{center}

\section{Problem E.1.3}\label{problem-e.1.3}

\textbf{Given:} A user wants to create a virtual environment for the
EE-Book scripts in the \texttt{scripts/} directory. The system Python is
3.12.

\textbf{Find:} The sequence of commands to (a) navigate to the
directory, (b) create the virtual environment, (c) activate it, and (d)
verify it is active.

\textbf{Solution:}

\begin{enumerate}
\def\labelenumi{(\alph{enumi})}
\tightlist
\item
  Navigate:
\end{enumerate}

\begin{verbatim}
cd /path/to/EE-Book/scripts
\end{verbatim}

\begin{enumerate}
\def\labelenumi{(\alph{enumi})}
\setcounter{enumi}{1}
\tightlist
\item
  Create virtual environment:
\end{enumerate}

\begin{verbatim}
python3 -m venv .venv
\end{verbatim}

\begin{enumerate}
\def\labelenumi{(\alph{enumi})}
\setcounter{enumi}{2}
\tightlist
\item
  Activate (macOS/Linux):
\end{enumerate}

\begin{verbatim}
source .venv/bin/activate
\end{verbatim}

On Windows:

\begin{verbatim}
.venv\Scripts\activate
\end{verbatim}

\begin{enumerate}
\def\labelenumi{(\alph{enumi})}
\setcounter{enumi}{3}
\tightlist
\item
  Verify: The terminal prompt changes to show \texttt{(.venv)} at the
  beginning. Additionally:
\end{enumerate}

\begin{verbatim}
which python
\end{verbatim}

Should output: \textbf{/path/to/EE-Book/scripts/.venv/bin/python} (not
the system Python).

\begin{verbatim}
python --version
\end{verbatim}

Should output: \textbf{Python 3.12.x} (the version used to create the
venv).

\begin{center}\rule{0.5\linewidth}{0.5pt}\end{center}

\section{Problem E.1.4}\label{problem-e.1.4}

\textbf{Given:} A virtual environment is active and the user needs to
install the project dependencies from \texttt{requirements.txt}, which
contains:

\begin{verbatim}
marimo>=0.19.0
numpy>=1.26.0
matplotlib>=3.8.0
\end{verbatim}

\textbf{Find:} (a) The install command, (b) how to verify all packages
are installed, and (c) how to check the installed version of numpy.

\textbf{Solution:}

\begin{enumerate}
\def\labelenumi{(\alph{enumi})}
\tightlist
\item
  Install command:
\end{enumerate}

\begin{verbatim}
pip install -r requirements.txt
\end{verbatim}

\begin{enumerate}
\def\labelenumi{(\alph{enumi})}
\setcounter{enumi}{1}
\tightlist
\item
  Verify all packages:
\end{enumerate}

\begin{verbatim}
pip list
\end{verbatim}

This displays all installed packages. Confirm marimo, numpy, and
matplotlib appear in the list.

\begin{enumerate}
\def\labelenumi{(\alph{enumi})}
\setcounter{enumi}{2}
\tightlist
\item
  Check numpy version:
\end{enumerate}

\begin{verbatim}
python -c "import numpy; print(numpy.__version__)"
\end{verbatim}

Expected output: \textbf{1.26.x} (or newer). Alternatively:

\begin{verbatim}
pip show numpy
\end{verbatim}

This displays the version, location, and dependencies of the numpy
package.

\begin{center}\rule{0.5\linewidth}{0.5pt}\end{center}

\section{Problem E.1.5}\label{problem-e.1.5}

\textbf{Given:} A user has two projects that require different versions
of matplotlib: Project A needs matplotlib 3.7.x and Project B needs
matplotlib 3.9.x. Both use Python 3.12.

\textbf{Find:} How to set up separate virtual environments so both
projects can coexist without conflicts.

\textbf{Solution:}

Create separate virtual environments in each project directory:

Project A:

\begin{verbatim}
cd /path/to/project-a
python3 -m venv .venv
source .venv/bin/activate
pip install matplotlib==3.7.5
python -c "import matplotlib; print(matplotlib.__version__)"
\end{verbatim}

Output: \textbf{3.7.5}

Project B:

\begin{verbatim}
cd /path/to/project-b
python3 -m venv .venv
source .venv/bin/activate
pip install matplotlib==3.9.2
python -c "import matplotlib; print(matplotlib.__version__)"
\end{verbatim}

Output: \textbf{3.9.2}

Each virtual environment is isolated. Activating one deactivates the
other. The \texttt{.venv} directories should be added to
\texttt{.gitignore} to avoid committing them to version control.

\begin{center}\rule{0.5\linewidth}{0.5pt}\end{center}

\section{Problem E.1.6}\label{problem-e.1.6}

\textbf{Given:} A user accidentally installed packages into the system
Python instead of the virtual environment and wants to clean up.

\textbf{Find:} (a) How to check whether the virtual environment is
active, (b) how to deactivate it, and (c) how to delete and recreate the
virtual environment for a clean start.

\textbf{Solution:}

\begin{enumerate}
\def\labelenumi{(\alph{enumi})}
\tightlist
\item
  Check if venv is active:
\end{enumerate}

\begin{verbatim}
echo $VIRTUAL_ENV
\end{verbatim}

If this prints a path (e.g., \texttt{/path/to/scripts/.venv}), the
virtual environment is active. If it prints nothing, the system Python
is being used.

\begin{enumerate}
\def\labelenumi{(\alph{enumi})}
\setcounter{enumi}{1}
\tightlist
\item
  Deactivate:
\end{enumerate}

\begin{verbatim}
deactivate
\end{verbatim}

\begin{enumerate}
\def\labelenumi{(\alph{enumi})}
\setcounter{enumi}{2}
\tightlist
\item
  Delete and recreate:
\end{enumerate}

\begin{verbatim}
rm -rf .venv
python3 -m venv .venv
source .venv/bin/activate
pip install -r requirements.txt
\end{verbatim}

This creates a \textbf{completely fresh environment} with only the
packages listed in \texttt{requirements.txt}. No leftover packages from
previous installations will be present.

\begin{center}\rule{0.5\linewidth}{0.5pt}\end{center}

\section{Problem E.1.7}\label{problem-e.1.7}

\textbf{Given:} A user needs to freeze the current virtual environment's
package list to share with a collaborator who will reproduce the exact
setup.

\textbf{Find:} (a) The command to generate a frozen requirements file,
(b) what the output looks like, and (c) how the collaborator installs
from it.

\textbf{Solution:}

\begin{enumerate}
\def\labelenumi{(\alph{enumi})}
\tightlist
\item
  Freeze command:
\end{enumerate}

\begin{verbatim}
pip freeze > requirements-lock.txt
\end{verbatim}

\begin{enumerate}
\def\labelenumi{(\alph{enumi})}
\setcounter{enumi}{1}
\tightlist
\item
  Example output in \texttt{requirements-lock.txt}:
\end{enumerate}

\begin{verbatim}
contourpy==1.2.0
cycler==0.12.1
fonttools==4.47.0
kiwisolver==1.4.5
marimo==0.19.4
matplotlib==3.8.2
numpy==1.26.3
packaging==23.2
pillow==10.2.0
pyparsing==3.1.1
python-dateutil==2.8.2
six==1.16.0
\end{verbatim}

All transitive dependencies are pinned to exact versions.

\begin{enumerate}
\def\labelenumi{(\alph{enumi})}
\setcounter{enumi}{2}
\tightlist
\item
  Collaborator installs:
\end{enumerate}

\begin{verbatim}
python3 -m venv .venv
source .venv/bin/activate
pip install -r requirements-lock.txt
\end{verbatim}

This guarantees \textbf{identical package versions} across machines,
ensuring reproducible results.

\begin{center}\rule{0.5\linewidth}{0.5pt}\end{center}

\section{Problem E.1.8}\label{problem-e.1.8}

\textbf{Given:} A Linux user (Ubuntu 22.04) finds that
\texttt{python3\ -m\ venv\ .venv} fails with the error: ``The virtual
environment was not created successfully because ensurepip is not
available.''

\textbf{Find:} The cause of the error and the commands to fix it.

\textbf{Solution:}

The error occurs because Ubuntu packages \texttt{python3} and
\texttt{python3-venv} separately. The \texttt{venv} module is not
included in the base Python installation on Debian/Ubuntu systems.

Fix:

\begin{verbatim}
sudo apt update
sudo apt install python3-venv python3-pip
\end{verbatim}

After installation, retry:

\begin{verbatim}
python3 -m venv .venv
source .venv/bin/activate
pip install -r requirements.txt
\end{verbatim}

The virtual environment should now be created successfully. This is a
\textbf{Debian/Ubuntu-specific issue} --- other Linux distributions
(Fedora, Arch) and macOS/Windows include venv with the base Python
package.

\begin{center}\rule{0.5\linewidth}{0.5pt}\end{center}

\section{Problem E.1.9}\label{problem-e.1.9}

\textbf{Given:} A user wants to upgrade pip inside their virtual
environment. The current pip version is 23.0.1 and the latest available
is 24.2.

\textbf{Find:} (a) The command to upgrade pip, (b) how to verify the
upgrade, and (c) why upgrading pip matters.

\textbf{Solution:}

\begin{enumerate}
\def\labelenumi{(\alph{enumi})}
\tightlist
\item
  Upgrade command (from within the active virtual environment):
\end{enumerate}

\begin{verbatim}
pip install --upgrade pip
\end{verbatim}

\begin{enumerate}
\def\labelenumi{(\alph{enumi})}
\setcounter{enumi}{1}
\tightlist
\item
  Verify:
\end{enumerate}

\begin{verbatim}
pip --version
\end{verbatim}

Expected output: \textbf{pip 24.2 from
/path/to/.venv/lib/python3.12/site-packages/pip (python 3.12)}

\begin{enumerate}
\def\labelenumi{(\alph{enumi})}
\setcounter{enumi}{2}
\tightlist
\item
  Upgrading pip matters because:
\end{enumerate}

\begin{itemize}
\tightlist
\item
  Newer pip versions resolve dependencies more accurately (using the
  resolver introduced in pip 20.3)
\item
  Bug fixes for package installation edge cases
\item
  Support for newer packaging standards (PEP 517, PEP 660)
\item
  Better error messages when installations fail
\item
  Security patches
\end{itemize}

It is good practice to upgrade pip \textbf{immediately after creating a
new virtual environment}.

\begin{center}\rule{0.5\linewidth}{0.5pt}\end{center}

\section{Problem E.1.10}\label{problem-e.1.10}

\textbf{Given:} A user is working on a machine without internet access
and needs to install packages for the EE-Book scripts. They have a USB
drive with the \texttt{.whl} (wheel) files for marimo, numpy, and
matplotlib (and all dependencies).

\textbf{Find:} The procedure to install packages offline from local
wheel files.

\textbf{Solution:}

First, on an internet-connected machine, download all packages and
dependencies:

\begin{verbatim}
pip download -r requirements.txt -d ./packages
\end{verbatim}

Copy the \texttt{packages/} directory to the USB drive.

On the offline machine:

\begin{verbatim}
python3 -m venv .venv
source .venv/bin/activate
pip install --no-index --find-links=/path/to/usb/packages -r requirements.txt
\end{verbatim}

The \texttt{-\/-no-index} flag tells pip not to query PyPI, and
\texttt{-\/-find-links} points to the local directory containing the
\texttt{.whl} files. Pip resolves dependencies from the local files
only.

Verify the installation:

\begin{verbatim}
python -c "import marimo; import numpy; import matplotlib; print('All packages installed successfully')"
\end{verbatim}

Expected output: \textbf{All packages installed successfully}

\chapter{Appendix E --- Section E.2: Installing and Running
marimo}\label{appendix-e-section-e.2-installing-and-running-marimo}

Practice problems covering marimo installation, running notebooks in
edit and run modes, and navigating the interface.

\begin{center}\rule{0.5\linewidth}{0.5pt}\end{center}

\section{Problem E.2.1}\label{problem-e.2.1}

\textbf{Given:} A user has an active virtual environment with pip
installed and wants to install marimo.

\textbf{Find:} (a) The installation command, (b) the command to verify
the installation, and (c) the expected output.

\textbf{Solution:}

\begin{enumerate}
\def\labelenumi{(\alph{enumi})}
\tightlist
\item
  Install:
\end{enumerate}

\begin{verbatim}
pip install marimo
\end{verbatim}

\begin{enumerate}
\def\labelenumi{(\alph{enumi})}
\setcounter{enumi}{1}
\tightlist
\item
  Verify:
\end{enumerate}

\begin{verbatim}
marimo --version
\end{verbatim}

\begin{enumerate}
\def\labelenumi{(\alph{enumi})}
\setcounter{enumi}{2}
\tightlist
\item
  Expected output: \textbf{marimo 0.19.4} (or newer version number).
\end{enumerate}

If the command is not found, ensure the virtual environment is
activated. marimo installs a command-line entry point in
\texttt{.venv/bin/marimo} which is only available when the venv is
active.

\begin{center}\rule{0.5\linewidth}{0.5pt}\end{center}

\section{Problem E.2.2}\label{problem-e.2.2}

\textbf{Given:} A user wants to open the signal processing notebook
\texttt{08\_signal\_processing.py} for interactive exploration and
modification.

\textbf{Find:} (a) The command to open in edit mode, (b) what happens in
the browser, and (c) the keyboard shortcut to run a modified cell.

\textbf{Solution:}

\begin{enumerate}
\def\labelenumi{(\alph{enumi})}
\tightlist
\item
  Edit mode:
\end{enumerate}

\begin{verbatim}
cd scripts
marimo edit 08_signal_processing.py
\end{verbatim}

\begin{enumerate}
\def\labelenumi{(\alph{enumi})}
\setcounter{enumi}{1}
\item
  The default web browser opens with the marimo editor at
  \texttt{http://localhost:2718}. The notebook displays markdown cells
  with section descriptions and code cells with matplotlib graphs. Each
  cell shows its output (graph or text) below the code.
\item
  After modifying a cell, press \textbf{Shift+Enter} to run it. Because
  marimo is reactive, all downstream cells that depend on modified
  variables automatically re-run and update their outputs.
\end{enumerate}

\begin{center}\rule{0.5\linewidth}{0.5pt}\end{center}

\section{Problem E.2.3}\label{problem-e.2.3}

\textbf{Given:} A professor wants to display the op-amps notebook to a
class without showing the Python code --- only the descriptions and
graphs.

\textbf{Find:} (a) The command for read-only run mode, (b) the
difference in display, and (c) how students can interact with it.

\textbf{Solution:}

\begin{enumerate}
\def\labelenumi{(\alph{enumi})}
\tightlist
\item
  Run mode:
\end{enumerate}

\begin{verbatim}
marimo run 13_op_amps.py
\end{verbatim}

\begin{enumerate}
\def\labelenumi{(\alph{enumi})}
\setcounter{enumi}{1}
\item
  In run mode, the browser displays only the cell outputs --- markdown
  text and matplotlib graphs --- without the Python source code. The
  interface looks like a clean presentation or report.
\item
  In standard run mode, students can view but not modify the code.
  However, if the notebook includes marimo UI elements (sliders,
  dropdowns), students can \textbf{interact with those controls} and see
  the graphs update in real time. For example, a slider controlling the
  feedback resistance R\textsubscript{f} would let students explore how
  gain changes with resistance.
\end{enumerate}

\begin{center}\rule{0.5\linewidth}{0.5pt}\end{center}

\section{Problem E.2.4}\label{problem-e.2.4}

\textbf{Given:} A user opens the Chapter 12 electric motors notebook and
sees the cell dependency graph in the marimo menu.

\textbf{Find:} (a) How to access the dependency graph, (b) what it
shows, and (c) what happens when a cell with three downstream
dependencies is modified.

\textbf{Solution:}

\begin{enumerate}
\def\labelenumi{(\alph{enumi})}
\item
  In the marimo editor, click the \textbf{hamburger menu} (three lines)
  in the top-left corner, then select ``View dependency graph'' (or
  press Ctrl+Shift+D / Cmd+Shift+D on macOS).
\item
  The dependency graph shows cells as nodes connected by directed edges.
  An edge from cell A to cell B means B uses a variable defined in A.
  For example, if the imports cell defines \texttt{np}, and five
  computation cells use \texttt{np}, the imports cell has edges to all
  five.
\item
  When a cell with three downstream dependencies is modified and run:
\end{enumerate}

\begin{enumerate}
\def\labelenumi{\arabic{enumi}.}
\tightlist
\item
  The modified cell executes first
\item
  All three downstream cells automatically re-execute \textbf{in
  dependency order}
\item
  Any cells further downstream of those three also re-execute
\item
  The graphs and outputs update without manual intervention
\end{enumerate}

This is the core feature of marimo's \textbf{reactivity} --- changes
cascade through the dependency graph automatically.

\begin{center}\rule{0.5\linewidth}{0.5pt}\end{center}

\section{Problem E.2.5}\label{problem-e.2.5}

\textbf{Given:} A user wants to run two notebooks simultaneously --- one
for circuit analysis and one for signal processing --- to compare graphs
side by side.

\textbf{Find:} (a) How to run two marimo sessions at once, (b) the port
assignment, and (c) how to specify a custom port.

\textbf{Solution:}

\begin{enumerate}
\def\labelenumi{(\alph{enumi})}
\tightlist
\item
  Open two terminal windows (or tabs) with the virtual environment
  activated in each.
\end{enumerate}

Terminal 1:

\begin{verbatim}
marimo edit 07_circuit_analysis.py
\end{verbatim}

Terminal 2:

\begin{verbatim}
marimo edit 08_signal_processing.py --port 2719
\end{verbatim}

\begin{enumerate}
\def\labelenumi{(\alph{enumi})}
\setcounter{enumi}{1}
\item
  The first instance uses the default port \textbf{2718}. If a second
  instance is started without specifying a port, marimo automatically
  finds the next available port (2719, 2720, etc.).
\item
  Explicit port specification:
\end{enumerate}

\begin{verbatim}
marimo edit 08_signal_processing.py --port 3000
\end{verbatim}

The notebook opens at \texttt{http://localhost:3000}.

Both notebooks run independently in separate browser tabs and do not
share state.

\begin{center}\rule{0.5\linewidth}{0.5pt}\end{center}

\section{Problem E.2.6}\label{problem-e.2.6}

\textbf{Given:} A user modifies parameters in the antenna design
notebook but wants to undo all changes and return to the saved version.

\textbf{Find:} (a) How to undo individual changes, (b) how to revert to
the last saved state, and (c) how marimo handles unsaved changes.

\textbf{Solution:}

\begin{enumerate}
\def\labelenumi{(\alph{enumi})}
\item
  Individual undo within a cell: \textbf{Ctrl+Z} (Cmd+Z on macOS). Each
  cell maintains its own undo history.
\item
  To revert all unsaved changes, close the browser tab without saving.
  When you reopen the file with \texttt{marimo\ edit}, it loads the last
  saved version from disk. Alternatively, reload the page (F5 or Ctrl+R)
  --- marimo prompts whether to discard unsaved changes.
\item
  Unsaved changes are indicated by a \textbf{dot or indicator} in the
  tab title or editor header. Pressing \textbf{Ctrl+S} (Cmd+S on macOS)
  saves all changes to the \texttt{.py} file on disk. If the browser is
  closed without saving, \textbf{all unsaved changes are lost} ---
  marimo does not auto-save.
\end{enumerate}

\begin{center}\rule{0.5\linewidth}{0.5pt}\end{center}

\section{Problem E.2.7}\label{problem-e.2.7}

\textbf{Given:} A user wants to share a notebook with someone who does
not have Python or marimo installed.

\textbf{Find:} (a) The command to export to a static HTML file, (b) the
command to export to a flat Python script (no marimo dependency), and
(c) the limitations of each format.

\textbf{Solution:}

\begin{enumerate}
\def\labelenumi{(\alph{enumi})}
\tightlist
\item
  Export to HTML:
\end{enumerate}

\begin{verbatim}
marimo export html 07_circuit_analysis.py -o circuit_analysis.html
\end{verbatim}

This produces a self-contained HTML file with all graphs rendered as
static images.

\begin{enumerate}
\def\labelenumi{(\alph{enumi})}
\setcounter{enumi}{1}
\tightlist
\item
  Export to flat script:
\end{enumerate}

\begin{verbatim}
marimo export script 07_circuit_analysis.py -o circuit_analysis_flat.py
\end{verbatim}

This produces a standard Python script that can be run with
\texttt{python\ circuit\_analysis\_flat.py} (requires numpy and
matplotlib but not marimo).

\begin{enumerate}
\def\labelenumi{(\alph{enumi})}
\setcounter{enumi}{2}
\tightlist
\item
  Limitations:
\end{enumerate}

\begin{itemize}
\tightlist
\item
  \textbf{HTML export}: Static only --- sliders and interactive widgets
  do not work. Graphs are fixed images. The file can be large if there
  are many high-resolution plots.
\item
  \textbf{Script export}: No reactivity --- modifying a variable does
  not cascade updates. The script runs top-to-bottom like a normal
  Python program. Interactive marimo UI elements are stripped out.
\end{itemize}

\begin{center}\rule{0.5\linewidth}{0.5pt}\end{center}

\section{Problem E.2.8}\label{problem-e.2.8}

\textbf{Given:} A user encounters an error when starting marimo:
``Address already in use: (`localhost', 2718).''

\textbf{Find:} (a) The cause of the error, (b) how to find and stop the
process using the port, and (c) how to start marimo on a different port.

\textbf{Solution:}

\begin{enumerate}
\def\labelenumi{(\alph{enumi})}
\item
  The error means another process (likely a previous marimo session) is
  already listening on port 2718.
\item
  Find the process: On macOS/Linux:
\end{enumerate}

\begin{verbatim}
lsof -i :2718
\end{verbatim}

This shows the PID of the process. Kill it:

\begin{verbatim}
kill <PID>
\end{verbatim}

On Windows:

\begin{verbatim}
netstat -ano | findstr :2718
taskkill /PID <PID> /F
\end{verbatim}

\begin{enumerate}
\def\labelenumi{(\alph{enumi})}
\setcounter{enumi}{2}
\tightlist
\item
  Start on a different port:
\end{enumerate}

\begin{verbatim}
marimo edit 07_circuit_analysis.py --port 8080
\end{verbatim}

The notebook opens at \textbf{http://localhost:8080} instead.

\begin{center}\rule{0.5\linewidth}{0.5pt}\end{center}

\section{Problem E.2.9}\label{problem-e.2.9}

\textbf{Given:} A user wants to run a marimo notebook on a remote server
and access it from their laptop's browser.

\textbf{Find:} The commands to (a) start marimo on the server allowing
external connections, (b) connect from the laptop, and (c) set up SSH
port forwarding for security.

\textbf{Solution:}

\begin{enumerate}
\def\labelenumi{(\alph{enumi})}
\tightlist
\item
  Start on the server (bind to all interfaces):
\end{enumerate}

\begin{verbatim}
marimo edit 07_circuit_analysis.py --host 0.0.0.0 --port 8080
\end{verbatim}

\begin{enumerate}
\def\labelenumi{(\alph{enumi})}
\setcounter{enumi}{1}
\tightlist
\item
  From the laptop browser, navigate to:
\end{enumerate}

\begin{verbatim}
http://<server-ip>:8080
\end{verbatim}

This works but is \textbf{not secure} --- the connection is unencrypted.

\begin{enumerate}
\def\labelenumi{(\alph{enumi})}
\setcounter{enumi}{2}
\tightlist
\item
  Secure approach with SSH port forwarding: On the laptop:
\end{enumerate}

\begin{verbatim}
ssh -L 8080:localhost:8080 user@server-ip
\end{verbatim}

Then on the server (in the SSH session):

\begin{verbatim}
marimo edit 07_circuit_analysis.py --port 8080
\end{verbatim}

On the laptop browser: \textbf{http://localhost:8080}

The SSH tunnel encrypts all traffic between the laptop and server, and
the marimo port is not exposed to the network.

\begin{center}\rule{0.5\linewidth}{0.5pt}\end{center}

\section{Problem E.2.10}\label{problem-e.2.10}

\textbf{Given:} A user wants to launch a marimo notebook in a headless
environment (no display, such as a CI server) to validate that all cells
execute without errors.

\textbf{Find:} (a) The command to run non-interactively, (b) how to
check for errors, and (c) how to capture output.

\textbf{Solution:}

\begin{enumerate}
\def\labelenumi{(\alph{enumi})}
\tightlist
\item
  Run as a Python script:
\end{enumerate}

\begin{verbatim}
python 07_circuit_analysis.py
\end{verbatim}

This executes all cells in dependency order without opening a browser.

Alternatively, use marimo's export to validate:

\begin{verbatim}
marimo export html 07_circuit_analysis.py -o /dev/null
\end{verbatim}

If any cell raises an exception, the export fails with an error message.

\begin{enumerate}
\def\labelenumi{(\alph{enumi})}
\setcounter{enumi}{1}
\tightlist
\item
  Check for errors by examining the exit code:
\end{enumerate}

\begin{verbatim}
python 07_circuit_analysis.py
echo $?
\end{verbatim}

Exit code \textbf{0} means all cells executed successfully. A non-zero
exit code indicates an error.

\begin{enumerate}
\def\labelenumi{(\alph{enumi})}
\setcounter{enumi}{2}
\tightlist
\item
  Capture output:
\end{enumerate}

\begin{verbatim}
marimo export html 07_circuit_analysis.py -o output.html 2>&1 | tee build.log
\end{verbatim}

The HTML file contains all rendered outputs, and \texttt{build.log}
captures any warnings or errors from the execution.

This approach is useful for \textbf{continuous integration} ---
automatically validating that all notebooks run without errors after
code changes.

\chapter{Appendix E --- Section E.3: Understanding the Script
Structure}\label{appendix-e-section-e.3-understanding-the-script-structure}

Practice problems covering imports and app initialization, markdown and
code cells, and running marimo notebooks as Python scripts.

\begin{center}\rule{0.5\linewidth}{0.5pt}\end{center}

\section{Problem E.3.1}\label{problem-e.3.1}

\textbf{Given:} A user opens a marimo script and sees the following
header:

\begin{Shaded}
\begin{Highlighting}[]
\ImportTok{import}\NormalTok{ marimo}
\NormalTok{\_\_generated\_with }\OperatorTok{=} \StringTok{"0.19.4"}
\NormalTok{app }\OperatorTok{=}\NormalTok{ marimo.App()}
\end{Highlighting}
\end{Shaded}

\textbf{Find:} (a) What each line does, (b) whether the version string
affects execution, and (c) what \texttt{app} represents.

\textbf{Solution:}

\begin{enumerate}
\def\labelenumi{(\alph{enumi})}
\tightlist
\item
  Line-by-line:
\end{enumerate}

\begin{itemize}
\tightlist
\item
  \texttt{import\ marimo} --- imports the marimo library
\item
  \texttt{\_\_generated\_with\ =\ "0.19.4"} --- records the marimo
  version that created the notebook (metadata only)
\item
  \texttt{app\ =\ marimo.App()} --- creates the \textbf{application
  object} that manages all cells and their dependencies
\end{itemize}

\begin{enumerate}
\def\labelenumi{(\alph{enumi})}
\setcounter{enumi}{1}
\item
  The version string is \textbf{informational only} and does not affect
  execution. A notebook created with version 0.19.4 can be run with any
  compatible version of marimo. It serves as a hint for debugging
  compatibility issues.
\item
  The \texttt{app} object is the central coordinator. All cells
  decorated with \texttt{@app.cell} register themselves with this
  object. When the notebook runs, \texttt{app} determines execution
  order based on the dependency graph.
\end{enumerate}

\begin{center}\rule{0.5\linewidth}{0.5pt}\end{center}

\section{Problem E.3.2}\label{problem-e.3.2}

\textbf{Given:} The first code cell of a marimo notebook is:

\begin{Shaded}
\begin{Highlighting}[]
\AttributeTok{@app.cell}
\KeywordTok{def}\NormalTok{ \_():}
    \ImportTok{import}\NormalTok{ marimo }\ImportTok{as}\NormalTok{ mo}
    \ImportTok{import}\NormalTok{ numpy }\ImportTok{as}\NormalTok{ np}
    \ImportTok{import}\NormalTok{ matplotlib.pyplot }\ImportTok{as}\NormalTok{ plt}
    \ControlFlowTok{return}\NormalTok{ mo, np, plt}
\end{Highlighting}
\end{Shaded}

\textbf{Find:} (a) Why the function is named \texttt{\_}, (b) why there
are no parameters, (c) what the return statement does, and (d) what
happens if \texttt{return} is omitted.

\textbf{Solution:}

\begin{enumerate}
\def\labelenumi{(\alph{enumi})}
\item
  The function name \texttt{\_} is a convention meaning ``no specific
  name needed.'' marimo identifies cells by their position and
  dependencies, not by function names. Any valid Python name works.
\item
  No parameters means this cell \textbf{has no dependencies on other
  cells}. It is a root node in the dependency graph and executes first.
\item
  The \texttt{return\ mo,\ np,\ plt} statement \textbf{exports} these
  variables, making them available to other cells. Any cell that lists
  \texttt{np} as a function parameter will receive the numpy module.
\item
  If \texttt{return} is omitted, the imports are \textbf{local to this
  cell only} and no other cell can access \texttt{mo}, \texttt{np}, or
  \texttt{plt}. Other cells would fail with \texttt{NameError} when
  trying to use these modules.
\end{enumerate}

\begin{center}\rule{0.5\linewidth}{0.5pt}\end{center}

\section{Problem E.3.3}\label{problem-e.3.3}

\textbf{Given:} Two cells in a notebook:

\begin{Shaded}
\begin{Highlighting}[]
\CommentTok{\# Cell A}
\AttributeTok{@app.cell}
\KeywordTok{def}\NormalTok{ \_(np):}
\NormalTok{    R }\OperatorTok{=} \DecValTok{1000}  \CommentTok{\# Ω}
\NormalTok{    C }\OperatorTok{=} \FloatTok{10e{-}6}  \CommentTok{\# F}
\NormalTok{    tau }\OperatorTok{=}\NormalTok{ R }\OperatorTok{*}\NormalTok{ C}
\NormalTok{    t }\OperatorTok{=}\NormalTok{ np.linspace(}\DecValTok{0}\NormalTok{, }\DecValTok{5} \OperatorTok{*}\NormalTok{ tau, }\DecValTok{500}\NormalTok{)}
    \ControlFlowTok{return}\NormalTok{ R, C, tau, t}

\CommentTok{\# Cell B}
\AttributeTok{@app.cell}
\KeywordTok{def}\NormalTok{ \_(np, plt, R, C, tau, t):}
\NormalTok{    Vs }\OperatorTok{=} \FloatTok{5.0}
\NormalTok{    v }\OperatorTok{=}\NormalTok{ Vs }\OperatorTok{*}\NormalTok{ (}\DecValTok{1} \OperatorTok{{-}}\NormalTok{ np.exp(}\OperatorTok{{-}}\NormalTok{t }\OperatorTok{/}\NormalTok{ tau))}
\NormalTok{    fig, ax }\OperatorTok{=}\NormalTok{ plt.subplots(figsize}\OperatorTok{=}\NormalTok{(}\DecValTok{10}\NormalTok{, }\DecValTok{5}\NormalTok{))}
\NormalTok{    ax.plot(t }\OperatorTok{*} \FloatTok{1e3}\NormalTok{, v)}
\NormalTok{    ax.set\_xlabel(}\StringTok{"Time (ms)"}\NormalTok{)}
\NormalTok{    ax.set\_ylabel(}\StringTok{"Voltage (V)"}\NormalTok{)}
\NormalTok{    ax.set\_title(}\SpecialStringTok{f"RC Charging: R=}\SpecialCharTok{\{}\NormalTok{R}\SpecialCharTok{\}}\SpecialStringTok{Ω, C=}\SpecialCharTok{\{}\NormalTok{C}\OperatorTok{*}\FloatTok{1e6}\SpecialCharTok{:.0f\}}\SpecialStringTok{μF, τ=}\SpecialCharTok{\{}\NormalTok{tau}\OperatorTok{*}\FloatTok{1e3}\SpecialCharTok{:.1f\}}\SpecialStringTok{ms"}\NormalTok{)}
\NormalTok{    fig}
    \ControlFlowTok{return}
\end{Highlighting}
\end{Shaded}

\textbf{Find:} (a) The dependency relationship between the cells, (b)
what happens when R is changed in Cell A, and (c) why Cell B returns
nothing.

\textbf{Solution:}

\begin{enumerate}
\def\labelenumi{(\alph{enumi})}
\tightlist
\item
  Cell B depends on Cell A through the variables \texttt{R}, \texttt{C},
  \texttt{tau}, and \texttt{t}. Cell B also depends on the imports cell
  through \texttt{np} and \texttt{plt}. Cell A depends only on the
  imports cell through \texttt{np}.
\end{enumerate}

Dependency chain: \textbf{imports → Cell A → Cell B}

\begin{enumerate}
\def\labelenumi{(\alph{enumi})}
\setcounter{enumi}{1}
\tightlist
\item
  When R is changed in Cell A (e.g., from 1000 to 2200):
\end{enumerate}

\begin{enumerate}
\def\labelenumi{\arabic{enumi}.}
\tightlist
\item
  Cell A re-executes, computing new values for \texttt{tau} and
  \texttt{t}
\item
  Cell B \textbf{automatically re-executes} because its inputs
  (\texttt{R}, \texttt{tau}, \texttt{t}) have changed
\item
  The graph updates to show a slower charging curve with τ = 22 ms
  instead of 10 ms
\end{enumerate}

\begin{enumerate}
\def\labelenumi{(\alph{enumi})}
\setcounter{enumi}{2}
\tightlist
\item
  Cell B returns nothing because it \textbf{does not export any
  variables} for use by other cells. Its only purpose is to compute and
  display the graph. The last expression \texttt{fig} is displayed as
  the cell output but is not returned for reuse.
\end{enumerate}

\begin{center}\rule{0.5\linewidth}{0.5pt}\end{center}

\section{Problem E.3.4}\label{problem-e.3.4}

\textbf{Given:} A user wants to add a markdown description cell before a
computation cell.

\textbf{Find:} (a) The syntax for a markdown cell, (b) how to include
formatted text with bold and headings, and (c) how to include a Greek
letter and subscript.

\textbf{Solution:}

\begin{enumerate}
\def\labelenumi{(\alph{enumi})}
\tightlist
\item
  Markdown cell syntax:
\end{enumerate}

\begin{Shaded}
\begin{Highlighting}[]
\AttributeTok{@app.cell}
\KeywordTok{def}\NormalTok{ \_(mo):}
\NormalTok{    mo.md(}\StringTok{"\#\# RC Circuit Time Constant}\CharTok{\textbackslash{}n\textbackslash{}n}\StringTok{The voltage across a charging capacitor..."}\NormalTok{)}
    \ControlFlowTok{return}
\end{Highlighting}
\end{Shaded}

\begin{enumerate}
\def\labelenumi{(\alph{enumi})}
\setcounter{enumi}{1}
\tightlist
\item
  Formatted text example:
\end{enumerate}

\begin{Shaded}
\begin{Highlighting}[]
\AttributeTok{@app.cell}
\KeywordTok{def}\NormalTok{ \_(mo):}
\NormalTok{    mo.md(}\StringTok{"""}
\StringTok{    \#\# RC Circuit Charging}

\StringTok{    The **time constant** τ determines how quickly the capacitor charges.}
\StringTok{    After **5τ**, the capacitor is considered fully charged (99.3\%).}

\StringTok{    \#\#\# Key Equations}
\StringTok{    {-} Voltage: V(t) = Vs × (1 {-} e\^{}({-}t/τ))}
\StringTok{    {-} Current: I(t) = (Vs/R) × e\^{}({-}t/τ)}
\StringTok{    """}\NormalTok{)}
    \ControlFlowTok{return}
\end{Highlighting}
\end{Shaded}

\begin{enumerate}
\def\labelenumi{(\alph{enumi})}
\setcounter{enumi}{2}
\tightlist
\item
  Greek letters and subscripts use Unicode in the markdown string:
\end{enumerate}

\begin{Shaded}
\begin{Highlighting}[]
\NormalTok{mo.md(}\StringTok{"The time constant **τ = R × C** where R is resistance in Ω and C is capacitance in F. The voltage V₀ is the initial condition."}\NormalTok{)}
\end{Highlighting}
\end{Shaded}

Output renders as: The time constant \textbf{τ = R × C} where R is
resistance in Ω and C is capacitance in F. The voltage V₀ is the initial
condition.

\begin{center}\rule{0.5\linewidth}{0.5pt}\end{center}

\section{Problem E.3.5}\label{problem-e.3.5}

\textbf{Given:} A notebook has the following cell that creates a graph:

\begin{Shaded}
\begin{Highlighting}[]
\AttributeTok{@app.cell}
\KeywordTok{def}\NormalTok{ \_(np, plt):}
\NormalTok{    f }\OperatorTok{=}\NormalTok{ np.linspace(}\DecValTok{1}\NormalTok{, }\FloatTok{100e3}\NormalTok{, }\DecValTok{10000}\NormalTok{)}
\NormalTok{    H }\OperatorTok{=} \DecValTok{1} \OperatorTok{/}\NormalTok{ np.sqrt(}\DecValTok{1} \OperatorTok{+}\NormalTok{ (f }\OperatorTok{/} \FloatTok{10e3}\NormalTok{)}\OperatorTok{**}\DecValTok{2}\NormalTok{)}
\NormalTok{    H\_dB }\OperatorTok{=} \DecValTok{20} \OperatorTok{*}\NormalTok{ np.log10(H)}
\NormalTok{    fig, ax }\OperatorTok{=}\NormalTok{ plt.subplots(figsize}\OperatorTok{=}\NormalTok{(}\DecValTok{10}\NormalTok{, }\DecValTok{5}\NormalTok{))}
\NormalTok{    ax.semilogx(f, H\_dB)}
\NormalTok{    ax.set\_xlabel(}\StringTok{"Frequency (Hz)"}\NormalTok{)}
\NormalTok{    ax.set\_ylabel(}\StringTok{"Magnitude (dB)"}\NormalTok{)}
\NormalTok{    ax.set\_title(}\StringTok{"Low{-}Pass Filter Response"}\NormalTok{)}
\NormalTok{    ax.grid(}\VariableTok{True}\NormalTok{, which}\OperatorTok{=}\StringTok{\textquotesingle{}both\textquotesingle{}}\NormalTok{, alpha}\OperatorTok{=}\FloatTok{0.3}\NormalTok{)}
\NormalTok{    ax.set\_ylim(}\OperatorTok{{-}}\DecValTok{40}\NormalTok{, }\DecValTok{5}\NormalTok{)}
\NormalTok{    fig}
    \ControlFlowTok{return}
\end{Highlighting}
\end{Shaded}

\textbf{Find:} (a) What type of filter this plots, (b) the -3 dB
frequency, (c) how to modify it to show a second filter with
f\textsubscript{c} = 50 kHz on the same graph.

\textbf{Solution:}

\begin{enumerate}
\def\labelenumi{(\alph{enumi})}
\item
  This is a \textbf{first-order low-pass filter} frequency response
  (Bode magnitude plot). The transfer function H(f) = 1/√(1 +
  (f/f\textsubscript{c})²) has a -20 dB/decade rolloff above the cutoff
  frequency.
\item
  The -3 dB frequency is \textbf{f\textsubscript{c} = 10 kHz} (set by
  the denominator \texttt{f\ /\ 10e3}). At f = 10 kHz, H = 1/√2 = -3.01
  dB.
\item
  Add the second filter to the same axes:
\end{enumerate}

\begin{Shaded}
\begin{Highlighting}[]
\AttributeTok{@app.cell}
\KeywordTok{def}\NormalTok{ \_(np, plt):}
\NormalTok{    f }\OperatorTok{=}\NormalTok{ np.linspace(}\DecValTok{1}\NormalTok{, }\FloatTok{100e3}\NormalTok{, }\DecValTok{10000}\NormalTok{)}
\NormalTok{    H1 }\OperatorTok{=} \DecValTok{1} \OperatorTok{/}\NormalTok{ np.sqrt(}\DecValTok{1} \OperatorTok{+}\NormalTok{ (f }\OperatorTok{/} \FloatTok{10e3}\NormalTok{)}\OperatorTok{**}\DecValTok{2}\NormalTok{)}
\NormalTok{    H2 }\OperatorTok{=} \DecValTok{1} \OperatorTok{/}\NormalTok{ np.sqrt(}\DecValTok{1} \OperatorTok{+}\NormalTok{ (f }\OperatorTok{/} \FloatTok{50e3}\NormalTok{)}\OperatorTok{**}\DecValTok{2}\NormalTok{)}
\NormalTok{    H1\_dB }\OperatorTok{=} \DecValTok{20} \OperatorTok{*}\NormalTok{ np.log10(H1)}
\NormalTok{    H2\_dB }\OperatorTok{=} \DecValTok{20} \OperatorTok{*}\NormalTok{ np.log10(H2)}
\NormalTok{    fig, ax }\OperatorTok{=}\NormalTok{ plt.subplots(figsize}\OperatorTok{=}\NormalTok{(}\DecValTok{10}\NormalTok{, }\DecValTok{5}\NormalTok{))}
\NormalTok{    ax.semilogx(f, H1\_dB, label}\OperatorTok{=}\StringTok{"fc = 10 kHz"}\NormalTok{)}
\NormalTok{    ax.semilogx(f, H2\_dB, label}\OperatorTok{=}\StringTok{"fc = 50 kHz"}\NormalTok{)}
\NormalTok{    ax.set\_xlabel(}\StringTok{"Frequency (Hz)"}\NormalTok{)}
\NormalTok{    ax.set\_ylabel(}\StringTok{"Magnitude (dB)"}\NormalTok{)}
\NormalTok{    ax.set\_title(}\StringTok{"Low{-}Pass Filter Comparison"}\NormalTok{)}
\NormalTok{    ax.legend()}
\NormalTok{    ax.grid(}\VariableTok{True}\NormalTok{, which}\OperatorTok{=}\StringTok{\textquotesingle{}both\textquotesingle{}}\NormalTok{, alpha}\OperatorTok{=}\FloatTok{0.3}\NormalTok{)}
\NormalTok{    ax.set\_ylim(}\OperatorTok{{-}}\DecValTok{40}\NormalTok{, }\DecValTok{5}\NormalTok{)}
\NormalTok{    fig}
    \ControlFlowTok{return}
\end{Highlighting}
\end{Shaded}

\begin{center}\rule{0.5\linewidth}{0.5pt}\end{center}

\section{Problem E.3.6}\label{problem-e.3.6}

\textbf{Given:} The end of every marimo script contains:

\begin{Shaded}
\begin{Highlighting}[]
\ControlFlowTok{if} \VariableTok{\_\_name\_\_} \OperatorTok{==} \StringTok{"\_\_main\_\_"}\NormalTok{:}
\NormalTok{    app.run()}
\end{Highlighting}
\end{Shaded}

\textbf{Find:} (a) What this block does, (b) when it executes, and (c)
what \texttt{app.run()} triggers.

\textbf{Solution:}

\begin{enumerate}
\def\labelenumi{(\alph{enumi})}
\item
  This is the \textbf{entry point guard} --- a standard Python pattern
  that runs code only when the file is executed directly (not when
  imported as a module).
\item
  It executes when the user runs:
\end{enumerate}

\begin{verbatim}
python 07_circuit_analysis.py
\end{verbatim}

It does \textbf{not} execute when marimo loads the file via
\texttt{marimo\ edit} or \texttt{marimo\ run}, because in those cases
marimo imports the file rather than running it directly.

\begin{enumerate}
\def\labelenumi{(\alph{enumi})}
\setcounter{enumi}{2}
\tightlist
\item
  \texttt{app.run()} launches the marimo notebook viewer in the default
  web browser, identical to running
  \texttt{marimo\ run\ 07\_circuit\_analysis.py} from the command line.
  All cells execute in dependency order and their outputs are displayed
  in read-only mode.
\end{enumerate}

\begin{center}\rule{0.5\linewidth}{0.5pt}\end{center}

\section{Problem E.3.7}\label{problem-e.3.7}

\textbf{Given:} A user wants to use a computed value from one cell as a
label in a markdown cell.

\textbf{Find:} (a) How to pass a variable from a code cell to a markdown
cell, (b) the syntax for string interpolation in \texttt{mo.md()}, and
(c) an example.

\textbf{Solution:}

\begin{enumerate}
\def\labelenumi{(\alph{enumi})}
\item
  The markdown cell lists the variable as a function parameter, just
  like any other cell dependency.
\item
  Use Python f-strings inside \texttt{mo.md()}:
\end{enumerate}

\begin{Shaded}
\begin{Highlighting}[]
\AttributeTok{@app.cell}
\KeywordTok{def}\NormalTok{ \_(mo, tau):}
\NormalTok{    mo.md(}\SpecialStringTok{f"The computed time constant is **τ = }\SpecialCharTok{\{}\NormalTok{tau}\OperatorTok{*}\FloatTok{1e3}\SpecialCharTok{:.2f\}}\SpecialStringTok{ ms**."}\NormalTok{)}
    \ControlFlowTok{return}
\end{Highlighting}
\end{Shaded}

\begin{enumerate}
\def\labelenumi{(\alph{enumi})}
\setcounter{enumi}{2}
\tightlist
\item
  Full example:
\end{enumerate}

\begin{Shaded}
\begin{Highlighting}[]
\CommentTok{\# Code cell computes tau}
\AttributeTok{@app.cell}
\KeywordTok{def}\NormalTok{ \_():}
\NormalTok{    R }\OperatorTok{=} \DecValTok{4700}  \CommentTok{\# Ω}
\NormalTok{    C }\OperatorTok{=} \FloatTok{22e{-}6}  \CommentTok{\# F}
\NormalTok{    tau }\OperatorTok{=}\NormalTok{ R }\OperatorTok{*}\NormalTok{ C  }\CommentTok{\# = 0.1034 s}
    \ControlFlowTok{return}\NormalTok{ R, C, tau}

\CommentTok{\# Markdown cell displays the result}
\AttributeTok{@app.cell}
\KeywordTok{def}\NormalTok{ \_(mo, R, C, tau):}
\NormalTok{    mo.md(}\SpecialStringTok{f"""}
\SpecialStringTok{    \#\# Computed Parameters}
\SpecialStringTok{    {-} Resistance: **R = }\SpecialCharTok{\{}\NormalTok{R}\SpecialCharTok{:,\}}\SpecialStringTok{ Ω**}
\SpecialStringTok{    {-} Capacitance: **C = }\SpecialCharTok{\{}\NormalTok{C}\OperatorTok{*}\FloatTok{1e6}\SpecialCharTok{:.0f\}}\SpecialStringTok{ μF**}
\SpecialStringTok{    {-} Time constant: **τ = }\SpecialCharTok{\{}\NormalTok{tau}\OperatorTok{*}\FloatTok{1e3}\SpecialCharTok{:.1f\}}\SpecialStringTok{ ms**}
\SpecialStringTok{    {-} 5τ settling time: **}\SpecialCharTok{\{}\DecValTok{5}\OperatorTok{*}\NormalTok{tau}\OperatorTok{*}\FloatTok{1e3}\SpecialCharTok{:.1f\}}\SpecialStringTok{ ms**}
\SpecialStringTok{    """}\NormalTok{)}
    \ControlFlowTok{return}
\end{Highlighting}
\end{Shaded}

Output: R = 4,700 Ω, C = 22 μF, τ = 103.4 ms, 5τ settling time = 517.0
ms. When R or C changes, the markdown cell \textbf{automatically
updates} with the new values.

\begin{center}\rule{0.5\linewidth}{0.5pt}\end{center}

\section{Problem E.3.8}\label{problem-e.3.8}

\textbf{Given:} A user encounters a
\texttt{NameError:\ name\ \textquotesingle{}scipy\textquotesingle{}\ is\ not\ defined}
when adding a new cell that uses \texttt{scipy.signal.butter}.

\textbf{Find:} (a) The cause of the error, (b) how to fix it, and (c)
the proper way to add a new import.

\textbf{Solution:}

\begin{enumerate}
\def\labelenumi{(\alph{enumi})}
\item
  The error means \texttt{scipy} has not been imported in any cell, or
  it was imported but not returned (exported) from the imports cell.
\item
  Option 1 --- Add scipy to the existing imports cell:
\end{enumerate}

\begin{Shaded}
\begin{Highlighting}[]
\AttributeTok{@app.cell}
\KeywordTok{def}\NormalTok{ \_():}
    \ImportTok{import}\NormalTok{ marimo }\ImportTok{as}\NormalTok{ mo}
    \ImportTok{import}\NormalTok{ numpy }\ImportTok{as}\NormalTok{ np}
    \ImportTok{import}\NormalTok{ matplotlib.pyplot }\ImportTok{as}\NormalTok{ plt}
    \ImportTok{import}\NormalTok{ scipy.signal}
    \ControlFlowTok{return}\NormalTok{ mo, np, plt, scipy}
\end{Highlighting}
\end{Shaded}

Option 2 --- Create a new imports cell:

\begin{Shaded}
\begin{Highlighting}[]
\AttributeTok{@app.cell}
\KeywordTok{def}\NormalTok{ \_():}
    \ImportTok{import}\NormalTok{ scipy.signal}
    \ControlFlowTok{return}\NormalTok{ scipy,}
\end{Highlighting}
\end{Shaded}

\begin{enumerate}
\def\labelenumi{(\alph{enumi})}
\setcounter{enumi}{2}
\tightlist
\item
  The new cell that uses scipy must list it as a parameter:
\end{enumerate}

\begin{Shaded}
\begin{Highlighting}[]
\AttributeTok{@app.cell}
\KeywordTok{def}\NormalTok{ \_(np, plt, scipy):}
\NormalTok{    b, a }\OperatorTok{=}\NormalTok{ scipy.signal.butter(}\DecValTok{4}\NormalTok{, }\DecValTok{1000}\NormalTok{, fs}\OperatorTok{=}\DecValTok{44100}\NormalTok{)}
\NormalTok{    w, h }\OperatorTok{=}\NormalTok{ scipy.signal.freqz(b, a, fs}\OperatorTok{=}\DecValTok{44100}\NormalTok{)}
    \CommentTok{\# ... plot the response}
    \ControlFlowTok{return}
\end{Highlighting}
\end{Shaded}

Also ensure scipy is installed: \texttt{pip\ install\ scipy} and add it
to \texttt{requirements.txt}.

\begin{center}\rule{0.5\linewidth}{0.5pt}\end{center}

\section{Problem E.3.9}\label{problem-e.3.9}

\textbf{Given:} A user notices that two cells in a notebook both define
a variable called \texttt{R}, causing a marimo error: ``Variable `R' is
defined in multiple cells.''

\textbf{Find:} (a) Why marimo reports this error, (b) how to fix it, and
(c) the design principle behind this restriction.

\textbf{Solution:}

\begin{enumerate}
\def\labelenumi{(\alph{enumi})}
\item
  marimo requires that each exported variable is defined in
  \textbf{exactly one cell}. This is necessary for the reactive
  dependency graph to be unambiguous --- if \texttt{R} were defined in
  two cells, marimo cannot determine which value downstream cells should
  receive.
\item
  Fix by using distinct variable names:
\end{enumerate}

\begin{Shaded}
\begin{Highlighting}[]
\CommentTok{\# Cell 1: RC circuit}
\AttributeTok{@app.cell}
\KeywordTok{def}\NormalTok{ \_():}
\NormalTok{    R\_rc }\OperatorTok{=} \DecValTok{1000}  \CommentTok{\# Ω}
\NormalTok{    C\_rc }\OperatorTok{=} \FloatTok{10e{-}6}
    \ControlFlowTok{return}\NormalTok{ R\_rc, C\_rc}

\CommentTok{\# Cell 2: RL circuit}
\AttributeTok{@app.cell}
\KeywordTok{def}\NormalTok{ \_():}
\NormalTok{    R\_rl }\OperatorTok{=} \DecValTok{470}  \CommentTok{\# Ω}
\NormalTok{    L\_rl }\OperatorTok{=} \FloatTok{100e{-}3}
    \ControlFlowTok{return}\NormalTok{ R\_rl, L\_rl}
\end{Highlighting}
\end{Shaded}

Alternatively, if the variable is only used locally and does not need to
be exported, do not include it in the return statement. Variables not
returned are local to the cell.

\begin{enumerate}
\def\labelenumi{(\alph{enumi})}
\setcounter{enumi}{2}
\tightlist
\item
  The design principle is \textbf{single-definition, multiple-use} ---
  each variable has one authoritative source, enabling deterministic
  reactive updates. This prevents the ambiguity bugs common in Jupyter
  notebooks where cells can be run in arbitrary order.
\end{enumerate}

\begin{center}\rule{0.5\linewidth}{0.5pt}\end{center}

\section{Problem E.3.10}\label{problem-e.3.10}

\textbf{Given:} A user wants to understand the execution order of the
following four cells:

\begin{Shaded}
\begin{Highlighting}[]
\CommentTok{\# Cell 1}
\AttributeTok{@app.cell}
\KeywordTok{def}\NormalTok{ \_():}
    \ImportTok{import}\NormalTok{ numpy }\ImportTok{as}\NormalTok{ np}
    \ControlFlowTok{return}\NormalTok{ np,}

\CommentTok{\# Cell 2}
\AttributeTok{@app.cell}
\KeywordTok{def}\NormalTok{ \_(np):}
\NormalTok{    t }\OperatorTok{=}\NormalTok{ np.linspace(}\DecValTok{0}\NormalTok{, }\DecValTok{1}\NormalTok{, }\DecValTok{100}\NormalTok{)}
    \ControlFlowTok{return}\NormalTok{ t,}

\CommentTok{\# Cell 3}
\AttributeTok{@app.cell}
\KeywordTok{def}\NormalTok{ \_(np, t):}
\NormalTok{    y }\OperatorTok{=}\NormalTok{ np.sin(}\DecValTok{2} \OperatorTok{*}\NormalTok{ np.pi }\OperatorTok{*} \DecValTok{5} \OperatorTok{*}\NormalTok{ t)}
    \ControlFlowTok{return}\NormalTok{ y,}

\CommentTok{\# Cell 4}
\AttributeTok{@app.cell}
\KeywordTok{def}\NormalTok{ \_(t, y):}
    \ImportTok{import}\NormalTok{ matplotlib.pyplot }\ImportTok{as}\NormalTok{ plt}
\NormalTok{    fig, ax }\OperatorTok{=}\NormalTok{ plt.subplots()}
\NormalTok{    ax.plot(t, y)}
\NormalTok{    fig}
    \ControlFlowTok{return}
\end{Highlighting}
\end{Shaded}

\textbf{Find:} (a) The dependency graph, (b) the execution order, and
(c) what happens if Cell 2 changes \texttt{t} to
\texttt{np.linspace(0,\ 2,\ 200)}.

\textbf{Solution:}

\begin{enumerate}
\def\labelenumi{(\alph{enumi})}
\tightlist
\item
  Dependency graph:
\end{enumerate}

\begin{itemize}
\tightlist
\item
  Cell 1: no dependencies → exports \texttt{np}
\item
  Cell 2: depends on \texttt{np} → exports \texttt{t}
\item
  Cell 3: depends on \texttt{np} and \texttt{t} → exports \texttt{y}
\item
  Cell 4: depends on \texttt{t} and \texttt{y} → exports nothing
\end{itemize}

\begin{verbatim}
Cell 1 (np)
  ├── Cell 2 (t)
  │     ├── Cell 3 (y)   ← also depends on np from Cell 1
  │     │     └── Cell 4  ← also depends on t from Cell 2
  │     └── Cell 4
  └── Cell 3
\end{verbatim}

\begin{enumerate}
\def\labelenumi{(\alph{enumi})}
\setcounter{enumi}{1}
\item
  Execution order: \textbf{Cell 1 → Cell 2 → Cell 3 → Cell 4}
  (topologically sorted).
\item
  If Cell 2 changes:
\end{enumerate}

\begin{enumerate}
\def\labelenumi{\arabic{enumi}.}
\tightlist
\item
  Cell 2 re-executes → new \texttt{t} with 200 points over 0--2 seconds
\item
  Cell 3 re-executes (depends on \texttt{t}) → new \texttt{y} with 10
  full sine cycles
\item
  Cell 4 re-executes (depends on \texttt{t} and \texttt{y}) → graph
  updates to show 10 cycles over 2 seconds instead of 5 cycles over 1
  second
\end{enumerate}

Cell 1 does \textbf{not} re-execute because it has no upstream
dependency on Cell 2.

\chapter{Appendix E --- Section E.4: Modifying and Extending
Scripts}\label{appendix-e-section-e.4-modifying-and-extending-scripts}

Practice problems covering changing parameters, adding new cells,
creating new visualizations, and extending marimo notebooks.

\begin{center}\rule{0.5\linewidth}{0.5pt}\end{center}

\section{Problem E.4.1}\label{problem-e.4.1}

\textbf{Given:} The RC circuit charging cell in
\texttt{07\_circuit\_analysis.py} has the parameters:

\begin{Shaded}
\begin{Highlighting}[]
\NormalTok{R\_rc }\OperatorTok{=} \FloatTok{47e3}   \CommentTok{\# 47 kΩ}
\NormalTok{C\_rc }\OperatorTok{=} \FloatTok{10e{-}6}  \CommentTok{\# 10 μF}
\NormalTok{Vs\_rc }\OperatorTok{=} \DecValTok{9}     \CommentTok{\# 9 V supply}
\end{Highlighting}
\end{Shaded}

A user wants to model a 1 kΩ / 100 μF circuit with a 5 V supply.

\textbf{Find:} (a) The new parameter values, (b) the new time constant,
(c) the expected effect on the graph, and (d) the charging voltage at t
= 200 ms.

\textbf{Solution:}

\begin{enumerate}
\def\labelenumi{(\alph{enumi})}
\tightlist
\item
  New parameters:
\end{enumerate}

\begin{Shaded}
\begin{Highlighting}[]
\NormalTok{R\_rc }\OperatorTok{=} \FloatTok{1e3}     \CommentTok{\# 1 kΩ}
\NormalTok{C\_rc }\OperatorTok{=} \FloatTok{100e{-}6}  \CommentTok{\# 100 μF}
\NormalTok{Vs\_rc }\OperatorTok{=} \DecValTok{5}      \CommentTok{\# 5 V supply}
\end{Highlighting}
\end{Shaded}

\begin{enumerate}
\def\labelenumi{(\alph{enumi})}
\setcounter{enumi}{1}
\tightlist
\item
  New time constant: τ = R × C = 1,000 × 100 × 10⁻⁶ = \textbf{0.1 s =
  100 ms}
\end{enumerate}

The original τ = 47,000 × 10 × 10⁻⁶ = 0.47 s. The new circuit charges
\textbf{4.7× faster}.

\begin{enumerate}
\def\labelenumi{(\alph{enumi})}
\setcounter{enumi}{2}
\tightlist
\item
  The graph will show:
\end{enumerate}

\begin{itemize}
\tightlist
\item
  Lower final voltage (5 V instead of 9 V)
\item
  Faster exponential rise (100 ms time constant instead of 470 ms)
\item
  The 5τ settling time is 500 ms instead of 2.35 s
\end{itemize}

\begin{enumerate}
\def\labelenumi{(\alph{enumi})}
\setcounter{enumi}{3}
\tightlist
\item
  Voltage at t = 200 ms: V(200 ms) = 5 × (1 −
  e\textsuperscript{−0.2/0.1}) = 5 × (1 − e⁻²) = 5 × (1 − 0.1353) = 5 ×
  0.8647 = \textbf{4.32 V}
\end{enumerate}

The capacitor is 86.5\% charged after 2τ.

\begin{center}\rule{0.5\linewidth}{0.5pt}\end{center}

\section{Problem E.4.2}\label{problem-e.4.2}

\textbf{Given:} A user wants to add a cell to the circuit analysis
notebook that plots the power dissipated in the resistor during RC
charging.

\textbf{Find:} (a) The formula for resistor power versus time, (b) the
code for the new cell, and (c) the expected shape of the graph.

\textbf{Solution:}

\begin{enumerate}
\def\labelenumi{(\alph{enumi})}
\item
  The current during charging: I(t) = (V\textsubscript{s}/R) ×
  e\textsuperscript{−t/τ} Power in the resistor: P(t) = I²R =
  (V\textsubscript{s}²/R) × e\textsuperscript{−2t/τ}
\item
  New cell code:
\end{enumerate}

\begin{Shaded}
\begin{Highlighting}[]
\AttributeTok{@app.cell}
\KeywordTok{def}\NormalTok{ \_(np, plt, R\_rc, C\_rc, Vs\_rc):}
\NormalTok{    tau }\OperatorTok{=}\NormalTok{ R\_rc }\OperatorTok{*}\NormalTok{ C\_rc}
\NormalTok{    t }\OperatorTok{=}\NormalTok{ np.linspace(}\DecValTok{0}\NormalTok{, }\DecValTok{5} \OperatorTok{*}\NormalTok{ tau, }\DecValTok{500}\NormalTok{)}
\NormalTok{    P\_resistor }\OperatorTok{=}\NormalTok{ (Vs\_rc}\OperatorTok{**}\DecValTok{2} \OperatorTok{/}\NormalTok{ R\_rc) }\OperatorTok{*}\NormalTok{ np.exp(}\OperatorTok{{-}}\DecValTok{2} \OperatorTok{*}\NormalTok{ t }\OperatorTok{/}\NormalTok{ tau)}
\NormalTok{    fig, ax }\OperatorTok{=}\NormalTok{ plt.subplots(figsize}\OperatorTok{=}\NormalTok{(}\DecValTok{10}\NormalTok{, }\DecValTok{5}\NormalTok{))}
\NormalTok{    ax.plot(t }\OperatorTok{*} \FloatTok{1e3}\NormalTok{, P\_resistor }\OperatorTok{*} \FloatTok{1e3}\NormalTok{, }\StringTok{"r{-}"}\NormalTok{, linewidth}\OperatorTok{=}\DecValTok{2}\NormalTok{)}
\NormalTok{    ax.set\_xlabel(}\StringTok{"Time (ms)"}\NormalTok{)}
\NormalTok{    ax.set\_ylabel(}\StringTok{"Power (mW)"}\NormalTok{)}
\NormalTok{    ax.set\_title(}\SpecialStringTok{f"Resistor Power Dissipation (R=}\SpecialCharTok{\{}\NormalTok{R\_rc}\OperatorTok{/}\FloatTok{1e3}\SpecialCharTok{:.1f\}}\SpecialStringTok{kΩ)"}\NormalTok{)}
\NormalTok{    ax.grid(}\VariableTok{True}\NormalTok{, alpha}\OperatorTok{=}\FloatTok{0.3}\NormalTok{)}
\NormalTok{    fig}
    \ControlFlowTok{return}
\end{Highlighting}
\end{Shaded}

\begin{enumerate}
\def\labelenumi{(\alph{enumi})}
\setcounter{enumi}{2}
\tightlist
\item
  The graph shows an \textbf{exponential decay} starting at P(0) =
  V\textsubscript{s}²/R and decaying to zero with time constant τ/2
  (half the RC time constant). The power dissipation is highest at t = 0
  when the full supply voltage appears across the resistor, and drops to
  zero as the capacitor charges and the current falls to zero.
\end{enumerate}

\begin{center}\rule{0.5\linewidth}{0.5pt}\end{center}

\section{Problem E.4.3}\label{problem-e.4.3}

\textbf{Given:} A user wants to explore how the quality factor Q affects
the frequency response of a series RLC resonant circuit. The current
notebook has a fixed Q.

\textbf{Find:} (a) How to add a marimo slider for Q, (b) how to connect
it to the graph, and (c) the expected interactive behavior.

\textbf{Solution:}

\begin{enumerate}
\def\labelenumi{(\alph{enumi})}
\tightlist
\item
  Add a slider cell:
\end{enumerate}

\begin{Shaded}
\begin{Highlighting}[]
\AttributeTok{@app.cell}
\KeywordTok{def}\NormalTok{ \_(mo):}
\NormalTok{    Q\_slider }\OperatorTok{=}\NormalTok{ mo.ui.slider(}
\NormalTok{        start}\OperatorTok{=}\DecValTok{1}\NormalTok{, stop}\OperatorTok{=}\DecValTok{100}\NormalTok{, value}\OperatorTok{=}\DecValTok{10}\NormalTok{, step}\OperatorTok{=}\DecValTok{1}\NormalTok{,}
\NormalTok{        label}\OperatorTok{=}\StringTok{"Quality Factor Q"}
\NormalTok{    )}
\NormalTok{    Q\_slider}
    \ControlFlowTok{return}\NormalTok{ Q\_slider,}
\end{Highlighting}
\end{Shaded}

\begin{enumerate}
\def\labelenumi{(\alph{enumi})}
\setcounter{enumi}{1}
\tightlist
\item
  Connect to the computation cell:
\end{enumerate}

\begin{Shaded}
\begin{Highlighting}[]
\AttributeTok{@app.cell}
\KeywordTok{def}\NormalTok{ \_(np, plt, Q\_slider):}
\NormalTok{    Q }\OperatorTok{=}\NormalTok{ Q\_slider.value}
\NormalTok{    f0 }\OperatorTok{=} \DecValTok{1000}  \CommentTok{\# 1 kHz resonant frequency}
\NormalTok{    f }\OperatorTok{=}\NormalTok{ np.logspace(}\DecValTok{1}\NormalTok{, }\DecValTok{5}\NormalTok{, }\DecValTok{1000}\NormalTok{)}
\NormalTok{    H }\OperatorTok{=} \DecValTok{1} \OperatorTok{/}\NormalTok{ np.sqrt((}\DecValTok{1} \OperatorTok{{-}}\NormalTok{ (f}\OperatorTok{/}\NormalTok{f0)}\OperatorTok{**}\DecValTok{2}\NormalTok{)}\OperatorTok{**}\DecValTok{2} \OperatorTok{+}\NormalTok{ (f}\OperatorTok{/}\NormalTok{(Q}\OperatorTok{*}\NormalTok{f0))}\OperatorTok{**}\DecValTok{2}\NormalTok{)}
\NormalTok{    H\_dB }\OperatorTok{=} \DecValTok{20} \OperatorTok{*}\NormalTok{ np.log10(H)}
\NormalTok{    fig, ax }\OperatorTok{=}\NormalTok{ plt.subplots(figsize}\OperatorTok{=}\NormalTok{(}\DecValTok{10}\NormalTok{, }\DecValTok{5}\NormalTok{))}
\NormalTok{    ax.semilogx(f, H\_dB)}
\NormalTok{    ax.set\_xlabel(}\StringTok{"Frequency (Hz)"}\NormalTok{)}
\NormalTok{    ax.set\_ylabel(}\StringTok{"Magnitude (dB)"}\NormalTok{)}
\NormalTok{    ax.set\_title(}\SpecialStringTok{f"RLC Resonance: f₀ = 1 kHz, Q = }\SpecialCharTok{\{}\NormalTok{Q}\SpecialCharTok{\}}\SpecialStringTok{"}\NormalTok{)}
\NormalTok{    ax.grid(}\VariableTok{True}\NormalTok{, which}\OperatorTok{=}\StringTok{\textquotesingle{}both\textquotesingle{}}\NormalTok{, alpha}\OperatorTok{=}\FloatTok{0.3}\NormalTok{)}
\NormalTok{    fig}
    \ControlFlowTok{return}
\end{Highlighting}
\end{Shaded}

\begin{enumerate}
\def\labelenumi{(\alph{enumi})}
\setcounter{enumi}{2}
\tightlist
\item
  When the user moves the slider:
\end{enumerate}

\begin{itemize}
\tightlist
\item
  \textbf{Q = 1}: Broad, flat response with no visible peak (overdamped)
\item
  \textbf{Q = 10}: Moderate peak of +20 dB, bandwidth = f₀/Q = 100 Hz
\item
  \textbf{Q = 100}: Sharp peak of +40 dB, bandwidth = 10 Hz
\end{itemize}

The graph updates \textbf{instantly} as the slider moves, providing
real-time interactive exploration.

\begin{center}\rule{0.5\linewidth}{0.5pt}\end{center}

\section{Problem E.4.4}\label{problem-e.4.4}

\textbf{Given:} A user wants to create a new notebook from scratch for
visualizing transformer turns ratio effects.

\textbf{Find:} (a) The command to create a new empty notebook, (b) the
minimum cells needed, and (c) a complete minimal example.

\textbf{Solution:}

\begin{enumerate}
\def\labelenumi{(\alph{enumi})}
\tightlist
\item
  Create a new notebook:
\end{enumerate}

\begin{verbatim}
marimo edit transformer_demo.py
\end{verbatim}

If the file does not exist, marimo creates a new empty notebook.

\begin{enumerate}
\def\labelenumi{(\alph{enumi})}
\setcounter{enumi}{1}
\tightlist
\item
  Minimum cells:
\end{enumerate}

\begin{enumerate}
\def\labelenumi{\arabic{enumi}.}
\tightlist
\item
  Imports cell
\item
  At least one computation/output cell
\end{enumerate}

\begin{enumerate}
\def\labelenumi{(\alph{enumi})}
\setcounter{enumi}{2}
\tightlist
\item
  Complete minimal example:
\end{enumerate}

\begin{Shaded}
\begin{Highlighting}[]
\ImportTok{import}\NormalTok{ marimo}
\NormalTok{\_\_generated\_with }\OperatorTok{=} \StringTok{"0.19.4"}
\NormalTok{app }\OperatorTok{=}\NormalTok{ marimo.App()}

\AttributeTok{@app.cell}
\KeywordTok{def}\NormalTok{ \_():}
    \ImportTok{import}\NormalTok{ marimo }\ImportTok{as}\NormalTok{ mo}
    \ImportTok{import}\NormalTok{ numpy }\ImportTok{as}\NormalTok{ np}
    \ImportTok{import}\NormalTok{ matplotlib.pyplot }\ImportTok{as}\NormalTok{ plt}
    \ControlFlowTok{return}\NormalTok{ mo, np, plt}

\AttributeTok{@app.cell}
\KeywordTok{def}\NormalTok{ \_(mo):}
\NormalTok{    mo.md(}\StringTok{"\#\# Transformer Turns Ratio"}\NormalTok{)}
    \ControlFlowTok{return}

\AttributeTok{@app.cell}
\KeywordTok{def}\NormalTok{ \_(np, plt):}
\NormalTok{    N1\_N2 }\OperatorTok{=}\NormalTok{ np.linspace(}\FloatTok{0.1}\NormalTok{, }\DecValTok{10}\NormalTok{, }\DecValTok{100}\NormalTok{)}
\NormalTok{    V\_secondary }\OperatorTok{=} \DecValTok{120} \OperatorTok{/}\NormalTok{ N1\_N2  }\CommentTok{\# 120V primary}
\NormalTok{    fig, ax }\OperatorTok{=}\NormalTok{ plt.subplots(figsize}\OperatorTok{=}\NormalTok{(}\DecValTok{10}\NormalTok{, }\DecValTok{5}\NormalTok{))}
\NormalTok{    ax.plot(N1\_N2, V\_secondary)}
\NormalTok{    ax.set\_xlabel(}\StringTok{"Turns Ratio (N₁/N₂)"}\NormalTok{)}
\NormalTok{    ax.set\_ylabel(}\StringTok{"Secondary Voltage (V)"}\NormalTok{)}
\NormalTok{    ax.set\_title(}\StringTok{"Transformer: V₂ = V₁ / (N₁/N₂)"}\NormalTok{)}
\NormalTok{    ax.grid(}\VariableTok{True}\NormalTok{, alpha}\OperatorTok{=}\FloatTok{0.3}\NormalTok{)}
\NormalTok{    ax.axhline(y}\OperatorTok{=}\DecValTok{120}\NormalTok{, color}\OperatorTok{=}\StringTok{\textquotesingle{}r\textquotesingle{}}\NormalTok{, linestyle}\OperatorTok{=}\StringTok{\textquotesingle{}{-}{-}\textquotesingle{}}\NormalTok{, label}\OperatorTok{=}\StringTok{\textquotesingle{}V₁ = 120V\textquotesingle{}}\NormalTok{)}
\NormalTok{    ax.legend()}
\NormalTok{    fig}
    \ControlFlowTok{return}

\ControlFlowTok{if} \VariableTok{\_\_name\_\_} \OperatorTok{==} \StringTok{"\_\_main\_\_"}\NormalTok{:}
\NormalTok{    app.run()}
\end{Highlighting}
\end{Shaded}

\begin{center}\rule{0.5\linewidth}{0.5pt}\end{center}

\section{Problem E.4.5}\label{problem-e.4.5}

\textbf{Given:} A user wants to add a cell that displays a data table
alongside the graph, showing specific calculated values.

\textbf{Find:} (a) How to create a table using marimo's built-in
features, (b) the code for a table showing RC time constant values, and
(c) how to format numbers in the table.

\textbf{Solution:}

\begin{enumerate}
\def\labelenumi{(\alph{enumi})}
\item
  marimo can display tables using \texttt{mo.ui.table()} for interactive
  tables or markdown for static tables.
\item
  Code for a static markdown table:
\end{enumerate}

\begin{Shaded}
\begin{Highlighting}[]
\AttributeTok{@app.cell}
\KeywordTok{def}\NormalTok{ \_(mo):}
\NormalTok{    rows }\OperatorTok{=}\NormalTok{ []}
    \ControlFlowTok{for}\NormalTok{ R }\KeywordTok{in}\NormalTok{ [}\FloatTok{1e3}\NormalTok{, }\FloatTok{4.7e3}\NormalTok{, }\FloatTok{10e3}\NormalTok{, }\FloatTok{47e3}\NormalTok{, }\FloatTok{100e3}\NormalTok{]:}
\NormalTok{        C }\OperatorTok{=} \FloatTok{10e{-}6}
\NormalTok{        tau }\OperatorTok{=}\NormalTok{ R }\OperatorTok{*}\NormalTok{ C}
\NormalTok{        rows.append(}\SpecialStringTok{f"| }\SpecialCharTok{\{}\NormalTok{R}\OperatorTok{/}\FloatTok{1e3}\SpecialCharTok{:.1f\}}\SpecialStringTok{ kΩ | }\SpecialCharTok{\{}\NormalTok{C}\OperatorTok{*}\FloatTok{1e6}\SpecialCharTok{:.0f\}}\SpecialStringTok{ μF | }\SpecialCharTok{\{}\NormalTok{tau}\OperatorTok{*}\FloatTok{1e3}\SpecialCharTok{:.1f\}}\SpecialStringTok{ ms | }\SpecialCharTok{\{}\DecValTok{5}\OperatorTok{*}\NormalTok{tau}\OperatorTok{*}\FloatTok{1e3}\SpecialCharTok{:.0f\}}\SpecialStringTok{ ms |"}\NormalTok{)}

\NormalTok{    table }\OperatorTok{=} \StringTok{"}\CharTok{\textbackslash{}n}\StringTok{"}\NormalTok{.join(rows)}
\NormalTok{    mo.md(}\SpecialStringTok{f"""}
\SpecialStringTok{    \#\# RC Time Constant Reference}

\SpecialStringTok{    | Resistance | Capacitance | τ (ms) | 5τ (ms) |}
\SpecialStringTok{    |{-}{-}{-}{-}{-}{-}{-}{-}{-}{-}{-}|{-}{-}{-}{-}{-}{-}{-}{-}{-}{-}{-}{-}|{-}{-}{-}{-}{-}{-}{-}{-}|{-}{-}{-}{-}{-}{-}{-}{-}{-}|}
\SpecialStringTok{    }\SpecialCharTok{\{}\NormalTok{table}\SpecialCharTok{\}}
\SpecialStringTok{    """}\NormalTok{)}
    \ControlFlowTok{return}
\end{Highlighting}
\end{Shaded}

\begin{enumerate}
\def\labelenumi{(\alph{enumi})}
\setcounter{enumi}{2}
\tightlist
\item
  This produces a formatted table: \textbar{} Resistance \textbar{}
  Capacitance \textbar{} τ (ms) \textbar{} 5τ (ms) \textbar{}
  \textbar-----------\textbar------------\textbar--------\textbar---------\textbar{}
  \textbar{} 1.0 kΩ \textbar{} 10 μF \textbar{} 10.0 ms \textbar{} 50 ms
  \textbar{} \textbar{} 4.7 kΩ \textbar{} 10 μF \textbar{} 47.0 ms
  \textbar{} 235 ms \textbar{} \textbar{} 10.0 kΩ \textbar{} 10 μF
  \textbar{} 100.0 ms \textbar{} 500 ms \textbar{} \textbar{} 47.0 kΩ
  \textbar{} 10 μF \textbar{} 470.0 ms \textbar{} 2350 ms \textbar{}
  \textbar{} 100.0 kΩ \textbar{} 10 μF \textbar{} 1000.0 ms \textbar{}
  5000 ms \textbar{}
\end{enumerate}

\begin{center}\rule{0.5\linewidth}{0.5pt}\end{center}

\section{Problem E.4.6}\label{problem-e.4.6}

\textbf{Given:} A user wants to save a matplotlib figure generated in a
marimo notebook as a high-resolution PNG file for inclusion in a report.

\textbf{Find:} (a) The code to save the figure from within a marimo
cell, (b) the recommended resolution settings, and (c) how to specify
the output path.

\textbf{Solution:}

\begin{enumerate}
\def\labelenumi{(\alph{enumi})}
\tightlist
\item
  Add \texttt{fig.savefig()} before displaying the figure:
\end{enumerate}

\begin{Shaded}
\begin{Highlighting}[]
\AttributeTok{@app.cell}
\KeywordTok{def}\NormalTok{ \_(np, plt):}
\NormalTok{    t }\OperatorTok{=}\NormalTok{ np.linspace(}\DecValTok{0}\NormalTok{, }\DecValTok{1}\NormalTok{, }\DecValTok{1000}\NormalTok{)}
\NormalTok{    y }\OperatorTok{=}\NormalTok{ np.sin(}\DecValTok{2} \OperatorTok{*}\NormalTok{ np.pi }\OperatorTok{*} \DecValTok{5} \OperatorTok{*}\NormalTok{ t)}
\NormalTok{    fig, ax }\OperatorTok{=}\NormalTok{ plt.subplots(figsize}\OperatorTok{=}\NormalTok{(}\DecValTok{10}\NormalTok{, }\DecValTok{5}\NormalTok{))}
\NormalTok{    ax.plot(t, y)}
\NormalTok{    ax.set\_xlabel(}\StringTok{"Time (s)"}\NormalTok{)}
\NormalTok{    ax.set\_ylabel(}\StringTok{"Amplitude"}\NormalTok{)}
\NormalTok{    fig.savefig(}\StringTok{"../images/sine\_wave.png"}\NormalTok{, dpi}\OperatorTok{=}\DecValTok{150}\NormalTok{, bbox\_inches}\OperatorTok{=}\StringTok{"tight"}\NormalTok{,}
\NormalTok{                facecolor}\OperatorTok{=}\StringTok{"white"}\NormalTok{, edgecolor}\OperatorTok{=}\StringTok{"none"}\NormalTok{)}
\NormalTok{    fig}
    \ControlFlowTok{return}
\end{Highlighting}
\end{Shaded}

\begin{enumerate}
\def\labelenumi{(\alph{enumi})}
\setcounter{enumi}{1}
\tightlist
\item
  Recommended settings:
\end{enumerate}

\begin{itemize}
\tightlist
\item
  \textbf{dpi=150}: Good balance between quality and file size (300 dpi
  for print)
\item
  \textbf{bbox\_inches=``tight''}: Removes excess whitespace around the
  plot
\item
  \textbf{facecolor=``white''}: Ensures white background (not
  transparent)
\item
  \textbf{figsize=(10, 5)}: 10 × 5 inches at 150 dpi = 1500 × 750 pixels
\end{itemize}

\begin{enumerate}
\def\labelenumi{(\alph{enumi})}
\setcounter{enumi}{2}
\tightlist
\item
  The path \texttt{"../images/sine\_wave.png"} saves to the
  \texttt{images/} directory relative to the \texttt{scripts/} directory
  where the notebook runs. Use absolute paths if needed:
  \texttt{"/full/path/to/images/sine\_wave.png"}.
\end{enumerate}

\begin{center}\rule{0.5\linewidth}{0.5pt}\end{center}

\section{Problem E.4.7}\label{problem-e.4.7}

\textbf{Given:} A user wants to add a dropdown selector that lets the
viewer choose between different waveform types (sine, square, triangle,
sawtooth) in a signal visualization.

\textbf{Find:} The marimo code for (a) the dropdown UI element, (b) the
waveform generation based on selection, and (c) the graph update.

\textbf{Solution:}

\begin{enumerate}
\def\labelenumi{(\alph{enumi})}
\tightlist
\item
  Dropdown cell:
\end{enumerate}

\begin{Shaded}
\begin{Highlighting}[]
\AttributeTok{@app.cell}
\KeywordTok{def}\NormalTok{ \_(mo):}
\NormalTok{    waveform\_select }\OperatorTok{=}\NormalTok{ mo.ui.dropdown(}
\NormalTok{        options}\OperatorTok{=}\NormalTok{[}\StringTok{"Sine"}\NormalTok{, }\StringTok{"Square"}\NormalTok{, }\StringTok{"Triangle"}\NormalTok{, }\StringTok{"Sawtooth"}\NormalTok{],}
\NormalTok{        value}\OperatorTok{=}\StringTok{"Sine"}\NormalTok{,}
\NormalTok{        label}\OperatorTok{=}\StringTok{"Waveform Type"}
\NormalTok{    )}
\NormalTok{    waveform\_select}
    \ControlFlowTok{return}\NormalTok{ waveform\_select,}
\end{Highlighting}
\end{Shaded}

\begin{enumerate}
\def\labelenumi{(\alph{enumi})}
\setcounter{enumi}{1}
\tightlist
\item
  and (c) Computation and graph cell:
\end{enumerate}

\begin{Shaded}
\begin{Highlighting}[]
\AttributeTok{@app.cell}
\KeywordTok{def}\NormalTok{ \_(np, plt, waveform\_select):}
    \ImportTok{from}\NormalTok{ scipy }\ImportTok{import}\NormalTok{ signal }\ImportTok{as}\NormalTok{ sig}
\NormalTok{    t }\OperatorTok{=}\NormalTok{ np.linspace(}\DecValTok{0}\NormalTok{, }\FloatTok{0.01}\NormalTok{, }\DecValTok{1000}\NormalTok{)  }\CommentTok{\# 10 ms}
\NormalTok{    f }\OperatorTok{=} \DecValTok{1000}  \CommentTok{\# 1 kHz}
\NormalTok{    choice }\OperatorTok{=}\NormalTok{ waveform\_select.value}
    \ControlFlowTok{if}\NormalTok{ choice }\OperatorTok{==} \StringTok{"Sine"}\NormalTok{:}
\NormalTok{        y }\OperatorTok{=}\NormalTok{ np.sin(}\DecValTok{2} \OperatorTok{*}\NormalTok{ np.pi }\OperatorTok{*}\NormalTok{ f }\OperatorTok{*}\NormalTok{ t)}
    \ControlFlowTok{elif}\NormalTok{ choice }\OperatorTok{==} \StringTok{"Square"}\NormalTok{:}
\NormalTok{        y }\OperatorTok{=}\NormalTok{ sig.square(}\DecValTok{2} \OperatorTok{*}\NormalTok{ np.pi }\OperatorTok{*}\NormalTok{ f }\OperatorTok{*}\NormalTok{ t)}
    \ControlFlowTok{elif}\NormalTok{ choice }\OperatorTok{==} \StringTok{"Triangle"}\NormalTok{:}
\NormalTok{        y }\OperatorTok{=}\NormalTok{ sig.sawtooth(}\DecValTok{2} \OperatorTok{*}\NormalTok{ np.pi }\OperatorTok{*}\NormalTok{ f }\OperatorTok{*}\NormalTok{ t, width}\OperatorTok{=}\FloatTok{0.5}\NormalTok{)}
    \ControlFlowTok{elif}\NormalTok{ choice }\OperatorTok{==} \StringTok{"Sawtooth"}\NormalTok{:}
\NormalTok{        y }\OperatorTok{=}\NormalTok{ sig.sawtooth(}\DecValTok{2} \OperatorTok{*}\NormalTok{ np.pi }\OperatorTok{*}\NormalTok{ f }\OperatorTok{*}\NormalTok{ t)}
\NormalTok{    fig, ax }\OperatorTok{=}\NormalTok{ plt.subplots(figsize}\OperatorTok{=}\NormalTok{(}\DecValTok{10}\NormalTok{, }\DecValTok{4}\NormalTok{))}
\NormalTok{    ax.plot(t }\OperatorTok{*} \FloatTok{1e3}\NormalTok{, y, linewidth}\OperatorTok{=}\FloatTok{1.5}\NormalTok{)}
\NormalTok{    ax.set\_xlabel(}\StringTok{"Time (ms)"}\NormalTok{)}
\NormalTok{    ax.set\_ylabel(}\StringTok{"Amplitude"}\NormalTok{)}
\NormalTok{    ax.set\_title(}\SpecialStringTok{f"}\SpecialCharTok{\{}\NormalTok{choice}\SpecialCharTok{\}}\SpecialStringTok{ Wave at }\SpecialCharTok{\{}\NormalTok{f}\SpecialCharTok{\}}\SpecialStringTok{ Hz"}\NormalTok{)}
\NormalTok{    ax.grid(}\VariableTok{True}\NormalTok{, alpha}\OperatorTok{=}\FloatTok{0.3}\NormalTok{)}
\NormalTok{    ax.set\_ylim(}\OperatorTok{{-}}\FloatTok{1.5}\NormalTok{, }\FloatTok{1.5}\NormalTok{)}
\NormalTok{    fig}
    \ControlFlowTok{return}
\end{Highlighting}
\end{Shaded}

When the dropdown selection changes, the graph \textbf{reactively
updates} to show the selected waveform.

\begin{center}\rule{0.5\linewidth}{0.5pt}\end{center}

\section{Problem E.4.8}\label{problem-e.4.8}

\textbf{Given:} A user has created a useful plotting utility function
and wants to reuse it across multiple cells in the same notebook.

\textbf{Find:} (a) How to define a reusable function in one cell and use
it in others, (b) the code pattern, and (c) any limitations.

\textbf{Solution:}

\begin{enumerate}
\def\labelenumi{(\alph{enumi})}
\item
  Define the function in a cell and return it, just like any other
  variable.
\item
  Code pattern:
\end{enumerate}

\begin{Shaded}
\begin{Highlighting}[]
\CommentTok{\# Utility cell}
\AttributeTok{@app.cell}
\KeywordTok{def}\NormalTok{ \_(plt):}
    \KeywordTok{def}\NormalTok{ make\_bode\_plot(f, H\_dB, title}\OperatorTok{=}\StringTok{"Bode Plot"}\NormalTok{):}
\NormalTok{        fig, ax }\OperatorTok{=}\NormalTok{ plt.subplots(figsize}\OperatorTok{=}\NormalTok{(}\DecValTok{10}\NormalTok{, }\DecValTok{5}\NormalTok{))}
\NormalTok{        ax.semilogx(f, H\_dB, linewidth}\OperatorTok{=}\DecValTok{2}\NormalTok{)}
\NormalTok{        ax.set\_xlabel(}\StringTok{"Frequency (Hz)"}\NormalTok{)}
\NormalTok{        ax.set\_ylabel(}\StringTok{"Magnitude (dB)"}\NormalTok{)}
\NormalTok{        ax.set\_title(title)}
\NormalTok{        ax.grid(}\VariableTok{True}\NormalTok{, which}\OperatorTok{=}\StringTok{\textquotesingle{}both\textquotesingle{}}\NormalTok{, alpha}\OperatorTok{=}\FloatTok{0.3}\NormalTok{)}
\NormalTok{        ax.set\_ylim(}\OperatorTok{{-}}\DecValTok{60}\NormalTok{, }\DecValTok{10}\NormalTok{)}
        \ControlFlowTok{return}\NormalTok{ fig}
    \ControlFlowTok{return}\NormalTok{ make\_bode\_plot,}

\CommentTok{\# Usage cell 1}
\AttributeTok{@app.cell}
\KeywordTok{def}\NormalTok{ \_(np, make\_bode\_plot):}
\NormalTok{    f }\OperatorTok{=}\NormalTok{ np.logspace(}\DecValTok{1}\NormalTok{, }\DecValTok{6}\NormalTok{, }\DecValTok{1000}\NormalTok{)}
\NormalTok{    H }\OperatorTok{=} \DecValTok{20} \OperatorTok{*}\NormalTok{ np.log10(}\DecValTok{1} \OperatorTok{/}\NormalTok{ np.sqrt(}\DecValTok{1} \OperatorTok{+}\NormalTok{ (f}\OperatorTok{/}\FloatTok{1e3}\NormalTok{)}\OperatorTok{**}\DecValTok{2}\NormalTok{))}
\NormalTok{    fig }\OperatorTok{=}\NormalTok{ make\_bode\_plot(f, H, }\StringTok{"1 kHz Low{-}Pass Filter"}\NormalTok{)}
\NormalTok{    fig}
    \ControlFlowTok{return}

\CommentTok{\# Usage cell 2}
\AttributeTok{@app.cell}
\KeywordTok{def}\NormalTok{ \_(np, make\_bode\_plot):}
\NormalTok{    f }\OperatorTok{=}\NormalTok{ np.logspace(}\DecValTok{1}\NormalTok{, }\DecValTok{6}\NormalTok{, }\DecValTok{1000}\NormalTok{)}
\NormalTok{    H }\OperatorTok{=} \DecValTok{20} \OperatorTok{*}\NormalTok{ np.log10((f}\OperatorTok{/}\FloatTok{1e3}\NormalTok{) }\OperatorTok{/}\NormalTok{ np.sqrt(}\DecValTok{1} \OperatorTok{+}\NormalTok{ (f}\OperatorTok{/}\FloatTok{1e3}\NormalTok{)}\OperatorTok{**}\DecValTok{2}\NormalTok{))}
\NormalTok{    fig }\OperatorTok{=}\NormalTok{ make\_bode\_plot(f, H, }\StringTok{"1 kHz High{-}Pass Filter"}\NormalTok{)}
\NormalTok{    fig}
    \ControlFlowTok{return}
\end{Highlighting}
\end{Shaded}

\begin{enumerate}
\def\labelenumi{(\alph{enumi})}
\setcounter{enumi}{2}
\tightlist
\item
  Limitations: The function is only available within this notebook. For
  cross-notebook reuse, move the function to a separate Python module
  and import it.
\end{enumerate}

\begin{center}\rule{0.5\linewidth}{0.5pt}\end{center}

\section{Problem E.4.9}\label{problem-e.4.9}

\textbf{Given:} A user wants to add error handling to a cell that reads
data from an external CSV file, displaying a friendly message if the
file is not found.

\textbf{Find:} (a) The code pattern for error handling in a marimo cell,
(b) how to display the error in the notebook, and (c) how to provide a
fallback.

\textbf{Solution:}

\begin{enumerate}
\def\labelenumi{(\alph{enumi})}
\tightlist
\item
  and (b) Use try/except with \texttt{mo.md()} for error display:
\end{enumerate}

\begin{Shaded}
\begin{Highlighting}[]
\AttributeTok{@app.cell}
\KeywordTok{def}\NormalTok{ \_(mo, np, plt):}
    \ControlFlowTok{try}\NormalTok{:}
\NormalTok{        data }\OperatorTok{=}\NormalTok{ np.loadtxt(}\StringTok{"measurements.csv"}\NormalTok{, delimiter}\OperatorTok{=}\StringTok{","}\NormalTok{, skiprows}\OperatorTok{=}\DecValTok{1}\NormalTok{)}
\NormalTok{        t }\OperatorTok{=}\NormalTok{ data[:, }\DecValTok{0}\NormalTok{]}
\NormalTok{        v }\OperatorTok{=}\NormalTok{ data[:, }\DecValTok{1}\NormalTok{]}
\NormalTok{        fig, ax }\OperatorTok{=}\NormalTok{ plt.subplots(figsize}\OperatorTok{=}\NormalTok{(}\DecValTok{10}\NormalTok{, }\DecValTok{5}\NormalTok{))}
\NormalTok{        ax.plot(t, v)}
\NormalTok{        ax.set\_xlabel(}\StringTok{"Time (s)"}\NormalTok{)}
\NormalTok{        ax.set\_ylabel(}\StringTok{"Voltage (V)"}\NormalTok{)}
\NormalTok{        fig}
    \ControlFlowTok{except} \PreprocessorTok{FileNotFoundError}\NormalTok{:}
\NormalTok{        mo.md(}\StringTok{"""}
\StringTok{        ⚠ **File not found:** \textasciigrave{}measurements.csv\textasciigrave{}}

\StringTok{        Place the CSV file in the \textasciigrave{}scripts/\textasciigrave{} directory with columns: time, voltage.}
\StringTok{        Using demo data instead.}
\StringTok{        """}\NormalTok{)}
    \ControlFlowTok{return}
\end{Highlighting}
\end{Shaded}

\begin{enumerate}
\def\labelenumi{(\alph{enumi})}
\setcounter{enumi}{2}
\tightlist
\item
  Fallback with synthetic data:
\end{enumerate}

\begin{Shaded}
\begin{Highlighting}[]
\AttributeTok{@app.cell}
\KeywordTok{def}\NormalTok{ \_(mo, np, plt):}
    \ControlFlowTok{try}\NormalTok{:}
\NormalTok{        data }\OperatorTok{=}\NormalTok{ np.loadtxt(}\StringTok{"measurements.csv"}\NormalTok{, delimiter}\OperatorTok{=}\StringTok{","}\NormalTok{, skiprows}\OperatorTok{=}\DecValTok{1}\NormalTok{)}
\NormalTok{        t, v }\OperatorTok{=}\NormalTok{ data[:, }\DecValTok{0}\NormalTok{], data[:, }\DecValTok{1}\NormalTok{]}
\NormalTok{        source }\OperatorTok{=} \StringTok{"Measured Data"}
    \ControlFlowTok{except} \PreprocessorTok{FileNotFoundError}\NormalTok{:}
\NormalTok{        t }\OperatorTok{=}\NormalTok{ np.linspace(}\DecValTok{0}\NormalTok{, }\DecValTok{1}\NormalTok{, }\DecValTok{500}\NormalTok{)}
\NormalTok{        v }\OperatorTok{=} \FloatTok{3.3} \OperatorTok{*}\NormalTok{ (}\DecValTok{1} \OperatorTok{{-}}\NormalTok{ np.exp(}\OperatorTok{{-}}\NormalTok{t }\OperatorTok{/} \FloatTok{0.1}\NormalTok{)) }\OperatorTok{+}\NormalTok{ np.random.normal(}\DecValTok{0}\NormalTok{, }\FloatTok{0.05}\NormalTok{, }\DecValTok{500}\NormalTok{)}
\NormalTok{        source }\OperatorTok{=} \StringTok{"Simulated Data (file not found)"}
\NormalTok{    fig, ax }\OperatorTok{=}\NormalTok{ plt.subplots(figsize}\OperatorTok{=}\NormalTok{(}\DecValTok{10}\NormalTok{, }\DecValTok{5}\NormalTok{))}
\NormalTok{    ax.plot(t, v)}
\NormalTok{    ax.set\_title(source)}
\NormalTok{    ax.set\_xlabel(}\StringTok{"Time (s)"}\NormalTok{)}
\NormalTok{    ax.set\_ylabel(}\StringTok{"Voltage (V)"}\NormalTok{)}
\NormalTok{    fig}
    \ControlFlowTok{return}
\end{Highlighting}
\end{Shaded}

The notebook gracefully handles the missing file by substituting
simulated data.

\begin{center}\rule{0.5\linewidth}{0.5pt}\end{center}

\section{Problem E.4.10}\label{problem-e.4.10}

\textbf{Given:} A user wants to create a side-by-side comparison of two
graphs in a single cell --- one showing the time-domain waveform and one
showing the frequency spectrum.

\textbf{Find:} (a) The matplotlib code for subplots, (b) a complete cell
implementation, and (c) how to adjust relative subplot sizes.

\textbf{Solution:}

\begin{enumerate}
\def\labelenumi{(\alph{enumi})}
\item
  Use \texttt{plt.subplots(1,\ 2)} for horizontal side-by-side layout.
\item
  Complete cell:
\end{enumerate}

\begin{Shaded}
\begin{Highlighting}[]
\AttributeTok{@app.cell}
\KeywordTok{def}\NormalTok{ \_(np, plt):}
\NormalTok{    fs }\OperatorTok{=} \DecValTok{10000}  \CommentTok{\# 10 kHz sample rate}
\NormalTok{    t }\OperatorTok{=}\NormalTok{ np.arange(}\DecValTok{0}\NormalTok{, }\FloatTok{0.1}\NormalTok{, }\DecValTok{1}\OperatorTok{/}\NormalTok{fs)  }\CommentTok{\# 100 ms}
\NormalTok{    f\_sig }\OperatorTok{=} \DecValTok{200}  \CommentTok{\# 200 Hz signal}
\NormalTok{    x }\OperatorTok{=}\NormalTok{ np.sin(}\DecValTok{2} \OperatorTok{*}\NormalTok{ np.pi }\OperatorTok{*}\NormalTok{ f\_sig }\OperatorTok{*}\NormalTok{ t) }\OperatorTok{+} \FloatTok{0.3} \OperatorTok{*}\NormalTok{ np.sin(}\DecValTok{2} \OperatorTok{*}\NormalTok{ np.pi }\OperatorTok{*} \DecValTok{3} \OperatorTok{*}\NormalTok{ f\_sig }\OperatorTok{*}\NormalTok{ t)}

    \CommentTok{\# Compute FFT}
\NormalTok{    N }\OperatorTok{=} \BuiltInTok{len}\NormalTok{(t)}
\NormalTok{    X }\OperatorTok{=}\NormalTok{ np.fft.rfft(x)}
\NormalTok{    freqs }\OperatorTok{=}\NormalTok{ np.fft.rfftfreq(N, }\DecValTok{1}\OperatorTok{/}\NormalTok{fs)}
\NormalTok{    magnitude }\OperatorTok{=} \DecValTok{2} \OperatorTok{*}\NormalTok{ np.}\BuiltInTok{abs}\NormalTok{(X) }\OperatorTok{/}\NormalTok{ N}

\NormalTok{    fig, (ax1, ax2) }\OperatorTok{=}\NormalTok{ plt.subplots(}\DecValTok{1}\NormalTok{, }\DecValTok{2}\NormalTok{, figsize}\OperatorTok{=}\NormalTok{(}\DecValTok{14}\NormalTok{, }\DecValTok{5}\NormalTok{))}

    \CommentTok{\# Time domain}
\NormalTok{    ax1.plot(t }\OperatorTok{*} \FloatTok{1e3}\NormalTok{, x, linewidth}\OperatorTok{=}\DecValTok{1}\NormalTok{)}
\NormalTok{    ax1.set\_xlabel(}\StringTok{"Time (ms)"}\NormalTok{)}
\NormalTok{    ax1.set\_ylabel(}\StringTok{"Amplitude"}\NormalTok{)}
\NormalTok{    ax1.set\_title(}\StringTok{"Time Domain"}\NormalTok{)}
\NormalTok{    ax1.set\_xlim(}\DecValTok{0}\NormalTok{, }\DecValTok{20}\NormalTok{)  }\CommentTok{\# Show 20 ms}
\NormalTok{    ax1.grid(}\VariableTok{True}\NormalTok{, alpha}\OperatorTok{=}\FloatTok{0.3}\NormalTok{)}

    \CommentTok{\# Frequency domain}
\NormalTok{    ax2.stem(freqs, magnitude, linefmt}\OperatorTok{=}\StringTok{\textquotesingle{}C1{-}\textquotesingle{}}\NormalTok{, markerfmt}\OperatorTok{=}\StringTok{\textquotesingle{}C1o\textquotesingle{}}\NormalTok{, basefmt}\OperatorTok{=}\StringTok{\textquotesingle{}k{-}\textquotesingle{}}\NormalTok{)}
\NormalTok{    ax2.set\_xlabel(}\StringTok{"Frequency (Hz)"}\NormalTok{)}
\NormalTok{    ax2.set\_ylabel(}\StringTok{"Magnitude"}\NormalTok{)}
\NormalTok{    ax2.set\_title(}\StringTok{"Frequency Spectrum (FFT)"}\NormalTok{)}
\NormalTok{    ax2.set\_xlim(}\DecValTok{0}\NormalTok{, }\DecValTok{1000}\NormalTok{)}
\NormalTok{    ax2.grid(}\VariableTok{True}\NormalTok{, alpha}\OperatorTok{=}\FloatTok{0.3}\NormalTok{)}

\NormalTok{    fig.tight\_layout()}
\NormalTok{    fig}
    \ControlFlowTok{return}
\end{Highlighting}
\end{Shaded}

\begin{enumerate}
\def\labelenumi{(\alph{enumi})}
\setcounter{enumi}{2}
\tightlist
\item
  For unequal subplot widths, use \texttt{gridspec\_kw}:
\end{enumerate}

\begin{Shaded}
\begin{Highlighting}[]
\NormalTok{fig, (ax1, ax2) }\OperatorTok{=}\NormalTok{ plt.subplots(}\DecValTok{1}\NormalTok{, }\DecValTok{2}\NormalTok{, figsize}\OperatorTok{=}\NormalTok{(}\DecValTok{14}\NormalTok{, }\DecValTok{5}\NormalTok{),}
\NormalTok{                                 gridspec\_kw}\OperatorTok{=}\NormalTok{\{}\StringTok{\textquotesingle{}width\_ratios\textquotesingle{}}\NormalTok{: [}\DecValTok{2}\NormalTok{, }\DecValTok{1}\NormalTok{]\})}
\end{Highlighting}
\end{Shaded}

This makes the left plot twice as wide as the right plot.

\chapter{Appendix F --- Section F.1: Matrix
Fundamentals}\label{appendix-f-section-f.1-matrix-fundamentals}

Practice problems covering matrix definitions, notation, special
matrices, and construction of matrices from circuit analysis.

\begin{center}\rule{0.5\linewidth}{0.5pt}\end{center}

\section{Problem F.1.1}\label{problem-f.1.1}

\textbf{Given:} A 3 × 3 matrix A = {[}4, −1, 0; −1, 4, −1; 0, −1, 4{]}.

\textbf{Find:} (a) The element a₂₃, (b) the main diagonal elements, (c)
whether the matrix is symmetric, and (d) the trace (sum of diagonal
elements).

\textbf{Solution:}

\begin{enumerate}
\def\labelenumi{(\alph{enumi})}
\item
  Element a₂₃ is in row 2, column 3: a₂₃ = \textbf{−1}
\item
  Main diagonal elements (a₁₁, a₂₂, a₃₃): a₁₁ = 4, a₂₂ = 4, a₃₃ = 4 →
  diagonal = \textbf{\{4, 4, 4\}}
\item
  Check symmetry (A = A\textsuperscript{T}): a₁₂ = −1 = a₂₁ ✓, a₁₃ = 0 =
  a₃₁ ✓, a₂₃ = −1 = a₃₂ ✓ The matrix \textbf{is symmetric}. This is
  typical of admittance/impedance matrices in reciprocal circuits.
\item
  Trace: tr(A) = 4 + 4 + 4 = \textbf{12}
\end{enumerate}

\begin{center}\rule{0.5\linewidth}{0.5pt}\end{center}

\section{Problem F.1.2}\label{problem-f.1.2}

\textbf{Given:} A circuit has four nodes. Conductances connecting node
pairs: G₁₂ = 0.1 S, G₁₃ = 0.2 S, G₂₃ = 0.05 S, G₂₄ = 0.1 S, G₃₄ = 0.15
S. A 0.5 S conductance connects node 1 to ground.

\textbf{Find:} The 4 × 4 nodal admittance matrix Y.

\textbf{Solution:}

Diagonal entries (sum of conductances at each node): Y₁₁ = 0.5 + 0.1 +
0.2 = \textbf{0.8 S} Y₂₂ = 0.1 + 0.05 + 0.1 = \textbf{0.25 S} Y₃₃ = 0.2
+ 0.05 + 0.15 = \textbf{0.4 S} Y₄₄ = 0.1 + 0.15 = \textbf{0.25 S}

Off-diagonal entries (negative of mutual conductance): Y₁₂ = Y₂₁ = −0.1,
Y₁₃ = Y₃₁ = −0.2, Y₁₄ = Y₄₁ = 0 Y₂₃ = Y₃₂ = −0.05, Y₂₄ = Y₄₂ = −0.1 Y₃₄
= Y₄₃ = −0.15

\textbf{Y = {[}0.8, −0.1, −0.2, 0; −0.1, 0.25, −0.05, −0.1; −0.2, −0.05,
0.4, −0.15; 0, −0.1, −0.15, 0.25{]}}

The matrix is symmetric (reciprocal network) and sparse (Y₁₄ = 0 because
nodes 1 and 4 are not directly connected).

\begin{center}\rule{0.5\linewidth}{0.5pt}\end{center}

\section{Problem F.1.3}\label{problem-f.1.3}

\textbf{Given:} Two column vectors representing node voltages and
currents in a two-node circuit: V = {[}12, 8{]}\textsuperscript{T} V and
I = {[}0.5, −0.3{]}\textsuperscript{T} A.

\textbf{Find:} (a) The dimensions of each vector, (b) the total power
delivered by the sources (P = V\textsuperscript{T}I), and (c) the
physical meaning of the negative current at node 2.

\textbf{Solution:}

\begin{enumerate}
\def\labelenumi{(\alph{enumi})}
\item
  Both V and I are \textbf{2 × 1} column vectors (2 rows, 1 column).
\item
  Total power: P = V\textsuperscript{T}I = {[}12, 8{]} × {[}0.5,
  −0.3{]}\textsuperscript{T} = 12 × 0.5 + 8 × (−0.3) = 6.0 − 2.4 =
  \textbf{3.6 W}
\item
  The negative current at node 2 (I₂ = −0.3 A) means that 0.3 A is
  \textbf{leaving} node 2 (current is being extracted, not injected).
  This could represent a load or a current source oriented away from the
  node.
\end{enumerate}

\begin{center}\rule{0.5\linewidth}{0.5pt}\end{center}

\section{Problem F.1.4}\label{problem-f.1.4}

\textbf{Given:} The identity matrix I₃ (3 × 3) and a vector x = {[}5,
−2, 7{]}\textsuperscript{T}.

\textbf{Find:} (a) Write out I₃ explicitly, (b) compute I₃ × x, and (c)
explain why the result equals x.

\textbf{Solution:}

\begin{enumerate}
\def\labelenumi{(\alph{enumi})}
\item
  I₃ = \textbf{{[}1, 0, 0; 0, 1, 0; 0, 0, 1{]}}
\item
  I₃ × x: Row 1: 1×5 + 0×(−2) + 0×7 = 5 Row 2: 0×5 + 1×(−2) + 0×7 = −2
  Row 3: 0×5 + 0×(−2) + 1×7 = 7 Result: \textbf{{[}5, −2,
  7{]}\textsuperscript{T} = x}
\item
  Multiplying by the identity matrix leaves any vector (or matrix)
  unchanged. This is the matrix equivalent of multiplying a number by 1.
  In circuit terms, the identity matrix represents a direct connection
  --- the output equals the input with no transformation.
\end{enumerate}

\begin{center}\rule{0.5\linewidth}{0.5pt}\end{center}

\section{Problem F.1.5}\label{problem-f.1.5}

\textbf{Given:} A diagonal matrix D = diag(100, 200, 50) represents the
impedance (in ohms) of three independent branches, and I = {[}0.1, 0.05,
0.2{]}\textsuperscript{T} A are the branch currents.

\textbf{Find:} (a) Write D in full matrix form, (b) compute V = D × I,
and (c) the total power dissipated.

\textbf{Solution:}

\begin{enumerate}
\def\labelenumi{(\alph{enumi})}
\item
  D = \textbf{{[}100, 0, 0; 0, 200, 0; 0, 0, 50{]}} Ω
\item
  V = D × I: V₁ = 100 × 0.1 = 10 V V₂ = 200 × 0.05 = 10 V V₃ = 50 × 0.2
  = 10 V V = \textbf{{[}10, 10, 10{]}\textsuperscript{T} V}
\end{enumerate}

All branches have the same voltage (10 V) despite different impedances
and currents.

\begin{enumerate}
\def\labelenumi{(\alph{enumi})}
\setcounter{enumi}{2}
\tightlist
\item
  Total power: P = V₁I₁ + V₂I₂ + V₃I₃ = 10 × 0.1 + 10 × 0.05 + 10 × 0.2
  = 1.0 + 0.5 + 2.0 = \textbf{3.5 W}
\end{enumerate}

\begin{center}\rule{0.5\linewidth}{0.5pt}\end{center}

\section{Problem F.1.6}\label{problem-f.1.6}

\textbf{Given:} A mesh impedance matrix for a three-mesh circuit: Z =
{[}R₁ + R₂, −R₂, 0; −R₂, R₂ + R₃ + R₄, −R₄; 0, −R₄, R₄ + R₅{]} with R₁ =
10 Ω, R₂ = 20 Ω, R₃ = 15 Ω, R₄ = 30 Ω, R₅ = 25 Ω.

\textbf{Find:} (a) The numerical Z matrix, (b) verify it is symmetric,
and (c) explain why Z₁₃ = 0.

\textbf{Solution:}

\begin{enumerate}
\def\labelenumi{(\alph{enumi})}
\tightlist
\item
  Numerical values: Z₁₁ = 10 + 20 = 30 Ω Z₂₂ = 20 + 15 + 30 = 65 Ω Z₃₃ =
  30 + 25 = 55 Ω Z₁₂ = Z₂₁ = −20 Ω Z₂₃ = Z₃₂ = −30 Ω Z₁₃ = Z₃₁ = 0 Ω
\end{enumerate}

\textbf{Z = {[}30, −20, 0; −20, 65, −30; 0, −30, 55{]} Ω}

\begin{enumerate}
\def\labelenumi{(\alph{enumi})}
\setcounter{enumi}{1}
\item
  Z₁₂ = Z₂₁ = −20 ✓, Z₁₃ = Z₃₁ = 0 ✓, Z₂₃ = Z₃₂ = −30 ✓ The matrix
  \textbf{is symmetric}, as expected for a reciprocal passive network.
\item
  Z₁₃ = 0 because meshes 1 and 3 \textbf{share no common element}. R₂ is
  shared between meshes 1 and 2, and R₄ is shared between meshes 2 and
  3, but no resistor appears in both mesh 1 and mesh 3.
\end{enumerate}

\begin{center}\rule{0.5\linewidth}{0.5pt}\end{center}

\section{Problem F.1.7}\label{problem-f.1.7}

\textbf{Given:} A sparse bus admittance matrix for a 5-bus power system
where only adjacent buses are connected:

Bus connections: 1-2 (y = 0.5 − j2.0), 2-3 (y = 0.3 − j1.5), 3-4 (y =
0.4 − j1.8), 4-5 (y = 0.2 − j1.0). Each bus has a shunt admittance to
ground of j0.05.

\textbf{Find:} (a) The diagonal element Y₃₃, (b) the number of non-zero
elements in the 5 × 5 matrix, and (c) the sparsity (percentage of zero
elements).

\textbf{Solution:}

\begin{enumerate}
\def\labelenumi{(\alph{enumi})}
\item
  Y₃₃ = sum of all admittances at bus 3: Y₃₃ = y₂₃ + y₃₄ + y₃₀ = (0.3 −
  j1.5) + (0.4 − j1.8) + j0.05 = \textbf{0.7 − j3.25 S}
\item
  Non-zero elements:
\end{enumerate}

\begin{itemize}
\tightlist
\item
  5 diagonal elements (Y₁₁ through Y₅₅)
\item
  4 connections × 2 (symmetric upper and lower triangle) = 8
  off-diagonal elements Total non-zero: 5 + 8 = \textbf{13 elements}
\end{itemize}

\begin{enumerate}
\def\labelenumi{(\alph{enumi})}
\setcounter{enumi}{2}
\tightlist
\item
  Total elements in 5 × 5 matrix: 25. Zero elements: 25 − 13 = 12.
  Sparsity = 12/25 = \textbf{48\%}
\end{enumerate}

For larger power systems (hundreds of buses), sparsity exceeds 95\%,
making sparse matrix techniques essential.

\begin{center}\rule{0.5\linewidth}{0.5pt}\end{center}

\section{Problem F.1.8}\label{problem-f.1.8}

\textbf{Given:} A matrix represents the voltage gains of a three-stage
amplifier cascade. Each stage is characterized by a 2 × 2 matrix. Stage
1: {[}A₁, B₁; C₁, D₁{]} = {[}5, 0; 0, 1{]}, Stage 2: {[}1, 50; 0, 1{]},
Stage 3: {[}10, 0; 0, 1{]}.

\textbf{Find:} (a) What type of matrix Stage 1 represents, (b) what type
Stage 2 represents, and (c) whether multiplication order matters for
cascading.

\textbf{Solution:}

\begin{enumerate}
\def\labelenumi{(\alph{enumi})}
\item
  Stage 1 is a \textbf{diagonal matrix} (non-zero only on diagonal). It
  represents a voltage amplifier with gain A = 5 and no series
  impedance. This is an ideal voltage amplifier.
\item
  Stage 2 has ones on the diagonal and a non-zero B parameter. This
  represents a \textbf{series impedance} of 50 Ω (an ABCD representation
  where A = D = 1, B = Z, C = 0). It is a series element inserted in the
  signal path.
\item
  Matrix multiplication is \textbf{not commutative} (AB ≠ BA in
  general), so the cascade order matters.
\end{enumerate}

M = M₁ × M₂ × M₃ (signal flows left to right)

If the order were reversed (M₃ × M₂ × M₁), the result would be
different. Physically, the signal encounters Stage 1 first, then Stage
2, then Stage 3 --- this order must be preserved in the matrix product.

\begin{center}\rule{0.5\linewidth}{0.5pt}\end{center}

\section{Problem F.1.9}\label{problem-f.1.9}

\textbf{Given:} A matrix A = {[}3, 1; 1, 3{]} represents the admittance
matrix of a two-port network.

\textbf{Find:} (a) Whether A is symmetric, (b) the eigenvalues of A, and
(c) the physical meaning of the eigenvalues.

\textbf{Solution:}

\begin{enumerate}
\def\labelenumi{(\alph{enumi})}
\item
  a₁₂ = 1 = a₂₁, so the matrix \textbf{is symmetric}. This indicates a
  reciprocal two-port network.
\item
  Eigenvalues from det(A − λI) = 0: (3 − λ)(3 − λ) − (1)(1) = 0 (3 − λ)²
  − 1 = 0 (3 − λ)² = 1 3 − λ = ±1 λ₁ = 3 − 1 = \textbf{2}, λ₂ = 3 + 1 =
  \textbf{4}
\item
  The eigenvalues represent the \textbf{natural modes} of the network:
\end{enumerate}

\begin{itemize}
\tightlist
\item
  λ₁ = 2 S: The even mode (both ports driven in phase) --- the effective
  admittance is 2 S
\item
  λ₂ = 4 S: The odd mode (ports driven out of phase) --- the effective
  admittance is 4 S
\end{itemize}

The eigenvectors are {[}1, 1{]}\textsuperscript{T} (even mode) and {[}1,
−1{]}\textsuperscript{T} (odd mode). These modes decouple the two-port
into two independent one-port problems.

\begin{center}\rule{0.5\linewidth}{0.5pt}\end{center}

\section{Problem F.1.10}\label{problem-f.1.10}

\textbf{Given:} A 3 × 2 matrix represents the incidence matrix of a
circuit graph with 3 nodes and 2 branches: A = {[}1, 0; −1, 1; 0, −1{]}

\textbf{Find:} (a) The dimensions and their meaning, (b) what the
entries tell us about the circuit topology, and (c) the product
A\textsuperscript{T}A and its significance.

\textbf{Solution:}

\begin{enumerate}
\def\labelenumi{(\alph{enumi})}
\item
  A is \textbf{3 × 2} --- 3 rows (nodes) and 2 columns (branches). Each
  column represents a branch, and each row represents a node.
\item
  The entries:
\end{enumerate}

\begin{itemize}
\tightlist
\item
  Branch 1 (column 1): +1 at node 1, −1 at node 2 → branch 1 goes
  \textbf{from node 1 to node 2}
\item
  Branch 2 (column 2): +1 at node 2, −1 at node 3 → branch 2 goes
  \textbf{from node 2 to node 3}
\end{itemize}

Each branch has exactly one +1 (start node) and one −1 (end node),
indicating the assumed current direction.

\begin{enumerate}
\def\labelenumi{(\alph{enumi})}
\setcounter{enumi}{2}
\tightlist
\item
  A\textsuperscript{T}A: A\textsuperscript{T} = {[}1, −1, 0; 0, 1, −1{]}
  A\textsuperscript{T}A = {[}1×1+(−1)(−1)+0, 1×0+(−1)(1)+0; 0+(1)(−1)+0,
  0+1×1+(−1)(−1){]} = \textbf{{[}2, −1; −1, 2{]}}
\end{enumerate}

This is the \textbf{graph Laplacian} of the circuit. In circuit
analysis, if each branch has unit conductance, then
A\textsuperscript{T}A is the nodal admittance matrix (with the reference
node eliminated). The diagonal entries count the number of branches at
each node, and the off-diagonal entries are −1 for connected node pairs.

\chapter{Appendix F --- Section F.2: Matrix
Arithmetic}\label{appendix-f-section-f.2-matrix-arithmetic}

Practice problems covering matrix addition, subtraction, scalar
multiplication, and matrix multiplication.

\begin{center}\rule{0.5\linewidth}{0.5pt}\end{center}

\section{Problem F.2.1}\label{problem-f.2.1}

\textbf{Given:} Two admittance matrices from independent source
contributions in a three-node circuit: Y₁ = {[}0.4, −0.1, −0.2; −0.1,
0.3, −0.1; −0.2, −0.1, 0.5{]} S and Y₂ = {[}0.1, 0, −0.1; 0, 0.2, −0.1;
−0.1, −0.1, 0.3{]} S.

\textbf{Find:} The total admittance matrix Y\textsubscript{total} = Y₁ +
Y₂.

\textbf{Solution:}

Adding element by element: Y\textsubscript{total}(1,1) = 0.4 + 0.1 = 0.5
Y\textsubscript{total}(1,2) = −0.1 + 0 = −0.1
Y\textsubscript{total}(1,3) = −0.2 + (−0.1) = −0.3
Y\textsubscript{total}(2,1) = −0.1 + 0 = −0.1
Y\textsubscript{total}(2,2) = 0.3 + 0.2 = 0.5
Y\textsubscript{total}(2,3) = −0.1 + (−0.1) = −0.2
Y\textsubscript{total}(3,1) = −0.2 + (−0.1) = −0.3
Y\textsubscript{total}(3,2) = −0.1 + (−0.1) = −0.2
Y\textsubscript{total}(3,3) = 0.5 + 0.3 = 0.8

\textbf{Y\textsubscript{total} = {[}0.5, −0.1, −0.3; −0.1, 0.5, −0.2;
−0.3, −0.2, 0.8{]} S}

The result is symmetric, confirming both sub-networks are reciprocal.

\begin{center}\rule{0.5\linewidth}{0.5pt}\end{center}

\section{Problem F.2.2}\label{problem-f.2.2}

\textbf{Given:} A current source vector I\textsubscript{sources} = {[}3,
−1, 2{]}\textsuperscript{T} A and a load current vector
I\textsubscript{loads} = {[}1.5, 0.8, 1.2{]}\textsuperscript{T} A at
three nodes.

\textbf{Find:} (a) The net current injection vector I\textsubscript{net}
= I\textsubscript{sources} − I\textsubscript{loads}, and (b) the
physical meaning of each entry.

\textbf{Solution:}

\begin{enumerate}
\def\labelenumi{(\alph{enumi})}
\tightlist
\item
  I\textsubscript{net} = I\textsubscript{sources} −
  I\textsubscript{loads}: I\textsubscript{net}(1) = 3 − 1.5 = 1.5 A
  I\textsubscript{net}(2) = −1 − 0.8 = −1.8 A I\textsubscript{net}(3) =
  2 − 1.2 = 0.8 A
\end{enumerate}

\textbf{I\textsubscript{net} = {[}1.5, −1.8, 0.8{]}\textsuperscript{T}
A}

\begin{enumerate}
\def\labelenumi{(\alph{enumi})}
\setcounter{enumi}{1}
\tightlist
\item
  Physical meaning:
\end{enumerate}

\begin{itemize}
\tightlist
\item
  Node 1: Net \textbf{1.5 A injected} (source exceeds load)
\item
  Node 2: Net \textbf{1.8 A extracted} (load exceeds source; the
  original source was already extracting 1 A)
\item
  Node 3: Net \textbf{0.8 A injected} (source exceeds load)
\end{itemize}

KCL requires the sum of all net injections to equal zero for the
internal nodes, accounting for current through ground connections.

\begin{center}\rule{0.5\linewidth}{0.5pt}\end{center}

\section{Problem F.2.3}\label{problem-f.2.3}

\textbf{Given:} An impedance matrix Z = {[}50, 10; 10, 40{]} Ω is
specified at 60 Hz. The system is being converted to per-unit with a
base impedance Z\textsubscript{base} = 100 Ω.

\textbf{Find:} (a) The per-unit impedance matrix Z\textsubscript{pu},
(b) verify that the per-unit matrix preserves the symmetry, and (c) the
per-unit current if the base power is 1 MVA at 10 kV base voltage.

\textbf{Solution:}

\begin{enumerate}
\def\labelenumi{(\alph{enumi})}
\item
  Per-unit conversion is scalar multiplication by
  1/Z\textsubscript{base}: Z\textsubscript{pu} = Z /
  Z\textsubscript{base} = (1/100) × {[}50, 10; 10, 40{]} =
  \textbf{{[}0.50, 0.10; 0.10, 0.40{]} pu}
\item
  Z\textsubscript{pu}(1,2) = 0.10 = Z\textsubscript{pu}(2,1) ✓.
  \textbf{Symmetry is preserved} because scalar multiplication does not
  affect the relative values of elements.
\item
  Base current: I\textsubscript{base} = S\textsubscript{base} / (√3 ×
  V\textsubscript{base}) = 1 × 10⁶ / (√3 × 10,000) = \textbf{57.74 A}
\end{enumerate}

If a per-unit current is I\textsubscript{pu} = 0.8, the actual current
is 0.8 × 57.74 = \textbf{46.19 A}.

\begin{center}\rule{0.5\linewidth}{0.5pt}\end{center}

\section{Problem F.2.4}\label{problem-f.2.4}

\textbf{Given:} Two matrices: A = {[}2, 3; 1, 4{]} and B = {[}5, 1; 2,
3{]}.

\textbf{Find:} (a) AB, (b) BA, and (c) verify that AB ≠ BA.

\textbf{Solution:}

\begin{enumerate}
\def\labelenumi{(\alph{enumi})}
\item
  AB: (AB)₁₁ = 2×5 + 3×2 = 10 + 6 = 16 (AB)₁₂ = 2×1 + 3×3 = 2 + 9 = 11
  (AB)₂₁ = 1×5 + 4×2 = 5 + 8 = 13 (AB)₂₂ = 1×1 + 4×3 = 1 + 12 = 13
  \textbf{AB = {[}16, 11; 13, 13{]}}
\item
  BA: (BA)₁₁ = 5×2 + 1×1 = 10 + 1 = 11 (BA)₁₂ = 5×3 + 1×4 = 15 + 4 = 19
  (BA)₂₁ = 2×2 + 3×1 = 4 + 3 = 7 (BA)₂₂ = 2×3 + 3×4 = 6 + 12 = 18
  \textbf{BA = {[}11, 19; 7, 18{]}}
\item
  AB = {[}16, 11; 13, 13{]} ≠ BA = {[}11, 19; 7, 18{]}. \textbf{Matrix
  multiplication is not commutative.} In circuit analysis, this means
  the order of cascaded two-port networks matters.
\end{enumerate}

\begin{center}\rule{0.5\linewidth}{0.5pt}\end{center}

\section{Problem F.2.5}\label{problem-f.2.5}

\textbf{Given:} A 3 × 3 impedance matrix Z = {[}20, 5, 0; 5, 15, 5; 0,
5, 25{]} Ω and a current vector I = {[}1, 2, 0.5{]}\textsuperscript{T}
A.

\textbf{Find:} The voltage vector V = Z × I by performing the
matrix-vector multiplication.

\textbf{Solution:}

V₁ = 20×1 + 5×2 + 0×0.5 = 20 + 10 + 0 = \textbf{30 V} V₂ = 5×1 + 15×2 +
5×0.5 = 5 + 30 + 2.5 = \textbf{37.5 V} V₃ = 0×1 + 5×2 + 25×0.5 = 0 + 10
+ 12.5 = \textbf{22.5 V}

\textbf{V = {[}30, 37.5, 22.5{]}\textsuperscript{T} V}

Verification: Total power = V\textsuperscript{T}I = 30×1 + 37.5×2 +
22.5×0.5 = 30 + 75 + 11.25 = 116.25 W.

\begin{center}\rule{0.5\linewidth}{0.5pt}\end{center}

\section{Problem F.2.6}\label{problem-f.2.6}

\textbf{Given:} Two ABCD matrices representing cascaded transmission
line sections: M₁ = {[}cosh(γ₁l₁), Z₀₁ sinh(γ₁l₁); sinh(γ₁l₁)/Z₀₁,
cosh(γ₁l₁){]} with numerical values M₁ = {[}0.95, 25; 0.01, 0.95{]} and
M₂ = {[}0.90, 40; 0.008, 0.90{]}.

\textbf{Find:} The overall ABCD matrix M\textsubscript{total} = M₁ × M₂.

\textbf{Solution:}

A = 0.95×0.90 + 25×0.008 = 0.855 + 0.200 = \textbf{1.055} B = 0.95×40 +
25×0.90 = 38.0 + 22.5 = \textbf{60.5} C = 0.01×0.90 + 0.95×0.008 = 0.009
+ 0.0076 = \textbf{0.0166} D = 0.01×40 + 0.95×0.90 = 0.40 + 0.855 =
\textbf{1.255}

\textbf{M\textsubscript{total} = {[}1.055, 60.5; 0.0166, 1.255{]}}

Verification: det(M\textsubscript{total}) = 1.055 × 1.255 − 60.5 ×
0.0166 = 1.324 − 1.004 = 0.320. Note: det(M₁) = 0.95² − 25 × 0.01 =
0.9025 − 0.25 = 0.6525 and det(M₂) = 0.81 − 0.32 = 0.49.
det(M\textsubscript{total}) = det(M₁) × det(M₂) = 0.6525 × 0.49 = 0.320
✓

\begin{center}\rule{0.5\linewidth}{0.5pt}\end{center}

\section{Problem F.2.7}\label{problem-f.2.7}

\textbf{Given:} A scaling operation on a state-space system. The
original A matrix is: A = {[}−2, 1; 0, −3{]} The system is time-scaled
by a factor of 2 (running twice as fast), which multiplies A by 2.

\textbf{Find:} (a) The scaled A matrix, (b) the eigenvalues of the
original A, and (c) the eigenvalues of the scaled A.

\textbf{Solution:}

\begin{enumerate}
\def\labelenumi{(\alph{enumi})}
\item
  Scaled matrix: A\textsubscript{scaled} = 2A = 2 × {[}−2, 1; 0, −3{]} =
  \textbf{{[}−4, 2; 0, −6{]}}
\item
  Original eigenvalues (A is upper triangular, so eigenvalues are
  diagonal elements): λ₁ = \textbf{−2}, λ₂ = \textbf{−3}
\end{enumerate}

Both poles are in the left half-plane, confirming stability.

\begin{enumerate}
\def\labelenumi{(\alph{enumi})}
\setcounter{enumi}{2}
\tightlist
\item
  Scaled eigenvalues: λ₁\textsubscript{scaled} = 2 × (−2) = \textbf{−4},
  λ₂\textsubscript{scaled} = 2 × (−3) = \textbf{−6}
\end{enumerate}

Scalar multiplication by k multiplies all eigenvalues by k. The
time-scaled system is still stable but responds twice as fast (poles are
farther from the imaginary axis).

\begin{center}\rule{0.5\linewidth}{0.5pt}\end{center}

\section{Problem F.2.8}\label{problem-f.2.8}

\textbf{Given:} A 2 × 3 matrix A = {[}1, 2, 3; 4, 5, 6{]} and a 3 × 2
matrix B = {[}7, 8; 9, 10; 11, 12{]}.

\textbf{Find:} (a) The dimensions of AB, (b) the product AB, (c) the
dimensions of BA, and (d) the product BA.

\textbf{Solution:}

\begin{enumerate}
\def\labelenumi{(\alph{enumi})}
\item
  A is 2 × 3, B is 3 × 2, so AB is \textbf{2 × 2}.
\item
  AB: (AB)₁₁ = 1×7 + 2×9 + 3×11 = 7 + 18 + 33 = 58 (AB)₁₂ = 1×8 + 2×10 +
  3×12 = 8 + 20 + 36 = 64 (AB)₂₁ = 4×7 + 5×9 + 6×11 = 28 + 45 + 66 = 139
  (AB)₂₂ = 4×8 + 5×10 + 6×12 = 32 + 50 + 72 = 154 \textbf{AB = {[}58,
  64; 139, 154{]}}
\item
  B is 3 × 2, A is 2 × 3, so BA is \textbf{3 × 3}.
\item
  BA: Row 1: {[}7×1+8×4, 7×2+8×5, 7×3+8×6{]} = {[}39, 54, 69{]} Row 2:
  {[}9×1+10×4, 9×2+10×5, 9×3+10×6{]} = {[}49, 68, 87{]} Row 3:
  {[}11×1+12×4, 11×2+12×5, 11×3+12×6{]} = {[}59, 82, 105{]} \textbf{BA =
  {[}39, 54, 69; 49, 68, 87; 59, 82, 105{]}}
\end{enumerate}

Note: AB is 2 × 2 while BA is 3 × 3 --- even the dimensions differ when
the matrices are not square.

\begin{center}\rule{0.5\linewidth}{0.5pt}\end{center}

\section{Problem F.2.9}\label{problem-f.2.9}

\textbf{Given:} The mesh impedance equation ZI = V for a two-mesh
circuit with: Z = {[}R₁ + jωL₁, −jωM; −jωM, R₂ + jωL₂{]} R₁ = 10 Ω, R₂ =
20 Ω, ωL₁ = 30 Ω, ωL₂ = 40 Ω, ωM = 15 Ω (mutual inductance). Source
voltage V = {[}100∠0°, 0{]}\textsuperscript{T} V.

\textbf{Find:} (a) The numerical Z matrix (in complex form), and (b) the
product Z × I for I = {[}2 − j1, 1 + j0.5{]}\textsuperscript{T} A (to
check a proposed solution).

\textbf{Solution:}

\begin{enumerate}
\def\labelenumi{(\alph{enumi})}
\tightlist
\item
  Z matrix: Z₁₁ = 10 + j30 Ω Z₁₂ = Z₂₁ = −j15 Ω Z₂₂ = 20 + j40 Ω
\end{enumerate}

\textbf{Z = {[}10 + j30, −j15; −j15, 20 + j40{]} Ω}

\begin{enumerate}
\def\labelenumi{(\alph{enumi})}
\setcounter{enumi}{1}
\tightlist
\item
  ZI: V₁ = (10 + j30)(2 − j1) + (−j15)(1 + j0.5) = (20 − j10 + j60 −
  j²30) + (−j15 − j²7.5) = (20 − j10 + j60 + 30) + (−j15 + 7.5) = (50 +
  j50) + (7.5 − j15) = \textbf{57.5 + j35 V}
\end{enumerate}

V₂ = (−j15)(2 − j1) + (20 + j40)(1 + j0.5) = (−j30 + j²15) + (20 + j10 +
j40 + j²20) = (−15 − j30) + (0 + j50) = \textbf{−15 + j20 V}

Since V₁ should equal 100∠0° = 100 + j0, the proposed solution I = {[}2
− j1, 1 + j0.5{]}\textsuperscript{T} is \textbf{not correct}. The actual
mesh currents would need to be found by solving ZI = V using matrix
inversion.

\begin{center}\rule{0.5\linewidth}{0.5pt}\end{center}

\section{Problem F.2.10}\label{problem-f.2.10}

\textbf{Given:} A discrete-time state update requires the operation
x{[}k+1{]} = Ax{[}k{]} + Bu{[}k{]} where: A = {[}0.9, 0.1; 0, 0.8{]}, B
= {[}0; 1{]}, x{[}0{]} = {[}1, 0{]}\textsuperscript{T}, u{[}0{]} = 0.5.

\textbf{Find:} (a) The state x{[}1{]}, (b) the state x{[}2{]} (with
u{[}1{]} = 0.5), and (c) the steady-state response as k → ∞ (with
constant u = 0.5).

\textbf{Solution:}

\begin{enumerate}
\def\labelenumi{(\alph{enumi})}
\item
  x{[}1{]} = Ax{[}0{]} + Bu{[}0{]}: Ax{[}0{]} = {[}0.9×1+0.1×0,
  0×1+0.8×0{]} = {[}0.9, 0{]} Bu{[}0{]} = {[}0×0.5, 1×0.5{]} = {[}0,
  0.5{]} x{[}1{]} = {[}0.9 + 0, 0 + 0.5{]} = \textbf{{[}0.9,
  0.5{]}\textsuperscript{T}}
\item
  x{[}2{]} = Ax{[}1{]} + Bu{[}1{]}: Ax{[}1{]} = {[}0.9×0.9+0.1×0.5,
  0×0.9+0.8×0.5{]} = {[}0.81+0.05, 0.40{]} = {[}0.86, 0.40{]} Bu{[}1{]}
  = {[}0, 0.5{]} x{[}2{]} = {[}0.86, 0.40+0.50{]} = \textbf{{[}0.86,
  0.90{]}\textsuperscript{T}}
\item
  At steady state, x{[}∞{]} = Ax{[}∞{]} + Bu, so (I − A)x{[}∞{]} = Bu: I
  − A = {[}0.1, −0.1; 0, 0.2{]} det(I − A) = 0.1 × 0.2 − (−0.1)(0) =
  0.02 (I − A)⁻¹ = (1/0.02) × {[}0.2, 0.1; 0, 0.1{]} = {[}10, 5; 0, 5{]}
  x{[}∞{]} = (I − A)⁻¹Bu = {[}10, 5; 0, 5{]} × {[}0,
  0.5{]}\textsuperscript{T} = {[}10×0+5×0.5, 0+5×0.5{]} =
  \textbf{{[}2.5, 2.5{]}\textsuperscript{T}}
\end{enumerate}

\chapter{Appendix F --- Section F.3: Matrix
Operations}\label{appendix-f-section-f.3-matrix-operations}

Practice problems covering the transpose, determinant, and matrix
inverse.

\begin{center}\rule{0.5\linewidth}{0.5pt}\end{center}

\section{Problem F.3.1}\label{problem-f.3.1}

\textbf{Given:} A matrix A = {[}2, 5, 1; 3, 7, 4; 8, 6, 9{]}.

\textbf{Find:} (a) A\textsuperscript{T}, (b) the product
A\textsuperscript{T}A, and (c) verify that A\textsuperscript{T}A is
symmetric.

\textbf{Solution:}

\begin{enumerate}
\def\labelenumi{(\alph{enumi})}
\item
  A\textsuperscript{T} = \textbf{{[}2, 3, 8; 5, 7, 6; 1, 4, 9{]}} (rows
  become columns)
\item
  A\textsuperscript{T}A (3 × 3): (A\textsuperscript{T}A)₁₁ = 2×2+3×3+8×8
  = 4+9+64 = 77 (A\textsuperscript{T}A)₁₂ = 2×5+3×7+8×6 = 10+21+48 = 79
  (A\textsuperscript{T}A)₁₃ = 2×1+3×4+8×9 = 2+12+72 = 86
  (A\textsuperscript{T}A)₂₂ = 5×5+7×7+6×6 = 25+49+36 = 110
  (A\textsuperscript{T}A)₂₃ = 5×1+7×4+6×9 = 5+28+54 = 87
  (A\textsuperscript{T}A)₃₃ = 1×1+4×4+9×9 = 1+16+81 = 98
\end{enumerate}

\textbf{A\textsuperscript{T}A = {[}77, 79, 86; 79, 110, 87; 86, 87,
98{]}}

\begin{enumerate}
\def\labelenumi{(\alph{enumi})}
\setcounter{enumi}{2}
\tightlist
\item
  By construction, (A\textsuperscript{T}A)\textsuperscript{T} =
  A\textsuperscript{T}(A\textsuperscript{T})\textsuperscript{T} =
  A\textsuperscript{T}A. Checking: (A\textsuperscript{T}A)₁₂ = 79 =
  (A\textsuperscript{T}A)₂₁ ✓, (A\textsuperscript{T}A)₁₃ = 86 =
  (A\textsuperscript{T}A)₃₁ ✓, (A\textsuperscript{T}A)₂₃ = 87 =
  (A\textsuperscript{T}A)₃₂ ✓. \textbf{A\textsuperscript{T}A is always
  symmetric} --- this is a fundamental property used in least-squares
  fitting and normal equations.
\end{enumerate}

\begin{center}\rule{0.5\linewidth}{0.5pt}\end{center}

\section{Problem F.3.2}\label{problem-f.3.2}

\textbf{Given:} A 2 × 2 mesh impedance matrix: Z = {[}25, −8; −8, 18{]}
Ω.

\textbf{Find:} (a) The determinant, (b) whether the system has a unique
solution, and (c) the physical meaning of a non-zero determinant.

\textbf{Solution:}

\begin{enumerate}
\def\labelenumi{(\alph{enumi})}
\item
  det(Z) = (25)(18) − (−8)(−8) = 450 − 64 = \textbf{386}
\item
  Since det(Z) = 386 ≠ 0, the system \textbf{has a unique solution}. The
  mesh currents can be determined uniquely for any set of source
  voltages.
\item
  A non-zero determinant means:
\end{enumerate}

\begin{itemize}
\tightlist
\item
  The two mesh equations are \textbf{linearly independent} (neither is a
  multiple of the other)
\item
  The circuit has no redundant or contradictory constraints
\item
  The impedance matrix is \textbf{invertible}, so I = Z⁻¹V can be
  computed
\end{itemize}

If the determinant were zero, it would indicate that the two meshes are
not independent --- for example, if one mesh were completely contained
within another, or if a wire short removed a degree of freedom.

\begin{center}\rule{0.5\linewidth}{0.5pt}\end{center}

\section{Problem F.3.3}\label{problem-f.3.3}

\textbf{Given:} A 3 × 3 matrix B = {[}2, 1, 0; 1, 3, 1; 0, 1, 2{]}.

\textbf{Find:} The determinant using cofactor expansion along the first
row.

\textbf{Solution:}

det(B) = b₁₁ × C₁₁ − b₁₂ × C₁₂ + b₁₃ × C₁₃

where C\textsubscript{ij} is the cofactor (signed minor).

C₁₁ = det{[}3, 1; 1, 2{]} = 3×2 − 1×1 = 5 C₁₂ = det{[}1, 1; 0, 2{]} =
1×2 − 1×0 = 2 C₁₃ = det{[}1, 3; 0, 1{]} = 1×1 − 3×0 = 1

det(B) = 2×5 − 1×2 + 0×1 = 10 − 2 + 0 = \textbf{8}

Since det(B) ≠ 0, the matrix is invertible.

\begin{center}\rule{0.5\linewidth}{0.5pt}\end{center}

\section{Problem F.3.4}\label{problem-f.3.4}

\textbf{Given:} An admittance matrix Y = {[}0.2, −0.05; −0.05, 0.1{]} S
and current vector I = {[}0.5, 0.3{]}\textsuperscript{T} A.

\textbf{Find:} (a) The determinant of Y, (b) the inverse Y⁻¹, and (c)
the node voltages V = Y⁻¹I.

\textbf{Solution:}

\begin{enumerate}
\def\labelenumi{(\alph{enumi})}
\item
  det(Y) = (0.2)(0.1) − (−0.05)(−0.05) = 0.02 − 0.0025 = \textbf{0.0175}
\item
  For a 2 × 2 matrix {[}a, b; c, d{]}, the inverse is (1/det) × {[}d,
  −b; −c, a{]}: Y⁻¹ = (1/0.0175) × {[}0.1, 0.05; 0.05, 0.2{]} =
  \textbf{{[}5.714, 2.857; 2.857, 11.429{]}} Ω (the impedance matrix Z =
  Y⁻¹)
\item
  V = Y⁻¹I = Z × I: V₁ = 5.714 × 0.5 + 2.857 × 0.3 = 2.857 + 0.857 =
  \textbf{3.714 V} V₂ = 2.857 × 0.5 + 11.429 × 0.3 = 1.429 + 3.429 =
  \textbf{4.857 V}
\end{enumerate}

Verification: YV = {[}0.2×3.714+(−0.05)×4.857,
(−0.05)×3.714+0.1×4.857{]} = {[}0.743−0.243, −0.186+0.486{]} = {[}0.500,
0.300{]} = I ✓

\begin{center}\rule{0.5\linewidth}{0.5pt}\end{center}

\section{Problem F.3.5}\label{problem-f.3.5}

\textbf{Given:} A matrix M = {[}1, 2; 2, 4{]}.

\textbf{Find:} (a) The determinant, (b) whether the inverse exists, and
(c) the physical interpretation in a circuit context.

\textbf{Solution:}

\begin{enumerate}
\def\labelenumi{(\alph{enumi})}
\item
  det(M) = 1×4 − 2×2 = 4 − 4 = \textbf{0}
\item
  Since det(M) = 0, the matrix is \textbf{singular} and has \textbf{no
  inverse}. The system MX = B does not have a unique solution.
\item
  In a circuit context, a singular impedance or admittance matrix means:
\end{enumerate}

\begin{itemize}
\tightlist
\item
  The equations are \textbf{linearly dependent} (row 2 = 2 × row 1)
\item
  The circuit has a \textbf{redundant constraint} or a missing equation
\item
  There is either no solution or infinitely many solutions
\item
  Physically, this could represent a floating node, a short circuit that
  eliminates a degree of freedom, or a dependent source creating a
  proportional relationship between meshes
\end{itemize}

\begin{center}\rule{0.5\linewidth}{0.5pt}\end{center}

\section{Problem F.3.6}\label{problem-f.3.6}

\textbf{Given:} The property (AB)⁻¹ = B⁻¹A⁻¹ for invertible matrices A
and B. A = {[}3, 1; 2, 2{]} and B = {[}1, −1; 0, 2{]}.

\textbf{Find:} (a) A⁻¹, (b) B⁻¹, (c) AB, (d) (AB)⁻¹ directly, and (e)
B⁻¹A⁻¹, verifying that (d) equals (e).

\textbf{Solution:}

\begin{enumerate}
\def\labelenumi{(\alph{enumi})}
\item
  det(A) = 6 − 2 = 4. A⁻¹ = (1/4){[}2, −1; −2, 3{]} = \textbf{{[}0.5,
  −0.25; −0.5, 0.75{]}}
\item
  det(B) = 2 − 0 = 2. B⁻¹ = (1/2){[}2, 1; 0, 1{]} = \textbf{{[}1, 0.5;
  0, 0.5{]}}
\item
  AB: (AB)₁₁ = 3×1+1×0 = 3, (AB)₁₂ = 3×(−1)+1×2 = −1 (AB)₂₁ = 2×1+2×0 =
  2, (AB)₂₂ = 2×(−1)+2×2 = 2 \textbf{AB = {[}3, −1; 2, 2{]}}
\item
  det(AB) = 6+2 = 8. (AB)⁻¹ = (1/8){[}2, 1; −2, 3{]} = \textbf{{[}0.25,
  0.125; −0.25, 0.375{]}}
\item
  B⁻¹A⁻¹: (B⁻¹A⁻¹)₁₁ = 1×0.5+0.5×(−0.5) = 0.5−0.25 = 0.25 (B⁻¹A⁻¹)₁₂ =
  1×(−0.25)+0.5×0.75 = −0.25+0.375 = 0.125 (B⁻¹A⁻¹)₂₁ = 0×0.5+0.5×(−0.5)
  = −0.25 (B⁻¹A⁻¹)₂₂ = 0×(−0.25)+0.5×0.75 = 0.375 \textbf{B⁻¹A⁻¹ =
  {[}0.25, 0.125; −0.25, 0.375{]}} ✓ matches (AB)⁻¹
\end{enumerate}

\begin{center}\rule{0.5\linewidth}{0.5pt}\end{center}

\section{Problem F.3.7}\label{problem-f.3.7}

\textbf{Given:} A 3 × 3 admittance matrix for a three-node circuit: Y =
{[}0.5, −0.2, 0; −0.2, 0.5, −0.2; 0, −0.2, 0.5{]}

\textbf{Find:} (a) The determinant using cofactor expansion, and (b) the
condition number estimate (ratio of largest to smallest diagonal).

\textbf{Solution:}

\begin{enumerate}
\def\labelenumi{(\alph{enumi})}
\tightlist
\item
  Cofactor expansion along row 1: det(Y) = 0.5 × det{[}0.5, −0.2; −0.2,
  0.5{]} − (−0.2) × det{[}−0.2, −0.2; 0, 0.5{]} + 0 × (\ldots)
\end{enumerate}

det{[}0.5, −0.2; −0.2, 0.5{]} = 0.25 − 0.04 = 0.21 det{[}−0.2, −0.2; 0,
0.5{]} = (−0.2)(0.5) − (−0.2)(0) = −0.10

det(Y) = 0.5 × 0.21 + 0.2 × (−0.10) = 0.105 − 0.020 = \textbf{0.085}

\begin{enumerate}
\def\labelenumi{(\alph{enumi})}
\setcounter{enumi}{1}
\tightlist
\item
  All diagonal elements are equal (0.5), so the diagonal ratio is 1.0.
  However, a better condition estimate uses the eigenvalues. The
  tridiagonal structure with diagonal 0.5 and off-diagonal −0.2 gives
  eigenvalues: λ\textsubscript{k} = 0.5 − 2×0.2×cos(kπ/4) for k = 1,2,3
  λ₁ = 0.5 − 0.4×cos(π/4) = 0.5 − 0.283 = 0.217 λ₂ = 0.5 − 0.4×cos(π/2)
  = 0.5 λ₃ = 0.5 − 0.4×cos(3π/4) = 0.5 + 0.283 = 0.783
\end{enumerate}

Condition number ≈ λ\textsubscript{max}/λ\textsubscript{min} =
0.783/0.217 = \textbf{3.6}

This is well-conditioned; numerical solutions will be accurate.

\begin{center}\rule{0.5\linewidth}{0.5pt}\end{center}

\section{Problem F.3.8}\label{problem-f.3.8}

\textbf{Given:} A rotation matrix R(θ) = {[}cos θ, −sin θ; sin θ, cos
θ{]} for θ = 30°.

\textbf{Find:} (a) The numerical matrix, (b) the determinant, (c) the
inverse R⁻¹, and (d) verify that R⁻¹ = R\textsuperscript{T} (orthogonal
matrix).

\textbf{Solution:}

\begin{enumerate}
\def\labelenumi{(\alph{enumi})}
\item
  R(30°): cos 30° = √3/2 = 0.8660, sin 30° = 0.5 R = \textbf{{[}0.8660,
  −0.5; 0.5, 0.8660{]}}
\item
  det(R) = 0.8660² + 0.5² = 0.75 + 0.25 = \textbf{1.0}
\end{enumerate}

A rotation matrix always has determinant 1 (preserves area and
orientation).

\begin{enumerate}
\def\labelenumi{(\alph{enumi})}
\setcounter{enumi}{2}
\tightlist
\item
  R⁻¹ = (1/1.0) × {[}0.8660, 0.5; −0.5, 0.8660{]} = \textbf{{[}0.8660,
  0.5; −0.5, 0.8660{]}}
\end{enumerate}

This is R(−30°), which makes sense: the inverse of a 30° rotation is a
−30° rotation.

\begin{enumerate}
\def\labelenumi{(\alph{enumi})}
\setcounter{enumi}{3}
\tightlist
\item
  R\textsuperscript{T} = {[}0.8660, 0.5; −0.5, 0.8660{]} = R⁻¹ ✓
\end{enumerate}

\textbf{R is orthogonal} (R\textsuperscript{T}R = I), a property of all
rotation matrices. In signal processing, rotation matrices appear in
coordinate transformations (e.g., Park's transform for three-phase
systems).

\begin{center}\rule{0.5\linewidth}{0.5pt}\end{center}

\section{Problem F.3.9}\label{problem-f.3.9}

\textbf{Given:} A complex impedance matrix for an AC circuit: Z = {[}10
+ j5, −j3; −j3, 8 + j6{]}

\textbf{Find:} (a) The determinant (complex), (b) the inverse, and (c)
the current I when V = {[}20∠0°, 0{]}\textsuperscript{T}.

\textbf{Solution:}

\begin{enumerate}
\def\labelenumi{(\alph{enumi})}
\tightlist
\item
  det(Z) = (10 + j5)(8 + j6) − (−j3)(−j3) = 80 + j60 + j40 + j²30 − j²9
  = 80 + j100 − 30 + 9 = \textbf{59 + j100}
\end{enumerate}

\textbar det(Z)\textbar{} = √(59² + 100²) = √(3481 + 10000) = √13481 =
116.1

\begin{enumerate}
\def\labelenumi{(\alph{enumi})}
\setcounter{enumi}{1}
\tightlist
\item
  Z⁻¹ = (1/det) × {[}8 + j6, j3; j3, 10 + j5{]} = 1/(59 + j100) × {[}8 +
  j6, j3; j3, 10 + j5{]}
\end{enumerate}

Converting 1/(59 + j100): multiply by conjugate: = (59 − j100)/(59² +
100²) = (59 − j100)/13481 = 0.004376 − j0.007418

Z⁻¹₁₁ = (0.004376 − j0.007418)(8 + j6) = 0.0350 − j0.0594 + j0.0263 +
0.0445 = \textbf{0.0795 − j0.0331} Z⁻¹₁₂ = (0.004376 − j0.007418)(j3) =
j0.01313 + 0.02225 = \textbf{0.02225 + j0.01313}

\begin{enumerate}
\def\labelenumi{(\alph{enumi})}
\setcounter{enumi}{2}
\tightlist
\item
  I₁ = Z⁻¹₁₁ × V₁ = (0.0795 − j0.0331) × 20 = \textbf{1.590 − j0.662 A}
  \textbar I₁\textbar{} = √(1.590² + 0.662²) = √(2.528 + 0.438) = √2.966
  = \textbf{1.722 A} ∠I₁ = arctan(−0.662/1.590) = \textbf{−22.6°}
  (current lags voltage due to inductive impedance)
\end{enumerate}

\begin{center}\rule{0.5\linewidth}{0.5pt}\end{center}

\section{Problem F.3.10}\label{problem-f.3.10}

\textbf{Given:} The matrix equation AX = B where A = {[}4, 2; 3, 5{]}, X
is the unknown 2 × 2 matrix, and B = {[}10, 6; 11, 9{]}.

\textbf{Find:} (a) A⁻¹, (b) the solution X = A⁻¹B, and (c) verify by
computing AX.

\textbf{Solution:}

\begin{enumerate}
\def\labelenumi{(\alph{enumi})}
\item
  det(A) = 4×5 − 2×3 = 20 − 6 = 14. A⁻¹ = (1/14) × {[}5, −2; −3, 4{]} =
  \textbf{{[}0.3571, −0.1429; −0.2143, 0.2857{]}}
\item
  X = A⁻¹B: X₁₁ = 0.3571×10 + (−0.1429)×11 = 3.571 − 1.571 = 2.0 X₁₂ =
  0.3571×6 + (−0.1429)×9 = 2.143 − 1.286 = 0.857 X₂₁ = (−0.2143)×10 +
  0.2857×11 = −2.143 + 3.143 = 1.0 X₂₂ = (−0.2143)×6 + 0.2857×9 = −1.286
  + 2.571 = 1.286
\end{enumerate}

\textbf{X = {[}2.0, 0.857; 1.0, 1.286{]}}

More precisely: X₁₂ = 6/7, X₂₂ = 9/7.

\begin{enumerate}
\def\labelenumi{(\alph{enumi})}
\setcounter{enumi}{2}
\tightlist
\item
  Verify AX = B: (AX)₁₁ = 4×2 + 2×1 = 10 ✓ (AX)₁₂ = 4×(6/7) + 2×(9/7) =
  24/7 + 18/7 = 42/7 = 6 ✓ (AX)₂₁ = 3×2 + 5×1 = 11 ✓ (AX)₂₂ = 3×(6/7) +
  5×(9/7) = 18/7 + 45/7 = 63/7 = 9 ✓
\end{enumerate}

\chapter{Appendix F --- Section F.4: Solving Linear
Systems}\label{appendix-f-section-f.4-solving-linear-systems}

Practice problems covering Gaussian elimination and Cramer's rule for
solving systems of linear equations.

\begin{center}\rule{0.5\linewidth}{0.5pt}\end{center}

\section{Problem F.4.1}\label{problem-f.4.1}

\textbf{Given:} A two-node circuit yields the system: 0.3V₁ − 0.1V₂ = 2
−0.1V₁ + 0.2V₂ = 1

\textbf{Find:} Solve for V₁ and V₂ using Gaussian elimination.

\textbf{Solution:}

Augmented matrix: {[}0.3, −0.1, 2; −0.1, 0.2, 1{]}

Multiply row 1 by (0.1/0.3) = 1/3 and add to row 2 to eliminate V₁: R₂ →
R₂ + (1/3)R₁: New R₂: −0.1 + 0.1 = 0, 0.2 + (−0.1/3) = 0.2 − 0.0333 =
0.1667, 1 + 2/3 = 1.6667

Result: {[}0.3, −0.1, 2; 0, 0.1667, 1.6667{]}

Back-substitute: 0.1667 × V₂ = 1.6667 → V₂ = 1.6667/0.1667 =
\textbf{10.0 V} 0.3V₁ − 0.1 × 10 = 2 → 0.3V₁ = 3 → V₁ = \textbf{10.0 V}

Verification: 0.3(10) − 0.1(10) = 3 − 1 = 2 ✓ and −0.1(10) + 0.2(10) =
−1 + 2 = 1 ✓

\begin{center}\rule{0.5\linewidth}{0.5pt}\end{center}

\section{Problem F.4.2}\label{problem-f.4.2}

\textbf{Given:} A three-mesh circuit yields: 20I₁ − 10I₂ = 50 −10I₁ +
25I₂ − 5I₃ = 0 −5I₂ + 15I₃ = −20

\textbf{Find:} Solve for I₁, I₂, I₃ using Gaussian elimination.

\textbf{Solution:}

Augmented matrix: {[}20, −10, 0, 50; −10, 25, −5, 0; 0, −5, 15, −20{]}

Step 1: Eliminate I₁ from row 2. R₂ → R₂ + (10/20)R₁ = R₂ + 0.5R₁:
{[}−10+10, 25−5, −5+0, 0+25{]} = {[}0, 20, −5, 25{]}

Matrix: {[}20, −10, 0, 50; 0, 20, −5, 25; 0, −5, 15, −20{]}

Step 2: Eliminate I₂ from row 3. R₃ → R₃ + (5/20)R₂ = R₃ + 0.25R₂: {[}0,
−5+5, 15−1.25, −20+6.25{]} = {[}0, 0, 13.75, −13.75{]}

Matrix: {[}20, −10, 0, 50; 0, 20, −5, 25; 0, 0, 13.75, −13.75{]}

Back-substitute: 13.75I₃ = −13.75 → I₃ = \textbf{−1.0 A} 20I₂ − 5(−1) =
25 → 20I₂ = 20 → I₂ = \textbf{1.0 A} 20I₁ − 10(1) = 50 → 20I₁ = 60 → I₁
= \textbf{3.0 A}

Verification: 20(3) − 10(1) = 50 ✓, −10(3) + 25(1) − 5(−1) = −30+25+5 =
0 ✓, −5(1)+15(−1) = −5−15 = −20 ✓

\begin{center}\rule{0.5\linewidth}{0.5pt}\end{center}

\section{Problem F.4.3}\label{problem-f.4.3}

\textbf{Given:} A system of equations from nodal analysis: 5V₁ − 2V₂ =
10 −2V₁ + 6V₂ = 8

\textbf{Find:} Solve using Cramer's rule.

\textbf{Solution:}

Coefficient matrix A = {[}5, −2; −2, 6{]}, source vector B = {[}10,
8{]}\textsuperscript{T}.

det(A) = 5×6 − (−2)(−2) = 30 − 4 = \textbf{26}

For V₁: Replace column 1 with B: A₁ = {[}10, −2; 8, 6{]} det(A₁) = 10×6
− (−2)×8 = 60 + 16 = 76 V₁ = det(A₁)/det(A) = 76/26 = \textbf{2.923 V}

For V₂: Replace column 2 with B: A₂ = {[}5, 10; −2, 8{]} det(A₂) = 5×8 −
10×(−2) = 40 + 20 = 60 V₂ = det(A₂)/det(A) = 60/26 = \textbf{2.308 V}

Verification: 5(2.923) − 2(2.308) = 14.615 − 4.615 = 10.0 ✓

\begin{center}\rule{0.5\linewidth}{0.5pt}\end{center}

\section{Problem F.4.4}\label{problem-f.4.4}

\textbf{Given:} A three-node system: V₁ − 0.5V₂ = 4 −0.5V₁ + V₂ − 0.5V₃
= 0 −0.5V₂ + V₃ = 2

\textbf{Find:} Solve using Gaussian elimination with partial pivoting
(select the largest pivot at each step).

\textbf{Solution:}

Augmented matrix: {[}1, −0.5, 0, 4; −0.5, 1, −0.5, 0; 0, −0.5, 1, 2{]}

Step 1: Pivot element a₁₁ = 1 (already the largest in column 1). R₂ → R₂
+ 0.5R₁: {[}0, 0.75, −0.5, 2{]}

Matrix: {[}1, −0.5, 0, 4; 0, 0.75, −0.5, 2; 0, −0.5, 1, 2{]}

Step 2: Pivot column 2. \textbar0.75\textbar{} \textgreater{}
\textbar−0.5\textbar, so no row swap needed. R₃ → R₃ + (0.5/0.75)R₂ = R₃
+ (2/3)R₂: {[}0, 0, 1−1/3, 2+4/3{]} = {[}0, 0, 2/3, 10/3{]}

Back-substitute: (2/3)V₃ = 10/3 → V₃ = \textbf{5.0 V} 0.75V₂ − 0.5(5) =
2 → 0.75V₂ = 4.5 → V₂ = \textbf{6.0 V} V₁ − 0.5(6) = 4 → V₁ =
\textbf{7.0 V}

Verification: 7−3 = 4 ✓, −3.5+6−2.5 = 0 ✓, −3+5 = 2 ✓

\begin{center}\rule{0.5\linewidth}{0.5pt}\end{center}

\section{Problem F.4.5}\label{problem-f.4.5}

\textbf{Given:} The mesh equations for a bridge circuit: 30I₁ − 10I₂ −
20I₃ = 100 −10I₁ + 50I₂ − 10I₃ = 0 −20I₁ − 10I₂ + 60I₃ = 0

\textbf{Find:} Solve using Cramer's rule for I₁ only (to determine
source current).

\textbf{Solution:}

A = {[}30, −10, −20; −10, 50, −10; −20, −10, 60{]}

det(A): Expand along row 1: = 30 × det{[}50,−10;−10,60{]} − (−10) ×
det{[}−10,−10;−20,60{]} + (−20) × det{[}−10,50;−20,−10{]} = 30 × (3000 −
100) + 10 × (−600 − 200) − 20 × (100 + 1000) = 30 × 2900 + 10 × (−800) −
20 × 1100 = 87,000 − 8,000 − 22,000 = \textbf{57,000}

For I₁: A₁ = {[}100, −10, −20; 0, 50, −10; 0, −10, 60{]} det(A₁) = 100 ×
det{[}50,−10;−10,60{]} = 100 × (3000−100) = 100 × 2900 = 290,000 (The
other terms have a factor of 0 in column 1)

I₁ = det(A₁)/det(A) = 290,000/57,000 = \textbf{5.088 A}

The total current drawn from the 100 V source is 5.088 A, giving an
equivalent resistance of R\textsubscript{eq} = 100/5.088 = 19.65 Ω.

\begin{center}\rule{0.5\linewidth}{0.5pt}\end{center}

\section{Problem F.4.6}\label{problem-f.4.6}

\textbf{Given:} A system with a near-singular coefficient matrix: 100V₁
− 99V₂ = 1 99V₁ − 100V₂ = −1

\textbf{Find:} (a) The solution, (b) the determinant, and (c) a
discussion of numerical sensitivity.

\textbf{Solution:}

\begin{enumerate}
\def\labelenumi{(\alph{enumi})}
\tightlist
\item
  Using Cramer's rule: det(A) = 100×(−100) − (−99)(99) = −10,000 + 9,801
  = \textbf{−199}
\end{enumerate}

det(A₁) = {[}1, −99; −1, −100{]} = 1×(−100) − (−99)(−1) = −100 − 99 =
−199 V₁ = −199/(−199) = \textbf{1.0 V}

det(A₂) = {[}100, 1; 99, −1{]} = 100×(−1) − 1×99 = −100 − 99 = −199 V₂ =
−199/(−199) = \textbf{1.0 V}

\begin{enumerate}
\def\labelenumi{(\alph{enumi})}
\setcounter{enumi}{1}
\item
  det(A) = −199. While non-zero, it is small relative to the magnitude
  of the matrix entries (10,000).
\item
  The condition number is approximately
  max\textbar a\textsubscript{ij}\textbar/\textbar det/n\textbar{} ≈
  100/1 ≈ 100. More precisely, eigenvalues: λ₁ = 100−99 = 1 and λ₂ =
  −(100+99) = −199, giving condition number κ = 199/1 = 199. A small
  perturbation in the right-hand side could shift the solution
  significantly. For example, changing the RHS from {[}1, −1{]} to
  {[}1.01, −0.99{]} shifts V₁ to ≈1.01 and V₂ to ≈0.99 --- the solution
  change is proportional to κ times the input change, demonstrating
  \textbf{ill-conditioning}.
\end{enumerate}

\begin{center}\rule{0.5\linewidth}{0.5pt}\end{center}

\section{Problem F.4.7}\label{problem-f.4.7}

\textbf{Given:} A power system load flow equation (linearized): {[}B₁₁,
B₁₂; B₂₁, B₂₂{]} × {[}δ₁; δ₂{]} = {[}P₁; P₂{]} where B = {[}−20, 10; 10,
−15{]} (susceptance matrix, pu) and P = {[}1.5,
−0.8{]}\textsuperscript{T} pu.

\textbf{Find:} The voltage angles δ₁ and δ₂ in radians and degrees using
Gaussian elimination.

\textbf{Solution:}

Augmented matrix: {[}−20, 10, 1.5; 10, −15, −0.8{]}

R₂ → R₂ + (10/20)R₁ = R₂ + 0.5R₁: {[}10+(-10), −15+5, −0.8+0.75{]} =
{[}0, −10, −0.05{]}

Back-substitute: −10δ₂ = −0.05 → δ₂ = \textbf{0.005 rad = 0.286°} −20δ₁
+ 10(0.005) = 1.5 → −20δ₁ = 1.45 → δ₁ = \textbf{−0.0725 rad = −4.154°}

The negative angle at bus 1 indicates that bus 1 leads the reference
(generating power), and bus 2 is nearly at the reference angle (small
load).

\begin{center}\rule{0.5\linewidth}{0.5pt}\end{center}

\section{Problem F.4.8}\label{problem-f.4.8}

\textbf{Given:} Ohm's law in matrix form for a resistive network: V = RI
where R = {[}15, −5, 0; −5, 20, −10; 0, −10, 25{]} Ω and I = {[}I₁, I₂,
I₃{]}\textsuperscript{T}. The source voltages are V = {[}30, 0,
−10{]}\textsuperscript{T} V.

\textbf{Find:} Solve for I₁, I₂, I₃ using Gaussian elimination.

\textbf{Solution:}

Augmented matrix: {[}15, −5, 0, 30; −5, 20, −10, 0; 0, −10, 25, −10{]}

Step 1: R₂ → R₂ + (5/15)R₁ = R₂ + (1/3)R₁: {[}0, 20−5/3, −10, 0+10{]} =
{[}0, 18.333, −10, 10{]}

Matrix: {[}15, −5, 0, 30; 0, 18.333, −10, 10; 0, −10, 25, −10{]}

Step 2: R₃ → R₃ + (10/18.333)R₂ = R₃ + 0.5455R₂: {[}0, 0, 25−5.455,
−10+5.455{]} = {[}0, 0, 19.545, −4.545{]}

Back-substitute: 19.545I₃ = −4.545 → I₃ = \textbf{−0.2326 A} 18.333I₂ −
10(−0.2326) = 10 → 18.333I₂ = 7.674 → I₂ = \textbf{0.4186 A} 15I₁ −
5(0.4186) = 30 → 15I₁ = 32.093 → I₁ = \textbf{2.140 A}

Verification: 15(2.14)−5(0.419) = 32.1−2.1 = 30 ✓

\begin{center}\rule{0.5\linewidth}{0.5pt}\end{center}

\section{Problem F.4.9}\label{problem-f.4.9}

\textbf{Given:} A 2 × 2 system from a transformer equivalent circuit:
(R₁ + jX₁)I₁ − jX\textsubscript{m}I₂ = V₁ −jX\textsubscript{m}I₁ + (R₂ +
jX₂ + jX\textsubscript{m})I₂ = 0

With R₁ = 2 Ω, X₁ = 8 Ω, X\textsubscript{m} = 200 Ω, R₂ = 3 Ω, X₂ = 10
Ω, V₁ = 240∠0° V.

\textbf{Find:} Use Cramer's rule to find I₂ (secondary current).

\textbf{Solution:}

Z = {[}2+j8, −j200; −j200, 3+j210{]}

det(Z) = (2+j8)(3+j210) − (−j200)(−j200) = 6+j420+j24+j²1680 − j²40,000
= 6+j444−1680+40,000 = \textbf{38,326 + j444}

For I₂: Replace column 2 with {[}240, 0{]}: det(Z₂) = {[}2+j8, 240;
−j200, 0{]} = (2+j8)(0) − (240)(−j200) = 0 + j48,000 = \textbf{j48,000}

I₂ = det(Z₂)/det(Z) = j48,000/(38,326+j444)

\textbar det(Z)\textbar{} = √(38326² + 444²) = √(1.469×10⁹ + 1.972×10⁵)
≈ 38,329

I₂ = j48,000 × (38,326−j444)/(38,329²) = (j48,000×38,326 +
48,000×444)/1.469×10⁹ = (21,312+j1.840×10⁹)/1.469×10⁹ = 0.01451 +
j1.2526

Wait --- let me recalculate more carefully: I₂ = j48,000/(38,326+j444)
Multiply by conjugate: (38,326−j444)/(38,326²+444²) =
(38,326−j444)/1.469×10⁹

I₂ = j48,000 × (38,326−j444)/1.469×10⁹ = (j×48,000×38,326 −
j²×48,000×444)/1.469×10⁹ = (48,000×444 + j×48,000×38,326)/1.469×10⁹ =
(21,312,000 + j1,839,648,000)/1,469,459,432 = 0.01450 + j1.2519

\textbar I₂\textbar{} = √(0.0145² + 1.252²) ≈ \textbf{1.252 A} ∠I₂ ≈
arctan(1.252/0.0145) ≈ \textbf{89.3°}

The secondary current lags the applied voltage by nearly 90° (almost
purely magnetizing current with a small resistive component). This is
expected for a transformer operating near no-load.

\begin{center}\rule{0.5\linewidth}{0.5pt}\end{center}

\section{Problem F.4.10}\label{problem-f.4.10}

\textbf{Given:} A system of four equations from a DC circuit with four
node voltages:

\begin{verbatim}
4V₁ − V₂ − V₃ = 10
−V₁ + 3V₂ − V₃ = 5
−V₁ − V₂ + 4V₃ − V₄ = 0
−V₃ + 2V₄ = −3
\end{verbatim}

\textbf{Find:} Solve using Gaussian elimination (forward elimination,
then back-substitution).

\textbf{Solution:}

Augmented matrix: {[}4, −1, −1, 0, 10; −1, 3, −1, 0, 5; −1, −1, 4, −1,
0; 0, 0, −1, 2, −3{]}

Step 1: Eliminate V₁ from rows 2 and 3. R₂ → R₂ + (1/4)R₁: {[}0, 2.75,
−1.25, 0, 7.5{]} R₃ → R₃ + (1/4)R₁: {[}0, −1.25, 3.75, −1, 2.5{]}

Step 2: Eliminate V₂ from rows 3 and 4. R₃ → R₃ + (1.25/2.75)R₂ = R₃ +
0.4545R₂: {[}0, 0, 3.75−0.568, −1, 2.5+3.409{]} = {[}0, 0, 3.182, −1,
5.909{]}

Row 4 already has zero in column 2, no change needed.

Step 3: Eliminate V₃ from row 4. R₄ → R₄ + (1/3.182)R₃ = R₄ + 0.3143R₃:
{[}0, 0, 0, 2−0.314, −3+1.857{]} = {[}0, 0, 0, 1.686, −1.143{]}

Back-substitute: 1.686V₄ = −1.143 → V₄ = \textbf{−0.678 V} 3.182V₃ −
1×(−0.678) = 5.909 → 3.182V₃ = 5.231 → V₃ = \textbf{1.644 V} 2.75V₂ −
1.25×1.644 = 7.5 → 2.75V₂ = 9.555 → V₂ = \textbf{3.475 V} 4V₁ − 3.475 −
1.644 = 10 → 4V₁ = 15.119 → V₁ = \textbf{3.780 V}

Verification of row 4: −1.644 + 2×(−0.678) = −1.644 − 1.356 = −3.0 ✓

\chapter{Appendix F --- Section F.5: Applications in Electrical
Engineering}\label{appendix-f-section-f.5-applications-in-electrical-engineering}

Practice problems covering nodal analysis matrix form, two-port network
parameters, and state-space representation.

\begin{center}\rule{0.5\linewidth}{0.5pt}\end{center}

\section{Problem F.5.1}\label{problem-f.5.1}

\textbf{Given:} A four-node circuit with the following conductances: G₁₂
= 0.2 S, G₁₃ = 0.1 S, G₂₃ = 0.4 S, G₂₄ = 0.1 S, G₃₄ = 0.3 S, G₁₀ = 0.5 S
(node 1 to ground), G₄₀ = 0.2 S (node 4 to ground). A 3 A source feeds
into node 1, and a 1 A source feeds into node 3.

\textbf{Find:} (a) The 4 × 4 nodal admittance matrix Y, and (b) the
current source vector I.

\textbf{Solution:}

\begin{enumerate}
\def\labelenumi{(\alph{enumi})}
\tightlist
\item
  Diagonal entries: Y₁₁ = G₁₀ + G₁₂ + G₁₃ = 0.5 + 0.2 + 0.1 =
  \textbf{0.8 S} Y₂₂ = G₁₂ + G₂₃ + G₂₄ = 0.2 + 0.4 + 0.1 = \textbf{0.7
  S} Y₃₃ = G₁₃ + G₂₃ + G₃₄ = 0.1 + 0.4 + 0.3 = \textbf{0.8 S} Y₄₄ = G₂₄
  + G₃₄ + G₄₀ = 0.1 + 0.3 + 0.2 = \textbf{0.6 S}
\end{enumerate}

Off-diagonal entries: Y₁₂ = Y₂₁ = −0.2, Y₁₃ = Y₃₁ = −0.1, Y₁₄ = Y₄₁ = 0
Y₂₃ = Y₃₂ = −0.4, Y₂₄ = Y₄₂ = −0.1 Y₃₄ = Y₄₃ = −0.3

\textbf{Y = {[}0.8, −0.2, −0.1, 0; −0.2, 0.7, −0.4, −0.1; −0.1, −0.4,
0.8, −0.3; 0, −0.1, −0.3, 0.6{]}}

\begin{enumerate}
\def\labelenumi{(\alph{enumi})}
\setcounter{enumi}{1}
\tightlist
\item
  Current source vector: \textbf{I = {[}3, 0, 1, 0{]}\textsuperscript{T}
  A}
\end{enumerate}

The system YV = I can be solved by Gaussian elimination or matrix
inversion to find all four node voltages.

\begin{center}\rule{0.5\linewidth}{0.5pt}\end{center}

\section{Problem F.5.2}\label{problem-f.5.2}

\textbf{Given:} A two-port network has Z-parameters: Z₁₁ = 100 Ω, Z₁₂ =
Z₂₁ = 30 Ω, Z₂₂ = 80 Ω. The network is driven by a 10 V source at port 1
with internal resistance R\textsubscript{s} = 50 Ω, and port 2 is
terminated with R\textsubscript{L} = 200 Ω.

\textbf{Find:} (a) The port equations, (b) the port currents I₁ and I₂,
and (c) the voltage gain V₂/V\textsubscript{s}.

\textbf{Solution:}

\begin{enumerate}
\def\labelenumi{(\alph{enumi})}
\item
  Port equations with source and load: V₁ = Z₁₁I₁ + Z₁₂I₂ and V₂ = Z₂₁I₁
  + Z₂₂I₂ Source constraint: V₁ = V\textsubscript{s} −
  R\textsubscript{s}I₁ = 10 − 50I₁ Load constraint: V₂ =
  −R\textsubscript{L}I₂ = −200I₂
\item
  Substituting: 10 − 50I₁ = 100I₁ + 30I₂ → 150I₁ + 30I₂ = 10 \ldots{}
  (1) −200I₂ = 30I₁ + 80I₂ → 30I₁ + 280I₂ = 0 \ldots{} (2)
\end{enumerate}

From (2): I₁ = −280I₂/30 = −9.333I₂ Substitute into (1): 150(−9.333I₂) +
30I₂ = 10 −1400I₂ + 30I₂ = 10 −1370I₂ = 10 I₂ = \textbf{−7.30 × 10⁻³ A =
−7.30 mA} (current flows out of port 2 into load) I₁ = −9.333 × (−7.30 ×
10⁻³) = \textbf{68.1 mA}

\begin{enumerate}
\def\labelenumi{(\alph{enumi})}
\setcounter{enumi}{2}
\tightlist
\item
  V₂ = −200 × (−7.30 × 10⁻³) = 1.460 V Voltage gain:
  V₂/V\textsubscript{s} = 1.460/10 = \textbf{0.146 = −16.7 dB}
\end{enumerate}

\begin{center}\rule{0.5\linewidth}{0.5pt}\end{center}

\section{Problem F.5.3}\label{problem-f.5.3}

\textbf{Given:} Three two-port networks in cascade with ABCD matrices:
M₁ = {[}1, 0; 0.01, 1{]} (shunt admittance of 0.01 S) M₂ = {[}1, 100; 0,
1{]} (series impedance of 100 Ω) M₃ = {[}1, 0; 0.02, 1{]} (shunt
admittance of 0.02 S)

\textbf{Find:} (a) The overall ABCD matrix M = M₁ × M₂ × M₃, and (b) the
voltage gain V₁/V₂ with I₂ = 0 (open-circuit output).

\textbf{Solution:}

\begin{enumerate}
\def\labelenumi{(\alph{enumi})}
\tightlist
\item
  First compute M₁₂ = M₁ × M₂: A = 1×1+0×0 = 1 B = 1×100+0×1 = 100 C =
  0.01×1+1×0 = 0.01 D = 0.01×100+1×1 = 2
\end{enumerate}

M₁₂ = {[}1, 100; 0.01, 2{]}

Then M = M₁₂ × M₃: A = 1×1+100×0.02 = 1+2 = 3 B = 1×0+100×1 = 100 C =
0.01×1+2×0.02 = 0.01+0.04 = 0.05 D = 0.01×0+2×1 = 2

\textbf{M = {[}3, 100; 0.05, 2{]}}

Verification: det(M) = 3×2 − 100×0.05 = 6 − 5 = 1 ✓ (reciprocal network)

\begin{enumerate}
\def\labelenumi{(\alph{enumi})}
\setcounter{enumi}{1}
\tightlist
\item
  With I₂ = 0: V₁ = AV₂ + BI₂ = AV₂ = 3V₂ V₁/V₂ = A = \textbf{3} (or
  equivalently, V₂/V₁ = 1/3 = 0.333 = −9.54 dB)
\end{enumerate}

\begin{center}\rule{0.5\linewidth}{0.5pt}\end{center}

\section{Problem F.5.4}\label{problem-f.5.4}

\textbf{Given:} A series RLC circuit (R = 4 Ω, L = 0.5 H, C = 0.125 F)
driven by voltage source u(t). State variables: x₁ = capacitor voltage,
x₂ = inductor current. Output: y = inductor current.

\textbf{Find:} (a) The state-space matrices A, B, C, D, (b) the
eigenvalues of A (system poles), and (c) the natural frequency and
damping ratio.

\textbf{Solution:}

\begin{enumerate}
\def\labelenumi{(\alph{enumi})}
\tightlist
\item
  Circuit equations: dx₁/dt = (1/C)x₂ = 8x₂ dx₂/dt = (1/L)(u − Rx₂ − x₁)
  = 2(u − 4x₂ − x₁) = −2x₁ − 8x₂ + 2u Output: y = x₂
\end{enumerate}

\textbf{A = {[}0, 8; −2, −8{]}, B = {[}0; 2{]}, C = {[}0, 1{]}, D =
{[}0{]}}

\begin{enumerate}
\def\labelenumi{(\alph{enumi})}
\setcounter{enumi}{1}
\item
  Eigenvalues from det(A − λI) = 0: det({[}−λ, 8; −2, −8−λ{]}) =
  −λ(−8−λ) − 8(−2) = λ² + 8λ + 16 = 0 (λ + 4)² = 0 λ₁ = λ₂ = \textbf{−4}
  (repeated root)
\item
  From the characteristic equation λ² + 8λ + 16 = 0: Comparing with λ² +
  2ζω₀λ + ω₀² = 0: ω₀² = 16 → ω₀ = \textbf{4 rad/s} 2ζω₀ = 8 → ζ =
  8/(2×4) = \textbf{1.0 (critically damped)}
\end{enumerate}

The system is critically damped --- it returns to equilibrium as fast as
possible without oscillation.

\begin{center}\rule{0.5\linewidth}{0.5pt}\end{center}

\section{Problem F.5.5}\label{problem-f.5.5}

\textbf{Given:} A DC motor is modeled with state variables x₁ = angular
velocity (rad/s), x₂ = armature current (A). Parameters: J = 0.01 kg·m²
(inertia), B = 0.1 N·m·s/rad (friction), K\textsubscript{t} = 0.5 N·m/A
(torque constant), K\textsubscript{e} = 0.5 V·s/rad (back-EMF constant),
R\textsubscript{a} = 2 Ω, L\textsubscript{a} = 0.5 H. Input: armature
voltage u. Output: angular velocity y = x₁.

\textbf{Find:} (a) The state equations, (b) the A, B, C, D matrices, and
(c) the eigenvalues.

\textbf{Solution:}

\begin{enumerate}
\def\labelenumi{(\alph{enumi})}
\item
  Mechanical: J(dx₁/dt) = K\textsubscript{t}x₂ − Bx₁ → dx₁/dt =
  (K\textsubscript{t}/J)x₂ − (B/J)x₁ Electrical:
  L\textsubscript{a}(dx₂/dt) = u − R\textsubscript{a}x₂ −
  K\textsubscript{e}x₁ → dx₂/dt =
  −(K\textsubscript{e}/L\textsubscript{a})x₁ −
  (R\textsubscript{a}/L\textsubscript{a})x₂ + (1/L\textsubscript{a})u
\item
  Numerical values: dx₁/dt = −(0.1/0.01)x₁ + (0.5/0.01)x₂ = −10x₁ + 50x₂
  dx₂/dt = −(0.5/0.5)x₁ − (2/0.5)x₂ + (1/0.5)u = −x₁ − 4x₂ + 2u
\end{enumerate}

\textbf{A = {[}−10, 50; −1, −4{]}, B = {[}0; 2{]}, C = {[}1, 0{]}, D =
{[}0{]}}

\begin{enumerate}
\def\labelenumi{(\alph{enumi})}
\setcounter{enumi}{2}
\tightlist
\item
  Eigenvalues: det(A − λI) = (−10−λ)(−4−λ) − (50)(−1) = λ² + 14λ + 40 +
  50 = λ² + 14λ + 90 = 0 λ = (−14 ± √(196−360))/2 = (−14 ± √(−164))/2 =
  (−14 ± j12.81)/2 λ₁ = \textbf{−7 + j6.40}, λ₂ = \textbf{−7 − j6.40}
\end{enumerate}

The poles are complex conjugate with negative real parts, indicating a
stable, underdamped system with ω\textsubscript{n} = √90 = 9.49 rad/s
and ζ = 14/(2×9.49) = 0.738.

\begin{center}\rule{0.5\linewidth}{0.5pt}\end{center}

\section{Problem F.5.6}\label{problem-f.5.6}

\textbf{Given:} A two-port network has Y-parameters: Y₁₁ = 0.04 S, Y₁₂ =
Y₂₁ = −0.01 S, Y₂₂ = 0.03 S.

\textbf{Find:} (a) Convert to Z-parameters using Z = Y⁻¹, and (b) verify
by computing YZ = I.

\textbf{Solution:}

\begin{enumerate}
\def\labelenumi{(\alph{enumi})}
\tightlist
\item
  Y = {[}0.04, −0.01; −0.01, 0.03{]} det(Y) = 0.04×0.03 − (−0.01)² =
  0.0012 − 0.0001 = 0.0011
\end{enumerate}

Z = Y⁻¹ = (1/0.0011) × {[}0.03, 0.01; 0.01, 0.04{]} Z₁₁ = 0.03/0.0011 =
\textbf{27.27 Ω} Z₁₂ = Z₂₁ = 0.01/0.0011 = \textbf{9.091 Ω} Z₂₂ =
0.04/0.0011 = \textbf{36.36 Ω}

\textbf{Z = {[}27.27, 9.091; 9.091, 36.36{]} Ω}

\begin{enumerate}
\def\labelenumi{(\alph{enumi})}
\setcounter{enumi}{1}
\tightlist
\item
  YZ: (YZ)₁₁ = 0.04×27.27+(−0.01)×9.091 = 1.091−0.091 = 1.000 ✓ (YZ)₁₂ =
  0.04×9.091+(−0.01)×36.36 = 0.364−0.364 = 0.000 ✓ (YZ)₂₁ =
  (−0.01)×27.27+0.03×9.091 = −0.273+0.273 = 0.000 ✓ (YZ)₂₂ =
  (−0.01)×9.091+0.03×36.36 = −0.091+1.091 = 1.000 ✓
\end{enumerate}

\textbf{YZ = I₂} ✓

\begin{center}\rule{0.5\linewidth}{0.5pt}\end{center}

\section{Problem F.5.7}\label{problem-f.5.7}

\textbf{Given:} A state-space system with: A = {[}0, 1; −9, −6{]}, B =
{[}0; 1{]}, C = {[}9, 0{]}, D = {[}0{]}

\textbf{Find:} (a) The eigenvalues of A, (b) the transfer function H(s)
= C(sI − A)⁻¹B + D, and (c) the steady-state gain H(0).

\textbf{Solution:}

\begin{enumerate}
\def\labelenumi{(\alph{enumi})}
\item
  Eigenvalues: det(A − λI) = det{[}−λ, 1; −9, −6−λ{]} = −λ(−6−λ)+9 =
  λ²+6λ+9 = (λ+3)² = 0 λ₁ = λ₂ = \textbf{−3} (repeated pole)
\item
  Transfer function: sI − A = {[}s, −1; 9, s+6{]} det(sI − A) = s(s+6)+9
  = s²+6s+9 = (s+3)²
\end{enumerate}

(sI − A)⁻¹ = 1/(s+3)² × {[}s+6, 1; −9, s{]}

C(sI − A)⁻¹B: (sI − A)⁻¹B = 1/(s+3)² × {[}1; s{]} (from matrix-vector
multiplication) C × {[}1; s{]}/(s+3)² = {[}9, 0{]} × {[}1; s{]}/(s+3)² =
9/(s+3)²

\textbf{H(s) = 9/(s+3)² = 9/(s² + 6s + 9)}

\begin{enumerate}
\def\labelenumi{(\alph{enumi})}
\setcounter{enumi}{2}
\tightlist
\item
  Steady-state gain: H(0) = 9/(0+0+9) = \textbf{1.0}
\end{enumerate}

The system has unity DC gain and a critically damped response with time
constant τ = 1/3 s.

\begin{center}\rule{0.5\linewidth}{0.5pt}\end{center}

\section{Problem F.5.8}\label{problem-f.5.8}

\textbf{Given:} Convert the Y-parameters from Problem F.5.6 to ABCD
parameters using the conversion formulas: A = −Y₂₂/Y₂₁, B = −1/Y₂₁, C =
−det(Y)/Y₂₁, D = −Y₁₁/Y₂₁

\textbf{Find:} The ABCD matrix and verify det(ABCD) = 1 for a reciprocal
network.

\textbf{Solution:}

From Problem F.5.6: Y₁₁ = 0.04, Y₁₂ = Y₂₁ = −0.01, Y₂₂ = 0.03, det(Y) =
0.0011.

A = −Y₂₂/Y₂₁ = −0.03/(−0.01) = \textbf{3.0} B = −1/Y₂₁ = −1/(−0.01) =
\textbf{100 Ω} C = −det(Y)/Y₂₁ = −0.0011/(−0.01) = \textbf{0.11 S} D =
−Y₁₁/Y₂₁ = −0.04/(−0.01) = \textbf{4.0}

\textbf{ABCD = {[}3.0, 100; 0.11, 4.0{]}}

Verification: AD − BC = 3.0×4.0 − 100×0.11 = 12.0 − 11.0 = \textbf{1.0}
✓

The determinant equals 1, confirming the network is reciprocal (Y₁₂ =
Y₂₁).

\begin{center}\rule{0.5\linewidth}{0.5pt}\end{center}

\section{Problem F.5.9}\label{problem-f.5.9}

\textbf{Given:} A discrete-time state-space model for a digital
controller: A = {[}0, 1; −0.5, 1.2{]}, B = {[}0; 0.1{]}, C = {[}1, 0{]},
D = {[}0{]} Initial state x{[}0{]} = {[}0, 0{]}\textsuperscript{T},
input u{[}k{]} = 1 for all k ≥ 0 (unit step).

\textbf{Find:} (a) x{[}1{]}, x{[}2{]}, x{[}3{]}, and (b) the output
y{[}0{]}, y{[}1{]}, y{[}2{]}, y{[}3{]}.

\textbf{Solution:}

\begin{enumerate}
\def\labelenumi{(\alph{enumi})}
\tightlist
\item
  x{[}k+1{]} = Ax{[}k{]} + Bu{[}k{]}:
\end{enumerate}

x{[}1{]} = A{[}0;0{]} + B×1 = {[}0;0{]} + {[}0;0.1{]} = \textbf{{[}0,
0.1{]}\textsuperscript{T}}

x{[}2{]} = A{[}0;0.1{]} + B×1 = {[}0×0+1×0.1; −0.5×0+1.2×0.1{]} +
{[}0;0.1{]} = {[}0.1; 0.12{]} + {[}0; 0.1{]} = \textbf{{[}0.1,
0.22{]}\textsuperscript{T}}

x{[}3{]} = A{[}0.1;0.22{]} + B×1 = {[}0×0.1+1×0.22; −0.5×0.1+1.2×0.22{]}
+ {[}0;0.1{]} = {[}0.22; −0.05+0.264{]} + {[}0; 0.1{]} = {[}0.22;
0.314{]} = \textbf{{[}0.22, 0.314{]}\textsuperscript{T}}

\begin{enumerate}
\def\labelenumi{(\alph{enumi})}
\setcounter{enumi}{1}
\tightlist
\item
  y{[}k{]} = Cx{[}k{]} + Du{[}k{]} = x₁{[}k{]}: y{[}0{]} = \textbf{0},
  y{[}1{]} = \textbf{0}, y{[}2{]} = \textbf{0.1}, y{[}3{]} =
  \textbf{0.22}
\end{enumerate}

The output shows a delayed response (one-step delay due to B acting on
x₂ only, and C reading x₁ only) followed by an increasing ramp as the
step input accumulates through the integrator-like dynamics.

\begin{center}\rule{0.5\linewidth}{0.5pt}\end{center}

\section{Problem F.5.10}\label{problem-f.5.10}

\textbf{Given:} A three-phase power system uses the symmetrical
component transformation:

T = (1/3) × {[}1, 1, 1; 1, a, a²; 1, a², a{]}

where a = e\textsuperscript{j120°} = −0.5 + j0.866 and a² =
e\textsuperscript{j240°} = −0.5 − j0.866.

The unbalanced phase voltages are V\textsubscript{abc} = {[}100∠0°,
80∠−125°, 90∠115°{]}\textsuperscript{T} V.

\textbf{Find:} (a) The positive-sequence voltage V₁, (b) the
negative-sequence voltage V₂, and (c) the zero-sequence voltage V₀ using
V₀₁₂ = T × V\textsubscript{abc}.

\textbf{Solution:}

First, convert phase voltages to rectangular form: V\textsubscript{a} =
100 + j0 V\textsubscript{b} = 80∠−125° = 80(cos(−125°) + j sin(−125°)) =
80(−0.5736 − j0.8192) = −45.89 − j65.54 V\textsubscript{c} = 90∠115° =
90(cos 115° + j sin 115°) = 90(−0.4226 + j0.9063) = −38.04 + j81.57

\begin{enumerate}
\def\labelenumi{(\alph{enumi})}
\item
  Zero-sequence: V₀ = (1/3)(V\textsubscript{a} + V\textsubscript{b} +
  V\textsubscript{c}) = (1/3)(100 − 45.89 − 38.04 + j(0 − 65.54 +
  81.57)) = (1/3)(16.07 + j16.03) = \textbf{5.36 + j5.34 = 7.56∠44.9° V}
\item
  Positive-sequence: V₁ = (1/3)(V\textsubscript{a} + aV\textsubscript{b}
  + a²V\textsubscript{c}) aV\textsubscript{b} =
  (−0.5+j0.866)(−45.89−j65.54) = 22.945+j32.77−j39.74−j²56.76 =
  79.71−j6.97 a²V\textsubscript{c} = (−0.5−j0.866)(−38.04+j81.57) =
  19.02−j40.79+j32.94−j²70.64 = 89.66−j7.85
\end{enumerate}

V₁ = (1/3)(100+j0+79.71−j6.97+89.66−j7.85) = (1/3)(269.37−j14.82) =
\textbf{89.79 − j4.94 = 89.93∠−3.1° V}

\begin{enumerate}
\def\labelenumi{(\alph{enumi})}
\setcounter{enumi}{2}
\tightlist
\item
  Negative-sequence: V₂ = (1/3)(V\textsubscript{a} +
  a²V\textsubscript{b} + aV\textsubscript{c}) a²V\textsubscript{b} =
  (−0.5−j0.866)(−45.89−j65.54) = 22.945+j32.77+j39.74+j²56.76 =
  −33.82+j72.51 aV\textsubscript{c} = (−0.5+j0.866)(−38.04+j81.57) =
  19.02−j40.79−j32.94+j²70.64 = −51.62−j73.73
\end{enumerate}

V₂ = (1/3)(100+j0−33.82+j72.51−51.62−j73.73) = (1/3)(14.56−j1.22) =
\textbf{4.85 − j0.41 = 4.87∠−4.8° V}

The system is nearly balanced: \textbar V₁\textbar{} = 89.93 V
dominates, \textbar V₂\textbar{} = 4.87 V (5.4\% unbalance), and
\textbar V₀\textbar{} = 7.56 V (8.4\% zero-sequence, indicating neutral
current flow).

\end{document}
